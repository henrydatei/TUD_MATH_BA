\documentclass[11pt]{article}
\usepackage[a4paper,left=2cm,right=2cm,top=2cm,bottom=4cm,bindingoffset=5mm]{geometry}
\usepackage{paralist}
\usepackage{framed}
\usepackage{amssymb}
\usepackage{booktabs}

\title{\textbf{Analysis 1. Semester (WS2017/18)}}
\author{Dozent: Prof. Dr. Friedemann Schuricht\\
		Kursassistenz: Moritz Sch\"onherr}
\date{}
\begin{document}

\maketitle
\renewcommand*{\arraystretch}{1.4}

\raggedright 
Mathematik besitzt eine Sonderrolle unter den Wissenschaften, da
\begin{compactitem}
	\item Resultate nicht empirisch gezeigt werden m\"ussen
	\item Resultate nicht durch Experimente widerlegt werden k\"onnen
\end{compactitem}

\paragraph{Literatur}
\begin{compactitem}
	\item Forster: Analysis 1 + 2, Vieweg
	\item K\"onigsberger: Analysis 1 + 2, Springer
	\item Hildebrandt: Analysis 1 + 2, Springer
	\item Walter: Analysis 1 + 2, Springer
	\item Escher/Amann: Analysis 1 + 2, Birkh\"auser
	\item Ebbinghaus: Einf\"uhung in die Mengenlehre, BI-Wissenschaftsverlag
	\item Teubner-Taschenbuch der Mathematik, Teubner 1996
	\item Springer-Taschenbuch der Mathematik, Springer 2012
\end{compactitem}

\section{Grundlagen der Mathematik}

	\subsection{Grundbegriffe aus Mengenlehre und Logik}

		\textbf{Mengenlehre:} Universalit\"at von Aussagen \\
		\textbf{Logik:} Regeln des Folgerns, wahre/falsche Aussagen

		\begin{framed}
			\textbf{Definition Aussage:} Sachverhalt, dem man entweder den Wahrheitswert "wahr" oder "falsch" 				zuordnen kann, aber nichts anders.
		\end{framed}

		Beispiele:
		\begin{compactitem}
			\item 5 ist eine Quadratzahl $\to$ falsch (Aussage)
			\item Die Elbe flie{\ss}t durch Dresden $\to$ wahr (Aussage)
			\item Mathematik ist rot $\to$ ??? (keine Aussage)
		\end{compactitem}

		\begin{framed}
			\textbf{Definition Menge:} Zusammenfassung von bestimmten wohlunterscheidbaren Objekten der 			Anschauung oder des Denkens, welche die Elemente der Menge genannt werden, zu einem Ganzen. 			(\textsc{Cantor}, 1877)
		\end{framed}

		Beispiele:
		\begin{compactitem}
			\item $M_1 :=$ Menge aller St\"adte in Deutschland
			\item $M_2 := \{1;2;3\}$ 
		\end{compactitem}

		$\newline$ 
		F\"ur ein Objekt $m$ und eine Menge $M$ gilt stets $m \in M$ oder $m \notin M$ \\
		F\"ur die Mengen $M$ und $N$ gilt $M=N$, falls dieselben Elemente enthalten sind 
		$\{1;2;3\} = \{3;2;1\} = \{1;2;2;3\}$ \\
		- $N \subseteq M$, falls $n \in M$ f\"ur jedes $n \in N$ \\
		- $N \subset M$, falls zus\"atzlich $M \neq N$ \\

		\begin{framed}
			\textbf{Definition Aussageform:} Sachverhalt mit Variablen, der durch geeignete Ersetzung der 				Variablen zur Aussage wird.
		\end{framed}

		Beispiele:
		\begin{compactitem}
			\item $A(X) := $ Die Elbe flie{\ss}t durch X
			\item $B(X;Y;Z) := X + Y = Z$ 
		\end{compactitem}
		- aber $A(Dresden) ,B(2;3;4)$ sind Aussagen, $A(Mathematik)$ ist keine Aussage  \\
		- $A(X)$ ist eine Aussage f\"u jedes $X \in M_1$ $\to$ Generalisierung von Aussagen durch Mengen

		\paragraph{Bildung und Verkn\"upfung von Aussagen}
		\begin{tabular}{|c|c|c|c|c|c|c|}
			\hline
				$A$ & $B$ & $\lnot A$ & $A \land B$ & $A \lor B$ & $A \Rightarrow B$ & $A \iff B$\\
			\hline
				w & w & f & w & w & w & w\\
			\hline
				w & f & f & f & w & f & f\\
			\hline
				f & w & w & f & w & w & f\\
			\hline
				f & f & w & f & f & w & w\\
			\hline
		\end{tabular}

		$\newline$
		Beispiele:
		\begin{compactitem}
			\item $\lnot$(3 ist gerade) $\to$ w
			\item (4 ist gerade) $\land$ (4 ist Primzahl) $\to$ f
			\item (3 ist gerade) $\lor$ (3 ist Primzahl) $\to$ w
			\item (3 ist gerade) $\Rightarrow$ (Mond ist W\"urfel) $\to$ w
			\item (Die Sonne ist hei{\ss}) $\Rightarrow$ (es gibt Primzahlen) $\to$ w
		\end{compactitem}
		$\newline$
		Auschlie{\ss}endes oder: (entweder $A$ oder $B$) wird realisiert durch $\lnot(A \iff B)$.

		$\newline$
		Aussageform $A(X)$ sei f\"ur jedes $X \in M$ Aussage: neue Aussage mittels Quantoren
		\begin{compactitem}
			\item $\forall$: "f\"ur alle"
			\item $\exists$: "es existiert"
		\end{compactitem}

		Beispiele:
		\begin{compactitem}
			\item $\forall n \in \mathbb{N}: n$ ist gerade $\to$ f
			\item $\exists n \in \mathbb{N}: n$ ist gerade $\to$ w 
		\end{compactitem}

		\begin{framed}
			\textbf{Definition Tautologie bzw. Kontraduktion/Widerspruch:} zusammengesetzte Aussage, die 
			unabh\"angig vom Wahrheitsgehalt der Teilaussagen stest wahr bzw. falsch ist.
		\end{framed}

		Beispiele:
		\begin{compactitem}
			\item Tautologie (immer wahr): 
			$(A) \lor (\lnot A), \lnot (A  \land (\lnot A)), (A \land B) \Rightarrow A$
			\item Widerspruch (immer falsch): $A \land (\lnot A), A \iff \lnot A$  
			\item besondere Tautologie: $(A \Rightarrow B) \iff (\lnot B \Rightarrow \lnot A)$
		\end{compactitem}

		\begin{framed}
			\textbf{Satz (de Morgansche Regeln):} Folgende Aussagen sind Tautologien:
			\begin{compactitem}
				\item $\lnot(A \land B) \iff \lnot A \lor \lnot B$ 
				\item $\lnot(A \lor B) \iff \lnot A \land \lnot B$ 
			\end{compactitem}
		\end{framed}

		\paragraph{Bildung von Mengen}
		Seien $M$ und $N$ Mengen
		\begin{compactitem}
			\item Aufz\"ahlung der Elemente: $\{1;2;3\}$
			\item mittels Eigenschaften: $\{X \in M \mid A(X)\}$
			\item $\emptyset:=$ Menge, die keine Elemente enth\"alt
			\begin{compactitem}
				\item leere Menge ist immer Teilmenge jeder Menge $M$
				\item \textbf{Warnung:} $\{\emptyset\} \neq \emptyset$
			\end{compactitem}
			\item Verkn\"upfung von Mengen wie bei Aussagen
		\end{compactitem}

		\begin{framed}
			\textbf{Definition Mengensystem:} Ein Mengensystem $\mathcal M$ ist eine Menge, bestehend aus 				anderen Mengen.
			\begin{compactitem}
			\item $\bigcup M := \{X \mid \exists M \in \mathcal M: X \in M\}$ (Vereinigung aller Mengen in 
			$\mathcal M$)
			\item $\bigcap M := \{X \mid \forall M \in \mathcal M: X \in M\}$ (Durchschnitt aller Mengen in 
			$\mathcal M$)
			\end{compactitem}
		\end{framed}

		\begin{framed}
			\textbf{Definition Potenzmenge:} Die Potenzmenge $\mathcal P$ enth\"alt alle Teilmengen einer 				Menge $M$. \\
			$\mathcal P(X) := \{\tilde M \mid \tilde M \subset M\}$ 
		\end{framed}

		Beispiel:
		\begin{compactitem}
			\item $M_3 := \{1;3;5\}$ \\
			$\to \mathcal P(M_3) = \{\emptyset, \{1\}, \{3\}, \{5\}, \{1;3\}, \{1;5\}, \{3;5\}, \{1;3;5\}\}$
		\end{compactitem}

		\begin{framed}
			\textbf{Satz (de Morgansche Regeln f\"ur Mengen):}
			\begin{compactitem}
				\item $(\mathop{\bigcup}_{N \in \mathcal N} N)^C = \mathop{\bigcap}_{N \in \mathcal N} N^C$ 
				\item $(\mathop{\bigcap}_{N \in \mathcal N} N)^C = \mathop{\bigcup}_{N \in \mathcal N} N^C$ 
			\end{compactitem}
		\end{framed}

		\begin{framed}
			\textbf{Definition Kartesisches Produkt:} 
			$M \times N := \{m,n \mid m \in M \land n \in N\}$ \\
			$(m,n)$ hei{\ss}t geordnetes Paar (Reihenfolge wichtig!) \\
			allgemeiner: $M_1 \times ... \times M_k := \{(m_1,...,m_k) \mid m_j \in M_j, j=1, .., k\}$ \\
			$M^k := M \times ... \times M := \{(m_1,...,m_k) \mid m_j \in M_j, j=1, .., k\}$ 
		\end{framed}

		\begin{framed}
			\textbf{Satz (Auswahlaxiom): } Sei $\mathcal M$ ein Mengensystem nichtleerer paarweise disjunkter 				Mengen $M$.
			\begin{compactitem}
			\item Es existiert eine Auswahlmenge $\tilde M$, die mit jedem $M \in \mathcal M$ genau 1 Element 				gemeinsam hat.
			\item beachte: Die Auswahl ist nicht konstruktiv!
			\end{compactitem}
		\end{framed}


	\subsection{Aufbau einer mathematischen Theorie}
		Axiome $\to$ Beweise $\to$ S\"atze ("neue" wahre Aussagen) \\
		$\to$ ergibt Ansammlung (Menge) wahrer Aussagen
		
		\paragraph{Formulierung mathematischer Aussagen}
		\begin{compactitem}
		\item typische Form eines mathematischen Satzes: "Wenn A gilt, dann gilt auch B."
		\item formal: $A \Rightarrow B$ bzw. $A(X) \Rightarrow B(X)$ ist stets wahr (insbesondere falls 
		A wahr ist)
		\end{compactitem}
		
		$\newline$
		Beispiel
		\begin{compactitem}
			\item $X \in \mathbb N$ und ist durch 4 teilbar $\Rightarrow X$ ist durch 2 teilbar
			\item beachte: Implikation auch wahr, falls $X = 5$ oder $X =6$, dieser Fall ist aber 
			uninteressant
			\item genauer meint man sogar $A \land C \Rightarrow B$, wobei $C$ aus allen bekannten wahren
			Aussagen besteht
			\item man sagt: $B$ ist \textbf{notwendig} f\"ur $A$, da $A$ nur wahr sein kann, wenn $B$ 
			wahr ist
			\item man sagt: $A$ ist \textbf{hinreichend} f\"ur $B$, da $B$ stets wahr ist, wenn $A$ wahr ist 
		\end{compactitem}
		
		\paragraph{Mathematische Beweise}
		\begin{compactitem}
		\item \textbf{direkter Beweis:} finde Zwischenaussagen $A_1,...,A_k$, sodass f\"ur $A$ auch wahr: \\
		$(A \Rightarrow A_1) \land (A_1 \Rightarrow A_2) \land ... \land (A_k \Rightarrow B)$
		\item Beispiel: Zeige $x > 2 \Rightarrow x^2-3x+2>0$ \\
		$(x>2) \Rightarrow (x-2>0) \land (x-1>0) \Rightarrow (x-2) \cdot (x-1) \Rightarrow x^2-3x+2>0$
		\item \textbf{indirekter Beweis:} auf Grundlage der Tautologie $(A \Rightarrow B) \iff 
		(\lnot B \Rightarrow \lnot A)$  f\"uhrt man direkten Beweis $\lnot B \Rightarrow \lnot A$ (das 
		hei{\ss}t angenommen $B$ falsch, dann auch $A$ falsch)
		\item praktisch formuliert man das auch so: $(A \land \lnot B) \Rightarrow ... \Rightarrow (A 
		\land \lnot A)$
		\item Beispiel: Zeige $x^2-3x+2 \le 0$ sei wahr \\
		$\lnot B \Rightarrow (x-2) \cdot (x-1) \le 0 \Rightarrow 1 \le x \le 2 \Rightarrow x \le 2
		\Rightarrow \lnot A$
		\end{compactitem}

	\subsection{Relationen und Funktionen}
		\begin{framed}
			\textbf{Definition Relation:} Seien $M$ und $N$ Mengen. Dann ist jede Teilmenge $R$ von 
			$M \times N$ eine Relation. \\
			$(x,y) \in R$ hei{\ss}t: $x$ und $y$ stehen in Relation zueinander
		\end{framed}
		Beispiele
		\begin{compactitem}
		\item $M$ ist die Menge aller Menschen. Die Liebesbeziehung $x$ liebt $y$ sieht als geordnetes Paar
		geschrieben so aus: $(x,y)$. Das hei{\ss}t die Menge der Liebespaare ist das: $L := \{(x,y) \mid
		x \; liebt \; y\}$. Und es gilt: $L \subset M \times M$.
		\end{compactitem}
		$\newline$
		Die Relation $R \subset M \times N$ hei{\ss}t \textbf{Ordnungsrelation} (kurz. Ordnung) auf M, falls 			f\"ur alle $a,b,c \in M$ gilt:
		\begin{compactitem}
		\item $(a,a) \in R$ (reflexiv)
		\item $(a,b),(b,a) \in R$ (antisymetrisch)
		\item $(a,b), (b,c) \in R \Rightarrow (a,c) \in R$ (transitiv)
		\item z.B. $R = \{(X,Y) \in \mathcal P(Y) \times \mathcal P(Y) \mid X \subset Y\}$
		\end{compactitem}
		
		$\newline$
		Eine Ordnungsrelation hei{\ss}t \textbf{Totalordnung}, wenn zus\"atzlich gilt: $(a,b) \in R \lor 
		(b,a) \in R$ \\
		
		$\newline$
		Beispiel \\
		Seien $m$, $n$ und $o$ nat\"urliche Zahlen, dann ist $R = \{(m,n) \in \mathbb{N} \times \mathbb{N}
		\mid x \le y\}$ eine Totalordnung, da
		\begin{compactitem}
		\item $m \le m$ (reflexiv)
		\item $(m \le n \land n \le m) \Rightarrow m=n$ (antisymetrisch)
		\item $(m \le n \land n \le o) \Rightarrow m \le o$ (transitiv)
		\item $m \le n \lor n \le m$ (total)
		\end{compactitem}
		
		$\newline$
		Eine Relation auf $M$ hei{\ss}t \textbf{\"Aquivalenzrelation}, wenn f\"ur alle $a,b,c \in M$ gilt:
		\begin{compactitem}
		\item $(a,a) \in R$ (reflexiv)
		\item $(a,b),(b,a) \in R$ (symetrisch)
		\item $(a,b), (b,c) \in R \Rightarrow (a,c) \in R$ (transitiv)
		\end{compactitem}
		
		$\newline$
		Obwohl Ordnungs- und \"Aquivalenzrelation die gleichen Eigenschaften haben, haben sie
		unterschiedliche Zwecke: Ordnungsrelationen ordnen Elemente in einer Menge (z.B. das Zeichen $\le$ 
		ordnet die Menge der nat\"urlichen Zahlen), w\"ahrend \"Aquivalenzrelationen eine Menge in disjunkte
		Teilmengen (\"Aquivalenzklassen) ohne Rest aufteilen. \\
		$\newline$
		
		Wenn $R$ eine Ordnung auf M ist, so wird h\"aufig geschrieben: \\
		\noindent\hspace*{5mm} $a \le b$ bzw. $a \ge b$ falls $(a,b) \in \mathbb R$ \\
		\noindent\hspace*{5mm} $a < b$ bzw. $a > b$ falls zus\"atzlich $a \neq b$ \\
		
		\begin{framed}
			\textbf{Definition Abbildung/Funktion:} Eine Funktion $F$ von $M$ nach $N$ 
			(kurz: $F: M \mapsto N$), ist eine Vorschrift, die jedem Argument/Urbild $m \in M$ genau einen
			Wert/Bild $F(m) \in N$ zuordnet. \\
			$D(F) := M$ hei{\ss}t Definitionsbereich/Urbildmenge \\
			\noindent\hspace*{15mm} $N$ hei{\ss}t Zielbild \\
			$F(M') := \{n \in N \mid n=F(m)$ f\"ur ein $m \in M' \}$ ist Bild von $M' \subset M$ \\
			$F^{-1}(N') := \{m \in M \mid n=F(m)$ f\"ur ein $N' \}$ ist Urbild von $N' \subset N$ \\
			$R(F) := F(M)$ hei{\ss}t Wertebereich/Bildmenge \\
			$graph(F) := \{(m,n) \in M \times N \mid n=F(m)\}$ hei{\ss}t Graph von $F$ \\
			$F_{\mid M'}$ ist Einchr\"ankungvon $F$ auf $M' \subset M$
		\end{framed}
		
		Unterschied Zielmenge und Wertebereich: $f(x) = sin(x):$ \\
		\noindent\hspace*{5mm} Zielmenge: $\mathbb R$ \\
		\noindent\hspace*{5mm} Wertebereich: $[-1;1]$ \\
		$\newline$
		
		Funktionen $F$ und $G$ sind gleich, wenn
		\begin{compactitem}
		\item $D(F) = D(G)$
		\item $F(m) = G(m) \quad \forall m \in D(F)$
		\end{compactitem}
		$\newline$
		
		Manchaml wird auch die vereinfachende Schreibweise benutzt: \\
		- $F: M \mapsto N$, obwohl $D(F) \subsetneq M$ (z.B. $tan: \mathbb R \mapsto \mathbb R$, Probleme
		bei $\frac{\pi} {2}$) \\
		- gelegentlich spricht man auch von "Funktion $F(m)$" statt Funktion $F$ \\
		
		\begin{framed}
			\textbf{Definition Komposition/Verkn\"upfung:} Die Funktionen $F: M \mapsto N$ und $G: N \mapsto P$
			sind verkn\"upft, wenn \\
			$F \circ G: M \mapsto P$ mit $(F \circ G)(m) := G(F(m))$
		\end{framed}
		
		\textbf{Eigenschaften von Funktionen:} \\
		\begin{compactitem}
		\item injektiv: Zuordnung ist eineindeutig $\to F(m_1) = F(m_2) \Rightarrow m_1=m_2$
		\item Beispiel: $x^2$ ist nicht injektiv, da $F(2)=F(-2)=4$
		\item surjektiv: $F(M) = N \quad \forall n \in N \; \exists m \in M: F(m)=n$
		\item Beispiel: $sin(x)$ ist nicht surjektiv, da es kein $x$ f\"ur $y=27$ gibt
		\item bijektiv: injektiv und surjektiv
		\end{compactitem}
		$\newline$
		
		F\"ur bijektive Abbildung $F: M \mapsto N$ ist Umkehrabbildung/inverse Abbildung $F^{-1}: N \mapsto M$
		definiert durch: $F^{-1}(n) = m \iff F(m)=n$ \\
		Hinweis: Die Notation $F^{-1}(N')$ f\"ur Urbild bedeutet nicht, dass die inverse Abbildung $F^{-1}$
		existiert.
		
		\begin{framed}
			\textbf{Satz:} Sei $F: M \mapsto N$ surjektiv. Dann existiert die Abbildung $G: N \mapsto M$,
			sodass $F \circ G = id_N$ (d.h. $F(G(n))=n \quad \forall n \in N$)
		\end{framed}
		
		\begin{framed}
			\textbf{Definition Rechenoperation/Verkn\"upfung:} Eine Rechenoperation auf einer Menge $M$ ist
			die Abbildung $*: M \times M \mapsto M$ d.h. $(m,n) \in M$ wird das Ergbnis $m*n \in M$ zugeordnet.
		\end{framed}
		
		\textbf{Eigenschaften von Rechenoperationen:}
		\begin{compactitem}
		\item hat neutrales Element $e \in M: m*e=m$
		\item ist kommutativ $m*n=n*m$
		\item ist assotiativ $k*(m*n)=(k*m)*n$
		\item hat ein inverses Element $m' \in M$ zu $m \in M: m*m'=e$ 
		\end{compactitem}
		$e$ ist stets eindeutig, $m'$ ist eindeutig, wenn die Operation $*$ assoziativ ist. \\
		$\newline$
		
		Beispiele:
		\begin{compactitem}
			\item Addition $+$: $(m,n) \mapsto m+n$ Summe, neutrales Element hei{\ss}t Nullelement, inverses
			Element $-m$
			\item Multiplikation $\cdot$: $(m,n) \mapsto m \cdot n$ Produkt, neutrales Element Eins, inverses
			Element $m^{-1}$
		\end{compactitem}
		Addition und Multiplikation sind distributiv, falls $k(m+n) = k \cdot m + k \cdot n$
		
		\begin{framed}
			\textbf{Definition K\"orper:} Eine Menge $M$ ist ein K\"orper $K$, wenn man auf $K$ eine Addition
			und eine Multiplikation mit folgenden Eigenschaften durchf\"uhren kann:
			\begin{compactitem}
				\item es gibt neutrale Elemente 0 und 1 $\in K$
				\item Addition und Multiplikation sind jeweils kommutativ und assoziativ
				\item Addition und Multiplikation sind distributiv
				\item es gibt Inverse $-k$ und $k^{-1} \in K$ \\
				$\to$ die reellen Zahlen sind ein solcher K\"orper
			\end{compactitem}
		\end{framed}
		
		Eine Menge $M$ habe die Ordnung "$\le$" und diese erlaubt die Addition und Multiplikation, wenn
		\begin{compactitem}
			\item $a \le b \iff a+c \le b+c$
			\item $a \le b \iff a \cdot c \le b \cdot c \quad c >0$ \\
			$\to$ Man kann die Gleichungen in gewohnter Weise umformen.
		\end{compactitem}
		$\newline$
		
		Ein K\"orper $K$ hei{\ss}t angeordnet, wenn er eine Totalordnung besitzt, die mit Addition 
		und Multiplikation vertr\"aglich ist. \\
		$\newline$
		
		\textbf{Isomorphismus} bez\"uglich einer Struktur ist die bijektive Abbildung $I: M_1 
		\mapsto M_2$, die die vorhandene Struktur auf $M_1$ und $M_2$ erh\"alt, z.B.
		\begin{compactitem}
			\item Ordnung $\le_1$ auf $M_1$, falls $a \le_1 b \iff I(a) \le_2 I(b)$
			\item Abbildung $F_i: M_i \mapsto M_i$, falls $I(F_1(a)) = F_2(I(a))$
			\item Rechenoperation $*_i: M_i \times M_i \mapsto M_i$, falls $I(a*_1b) = I(a) *_2 I(b)$
			\item spezielles Element $a_i \in M_i$, falls $I(a_1) = a_2$
		\end{compactitem}
		$\newline$
		
		\textit{"Es gibt 2 verschiedene Arten von reellen Zahlen, meine und Prof. Schurichts. Wenn wir einen
		Isomorphismus finden, dann bedeutet das, dass unsere Zahlen strukturell die selben sind."}\\
		$\newline$
		
		Beispiele: $M_1 = \mathbb N$ und $M_2 = \{$gerade Zahlen$\}$, jeweils mit Addition, Multiplikation
		und Ordnung \\
		$\to I: M_2 \mapsto M_2$ mit $I(k)=2k \quad \forall k \in \mathbb N$ \\
		$\to$ Isomorphismus, der die Addition, Ordnung und die Null, aber nicht die Multiplikation erh\"alt
		
	\subsection{Bemerkungen zum Fundament der Mathematik}
		Forderungen an eine mathematische Theorie:
		\begin{compactitem}
			\item widerspruchsfrei: Satz und Negation nicht gleichzeitig herleitbar
			\item vollst\"andig: alle Aussagen innerhalb der Theorie sind als wahr oder falsch beweisbar
		\end{compactitem} 
		$\newline$
		
		2 Unvollst\"andigkeitss\"atze:
		\begin{compactitem}
			\item jedes System ist nicht gleichzeitig widerspruchsfrei und vollst\"andig
			\item in einem System kann man nicht die eigene Widerspruchsfreiheit zeigen
		\end{compactitem}
		
\section{Zahlenbereiche}
	\subsection{Nat\"urliche Zahlen}
		$\mathbb N$ sei diejenige Menge, die die \textbf{Peano-Axiome} erf\"ullt, das hei{\ss}t
		 \begin{compactitem}
		 	\item $\mathbb N$ sei induktiv, d.h. es existiert ein Nullelement und eine injektive Abbildung
		 	$\mathbb N \mapsto \mathbb N$ mit $\nu(n) \neq 0 \quad \forall n$
		 	\item Falls $N \subset \mathbb N$ induktiv in $\mathbb N$ (0, $\nu(n) \in N$ falls $n \in N
		 	\Rightarrow N = \mathbb N$
		 \end{compactitem}
		 $\to \mathbb N$ ist die kleinste induktive Menge \\
		 $\newline$
		 
		 Nach der Mengenlehre ZF (Zermelo-Fraenkel) existiert eine solche Menge $\mathbb N$ der nat\"urlichen
		 Zahlen. Mit den \"ublichen Symbolen hat man:
		 \begin{compactitem}
		 	\item $0 := \emptyset$
		 	\item $1 := \nu(0) := \{\emptyset\}$
		 	\item $2 := \nu(1) := \{\emptyset, \{\emptyset\}\}$
		 	\item $3 := \nu(2) := \{\emptyset, \{\emptyset, \{\emptyset\}\}\}$
		 \end{compactitem}
		 Damit ergibt sich in gewohnter Weise $\mathbb N = \{1; 2; 3; ...\}$ \\
		 anschauliche Notation $\nu(n) = n+1$ (beachte: noch keine Addition definiert!) \\
		 
		 \begin{framed}
			\textbf{Theorem:} Falls $\mathbb N$ und $\mathbb N'$ die Peano-Axiome erf\"ullen, sind sie 
			isomorph bez\"uglich Nachfolgerbildung und Nullelement. Das hei{\ss}t alle solche $\mathbb N'$
			sind strukturell gleich und k\"onnen mit obigem $\mathbb N$ identifiziert werden.
		\end{framed}
		
		\begin{framed}
			\textbf{Satz (Prinzip der vollst\"andigen Induktion):} Sei $\{A_n \mid n \in N\}$ eine Menge 
			von Aussagen $A_n$ mit der Eigenschaft \\
			\noindent\hspace*{5mm}IA: $A_0$ ist wahr \\
			\noindent\hspace*{5mm}IS: $\forall n \in \mathbb N$ gilt $A_n \Rightarrow A_{n+1}$ \\
			$A_n$ ist wahr f\"ur alle $n \in \mathbb N$
		\end{framed}
		
		\begin{framed}
			\textbf{Lemma:} Es gilt:
			\begin{compactitem}
				\item $\nu(n) \cup \{0\} = \mathbb N$
				\item $\nu(n) \neq n \quad \forall n \in \mathbb N$
			\end{compactitem}
		\end{framed}
		
		\begin{framed}
			\textbf{Satz (rekursive Definition/Rekursion):} Sei $B$ eine Menge und $b \in B$. Sei $F$ eine 
			Abbildung mit $F: B \times \mathbb N \mapsto B$. Dann liefert nach Vorschrift: $f(0) := b$  und
			$f(n+1) = F(f(n),n) \quad \forall n \in \mathbb N$ genau eine Abbildung $f: \mathbb N \mapsto B$. 
			Das hei{\ss}t eine solche Abbildung exstiert und ist eindeutig.
		\end{framed}
		$\newline$
		
		\textbf{Rechenoperationen:}
		\begin{compactitem}
			\item Definition Addition '$+$': $\mathbb N \times \mathbb N \mapsto \mathbb N$ auf $\mathbb N$ 
			durch $n+0:=n$, $n+\nu(m):=\nu(n+m) \quad \forall n,m \in \mathbb N$
			\item Definition Multiplikation '$\cdot$': $\mathbb N \times \mathbb N \mapsto \mathbb 
			N$ auf $\mathbb N$ durch $n \cdot 0 := 0$, $n \cdot \nu(m) := n \cdot m + n \quad \forall 
			n,m \in \mathbb N$
		\end{compactitem}
		F\"ur jedes feste $n \in \mathbb N$ sind beide Definitionen rekursiv und eindeutig definiert. \\
		$\forall n \in \mathbb N$ gilt: $n+1=n+\nu(0)=\nu(n+0) = \nu(n)$
		
		\begin{framed}
			\textbf{Satz:} Addition und Multiplikation haben folgende Eigenschaften:
			\begin{compactitem}
				\item es existiert jeweils ein neutrales Element
				\item kommutativ
				\item assoziativ
				\item distributiv
			\end{compactitem}
		\end{framed}
		$\newline$
		
		Es gilt $\forall k,m,n \in \mathbb N$:
		\begin{compactitem}
			\item $m \neq 0 \Rightarrow m+n \neq 0$
			\item $m \cdot n = 0 \Rightarrow n=0$ oder $m=0$
			\item $m+k=n+k \Rightarrow m=n$ (K\"urzungsregel der Addition)
			\item $m \cdot k=n \cdot k \Rightarrow m=n$ (K\"urzungsregel der Multiplikation)
		\end{compactitem}
		$\newline$
		
		Ordnung auf $\mathbb N:$ Relation $R := \{(m,n) \in \mathbb N \times \mathbb N \mid m \le n\}$ \\
		wobei $m \le n \iff n=m+k$ f\"ur ein $k \in \mathbb N$ \\
		
		\begin{framed}
			\textbf{Satz:} Es gilt auf $\mathbb N:$
			\begin{compactitem}
				\item $m \le n \Rightarrow \exists ! k \in \mathbb N: n=m+k$, nenne $n-m:=k$ (Differenz)
				\item Relation $R$ (bzw. $\le$) ist eine Totalordnung auf $\mathbb N$
				\item Ordnung $\le$ ist vertr\"aglich mit der Addition und Multiplikation
			\end{compactitem}
		\end{framed}
		
		\textit{Bweis: \\
		\begin{compactitem}
			\item Sei $n=m+k=m+k' \Rightarrow k=k'$
			\item Sei $n=n+0 \Rightarrow n \le n \Rightarrow$ reflexiv \\
			sei $k\le m, m \le n \Rightarrow \exists l,j: m=k+l, n=m+j=(k+l)+j=k+(l+j) \Rightarrow
			 k \le n \Rightarrow$ transitiv \\
			sei nun $m \le n und n \le m \Rightarrow n=m+j=n+l+j \Rightarrow 0=l+j \Rightarrow j=0 
			\Rightarrow n=m \Rightarrow$ antisymmetrisch \\
			Totalordnung, d.h. $\forall m,n \in \mathbb N: m\le n$ oder $n\le m$ \\
			IA: $m=0$ wegen $0=n+0$ folgt $0 \le n \forall n$ \\
			IS: gelte $m\le n$ oder $n \le m$ mit festem $m$ und $\forall n \in \mathbb N$, dann \\
			falls $n \le m \Rightarrow n \le m+1$ \\
			falls $m < n \Rightarrow \exists k \in \mathbb N: n=m+(k+1)=(m+)1+k \Rightarrow m+1 \le n$ \\
			$m\le n$ oder $n \le m$ gilt für $m+1$ und $\forall n \in \mathbb N$, also $\forall n,m \in 
			\mathbb N$
			\item sei $m \le n \Rightarrow \exists j: n=m+j \Rightarrow n+k=m+j+k \Rightarrow m+k \le n+k$
		\end{compactitem}}
		
	\subsection{Ganze und rationale Zahlen}
		\textbf{Frage:} Existiert eine natürliche Zahl $x$ mit  $n=n'+x$ für ein gegebenes $n$ und $n'$? \\
		\textbf{Antwort:} Das geht nur falls $n \le n'$, dann ist $x=n-n'$ \\
		\textbf{Ziel:} Zahlenbereichserweiterung, sodass die Gleichung immer l\"osbar ist. Ordne jedem Paar
		$(n,n') \in \mathbb N \times \mathbb N$ eine neue Zahl als L\"osung zu. Gewisse Paare liefern die
		gleiche L\"osung, z.B. $(6,4),(5,3),(7,5)$. Diese m\"ussen mittels Relation identifiziert werden. \\
		$\newline$
		
		$\mathbb Q := \{(n_1,n_1'),(n_2,n_2') \in (\mathbb N \times \mathbb N) \times (\mathbb N \times 
		\mathbb N) \mid n_1+n_2'=n_1'+n_2\}$ \\
		$\newline$
		
		\begin{framed}
			\textbf{Satz:} $\mathbb Q$ ist die \"Aquivalenzrelation auf $\mathbb N \times \mathbb N$
		\end{framed}
		$\newline$
		
		\textbf{Beispiele:} \\
		\begin{compactitem}
				\item $(5,3) \sim (6,4) \sim (7,5)$ bzw. $(5-3) \sim (6-4) \sim (7-5)$
				\item $(3,6) \sim (5,8)$ bzw. $(3-6) \sim (5-8)$
		\end{compactitem}
		$\newline$
		
		\textit{Beweis: \\
		\begin{compactitem}
			\item offenbar $((n,n'),(n,n')) \in \mathbb Q \Rightarrow$ reflexiv
			\item falls $((n_1,n_1'),(n_2,n_2')) \in \mathbb Q \Rightarrow (n_2,n_2'),(n_1,n_1')) \in
			\mathbb Q \Rightarrow$ symmetrisch
			\item sei $((n_1,n_1'),(n_2,n_2')) \in \mathbb Q$ und $((n_2,n_2'),(n_3,n_3')) \in \mathbb Q
			 \Rightarrow n_1+n_2'=n_1'+n_2, n_2+n_3'=n_2'+n_3 \Rightarrow n_1+n_3'=n_1'+n_3 \Rightarrow
			  ((n_1,n_1'),(n_3,n_3')) \in \mathbb Q \Rightarrow$ transitiv
		\end{compactitem}
		} 
		$\newline$
		
		setze $\overline \mathbb Z := \{[(n,n')] \mid n,n' \in \mathbb N\}$ Menge der ganzen Zahlen, 
		[ganze Zahl] \\
		Kurzschreibweise: $\overline m := [(m,m')]$ oder $\overline n := [(n,n')]$ \\
		$\newline$
		
		\begin{framed}
			\textbf{Satz:} Sei $[(n,n')] \in \overline \mathbb Z$. Dann existiert eindeutig $n* \in 
			\mathbb N$ mit 
			$(n*,0) \in [(n,n')]$, falls $n \ge n'$ bzw. $(0,n*) \in [(n,n')]$ falls $n < n'$.
		\end{framed}
		
		\textit{Beweis: \\
		\begin{compactitem}
			\item $n \ge n' \Rightarrow \exists ! n* \in \mathbb N: n=n'+n* \Rightarrow (n*,0) \sim (n,n')$
			\item $n < n' \Rightarrow \exists ! n* \in \mathbb N: n+n*=n' \Rightarrow (0,n*) \sim (n,n')$
		\end{compactitem}}
		$\newline$
		
		\textbf{Frage:} Was hat $\overline \mathbb Z$ mit $\mathbb Z$ zu tun?\\
		\textbf{Antwort:} identifiziere $(n,0)$ bzw. $(n-0)$ mit $n \in \mathbb N$ und identifiziere $(0,n)$ 
		bzw. $(0-n)$ mit Symbol $-n$ \\
		$\Rightarrow$ ganze Zahlen kann man eindeutig den Elementen folgender Mengen zuordnen: $\mathbb Z :=
		\mathbb N \cup \{(-n) \mid n \in \mathbb N\}$ \\
		$\newline$
		
		\textbf{Rechenoperationen auf $\overline \mathbb Z$:} \\
		\begin{compactitem}
			\item Addition: $\overline m + \overline n = [(m,m')]+[(n,n')]=[(m+n,m'+n')]$
			\item Multiplikation: $\overline m \cdot \overline n = [(m,m')] \cdot [(n,n')]=[(mn+m'n',mn'+m'n)]$
		\end{compactitem}
		$\newline$
		
		\begin{framed}
			\textbf{Satz:} Addition und Multiplikation sind eindeutig definiert, d.h. unabh\"angig von 
			Repr\"asentant bez\"uglich $\mathbb Q$
		\end{framed}
		\textit{Beweis: \\
		Sei $(m_1,m_1') \sim (m_2,m_2'), (n_1,n_1') \sim (n_2,n_2') \Rightarrow m_1+m_2'=m_1'+m_2, n_1
		+n_2'=n_1'+n_2 \Rightarrow m_1+n_1+m_2'+n_2'=m_1'+n_1'+m_2+n_2 \Rightarrow (m_1,m_1')+(n_1,n_1')
		 \sim (m_2,m_2')+(n_2,n_2')$} \\
		 $\newline$
		 
		 \begin{framed}
			\textbf{Satz:} F\"ur Addition und Multiplikation auf $\mathbb Z$ gilt $\forall \overline m, 
			\overline n \in \overline \mathbb Z$:
			\begin{compactitem}
				\item es existiert eine neutrales Element: $0:=[(0,0)]$, $1:=[(1,0)]$
				\item jeweils kommutativ, assoziativ und gemeinsam distributiv
				\item $- \overline n := [(n',n)] \in \mathbb Z$ ist invers bez\"uglich der Addition zu 
				$[(n,n')] = \overline n$
				\item $(-1) \cdot \overline n = - \overline n$
				\item $\overline m \cdot \overline n = 0 \iff \overline m =0 \lor \overline n=0$
			\end{compactitem}
		\end{framed}
		
		\textit{Beweis: \\
		\begin{compactitem}
			\item offenbar $\overline n +0=0+\overline n=\overline n$ und $\overline n \cdot 1 = 1 \cdot 
			\overline n = \overline n$
			\item Flei{\ss}arbeit
			\item offenbar $\overline n+(- \overline n) = (- \overline n)+\overline n=[(n+n',m+m')]=0$
			\item $(-1)\cdot \overline n = [(0,1)]\cdot [n,n']=[n',n]=-\overline n$
			\item \"Ubungsaufgabe
		\end{compactitem}}
\end{document}
