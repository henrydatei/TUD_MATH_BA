\chapter{Grundbegriffe der Wahrscheinlichkeitstheorie}

\section{Wahrscheinlichkeitsräume}

\subsection*{Ergebnisraum}

Welche der möglichen Ausgänge eines zufälligen Geschehens interessieren uns?\\
Würfeln? Augenzahl, nicht die Lage und die Fallhöhe

\begin{definition}[Ergebnisraum]
	Die Menge der relevanten Ergebnisse eines Zufallsgeschehens nennen wir \begriff{Ergebnisraum} und bezeichnen diesen mit $\Omega$.
\end{definition}

\begin{*example}
	\begin{itemize}
		\item Würfeln: $\Omega = \{1,2, \dots, 6\}$
		\item Wartezeiten: $\Omega = \real_{+} = [0, \infty)$ (überabzählbar!)
	\end{itemize}
\end{*example}

\subsection*{Ereignisse}

Oft interessieren wir uns gar nicht für das konkrete Ergenis des Zufallsexperiments, sondern nur für das Eintreten gewisser Ereignisse.
\begin{*example}
	\begin{itemize}
		\item Würfeln: Zahl ist $\ge 3$
		\item Wartezeit: Wartezeit $\le 5$ Minuten
	\end{itemize}
\end{*example}
$\longrightarrow$ Teilmenge aus Ereignisraum, also Element der Potenzmenge $\mathscr{P}(\Omega)$, denen eine Wahrscheinlichkeit zugeordnet werden kann, d.h. welche \begriff{messbar} (mb) sind.

\begin{definition}[Ereignisraum, messbarer Raum]
	Sei $\Omega \neq \emptyset$ ein Ergebnisraum und $\mathscr{F}$ eine $\sigma$-Algebra auf $\Omega$, d.h. eine Familie von Teilmenge von $\Omega$, sodass
	\begin{enumerate}
		\item $\Omega \in \mathscr{F}$
		\item $A \in \mathscr{F} \Rightarrow A^C \in \mathscr{F}$
		\item $A_1, A_2, \dots \in \mathscr{F} \Rightarrow \bigcap_{i \ge 1} \in \mathscr{F}$
	\end{enumerate}
	Dann heißt $(\Omega, \mathscr{F})$ \begriff{Ereignisraum} bzw. \begriff{messbarer Raum}.
\end{definition}

\subsection*{Wahrscheinlichkeiten}