\section{Stetige Gleichverteilung}
\begin{*erinnerung}
	$\O \subset \Rn$ Borel-messbar mit \person{Lebesgue}-Volumen $0 < \lambda(\O) < \infty$. Wahrscheinlichkeitsmaß ist $(\O, \borel(\O))$ mit Dichte
	\begin{align*}
	q\rho(x) = \frac{1}{\lambda(\O)}
	\end{align*}
	heißt stetige Gleichverteilung auf $\O$: $\Gleich(\O)$.\\
	Für alle $A \in \borel(\O)$ gilt:
	\begin{align*}
		\P(A) = \int_{A} \rho(x) \d x = \frac{\lambda(A)}{\lambda(\O)}.
	\end{align*}
	Meist verwenden wir $\Gleich([a,b]), a < b$ (Gleichverteilung auf Intervall) mit $\rho(x) = \sfrac{1}{(b-a)}, a \le x \le b$ und Verteilungsfunktion
	\begin{align*}
		F(x) = 
		\begin{cases}
			0 & x < a\\
			\int_{a}^{x} \frac{x-a}{b-a} & a \le x \le b\\
			1 & x >b.
		\end{cases}
	\end{align*}
\end{*erinnerung}
\section{Wartezeitverteilungen}
\ul{Negative Binomialverteilung}:\\
Wir wiederholen ein Bernoulliexperiment mit Erfolgswahrscheinlichkeit $p \in [0,1]$ unendlich oft. Gesucht ist die Anzahl der Misserfolge bis zum $r$-ten Erfolg, $r \in \N$. Ein passender Ergebnisraum ist $\O = \N_0$. Für Modellierung ist es jedoch leichter in jedem Versuch erfolgt (``1'') oder Misserfolg (``0'') festzuhalten und $i$ mit dem unendlichen Produktmaß des Bernoullimaßes auf $\set{0,1}^{\N}$ zu arbeiten.\\
Als Zufallsvariable
\begin{align*}
	X_r : \set{0,1}^{\N} \to \O
\end{align*}
welche die Anzahl der Misserfolge bis zum $r$-ten Erfolg darstellt, setze
\begin{align*}
	X_r (\omega) &= \min \set{\sum_{i=1}^k \omega_i = r} = r.
	\intertext{Dann}
	\P(X_r = k) &= \sum_{\substack{\omega \in \set{0,1}^{\N}\\ X_r(\omega) = k}} \prod_{i=1}^{\infty} \rho(\omega_i)
	\intertext{mit $\rho(0) = 1-p, \rho(1) = 1$ (Zähldichte der Bernoulliverteilung), also}
	\P(X_r = k) &= \binom{r+k-1}{k} (1-p)^k p^r \quad r \in \N_0.
\end{align*}
\begin{definition}[negative Binomialverteilung, geometrische Verteilung]
	Sei $p \in[0,1]$ und $r \in \N$, dann heißt die Verteilung auf $\N_0$ mit Zähldichte
	\begin{align*}
		\rho(k) = \binom{r+k-1}{k} p^r (1-p)^k
	\end{align*}
	die \begriff{negative Binomialverteilung} mit Parametern $(r,p)$. Schreibe $\negBin(r,p)$. Im Fall $r = 1$ nennen wir die Verteilung mit Zähldichte
	\begin{align*}
		\rho(k) = p(1-p)^k \quad k \in \N_0
	\end{align*}
	\begriff{geometrische Verteilung} mit Parametern $p$. Schreibe $\Geom(p)$.
\end{definition}