\chapter[Bedingte Wahrscheinlichkeiten und (Un)-abbhängigkeit]{Bedingte Wheiten und (Un)-abbhängigkeit}
\chaptermark{Bedingte Wheiten und (Un)-abbhängigkeit}

\section{Bedingte Wahrscheinlichkeiten}
\begin{example}
	Das Würfeln mit zwei fairen, sechsseitigen Würfeln können wir mit 
	\begin{align}
		\Omega = \set{(i,j,), i,j \in \set{1,\dots,6}}\notag
	\end{align}
	und $\probp = \Gleich(\Omega)$. Da $\abs{\Omega} = 36$ gilt also
	\begin{align}
		\probp(\set{\omega}) = \frac{1}{36} \quad \forall \omega \in \Omega.\notag
	\end{align}
	Betrachte das Ereignis
	\begin{align}
		A = \set{(i,j) \in \Omega : i + j = 8},\notag
	\end{align}
	dann folgt
	\begin{align}
		\probp(A) = \frac{5}{36}.\notag
	\end{align}
	Werden die beiden Würfel nach einander ausgeführt, so kann nach dem ersten Wurf eine Neubewertung der Wahrscheinlichkeit von $A$ erfolgen.\\
	Ist z.B.:
	\begin{align}
		B = \set{(i,j) \in \Omega, i = 4}\notag
	\end{align}
	eingetreten, so kann die Summe 8 nur durch eine weitere 4 realisiert werden, also mit Wahrscheinlichkeit
	\begin{align}
		\frac{1}{6} = \frac{\abs{A \cap B}}{\abs{B}}.\notag 
	\end{align}
	Das Eintreten von $B$ führt also dazu, dass das Wahrscheinlichkeitsmaß $\probp$ durch ein neues Wahrscheinlichkeitsmaß $\probp_{B}$ ersetzt werden muss. Hierbei sollte gelten:
	\begin{align} %TODO add references!
		 &\text{Renormierung: }\probp_{B} = 1\label{Renorm}\tag{R}\\
		 &\text{Proportionalität: Für alle} A \subset \sigF \mit A \subseteq B \text{ gilt }
		 \probp_{B}(A) = c_B \probp(A) \text{ mit einer Konstante } c_B.\label{Prop}\tag{P}
    \end{align}
\end{example}

\begin{lemma}
	Sei $(\Omega, \sigF, \probp)$ Wahrscheinlichkeitsraum und $B \in \sigF$ mit $\probp(B) > 0$. Dann gibt es genau ein Wahrscheinlichkeitsmaß $\probp_B$ auf $(\Omega, \sigF)$ mit den Eigenschaften \eqref{Renorm} und \eqref{Prop}. Dieses ist gegeben durch
	\begin{align}
		\probp_{B}(A) = \frac{\probp(A\cap B)}{\probp(B)} \quad \forall A \in \sigF.\notag
	\end{align}
\end{lemma}

\begin{proof} %TODO surpress ``Gleichung'' here?!
	Offenbar erfüllt $\probp_{B}$ wie definiert \eqref{Renorm} und \eqref{Prop}. Umgekehrt erfüllt $\probp_{B}$ \eqref{Renorm} und \eqref{Prop}. Dann folgt für $A \in \sigF$:
	\begin{align}
		\probp_{B}(A) = \probp_{B}(A\cap B) + \underbrace{\probp_{B}(A\setminus B)}_{= 0, \text{ wegen } \eqref{Renorm}} \overset{\eqref{Prop}}{=} c_B \probp(A \cap B).\notag
	\end{align}
	Für $A=B$ folgt zudem aus \eqref{Renorm}
	\begin{align}
		1 = \probp_{B}(B) = c_B \probp(B)\notag
	\end{align}
	also $c_B = \probp(B)^{-1}$.
\end{proof}

% % % % % % % % % % % % % % % % % % % % % % % % % % % 5th lecture % % % % % % % % % % % % % % % % % % % % % % % % % % %


\section{(Un)-abhängigkeit} \label{sec_unabhangigkeit}