\section{Wurzelkörper und Zerfällungskörper}
Sei $K$ ein Körper, $f \in K[X]$ mit $n = \deg(f) > 0$.
\begin{example}
	Sei $K=\Q$. Dann hat $f$ eine Nullstelle (``Wurzel'') $\alpha \in \C$, und $L:= K(\alpha) = K[\alpha]$ ist die kleinste Erweiterung von $\Q$ in $\C$, die diese Nullstelle enthält.
\end{example}
\begin{definition}[Wurzelkörper]
	Ein \begriff{Wurzelkörper} von $f$ ist eine Körpererweiterung $L \mid K$ der Form $L = K(\alpha)$ mit $f(\alpha) = 0$.
\end{definition}
\begin{lemma}
	\proplbl{1_3_3}
	Sei $L = K(\alpha)$ mit $f(\alpha) = 0$ ein Wurzelkörper von $f$. Dann ist $[L:K] \le n$. Ist $f$ irreduzibel, so ist $[L:K] = n$ und $g \mapsto g(\alpha)$ induziert einen Isomorphismus $\lnkset{K[X]}{(f)} \overset{\cong}{\longrightarrow}_K L$.
\end{lemma}
\begin{proof} %TODO fix b) ref
	Sei zunächst $f$ irreduzibel, $f_{\alpha} = \MinPol(\alpha \mid K)$. Dann ist $f = cf_{\alpha}$, die Behauptung folgt somit aus \propref{korpererweiterungen:prop:1:2:7:b}. Für $f \in K[X]$ beliebig, schreibe $f = f_1\cdots f_r$ mit $f_i \in K[X]$ irreduzibel und
	\begin{flalign*}
		\qquad &f(\alpha) = 0 \quad\Rightarrow\quad \text{O.E. } f_1(\alpha) = 0 \quad\Rightarrow\quad [L:K] = \deg(f_1) \le \deg(f) = n& %TODO is it really 1 in the index?
	\end{flalign*}
\end{proof}
\begin{lemma}
	\proplbl{1_3_4}
	Sei $f$ irreduzibel. Dann ist $L := \lnkset{K[X]}{(f)}$ ein Wurzelkörper von $f$.
\end{lemma}
\begin{proof}
	Betrachte den Epimorphismus $\pi = \pi_f\colon K[X] \to \lnkset{K[X]}{(f)} = L$, setze $\alpha = \pi(X)$
	\begin{itemize}[topsep=-6pt]
		\item $K$ Körper $\Rightarrow \pi_{\mid K}$ injektiv\\
		\hspace*{0.5em}$\Rightarrow$ können $K$ mit Teilkörper von $L$ identifizieren, sodass $\pi_{\mid K} = \id_K$
		\item $(f)$ irreduzibel $\Rightarrow$ $(f)$ prim $\xRightarrow{\text{GEO II.4.7}}$ $(f)$ maximal $\Rightarrow L = \lnkset{K[X]}{(f)}$ ist Körper
		\item $f(\alpha) = f(\pi(X)) \overset{(\ast)}{=} \pi(f(X)) = 0$ $\Rightarrow$ $f(X) \in \Ker(\pi)$\\
		($\ast$ gilt, da $f = \sum a_i X^i$ $\Rightarrow$ $\pi(f) = \sum \pi(a_i)\pi(X)^i = \sum a_i \pi(X)^i = f(\pi(X))$)
		\item $L=\pi(K[X]) = K[\pi(X)] = K[\alpha] \overset{\alpha \text{ alg.}}{=} K(\alpha)$
	\end{itemize}
\end{proof}
\begin{proposition}
	\proplbl{1_3_5}
	Sei $f$ irreduzibel. Ein Wurzelkörper von $f$ existiert und ist eindeutig in folgendem Sinn:\\
	Sind $L_1 = K(\alpha_1), L_2 = K(\alpha_2)$ mit $f(\alpha_1) = 0 = f(\alpha_2)$, so existiert genau ein $K$-Isomorphismus $\varphi\colon L_1 \to L_2$ mit $\varphi(\alpha_1) = \alpha_2$.
\end{proposition}
\begin{proof}\leavevmode\vspace*{\dimexpr-\baselineskip+2\lineskip}
	\begin{itemize}
		\item Existenz gibt \propref{1_3_4}
		\item \propref{1_3_3} liefert Isomorphismus
		\begin{align*}
			\left.\begin{array}{@{}l@{\;}c@{\;}c@{\;}c@{\;}l}
				L_1 & \xleftarrow[\varphi_1]{\cong} & \lnkset{K[X]}{(f)} & \xrightarrow[\varphi_2]{\cong} & L_2\\
				\alpha_1 & \mapsfrom & X + (f) & \mapsto & \alpha_2
			\end{array}\right\rbrace
			\quad\Rightarrow \;\;\varphi_2 \circ \varphi_1\colon L_1 \xrightarrow{\cong}_K L_2 \mit \alpha_1 \mapsto \alpha_2 &
		\end{align*}
		Umgekehrt ist jeder $K$-Isomorphismus $\varphi\colon L_1 \to_K L_2$ wegen $L_1 = K(\alpha_1)$ schon durch $\varphi(\alpha_1)$ festgelegt.
	\end{itemize}
\end{proof}
\begin{conclusion}
	\proplbl{1_3_6}
	$f$ hat einen Wurzelkörper.
\end{conclusion}
\begin{proof}
	Schreibe $f=f_1\cdots f_r, f_1,\dots,f_r \in K[X]$ irreduzibel, nehme einen Wurzelkörper von $f_1$.
\end{proof}
\begin{conclusion}
	\proplbl{1_3_7}
	Es gibt eine Erweiterung $L\mid K$, über der $f$ in Linearfaktoren zerfällt, also $f=c\prod_{i=0}^{n}(X-\alpha_i)$ mit $c \in K^{\times}$, $\alpha_1$, $\dots$, $\alpha_n \in L$. 
\end{conclusion}
\begin{proof}\NoEndMark
	Schreibe $f=c\cdot f_0 \mit c \in K^{\times}, f_0 \in K[X]$ normiert. Induktion nach $n$:
	\vspace*{\dimexpr-\baselineskip+2\lineskip}
	\begin{description}[leftmargin=4em,labelindent=1em]
		\item[$n=1{:}$] $f = X-a$, nehme $L=K$.
		\item[$n>1{:}$] Nach \propref{1_3_6} existiert $L_1 \mid K$, $\alpha_1 \in L_1$ mit $f_0 (\alpha_1) = 0$\\
		\begin{tabularx}{\linewidth}{@{\hspace*{0.5em}}r@{$\;\;$}X}
		$\Rightarrow$ & $f_0 = (X-\alpha_1)\cdot f_1 \mit f_1 \in L_1 [X]$ normiert\\
		$\xRightarrow{\text{(IH)}}$ & Es existiert $L\mid L_1$, $\alpha_1$, $\dots$, $\alpha_n \in L$ mit $f_1 = \prod_{i=2}^n (X - \alpha_i)$\\
		$\Rightarrow$  & $f = c\cdot f_0 = c\cdot (X-\alpha_1) \cdot f_1 = c \prod_{i=1}^n (X- \alpha_i)$\hfill\csname\InTheoType Symbol\endcsname
		\end{tabularx}
	\end{description}
\end{proof}
\begin{definition}[Zerfällungskörper]
	Ein \begriff{Zerfällungskörper} von $K$ ist eine Erweiterung $L\mid K$ der Form $L = K(\alpha_1,\dots,\alpha_n)$ mit \linebreak $f=c\mal \prod_{i=1}^n (X-\alpha_i)$ und $c \in K^{\times}$.
\end{definition}
\begin{proposition}
	\proplbl{1_3_9}
	Ein Zerfällungskörper von $f$ existiert.
\end{proposition}
\begin{proof}
	Ist $L\mid K$ wie in \propref{1_3_7}, ist $K(\alpha_1,\dots,\alpha_n)$ ein Zerfällungskörper von $f$.
\end{proof}
\begin{lemma}
	Ist $L \mid K$ ein Zerfällungskörper von $f$, so ist $[L:K] \le n$!
\end{lemma}
\begin{proof}\NoEndMark
	Sei $L = K(\alpha_1,\dots,\alpha_n)$, $f = c\prod_{i=1}^n (x-\alpha_i)$. Induktion nach $n$:
	\vspace*{\dimexpr-\baselineskip+3\lineskip}
	\begin{description}[leftmargin=4em,labelindent=1em]
		\item[$n=1{:}$] $L=K$, $[K:K] = 1$
		\item[$n>1{:}$] $L_1 = K(\alpha_1)$ ist Wurzelkörper von $f$
		
		\begin{tabularx}{\linewidth}{@{\hspace*{0.5em}}r@{$\;\;$}X}
			 $\xRightarrow{\propref{1_3_3}}$  &  $[L_1:K] \le n$ und schreibe $f=c\mal (X-\alpha_1)\mal f_1$, $f_1 = \prod_{i=2}^n (X-\alpha_i) \in L_1[X]$\\
			$\Rightarrow$ & $L = K(\alpha_1,\dots,\alpha_n) = L_1(\alpha_2,\dots,\alpha_n)$ ist Zerfällungskörper von $f_1$ (über $L_1$)\\
			$\xRightarrow{\text{IH}}$ & $[L:L_1] \le \deg(f_1)! = (n-1)!$\\
			$\Rightarrow$ & $[L:K] = [L:L_1]\cdot[L_1:K] = (n-1)!\,n = n!$\hfill\csname\InTheoType Symbol\endcsname
		\end{tabularx}
	\end{description}
\end{proof}
\begin{example}
	\begin{expenum}
		\item Ist $n=2$, so ist jeder Wurzelkörper $L$ von $f$, schon ein Zerfällungskörper: $[L:K]\le 2$.
		\item \proplbl{1_3_11_b} Ist $n =3$, $f$ irreduzibel. Schreibe $L_1 = K(\alpha), f = c(X-\alpha)f_1 \mit f_1 \in L_1[X]$
			\begin{itemize}
				\item $f_1$ reduzibel: $L_1$ ist schon Zerfällungskörper von $f$, $[L_1:K] = 3$
				\item $f_1$ irreduzibel: $L_1$ ist kein Zerfällungskörper von $f$. Ist $L$ Wurzelkörper von $f_1$, so ist $L$ Zerfällungskörper von $f$, $[L:K] = 3! = 6$
			\end{itemize}
	\end{expenum}
\end{example}
\begin{*example}
	Sei $f = X^3 -2 \in \Q[X]$, dann sind die Nullstellen von $f$: $\sqrt[3]{2} \in \R$, $\zeta_3\sqrt[2]{2}$, $\zeta_3^2 \sqrt[2]{2}$
	\begin{itemize}
		\item $\Q(\sqrt[3]{2})$ ist Wurzelkörper von $f$. $\Q(\sqrt[3]{2}) \subseteq \R$, $\zeta_3\sqrt[3]{2}$, $\zeta_3^2 \sqrt[3]{2} \notin \R$, daher kein Zerfällungskörper. Der Zerfällungskörper von $f$ ist
		\begin{align*}
			\Q(\sqrt[3]{2},\zeta_3\sqrt[3]{2}, \zeta_3^2 \sqrt[3]{2}) = \Q(\sqrt[3]{2}, \zeta_3 \sqrt[3]{2})
		\end{align*}
	\end{itemize}
\end{*example}
\begin{mathematica}
	Will man die Nullstellen von $f = X^3 - 2 \in \Q[X]$ finden, dann bietet Mathematica folgende Funktion:
	\begin{align*}
		\texttt{Solve[f==0,x,Complexes]},
	\end{align*}
	der letzte Parameter lässt einem den Körper wählen, in dem Mathematica suchen soll. Es gibt zur Auswahl \texttt{Integers, Rationals, Reals, Complexes}. Für das Beispiel erhält man folgenden Output:
	\begin{align*}
		\set{x \to -(-2)^{(1/3)}, x \to 2^{(1/3)}, x \to (-1)^{(2/3)} 2^{(1/3)}}.
	\end{align*}
	Dabei müsste man die Einheitswurzeln identifizieren:
	\begin{align*}
		\set{x \to \zeta_3\sqrt[3]{2}, x \to \sqrt[3]{2}, x \to \zeta_3^2 \sqrt[3]{2}}
	\end{align*}
\end{mathematica}
\begin{lemma}
	\proplbl{1_3_12}
	Sei $f = \sum_{i=0}^n a_i X^i$ irreduzibel und sei $L = K(\alpha)$ mit $f(\alpha)=0$ ein Wurzelkörper von $f$. Sei $L'\mid K'$ eine weitere Körpererweiterung und $\varphi \in \Hom(K,K')$. Ist $\sigma \in \Hom(L,L')$ eine Fortsetzung von $\varphi$ (d.h. $\sigma|_{K} = \varphi$), so ist $\sigma(\alpha)$ eine Nullstelle von $f^{\varphi}=\sum_{i=0}^n \varphi(\alpha_i)X^i \in K[X]$.
	
	Ist umgekehrt $\beta \in L' $ eine Nullstelle von $f^{\varphi}$, so gibt es genau eine Fortsetzung $\sigma \in \Hom(L,\tilde{L})$ von $\varphi$ mit $\sigma(\alpha) = \beta$.
\begin{center} % tikzcd was bitchy, compiled and included the pdf.
	\includegraphics{./tikz/lemma_1_3_12.pdf}
\end{center}
\end{lemma}
\begin{proof}[was für die Prüfung!]\leavevmode\vspace*{\dimexpr-\baselineskip+2\lineskip}
	\begin{itemize}
		\item $f(\alpha) = 0$ $\Rightarrow$ $0 = \sigma(0) = \sigma\big(f(\alpha)\big) = \sigma\big(\sum_{i=0}^n a_i \alpha^i\big) = \sum_{i=0}^n \varphi(a_i)\sigma(\alpha)^i = f^{\varphi}\big(\sigma(\alpha)\big)$
		\item Eindeutigkeit klar, da $L=K(\alpha)$
		\item Existenz: Betrachte 
		\begin{align*}
			\eta\colon&
			\left\lbrace\begin{array}{@{}l@{\;}c@{\;}l}
				K[X] &\to& L\\
				g &\mapsto& g(\alpha)
			\end{array}\right.&
			\psi\colon&
			\left\lbrace\begin{array}{@{}l@{\;}c@{\;}l}
				K[X] &\to& L'\\
				g &\mapsto& g^{\varphi}(\beta) 
			\end{array}\right.
		\end{align*}
		Beide sind Homomorphismen nach der universellen Eigenschaft.
		(Bemerke: $\eta$ surjektiv: $\eta|_{K} = \id \to K \subset \Image(\eta)$ mit $\eta(X) = \alpha \to \alpha \in \Image(\eta)$)\\
		Aus $\Ker(\eta)=(f)$ folgt der Isomorphismus $\bar{\eta}\colon \lnkset{K[X]}{(f)} \xrightarrow{\cong}L$ und\\
		$f \in \Ker(\psi) \Rightarrow \Ker(\psi) = (f)$ liefert Homomorphismus $\bar{\psi}\colon \lnkset{K[X]}{(f)} \to L'$\\
		$\sigma:= \bar{\psi}\circ \bar{\eta}^{-1}\colon L \to L'$ ist eine Fortsetzung von $\phi$ und
		\begin{align*}
			\sigma(\alpha) = \bar{\psi}\big(X+(f)\big) = \beta
		\end{align*}
	\end{itemize}
\end{proof}
\begin{proposition}
	\proplbl{1_3_13}
	Der Zerfällungskörper von $f$ ist eindeutig bestimmt bis auf $K$-Isomorphie.
\end{proposition}
\begin{proof} Für den Beweis betrachte erst folgende Aussage.
	\begin{adjustwidth}{1em}{-6pt}
	\begin{underlinedenvironment}[Behauptung]
		Ist $\varphi\colon K \to K'$ ein Isomorphismus, $L$ ein Zerfällungskörper und $L'$ ein Zerfällungskörper von $f^{\varphi}$, so setzt sich $\varphi$ zu einem Isomorphismus $L \to L'$ fort.
	\end{underlinedenvironment}
	\vspace*{-\baselineskip}
	\begin{proof}\NoEndMark
			Induktion nach $n = \deg(f)$:
			\vspace*{-4\lineskip}
			\begin{itemize}[leftmargin=4.5em]
				\item[$n=1$:] $L = K \xrightarrow[\varphi]{\cong} K' = L'$ \checkmark
				\item[$n>1$:] Schreibe $f = cg_1\cdots g_r$ mit $g_i \in K[X]$ normiert und irreduzibel, $c \in K^{\times}$\\[-0.8mm]
				\begin{tabularx}{\linewidth}{@{\hspace{0.5em}}r@{$\;\;$}X}
					$\Rightarrow$ & $f^{\varphi} = c^{\varphi}g_1^{\varphi}\cdots g_r^{\varphi}$ mit $c^{\varphi}\in (K')^{\varphi}$ und $g_i^{\varphi}\in K' [X]$ normiert und irreduzibel (weil $\varphi$ Isomorphismus ist). Sei $\alpha_1 \in L$ mit $g_1 (\alpha_1) = 0$, $\alpha'_1 \in L'$ mit $g_1^{\varphi}(\alpha'_1) = 0$\\
					$\xRightarrow{\propref{1_3_12}}$ & \begin{minipage}[t]{\linewidth}
						$\varphi$ setzt man zu einem Isomorphismus
					\[
						\sigma\colon K_1 := K(\alpha_1) \to K' (\alpha'_1) \quad \mit \sigma(\alpha_1) = \alpha'_1
					\]
					fort.
					\end{minipage} \\[16mm]
					\multicolumn{2}{l}{Schreibe $f=(X - \alpha_1)\cdot f_1$ mit $f_1 \in K_1 [X]$} \\[-2mm]
					$\Rightarrow$ & $f^{\varphi} = \big(x - \underbrace{\sigma(\alpha_1)}_{\alpha'_1}\big)\cdot f_1^{\sigma}$ mit $f_1^{\sigma}\in K'_1 [X]$.\\[-2mm]
					&$L$ ist Zerfällungskörper von $f_1$, $L'$ ist Zerfällungskörper von $f_1^{\sigma}$\\
					$\Rightarrow$ &  $\sigma$ setzt sich fort zu einem Isomorphismus $L \to L'$\hfill\csname\InTheoType Symbol\endcsname
				\end{tabularx}
			\end{itemize}
		\end{proof}
		\end{adjustwidth}
		Die Behauptung im Fall $\varphi = \id_K$ ist genau die Aussage von \propref{1_3_13}.
\end{proof}
\begin{remark}
	\proplbl{1_3_14}
	Ist $M\mid K$ eine Erweiterung, die einem Zerfällungskörper $L$ von $f$ enthält, dann ist dieser nicht nur bis auf die Isomorphie sondern als Teilkörper eindeutig bestimmt: $L = K(\alpha_1, \dots, \alpha_n)$, wobei $\alpha_1, \dots, \alpha_n$ genau die $n$ Nullstellen von $f$ in $M$ sind.
\end{remark}