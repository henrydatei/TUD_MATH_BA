\section{Algebraische Körpererweiterungen} \label{sec:sec_2}

Sei $L \mid K$ eine Körpererweiterung.

\begin{definition}[algebraisch, transzendent]
	Sei $\alpha \in L$. Gibt es ein $0 \neq f \in K$ mit $f(\alpha) = 0$, so heißt $\alpha$ \begriff{algebraisch} über $K$, andernfalls \begriff{transzendent} über $K$.
\end{definition}

\begin{example}
	\begin{enumerate}[label=(\alph*)]
		\item $\alpha \in K$ $\Rightarrow$ $\alpha$ ist algebraisch über $K$ (denn $f(\alpha) = 0$ für $f = X - \alpha \in K[X]$)
		\item $\sqrt{-1} \in \Q(\sqrt{-1})$ ist algebraisch über $\Q$ (denn $f(\sqrt{-1})=0$ für $f = X^2 + 1 \in \Q[X]$) \\
		$\sqrt{-1} \in \C$ ist algebraisch über $\R$        
	\end{enumerate}
\end{example}

\begin{remark}
	\proplbl{1_2_3}
	Sind $K \subseteq L \subseteq M$ Körper und $\alpha \in M$ algebraisch über $K$, so auch über $L$.
\end{remark}

\begin{lemma} 
	\proplbl{1_2_4}
	Genau dann ist $\alpha \in L$ algebraisch über $K$, wenn $1$, $\alpha$, $\alpha^2$, $\dots$ $K$-linear abhängig sind.
\end{lemma}

\begin{proof}
	Sei $\lambda_0$, $\lambda_1$, $\ldots \in K$, fast alle gleich Null, so ist
	\begin{align}
	\sum_{i=0}^\infty \lambda_i \alpha^i = 0\quad \Leftrightarrow\quad f(\alpha) = 0 \text{ für } f = \sum_{i=0}^\infty \lambda_i X^i \in K[X]\notag
	\end{align}
\end{proof}

\begin{lemma}
	Betrachte den Epimorphismus \begin{align*}
	\phi_{\alpha}\colon \left\lbrace\begin{array}{@{}l@{\;}c@{\;}l}
	K[X] &\to & K[\alpha]\\
	f &\mapsto & f(\alpha).
	\end{array}\right.
	\end{align*}
	Genau dann ist $\alpha$ algebraisch über $K$, wenn $\Ker(\phi_\alpha) \neq (0)$. In diesem Fall ist $\Ker(\phi_\alpha) = (f_\alpha)$ mit einem eindeutig bestimmten irreduziblen, normierten $f_\alpha \in K$.
\end{lemma}

\begin{proof}
	$K$ Hauptidealring $\Rightarrow \Ker(\phi_\alpha) = (f_\alpha)$, $f_\alpha \in K$, und o.E. sei $f_{\alpha}$ normiert. Aus $K[\alpha] \subseteq L$ nullteilerfrei folgt, dass $\Ker(\phi_\alpha)$ prim ist. Somit ist $f_\alpha$ prim im Hauptidealring, also auch irreduzibel.
\end{proof}

\begin{definition}[Minimalpolynom, Grad]
	Sei $\alpha \in L$ algebraisch über $K$, $\Ker(\phi_\alpha) = (f_\alpha)$ mit $f_\alpha \in K$ normiert und irreduzibel.
	\begin{enumerate}[label=(\alph*)]
		\item $\MinPol(\alpha\mid K) := f_\alpha$, das \begriff{Minimalpolynom} von $\alpha$ über $K$.
		\item $\deg(\alpha\mid K) :\Leftrightarrow \deg(f_\alpha)$, der \begriff{Grad} von $\alpha$ über $K$.
	\end{enumerate}
\end{definition}

\begin{proposition}
	\proplbl{1_2_7}
	Sei $\alpha \in L$.
	\begin{propenum}[]
		\item $\alpha$ transzendent über $K$ \\
		\hspace*{0.5em}$\Rightarrow$ $K[\alpha] \cong K[X]$, $K(\alpha) \cong_K K(X)$, $[K(\alpha) : K] = \infty$.
		\item \proplbl{korpererweiterungen:prop:1:2:7:b} $\alpha$ algebraisch über $K$ \\
		\begin{tabularx}{\linewidth}{@{\hspace*{0.5em}}c@{$\;$}X}
		$\Rightarrow$ & $K[\alpha] = K(\alpha) \cong \lnkset{K}{\MinPol(\alpha\mid K)}$, $[ K(\alpha) \colon K)]  = \deg(\alpha \mid K) < \infty$, und\hfill\newline
		$1$, $\alpha$, $\dots$, $\alpha^{\deg(\alpha \mid K) -1}$ ist $K$-Basis von $K(\alpha)$. 
		\end{tabularx}
	\end{propenum}
\end{proposition}

\begin{proof}\hspace*{0pt}
	\vspace*{\dimexpr-\baselineskip+\lineskip}
	\begin{enumerate}[label=(\alph*)]
		\item $\Ker(\phi_\alpha) = (0)$ $\Rightarrow$ $\phi_\alpha$ ist Isomorphismus (da zusätzlich injektiv) \\
		\hspace*{0.5em}$\Rightarrow$ $K(\alpha) \cong_K \Quot(K[\alpha]) \cong_K \Quot(K[X]) = K(X)$ \\
		\hspace*{0.5em}$\Rightarrow$ $[K(\alpha) \colon K] = [K(X) \colon K] = \infty$
		\item Sei $f = f_\alpha = \MinPol(\alpha \mid K)$, und $n = \deg(\alpha \mid K) = \deg(f)$.
		\begin{itemize}
			\item $f$ irreduzibel $\Rightarrow$ $(f) \neq (0)$ prim ${\xRightarrow{\text{GEO II.4.7}}}$ $(f)$ ist maximal \\
			\hspace*{0.5em}$\Rightarrow$ $K[\alpha] \cong \lnkset{K[X]}{(f)}$ ist Körper $\Rightarrow$ $K[\alpha] = K(\alpha)$
			\item $1$, $\alpha$, $\dots$, $\alpha^{n-1}$ sind $K$-linear unabhängig: 
			\begin{align*}
			\sum_{i=0}^{n-1} \lambda_i \alpha^i = 0 \quad \Rightarrow \quad \sum_{i=0}^{n-1} \lambda_i X^i \in (f) \quad \xRightarrow{\deg f = n} \quad \lambda_i = 0 \enskip \forall i
			\end{align*}
			\item $1, \alpha, \dots , \alpha^{n-1}$ ist Erzeugendensystem: Für $g \in K[X]$ ist 
			\begin{flalign*}
			 \qquad g &= qf + r &
			\end{flalign*}
			mit $q$, $r \in K[X]$ und $\deg(r) < \deg(f) = n$ und  
			\begin{flalign*}
			 \qquad g(\alpha) &= q(\alpha) \underbrace{f(\alpha)}_{=0} + r(\alpha) = r(\alpha).&
			\end{flalign*}
			Somit folgt: \begin{flalign*}
				\qquad K[\alpha] &= \Image(\phi_\alpha) = \big\{g(\alpha) \;\big|\; g \in K\big\} = \big\{r(\alpha) \;\big|\; r \in K,\, \deg(r) < n\big\} = \sum_{i=0}^{n-1} K \cdot \alpha^i&
			\end{flalign*}
		\end{itemize}
	\end{enumerate}
\end{proof}

\begin{example}
	\begin{enumerate}[label=(\alph*)]
		\item $p \in \Z$ prim $\Rightarrow$ $\sqrt{p} \in \C$ ist algebraisch über $\Q$. \\
		Da $f(X) = X^2 - p$ irreduzibel in $\Q$ ist (GEO II.7.3), ist $\MinPol(\sqrt{p}\mid\Q) = X^2 - p$, und\linebreak $[\Q(\sqrt{p}) : \Q] = 2$.
		\item Sei $\zeta_p = e^{\frac{2\pi i}{p}} \in \C$ ($p \in \N$ prim). Da $\Phi_p =  \frac{X^p-1}{X-1} = X^{p-1} + X^{p-2} + \cdots + X + 1 \in \Q$ irreduzibel in $\Q$ ist (GEO II.7.9), ist $\MinPol(\zeta_p \mid \Q) = \Phi_p$, $[\Q(\zeta_p) : \Q] = p-1$.
		
		Daraus folgt schließlich $[\C : \Q] \ge [\Q(\zeta_p) : \Q] = p-1 \enskip \forall p$ $\Rightarrow$ $[\C : \Q] = \infty$ $\Rightarrow$ $[R : \Q] = \infty$.
		\item $e$, $\pi \in \R$ sind transzendent über $\Q$ (\person{Hermite} 1873, \person{Lindemann} 1882). \\
		Daraus folgt: $[R : \Q] \ge [\Q(\pi): \Q] = \infty$. Jedoch ist unbekannt, ob z.B. $\pi + e$ transzendent ist.
	\end{enumerate}
\end{example}

\begin{definition}
	$L \mid K$ ist \begriff{algebraisch} $:\Leftrightarrow$ jedes $\alpha \in L$ ist algebraisch über $K$.
\end{definition}

\begin{proposition}
	\proplbl{1_2_10}
	$L \mid K$ endlich $\Rightarrow$ $L \mid K$ algebraisch.
\end{proposition}

\begin{proof}
	Sei $\alpha \in L$, $[L : K] = n$. Dann ist  $1,$ $\alpha$, $\dots$, $\alpha^n$ $K$-linear abhängig $\xRightarrow{\propref{1_2_4}} \alpha$ algebraisch über $K$.
\end{proof}

\begin{conclusion}
	\proplbl{1_2_11}
	Ist $L = K(\alpha_1, \dots, \alpha_n)$ mit $\alpha_1$, $\dots$, $\alpha_n$ algebraisch über $K$, so ist $L \mid K$ endlich, insbesondere algebraisch.
\end{conclusion}

\begin{proof}
	Induktion nach $n$:
	\vspace*{\dimexpr-\baselineskip+2\lineskip\relax}
	\begin{itemize}
		\item $n=0$: \checkmark
		\item \renewcommand*{\arraystretch}{1.0}
		\begin{tabular}[t]{@{}l@{\;}l}
		$n > 0$: & $K_1 :=  K(\alpha_1, \dots, \alpha_{n-1})$ \\
				 & \hspace*{0.5em}$\Rightarrow L=K_1(\alpha_n)$, $\alpha_n$ algebraisch über $K_1$ (\propref{1_2_3}) \\
				 & \hspace*{0.5em}$\Rightarrow [L : K] = \underbrace{[K_1(\alpha_n) : K_1]}_{< \infty \text{ nach \propref{1_2_7}}}\cdot \underbrace{[K_1 : K]}_{< \infty \text{ nach IH}}$
	\end{tabular}
	\end{itemize}
\end{proof}

\begin{conclusion}
	Es sind äquivalent:
	\begin{enumerate}[label=(\alph*)]
		\item $L \mid K$ ist endlich.
		\item $L \mid K$ ist endlich erzeugt und algebraisch.
		\item $L = K(\alpha_1, \dots , \alpha_n)$ mit $\alpha_1, \dots, \alpha_n$ algebraisch über $K$.
	\end{enumerate}
\end{conclusion}

\begin{proof}
	\leavevmode\vspace*{\dimexpr-\baselineskip+2\lineskip}
	\begin{itemize}[widest={\ (1) $\Rightarrow$ (2)},leftmargin=*]
		\item[(1) $\Rightarrow$ (2):] \propref{1_1_15} und \propref{1_2_10}
		\item[(2) $\Rightarrow$ (3):] trivial
		\item[(3) $\Rightarrow$ (1):] \propref{1_2_11}
	\end{itemize}
\end{proof}

\begin{remark}
	Nach \propref{1_2_7} ist
	\begin{align*}
		\alpha \text{ algebraisch über } K\quad \Leftrightarrow \quad K[\alpha] = K(\alpha). &
	\end{align*}
	Direkter Beweis für $(\Rightarrow)$: \\
	Sei $0 \neq \beta \in K[\alpha]$. Daraus folgt, dass $f(\beta) = 0$ für ein irreduzibles $0 \neq f = \sum_{i=0}^n a_i X^i \in K[X]$. Durch Einsetzen von $\beta$ und Division durch $\beta$ erhält man
	\begin{flalign*}
	\qquad &\xRightarrow{a_0 \neq 0}\beta^{-1} = -a_0^{-1} ( a_1 + a_2 \beta + \dots + a_n \beta^{n-1}) \in K[\beta] \subseteq K[\alpha]&
	\end{flalign*}
\end{remark}
% % % % % % % % % % % % % % % % 3rd lecture % % % % % % % % % % % % % % % % % % %
\begin{proposition}
	\proplbl{1_2_14}
	Seien $K \subseteq L \subseteq M$ Körper. Dann gilt:
	\begin{align*}
		M\mid K \text{ algebraisch } \Leftrightarrow M\mid L \text{ algebraisch und } L \mid K \text{ algebraisch }
	\end{align*}
\end{proposition}

\begin{proof}\NoEndMark
	\begin{itemize}
		\item[($\Rightarrow$)] klar, siehe \propref{1_2_3}.
		\item[($\Leftarrow$)] Sei $\alpha \in M$. Schreibe $f=\MinPol(\alpha \mid L) = \sum_{i=0}^{n} a_i x^i$, $L_0 := K(a_0,\dots,a_n)$\\
		\renewcommand*{\arraystretch}{1.0}
		\begin{tabularx}{\linewidth}{@{\hspace*{0.5em}}c@{$\;$}X}
		$\Rightarrow$ & $f \in L_0[X]$\\
		$\Rightarrow$ & $[L_0(\alpha): L_0] \le \deg(f) < \infty$\\
		$\Rightarrow$ & $[K(\alpha): K] \le [K(a_0,\dots,a_n,\alpha):K] = \underbrace{[L_0(\alpha):L_0]}_{< \infty}\underbrace{[L_0 :K]}_{< \text{ nach } \propref{1_2_7}}$ \\
		$\Rightarrow$ &  $\alpha$ algebraisch über $K$ \\
		$\overset{\alpha \text{ bel.}}{\Rightarrow}$ & $M \mid K$ algebraisch.\hfill\csname\InTheoType Symbol\endcsname
		\end{tabularx}
	\end{itemize}
\end{proof}

\begin{conclusion}
	\proplbl{1_2_15}
	$\tilde{K} = \set{\alpha \in L\mid \alpha \text{ algebraisch über }K}$ ist ein Körper, und ist $\alpha \in L$ algebraisch über $\tilde{K}$, so ist schon $\alpha \in \tilde{K}$.
\end{conclusion}

\begin{proof}\leavevmode\vspace*{\dimexpr-\baselineskip+2\lineskip} %TODO find a good way to format the RIGHTARROWS?
	\begin{itemize}
		\item $\alpha$, $\beta \in \tilde{K}$:\\
			\hspace*{0.5em}$\Rightarrow$ $K(\alpha, \beta)\mid K$ endlich, insbesondere algebraisch\\
			\hspace*{0.5em}$\Rightarrow$ $\alpha + \beta$, $\alpha - \beta$, $\alpha \cdot \beta$, $\alpha^{-1} \in K(\alpha,\beta)$ alle algebraisch über $K$, also $K(\alpha, \beta) \subseteq \tilde{K}$.
		\item $\alpha \in L$ algebraisch über $\tilde{K}$:\\
			\hspace*{0.5em}$\Rightarrow \tilde{K}(\alpha)\mid \tilde{K}$ algebraisch\\
			\hspace*{0.5em}$\Rightarrow \tilde{K}\mid K$ algebraisch \\
			\hspace*{0.5em}$\overset{\mathclap{\propref{1_2_14}}}{\Rightarrow}$ $\tilde{K}(\alpha)\mid K$ algebraisch, insbesondere $\alpha \in \tilde{K}$.
	\end{itemize}
\end{proof}

\begin{definition}[relative algebraische Abschluss]
	$\tilde{K} = \set{\alpha \in L\mid \alpha \text{ algebraisch über }K}$ heißt der \begriff{relative algebraische Abschluss} von $K$ in $L$.
\end{definition}

\begin{example}
	$\tilde{\Q} = \set{\alpha \in \C \mid \alpha \text{ algebraisch über }\Q}$ ist ein Körper, der Körper der algebraischen Zahlen. Es ist $[\tilde{\Q}:\Q] = \infty$, z.B. da $[\Q(\zeta_p):\Q] = p-1$ für jedes $p$ prim. (algebraische Erweiterung die nicht endlich ist.)
\end{example}