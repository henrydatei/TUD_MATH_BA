\section{Separable Polynome}
Sei $K$ ein Körper, $f \in K[X]$, $n = \deg(f)$.
\begin{definition}
	Sei $a \in K$.
	\begin{enumerate}[label={(\arabic*)}]
		\item $\mu(f,a) := v_{x-a}(f) := \sup \set{k \in \N_0 : (x-a)^k \mid f} \in \N_0 \cup \set{\infty}$ die \begriff{Vielfachheit} der Nullstelle $a$ von $f$
		\item Nullstelle $a$ von $f$ ist \begriff{einfach} :$\Leftrightarrow$ $\mu(f,a) = 1$
		\item $f$ ist \begriff{separabel} :$\Leftrightarrow$ jede Nullstelle $a\in\bar K$ von $f\in\bar K[X]$ ist einfach.
	\end{enumerate}
\end{definition}
\begin{remark}
	\begin{enumerate}[label={(\alph*)}]
		\item Ist $L\mid K$ eine Körpererweiterung und $g\in K[X]$, so gilt \begin{flalign*}
			\qquad & f\mid g\;\text{in}\; K[X]\quad\Leftrightarrow\quad f\mid g\;\text{in}\; L[X]
		\end{flalign*}
		Insbesondere ist die Nullstelle $\mu_K(f,a) = \mu_L(f,a)$. Wir können deshalb von der Vielfachheit der Nullstelle von $f$ sprechen.
		\item \proplbl{1_6_2_b} $\displaystyle\#\{a\in K\mid f(a) = 0\} \le \sum_{a\in K} \mu(f,a) \le \sum_{a\in \bar K} \mu(f,a) = \deg(f)$, falls ($f\neq 0$)
		\item Aus \ref{1_6_2_b} folgt insbesondere:\\
		\begin{tabularx}{\linewidth}{XcX}
			\hfill$f$ ist separabel & $\Leftrightarrow$ & $f$ hat genau $\deg(f)$ paarweise verschiedene Nullstellen in $\bar K$
		\end{tabularx}
	\end{enumerate}
\end{remark}
\begin{definition}
	Die \begriff{formale Ableitung} von $f = \sum_{i=1}^n a_i X^{i-1}$ ist \begin{flalign*}
		\qquad &f' := \frac{\d}{\d x} f(x) := \sum_{i=1} i a_i X^{i-1} &
	\end{flalign*}
\end{definition}
\begin{lemma}
	Für $f,g \in K[X], a,b \in K$ gelten\begin{enumerate}[label={(\alph*)}]
		\item $(af + bg)' = a f' + b g'$ (Linearität)
		\item $(fg)' = f'g + fg'$ (Produktregel)
		\item $(f(g(x)))' = f'(g(x))\cdot g'(x)$ (Kettenregel)
	\end{enumerate}
\end{lemma}
\begin{proof}
	Übung.
\end{proof}
\begin{lemma}
	\proplbl{1_6_5}
	Sei $f \neq 0$. Für $a \in K$ gilt
	\begin{flalign*}
		\qquad&\mu(f' , a) \ge \mu(f,a) - 1&
	\end{flalign*}
	mit Gleichheit genau dann, wenn $\chara(K) \nmid \mu(f,a)$.
\end{lemma}
\begin{proof}
	Schreibe $f = (X-a)^k \cdot g$, $k = \mu(f,a)$
	\begin{itemize}[topsep=-.5em,left=3.5em]
		\item[$k=0$:] $\mu(f', a) \ge 0 > -1$ und $\chara(K) \mid 0$
		\item[$k>0$:] $f' = k(X-a)^{k-1}g + (X-a)^k \cdot g'$ $\implies \mu(f',a) \ge k$, sowie 
		
		\vspace*{-2pt}
		\begin{tabular}{@{}>{$}r<{$}>{$}c<{$}>{$}l<{$}}
			\mu(f', a) \ge k & \Leftrightarrow & (X-a)^k \mid k(X-a)^{k-1}\cdot g \\
							 & \Leftrightarrow & X-a \mid k\cdot g \\
							 & \Leftrightarrow & X-a\mid k \\
							 & \Leftrightarrow & k=0\;\text{in}\; K\\
							 & \Leftrightarrow & \chara(K) \mid k
		\end{tabular}
	\end{itemize}
\end{proof}

\begin{proposition}
	\proplbl{1_6_6}
	Sei $f\neq 0$. Dann gilt:
	
	\begin{tabular}{@{$\qquad$}rcl}
		$f$ separabel & $\Leftrightarrow$ & $\ggT(f,f') = 1$
	\end{tabular}
\end{proposition}

\begin{proof}\leavevmode\vspace*{\dimexpr-\baselineskip+2\lineskip}
	\begin{itemize}
		\item[($\Rightarrow$)] $f$ separabel \\[-0.2em]
			\begin{tabular}[t]{@{}>{$}r<{$}@{$\;$}l}
			\Rightarrow & $f = c\cdot\prod_{i=1}^{n}(X-a_i)$ mit $c\in K$, $a_1$, $\dots$, $a_n\in \bar K$ paarweise verschieden und $\mu(f,a_i) = 1$ \\
			\xRightarrow[\chara(K)\nmid 1]{\propref{1_6_5}} & $\mu(f',a_i) = 0$ $\forall i$\\
			\Rightarrow & $\displaystyle\ggT(f,f') = \prod_{a\in\bar K} (X-a)^{\min\{ \mu(f,a), \mu(f',a) \}} = 1$
		\end{tabular}
		\item[($\Leftarrow$)] $f$ nicht separabel $\Rightarrow$ $\exists a\in\bar K$ mit $\mu(f,a)\ge 2$ $\xRightarrow{\propref{1_6_5}}$ $\mu(f',a)\ge 1$.
		
		Mit $g = \MinPol(a\mid K)$ gilt: $g\mid f$ $\Rightarrow$ $\ggT(f,f') \neq 1$
	\end{itemize}
\end{proof}

\begin{lemma}
	$f' = 0$ $\Leftrightarrow$ $\exists g\in K[X]$ mit $f(X) = g(X^p)$ und $p=\chara(K)$.
\end{lemma}
\begin{proof}
	Ist $f = \sum_{i=1}^{n} a_iX^i$ $\Rightarrow$ $f' = \sum_{i=1}^{n}i a_{i-1}X^{i-1}$ und \\[-0.3em]
	\begin{tabular}{@{}r>{$}c<{$}l}
		$f' = 0$	& \Leftrightarrow & $i a_i = 0$ in $K$ $\forall i$\\
					& \Leftrightarrow & $\forall i$: $i = 0$ in $K$ oder $a_i = 0$ \\
					& \Leftrightarrow & $f = a_0 + a_p X^p + \dots + a_{pm} X^{pm} = g(X^p)$ mit $g = a_0 + a_p X + \dots + a_{pm} X^m$
	\end{tabular}
\end{proof}

%TODO
\begin{conclusion}
	\proplbl{1_6_8}
	Sei $f$ irreduzibel
	\begin{enumerate}[label={(\alph*)}]
		\item Ist $\chara(K) = 0$, so ist $f$ separabel
		\item Ist $\chara(K) = p>0$, so sind äquivalent
		\begin{enumerate}[label={(\arabic*)}]
			\item $f$ ist inseparabel
			\item $f' = 0$
			\item $f(X) = g(X^p)$ für ein $g \in K[X]$
		\end{enumerate}
	\end{enumerate}
\end{conclusion}
\begin{proof}
	$f$ irreduzibel $\implies \underbrace{\ggT(f,f') \sim 1}_{\xLeftrightarrow{\propref{1_6_6}} f \text{ sep}} \oder \underbrace{\ggT(f,f') \sim f}_{\xLeftrightarrow{\propref{1_6_6}} f \text{ sep.}}$.
	
	Da $\deg(f') = \deg(f)$ ist
	\begin{flalign*}
		\qquad & f \mid f' \quad \iff \quad  f' = 0 \quad \iff \quad f(X) = g(X^p) \; \text{für ein}\;g &
	\end{flalign*}
	Im Fall $\chara(K) = 0$ tritt dieser Fall nicht ein.
\end{proof}
\begin{definition}[vollkommen]
	$K$ ist \begriff{vollkommen} $\iff$ jedes irreduzibel $f \in K[X]$ ist separabel.
\end{definition}
\begin{example}
	\begin{expenum}
		\item \proplbl{1_6_12_a} $\chara(K) = 0 \implies K$ ist vollkommen
		\item $K = \bar{K} \implies K$ ist vollkommen
		\item $K = \F_p (t)$ ist nicht vollkommen:
		\begin{flalign*}
			\qquad f &= X^p - t \in K[X] \text{ ist irreduzibel} &\\
			f' &= pX^{p-1} = 0 \implies f \text{ nicht seperabel.} &
		\end{flalign*}
		Tatsächlich hat $f$ nur eine Nullstelle in $\bar{K}$: $f = X^p - t \overset{\text{V1}}{=} (X - t^{\frac{1}{p}})^p.$
	\end{expenum}
\end{example}
\begin{definition}
	Sei $\chara(K) = p > 0$.
	\begin{enumerate}[label={(\arabic*)}]
		\item Der \person{Frobenius}-Endomorphismus von $K$ ist
		\begin{flalign*}
		\qquad &\Phi_p\colon \left\lbrace\begin{array}{@{}l@{\;}c@{\;}l}
		K &\to& K\\
		X &\mapsto& X^p 
		\end{array}\right. &
		\end{flalign*}
		\item $K^p = \Image(\Phi_p) = \set{a^p \mid a \in K}$
	\end{enumerate}
\end{definition}
\begin{proposition}
	Sei $\chara(K) = p > 0$. Dann ist $\Phi_p \in \End(K): =\Hom(K,K)$
\end{proposition}
\begin{proof}
	Für $a, b \in K$ ist
	\begin{itemize}[topsep=-6pt]
		\item $\Phi_p = (ab)^p = a^p \cdot b^p = \Phi_p (a) \cdot \Phi_p(b)$
		\item $\Phi_p(a+b) = (a+b)^p = \sum_{i=0}^p\binom{p}{i} a^i b^{p-i} = b^p + a^p = \Phi_p(a) + \Phi_p(b)$, da $p \mid \binom{p}{i}$ für $i = 1, \dots, p-1$ (V1).
		\item $\Phi_p(1) = 1^p = 1$
	\end{itemize}
\end{proof}
\begin{remark}
	\begin{remarkenum}
		\item \proplbl{1_6_13_a} Da $\Phi_p \in \End(K) $ ist $K^p$ ein Teilkörper von $K$ und $\Phi_p$ ist injektiv.
		\item Insbesondere gibt es zu jedem $a \in K$ ein eindeutig bestimmtes $a^{\frac{1}{p}} \in \bar{K}$ mit
		\begin{flalign*}
			\qquad & \Phi_p(a^{\frac{1}{p}}) = (a^{\frac{1}{p}})^p = a &
		\end{flalign*}
		\item Für $a \in \F \cong \F_p$ ist $\Phi_p(a) = a$. (z.B. $\Phi_p(1) = 1$ oder kleiner Satz von \person{Fermat})
	\end{remarkenum}
\end{remark}
\begin{lemma}
	\proplbl{1_6_14}
	Sei $\chara(K) = p > 0$, $a \in K \setminus K^p$. Dann ist $f = X^p -a$ irreduzibel und inseparabel
\end{lemma}
\begin{proof}
	Sei $\alpha \in \bar{K}$ mit $f(\alpha) = 0$, $g= \MinPol(\alpha \mid K)$
	\begin{itemize}[topsep=-6pt]
	\item[$\implies$] $g \mid f = X^p - \alpha = (X-\alpha)^p$
	\item[$\implies$] $g \equiv (X - \alpha)^k$ mit $k \le p$. 
	\end{itemize}
	\medskip
	$a \notin K^p$
	\begin{itemize}[topsep=-6pt,widest=$\xRightarrow{g \text{ irred.}}$,leftmargin=*]
	\item [$\implies$] $\alpha \notin K \implies k >1$
	\item[$\implies$] $g$ ist inseperabel
	\item[$\xRightarrow{g \text{ irred.}}$] $g(X) = h(X^p)$ für ein $h$
	\item[$\implies$] $k = p$ $\implies f = g$ irreduzibel 
	\end{itemize}
\end{proof}

\begin{proposition}
	\proplbl{1_6_15}
	Genau dann ist $K$ vollkommen, wenn \begin{enumerate}[label={(\roman*)}]
		\item $\chara(K) = 0$ oder
		\item $\chara(K) = p > 0$ und $K^P = K$
	\end{enumerate}
\end{proposition}
\begin{proof}
\leavevmode
\begin{itemize}[topsep=-6pt]
\item $\chara(K) = 0$: klar (\cref{1_6_12_a})
\item $\chara(K) = p > 0$: \begin{itemize}
	\item[($\Rightarrow$)] Es existiert ein $a\in K\setminus K^p$, so ist $K$ nicht vollkommen nach \cref{1_6_14}.
	\item[($\Leftarrow$)] Sei $f(X)\in K[X]$ irreduzibel und inseparabel. Nach \cref{1_6_8} existiert ein $g(X)\in K[X]$ mit \begin{flalign*}
		\qquad & f(X) = g(X^p) &
	\end{flalign*}
	Setze $g(X) = \sum_{i=0}^n a_i X^i\in K[X]$. Dann ist \begin{flalign*}
		\qquad & f(X) = g(X^p) = \sum_{i=0}^n a_i \big(X^i)^p \overset{=}{\text{V1}} \Bigg(\sum_{i=0}^n \underbrace{a_i^{1\!\slash\!n}}_{\mathclap{\text{$\in K$ da $K^p=K$}}} X^i\Bigg)^p, &
	\end{flalign*}
	folglich ein Widerspruch.
	\end{itemize}
\end{itemize}
\end{proof}
\begin{example}
	\proplbl{1_6_16}
	$K$ endlich $\Rightarrow$ $K$ vollkommen (\propref{1_6_13_a}, \propref{1_6_15}).
\end{example}