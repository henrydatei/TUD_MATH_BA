\section{Körpererweiterungen}

Seien $K,L,M$ Körper.

\begin{remark}
	In diesem Kapitel bedeutet ``Ring'' \emph{immer} kommutativer Ring mit Einselement, und ein Ringhomomorphismus bildet stets das Einselement auf das Einselement ab.
	Insbesondere gibt es für jeden Ring einen eindeutig bestimmten Ringhomomorphismus $\Z \to R$.
\end{remark}

\begin{remark}
	\proplbl{1_1_2}
	\begin{remarkenum}[]
		\item Ein \emph{Körper} ist ein Ring $R$, in dem eine der folgenden äquivalenten Bedingungen gilt:
		\begin{enumerate}[label=\arabic*)]
			\item $0 \neq 1$ und jedes $0 \neq x \in R$ ist invertierbar
			\item $R^{\times} = R \setminus \set{0}$
			\item $R$ hat genau zwei Hauptideale (nämlich $(0)$ und $(1)$)
			\item $(0)$ ist ein maximales Ideal von $R$
			\item $(0)$ ist das einzige echte Ideal von $R$
			\item $(0)$ ist das einzige Primideal von $R$
		\end{enumerate}
		\item Insbesondere sind Körper \emph{nullteilerfrei}, weshalb $\Ker(\Z \to K)$ prim ist.
		\item Aus (5) folgt: Jeder Ringhomomorphismus $K \to L$ ist injektiv %TODO add ref to (5) and check the ringhomo!
		\item \proplbl{koerpererweiterungen:rem:1.2.d} Der Durchschnitt einer Familie von Teilkörpern von $K$ ist wieder ein Teilkörper von $K$.
	\end{remarkenum}
\end{remark}

\begin{definition}[Charakteristik]
	Die \begriff{Charakteristik} von $K$, $\chara(K)$, ist das $p \in \set{0,2,3,5,7, \dots}$ mit $\Ker(\Z \to K) = (p)$.
\end{definition}

\begin{example}
	\proplbl{1_1_4}
	\begin{enumerate}[label=(\alph*)]
		\item $\chara(\Q) = 0$ und $\chara(\Fp) = (p)$ ($p =$ Primzahl), wobei $\Fp = \lnkset{\Z}{p\Z}$
		\item Ist $K_0 \subseteq K$ Teilkörper, so ist $\chara(K_0) = \chara(K)$.
	\end{enumerate}
\end{example}

\begin{definition}[Primkörper]
	Der \begriff{Primkörper} von $K$ ist der kleinste Teilkörper von $K$. (existiert nach \propref{koerpererweiterungen:rem:1.2.d}).
\end{definition}

\begin{proposition}
	Sei $\field$ der Primkörper von $K$.
	\begin{enumerate}[label=(\alph*)]
		\item $\chara(K)  = 0$ $\Leftrightarrow$ $\field \cong \Q$
		\item $\chara(K)  = p > 0$ $\Leftrightarrow$ $\field \cong \Fp$
	\end{enumerate}
\end{proposition}

\begin{proof}\hspace{0pt}
	\vspace*{\dimexpr-\baselineskip+2\lineskip\relax}
	\begin{itemize}
	\item[($\Leftarrow$)] \propref{1_1_4} % (beide?)
	\item[($\Rightarrow$)] Es ist $\Image(\Z \to K) \subseteq \field$ und $\Image(\Z \to K) \cong \lnkset{\Z}{\Ker(\Z \to K)}$, sowie
		\begin{enumerate}[label=(\alph*)]
			\item $\Image(\Z \to K) \cong \lnkset{Z}{(0)} \cong \Z$ $\Rightarrow$ $\field = \Quot(\Image(\Z \to K)) \cong \Quot(\Z) \cong \Q$
			\item $\Image(\Z \to K) \cong \lnkset{Z}{(p)} \cong \Fp$ ist Teilkörper von K $\Rightarrow$ $\field = \Image(\Z \to K) \cong \Fp$
		\end{enumerate}
	\end{itemize}
\end{proof}

\begin{definition}[Körpererweiterung]
	Ist $K$ ein Teilkörper von $L$, so nennt man $L$ eine \begriff{Köpererweiterung} von $K$, auch geschrieben $L \mid K$.
\end{definition}

\begin{definition}[$K$-Homomorphismus]
	Seien $L_1 \mid K$ und $L_2 \mid K$ Körpererweiterungen.
	\begin{enumerate}[label=(\alph*)]
		\item Ein Ringhomomorphismus $\phi\colon L_1 \to L_2$ ist ein $K$-Homomorphismus, wenn $\phi\vert_K = \id_K$ (i.Z. $\phi\colon L_1 \to_K L_2$)
		\item $\Hom_K(L_1,L_2) = \set{\phi \mid \phi\colon L_1 \to L_2 \text{ ist $K$-Homomorphismus}}$
		\item $L_1$ und $L_2$ sind $K$-isomorph (i.Z. $L_1 \cong_K L_2$), wenn es einen Isomorphismus $\phi \in \Hom_K(L_1, L_2)$ gibt.
	\end{enumerate}
\end{definition}

\begin{remark}
	Ist $L\mid K$ eine Körpererweiterung, so wird $L$ durch Einschränkung der Multiplikation zu einem $K$-Vektorraum.
\end{remark}

\begin{definition}[Körpergrad]
	Es ist $[L:K]:= \dim_K(L) \in \N \cup \{\infty\}$ der \begriff{Körpergrad} der Körpererweiterungen $L\mid K$.
\end{definition}

\begin{example}
	\begin{enumerate}[label=(\alph*)]
		\item $[K: K] = 1$
		\item $[\C:\R] = 2$ (Basis $(1,i)$) (aber $(\C:\R) = \infty$)
		\item $[\R:\Q] = \infty$ (mit Abzählarbarkeitsargument oder siehe \cref{sec:sec_2})
		\item $[K(X):K] = \infty$ ($K(X) = \Quot(K[X])$ (vgl. GEO II.8)
	\end{enumerate}
\end{example}

\begin{proposition}
	\proplbl{1_1_12}
	Für $K \subseteq L \subseteq M$ Körper ist $[M:K] = [M:L]\cdot [L:K]$ \hspace*{1.5em} (``Körpergrad ist multiplikativ'')
\end{proposition}

\bgroup
\begin{proof}
	Für den Beweis betrachte folgende Aussage:
	\begin{adjustwidth}{1em}{0pt}
	\begin{underlinedenvironment}[Behauptung]
			Sei $x_1$, $\dots$, $x_n \in L$ $K$-linear unabhängig und $y_1$, $\dots$, $y_m \in M$ $L$-linear unabhängig\\[2\lineskip]
	$\Rightarrow \big\lbrace x_i y_j \mid  i \in \set{1,\dots,n}, j \in \set{1, \dots, m}\big\rbrace$ $K$-linear unabhängig.
		\end{underlinedenvironment}
	
	\begin{underlinedenvironment}[Beweis]
			 Sei $\sum_{i,j} \lambda_{ij}x_i y_j = 0$ mit $\lambda_{ij} \in K$
			 \vspace*{2\lineskip}
			 \zeroAmsmathAlignVSpaces
			 \begin{flalign*}
			\Rightarrow\; & \sum_{j}\underbrace{\bigg[\! \sum_{i} \lambda_{ij}x_i \bigg]}_{\in L}y_j = 0 
			\quad\; \xRightarrow{y_j\, L\text{-l.u.}}\; \sum_{i} \lambda_{ij} x_i = 0\quad\forall j
			\quad\; \xRightarrow{x_i\, K\text{-l.u.}}\; \lambda_{ij} = 0\quad\forall i, \forall j&
			\end{flalign*}
			\hfill\proofSymbol
	\end{underlinedenvironment}
	\end{adjustwidth}
	Dann:
	\vspace*{\dimexpr-\baselineskip+2\lineskip}
	\begin{itemize}
		\item $[L:K] = \infty$ oder $[M:L] = \infty$ $\Rightarrow$ $[M:K] = \infty$
		\item $[L:K] = n$, $[M:L] = m < \infty$:\\[2\lineskip]
		Sei $(x_1, \dots, x_n)$ Basis des $K$-Vektorraum $L$ und $(y_1, \dots, y_m)$ Basis des $L$-Vektorraums M\\
		\hspace*{0.5em}$\Rightarrow\;$\begin{minipage}[t]{\dimexpr\linewidth-\labelindent+0.5em}
		 $\set{x_i y_j \mid i = 1,\, \dots,\, n;\; j = 1,\, \dots,\, m}$ $K$-linear unabhängig und
		\begin{flalign*}
			\quad&\sum_{i,j} Kx_i y_j = \sum_{j}\underbrace{\!\bigg[\! \sum_{i} \lambda_{ij}x_i \bigg]\!}_{=L} y_j = M,&
		\end{flalign*}
		also ist $\set{x_i y_j \mid i = 1,\, \dots ,\,n;\; j = 1,\, \dots,\, m}$ Basis von $M$.\hfill\csname\InTheoType Symbol\endcsname
		\end{minipage}
	\end{itemize}
	\setendmarkfalse
\end{proof}
\egroup

\begin{definition}[Körpergrad endlich]
	$L\mid K$ endlich $:\Leftrightarrow [L:K] < \infty$.
\end{definition}

%%%%%%%%%%%%%%%%%%%%%%%%%%%% 2nd lecture %%%%%%%%%%%%%%%%%%%%%%%%%%%%%%%%%%%%%%%%%%%%%%%%%%%%%%

\begin{definition}[Unterring, Teilkörper]
	Sei $L\mid K$ eine Körpererweiterung $a_1$, $a_2$, $\dots$, $a_n \in L$.
	\begin{enumerate}[label=(\alph*)]
		\item $K[a_1,\dots,a_n]$ ist kleinster \begriff{Unterring} von $L$, der $K \cup \set{a_1,\, \dots,\, a_n}$ enthält ("`$a_1, \dots, a_n$ über $K$ erzeugt'')
		\item $K(a_1,\dots,a_n)$ ist kleinster \begriff{Teilkörper} von $L$, der $K \cup \set{a_1,\, \dots,\, a_n}$ enthält (von "`$a_1$, $\dots$, $a_n$ über $K$ erzeugte'', "`$a_1$, $\dots$, $a_n$'' zu $K$ adjungieren)
		\item $L | K$ ist \begriff{endlich erzeugt} $:\Leftrightarrow$ $a_1$, $\dots$, $a_n \in L$: $L=K(a_1, \dots, a_n)$
		\item $L | K$ ist \begriff{einfach} $:\Leftrightarrow$ existiert $a \in L$: $L=K(a)$  
	\end{enumerate}
\end{definition}

\begin{remark}
	\proplbl{1_1_15}
	\begin{enumerate}[label=(\alph*)]
		\item $L\mid K$ endlich $\Rightarrow L\mid K$ endlich erzeugt.
		\item $K[a_1, \dots, a_n]$ ist das Bild des Homomorphismus
		\begin{align*}
		\left\lbrace\begin{array}{@{}l@{\;}c@{\;}l}
		K[X_1, \dots, X_n] &\to& L\\
		f &\mapsto& f(a_1, \dots, a_n)
		\end{array}\right.
		\end{align*}
		und $K(a_1, \dots , a_n) = \big\{\alpha\mskip-1mu\big\slash\mskip-1mu \beta \mid \alpha, \beta \in K[a_1, \dots, a_n],\, \beta \neq 0\big\} \cong \Quot\big(K[a_1, \dots, a_n]\big)$
	\end{enumerate}
\end{remark}