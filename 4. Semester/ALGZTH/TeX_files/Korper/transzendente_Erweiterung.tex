\section{Die transzendente Erweiterung}
Sei $L\mid K$ eine Körpererweiterung.
\begin{definition}[algebraisch abhängig]
	\begin{enumerate}[label=(\alph*)]
		\item $a_1, \dots, a_n \in L$ \begriff{algebraisch abhängig} über $K$ $: \iff $ existiert \\$0 \neq f \in K(X_1,\dots, X_n) \colon f(a_1, \dots, a_n) = 0$
		\item $(a_i)_{i\in I}$ ist \begriff{algebraisch abhängig} über $K$ $:\iff$ existiert $J \subseteq I$ endlich: $(a_i)_{i\in I}$ algebraisch abhängig über $K$
	\end{enumerate}
\end{definition}
\begin{*example}[nicht aus VL, sondern ergänzt!]
	Betrachte die reellen Zahlen $\sqrt{\pi} \und 2\pi +1$, beide sind transzendent über $\Q$. Die Singletons $\set{\sqrt{\pi}}\und \set{2\pi +1}$ sind algebraisch unabhängig über $\Q$. Aber die Vereinigung $\set{\sqrt{\pi}, 2\pi +1}$ ist nicht algebraisch unabhängig in $\Q$, da
	\begin{align*}
		P(x,y) = 2x^2 - y + 1 = 0
	\end{align*}
	ist, wenn $x = \sqrt{\pi} \und y = 2\pi +1$ gesetzt sind.
\end{*example}
\begin{remark}
	\begin{enumerate}[label=(\alph*)]
		\item $(a)$ ist algebraisch abhängig über $K \iff a$ ist algebraisch über $K$
		\item $L = K(X_1,\dots, X_n) = \Quot(K([X_1,\dots, X_n])) \implies X_1,\dots, X_n$ sind algebraisch unabhängig über $K$
		\item Sind $\pi, e$ unabhängig über $\Q$?\\
		Falls ``Ja'', wäre z.B. $\pi+e$ transzendent über $\Q$
	\end{enumerate}
\end{remark}
\begin{definition}[rein transzendent]
	$L \mid K$ \begriff{rein transzendent} $:\iff L = K(\Halb) \mit \Halb = (a_i)_{i\in I}$ algebraisch unabhängig über $K$.
\end{definition}
\begin{lemma}
	\proplbl{1_5_4}
	$\Halb = (a_i)_{i \in I}$ algebraisch unabhängig über $K \implies K(\Halb) \cong_K K(X_i \colon i \in I) = \Quot(K[X_i \colon i \in I])$. 
\end{lemma}
\begin{proof}
	Betrachte $K$-Isomorphismus
	\begin{align*}
		\varphi = \begin{cases}
			K[X_i \colon I \in I] &\to K[a_i : i \in I]\\
			f & \mapsto f(\Halb)
		\end{cases} 
	\end{align*}
	($a_i$ für $x_i$ einsetzen.) Da $\Halb$ algebraisch unabhängig über $K$, ist $\Ker(\varphi) = (0)$\\
	$\implies K(\Halb) = \Quot(K[\Halb]) \cong_K \Quot(K[X_i : i \in I])$.
\end{proof}
\begin{proposition}
	$L\mid K$ rein transzendent $\implies \tilde{K} = K$.
\end{proposition}
\begin{proof}
	Nach \propref{1_5_4} o.E. $L = K(X_i : i \in I)$. Weiter o. E. $I = \set{1, \dots,n}$ endlich. Sei $\alpha \in L$ algebraisch über $K$. Definiere $f = \MinPol(\alpha \mid K)$.\\
	$f$ irreduzibel in $K[X] \xRightarrow{\text{\person{Gauss}}} f$ irreduzibel in $K[X_1, \dots, X_n][X]$\\
	$\xRightarrow{\text{\person{Gauss}}} f$ irreduzibel $K(X_1, \dots, X_n)[X]$\\
	$\implies \deg(f) = 1$\\
	$\implies \alpha \in K$.
\end{proof}
\begin{remark}
	Die Umkehrung gilt nicht, da z.B. $L = \C$. Sei $K = \tilde{\Q}$, dann $\tilde{K} = K$, aber $L\mid K$ nicht rein transzendent. Ist $L = K[\Halb]$, $\Halb = (a_i)_{i\in I}$ algebraisch unabhängig, so wäre $a_i \in L$ aber $\sqrt{a_i} \in \bar{L}\setminus L$.
\end{remark}
\begin{definition}[Transzendentbasis]
	$\Halb = (a_i)_{i\in I}$ ist eine \begriff{Transzendentbasis} von $L \mid K : \iff \Halb$ ist algebraisch unabhängig über $K$ und $L\mid K(\Halb)$ algebraisch.
\end{definition}
\begin{lemma}
	\proplbl{1_5_8}
	$\Halb = (a_i)_{i \in I} \subseteq L$ ist genau dann eine Transzendentbasis von $L \mid K$, wenn $\Halb$ maximal algebraisch unabhängig über $K$ ist.
\end{lemma}
\begin{proof}
	\begin{itemize}
		\item ``$\Longleftarrow$'': $a \in L \mid K(\Halb) \xRightarrow{\text{maximal}} \Halb \cup \set{a}$ algebraisch abhängig, d.h. existieren $i_1, \dots, i_n \in I, 0 \neq f \in K[x_1, \dots, x_n, x]$ mit $f(a_{i_1}, \dots, a_{i_n}, a) = 0$\\ 
		$a_{i_1}, \dots, a_{i_n}$ unabhängig über $K$\\
		$\implies \underbrace{f(a_{i_1}, \dots, a_{i_n}, x)}_{\in K(a_{i_1}, \dots, a_{i_n})[X]\subseteq K(\Halb)[X]}\neq 0$\\
		$\implies a$ ist algebraisch über $K(\Halb)$
		\item ``$\implies$'': $a \in L\setminus K(\Halb) \xRightarrow{L \setminus K(\Halb) \text{ alg.}}$ existiert $0 \neq f \in K(\Halb)[X] \mit f(a) = 0$\\
		O.E. (Problem: Nenner kann Koeffizienten in $K(\Halb)$ haben $\to$ Multiplikation mit Nenner, weil $f(a) =0$) $f \in K[\Halb][X]$, d.h. es existiert $g \in K[X_1, \dots, X_n][X]$ und $i_1,\dots, i_n \in I \mit f(X) = g(a_{i_1},\dots, a_{i_n}, x)$ und $\Halb \cup \set{a}$ ist algebraisch abhängig.
	\end{itemize}
\end{proof}
\begin{proposition}
	Es gibt eine Transzendenzbasis von $L\mid K$.
\end{proposition}
\begin{proof}
	Nach Lemma von \person{Zorn} gibt es eine Familie $\Halb$ in $L$, die maximal algebraisch unabhängig über $K$ ist.
\end{proof}
\begin{conclusion}
	Jede Erweiterung $L\mid K$ lässt sich zerlegen als
\begin{center} % tikzcd was bitchy, compiled and included the pdf.
	\includegraphics{./tikz/lemma_1_5_10.pdf}
\end{center}
\end{conclusion}
\begin{lemma}
	Ist $\caly = (b_j)_{j \in J}$ mit $L\mid K(\caly)$ algebraisch und $\Halb = (a_i)_{i\in I}$ algebraisch unabhängig über $K$, so existiert $J_0 \subseteq J$ mit $\Halb \cup (b_j)_{j \in J_0}$ eine Transzendenzbasis von $L \mid K$.
\end{lemma}
\begin{proof}
	Nach dem Lemma von \person{Zorn} existiert $J_0 \subseteq J$ maximal mit $\Halb' = \Halb \cup (b_j)_{j \in J_0}$ algebraisch unabhängig über $K$. Für jedes $j \in J$ ist $\Halb' \cup \set{b_j}$ algebraisch abhängig über $K$, somit $b_j$ algebraisch über $K(\Halb') \implies K(\Halb \cup \caly) \mid K(\Halb')$ algebraisch\\
	$L \mid K(\caly)$ algebraisch $\implies L\mid K(\Halb \cup \caly)$ algebraisch $\xRightarrow{\text{alg. transitiv}} L \mid K(\Halb')$ algebraisch. Somit ist $\Halb'$ eine Transzendenzbasis.
\end{proof}
\begin{theorem}[Steinitz, 1910]
	Je zwei Transzendenzbasen von $L \mid K$ besitzen die gleiche Mächtigkeit.
\end{theorem}
\begin{proof}
	Seien $\Halb = (a_i)_{i\in I}, \caly = (b_j)_{j \in J}$ Transzendenzbasis von $L \mid K$.\\ 
	Beweisen hier nur für $J$ endlich:\\
	Wegen Symmetrie genügt es zu zeigen, dass $\abs{I} \le \abs{J}$.\\
	Induktion nach $n = \abs{J}$:
	\begin{itemize}
		\item (IA) $n =0:$ $L \mid K$ algebraisch $\implies \abs{I} = 0$
		\item (IS) $n > 0:$ $L \mid K$ nicht algebraisch $\implies \abs{I} > 0$ %TODO
	\end{itemize}
\end{proof}
\begin{definition}[Transzendenzgrad]
	Der \begriff{Transzendenzgrad} von $L \mid K$ ist die Mächtigkeit $\transdeg(L\mid K)$ einer Tarnszendenzbasis von $L \mid K$.
\end{definition}
\begin{conclusion}
	Sind $L \subseteq L \subseteq M$ Körper, so ist
	\[
		\transdeg(M\mid K) = \transdeg(M \mid L) + \transdeg(L \mid K).
	\]
\end{conclusion}
\begin{proof}
	... %TODO
\end{proof}