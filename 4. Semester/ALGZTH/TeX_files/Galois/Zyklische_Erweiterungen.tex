\section{Zyklische Erweiterungen}
Sei $L \mid K$ eine endliche Galoiserweiterung.
\begin{definition}
	$L\mid K$ ist \begriff{zyklisch} (\begriff{abelsch}, \begriff{auflösbar}) $:\Leftrightarrow$ $\Gal(L\mid K)$ ist zyklisch (bzw. abelsch, auflösbar).
\end{definition}
\begin{remark}[Erinnerung]
	Sei $\Hom_K(L,\overline{K}) = \set{\sigma_1,\dots, \sigma_n}$
	\begin{itemize}
		\item $N_{L \mid K}(\alpha) = \det(\mu_{\alpha}) = \prod_{i=1}^n \sigma_i(\alpha)$
		\item $\Tr_{L\mid K}(\alpha) = \Tr(\mu_{\alpha}) = \sum_{i=1}^n \sigma_i(\alpha)$
		\item $L \mid K$ galoissch $\implies$ $\Hom_K(L, \bar K) = \Gal(L\mid K)$ (wie üblich nehmen wir hier an, dass $L \subseteq \overline{K}$)
	\end{itemize}
\end{remark}
\begin{proposition}[Hilbert 90, multiplikativ]
	\proplbl{2_7_3}
	Sei $L \mid K$ endlich zyklisch mit $\Gal(L \mid K) = \langle \sigma\rangle$. Für $b \in L$ gilt:
	\begin{align*}
		N_{L\mid K}(b)  = 1 \quad \Leftrightarrow \quad b = \frac{a}{\sigma(a)} \text{ für ein } a \in L^{\times}.
	\end{align*}
\end{proposition}

\begin{proof}
	Sei $n= [L:K]$.
	\begin{itemize}[topsep=-6pt]
		\item[($\Leftarrow$)]
		Ist $b = \frac{a}{\sigma(a)}$, so folgt \begin{align*}
			N_{L\mid K} (b) &= \prod_{\sigma \in \Gal(L\mid K)} \sigma(b)
			= \prod_{i=0}^{n-1} \sigma^i(b) = \frac{a}{\sigma(a)}\cdot \sigma(\frac{a}{\sigma(a)})\cdot \dots \cdot  \sigma^{n-1}\Big(\frac{a}{\sigma(a)}\Big)
			= \frac{a}{\sigma^n(a)} = \frac{a}{a} = 1.
		\end{align*}
		\item[($\Rightarrow$)] Sei $1 = N_{L\mid K}(b) = \prod_{i=0}^{n-1} \sigma^i(b)$. Nach \propref{1_8_9} sind $\sigma^0, \dots, \sigma^{n-1}$ linear unabhängig in $\Abb(L^{\times},L)$, insbesondere existiert $c \in L$ mit 
		\begin{align*}
			a:= \sigma^0(c) + b\sigma^1(c) + b\sigma(b)\sigma^2(c) + \dots + b \sigma(b)\dots\sigma^{n-2}(b)\sigma^{n-1}(c) \neq 0
		\end{align*}
		Es gilt $b\sigma(a) = a$, da $\sigma^n(c) = \sigma^0(c)$ und $b\sigma(b)\dots \sigma^{n-1}(b) = N_{L\mid K}(b) = 1$. Also folgt $b = \frac{a}{\sigma(a)}$.
	\end{itemize}
\end{proof}
\begin{proposition}[Kummer]
	\proplbl{2_7_4}
	Sei $p = \chara(K) \ge 0$ und $n \in \N$ mit $p \nmid n$. Sei $\mu_n \subseteq K$.
	\begin{enumerate}[label={(\alph*)}]
		\item Ist $\alpha \in \overline{K} \mit \alpha^n = a \in K$, so ist $K(\alpha) \mid K$ eine zyklische Galoiserweiterung vom Grad $d \mid n$.
		\item Ist $L \mid K$ zyklisch vom Grad $n$, so ist $L = K(\alpha) \mit \alpha^n \in K$.
	\end{enumerate}
\end{proposition}
\begin{proof}
	O.E. $\alpha \neq 0$.
	\begin{enumerate}[topsep=-6pt,label={(\alph*)}]
		\item $\alpha$ ist Nullstelle von $f = X^n - a = \prod_{\zeta \in \mu_n} (X-\zeta \alpha)$\\
		\hspace*{0.5em}$\xRightarrow{\mu_n \subseteq K}$ $K(\alpha) = K(\zeta \alpha\colon \zeta \in \mu_n)$ ist Zerfällungskörper von $f$ und $f$ ist separabel, also ist $K(\alpha) \mid K$ galoissch.
		
		Betrachte $\chi\colon G=\Gal(K(\alpha)\mid K) \to \mu_n$, $\sigma \mapsto \frac{\sigma(\alpha)}{\alpha}$
		\begin{itemize}
			\item $\chi$ ist unabhängig von der Wahl von $\alpha$: $\alpha' = \zeta \alpha \mit \zeta \in \mu_n \subseteq K$\\
			$\implies$ $\frac{\sigma(\alpha')}{\alpha'} = \frac{\sigma(\zeta\alpha)}{\zeta \alpha} = \frac{\zeta \cdot \sigma(\alpha)}{\zeta \alpha} = \frac{\sigma(\alpha)}{\alpha}$
			\item $\chi$ ist Homomorphismus: Für $\sigma, \tau \in G$ ist
			\begin{align*}
				\chi(\sigma \tau) = \frac{\sigma\tau(\alpha)}{\alpha} = \frac{\sigma\tau(\alpha)}{\tau(\alpha)}\cdot \frac{\tau(\alpha)}{\alpha} = \chi(\sigma)\cdot \chi(\tau)
			\end{align*}
			\item $\chi$ ist injektiv:
			\begin{align*}
				1 = \chi(\sigma) = \frac{\sigma(\alpha)}{\alpha} \quad \implies \quad \sigma(\alpha) = \alpha \quad \implies \quad \sigma = \id_{K(\alpha)}
			\end{align*}
		\end{itemize}
		Haben somit Einbettung
		\begin{align*}
		G  \hookrightarrow \mu_n \cong C_n
		\end{align*}
		und damit $G \cong C_d$ für ein $d\mid n$.
		\item Sei $\Gal(L \mid K) = \langle \sigma \rangle$. Dann \begin{itemize}
			\item $\displaystyle N_{L\mid K}(\zeta_n) = \zeta_n^n = 1$ $\xRightarrow{\propref{2_7_3}}$ $\zeta_n = \frac{\alpha}{\sigma(\alpha)} \text{ für ein }\alpha \in L^{\times}$
			\item  $\displaystyle \sigma(\alpha^n) = \sigma(\alpha)^n = \Big(\frac{\alpha}{\zeta_n}\Big)^n = \alpha^n$ $\implies$ $\alpha^n \in L^{\Gal(L\mid K)} = K$
			\item $\displaystyle \alpha, \sigma(\alpha) = \zeta_n^{-1}(\alpha),\dots,\sigma^{n-1}(\alpha) = \zeta_n^{-(n-1)}(\alpha)$ sind paarweise verschieden\\
			\hspace*{0.5em}$\implies$ $\deg(\alpha \mid K) \ge n$ $\implies$ $L = K(\alpha)$
		\end{itemize}
	\end{enumerate}
\end{proof}
\begin{proposition}[Hilbert 90, additiv]
	\proplbl{2_7_5}
	Sei $L\mid K$ endlich zyklisch mit $\Gal(L\mid K) = \langle \sigma \rangle$. Für $b \in L$ gilt
	\begin{align*}
		\Tr_{L\mid K} (b) = 0 \quad \Longleftrightarrow \quad b = a - \sigma(a) \quad\text{für ein}\; a \in L.
	\end{align*}
\end{proposition}
\begin{proof}
	Sei $n = [L:K]$.
	\begin{itemize}[topsep=-6pt]
		\item[($\Leftarrow$)]
		Ist $b = a - \sigma(a)$, so folgt
		\begin{align*}
			\Tr_{L\mid K}(b) &= \sum_{i=0}^{n-1}\sigma^i(b)\\
			&= a - \sigma(a) + \sigma(a-\sigma(a)) + \dots + \sigma^{n-1}(a - \sigma(a))\\
			&= a - \sigma^n(a) = a - a = 0 
		\end{align*}
		\item[($\Rightarrow$)]
		Sei $0 = \Tr_{L\mid K} (b) = \sum_{i=0}^{n-1} \sigma^i(b)$. Nach \propref{1_8_10} ist $\Tr_{L\mid K} \neq 0$, insbesondere existiert $c \in L$ mit $\Tr_{L\mid K}(c) \neq 0$. Setze
		\begin{align*}
			a &:= \Tr_{L\mid K} (c)^{-1}\Big(b\sigma(c) + \big(b+\sigma(b)\big)\sigma^2(c) + \dots + \big(b+\sigma(b) + \dots + \sigma^{n-2}(b)\big)\sigma^{n-1}(c)\Big),
		\end{align*}
		und es folgt
		\begin{align*}
			a- \sigma(a) &= \Big(b \sigma(c)+b\sigma^2(c) + \dots + b \sigma^{n-1}(c) - \underbrace{\big(\sigma(b)+\sigma^2(b)+ \dots + \sigma^{n-1}(b)\big)}_{\Tr_{L\mid K}(b)-b} \underbrace{\sigma^n(c)}_{=c}\Big) \cdot \Tr_{L\mid K}(c)^{-1}\\
			&= b \cdot \Tr_{L\mid K}(c)^{-1}\big(\sigma(c)+\sigma^2(c)+\dots+\sigma^{n-1}(c)+c\big) = b.
		\end{align*}
	\end{itemize}
\end{proof}
\begin{proposition}[Artin-Schreier]
	\proplbl{2_7_6}
	Sei $p = \chara(K) > 0$.
	\begin{enumerate}[label={(\alph*)}]
		\item Ist $\alpha \in \overline{K}$ eine Nullstelle von $f = X^p - X - a \in K[X]$, so ist entweder $\alpha \in K$ oder $f$ ist irreduzibel und $K(\alpha)\mid K$ ist zyklisch vom Grad $p$.
		\item Ist $L \mid K$ zyklisch vom Grad $p$, so ist $L = K(\alpha)$ mit $\alpha^p - \alpha = a \in K$.
	\end{enumerate}
\end{proposition}
\begin{proof}\leavevmode
	\begin{enumerate}[topsep=-6pt,label={(\alph*)}]
		\item $\alpha$ ist Nullstelle von $f = X^p - X - a = \prod_{i=1}^{p-1} \big(X-(\alpha + i)\big)$\\
		\hspace*{0.5em}$\implies$ $K(\alpha) = K(\alpha + i, i=0, \dots, p-1)$ ist Zerfällungskörper von $f$ und mit $f$ separabel folgt: $K(\alpha)\mid K$ galoissch.
		
		Betrachte $\chi\colon G = \Gal(K(\alpha)\mid K) \to \F_p, \sigma \mapsto \alpha - \sigma(\alpha)$.
		\begin{itemize}
			\item $\chi$ ist unabhängig von der Wahl von $\alpha$:
			\begin{itemize}
				\item $\alpha' = \alpha +i, i \in \F_p \subseteq K$ $\implies$ $\sigma(\alpha') = \sigma(\alpha)+i$
				\item $\alpha' - \sigma(\alpha') =\alpha + i - (\sigma(\alpha)+i) = \alpha - \sigma(\alpha)$
			\end{itemize}
			\item $\chi$ ist Hom.:
			\begin{align*}
				\chi(\sigma \tau) &= \alpha - \sigma \tau(\alpha) = \alpha - \tau(\alpha) + \tau(\alpha) - \sigma\big(\tau(\alpha)\big) &= \chi(\tau) + \chi(\sigma)
			\end{align*}
			\item $\chi$ ist injektiv: $\chi(\sigma) = 0$ $\implies$ $\sigma(\alpha) = \alpha$ $\implies$ $\sigma=\id_{K(\alpha)}$.
		\end{itemize}
		Dies liefert Einbettung $\chi\colon \Gal(K(\alpha)\mid K) \hookrightarrow \F_p \cong C_p$, sodass $G \cong C_p$ ($f$ irreduzibel) oder $G = 1$ ($\alpha \in K$).
	\item Sei $G = \Gal(L \mid K) = \langle \sigma \rangle$ und $\# G = p$. Einerseits gilt
		\begin{align*}
			\Tr_{L\mid K}(-1) = p (-1) = 0 \quad \xRightarrow{\propref{2_7_5}} \quad -1 = \alpha - \sigma(\alpha) \text{ für ein } \alpha \in L,
		\end{align*}
		andererseits ergibt sich
		\begin{align*}
			\sigma(\alpha^p - \alpha) = \sigma(\alpha)^p - \sigma(\alpha) = (\alpha + 1)^p - (\alpha+1) = \alpha^p + 1^p - \alpha -1 = \alpha^p - \alpha,
		\end{align*}
		sodass $a:= \alpha^p - \alpha\in L^{G} = K$.
		
		$\alpha,$ $\sigma(\alpha) = \alpha +1$, $\sigma^2(\alpha) = \alpha +2$, $\dots$, $\sigma^{p-1}(\alpha) = \alpha + p -1$ sind paarweise verschieden und damit folgt $\deg(\alpha \mid K) \ge p \implies L = K(\alpha)$.
	\end{enumerate}
\end{proof}
\begin{lemma}
	\proplbl{2_7_7}
	Sei $F \mid K$ eine weitere Körpererweiterung. Dann: Ist $L \mid K$ zyklisch (abelsch, auflösbar), so auch $LF \mid F$.
\end{lemma}
\begin{proof}
	$FL \mid L$ ist galoissch und
	\begin{align*}
		\Gal(FL \mid L) \cong \Gal(L \mid L \cap F) \le \Gal(L \mid K)
	\end{align*}
	Untergruppen zyklischer (abelscher, auflösbarer) Gruppen sind wieder zyklisch (abelsch, auflösbar). ($\nearrow$ \cref{sec:2_8}).
\end{proof}