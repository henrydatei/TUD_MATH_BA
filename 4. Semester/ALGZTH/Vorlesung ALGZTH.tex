\documentclass[ngerman,a4paper,order=firstname]{../../texmf/tex/latex/mathscript/mathscript}
\usepackage{../../texmf/tex/latex/mathoperators/mathoperators}

% % % local commands
\newcommand{\Halb}{\mathfrak{X}}            % Halbordnung
\newcommand{\caly}{\mathcal{Y}}				% caligraphic Y
\DeclareMathOperator{\transdeg}{tr.deg}		% transcendence degree
\DeclareMathOperator{\Tr}{Tr}				% Trace Matrix or here for L\mid K
\newcommand{\Zwischen}{\mathscr M}

\newlist{remarkenum}{enumerate}{1}
\setlist[remarkenum]{label=(\alph*),ref=\theremark~(\alph*)}
\crefalias{remarkenumi}{remark}

\newlist{propenum}{enumerate}{1}
\setlist[propenum]{label=(\alph*),ref=\theproposition~(\alph*)}
\crefalias{propenumi}{proposition}

\newlist{expenum}{enumerate}{1}
\setlist[expenum]{label=(\alph*),ref=\theexample~(\alph*)}
\crefalias{expenumi}{example}

\title{\textbf{Algebra und Zahlentheorie SS 2019}}
\author{Dozent: Prof. Dr. \person{Arno Fehm}}

\begin{document}
\pagenumbering{roman}
\pagestyle{plain}

\maketitle

\hypertarget{tocpage}{}
\tableofcontents
\bookmark[dest=tocpage,level=1]{Inhaltsverzeichnis}

\pagebreak
\pagenumbering{arabic}
\pagestyle{fancy}

\chapter*{Vorwort}
Wir freuen uns, dass du unser Skript für die Vorlesung \textit{Geometrie} bei Prof. Dr. Arno Fehm im WS2018/19 gefunden hast. Da du ja offensichtlich seit einem Jahr Mathematik studierst, kannst du dich glücklich schätzen zu dem einen Drittel zu gehören, dass nicht bis zum zweiten Semester abgebrochen hat.

Wenn du schon das Vorwort zu \textit{Lineare Algebra und analytische Geometrie 1+2} gelesen hast, weißt du sicherlich, dass Prof. Fehm ein Freud der Algebra ist.\footnote{In Zukunft wird sich Prof. Fehm richtig freuen dürfen, denn im Zuge einer neuen Studienordnung, die am 1.4.2019 in Kraft tritt, kommt so gut wie keine Geometrie im \textit{Bachelor Mathematik} vor.} Auf die Frage eines Kommilitonen, wo in seinem Inhaltsverzeichnis (Gruppen, Ringe, Körper) die Geometrie vorkomme, antwortete er:
\begin{quote}
	\textit{Die Frage ist nicht, wieso wir in dieser Vorlesung Algebra statt Geometrie machen, sondern warum hier seit 20 Jahren Geometrie unterrichtet wird.}
\end{quote}

Wie auch im letzten Vorwort können wir dir nur empfehlen die Vorlesung immer zu besuchen, denn dieses Skript ist kein Ersatz dafür. Es soll aber ein Ersatz für deine unleserlichen und (hoffentlich nicht) unvollständigen Mitschriften sein und damit die Prüfungsvorbereitung einfacher machen. Im Gegensatz zu letztem Semester veröffentlicht Prof. Fehm auf seiner Homepage (\url{http://www.math.tu-dresden.de/~afehm/lehre.html}) kein vollständiges Skript mehr, sondern nur noch eine Zusammenfassung.

Der Quelltext dieses Skriptes ist bei Github (\url{https://github.com/henrydatei/TUD_MATH_BA}) gehostet; du kannst ihn dir herunterladen, anschauen, verändern, neu kompilieren, ... Auch wenn wir das Skript immer wieder durchlesen und Fehler beheben, können wir leider keine Garantie auf Richtigkeit geben. Wenn du Fehler finden solltest, wären wir froh, wenn du ein neues Issue auf Github erstellst und dort beschreibst, was falsch ist. Damit wird vielen (und besonders nachfolgenden) Studenten geholfen.

Und jetzt viel Spaß bei \textit{Geometrie}!

\begin{flushright}
	Henry, Pascal und Daniel
\end{flushright}
\chapter*{Motivation und Einführung}
\input{./TeX_files/Motivation}
\chapter{Körper}
\section{Körpererweiterungen}

Sei $K,L,M$ Körper.

\begin{remark}
	In diesem Kapitel bedeutet ``Ring'' \emph{immer} kommutativer Ring mit Einselement, und ein Ringhomomorphismus bildet stets das Einselement auf das Einselement ab.
	Insbesondere gibt es für jeden Ring einen eindeutig bestimmten Ringhomomorphismus $: \Z \to R$.
\end{remark}

\begin{remark}
	\proplbl{1_1_2}
	\begin{enumerate}[label=(\alph*)]
		\item Ein \emph{Körper} ist ein Ring $R$, in dem eine der folgenden äquivalenten bedingungen gilt:
		\begin{enumerate}[label=\arabic*)]
			\item $0 \neq 1$ und jedes $0 \neq x \in R$ ist invertierbar
			\item $R^{\times} = R \setminus \set{0}$
			\item $R$ hat genau zwei Hauptideale (nämlich $(0)$ und $(1)$)
			\item $(0)$ ist ein maximales Ideal von $R$
			\item $(0)$ ist das einzige echte Ideal von $R$
			\item $(0)$ ist das einzigste Primideal von $R$
		\end{enumerate}
		\item Insbesondere sind Körper \emph{nullteilerfrei}, weshalb $\Ker(\Z \to K)$ prim ist.
		\item Aus (5) folgt: Jeder Ringhomomorphismus $K \to L$ ist injektiv %TODO add ref to (5) and check the ringhomo!
		\item Der Durchschnitt einer Familie von Teilkörpern von $K$ ist wieder ein Teilkörper von $K$.
	\end{enumerate}
\end{remark}

\begin{definition}[Charakteristik]
	Die \begriff{Charakteristik} von $K$, $\chara(K)$, ist das $p \in \set{0,2,3,5,7, \dots}$ mit $\Ker(\Z \to K) = (p)$.
\end{definition}

\begin{example}
	\proplbl{1_1_4}
	\begin{enumerate}
		\item $\chara(\Q) = 0$ und $\chara(\Fp) = (p)$ ($p =$ Primzahl), wobei $\Fp = \lnkset{\Z}{p\Z}$
		\item Ist $K_0 \subseteq K$ Teilkörper, so ist $\chara(K_0) = \chara(K)$.
	\end{enumerate}
\end{example}

\begin{definition}[Primkörper]
	Der \begriff{Primkörper} von $K$ ist der kleinste Teilkörper von $K$. (existiert nach \propref{1_1_2}(d)) %TODO add ref!
\end{definition}

\begin{proposition}
	Sei $\field$ der Primkörper von $K$.
	\begin{enumerate}[label=(\alph*)]
		\item $\chara(K)  = 0 \Leftrightarrow \field \cong \Q$
		\item $\chara(K)  = p > 0 \Leftrightarrow \field \cong \Fp$
	\end{enumerate}
\end{proposition}

\begin{proof}
	``$\Leftarrow$'': \propref{1_1_4}\\ % (beide?)
	``$\Rightarrow$'': $\Image(\Z \to K) \subseteq \field$ und $\Image(\Z \to K) \cong \lnkset{\Z}{\Ker(\Z \to K)}$
	\begin{enumerate}[label=(\alph*)]
		\item $\Image(\Z \to K) \cong \lnkset{Z}{(0)} \cong \Z \Rightarrow \field = \Quot(\Image(\Z \to K)) \cong \Quot(\Z) \cong \Q$
		\item $\Image(\Z \to K) \cong \lnkset{Z}{(p)} \cong \Fp$ ist Teilkörper von K $\Rightarrow \field = \Image(\Z \to K) \cong \Fp$
	\end{enumerate}
\end{proof}

\begin{definition}[Körpererweiterung]
	Ist $K$ ein Teilkörper von $L$, so nennt man $L$ eine \begriff{Köpererweiterung} von $K$, auch geschrieben $L\vert K$.
\end{definition}

\begin{definition}[$K$-Homomorphismus]
	Seien $L_1\vert K$ und $L_2 \vert K$ Körpererweiterungen.
	\begin{enumerate}
		\item Ein Ringhomomorphismus $\phi\colon L_1 \to L_2$ ist ein $K$-Homomorphismus, wenn $\phi\vert_K = \id_K$ (i.Z. $\phi: L_1 \to L_2$)
		\item $\Hom_K(L_1,L_2) = \set{\phi \mid \phi: L_1 \to L_2 \text{ ist $K$-Homomorphismus}}$
		\item $L_1$ und $L_2$ sind $K$-isomorph (i.Z. $L_1 \cong L_2$), wenn es einen Isomorphismus: $\phi \in \Hom_K(L_1, L_2)$ gibt.
	\end{enumerate}
\end{definition}

\begin{remark}
	$L\vert K$ eine Körpererweiterung, so wird $L$ durch Einschränkung der Multiplikation zu einem $K$-Vektorraum.
\end{remark}

\begin{definition}[Körpergrad]
	$[L:K]:= \dim_k(L) \in \N \cup \{\infty\}$, der \begriff{Körpergrad} der Körpererweiterungen $L\vert K$.
\end{definition}

\begin{example}
	\begin{enumerate}[label=(\alph*)]
		\item $[K: K] = 1$
		\item $[\C:\R] = 2$ (Basis $(1,i)$) (aber $(\C:\R) = \infty$)
		\item $[\R:\Q] = \infty$ (mit Abzählarbarkeitsargument oder siehe §2) %TODO ref later, when we are in chap 2!
		\item $[K(x):K] = \infty$ ($K(x) = \Quot(K[x])$ (vgl. GEO II.8)
	\end{enumerate}
\end{example}

\begin{proposition}
	Für $K \subseteq L \subseteq M$ Körper ist $[M:K] = [M:L]\cdot [L:K]$ \\
	(``Körpergrad ist multiplikativ'')
\end{proposition}

\begin{proof} %TODO maybe make unnumbered lemma here for the claim?
	Behauptung: Sei $x_1, \dots, x_n \in L$ $K$-linear unabhängig und $y_1, \dots, y_m \in M$ $L$-linear unabhängig\\
	$\Rightarrow x_i y_j, i \in \set{1,\dots,n}, j \in \set{1, \dots, m}$ $K$-linear unabhängig.\\
	Beweis: $\sum_{i,j} \lambda_{ij}x_i y_j = 0$ mit $\lambda_{ij} \in K$\\
	$\Rightarrow \sum_{j}\underbrace{\left( \sum_{i} \lambda_{ij}x_i \right)}_{\in L}y_j = 0 
	\xRightarrow{y_j L\text{-l.u.}} \sum_{i} \lambda_{ij} x_i = 0\quad\forall j
	\xRightarrow{y_j K\text{-l.u.}} \lambda_{ij} = 0\quad\forall i, \forall j$
	\begin{itemize}
		\item $[L:K] = \infty$ oder $[M:L] = \infty \Rightarrow [M:K] = \infty$
		\item $[L:K] = n, [M:L] = m < \infty$\\
		$(x_1, \dots, x_n)$ Basis des $K$-Vektorraum $L$ und $(y_1, \dots, y_m)$ Basis des $L$-Vektorraums M\\
		$\Rightarrow \set{x_i y_j \colon i = 1, \dots, n; j = 1, \dots, m}$ $K$-linear unabhängig und \\
		$\sum_{i,j} Kx_i y_j = \sum_{j}\left( \sum_{i} \lambda_{ij}x_i \right)y_j = M$, also ist \\
		$\set{x_i y_j \colon i = 1, \dots ,n; j = 1, \dots, m}$ Basis von $M$ 
	\end{itemize}
\end{proof}

\begin{definition}[Körpergrad endlich]
	$L\vert K$ endlich $:\Leftrightarrow [L:K] < \infty$.
\end{definition}

%%%%%%%%%%%%%%%%%%%%%%%%%%%% 2nd lecture %%%%%%%%%%%%%%%%%%%%%%%%%%%%%%%%%%%%%%%%%%%%%%%%%%%%%%

\begin{definition}[Unterring, Teilkörper]
	Sei $L\vert K$ eine Körpererweiterung $a_1, a_2, \dots, a_n \in L$.
	\begin{enumerate}
		\item $K[a_1, \dots, a_n]$ ist kleinster \begriff{Unterring} von $L$, der $K \cup \set{a_1, \dots, a_n}$ enthält (``$a_1, \dots, a_n$ über $K$ erzeugt'')
		\item $K[a_1, \dots, a_n]$ ist kleinster \begriff{Teilkörper} von $L$, der $K \cup \set{a_1, \dots, a_n}$ enthält (``von ``$a_1, \dots, a_n$ über $K$ erzeugte'', ``$a_1, \dots, a_n$'' zu $K$ adjungieren)
		\item $L | K$ ist \begriff{endlich erzeugt} $:\Leftrightarrow a_1, \dots , a_n \in L : L=K(a_1, \dots, a_n)$
		\item $L | K$ ist \begriff{einfach} $:\Leftrightarrow$ existiert $a \in L: L=K(a)$  
	\end{enumerate}
\end{definition}

\begin{remark}
	\proplbl{1_1_15}
	\begin{enumerate}[label=(\alph*)]
		\item $L\vert K$ endlich $\Rightarrow L\vert K$ endlich erzeugt.
		\item $K[a_1, \dots, a_n]$ ist das Bild des Homomorphismus
		\begin{align}
		\begin{cases}
		K[x_1, \dots, x_n] &\to L\\
		f &\mapsto f(a_1, \dots, a_n)
		\end{cases}\notag
		\end{align}
		und $K(a_1, \dots , a_n) = \set{\alpha \beta \colon \alpha, \beta \in K[a_1, \dots, a_n], \beta \neq 0} \cong \Quot(K[a_1, \dots, a_n])$
	\end{enumerate}
\end{remark}
\section{Algebraische Körpererweiterungen} \label{sec:sec_2}

Sei $L \mid K$ eine Körpererweiterung.

\begin{definition}[algebraisch, transzendent]
	Sei $\alpha \in L$. Gibt es ein $0 \neq f \in K$ mit $f(\alpha) = 0$, so heißt $\alpha$ \begriff{algebraisch} über $K$, andernfalls \begriff{transzendent} über $K$.
\end{definition}

\begin{example}
	\begin{enumerate}[label=(\alph*)]
		\item $\alpha \in K$ $\Rightarrow$ $\alpha$ ist algebraisch über $K$ (denn $f(\alpha) = 0$ für $f = X - \alpha \in K[X]$)
		\item $\sqrt{-1} \in \Q(\sqrt{-1})$ ist algebraisch über $\Q$ (denn $f(\sqrt{-1})=0$ für $f = X^2 + 1 \in \Q[X]$) \\
		$\sqrt{-1} \in \C$ ist algebraisch über $\R$        
	\end{enumerate}
\end{example}

\begin{remark}
	\proplbl{1_2_3}
	Sind $K \subseteq L \subseteq M$ Körper und $\alpha \in M$ algebraisch über $K$, so auch über $L$.
\end{remark}

\begin{lemma} 
	\proplbl{1_2_4}
	Genau dann ist $\alpha \in L$ algebraisch über $K$, wenn $1$, $\alpha$, $\alpha^2$, $\dots$ $K$-linear abhängig sind.
\end{lemma}

\begin{proof}
	Sei $\lambda_0$, $\lambda_1$, $\ldots \in K$, fast alle gleich Null, so ist
	\begin{align}
	\sum_{i=0}^\infty \lambda_i \alpha^i = 0\quad \Leftrightarrow\quad f(\alpha) = 0 \text{ für } f = \sum_{i=0}^\infty \lambda_i X^i \in K[X]\notag
	\end{align}
\end{proof}

\begin{lemma}
	Betrachte den Epimorphismus \begin{align*}
	\phi_{\alpha}\colon \left\lbrace\begin{array}{@{}l@{\;}c@{\;}l}
	K[X] &\to & K[\alpha]\\
	f &\mapsto & f(\alpha).
	\end{array}\right.
	\end{align*}
	Genau dann ist $\alpha$ algebraisch über $K$, wenn $\Ker(\phi_\alpha) \neq (0)$. In diesem Fall ist $\Ker(\phi_\alpha) = (f_\alpha)$ mit einem eindeutig bestimmten irreduziblen, normierten $f_\alpha \in K$.
\end{lemma}

\begin{proof}
	$K$ Hauptidealring $\Rightarrow \Ker(\phi_\alpha) = (f_\alpha)$, $f_\alpha \in K$, und o.E. sei $f_{\alpha}$ normiert. Aus $K[\alpha] \subseteq L$ nullteilerfrei folgt, dass $\Ker(\phi_\alpha)$ prim ist. Somit ist $f_\alpha$ prim im Hauptidealring, also auch irreduzibel.
\end{proof}

\begin{definition}[Minimalpolynom, Grad]
	Sei $\alpha \in L$ algebraisch über $K$, $\Ker(\phi_\alpha) = (f_\alpha)$ mit $f_\alpha \in K$ normiert und irreduzibel.
	\begin{enumerate}[label=(\alph*)]
		\item $\MinPol(\alpha\mid K) := f_\alpha$, das \begriff{Minimalpolynom} von $\alpha$ über $K$.
		\item $\deg(\alpha\mid K) :\Leftrightarrow \deg(f_\alpha)$, der \begriff{Grad} von $\alpha$ über $K$.
	\end{enumerate}
\end{definition}

\begin{proposition}
	\proplbl{1_2_7}
	Sei $\alpha \in L$.
	\begin{propenum}[]
		\item $\alpha$ transzendent über $K$ \\
		\hspace*{0.5em}$\Rightarrow$ $K[\alpha] \cong K[X]$, $K(\alpha) \cong_K K(X)$, $[K(\alpha) : K] = \infty$.
		\item \proplbl{korpererweiterungen:prop:1:2:7:b} $\alpha$ algebraisch über $K$ \\
		\begin{tabularx}{\linewidth}{@{\hspace*{0.5em}}c@{$\;$}X}
		$\Rightarrow$ & $K[\alpha] = K(\alpha) \cong \lnkset{K}{\MinPol(\alpha\mid K)}$, $[ K(\alpha) \colon K)]  = \deg(\alpha \mid K) < \infty$, und\hfill\newline
		$1$, $\alpha$, $\dots$, $\alpha^{\deg(\alpha \mid K) -1}$ ist $K$-Basis von $K(\alpha)$. 
		\end{tabularx}
	\end{propenum}
\end{proposition}

\begin{proof}\hspace*{0pt}
	\vspace*{\dimexpr-\baselineskip+\lineskip}
	\begin{enumerate}[label=(\alph*)]
		\item $\Ker(\phi_\alpha) = (0)$ $\Rightarrow$ $\phi_\alpha$ ist Isomorphismus (da zusätzlich injektiv) \\
		\hspace*{0.5em}$\Rightarrow$ $K(\alpha) \cong_K \Quot(K[\alpha]) \cong_K \Quot(K[X]) = K(X)$ \\
		\hspace*{0.5em}$\Rightarrow$ $[K(\alpha) \colon K] = [K(X) \colon K] = \infty$
		\item Sei $f = f_\alpha = \MinPol(\alpha \mid K)$, und $n = \deg(\alpha \mid K) = \deg(f)$.
		\begin{itemize}
			\item $f$ irreduzibel $\Rightarrow$ $(f) \neq (0)$ prim ${\xRightarrow{\text{GEO II.4.7}}}$ $(f)$ ist maximal \\
			\hspace*{0.5em}$\Rightarrow$ $K[\alpha] \cong \lnkset{K[X]}{(f)}$ ist Körper $\Rightarrow$ $K[\alpha] = K(\alpha)$
			\item $1$, $\alpha$, $\dots$, $\alpha^{n-1}$ sind $K$-linear unabhängig: 
			\begin{align*}
			\sum_{i=0}^{n-1} \lambda_i \alpha^i = 0 \quad \Rightarrow \quad \sum_{i=0}^{n-1} \lambda_i X^i \in (f) \quad \xRightarrow{\deg f = n} \quad \lambda_i = 0 \enskip \forall i
			\end{align*}
			\item $1, \alpha, \dots , \alpha^{n-1}$ ist Erzeugendensystem: Für $g \in K[X]$ ist 
			\begin{flalign*}
			 \qquad g &= qf + r &
			\end{flalign*}
			mit $q$, $r \in K[X]$ und $\deg(r) < \deg(f) = n$ und  
			\begin{flalign*}
			 \qquad g(\alpha) &= q(\alpha) \underbrace{f(\alpha)}_{=0} + r(\alpha) = r(\alpha).&
			\end{flalign*}
			Somit folgt: \begin{flalign*}
				\qquad K[\alpha] &= \Image(\phi_\alpha) = \big\{g(\alpha) \;\big|\; g \in K\big\} = \big\{r(\alpha) \;\big|\; r \in K,\, \deg(r) < n\big\} = \sum_{i=0}^{n-1} K \cdot \alpha^i&
			\end{flalign*}
		\end{itemize}
	\end{enumerate}
\end{proof}

\begin{example}
	\begin{enumerate}[label=(\alph*)]
		\item $p \in \Z$ prim $\Rightarrow$ $\sqrt{p} \in \C$ ist algebraisch über $\Q$. \\
		Da $f(X) = X^2 - p$ irreduzibel in $\Q$ ist (GEO II.7.3), ist $\MinPol(\sqrt{p}\mid\Q) = X^2 - p$, und\linebreak $[\Q(\sqrt{p}) : \Q] = 2$.
		\item Sei $\zeta_p = e^{\frac{2\pi i}{p}} \in \C$ ($p \in \N$ prim). Da $\Phi_p =  \frac{X^p-1}{X-1} = X^{p-1} + X^{p-2} + \cdots + X + 1 \in \Q$ irreduzibel in $\Q$ ist (GEO II.7.9), ist $\MinPol(\zeta_p \mid \Q) = \Phi_p$, $[\Q(\zeta_p) : \Q] = p-1$.
		
		Daraus folgt schließlich $[\C : \Q] \ge [\Q(\zeta_p) : \Q] = p-1 \enskip \forall p$ $\Rightarrow$ $[\C : \Q] = \infty$ $\Rightarrow$ $[R : \Q] = \infty$.
		\item $e$, $\pi \in \R$ sind transzendent über $\Q$ (\person{Hermite} 1873, \person{Lindemann} 1882). \\
		Daraus folgt: $[R : \Q] \ge [\Q(\pi): \Q] = \infty$. Jedoch ist unbekannt, ob z.B. $\pi + e$ transzendent ist.
	\end{enumerate}
\end{example}

\begin{definition}
	$L \mid K$ ist \begriff{algebraisch} $:\Leftrightarrow$ jedes $\alpha \in L$ ist algebraisch über $K$.
\end{definition}

\begin{proposition}
	\proplbl{1_2_10}
	$L \mid K$ endlich $\Rightarrow$ $L \mid K$ algebraisch.
\end{proposition}

\begin{proof}
	Sei $\alpha \in L$, $[L : K] = n$. Dann ist  $1,$ $\alpha$, $\dots$, $\alpha^n$ $K$-linear abhängig $\xRightarrow{\propref{1_2_4}} \alpha$ algebraisch über $K$.
\end{proof}

\begin{conclusion}
	\proplbl{1_2_11}
	Ist $L = K(\alpha_1, \dots, \alpha_n)$ mit $\alpha_1$, $\dots$, $\alpha_n$ algebraisch über $K$, so ist $L \mid K$ endlich, insbesondere algebraisch.
\end{conclusion}

\begin{proof}
	Induktion nach $n$:
	\vspace*{\dimexpr-\baselineskip+2\lineskip\relax}
	\begin{itemize}
		\item $n=0$: \checkmark
		\item \renewcommand*{\arraystretch}{1.0}
		\begin{tabular}[t]{@{}l@{\;}l}
		$n > 0$: & $K_1 :=  K(\alpha_1, \dots, \alpha_{n-1})$ \\
				 & \hspace*{0.5em}$\Rightarrow L=K_1(\alpha_n)$, $\alpha_n$ algebraisch über $K_1$ (\propref{1_2_3}) \\
				 & \hspace*{0.5em}$\Rightarrow [L : K] = \underbrace{[K_1(\alpha_n) : K_1]}_{< \infty \text{ nach \propref{1_2_7}}}\cdot \underbrace{[K_1 : K]}_{< \infty \text{ nach IH}}$
	\end{tabular}
	\end{itemize}
\end{proof}

\begin{conclusion}
	Es sind äquivalent:
	\begin{enumerate}[label=(\alph*)]
		\item $L \mid K$ ist endlich.
		\item $L \mid K$ ist endlich erzeugt und algebraisch.
		\item $L = K(\alpha_1, \dots , \alpha_n)$ mit $\alpha_1, \dots, \alpha_n$ algebraisch über $K$.
	\end{enumerate}
\end{conclusion}

\begin{proof}
	\leavevmode\vspace*{\dimexpr-\baselineskip+2\lineskip}
	\begin{itemize}[widest={\ (1) $\Rightarrow$ (2)},leftmargin=*]
		\item[(1) $\Rightarrow$ (2):] \propref{1_1_15} und \propref{1_2_10}
		\item[(2) $\Rightarrow$ (3):] trivial
		\item[(3) $\Rightarrow$ (1):] \propref{1_2_11}
	\end{itemize}
\end{proof}

\begin{remark}
	Nach \propref{1_2_7} ist
	\begin{align*}
		\alpha \text{ algebraisch über } K\quad \Leftrightarrow \quad K[\alpha] = K(\alpha). &
	\end{align*}
	Direkter Beweis für $(\Rightarrow)$: \\
	Sei $0 \neq \beta \in K[\alpha]$. Daraus folgt, dass $f(\beta) = 0$ für ein irreduzibles $0 \neq f = \sum_{i=0}^n a_i X^i \in K[X]$. Durch Einsetzen von $\beta$ und Division durch $\beta$ erhält man
	\begin{flalign*}
	\qquad &\xRightarrow{a_0 \neq 0}\beta^{-1} = -a_0^{-1} ( a_1 + a_2 \beta + \dots + a_n \beta^{n-1}) \in K[\beta] \subseteq K[\alpha]&
	\end{flalign*}
\end{remark}
% % % % % % % % % % % % % % % % 3rd lecture % % % % % % % % % % % % % % % % % % %
\begin{proposition}
	\proplbl{1_2_14}
	Seien $K \subseteq L \subseteq M$ Körper. Dann gilt:
	\begin{align*}
		M\mid K \text{ algebraisch } \Leftrightarrow M\mid L \text{ algebraisch und } L \mid K \text{ algebraisch }
	\end{align*}
\end{proposition}

\begin{proof}\NoEndMark
	\begin{itemize}
		\item[($\Rightarrow$)] klar, siehe \propref{1_2_3}.
		\item[($\Leftarrow$)] Sei $\alpha \in M$. Schreibe $f=\MinPol(\alpha \mid L) = \sum_{i=0}^{n} a_i x^i$, $L_0 := K(a_0,\dots,a_n)$\\
		\renewcommand*{\arraystretch}{1.0}
		\begin{tabularx}{\linewidth}{@{\hspace*{0.5em}}c@{$\;$}X}
		$\Rightarrow$ & $f \in L_0[X]$\\
		$\Rightarrow$ & $[L_0(\alpha): L_0] \le \deg(f) < \infty$\\
		$\Rightarrow$ & $[K(\alpha): K] \le [K(a_0,\dots,a_n,\alpha):K] = \underbrace{[L_0(\alpha):L_0]}_{< \infty}\underbrace{[L_0 :K]}_{< \text{ nach } \propref{1_2_7}}$ \\
		$\Rightarrow$ &  $\alpha$ algebraisch über $K$ \\
		$\overset{\alpha \text{ bel.}}{\Rightarrow}$ & $M \mid K$ algebraisch.\hfill\csname\InTheoType Symbol\endcsname
		\end{tabularx}
	\end{itemize}
\end{proof}

\begin{conclusion}
	\proplbl{1_2_15}
	$\tilde{K} = \set{\alpha \in L\mid \alpha \text{ algebraisch über }K}$ ist ein Körper, und ist $\alpha \in L$ algebraisch über $\tilde{K}$, so ist schon $\alpha \in \tilde{K}$.
\end{conclusion}

\begin{proof}\leavevmode\vspace*{\dimexpr-\baselineskip+2\lineskip} %TODO find a good way to format the RIGHTARROWS?
	\begin{itemize}
		\item $\alpha$, $\beta \in \tilde{K}$:\\
			\hspace*{0.5em}$\Rightarrow$ $K(\alpha, \beta)\mid K$ endlich, insbesondere algebraisch\\
			\hspace*{0.5em}$\Rightarrow$ $\alpha + \beta$, $\alpha - \beta$, $\alpha \cdot \beta$, $\alpha^{-1} \in K(\alpha,\beta)$ alle algebraisch über $K$, also $K(\alpha, \beta) \subseteq \tilde{K}$.
		\item $\alpha \in L$ algebraisch über $\tilde{K}$:\\
			\hspace*{0.5em}$\Rightarrow \tilde{K}(\alpha)\mid \tilde{K}$ algebraisch\\
			\hspace*{0.5em}$\Rightarrow \tilde{K}\mid K$ algebraisch \\
			\hspace*{0.5em}$\overset{\mathclap{\propref{1_2_14}}}{\Rightarrow}$ $\tilde{K}(\alpha)\mid K$ algebraisch, insbesondere $\alpha \in \tilde{K}$.
	\end{itemize}
\end{proof}

\begin{definition}[relative algebraische Abschluss]
	$\tilde{K} = \set{\alpha \in L\mid \alpha \text{ algebraisch über }K}$ heißt der \begriff{relative algebraische Abschluss} von $K$ in $L$.
\end{definition}

\begin{example}
	$\tilde{\Q} = \set{\alpha \in \C \mid \alpha \text{ algebraisch über }\Q}$ ist ein Körper, der Körper der algebraischen Zahlen. Es ist $[\tilde{\Q}:\Q] = \infty$, z.B. da $[\Q(\zeta_p):\Q] = p-1$ für jedes $p$ prim. (algebraische Erweiterung die nicht endlich ist.)
\end{example}
\section{Wurzelkörper und Zerfällungskörper}
Sei $K$ ein Körper, $f \in K[X]$ mit $n = \deg(f) > 0$.
\begin{example}
	Sei $K=\Q$. Dann hat $f$ eine Nullstelle (``Wurzel'') $\alpha \in \C$, und $L:= K(\alpha) = K[\alpha]$ ist die kleinste Erweiterung von $\Q$ in $\C$, die diese Nullstelle enthält.
\end{example}
\begin{definition}[Wurzelkörper]
	Ein \begriff{Wurzelkörper} von $f$ ist eine Körpererweiterung $L \mid K$ der Form $L = K(\alpha)$ mit $f(\alpha) = 0$.
\end{definition}
\begin{lemma}
	\proplbl{1_3_3}
	Sei $L = K(\alpha)$ mit $f(\alpha) = 0$ ein Wurzelkörper von $f$. Dann ist $[L:K] \le n$. Ist $f$ irreduzibel, so ist $[L:K] = n$ und $g \mapsto g(\alpha)$ induziert einen Isomorphismus $\lnkset{K[X]}{(f)} \overset{\cong}{\longrightarrow}_K L$.
\end{lemma}
\begin{proof} %TODO fix b) ref
	Sei zunächst $f$ irreduzibel, $f_{\alpha} = \MinPol(\alpha \mid K)$. Dann ist $f = cf_{\alpha}$, die Behauptung folgt somit aus \propref{1_2_7}b). Für $f \in K[X]$ beliebig, schreibe $f = f_1\cdots f_r$ mit $f_i \in K[X]$ irreduzibel
	\begin{align*}
		f(\alpha) = 0 \Rightarrow \text{ OE } f_1(\alpha) = 0 \Rightarrow [L:K] = \deg(f_1) \le \deg(f) = n %TODO is it really 1 in the index?
	\end{align*}
\end{proof}
\begin{lemma}
	\proplbl{1_3_4}
	Sei $f$ irreduzibel. Dann ist $L := \lnkset{K[X]}{(f)}$ ein Wurzelkörper von $f$.
\end{lemma}
\begin{proof}
	Betrachte den Epimorphismus $\pi = \pi_f: K[X] \to \lnkset{K[X]}{(f)} = L$, setze $\alpha = \pi(X)$
	\begin{itemize}
		\item $K$ Körper $\Rightarrow \pi_{\mid K}$ injektiv\\
		$\Rightarrow$ können $K$ mit Teilkörper von $L$ identifizieren, sodass $\pi_{\mid K} = \id_K$
		\item $(f)$ irreduzibel $\Rightarrow$ prim $\xrightarrow{\text{GEO II.4.7}}$ $(f)$ maximal $\Rightarrow L = \lnkset{K[X]}{(f)}$ ist Körper
		\item $f(\alpha) = f(\pi(X)) \overset{(\ast)}{=} \pi(f(X)) = 0 \quad f(X) \in \Ker(\pi)$\\
		($\ast$ gilt, da $f = \sum a_i x^i = \pi(f) = \sum \pi(a_i)\pi(x)^i = \sum a_i \pi(x)^i = f(\pi(x))$)
		\item $L=\pi(K[X]) = K[\pi(X)] = k[\alpha] \overset{\alpha \text{ alg.}}{=} K(\alpha)$
	\end{itemize}
\end{proof}
\begin{proposition}
	\proplbl{1_3_5}
	Sei $f$ irreduzibel. Ein Wurzelkörper von $f$ existiert und ist eindeutig in folgendem Sinn:\\
	Sind $L_1 = K(\alpha_1), L_2 = K(\alpha_2)$ mit $f(\alpha_1) = 0 = f(\alpha_2)$, so existiert genau ein $K$-Isomorphismus $\varphi: L_1 \to L_2$ mit $\varphi(\alpha_1) = \alpha_2$.
\end{proposition}
\begin{proof}\
	\begin{itemize}
		\item Existenz gibt \propref{1_3_4}
		\item \propref{1_3_3} liefert Isomorphismus
		\begin{align*}
			\begin{Bmatrix} %TODO find a way to only have the right curly bracket!
				L_1 \xleftarrow[\varphi_1]{\cong} & \lnkset{K[X]}{(f)} & \xrightarrow[\varphi_2]{\cong} L_2\\
				\alpha_1 \mapsfrom & X + (f) & \mapsto \alpha_2\\
			\end{Bmatrix}
			\Rightarrow \varphi_2 \circ \varphi_1 : L_1 \xrightarrow{\cong}_K L_2 \mit \alpha_1 \mapsto \alpha_2
		\end{align*}
		Umgekehrt ist jeder $K$-Isomorphismus $\varphi: L_1 \to_K L_2$ wegen $L_1 = K(\alpha_1)$ schon durch $\varphi(\alpha_1)$ festgelegt.
	\end{itemize}
\end{proof}
\begin{conclusion}
	\proplbl{1_3_6}
	$f$ hat einen Wurzelkörper.
\end{conclusion}
\begin{proof}
	Schreibe $f=f_1\cdots f_r, f_1,\dots,f_r \in K[X]$ irreduzibel, nehme einen Wurzelkörper von $f_1$.
\end{proof}
\begin{conclusion}
	\proplbl{1_3_7}
	Es gibt eine Erweiterung $L\mid K$, über der $f$ in Linearfaktoren zerfällt, also $f=c\prod_{i=0}^{n}(x-\alpha_i)$ mit $c \in K^{\times}, \alpha,\dots,\alpha_n \in L$. 
\end{conclusion}
\begin{proof}
	Schreibe $f=c\cdot f_0 \mit c \in K^{\times}, f_0 \in K[X]$ normiert.\\ Induktion nach $n$:
	\begin{itemize}
		\item $n=1:$ $f = x-a$, nehme $L=K$.
		\item $n>1:$ Nach \propref{1_3_6} existiert $L_1 \mid K, \alpha_1 \in L_1 \mit f_0 (\alpha_1) = 0$\\
		$\Rightarrow$ $f_0 = (x-\alpha_1)\cdot f_1 \mit f_1 \in L_1 [X]$ normiert\\
		$\xRightarrow{\text{(IH)}}$ existiert $L\mid L_1 , \alpha_1, \dots, \alpha_n \in L \mit f_1 = \prod_{i=2}^n (x - \alpha_i)$\\
		$\Rightarrow$ $f = c\cdot f_0 = c\cdot (x-\alpha_1) \cdot f_1 = c \prod_{i=1}^n (x- \alpha_i)$
	\end{itemize}
\end{proof}
\begin{definition}[Zerfällungskörper]
	Ein \begriff{Zerfällungskörper} von $K$ ist eine Erweiterung $L\mid K$ der Form $L = K(\alpha_1,\dots,\alpha_n)$ mit $f=c\mal \prod_{i=1}^n (x-\alpha_i) \mit c \in K^{\times}$.
\end{definition}
\begin{proposition}
	\proplbl{1_3_9}
	Ein Zerfällungskörper von $f$ existiert.
\end{proposition}
\begin{proof}
	Ist $L\mid K$ wie in \propref{1_3_7}, ist $K(\alpha_1,\dots,\alpha_n)$ ein Zerfällungskörper von $f$.
\end{proof}
\begin{lemma}
	Ist $L \mid K$ ein Zerfällungskörper vpn $f$, so ist $[L:K] \le n$!
\end{lemma}
\begin{proof}
	Sei $L = K(\alpha_1,\dots,\alpha_n), f = c\prod_{i=1}^n (x-\alpha_i)$.\\
	Induktion nach $n$:
	\begin{itemize}
		\item $n=1:$ $L=K, [K:K] = 1$
		\item $n>1:$ $L_1 = K(\alpha_1)$ ist Wurzelkörper von $f \xRightarrow{\propref{1_3_3}} [L_1:K] \le n$ und schreibe $f=c\mal (x-\alpha_1)\mal f_1, f_1 = \prod_{i=2}^n (x-\alpha_i) \in L_1[X]$\\
		$\Rightarrow L = K(\alpha_1,\dots,\alpha_n) = L_1(\alpha_1,\dots,\alpha_n)$ ist Zerfällungskörper von $f_1$ (über $L_1$)\\
		$\xRightarrow{\text{IH}} [L:L_1] \le \deg(f_1)! = (n-1)!$\\
		$\Rightarrow [L:K] = [L:L_1][L_1:K] = (n-1)!n = n!$
	\end{itemize}
\end{proof}
\begin{example}
	\begin{enumerate}
		\item Ist $n=2$, so ist jeder Wurzelkörper $L$ von $f$, schon ein Zerfällungskörper: $[L:K]\le 2$.
		\item Ist $n =3$, $f$ irreduzibel. Schreibe $L_1 = K(\alpha), f = c(x-\alpha_1)f_1 \mit f_1 \in L_1[X]$
			\begin{itemize}
				\item $f_1$ reduzibel: $L_1$ ist schon Zerfällungskörper von $f$, $[L_1:K] = 3$
				\item $f_1$ irreduzibel: $L_1$ ist kein Zerfällungskörper von $f$. Ist $L$ Wurzelkörper von $f_1$, so ist $L$ Zerfällungskörper von $f$, $[L:K] = 3! = 6$
			\end{itemize}
	\end{enumerate}
\end{example}
\begin{*example}
	Sei $f = x^3 -2 \in \Q[X]$, dann sind die Nullstellen von $f$: $\sqrt[2]{2} \in \R, \zeta_3\sqrt[2]{2}, \zeta_3^2 \sqrt[2]{2}$
	\begin{itemize}
		\item $\Q(\sqrt[2]{2})$ ist Wurzelkörper von $f$. $\Q(\sqrt[3]{2}) \subseteq \R, \zeta_3\sqrt[3]{2}, \zeta_3^2 \sqrt[3]{2} \notin \R$, aber kein Zerfällungskörper. Der Zerfällungskörper von $f$ ist
		\begin{align*}
			\Q(\sqrt[3]{2},\zeta_3\sqrt[3]{2}, \zeta_3^2 \sqrt[3]{2}) = \Q(\sqrt[3]{2}, \zeta_3^2 \sqrt[3]{2})
		\end{align*}
	\end{itemize}
\end{*example}
\begin{mathematica}
	Will man die Nullstellen von $f = X^3 - 2 \in \Q[X]$ finden, dann bietet Mathematica folgende Funktion:
	\begin{align*}
		\texttt{Solve[f==0,x,Complexes]},
	\end{align*}
	der letzte Parameter lässt einem den Körper wählen, in dem Mathematica suchen soll. Es gibt zur Auswahl \texttt{Integers, Rationals, Reals, Complexes}. Für das Beispiel erhält man folgenden Output:
	\begin{align*}
		\set{x \to -(-2)^{(1/3)}, x \to 2^{(1/3)}, x \to (-1)^{(2/3)} 2^{(1/3)}}.
	\end{align*}
	Dabei müsste man die Einheitswurzeln identifizieren:
	\begin{align*}
		\set{x \to \zeta_3\sqrt[3]{2}, x \to \sqrt[3]{2}, x \to \zeta_3^2 \sqrt[3]{2}}
	\end{align*}
\end{mathematica}
\begin{*anmerkung}
	Wenn $f$ irreduzibel $\Rightarrow \lnkset{K[X]}{(f)}$ ist Wurzelkörper.
\end{*anmerkung}
\begin{lemma}
	\proplbl{1_3_12}
	Sei $f = \sum_{i=0}^n a_i x^i$ irreduzibel und sei $L = K(\alpha)$ mit $f(\alpha)$ ein Wurzelkörper von $f$. Sei $\tilde{L}\mid \tilde{K}$ eine weitere Körpererweiterung und $\varphi \in \Hom(K,\tilde{K})$. Ist $\sigma \in \Hom(L,\tilde{L})$ eine Fortsetzung von $\varphi$ (d.h. $\sigma_{\mid K} = \varphi$), so ist $\sigma(\alpha)$ eine Nullstelle von $f^{\varphi}=\sum_{i=0}^n \varphi(\alpha_i)x^i \in K[X]$. Ist umgekehrt $\beta \in L' $ eine Nullstelle von $f^{\varphi}$, so gibt es genau eine Fortsetzung $\sigma \in \Hom(L,\tilde{L})$ von $\varphi$ mit $\sigma(\alpha) = \beta$.
\begin{center} % tikzcd was bitchy, compiled and included the pdf.
	\includegraphics{./tikz/lemma_1_3_12.pdf}
\end{center}
\end{lemma}
\begin{proof}[was für die Prüfung!]
	\begin{itemize}
		\item $f(\alpha) = 0 \Rightarrow 0 = \sigma(0) = \sigma(f(\alpha)) = \sigma\brackets{\sum_{i=0}^n a_i \alpha_i} = \sum_{i=0}^n \varphi(a_i)\sigma(\alpha)^i = f^{\varphi}(\sigma(\alpha))$
		\item Eindeutigkeit klar, da $L=K(\alpha)$
		\item Existenz: Betrachte 
		\begin{align*}
			\eta: 
			\begin{cases}
				K[X] &\to L\\
				g &\mapsto g(\alpha)
			\end{cases}\qquad
			\psi:
			\begin{cases}
				K[X] &\to L'\\
				g &\to g^{\varphi}(\beta) 
			\end{cases}\qquad \to \text{ sind Homomorphismen nach univer. Eigenschaft}
		\end{align*}
		(Bemerke: $\eta$ surjektiv: $\eta_{\mid K} = \id \to K \in \Image(\eta) \mit \eta(x) = \alpha \to \alpha \in \Image(\eta)$)\\
		$Ker(\eta)=(f)$ ist Isomorphismus und $\bar{\eta}: \lnkset{K[X]}{(f)} \xrightarrow{\cong}L$ und\\
		$f \in \Ker(\psi) \Rightarrow \Ker(\psi) = (f)$ ist Homomorphismus $\bar{\psi}: \lnkset{K[X]}{(f)} \to L'$\\
		$\sigma:= \bar{\psi}\circ \bar{\eta}^{-1}: L \to L'$ ist eine Fortsetzung von $\psi$ und
		\begin{align*}
			\sigma(\alpha) = \bar{\psi}(x+(f)) = \beta
		\end{align*}
	\end{itemize}
\end{proof}
\begin{proposition}
	\proplbl{1_3_13}
	Der Zerfällungskörper von $f$ ist eindeutig bestimmt bis auf $K$-Isomorphie.
\end{proposition}
\begin{proof}
	\begin{enumerate}[label=]
		\item Behauptung: Ist $\varphi:K \to K'$ ein Isomorphismus, $L$ ein Zerfällungskörper, $L'$ ein Zerfällungs- körper von $f^{\varphi}$, so setzt sich $\varphi$ zu einem Isomorphismus $L \to L'$ fort.
		\item Beweis: Induktion nach $n = \deg(f)$
			\begin{enumerate}[label=] %TODO maybe use enum here without label?
				\item (IA) $n=1:$ $L = K \xrightarrow[\varphi]{\cong} K' = L'$ \checkmark
				\item (IS) $n>1:$ Schreibe $f = cg_1\cdots g_r$ mit $g_i \in K[x]$ normiert und irreduzibel, $c \in K^{\times}$\\
				$\Rightarrow f^{\varphi} = c^{\varphi}g_1^{\varphi}\cdots g_r^{\varphi}$ mit $c^{\varphi}\in (K')^{\varphi}$ und $g_i^{\varphi}\in K' [X]$ normiert und irreduzibel (weil $\varphi$ Isomorphismus ist). Sei $\alpha_1 \in L$ mit $g_1 (\alpha_1) = 0, \alpha'_1 \in L'$ mit $g_1^{\varphi}(\alpha'_1) = 0$\\
				$\xRightarrow{\propref{1_3_12}} \varphi$ setzt man zu einem Isomorphismus
				\begin{align*}
					\sigma: K_1 := K(\alpha_1) \to K' (\alpha'_1) \mit \sigma(\alpha_1) = \alpha'_1
				\end{align*} 
				fort. Schreibe $f=(x - \alpha_1)\cdot f_1^{\sigma}$ mit $f_1 \in K_1 [X]$ \\
				$\Rightarrow f^{\varphi} = (x - \underbrace{\sigma(\alpha_1)} _{\alpha'_1})\cdot f_1^{\sigma}$ mit $f_1^{\sigma}\in K'_1 [X]$. $L$ ist Zerfällungskörper von $f_1,L'$ ist Zerfällungskörper von $f_1^{\sigma}$\\
				$\Rightarrow \sigma$ setzt sich fort zu einem Isomorphismus $L \to L'$
			\end{enumerate}
		Die Behauptung im Fall $\varphi = \id_K$ ist genau die Aussage von \propref{1_3_13}.
	\end{enumerate}
\end{proof}
\begin{remark}
	Ist $M\mid K$ eine Erweiterung, die einem Zerfällungskörper $l$ von $f$ enthält, dann ist dieser nicht nur bis auf die Isomorphie sondern als Teilkörper eindeutig bestimmt $L = K(\alpha_1, \dots, \alpha_n)$, wobei $\alpha_1, \dots, \alpha_n$ genau die $n$ Nullstellen von $f$ in $M$ sind.
\end{remark}
\section{Der algebraische Abschluss}
Sei $L\mid K$ eine Körpererweiterung.
\begin{definition}[algebraisch abgeschlossen]
	$K$ ist algebraisch abgeschlossen $\Longleftrightarrow$ jedes $f \in K[X] \mit \deg(f) > 0$ hat eine Nullstelle in $K$.
\end{definition}
\begin{lemma}
	\proplbl{1_4_2}
	Es ist äquivalent:
	\begin{enumerate}[label=(\alph*)]
		\item $K$ ist algebraisch abgeschlossen. \label{aussage:1_4_2:1}
		\item Jedes $0 \neq f \in K[X]$ zerfällt über $K$ in Linearfaktoren. \label{aussage:1_4_2:2}
		\item $K$ hat keine echte algebraische Erweiterung. \label{aussage:1_4_2:3}
	\end{enumerate}
\end{lemma}
\begin{proof}\leavevmode %TODO ref
	\vspace*{\dimexpr-\baselineskip+2\lineskip}
	\begin{itemize}[leftmargin=6em]
		\item[\ref{aussage:1_4_2:1} $\Rightarrow$ \ref{aussage:1_4_2:2}:] Induktion nach $\deg(f)$ (siehe LAAG)
		\item[\ref{aussage:1_4_2:2} $\Rightarrow$ \ref{aussage:1_4_2:3}:] Sei $L \mid K$ algebraisch, $\alpha \in L$. Schreibe $f = \MinPol(\alpha \mid K)$. Nach \ref{aussage:1_4_2:2} zerfällt $f$ in Linearfaktoren über $K \Rightarrow \alpha \in K$
		\item[\ref{aussage:1_4_2:3} $\Rightarrow$ \ref{aussage:1_4_2:1}:] Sei $f \in K[X]$, $\deg(f) > 0$. Nach \propref{1_3_9} existiert ein Zerfällungskörper $L$ von $f$. Da $L\overset{(*)}{=}K$ nach \ref{aussage:1_4_2:3} hat $f$ Nullstellen in $K$. \\
		($(*)$ $L$ ist Erweiterung $\rightarrow$ die nach \ref{aussage:1_4_2:3} trivial ist)
	\end{itemize}
\end{proof}
\begin{definition}[algebraischer Abschluss]
	$L$ ist \begriff{algebraischer Abschluss} von $K :\Longleftrightarrow L$ ist algebraisch abgeschlossen und $L\mid K$ algebraisch.
\end{definition}
\begin{lemma}
	\proplbl{1_4_4}
	Ist $L$ algebraischer Abschluss, so ist der relative algebraische Abschluss $\tilde{K}$ ein algebraischer Abschluss von $K$.
\end{lemma}
\begin{proof}\leavevmode\vspace*{\dimexpr-\baselineskip+4\lineskip}
	\begin{itemize}
		\item $\tilde{K}$ ist Körper: \propref{1_2_15}
		\item $\tilde{K} \mid K$ ist algebraisch: Definition
		\item $\tilde{K}$ ist algebraisch abgeschlossen: Sei $f \in \tilde{K}[X] \mit \deg(f) > 0$.\\
		$L$ algebraisch abgeschlossen $\Rightarrow$ existiert $\alpha \in L \mit f(\alpha) = 0$ $\Rightarrow \alpha$ algebraisch über $\tilde{K} \xRightarrow{\propref{1_2_15}} \alpha \in \tilde{K}$.
	\end{itemize}
\end{proof}
\begin{example}
	\begin{enumerate}[label=(\alph*)]
		\item $\C$ ist algebraisch abgeschlossen (Fundamentalsatz der Algebra, $\nearrow$ II.) %TODO \nearrow II later
		\item $\C$ ist algebraischer Abschluss von $\R$.
		\item $\tilde{\Q} := \set{\alpha \in \C \mid \alpha \text{ algebraisch über }\Q}$ ist nach \propref{1_4_4} ein algebraischer Abschluss von $\Q$.
	\end{enumerate}
\end{example}
\begin{lemma}
	\proplbl{1_4_6}
	Sei $L\mid K$ algebraisch, $E$ ein algebraisch abgeschlossener Körper und $\varphi \in \Hom(K,E)$. Dann existiert eine Fortsetzung von $\varphi$ auf $L$, d.h. ein $\sigma \in \Hom(L,E) \mit \sigma_{\mid K} = \varphi$.
\end{lemma}
\begin{proof}
	Definiere die Halbordnung
	\begin{flalign*}
		\qquad &\Halb := \big\lbrace(M,\sigma) \;\big|\; K \subseteq M \subseteq L \text{ Zwischenkörper, }\sigma\in \Hom(M,E),\, \sigma_{\mid K} = \varphi\big\rbrace &\\
		\intertext{mit der Ordnung}
		&(M,\sigma) \subseteq (M' , \sigma') :\mskip-4mu\iff M \subset M' \und \sigma'_{\mid M} = \sigma&
	\end{flalign*}
	\begin{itemize}
		\item $\Halb \neq \emptyset$: $(K,\varphi) \in \Halb$
		\item Ist $(M,\sigma)_{i \in I}$ eine Kette in $\Halb$, so definieren wir $M:= \bigcup_{i\in I} M_i$ und $\sigma\colon M \to E$ durch $\sigma(x) = \sigma_i (x)$ falls $x \in M_i$.
		
		Dann ist $(M,\sigma) \in \Halb$ eine obere Schranke der Kette $(M_i , \sigma_i)_{i\in I}$. Nach Lemma von \person{Zorn} existiert $(M, \sigma)$ maximal. 
		Es ist $M = L$: Sei $\alpha \in L$, $f= \MinPol(\alpha\mid M)$. $f \in E[X]$ hat Nullstelle $\beta \in E$, da $E$ algebraisch abgeschlossen ist
		$\xRightarrow{\propref{1_3_12}}$ existiert Fortsetzung $\sigma' \in \Hom(M(\alpha), E)$ von $\sigma$\\
		$(M,\sigma) \le (M(\alpha), \sigma') \in \Halb \xRightarrow{(M(\alpha), \sigma) \text{ max.}} M = M(\alpha)$, $\alpha \in M.$
	\end{itemize}
\end{proof}
\begin{theorem}[\person{Steinitz}, 1910]
	Jeder Körper $K$ besitzt einen bis auf $K$-Isomorphie eindeutig bestimmten algebraischen Abschluss.
\end{theorem} %TODO tikzcd!
\begin{proof}\leavevmode\vspace*{\dimexpr-\baselineskip+2\lineskip}
	\begin{itemize}
		\item Eindeutigkeit:\\
		Seien $L_1$, $L_2$ algebraische Abschlüsse von $K$\\
		$L_1 \mid K$, $L_2$ algebraisch abgeschlossen $\xRightarrow{\propref{1_4_6}}$ existiert $\sigma \in \Hom(L_1 , L_2)$
		\begingroup
		\zeroAmsmathAlignVSpaces
		\begin{flalign*} %TODO find a way to have only the right curly bracket?
		\hspace*{-0.2em}&\left.\begin{array}{@{}l}
		L_1 \text{ algebraisch abgeschlossen }\Rightarrow \sigma(L_1) \cong L_1 \text{ algebraisch abgeschlossen}\\[-3pt]
		\vspace*{5pt}L_2 \mid K \text{ algebraisch } \Rightarrow L_2 \mid \sigma(L_1) \text{  algebraisch }
		\end{array}\right\rbrace\xRightarrow{\propref{1_4_2}} L_2 = \sigma(L_1).&
		\end{flalign*}
		\endgroup
		\vspace{4\lineskip}
		\hspace*{-3pt}Somit ist $\sigma\colon L_1 \to L_2$ ein $K$-Isomorphismus.
		\item Existenz: Seien
		\begin{itemize}
			\item $\mathscr{F} = \set{f \in K[X] \mid \deg(f) > 0}$
			\item $\Halb = (X_f)_{f \in \mathscr{F}}$ Familie von Variablen
			\item $R := K[\Halb]$ Polynomring in den Variablen $X_f$ ($f \in \mathscr{F}$)
			\item $I := (f(X_f) \colon f \in \mathscr{F}) \properideal R$
		\end{itemize}
		\begin{underlinedenvironment}[Behauptung 1]
			Es gilt $I \properideal R$.
		\end{underlinedenvironment}
		\vspace*{\dimexpr-\baselineskip-3pt}
		\begin{proof}
			Angenommen $I = R$. Dann existieren $f_1$, $\dots$, $f_n \in \mathscr{F}$ und $g_1$, $\dots$, $g_n \in R$ mit
			\begin{flalign*}
				\qquad &\sum_{i=1}^n g_i \cdot f_i (X_f{f_i}) = 1.&
			\end{flalign*}
			Sei $L$ ein Zerfällungskörper von $f_1$, $\dots$, $f_n$. Dann existieren $\alpha_1$, $\dots$, $\alpha_n \in L$ mit $f_i(\alpha_i) = 0$ für alle $i$. Sei $\varphi: R \to L$ der Einsetzungshomomorphismus gegeben durch
			\begin{flalign*}
				\qquad &\varphi_{\mid K} = \id_K, \quad \varphi(X_{f_i}) = \alpha_i, \quad \varphi(X_f) = 0 \text{ für } f \in \lnkset{\mathscr{F}}{\set{f_1,\dots, f_n}}&
			\end{flalign*}
			Dann folgt
			\begingroup
			\zeroAmsmathAlignVSpaces*
			\begin{flalign*}
				\qquad &1 = \varphi(1) = \sum_{i=1}^n \varphi(g_i) \cdot \varphi(f_i (X_f))
				= \sum_{i=1}^n \varphi(g_i) \cdot f_i (\underbrace{\varphi(X_f)}_{= \alpha_i}) = \sum_{i=1}^n \varphi(g_i) \cdot \underbrace{f_i (\alpha_i)}_{=0} = 0 &
			\end{flalign*}
			\endgroup
		\end{proof}
		Jedes echte Ideal ist in einem maximalen Ideal von $R$ enthalten (GEO II 2.13) \\[-5\lineskip]
		\begin{tabularx}{\linewidth}{@{\hspace{0.5em}}>{$}r<{$}@{\;\;}X}
		\implies & \begin{minipage}[t]{\linewidth}existiert maximales Ideal $m \unlhd R \mit I \subseteq m$. $L_1 := \lnkset{R}{m}$ ist Körpererweiterung von $K$, und jedes $f \in \mathscr{F}$ hat eine Nullstelle in $L_1$, nämlich $f(X_f + m) = f(X_f) + m = 0 + m$. Iteriere dies und
		\vspace*{0pt}	
		\[
			K := L_0 \subseteq L_1 \subseteq L_2 \subseteq \cdots,
			\vspace*{0pt}
		\]
		\leavevmode
		wobei jedes $f \in L_i [X]$, $\deg(f) >0$ eine Nullstelle in $L_{i+1}$ hat. Setze nun $L = \bigcup_{i=1}^{\infty} L_i$.
		\end{minipage}
		\end{tabularx}
		\begin{underlinedenvironment}[Behauptung 2]
			$L$ ist algebraisch abgeschlossen.
		\end{underlinedenvironment}
		\vspace*{\dimexpr-\baselineskip-3pt}
		\begin{proof}
			Sei $f \in L[X]$, $\deg(f) > 0$ $\implies$ $f \in L_i [X]$ für ein $i$ $\implies$ $f$ hat eine Nullstelle in $L_{i+1} \subseteq L$
		\end{proof}
		Nach \propref{1_4_4} ist somit
		\begin{flalign*} \qquad &\tilde{K} = \big\lbrace\alpha \in L \,\big|\, \alpha \text{ algebraisch über }K\big\rbrace&
		\end{flalign*}
		ein algebraischer Abschluss von $K$.
	\end{itemize}
\end{proof}
\begin{definition}[algebraischer Abschluss]
	Mit $\bar{K}$ bezeichnen wir den (bis auf $K$-Isomorphie eindeutig bestimmten) \begriff{algebraischen Abschluss} von $K$.
\end{definition}
\begin{definition}[Automorphismengruppe]
	$\Aut(L\mid K) := \{ \sigma \in \mathrm{Hom}_K (L,L)\mid \sigma \text{ Isomorphismus}\}$, die \begriff{Automorphismengruppe} von $L\mid K$.
\end{definition}
\begin{remark}
	$\Aut(L \mid K)$ ist Gruppe unter $\sigma \cdot \sigma' = \sigma' \circ \sigma$ und wirkt auf $L$ durch $x^{\sigma} := \sigma(x)$.
\end{remark}
\begin{proposition}
	\proplbl{1_4_11}
	Sei $K \subseteq L \subseteq \bar{K}$ ein Zwischenkörper. Jedes $\varphi \in \Hom_K (L, \bar{K})$ lässt sich zu einem $\sigma \in \Aut(\bar{K}\mid K)$ fortsetzen.
\end{proposition}
\begin{proof}
	Sei $\bar{K} \mid K$ algebraisch abgeschlossen und $\bar{K}$ algebraisch abgeschlossen\\
	\hspace*{0.5em}$\xRightarrow{\propref{1_4_6}}$ existiert Fortsetzung $\sigma \in \Hom_K (\bar{K}, \bar{K})$ von $\varphi$
	\bgroup
	\zeroAmsmathAlignVSpaces
	\begin{flalign*}
	\hspace*{-.3em}&\left.\begin{array}{@{}l@{\;}c@{\;}l}
	\bar{K} \;\text{algebraisch abgeschlossen} &\implies & \sigma(\bar{K})\; \text{algebraisch abgeschlossen} \\[-2pt]
	\bar{K} \mid K\; \text{algebraisch ist} & \implies& \bar{K} \mid \sigma(\bar{K})\;\text{algebraisch}
	\end{array}\right\rbrace \; \bar{K} = \sigma(\bar{K}) &
	\end{flalign*}
	\egroup
	 somit ist $\sigma \in \Aut(\bar{K}, K)$.
\end{proof}
\begin{definition}[konjugiert]
	$\alpha, \beta \in \bar{K}$ sind $K$-\begriff{konjugiert} $\Longleftrightarrow$ existiert $\sigma \in \Aut(\bar{K}, K)\mit \sigma(\alpha) = \beta$.
\end{definition}
\begin{remark}
	$K$-Konjugiertheit ist eine Äquivalenzrelation auf $\bar{K}$.
\end{remark}
\begin{conclusion}
	$\alpha, \beta \in \bar{K}$ sind $K$-konjugiert $\iff$ $\MinPol(\alpha \mid K) = \MinPol(\beta \mid K)$.
\end{conclusion}
\begin{proof}\leavevmode\vspace{\dimexpr-\baselineskip+2\lineskip}
	\begin{itemize}
		\item[($\Rightarrow$)] $\sigma(\alpha) = \beta \mit \sigma \in \Aut(\bar{K}\mid K)$, $f \in K[X]$, $f(\alpha) = 0 \implies 0 = \sigma(0) = \sigma(f(\alpha)) = f(\sigma(\alpha)) = f(\beta)$
		\item[($\Leftarrow$)] $\MinPol(\alpha \mid K) = \MinPol(\beta \mid K)$
		\begin{itemize}[topsep=0pt]
		\item[$\xRightarrow{\propref{1_3_5}}$] existiert $K$-Isomorphismus $\varphi\colon K(\alpha) \to K(\beta)$ mit $\varphi(\alpha) = \beta$
		\item[$\xRightarrow{\propref{1_4_11}}$] existiert Fortsetzung $\sigma \in \Aut(\bar{K}, K)$ von $\varphi$. 
		\end{itemize}
	\end{itemize}
\end{proof}
\begin{example}
	\begin{itemize}
		\item $\ii, -\ii \in\tilde{\Q}$ sind $\Q$-konjugiert: komplex Konjugation (eingeschränkt auf $\tilde{\Q}$)
		\item $\sqrt{2}$, $-\sqrt{2} \in \tilde{\Q}$ sind $\Q$-konjugiert: $\MinPol(\sqrt{2}\mid \Q) = x^2 -2 = \MinPol(-\sqrt{2}\mid \Q)$
	\end{itemize}
\end{example}
\section{Die transzendente Erweiterung}
Sei $L\mid K$ eine Körpererweiterung.
\begin{definition}[algebraisch abhängig]
	\begin{enumerate}
		\item $a_1, \dots, a_n \in L$ \begriff{algebraisch abhängig} über $K$ $: \equi $ existiert \\$0 \neq f \in K(X_1,\dots, X_n) \colon f(a_1, \dots, a_n) = 0$
		\item $(a_i)_{i\in I}$ ist \begriff{algebraisch abhängig} über $K$ $:\equi$ existiert $J \subseteq I$ endlich: $(a_i)_{i\in I}$ algebraisch abhängig über $K$
	\end{enumerate}
\end{definition}
\begin{*example}
	Betrachte die reellen Zahlen $\sqrt{\pi} \und 2\pi +1$, beide sind transzendent über $\Q$. Die Singletons $\set{\sqrt{\pi}}\und \set{2\pi +1}$ sind algebraisch unabhängig über $\Q$. Aber die Vereinigung $\set{\sqrt{\pi}, 2\pi +1}$ ist nicht algebraisch unabhängig in $\Q$, da
	\begin{align*}
		P(x,y) = 2x^2 - y + 1 = 0
	\end{align*}
	ist, wenn $x = \sqrt{\pi} \und y = 2\pi +1$ gesetzt sind.
\end{*example}
\begin{remark}
	\begin{enumerate}
		\item $(a)$ ist algebraisch abhängig über $K \equi a$ ist algebraisch über $K$
		\item $L = K(X_1,\dots, X_n) = \Quot(K([X_1,\dots, X_n])) \implies X_1,\dots, X_n$ sind algebraisch unabhängig über $K$
		\item Sind $\pi, e$ unabhängig über $\Q$?\\
		Falls ``Ja'', wäre z.B. $\pi+e$ transzendent über $\Q$
	\end{enumerate}
\end{remark}
\begin{definition}[rein transzendent]
	$L \mid K$ \begriff{rein transzendent} $:\equi L = K(\Halb) \mit \Halb = (a_i)_{i\in I}$ algebraisch unabhängig über $K$.
\end{definition}
\begin{lemma}
	\proplbl{1_5_4}
	$\Halb = (a_i)_{i \in I}$ algebraisch unabhängig über $K \implies K(\Halb) \cong_K K(X_i \colon i \in I) = \Quot(K[X_i \colon i \in I])$. 
\end{lemma}
\begin{proof}
	Betrachte $K$-Isomorphismus
	\begin{align*}
		\varphi = \begin{cases}
			K[X_i \colon I \in I] &\to K[a_i : i \in I]\\
			f & \mapsto f(\Halb)
		\end{cases} 
	\end{align*}
	($a_i$ für $x_i$ einsetzen.) Da $\Halb$ algebraisch unabhängig über $K$, ist $\Ker(\varphi) = (0)$\\
	$\implies K(\Halb) = \Quot(K[\Halb]) \cong_K \Quot(K[X_i : i \in I])$.
\end{proof}
\begin{proposition}
	$L\mid K$ rein transzendent $\implies \tilde{K} = K$.
\end{proposition}
\begin{proof}
	Nach \propref{1_5_4} o.E. $L = K(X_i : i \in I)$. Weiter o. E. $I = \set{1, \dots,n}$ endlich. Sei $\alpha \in L$ algebraisch über $K$. Definiere $f = \MinPol(\alpha \mid K)$.\\
	$f$ irreduzibel in $K[X] \xRightarrow{\text{Gauß}} f$ irreduzibel in $K[X_1, \dots, X_n][X]$\\
	$\xRightarrow{\text{Gauß}} f$ irreduzibel $K(X_1, \dots, X_n)[X]$\\
	$\implies \deg(f) = 1$\\
	$\implies \alpha \in K$.
\end{proof}
\section{Separable Polynome}
Sei $K$ ein Körper, $f \in K[X]$, $n = \deg(f)$.
\begin{definition}
	Sei $a \in K$.
	\begin{enumerate}[label={(\arabic*)}]
		\item $\mu(f,a) := v_{x-a}(f) := \sup \set{k \in \N_0 : (x-a)^k \mid f} \in \N_0 \cup \set{\infty}$ die \begriff{Vielfachheit} der Nullstelle $a$ von $f$
		\item Nullstelle $a$ von $f$ ist \begriff{einfach} :$\Leftrightarrow$ $\mu(f,a) = 1$
		\item $f$ ist \begriff{separabel} :$\Leftrightarrow$ jede Nullstelle $a\in\bar K$ von $f\in\bar K[X]$ ist einfach.
	\end{enumerate}
\end{definition}
\begin{remark}
	\begin{enumerate}[label={(\alph*)}]
		\item Ist $L\mid K$ eine Körpererweiterung und $g\in K[X]$, so gilt \begin{flalign*}
			\qquad & f\mid g\;\text{in}\; K[X]\quad\Leftrightarrow\quad f\mid g\;\text{in}\; L[X]
		\end{flalign*}
		Insbesondere ist die Nullstelle $\mu_K(f,a) = \mu_L(f,a)$. Wir können deshalb von der Vielfachheit der Nullstelle von $f$ sprechen.
		\item \proplbl{1_6_2_b} $\displaystyle\#\{a\in K\mid f(a) = 0\} \le \sum_{a\in K} \mu(f,a) \le \sum_{a\in \bar K} \mu(f,a) = \deg(f)$, falls ($f\neq 0$)
		\item Aus \ref{1_6_2_b} folgt insbesondere:\\
		\begin{tabularx}{\linewidth}{XcX}
			\hfill$f$ ist separabel & $\Leftrightarrow$ & $f$ hat genau $\deg(f)$ paarweise verschiedene Nullstellen in $\bar K$
		\end{tabularx}
	\end{enumerate}
\end{remark}
\begin{definition}
	Die \begriff{formale Ableitung} von $f = \sum_{i=1}^n a_i X^{i-1}$ ist \begin{flalign*}
		\qquad &f' := \frac{\d}{\d x} f(x) := \sum_{i=1} i a_i X^{i-1} &
	\end{flalign*}
\end{definition}
\begin{lemma}
	Für $f,g \in K[X], a,b \in K$ gelten\begin{enumerate}[label={(\alph*)}]
		\item $(af + bg)' = a f' + b g'$ (Linearität)
		\item $(fg)' = f'g + fg'$ (Produktregel)
		\item $(f(g(x)))' = f'(g(x))\cdot g'(x)$ (Kettenregel)
	\end{enumerate}
\end{lemma}
\begin{proof}
	Übung.
\end{proof}
\begin{lemma}
	\proplbl{1_6_5}
	Sei $f \neq 0$. Für $a \in K$ gilt
	\begin{flalign*}
		\qquad&\mu(f' , a) \ge \mu(f,a) - 1&
	\end{flalign*}
	mit Gleichheit genau dann, wenn $\chara(K) \nmid \mu(f,a)$.
\end{lemma}
\begin{proof}
	Schreibe $f = (X-a)^k \cdot g$, $k = \mu(f,a)$
	\begin{itemize}[topsep=-.5em,left=3.5em]
		\item[$k=0$:] $\mu(f', a) \ge 0 > -1$ und $\chara(K) \mid 0$
		\item[$k>0$:] $f' = k(X-a)^{k-1}g + (X-a)^k \cdot g'$ $\implies \mu(f',a) \ge k$, sowie 
		
		\vspace*{-2pt}
		\begin{tabular}{@{}>{$}r<{$}>{$}c<{$}>{$}l<{$}}
			\mu(f', a) \ge k & \Leftrightarrow & (X-a)^k \mid k(X-a)^{k-1}\cdot g \\
							 & \Leftrightarrow & X-a \mid k\cdot g \\
							 & \Leftrightarrow & X-a\mid k \\
							 & \Leftrightarrow & k=0\;\text{in}\; K\\
							 & \Leftrightarrow & \chara(K) \mid k
		\end{tabular}
	\end{itemize}
\end{proof}

\begin{proposition}
	\proplbl{1_6_6}
	Sei $f\neq 0$. Dann gilt:
	
	\begin{tabular}{@{$\qquad$}rcl}
		$f$ separabel & $\Leftrightarrow$ & $\ggT(f,f') = 1$
	\end{tabular}
\end{proposition}

\begin{proof}\leavevmode\vspace*{\dimexpr-\baselineskip+2\lineskip}
	\begin{itemize}
		\item[($\Rightarrow$)] $f$ separabel \\[-0.2em]
			\begin{tabular}[t]{@{}>{$}r<{$}@{$\;$}l}
			\Rightarrow & $f = c\cdot\prod_{i=1}^{n}(X-a_i)$ mit $c\in K$, $a_1$, $\dots$, $a_n\in \bar K$ paarweise verschieden und $\mu(f,a_i) = 1$ \\
			\xRightarrow[\chara(K)\nmid 1]{\propref{1_6_5}} & $\mu(f',a_i) = 0$ $\forall i$\\
			\Rightarrow & $\displaystyle\ggT(f,f') = \prod_{a\in\bar K} (X-a)^{\min\{ \mu(f,a), \mu(f',a) \}} = 1$
		\end{tabular}
		\item[($\Leftarrow$)] $f$ nicht separabel $\Rightarrow$ $\exists a\in\bar K$ mit $\mu(f,a)\ge 2$ $\xRightarrow{\propref{1_6_5}}$ $\mu(f',a)\ge 1$.
		
		Mit $g = \MinPol(a\mid K)$ gilt: $g\mid f$ $\Rightarrow$ $\ggT(f,f') \neq 1$
	\end{itemize}
\end{proof}

\begin{lemma}
	$f' = 0$ $\Leftrightarrow$ $\exists g\in K[X]$ mit $f(X) = g(X^p)$ und $p=\chara(K)$.
\end{lemma}
\begin{proof}
	Ist $f = \sum_{i=1}^{n} a_iX^i$ $\Rightarrow$ $f' = \sum_{i=1}^{n}i a_{i-1}X^{i-1}$ und \\[-0.3em]
	\begin{tabular}{@{}r>{$}c<{$}l}
		$f' = 0$	& \Leftrightarrow & $i a_i = 0$ in $K$ $\forall i$\\
					& \Leftrightarrow & $\forall i$: $i = 0$ in $K$ oder $a_i = 0$ \\
					& \Leftrightarrow & $f = a_0 + a_p X^p + \dots + a_{pm} X^{pm} = g(X^p)$ mit $g = a_0 + a_p X + \dots + a_{pm} X^m$
	\end{tabular}
\end{proof}

%TODO
\begin{conclusion}
	\proplbl{1_6_8}
	Sei $f$ irreduzibel
	\begin{enumerate}[label={(\alph*)}]
		\item Ist $\chara(K) = 0$, so ist $f$ separabel
		\item Ist $\chara(K) = p>0$, so sind äquivalent
		\begin{enumerate}[label={(\arabic*)}]
			\item $f$ ist inseparabel
			\item $f' = 0$
			\item $f(X) = g(X^p)$ für ein $g \in K[X]$
		\end{enumerate}
	\end{enumerate}
\end{conclusion}
\begin{proof}
	$f$ irreduzibel $\implies \underbrace{\ggT(f,f') \sim 1}_{\xLeftrightarrow{\propref{1_6_6}} f \text{ sep}} \oder \underbrace{\ggT(f,f') \sim f}_{\xLeftrightarrow{\propref{1_6_6}} f \text{ sep.}}$.
	
	Da $\deg(f') = \deg(f)$ ist
	\begin{flalign*}
		\qquad & f \mid f' \quad \iff \quad  f' = 0 \quad \iff \quad f(X) = g(X^p) \; \text{für ein}\;g &
	\end{flalign*}
	Im Fall $\chara(K) = 0$ tritt dieser Fall nicht ein.
\end{proof}
\begin{definition}[vollkommen]
	$K$ ist \begriff{vollkommen} $\iff$ jedes irreduzibel $f \in K[X]$ ist separabel.
\end{definition}
\begin{example}
	\begin{expenum}
		\item \proplbl{1_6_12_a} $\chara(K) = 0 \implies K$ ist vollkommen
		\item $K = \bar{K} \implies K$ ist vollkommen
		\item $K = \F_p (t)$ ist nicht vollkommen:
		\begin{flalign*}
			\qquad f &= X^p - t \in K[X] \text{ ist irreduzibel} &\\
			f' &= pX^{p-1} = 0 \implies f \text{ nicht seperabel.} &
		\end{flalign*}
		Tatsächlich hat $f$ nur eine Nullstelle in $\bar{K}$: $f = X^p - t \overset{\text{V1}}{=} (X - t^{\frac{1}{p}})^p.$
	\end{expenum}
\end{example}
\begin{definition}
	Sei $\chara(K) = p > 0$.
	\begin{enumerate}[label={(\arabic*)}]
		\item Der \person{Frobenius}-Endomorphismus von $K$ ist
		\begin{flalign*}
		\qquad &\Phi_p\colon \left\lbrace\begin{array}{@{}l@{\;}c@{\;}l}
		K &\to& K\\
		X &\mapsto& X^p 
		\end{array}\right. &
		\end{flalign*}
		\item $K^p = \Image(\Phi_p) = \set{a^p \mid a \in K}$
	\end{enumerate}
\end{definition}
\begin{proposition}
	Sei $\chara(K) = p > 0$. Dann ist $\Phi_p \in \End(K): =\Hom(K,K)$
\end{proposition}
\begin{proof}
	Für $a, b \in K$ ist
	\begin{itemize}[topsep=-6pt]
		\item $\Phi_p = (ab)^p = a^p \cdot b^p = \Phi_p (a) \cdot \Phi_p(b)$
		\item $\Phi_p(a+b) = (a+b)^p = \sum_{i=0}^p\binom{p}{i} a^i b^{p-i} = b^p + a^p = \Phi_p(a) + \Phi_p(b)$, da $p \mid \binom{p}{i}$ für $i = 1, \dots, p-1$ (V1).
		\item $\Phi_p(1) = 1^p = 1$
	\end{itemize}
\end{proof}
\begin{remark}
	\begin{remarkenum}
		\item \proplbl{1_6_13_a} Da $\Phi_p \in \End(K) $ ist $K^p$ ein Teilkörper von $K$ und $\Phi_p$ ist injektiv.
		\item Insbesondere gibt es zu jedem $a \in K$ ein eindeutig bestimmtes $a^{\frac{1}{p}} \in \bar{K}$ mit
		\begin{flalign*}
			\qquad & \Phi_p(a^{\frac{1}{p}}) = (a^{\frac{1}{p}})^p = a &
		\end{flalign*}
		\item Für $a \in \F \cong \F_p$ ist $\Phi_p(a) = a$. (z.B. $\Phi_p(1) = 1$ oder kleiner Satz von \person{Fermat})
	\end{remarkenum}
\end{remark}
\begin{lemma}
	\proplbl{1_6_14}
	Sei $\chara(K) = p > 0$, $a \in K \setminus K^p$. Dann ist $f = X^p -a$ irreduzibel und inseparabel
\end{lemma}
\begin{proof}
	Sei $\alpha \in \bar{K}$ mit $f(\alpha) = 0$, $g= \MinPol(\alpha \mid K)$
	\begin{itemize}[topsep=-6pt]
	\item[$\implies$] $g \mid f = X^p - \alpha = (X-\alpha)^p$
	\item[$\implies$] $g \equiv (X - \alpha)^k$ mit $k \le p$. 
	\end{itemize}
	\medskip
	$a \notin K^p$
	\begin{itemize}[topsep=-6pt,widest=$\xRightarrow{g \text{ irred.}}$,leftmargin=*]
	\item [$\implies$] $\alpha \notin K \implies k >1$
	\item[$\implies$] $g$ ist inseperabel
	\item[$\xRightarrow{g \text{ irred.}}$] $g(X) = h(X^p)$ für ein $h$
	\item[$\implies$] $k = p$ $\implies f = g$ irreduzibel 
	\end{itemize}
\end{proof}

\begin{proposition}
	\proplbl{1_6_15}
	Genau dann ist $K$ vollkommen, wenn \begin{enumerate}[label={(\roman*)}]
		\item $\chara(K) = 0$ oder
		\item $\chara(K) = p > 0$ und $K^P = K$
	\end{enumerate}
\end{proposition}
\begin{proof}
\leavevmode
\begin{itemize}[topsep=-6pt]
\item $\chara(K) = 0$: klar (\cref{1_6_12_a})
\item $\chara(K) = p > 0$: \begin{itemize}
	\item[($\Rightarrow$)] Es existiert ein $a\in K\setminus K^p$, so ist $K$ nicht vollkommen nach \cref{1_6_14}.
	\item[($\Leftarrow$)] Sei $f(X)\in K[X]$ irreduzibel und inseparabel. Nach \cref{1_6_8} existiert ein $g(X)\in K[X]$ mit \begin{flalign*}
		\qquad & f(X) = g(X^p) &
	\end{flalign*}
	Setze $g(X) = \sum_{i=0}^n a_i X^i\in K[X]$. Dann ist \begin{flalign*}
		\qquad & f(X) = g(X^p) = \sum_{i=0}^n a_i \big(X^i)^p \overset{=}{\text{V1}} \Bigg(\sum_{i=0}^n \underbrace{a_i^{1\!\slash\!n}}_{\mathclap{\text{$\in K$ da $K^p=K$}}} X^i\Bigg)^p, &
	\end{flalign*}
	folglich ein Widerspruch.
	\end{itemize}
\end{itemize}
\end{proof}
\begin{example}
	\proplbl{1_6_16}
	$K$ endlich $\Rightarrow$ $K$ vollkommen (\propref{1_6_13_a}, \propref{1_6_15}).
\end{example}
\section{Separable Erweiterungen_}
Sei $K$ ein Körper und $L\mid K$ algebraische Körpererweiterung.

\begin{remark}
\proplbl{1_7_1}
Für $L = K(\alpha)$ mit $f = \MinPol(\alpha\mid K$ ist \begin{flalign*}
	\qquad & [L:K] = \deg(f) \ge \big\vert\big\lbrace\beta\in\bar K\;\big|\;f(\beta) = 0\big\rbrace\big\vert \overset{\propref{1_3_12}}{=} \vert \Hom_{\mathbb K} (L,\bar K)\vert
	\end{flalign*}
	
	mit Gleichheit genau dann wenn $f$ separabel.
\end{remark}

\begin{definition}
\proplbl{1_7_2}
Sei $\alpha\in L$. \begin{enumerate}
	\item $\alpha$ ist \begriff{separabel} über $K$ :$\Leftrightarrow$ $\MinPol(\alpha\mid K)$ ist separabel.
	\item $L\mid K$ ist \begriff{separabel} :$\Leftrightarrow$ jedes $\alpha\in L$ ist separabel über $K$.
	\item Der \begriff{Separabilitätsgrad} von $L\mid K$ ist \begin{align*}
		[L:K]_{\mathrm S} = \vert\Hom_{\mathbb K} (L,\bar K)\vert
	\end{align*}
\end{enumerate}
\end{definition}

\begin{lemma}
	\proplbl{1_7_3}
	Sei $E$ algebraisch abgeschlossen, $\phi\in\Hom(K,E)$. Dann ist \begin{flalign*}
		\qquad & \big\vert\big\lbrace\psi\in\Hom(L,E)\;\big|\;\psi_{\mathbb K}=\phi\right\rbrace\right\vert = [L:K]_{\mathrm S}
	\end{flalign*}
\end{lemma}
\begin{proof}
	Nach \propref{1_4_6} existiert ein $g\in\Hom(\bar K, E)$ mit $g_{|K} = \phi$. Ohne Einschränkung ist $E=\widetilde{\phi(K)} = g(\bar K)$, d.h. $g$ ist Isomorphismus. Dann ist die Abbildung \begin{flalign*}
	\qquad & \left\lbrace\begin{array}{c@{\;}c@{\;}l}
		\Hom_{\mathbb K}(L,\bar K) & \Rightarrow & \big\lbace \psi\in\Hom(L,E)\;\big|\;\psi_{|K} = \phi\right\rbrace \\
		\sigma & \mapsto & g\circ \sigma
	\end{array}.\right. &
	\end{flalign*}
	Diese ist bijektiv mit Umkehrabbildung $\psi\mapsto g^{-1}\circ\psi$.
\end{proof}

\begin{proposition}
	\proplbl{1_7_4}
	Sind $K\subset L\subset M$ Körer mit $M\mid K$ algebraisch, so ist \begin{flalign*}
		\qquad & [M:K]_{\mathrm S} = [M:L]_{\mathrm S}[L:K]_{\mathrm S} &
	\end{flalign*}
	Insbesonder ist $[L:K]_{\mathrm S} \le [M:K]_{\mathrm S]$.
\end{proposition}

\begin{proof}
	Betrachte die Abbildung \begin{flalign*}
		\qquad & f\colon\;\left\lbrace\begin{array}{l@{\;}c@{\;}l}
			\Hom(M,\barK) & \rightarrow & \Hom_{\mathbb K}(L,\bar K) \\
			\sigma & \mapsto & \sigma_{|L}
		\end{array}.\right. &
	\end{flalign*}
	Für $\tau\in\Hom_{\mathbb K}(L,\bar K)$ ist \begin{flalign*}
		\qquad & f^{-1}(\lbrace\tau\rbrace) = \big\vert\big\lbrace \sigma\in\Hom_{\mathbb K}(M,\bar K)\;\big|\;\sigma_{|L}=\tau\big\rbrace\big\vert = [M:L]_{\mathrm S}
	\end{flalign*}
	Daher gilt $[M:K]_{\mathrm S} = [M:L]_{\mathrm S}[L:K]_{\mathrm S}$.
\end{proof}

\begin{lemma}
	\proplbl{1_7_5}
	Sei $L\mid K$ endlch und $p=\chara(K) > 0$. Dann ist \begin{align*}
		[L:K] = p^l [L:K]_{\mathrm S}
	\end{align*}
	für ein $L\in\mathbb N$. Insbesondere ist $[L:K]_{\mathrm S} \le [L:K]$.
\end{lemma}
\begin{proof}
	Schreibe $L=K(\alpha_1,\dots,\alpha_n)$, ohne Einschränkung ist $n=1$ (nach \cref{1_7_4,1_1_12}). Sei $f=\MinPol(\alpha_1\mid K)$ und $l\in\mathbb N$ die größte Zähl mit \begin{flalign*}
	\qquad & f(X) = g(X^{lp}),\quad g(X)\in K[X].
	\end{flalign*}
	Dann ist $g(X)$ irreduzibel und separabel nach \propref{1_6_8}. Daher gilt \begin{flalign*}
	\qquad & [L:K]_{\mathrm S} \overset{\propref{1_7_1},\propref{1_7_2}}{=} \big\vert \big\lbrace x\in\bar K\;\big|\; f(x) = 0\big\rbrace\big\vert = \big\vert\big\lbrace x\in\bar K\;\big|\; g(x) = 0\big\rbrace\big\vert = \deg(g) = \frac{\deg(f)}{p^l} = \frac{[L:K]}{p^l},&
	\end{flalign*}
	sodass $[L:K] = p^l [L:K]_{\mathrm S}$.
\end{proof}

\begin{proposition}
	\proplbl{1_7_6}
	Für $L\mid K$ endlich sind äquivalent \begin{enumerate}[label={(\arabic*)}]
		\item $L\mid K$ ist separabel.
		\item $L = K(\alpha_1,\dots,\alpha_n)$ mit $\alpha_1$, $\dots$, $\alpha_n$ separabel über $K$
		\item $[L:K]_{\mathrm S} = [L:K]$.
	\end{enumerate}
\end{proposition}
\begin{proof}\leavevmode\begin{itemize}[topsep=-6pt,widest={$(1)$ $\Rightarrow$ $(2)$},leftmargin=*]
	\item[$(1)$ $\Rightarrow$ $(2)$] klar nach \propref{1_7_2}
	\item[$(2)$ $\Rightarrow$ $(3)$] Da $\alpha_i$ separabel über $K$ ist $\alpha_i$ separabel über $K(\alpha_1,\dots,\alpha_{n-1})$. Daher ist \begin{flalign*}
		\qquad & [K(\alpha_1,\dots,\alpha_i):K(\alpha_1,\dots,\alpha_{i-1})]_{\mathrm S} \overset{\propref{1_7_1}}{=} [K(\alpha_1,\dots,\alpha_i):K(\alpha_1,\dots,\alpha_{i-1})] &
	\end{flalign*}
	Nach \cref{1_1_12,1_7_4} gilt dann \begin{flalign*}
		\qquad & [L:K]_{\mathrm S} = [L:K] &
	\end{flalign*}
	\item[$(3)$ $\Rightarrow$ $(1)$] Für $\alpha\in L$ ist mit $l\in\mathbb N$ \begin{flalign*}
		\qquad & [L:K] \overset{\propref{1_1_12}}{=} [L:K(\alpha)][K(\alpha):K] \overset{\propref{1_7_5}}{\ge} [L:K(\alpha)]_{\mathrm S} \cdot p^l [K(\alpha):K]_{\mathrm S} \overset{\propref{1_7_4}}{=} [L:K]_{\mathrm S} p^l \overset{(3)}= [L:K]p^l, &
	\end{flalign*}
	daher $l=0$, d.h. $[K(\alpha):K] = [K(\alpha):K]_{\mathrm S}$. Nach \propref{1_7_1} ist $\alpha$ separabel über $K$, d.h. $(1)$ gilt.
\end{itemize}
\end{proof}

\begin{conclusion}
	\proplbl{1_7_7}
	Der relative, separable Abschluss \begin{flalign*}
		\qquad & L_{\mathrm S} = \big\lbrace \alpha\in L\;\big|\; \alpha\;\text{separabel über}\; K\big\rbrace &
	\end{flalign*}
	von $K$ in $L$ ist Teilkörper in $L$.
\end{conclusion}

\begin{proof}
	Folgt aus \cref{1_7_6} (vergleiche \propref{1_2_15}).
\end{proof}

\begin{conclusion}
	\proplbl{1_7_8}
	Seien $K\subset L\subset M$ mit $M\mid K$ algebraisch. Dann gilt: \begin{tabularx}{\linewidth}{X@{\quad}c@{\quad}X}
		\hfill $M\mid K$ separabel & $\Leftrightarrow$ & $M\mid L$ separabel und $L\mid K$ separabel
	\end{tabularx}
\end{conclusion}
\begin{proof}
	\leavevmode
	\begin{itemize}[topsep=-6pt]
		\item[($\Rightarrow$)] klar
		\item[($\Leftarrow$)] Sei $\alpha\in M$, setzte $f=\MinPol(\alpha\mid L) = \sum_{i=0}^n a_i X^i$ und $L_0 = K(a_0,\dots,a_n)$. Da $M\mid L$ separabel ist $f$ separabel. Daher ist $\alpha$ separabel über $L_0$, d.h $L_0(\alpha)\mid L_0$ ist separabel (siehe \cref{1_7_6}). Da $L\mid K$ separabel ist, ist auch $L_0\mid K$ separabel und es gilt \begin{flalign*}
		\qquad & [L_0(\alpha)\mid K]_{\mathrm S} \overset{\propref{1_7_4}}= [L_0(\alpha):L_0]_{\mathrm S} [L_0:K]_{\mathrm S} \overset{\propref{1_7_5}}= [L_0(\alpha)\mid L_0] [L_0: K] \overset{\propref{1_1_2}}= [L_0(\alpha)\mid K]&
	\end{flaling*}
	Deswegen ist $L_0(\alpha)\mid K$ separabel (siehe \cref{1_7_6}). Insbesondere ist $\alpha$ separabel über $K$.
\end{proof}

\begin{conclusion}
\proplbl{1_7_9}
Sei $K\subset L_1$, $L_2\subset M$ Körper mit $M\mid K$ algebraisch. Sind $L_1\mid K$ und $L_2\mid K$ separabel,so auch die Komposition $L1\cdot _2 := K(L_1,L_2)$.
\end{conclusion}
\begin{proof}
Es sei $\alpha\in L_1L_2$. Dann gibt es $\x_1$, $\dots$, $x_n\in L_1$ und $y_1$, $\dots$, $y_m\in L_2$ mit $\alpha\in K(x_1,\dots,x_n,y_1,\dots,y_n) =: L_0$. Da $x_i$, $y_i$ separabel über $K$, so ist $L_0\mid K$ separabel. Nach \propref{1_7_6}. Insbesondere ist $\alpha$ separabel über $K$.
\end{proof}

\begin{defintion}
Die Erweiterung $L\mid K$ ist \begriff{rein separabel} :$\Leftrightarrow$ jedes $\alpha\in L\setminus K$ ist inseparabel über $K$.
\end{definition}

\begin{proof}
Ist $p=\chara(K) > 0$, so sind äquivalent\begin{enumerate}[label={(\arabic*)},widest={$(1)$ $\Rightarrow$ $(2)$,leftmargin=*]
	\item[$(1)$ $\Rightarrow$ $(2)$] Sei $\alpha\in L$, $f=\MinPol(\alpha\mid K) = g(X^{p^l})$ mit $l$ maximal und $g\in K[X]$ (wie in \propref{1_7_5}). Dann ist $\alpha^{pl}$ separabel über $K$. Da $L\mid K$ rein inseparabel ist, folgt $\alpha^{p^l}\in K$.
	\item[$(2)$ $\Rightarrow$ $(3)$] Sei $\phi\in\Hom_{\mathbb K}(L,\bar K)$ Für $\sigma\in L$ ist \begin{flalign*}
		\qquad & \sgima(\alpha) = \siga(\underbrace{\alpha^{p^l}}_{\in K})^{1\!\slash\! p\,l} = (\alpha^{p^l})^{1\!\slash\! p\,l} = \alpha, &
	\end{flalign*}
	also $\sigma_{|L} = \id_L$ und daher $[L:K]_{\mathrm S} = 1$.
	\item[$(3)$ $\Rightarrow$ $(1)$] Es sei $\alpha\in L\setminus K$. Es ist \begin{flalign*}
		\qquad & [K(\alpha):K] > 1 \overset{(3)}= [L:K]_{\mathrm S}\overset{\propref{1_7_4}}\ge [K(\alpha):K]_{\mathrm S}, &
	\end{flalign*}
	also ist $\alpha$ inseparabel über $\alpha$ nach \cref{1_7_6}.
\end{enumerate}
\end{proof}

\begin{example}
	\proplbl{1_7_12}
	Die Erweiterung $\mathbb F_p(t)\mid \mathbb F_p(t)^p = \mathbb F_p(t)$ ist rein inseparabel vom Grad $p$.
\end{example}

\begin{remark}
	\proplbl{1_7_13}
	Jede algebraische Erweiterung $L\mid K$ hat also eine Unterteilung in eine separablen und inseparablen Teil. Es gilt \begin{flalign*}
		\qquad & [L:K]_{\mathrm S} \overset{\propref{1_7_4}}= [L:L_{\mathrm S}]_{\mathrm S} [ L_{\mathrm S} : K] \underset{\overset{\propref{1_7_11}}=}{\propref{1:7:6}} 1\cdot [L_{\mathrm S}:K] = [L_{\mathrm S}:K] &
	\end{flalign*}
\end{remark}

\section{Norm und Spur}
Sei $L\mid K$ endliche Körpererweiterung und $\alpha \in L$.
\begin{remark}
	$L$ ist ein $K$-Vektorraum $\implies$ $\End_K (L)$ ist ein $K$-Vektorraum und ein (nichtkommutativer) Ring unter Komposition.
\end{remark}
\begin{definition}[Spur, Norm]
	\begin{enumerate}
		\item \begin{align*}
		\mu_{\alpha} : \begin{cases}
		L & \quad L\\
		x &\mapsto \alpha x
		\end{cases} \in \End_K (L)
		\end{align*}
		\item $N_{L \mid K} := \det(\mu_{\alpha}$, die $(L \mid K)$- Norm von $\alpha$
	\end{enumerate}
\end{definition}
%to add 16 May 2019 %TODO
\begin{conclusion}
	Sei $K \subseteq L \subseteq M$, dann
	\begin{itemize}
		\item $N_{L \mid K}(\alpha) = N_{L\mid K}(N_{L\mid K}(\alpha))$
		\item $\Tr_{L \mid K}(\alpha) = \Tr_{L \mid K}(\Tr_{L \mid K}(\alpha))$
	\end{itemize}
\end{conclusion}
\begin{proof}
	Sei $[L:K] = q_1 \cdot r_1, [M:L] = q_2 \cdot r_2$. $\Hom(L, \bar{K}) = \set{\tau_1, \dots, \tau_{r_1}}, \Hom(M, \bar{L}) = \set{\sigma_1, \dots, \sigma_{r_2}}$. Fixiere Einbettung $L \subseteq \bar{K}$ und setze $\tau_i$ fort zu $\tilde{\tau}_i \in \Aut(\bar{K}, K)$ \propref{1_4_11} %TODO
	Dann ist
	\[
		\Hom_K(M, \bar{K}) = \set{\tilde{\tau}_i \circ \sigma_j, i =1, \dots, r_1, j = 1, \dots, r_2}
	\]
	denn $\neq \Hom(M, \bar{K}) = [M, K] = r_1 \cdot r_2$ und
	\begin{align*} %TODO check for { } misssing
		\tilde{\tau}_{i'} \circ \sigma_{j'} \implies \sigma_j = (\tilde{\tau}_i^{-1} \circ \tilde{\tau}_{i'})\circ \sigma_{j'}\\
		\implies \tilde{\tau}_i^{-1} \circ \tilde{\tau}_{i\mid L} = \id_L = \tau_i = \tau_{i'} \implies i = i'\\
		\implies \sigma_j = \sigma_{j'} \implies j = j'.\\
		\implies N_{L\mid K}(N_{M\mid K}(\alpha)) = N_{L\mid K}\brackets{\prod_{i=j}^{r_2} \sigma_j (\alpha)}^{q_2}\ \overset{\cref*{1_2_7}}{=}\\
		= \brackets{ \prod_{i=1}^{r_1} \tau_i \brackets{\prod_{j=1}^{r_2} \sigma_j (\alpha) }}{q_1 q_2} = \brackets{\prod_{i,j} (\tilde{\tau}_i \circ \sigma_j (\alpha))}^{q_1 q_2} \overset{\cref{1_8_7}}{=} N_{M \mid K}(\alpha).
	\end{align*}
	Analog für die Spur.
\end{proof}
\begin{theorem}[Unabhängigkeit der Charaktere, \person{Artin}]
	\proplbl{1_8_9}
	Sei $G$ eine Gruppe. Sind $\chi_1, \dots, \chi_n \in \Hom(G, K^{\times})$ paarweise verschieden, so sind sie linear unabhängig im $K$-Vektorraum $\Abb(G,K)$.
\end{theorem}
\begin{proof} %TODO reformat!
	Seien $\chi_1, \dots, \chi_n$ linear abhängig, oE $n \ge 2$ minimal, d.h. $\sum_{i=1}^n a_i \chi_i = 0 \mit a_1, \dots, a_n \in K^{\times}$. Sind $\chi_1 \neq \chi_n \implies \exists g \in G \mit \chi_1(g) \neq \chi_n(g)$. $\sum a_i \chi_i = \implies \forall h \in G$ ist $\sum_{i=1}^n a_i \chi_i (h) = 0$.
	\begin{align*}
		\implies \begin{cases}
			\sum_{i=1}^n a_i \cdot \underbrace{\chi_i (hg)}_{\chi_i(h)\cdot \chi_i(g)} &= 0\\
			\sum_{i=1}^n a_i \cdot \chi_i(h)\cdot \chi_i(g) = 0
		\end{cases}
		\implies 0 &= \sum_{i=1}^n a_i \cdot \chi_i(h)(\chi_i(g) - \chi_n(g))\\
		&= \sum_{i=1}^{n-1} a_i (\chi_i(g) - \chi_n(g))\cdot \chi_i(h)\\
		\implies \sum_{i=1}^{n-1} a_i \cdot (\chi_i(g) - \chi_n(g))\cdot \chi_i = 0\\
		a_n (\chi_1(g) - \chi_n(g)) \neq 0
	\end{align*}
	das ist ein Widerspruch zur Minimalität von $n$.
\end{proof}
\begin{conclusion}
	\proplbl{1_8_10}
	Genau dann ist $\Tr_{L \mid K} \neq 0$, wenn $L \mid K$ separabel.
\end{conclusion}
\begin{proof}
	\begin{itemize}
		\item ``$\implies$'': \propref{1_8_6}
		\item ``$\Longleftarrow$'': Sei $\Hom_K(L, \bar{K}) = \set{\sigma_1, \dots, \sigma_n}$. $\sigma_{i \mid L^{times}} \in \Hom_K(L^{times}, K^{times})$\\
		$\xRightarrow{\cref{1_8_7}} \sigma_1,\dots, \sigma_n$ sind $\bar{K}$-linear unabhängig. Insbesondere ist $\Tr_{L \mid K} = \sum_{i=1}^n \sigma_i \neq 0$.
	\end{itemize}
\end{proof}
\section{Einfache Erweiterung}
Sei $K$ unendlich, $L \mid K$ endliche Erweiterung.
\begin{remark}
	$L \mid K$ einfach $\Longleftrightarrow L = K(\alpha)$ für ein $\alpha \in L$. Ein solches $\alpha$ heißt ein \begriff{primitives Element} von $L \mid K$.
\end{remark}
\begin{proposition}
	\proplbl{1_9_2}
	Die Erweiterung ist einfach $\Longleftrightarrow$ \\
	Die Menge der Zwischenkörper
	\[
		\Zwischen = \set{M \colon K \subseteq M \subseteq L, M \text{ Körper}}
	\]
	ist endlich.
\end{proposition}
\begin{proof}
	\begin{itemize}\
		\item ``$\implies$'': Sei $L = K(\alpha), f= \MinPol(\alpha \mid K)$. Für $M \in \Zwischen$ setze
		\begin{align*}
			g&:= \MinPol(\alpha \mid M) = \sum_{i=0}^n a_i X^i,\\
			M_0 &:= K(a_0, \dots, a_n).
		\end{align*} 
		Dann gilt $g \mid f$ in $L[X]$, es gibt also nur endlich viele solche $g$. Da $K \subseteq M_0 \subseteq M \subseteq L$ und $[L:M_0] = [M_(\alpha):M_0]= \deg(g) = [M(\alpha):M] = [L:M]$\\
		ist $M = M_0$ durch g bestimmt.
		\item ``$\Longleftarrow$'': Sei $L = K(\alpha_1, \dots, \alpha_r)$. Es genügt, die Behauptung für $r = 2$ zu zeigen. Sei also $L = K(\alpha, \beta)$, oE $\beta \neq 0$. Da $\abs{K} = \infty$ ist $\abs{\set{\alpha + c\beta \colon c \in K}} = \infty$. Ist $\abs{\Zwischen} < \infty$, so existiert somit $c, c' \in K$ mit $c \neq c' \und K(\alpha + c \beta) = K(\alpha + c' \beta) = M \in \Zwischen$\\
		$\implies M \ni (\alpha + c \beta) = (\alpha + c' \beta) = (\underbrace{c-c'}_{\in K^{\times}})\beta$\\
		$\implies \beta \in M \implies \alpha \in M$\\
		$L = K(\alpha, \beta) \subseteq M \subseteq L$\\
		$L = M = K(\alpha + c\beta)$.
	\end{itemize}
\end{proof}
\begin{remark}
	\begin{enumerate}
		\item Insbesondere gilt:
		$K \subseteq M \subseteq L, L \mid K$ endlich und einfach\\
		$\implies M \mid K$ endlich und einfach
		\item Dies gilt auch für transzendente einfache Erweiterungen. $K \subseteq M \subseteq L = K(X) \implies M = K(f)$ für ein $f \in K(X)$. ($\nearrow$ Satz von \person{Lüroth})
	\end{enumerate}
\end{remark}
\begin{theorem}[Satz vom primitiven Element, \person{Abel}]
	\proplbl{1_9_4:primitiv}
	Sei $L = K(\alpha_1, \dots, \alpha_r)$ eine endliche Erweiterung von $K$. Ist höchstens eines der $\alpha_i$ inseparabel über $K$, so ist die $L \mid K$ einfach.
\end{theorem}
\begin{proof}
	Es genügt, den Fall $r = 2$ zu betrachten. %TODO next week! ;)
\end{proof}
\begin{conclusion}
	Jede endliche separable Erweiterung von $K$ ist einfach und besitzt nur endliche viele Zwischenkörper. Dies gilt insbesondere für jede endliche Erweiterung in Charakteristik 0.
\end{conclusion}
\begin{proof}
	Folgt aus \propref{1_9_2}, \propref{1_9_4} und \propref{1_6_15}.
\end{proof}
\begin{example}
	$\Q(\sqrt{2}, \sqrt{3})\mid \Q$ besitzt ein primitives Element, z.B. $\sqrt{2} + \sqrt{3}$ ($\nearrow$ Übung 21). Tatsächlich ist $\Q(\sqrt{2}, \sqrt{3}) = \Q(\sqrt{2}+c\sqrt{3})$ für jedes $c \in \Q^{times}$.
\end{example}
\begin{example}
	Sei $L = \F_p(t,s) = \Quot(\F_p[t,s]), K = L^p$. Dann ist $[L:K] = p^2$ ($\nearrow$ Ü??) aber $L\mid K$ ist \emph{nicht} einfach und besitzt unendliche viele Zwischenkörper. (Nach \propref{1_9_2}) ($\nearrow$ Übung)
\end{example}
\begin{remark}
	Das \propref{1_9_4:primitiv} gilt für $K$ endlich, siehe II.3. %TODO ref later!
\end{remark}

\chapter{Galoistheorie}
\section{Normale Körpererweiterungen}
Sei $K$ Körper, $\bar K$ ein fixierter algebraischer Abschluss von $K$ und $L$ ein Zwischenkörper $K\subseteq L\subseteq \bar K$.

\begin{definition}
	$L\mid K$ ist \begriff{Normal} :$\Leftrightarrow$ Ist $\alpha\in L$ und $\beta\in\bar K$ $K$-konjugiert, so ist $\beta\in L$.
\end{definition}

\begin{proposition}
	\proplbl{2_1_2}
	Ist $L\mid K$ endlich, so sind äquivalent \begin{enumerate}[label={(\arabic*)}]
		\item $L\mid K$ ist normal
		\item Jedes irreduzible $f\in K[X]$, das eine Nullstelle in $L$ hat, zerfällt über $L$ in Linearfaktoren
		\item $L$ ist der Zerfällungskörper von $f\in K[X]$
		\item Für jedes $\sigma\in\Aut(\bar K\mid K)$ ist $\sigma(L) = L$
		\item Jedes $\sigma\in\Aut(\bar K\mid K)$ ist $\sigma(L)\subseteq L$
	\end{enumerate}
\end{proposition}

\begin{proof}\leavevmode
	\begin{itemize}[widest={(1) $\Rightarrow$ (2)},leftmargin=*,topsep=-6pt]
		\item[(1) $\Rightarrow$ (2)] klar nach \propref{1_4_14}
		\item[(2) $\Rightarrow$ (3)] Sei $L = K(\alpha_1,\dots,\alpha_n)$. Mit \begin{equation*}
			f = \prod_{i=1}^n \MinPol(\alpha_i\mid K)
		\end{equation*}
		ist $L$ der Zerfällungskörper von $f$.
		\item[(3) $\Rightarrow$ (4)] Ist $f$ der Zerfällungskörper von\begin{equation*}
			f = \prop_{i=1}^n (X - X_i),
		\end{equation*}
		und $\sigma\in\Aut(\bar K\mid K)$, so permutiert $\sigma$ die Nullstellen $\lbrace \alpha_1,\dots,\alpha_n\rbrace$ von $f$, folglich \begin{equation*}
			\sigma(L) = \sigma\big( K(\alpha_1,\dots,\alpha_n)\big) = K\big(\sigma(\alpha_1),\dots,\sigma(\alpha_n)\big) = K(\alpha_1,\dots,\alpha_n) = L.
		\end{equation*}
		\item[(4) $\Rightarrow$ (5)] trivial
		\item[(5) $\Rightarrow$ (1)] trivial
	\end{itemize}
\end{proof}

\begin{example}
	\begin{enumerate}[label={\alph*)}]
		\item $K\mid K$ ist normal
		\item $\bar K\mid K$ ist normal
		\item $\bar K_{\mathrm S} \mid K$ ist normal (\propref{1_7_7})
		\item $[L:K] = 2$ $\Rightarrow$ $L\mid K$ ist normal
		
		($\deg(f) = 2$, $f$ hat Nullstelle $\Rightarrow$ $f$ zerfällt in Linearfaktoren)
		\item $L = \mathbb Q(\sqrt[3]2)$, $[L:\mathbb Q] = 3$ $L\mid Q$ ist nicht normal, die zu $\sqrt[3]2$ $\mathbb Q$-konjugierte Elemente $\zeta_3 \sqrt[3]2$ und $\zeta_3^2 \sqrt[3]2$ liegen \emph{nicht} in $L$ (\propref{1_3_11_b})
		\item $Sei \alpha = \sqrt[4]2\in\mathbb R_{\ge 0}$  und $f = \MinPol(\alpha\mid\mathbb Q) = X^4 - 2$. Dann sind die $\mathbb Q$-konjugierten $\pm \sqrt[4]2$ und $i\sqrt[4]2$. Da $i\sqrt[4]2\notin\mathbb R$ ist $\mathbb Q(\alpha)\mid \mathbb Q$ nicht normal und \begin{equation*}
			\underbrace{\mathbb Q(\sqrt[4]2) \;\, \underset{\text{normal}}{\overset{2}{\rule[0.1\baselineskip]{3em}{0.1pt}}} \;\, \mathbb  Q(\sqrt 2) \;\, \underset{\text{normal}}{\overset{2}{\rule[0.1\baselineskip]{3em}{0.1pt}}} \;\, \mathbb Q,}_{\text{nicht normal}}
		\end{equation*}
		also ist Normalität nicht transitiv.
	\end{enumerate}
\end{example}

\begin{conclusion}
	\proplbl{2_1_4}
	Sei $L\mid K$ endlich und seien $K\subseteq L_1$, $L_2\subseteq L$ Zwischenkörper. Dann \begin{enumerate}[label={(\alph*)}]
		\item Sind $L_1\mid K$ und $L_2\mid K$ normal, so auch $L_1\cap L_2\mid K$ und $L_1L_2 \mid K$
		\item Ist $L\mid K$ normal, so auch $L\mid L_1$
	\end{enumerate}
\end{conclusion}

\begin{proof}\leavevmode
\begin{enumerate}[label={\alph*)},topsep=-6pt]
	\item \begin{itemize}[left=0pt]
		\item $L1\cap L_2$: klar aus Definition
		\item $L_1L_2$: Sei $\sigma\in\Aut(\bar K\mid K)$ $\Rightarrow$ $\sigma(L_1L_2)  = \sigma(L_1)\sigma(L_2) = L_1 L_2$
	\end{itemize}
	\item klar, da $\Aut(\bar L_1\mid L_1)\subseteq \Aut(\bar K\mid K)$
\end{enumerate}
\end{proof}

\begin{proposition}
	\proplbl{2_1_5}
	Sei $L\mid K$ endlich. Es ist \begin{equation*}
		\# \Aut(L\mid K) \le [L:K]_{\mathrm S}
	\end{equation*}
	mit Gleichheit, wenn die Erweiterung normal ist.
\end{proposition}

\begin{proof}
	Es ist \begin{equation*}
		\Aut(L\mid K) = \Hom_K(L, L) = \big\lbrace \sigma\in\Hom_K(L,\bar K)\;\big|\; \sigma(L)\subseteq L\big\rbrace \subseteq \Hom_K(L,\bar K),
	\end{equation*}
	sodass $\# \Aut(L\mid K) \le \# \Hom_K(L,\bar K) = [L:K]_{\mathrm S}$.
	
	Es gilt: $\Aut(L\mid K) = \Hom_K(L\mid \bar K)$ \begin{itemize}[topsep=0pt,label={$\Leftrightarrow$},widest={<I.4.11>},leftmargin=*]
		\item $\forall \sigma\in \Hom_K(L\mid \bar K)$: $\sigma(L)\subseteq L$
		\item[$\xLeftrightarrow{\propref{1_4_11}}$] $\forall \sigma\in\Aut(\bar K\mid K)$: $\sigma(L)\subseteq L$
		\item[$\xLeftrightarrow{\propref{2_1_2}}$] $L\mid K$ normal.
	\end{itemize}
\end{proof}

\begin{remark}
	\proplbl{2_1_6}
	Es ist also \begin{equation*}
		\Aut(L\mid K) \overset{\circled{\tiny  1}}{\le} [L:K]_{\mathrm S} \overset{\circled{\tiny2}}{\le} [L:K],
	\end{equation*}
	wobei gilt: \begin{enumerate}[label=\protect\circled{\arabic*}]
		\item ist Gleichheit :$\xLeftrightarrow{\propref{2_1_5}}$ $L\mid K$ normal
		\item ist Gleichheit :$\xLeftrightarrow{\propref{1_7_6}}$ $L\mid K$ separabel
	\end{enumerate}
\end{remark}

\begin{definition}
	$L\mid K$ ist \begriff{galoissch} (oder Galoiserweiterung) $\Leftrightarrow$ $L\mid K$ ist normal und separabel
\end{definition}

\begin{proposition}
	\proplbl{2_1_8}
	Ist $L\mid K$ endlich, so sind äquivalent \begin{enumerate}[label={(\arabic*)}]
		\item $L\mid K$ ist galoissch
		\item Jedes $\alpha\in L$ hat $\deg(\alpha\mid L)$ viele $K$-konjugierte in $L$
		\item $L$ ist Zerfällungskörper eines irreduziblen, separablen Polynoms $f\in K[X]$
		\item $L$ ist Zerfällungskörper eines separablen Polynoms $f\in K[X]$
		\item $\#\Aut(L\mid K) = [L:K]$
	\end{enumerate}
\end{proposition}
\begin{proof}\leavevmode
	\begin{itemize}[topsep=-6pt,widest={(1) $\Leftrightarrow$ (3)},leftmargin=*]
		\item[(1) $\Leftrightarrow$ (5)] \propref{2_1_6}
		\item[(1) $\Leftrightarrow$ (2)] $L\mid K$ separabel $\Leftrightarrow$ jdes $\alpha\in L$ hat $\deg(\alpha\mid K)$ viele $K$-konjugierte in $\bar K$. \\
		$L\mid K$ normal $\Leftrightarrow$ alle $K$-konjugierte von $\alpha\in L$ liegen in $L$.
		\item[(1) $\Rightarrow$ (3)] $L\mid K$ separabel $\xRightarrow{\propref{1_9_4}}$ $L = K(\alpha)$ einfach.\\
		$L\mid K$ normal $\Rightarrow$ $L$ ist Zerfällungskörper von $\MinPol(\alpha\mid K)$
		\item[(3) $\Rightarrow$ (4)] trivial
		\item[(4) $\Rightarrow$ (1)] \propref{2_1_2} und \propref{1_7_6}
	\end{itemize}
\end{proof}

\begin{conclusion}
	\proplbl{2_1_9}
	Sei $L\mid K$ endlich und seien $K\subseteq L_1$, $L_2\subseteq L$ Zwischenkörper. \begin{enumerate}[label={(\alph*)}]
		\item Sind $L_1\mid K$ und $L_2\mid K$ galoissch, so auch $L_1\cap L_2\mid K$ und $L_1L_2 \mid K$
		\item Ist $L\mid K$ galoissch, so auch $L\mid L_1$
	\end{enumerate}
\end{conclusion}

\begin{proof}
	\propref{2_1_4}, \propref{1_7_8} und \propref{1_7_9}.
\end{proof}
\section{Der Hauptsatz der Galoistheorie}

$L\mid K$ ist endliche Galoiserweiterung mit $G = \Gal(L\mid K)$.

\begin{definition}
	Es sind \begin{itemize}
		\item $\zwk(L\mid K) = \lbrace F\mid K\subset F\subset L,\;F\,\text{Zwischenkörper}\rbrace$ die Menge der Zwischenkörper und
		\item $\ugr(G) = \lbrace H\mid H\le G\rbrace$ die Menge der Untergruppen.
	\end{itemize}
\end{definition}

\begin{theorem}[Galoiskorrespondenz]
	Es sind \begin{align*}
		&\left\lbrace\begin{array}{@{}c@{\;}c@{\;}c}
			\zwk(L\mid K) & \rightarrow & \ugr(G) \\
			F & \mapsto & F^\circ := \Gal(L\mid F)
		\end{array}\right.&
		&\left\lbrace \begin{array}{@{}c@{\;}c@{\;}c}
			\ugr(G) & \rightarrow & \zwk(L\mid K) \\
			H & \mapsto & H^\circ := L^H
		\end{array}\right.
	\end{align*}
	zueinander inverse Bijektionen. Weiterhin gilt für $F$, $F_1$, $F_2\in\zwk(L\mid K)$ mit $H = F^\circ$, $H_1 = F_1^\circ$ und $H_2 = F_2^\circ$ \begin{enumerate}[label={\roman*)}]
		\item \label{2_2_2_1} die Bijketion ist antiton \begin{align*}
			F_1\subset F_2 \quad\Leftrightarrow\quad H_1\supset H_2
		\end{align*}
		\item die Bijektion ist indextreu, d.h.\begin{align*}
			[F_2:F_1] = (H_1:H_2),\quad\text{wenn}\;F_1\subset F_2
		\end{align*}
		\item die Bijektion vertauscht Erzeugnis und Durchschnitt \begin{align*}
			(F_1\cap F_2)^\circ = \langle H_1,H_2\rangle\quad\text{und}\quad(F_1F_2)^\circ = H_1\cap H_2
		\end{align*}
		\item \label{2_2_2_4} die Bijektion ist mit Konjugation verträglich: $\forall \sigma\in G$ \begin{align*}
			\left( F^\sigma\right)^\circ = \left( F^\circ\right)^\sigma
		\end{align*}
		\item die Bijektion erhält Normalität: \begin{align*}
			F\mid K\;\text{normal}\quad\Leftrightarrow\quad H\unlhd G
		\end{align*}
		In disem Fall gilt: \begin{align*}
			\Gal(F\mid K) \cong \lnkset{G}{H} = \lnkset{\Gal(L\mid K)}{\Gal(L\mid F)}
		\end{align*}
	\end{enumerate}
\end{theorem}

\begin{proof}
	\leavevmode
	\begin{itemize}[topsep=-6pt]
		\item $F\in\zwk(L\mid K)$ $\xRightarrow{\propref{2_1_9}}$ $F$ galoissch $\xRightarrow{\propref{2_1_15}}$ $(F^\circ)^\circ = L^{F^\circ} = F$
		\item $H\in\ugr(G)$ $\xRightarrow{\propref{2_1_14}}$ $L\mid H^\circ$ galoissch mit $(H^\circ)^\circ = \Gal(L\mid H^\circ) = H$
	\end{itemize}
	\begin{enumerate}[label={\roman*)}]
		\item \begin{itemize}[]
			\item[$(\Leftarrow)$] klar, da $F_1 = H_1^\circ$, $F_2 = H_2^\circ$
			\item[($\Rightarrow$)] klarer
		\end{itemize}
		\item $L\mid F_i$ ist galoissch, daher folgt aus \propref{2_1_11} $[L:F_i] = \# H_i$ für $i=1$,$2$ und \begin{align*}
			[F_2:F_1] = \frac{[L:F_1]}{[L:F_2]} = \frac{\#H_1}{\#H_2} = (H_1 : H_2)
		\end{align*}
		\item \begin{itemize}[left=0pt]
			\item $F_1\cap F_2 \subset F_1 F_2$ $\Rightarrow$ $(F_1\cap F_2)^\circ \overset{\text{\hyperref[2_2_2_1]{i)}}}{\supset} \langle H_1,H_2\rangle$,
			
			$H_1$, $H_2\subset \langle H_1,H_2\rangle$ $\Rightarrow$ $F_1\cap F_2 \supset (\langle H_1,H_2\rangle)^\circ$ $\Rightarrow$ $(F_1\cap F_2)^\circ \subset \big( (\langle H_1,H_2\rangle)^\circ\big)^\circ = \langle H_1,H_2\rangle$
			\item $F_1$, $F_2\subset F_1F_2$ $\Rightarrow$ $H_1\cap H_2 \supset (F_1F_2)^\circ$
			
			$H_1\cap H_2\subset H_1$, $H_2$ $\Rightarrow$ $(H_1\cap H_2)^\circ \supset F_1F_2$ $\Rightarrow$ $(F_1F_2)^\circ \supset \big((H_1\cap H_2)^\circ\big)^\circ = H_1\cap H_2$
		\end{itemize}
		\item $(F^\sigma)^\circ = \lbrace \tau\in G\mid \tau|_{F^\sigma} = \id \rbrace = \lbrace \tau\in G\mid \tau(x) = x\;\forall x\in F^\sigma\rbrace = \lbrace \tau\in G\mid \tau(x^\sigma) = x^\sigma\;\forall x\in F\rbrace = \lbrace \tau\in G\mid \tau^{\sigma^{-1}}\in F^\circ\rbrace = (F^\circ)^\sigma$
			
		\item $F\mid K$ normal \begin{itemize}[topsep=\dimexpr-\baselineskip+2\lineskip\relax,widest={$\xLeftrightarrow{\propref{2_1_16}}$},leftmargin=*]
			\item[$\xLeftrightarrow{\propref{2_1_2}}$] $F^\sigma = F$ $\forall \sigma\in\Aut(\bar K\mid K)$
			\item[$\xLeftrightarrow{\propref{2_1_16}}$] $F^\sigma = F$ $\forall \sigma \in G$
			\item[$\xLeftrightarrow{\text{\hyperref[2_2_2_4]{iv)}}}$] $H^\sigma = H$ $\forall \sigma\in G$
			\item[$\Leftrightarrow$] $H\unlhd G$
			
			Sei $F\mid K$ normal. Nach \propref{2_1_16} gilt \begin{align*}
				\res\colon\;\left\lbrace\begin{array}{@{}c@{\;}c@{\;}c}
					\Gal(L\mid K) & \rightarrow & \Gal(F\mid K) \\
					\sigma & \mapsto & \sigma|_F
				\end{array}\right.
			\end{align*}
			ist ein Epimorphismus. \begin{itemize}[topsep=0pt]
				\item[$\Rightarrow$] $\Gal(F\mid K)\cong \Im(\res) \cong \lnkset{\Gal(L\mid K)}{\ker(\res)} \cong \lnkset{\Gal(L\mid K)}{\Gal(L\mid F)} = \lnkset{G}{H}$
			\end{itemize}
		\end{itemize}
	\end{enumerate}
\end{proof}

\part*{Anhang}
\addcontentsline{toc}{part}{Anhang}
\appendix

%\printglossary[type=\acronymtype]

\printindex

\end{document}
