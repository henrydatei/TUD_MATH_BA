\documentclass[ngerman,a4paper,order=firstname]{../../texmf/tex/latex/mathscript/mathscript}
\usepackage{../../texmf/tex/latex/mathoperators/mathoperators}

\title{\textbf{Numerische Mathematik SS 2019}}
\author{Dozent: Prof. Dr. \person{Andreas Fischer}}

\begin{document}
\pagenumbering{roman}
\pagestyle{plain}

\maketitle

\hypertarget{tocpage}{}
\tableofcontents
\bookmark[dest=tocpage,level=1]{Inhaltsverzeichnis}

\pagebreak
\pagenumbering{arabic}
\pagestyle{fancy}

\chapter*{Vorwort}
Wir freuen uns, dass du unser Skript für die Vorlesung \textit{Geometrie} bei Prof. Dr. Arno Fehm im WS2018/19 gefunden hast. Da du ja offensichtlich seit einem Jahr Mathematik studierst, kannst du dich glücklich schätzen zu dem einen Drittel zu gehören, dass nicht bis zum zweiten Semester abgebrochen hat.

Wenn du schon das Vorwort zu \textit{Lineare Algebra und analytische Geometrie 1+2} gelesen hast, weißt du sicherlich, dass Prof. Fehm ein Freud der Algebra ist.\footnote{In Zukunft wird sich Prof. Fehm richtig freuen dürfen, denn im Zuge einer neuen Studienordnung, die am 1.4.2019 in Kraft tritt, kommt so gut wie keine Geometrie im \textit{Bachelor Mathematik} vor.} Auf die Frage eines Kommilitonen, wo in seinem Inhaltsverzeichnis (Gruppen, Ringe, Körper) die Geometrie vorkomme, antwortete er:
\begin{quote}
	\textit{Die Frage ist nicht, wieso wir in dieser Vorlesung Algebra statt Geometrie machen, sondern warum hier seit 20 Jahren Geometrie unterrichtet wird.}
\end{quote}

Wie auch im letzten Vorwort können wir dir nur empfehlen die Vorlesung immer zu besuchen, denn dieses Skript ist kein Ersatz dafür. Es soll aber ein Ersatz für deine unleserlichen und (hoffentlich nicht) unvollständigen Mitschriften sein und damit die Prüfungsvorbereitung einfacher machen. Im Gegensatz zu letztem Semester veröffentlicht Prof. Fehm auf seiner Homepage (\url{http://www.math.tu-dresden.de/~afehm/lehre.html}) kein vollständiges Skript mehr, sondern nur noch eine Zusammenfassung.

Der Quelltext dieses Skriptes ist bei Github (\url{https://github.com/henrydatei/TUD_MATH_BA}) gehostet; du kannst ihn dir herunterladen, anschauen, verändern, neu kompilieren, ... Auch wenn wir das Skript immer wieder durchlesen und Fehler beheben, können wir leider keine Garantie auf Richtigkeit geben. Wenn du Fehler finden solltest, wären wir froh, wenn du ein neues Issue auf Github erstellst und dort beschreibst, was falsch ist. Damit wird vielen (und besonders nachfolgenden) Studenten geholfen.

Und jetzt viel Spaß bei \textit{Geometrie}!

\begin{flushright}
	Henry, Pascal und Daniel
\end{flushright}

\chapter{Das gewöhnliche Iterationsverfahren}
\section{Fixpunkte}

Seien ein Vektorraum $V$, eine Menge $U \subseteq V$ und eine Abbildung $\Phi: U \to V$ gegeben.
Dann heißt $x^{*} \in U$ \begriff{Fixpunkt} der Abbildung $\Phi$, falls $\Phi(x^{*}) = x^{*}$ gilt.
Die Aufgabe
\begin{align}
\Phi(x) = x\notag
\end{align}
eigentlich die Aufgabe, diese Gleichung zu lösen) wird als \begriff{Fixpunktaufgabe} bezeichnet.
Die Abbildung $\Phi$ heißt \begriff{Fixpunktabbildung}. Im Unterschied zur Fixpunktaufgabe heißt
\begin{align}
F(x) = 0 \notag
\end{align}
\begriff{Nullstellenaufgabe}. 
Zu jeder Nullstellenaufgabe gibt es eine äquivalente Fixpunktaufgabe (z.B. $F(x) = 0 \Leftrightarrow \Phi(x) = x $ mit $\Phi(x) := F(x) + x$) und umgekehrt (z.B.
$\Phi(x) = x \Leftrightarrow F(x) = 0$ mit $F(x) := \Phi(x) -x$).
\section{Der Fixpunktsatz von \person{Banach}}

Der folgende Satz gibt (unter gewissen Bedingungen) eine konstruktive Möglichkeit an, einen Fixpunkt näherungsweise zu ermitteln.

\begin{proposition}[Banach]
	\proplbl{1_1_1}
	%TODO use \norm here and find out why undefined control sequence for \norm shows up?!?!?
	Seien $(V, \Vert \cdot \Vert)$ ein Banach-Raum, $U \subseteq V$ eine abgeschlossene Menge und $\Phi: U \to V$ eine Abbildung. Die Abbildung $\Phi$ sei \begriff{selbstabbildend}, d.h. es gilt
	\begin{align}
	\Phi(U) \subseteq U.\notag
	\end{align}
	Außerdem sei $\Phi$ \begriff{kontraktiv}, d.h. es gibt $\lambda \in [0,1)$, so dass
	\begin{align}
	\Vert\Phi(x) - \Phi(y)\Vert \le \lambda \Vert x-y\Vert, \text{für alle } x,y \in U.\notag
	\end{align}
	Dann besitzt $\Phi$ genau einen Fixpunkt $x^{*} \in U$. Weiterhin konvergiert die durch
	\begin{align}
	x^{k+1} := \Phi(x^k) \label{eq_1_1_1}
	\end{align}
	erzeugte Folge $\{x^k\}$ für jeden Startwert $x^0 \in U$ gegen $x^{*}$ und es gilt für alle $k \in \natur$
	\begin{align}
	\Vert x^{k+1} - x^{*}\Vert \le \frac{\lambda}{1 - \lambda}\Vert x^{k+1} - x^k\Vert \text{ a posteriori Fehlerabschätzung},\\
	\Vert x^{k+1} - x^{*}\Vert \le \frac{\lambda^{k+1}}{1 - \lambda}\Vert x^1 - x^0\Vert \text{ a priori Fehlerabschätzung},\\
	\Vert x^{k+1} - x^{*}\Vert \le \frac{\lambda}{1 - \lambda}\Vert x^{k} - x^{*}\Vert \text{ Q-lineare Konvergenz mit Ordnung }\lambda.
	\end{align}
\end{proposition}

\begin{proof}
	Verlesung zur Analysis.
\end{proof}

Die in \propref{1_1_1} vorkommende Zahl $\lambda \in [0,1)$ wird \begriff{Kontraktionskonstante} genannt. 
\section{Gewöhnliche Iterationsverfahren}

Durch \ref{eq_1_1_1} erklärte Verfahren heißt \begriff{gewöhnliches Interationsverfahren} oder \begriff{Fixpunktiteration}. Kritisch ist dabei, ob die Voraussetzungen ($\Phi$ ist selbstabbildend und kontraktiv) erfüllt werden können. Dies wird in diesem Abschnitt im Fall $V = \Rn$ mit einer beliebigen aber festen Vektornorm $\Vert \cdot \Vert$ untersucht. Die zugeordnete Matrixnorm wurde mit $\Vert \cdot \Vert_{\ast}$ bezeichnet.

\begin{lemma}
	\proplbl{1_1_2}
	Sei $S \subseteq \Rn$ offen und konvex und $\Phi: D \to \Rn$ stetig differenzierbar. Falls $L > 0$ existiert mit
	\begin{align}
	\Vert \Phi'(x) \Vert_{\ast} \le L \text{ für alle } x \in D, \label{eq_1_1_5}
	\end{align}
	dann ist $\Phi$ Lipschitz-stetig in $D$ mit der Lipschitz-Konstante $L$, d.h. es gilt
	\begin{align}
	\Vert \Phi(x) - \Phi(y)\Vert \le L \Vert x-y \Vert \text{ für alle } x \in D. \label{eq_1_1_6}
	\end{align}
	Die Umkehrung dieser Aussage ist ebenfalls richtig.
\end{lemma}

\begin{proof}
	\begin{enumerate}
		\item Sei \ref{eq_1_1_5} erfüllt. Mit Satz 5.1 aus der Vorlesung ENM folgt %TODO find out which prop is meant!
		\begin{align}
		\norm{\Phi(x) - \Phi(y)}_{\ast} = \norm{\int_{0}^{1} \Phi'(y + t(x-y))(x-y) dt} \le  \norm{x-y} \sup_{t \in [0,1]} \norm{\Phi'(y+t(x-y))}_{\ast}
		\end{align}
		für alle $x,y \in D$. Also liefert \ref{eq_1_1_5} unter Beachtung der Konvexität von $D$ die Behauptung.
		\item Sei nun \ref{eq_1_1_6} erfüllt. Angenommen es gibt $\hat{y} \in D$ mit
		\begin{align}
		\norm{\Phi'(\hat{y})}_{\ast} > L. \label{eq_1_1_7}
		\end{align}
		Unter Berücksichtigung der Definition der zugeordneten Matrixnorm $\norm{\cdot}_{\ast}$ folgt, dass $d \in \Rn$ existiert mit $\Vert d \Vert = 1$ und $\norm{\Phi'(\hat{y}d)} = \norm{\Phi(\hat{y})}_{\ast}$. Wendet man nun ENM mit $x := \hat{y} + sd$ und $y := \hat{y}$ an, so folgt für alle $s > 0$ hinreichend klein
		\begin{align}
		\norm{\Phi(\hat{y} + sd) - \Phi(\hat{y})} \le L \norm{sd} = sL
		\end{align}
		und 
		\begin{align}
		\norm{\Phi(\hat{y} + sd) - \Phi(\hat{y})} &= \norm{\int_{0}^{1} \Phi'(\hat{y} + tsd)(sd)dt}\notag \\
		&= \norm{\int_{0}^{1} \Phi'(\hat{y} + tsd)(sd)dt + \int_{0}^{1} \Phi'(\hat{y})(sd)(sd)dt - \int_{0}^{1} \Phi'(\hat{y})(sd)(sd)dt}\notag \\
		&\ge s\norm{\Phi'(\hat{y}d)} - s\norm{d} \sup_{t \in [0,1]}\norm{\Phi'(\hat{y} + tsd) - \Phi'(\hat{y})}_{\ast} \notag \\
		&= s (\norm{\Phi'(\hat{y})}_{\ast} - \sup_{t \in [0,1]}\norm{\Phi'(\hat{y} + tsd) - \Phi'(\hat{y})}_{\ast}) \notag\\
		&> sL, \notag
		\end{align}
		wobei sich die letzte Ungleichung wegen \ref{eq_1_1_7} und der Stetigkeit von $\Phi'$ ergibt. Offenbar hat man damit einen Widerspruch, so dass die Annahme falsch ist.
	\end{enumerate}
\end{proof}

\begin{example}
	Die Nullstellenaufgabe $\cos x - 2x = 0$ sei zu lösen. Eine mögliche Formulierung als Fixpunktaufgabe ist
	\begin{align}
	\Phi(x) = x \text{   mit  } \Phi(x) := -x + \cos x \notag
	\end{align}
	Offenbar ist $\Phi: \R \to \R$ selbstabbildend. Weiter ergibt sich
	\begin{align}
	\Phi'(x) = -1 - \sin x \notag
	\end{align}
	Für $x \in D := (0,1)$ gilt daher $\vert \Phi' (x)\vert > 1$. Mit \propref{1_1_2} folgt $\vert \Phi(x) - \Phi(y)\vert \ge \abs{x-y}$ für mindestens ein Paar $(x,y) \in D \times D$. Somit ist $\Phi$ in $D$ nicht kontrahierend.
	Definiert man $\Phi$ aber durch $\Phi(x) := \sfrac{1}{2}\cos x$, so ist die Fixpunktaufgabe $\sfrac{1}{2}\cos x = x$ wiederum zur Nullstellenaufgabe äquivalent und es folgt
	\begin{align}
	\Phi'(x) = \frac{1}{2}\sin x. \notag
	\end{align} 
	Damit hat man $\vert \Phi'(x)\vert \le \sfrac{1}{2}$ für alle $x \in \R$. Also ist die zuletzt definierte Abbildung $\Phi$ kontrahierend auf $\R$ (und dort natürlich selbstabbildend), so dass die Voraussetzungen des Banachschen Fixpunktsatzes erfüllt sind. Die Fixpunktiteration mit $\Phi(x) = \sfrac{1}{2}\cos x$ und $x^0 := 1$ ergibt:
	\begin{align}
	\text{ add table here!!!}
	\end{align} %TODO add plot!
\end{example}

Nehmen wir an, die Voraussetzungen des Banachschen Fixpunktsatzes seien gegeben. Dann hängt die Konvergenzgeschwindigkeit der Fixpunktiteration offenbar von der Kontraktionskonstanten $\lambda \in [0,1)$ ab. Je kleiner $\lambda$ ist, desto schneller ist ist die Konvergenzgeschwindigkeit. Unter Umständen kann die Umformulierung einer Fixpunktaufgabe mit Hilfe einer anderen Fixpunktabbildung helfen, die Konvergenzgeschwindigkeit zu verbessern (ggf. auf Kosten der Größe der Menge $U$, in der die Voraussetzungen des Banachschen Fixpunktsatzes erfüllt sind.) Ein Beispiel zu Konstruktion einer Fixpunktabbildung mit lokal beliebig kleiner Kontraktionskonstante gibt Abschnitt 1.4. In Abschnitt 2.1 wird gezeigt, wie Fixpunktabbildungen zu iterativen Lösung von linearen Gleichungssystemen eingesetzt werden können.
Im Weiteren bezeichne $B(x^{\ast}, r) :=$ die abgeschlossene Kugel um $x^{\ast}$ mit Radius $r$ (bzgl. einer passenden Norm).

\begin{proposition}[Ostrowski]
		\proplbl{1_3_4}
	Seien $D \subseteq \Rn$ offen und $\Phi: D \to \Rn$ stetig differenzierbar. Die Abbildung $\Phi$ besitze einen Fixpunkt $x^{\ast} \in D$ mit $\Vert \Phi'(x^{\ast})\Vert_{\ast} < 1$. Dann existiert $r > 0$, so dass das gewöhnliche Iterationsverfahren für jeden Startpunkt $x^0 \in B(x^{\ast}, r)$ gegen $x^{\ast}$ konvergiert.
\end{proposition}

\begin{proof}
	Da $\Phi$ stetig differenzierbar ist und $\norm{\Phi'(x*)}_{\ast} < 1$, gibt es $\lambda \in[0,1]$ und $r > 0$, sodass
	\begin{align}
		\norm{\Phi'(x)}_{\ast} \le \lambda \quad \text{ für alle } x\in B(x*,r).\notag
	\end{align}
	Nach \propref{1_1_2} gilt daher
	\begin{align}
		\norm{\Phi(x) - \Phi(y)}\le \lambda\norm{x-y} \quad\text{ für alle }x,y \in B(x*,r).
	\end{align}
	Insbesondere folgt hieraus
	\begin{align}
		\norm{\Phi(x) - \Phi(x*)} = \norm{\Phi(x) - x*} \le \lambda \norm{x-x*} \quad \text{ für alle } x \in B(x*,r)
	\end{align}
	und damit $\Phi(x) \in B(x*,r)$ für alle $x \in B(x*,r)$. Also ist $\Phi$ bzgl. $B(x*,r)$ selbstabbildend und kontraktiv. Daher liefert \propref{1_1_1} die gewünschte Aussage.
\end{proof}
\section{Das \person{Newton}-Verfahren als Fixpunktiteration}

Sei $D \subseteq \Rn$ offen und $F: D \to \Rn$ stetig differenzierbar. Die Nullstellenaufgabe
\begin{align}
	F(x) = 0\notag
\end{align}
wird nun in eine äquivalente Fixpunktaufgabe überführt. Dazu nehmen wir an, dass $x^{\ast}$ eine reguläre Nullstelle von $F$ ist. Wegen der vorausgesetzten Stetigkeit von $F'$ gibt es $r>0$ hinreichend klein, so dass $F'(x)$ für $x \in B(x^{\ast},r)$ regulär ist. Damit erhält man
\begin{align}
	F(x) = 0 \Leftrightarrow 0 = -F'(x)^{-1}F(x) \Leftrightarrow x = x - F'(x)^{-1}F(x).\notag
\end{align}
für $x \in B(x^{\ast},r)$. Definiert man $\Phi: B(x^{\ast},r) \to \Rn$ durch
\begin{align}
	\Phi(x):= x - F^{\ast}(x)^{-1}F(x). \label{eq_1_3_8}
\end{align}
so kann das Newton-Verfahren als Fixpunktverfahren mit $\Phi$ als Fixpunktabbildung interpretiert werden. Ob $\Phi$ selbstabbildend und kontrahierend ist, müsste noch untersucht werden. Hier soll nur die Kontraktionseigenschaft in $B(x^{\ast},r)$ für $r>1$ hinreichend klein betrachtet werden. Die Eigenschaft der Selbstabbildung ergibt sich dann wie im Beweis zu \propref{1_3_4}.

\begin{lemma}
	Sei $D\subseteq \Rn$ offen und $F: D \to \Rn$ stetig differenzierbar. Weiter sei $x^{\ast}\in D$ eine reguläre Nullstelle von $F$. Dann ist $\Phi$ in $x^{\ast}$ differenzierbar mit $\Phi'(x^{\ast}) = 0$.
\end{lemma}

\begin{proof}
	Wie zuvor gezeigt wurde, ist die durch \cref{eq_1_3_8} definierte Abbildung $\Phi$ in $B(x^{\ast},r)\subset D$ hinreichend kleines $r>0$ wohldefiniert. Falls
	\begin{align}
		\lim_{x\to x^{\ast}} \frac{\norm{\Phi(x) - \Phi(x^{\ast}) - G(x-x^{\ast})}}{\norm{x-x^{\ast}}} \label{eq_1_4_9}
	\end{align}
	mit $G = 0\in \Rnn$ gilt, folgt die Behauptung des Lemmas aus der Definition der Fréchet-Differenzierbarkeit. Unter Beachtung von $\Phi(x^{\ast}) = x^{\ast}$ ergibt sich
	\begin{align}
	\Phi(x) - \Phi(x^{\ast}) = x - F'(x)^{-1}F(x) - x^{\ast} = -F'(x)^{-1}(F'(x))(x^{\ast}-x)+F(x))\notag
	\end{align}
	und mit Satz 5.1 aus der Vorlesung ENM folgt weiter
	\begin{align}
		\Phi(x) - \Phi(x^{\ast}) = F'(x)^{-1}\left( -F(x^{\ast}) + \int_{0}^{1} (F'(x+t(x^{\ast}-x)) - F'(x))(x^{\ast}-x)\diff t\right)\label{eq_1_4_10}
	\end{align}
	für alle $x \inn B(x^{\ast},r)$. Die Stetigkeit von $F'$ auf der kompakten Menge $B(x^{\ast},r)$ impliziert, dass $F'$ dort auch gleichmäßig stetig ist. Also gibt es zu jedem $\epsilon > 0$ ein $\delta(\epsilon) > 0$, so dass auch
	\begin{align}
		\norm{x+t(x^{\ast}-x) - x}\le \delta(\epsilon) \quad \text{ die Beziehung } \norm{F'(x+t(x^{\ast}-x)) -F'(x)}_{\ast} \le \epsilon\notag
	\end{align}
	für beliebige $x \in B(x^{\ast},r)$ und $t \in [0,1]$ folgt. Damit hat man
	\begin{align}
		\lim_{x\to x^{\ast}}\max_{t\in[0,1]} \norm{F'(x+t(x^{\ast}-x)) -F'(x)}_{\ast} = 0 \notag
	\end{align}
	und
	\begin{align}
		\lim_{x\to x^{\ast}} \frac{\norm{\int_{0}^{1} (F'(x+t(x^{\ast} - x)) -F'(x))(x^{\ast}-x)\diff t}_{\ast}}{\norm{x-x^{\ast}}} = 0 \notag
	\end{align}
	Somit erhält man aus \cref{eq_1_4_10} unter Beachtung von $F(x^{\ast}) = 0$ und der Regularität von $F'(x)$
	\begin{align}
		\lim_{x\to x^{\ast}} \frac{\norm{\Phi(x) - \Phi(x^{\ast})}}{\norm{x-x^{\ast}}O(x-x^{\ast})} = 0,\notag
	\end{align} % different to the script!
	d.h. \cref{eq_1_4_9} ist für $G=0$ erfüllt.
\end{proof}

\begin{remark}
	Falls $F$ in einer Umgebung von $x^{\ast}$ sogar zweimal stetig differenzierbar und damit $\Phi$ dort stetig differenzierbar ist, zeigt \propref{1_1_2}, dass $\norm{\Phi'(x)}_{\ast} \le L$ für alle $x \in D \cap B(x^{\ast},r(L))$ gilt. D.h. die Kontraktionskonstante der Fixpunktabbildung $\Phi$ in \cref{eq_1_3_8} in einer Kugel $B(x^{\ast},r)$ konvergiert gegen $0$, wenn man den Radius $r$ gegen $0$ gehen lässt. Ferner gibt es Sätze, bei denen unter geeigneten Voraussetzungen eine bestimmte lokale Konvergenzgeschwindigkeit (Q-Ordnung) gezeigt wird (etwa die Q Ordnung $2$, wenn insbesondere $\Phi'$ stetig ist und $\Phi'(x^{\ast}) = 0$ gilt).
\end{remark}

\chapter{Iterative Verfahren für lineare Gleichungssysteme}
Seien eine reguläre Matrix $A \in \R^{n \times n}$ und $b \in Rn$ gegeben. In diesem Kapitel werden iterative Verfahren zur Lösung des linearen Gleichungssystems
\begin{align}
Ax = b\label{eq_2_2_1}
\end{align}
betrachtet.
%TODO fix counter, this equation should be the 1 and the one in the following section, 2!
\section{Fixpunktiteration}

Grundidee dieser Verfahren ist die geeignete Umformulierung des System $Ax = b$ als Fixpunktaufgabe und die Anwendung des gewöhnlichen Iterationsverfahrens. Die hier betrachtete (zu \eqref{eq_2_2_1} äquivalente) Fixpunktaufgabe lautet
\begin{align}
	x = x - B^{-1}(Ax - b),\notag
\end{align}
wobei $B \in \R^{n \times n}$ eine noch zu wählende reguläre Matrix ist. Bei Wahl eines Startpunktes $x^0 \in \Rn$ ergibt sich das gewöhnliche Iterationsverfahren damit zu
\begin{align}
	x^{k+1} := x^{k} - B^{-1}(A x^k -b) = (I - B^{-1}A)x^{k} + B^{-1}b, \qquad k = 0,1,2,\dots \label{eq_2_2_2} %TODO add underbraces for M = I - B^{-1}A and c = B^{-1}b
\end{align}
Mit den Bezeichnung $M := I - B^{-1}A$ und $c:= B^{-1}b$ untersuchen wir deshalb die Iterationsverschrift
\begin{align}
	x^{k+1} := Mx^k + c. \label{eq_2_2_3}
\end{align}
Die zugehörige Fixpunktabbildung $\Phi: \Rn \to \Rn$ ist damit offenbar gegeben durch
\begin{align}
	\Phi(x) := Mx + c.\notag
\end{align}
\begin{proposition}
	\proplbl{2_2_1}
	Es sei $B \in \Rnn$ regulär und mit $M:= I - B^{-1}A$ gelte
	\begin{align}
		\lambda := \norm{M}_{\ast} < 1 \label{eq_2_1_4}
	\end{align}
	wobei $\norm{\cdot}_{\ast}$ die einer Vektornorm $\norm{\cdot}$ zugeordnete Matrixnorm bezeichnet. Dann gilt:
	\begin{enumerate} %TODO add alph
		\item Die für eine beliebiges $x^0 \in \Rn$ durch \eqref{eq_2_2_3} erzeugte Folge $\set{x^k}$ konvergiert gegen die eindeutige Lösung $x*$ des linearen Gleichungssystems \eqref{eq_2_2_1}.
		\item Die Abschätzungen \eqref{eq_1_2_2} - \eqref{eq_1_2_4} sind für alle $k \in \N$ erfüllt.
	\end{enumerate}
\end{proposition}

\begin{proof}
	Direkte Folgerung aus dem Banachschen Fixpunktsatz (\propref{1_1_1})
\end{proof}

%%%%%%%%%%%%%%%%%%%%%%%%%%%%%%%%%% 3rd lecture %%%%%%%%%%%%%%%%%%%%%%%%%%%%%%%%%%%%%%%%%%

\begin{remark}
	In \propref{2_2_1}a) kann die Folgerung \eqref{eq_2_1_4} durch die Bedingung
	\begin{align}
	\rho(M) < 1\label{eq_2_1_5}
	\end{align}
	ersetzt werden. Da
	\begin{align}
		\rho(C) \le \norm{C}_{\ast} \quad \text{ für alle } C \in \Rnn \notag
	\end{align}
	für jede beliebige zugeordnete Matrixnorm $\norm{\cdot}_{\ast}$ gilt (vgl. Übungsaufgabe), ist \eqref{eq_2_1_5} eine schwächere Forderung als \eqref{eq_2_1_4}. Andererseits gibt es zu jedem Paar $(C,\epsilon) \in \Rnn \times (0,\infty)$ eine zugeordnete Matrixnorm $\norm{\cdot}_{(C,\epsilon)}$, so dass
	\begin{align}
		\norm{C}_{(C,\epsilon)} \le \rho(C) + \epsilon. \notag
	\end{align}
	Dabei ist $\rho(C)$ der \begriff{Spektralradius} der Matrix $C \in \Rnn$, d.h.
	\begin{align}
		\rho(C) := \max_{i = 1,\dots,n}\abs{\lambda_i}, \notag
	\end{align}
	wobei $\lambda_1,\dots,\lambda_n \in \C$ die Eigenwerte der Matrix $C \in \Rnn$ bezeichnen. Man kann weiter zeigen, dass \eqref{eq_2_1_5} auch notwendig dafür ist, dass die durch \eqref{eq_2_2_2} erzeugte Folge $\set{x^k}$ für jedes $x^0$ gegen $x*$ konvergiert.
\end{remark}

Um eine Matrix $B$ zu finden, so dass einerseits der Aufwand pro Iteration \eqref{eq_2_2_2} niedrig und andererseits die Bedingung \eqref{eq_2_1_4} bzw. \eqref{eq_2_1_5} erfüllt ist, betrachten wir die folgende Zerlegung
\begin{align}
	A = L + D + R \notag
\end{align}

der Matrix $A$, wobei $D:= \diag(a_{11}, \dots, a_{nn})$ die aus den Diagonalelementen von $A$ bestehende Diagonalmatrix bezeichnet und $L$ bzw. $R$ eine untere bzw. obere Dreiecksmatrix ist mit

\begin{align}
	L = 
	\begin{pmatrix}
		0      &        &        &           & \\
		a_{21} & 0      &        &           & \\
		a_{31} & a_{32} & 0      &           & \\
		\vdots &        & \ddots & \ddots    & \\
		a_{n1} & \cdots & \cdots & a_{n,n-1} & 0
	\end{pmatrix}
	 \bzw R = 
	 \begin{pmatrix}
	 0      & a_{12} & a_{13} & \dots     & a_{1n}\\
	        & 0      & a_{23} & \dots     & a_{2n}\\
	        &        & \ddots & \ddots    & \vdots\\
	        &        &        & 0         & a_{n-1,n}\\
	        &        &        & a_{n,n-1} & 0
	 \end{pmatrix}.\notag
\end{align}

\subsection{Das \person{Jacobi}-Verfahren}
Wir setzen hier voraus, dass $D$ regulär ist und wählen
\begin{align}
	B:= D \label{eq_1_2_6}
\end{align}
Damit ergibt sich die Iterationsvorschrift
\begin{align}
	x^{k+1} = x^k - D^{-1}(Ax^k - b) = -D^{-1}(L+R)x^k+D^{-1}b. \label{eq_1_2_7}
\end{align}
In \eqref{eq_2_2_3} ist entsprechend
\begin{align}
	M:= M_J := -D^{-1}(L+R) \text{ und } c:= c_J := D^{-1}b\notag 
\end{align}
zu wählen. Dieses Verfahren heißt \begriff{Gesamtschrittverfahren} oder \begriff{Jacobi-Verfahren}. Der Aufwand pro Schritt (Brechnung von $x^{k+1}$ aus $x^k$) beträgt $\Landau(n^2)$ bei voll besetzter Matrix $A$ und mindestens $\Landau(n)$, falls $A$ schwach besetzt ist.

\begin{proposition}
	Die Matrix $A$ sei streng diagonaldominant (vgl. Definition 3.1 der Vorlesung ENM). Dann ist die Matrix $B$ aus \eqref{eq_1_2_6} regulär und es gilt
	\begin{align}
	\norm{M_J}_{\infty} \le \lambda_{SD} := \max_{i = 1,\dots,n} \frac{1}{\abs{a_{ii}}} \sum_{\substack{j =1 \\ j\neq i}^{n}} \abs{a_{ij}} < 1.
	\end{align}
\end{proposition}

\begin{proof}
	Die Regularität von $B$ ergibt sich sofort aus der strengen Diagonaldominanz von $A$. Nutzt man die Definition der Zeilensummennorm $\norm{\cdot}_{\infty}$ erhält man sofort
	\begin{align}
		\norm{M_J}_{\infty} = \norm{D^{-1}(L+R)}_{\infty} =\max_{i = 1,\dots,n} \frac{1}{\abs{a_{ii}}} \sum_{\substack{j =1 \\ j\neq i}^{n}} \abs{a_{ij}}  = \lambda_{SD}.
	\end{align}
	Die vorrausgesetzte strenge Diagonaldominanz von $A$ sichert $\lambda_{SD} < 1$.
\end{proof}

\subsection{Das \person{Gauss-Seidel}-Verfahren}
Wir setzen hier voraus, dass $L + D$ regulär ist und wählen

\begin{align}
	B := L + D \label{1_2_8}
\end{align}

Damit ergibt sich die Iterationsvorschrift

\begin{align}
	x^{k+1} = x^k - (L+D)^{-1}(Ax^k - b) = - (L+D)^{-1}R x^k + (L+D)^{-1}b. \label{1_2_9}.
\end{align}

In \eqref{eq_2_2_3} ist entsprechend

\begin{align}
	M:= M{GS} := - (L+D)^{-1}R \text{ und } c:= c_{GS} := (L+D)^{-1}b \notag
\end{align}

zu wählen. Dieses Verfahren heißt \begriff{Einzelschrittverfahren} oder \begriff{Gauß-Seidel-Verfahren}. Der Aufwand pro Schritt beträgt im ungünstigsten Fall $\Landau(n^2)$. Verbesserungen sind möglich, wenn eine Sparse-Struktur in $A$ ausgenutzt werden kann.

\begin{proposition}
	Die Matrix $A$ sei streng diagonaldominant ($\nearrow$ Definition 3.1 der Vorlesung ENM). Dann ist die Matrix $B$ aus \eqref{1_2_8} regulär und es gilt
	\begin{align}
		\norm{M_{GS}}_{\infty} \le \lambda_{SD} <1.
	\end{align}
\end{proposition}

\begin{proof}
	Die Regularität von $B$ folgt sofort aus der strengen Diagonaldominanz von $A$. Weiter ergibt sich
	\begin{align}
		\norm{M_{GS}}_{\infty} = \norm{(L+D)^{-1}R}_{\infty} = \sup_{\norm{y}_{\infty}=1} \norm{(L+D)^{-1}Ry}_{\infty}. \notag
	\end{align}
	Um für einen festen Vektor $y$ mit $\norm{y}_{\infty} = 1$ eine Abschätzung für die rechte Seite zu erhalten, setzen wir $z:= (L+D)^{-1}Ry$. Damit gilt
	\begin{align}
		(D+L)z = Ry \label{1_2_10}
	\end{align}
	und
	\begin{align}
	z_1 = \frac{1}{a_{11}} \sum_{j=1}^{n} a_{1j}y_j. \notag
	\end{align}
	Daraus folgt (da $\lambda_{SD} < 1$ wegen der strengen Diagonaldominanz von $A$)
	\begin{align}
		\abs{z_1} 
		\le \frac{1}{\abs{a_{11}}} \sum_{j=2}^{n} \abs{a_{1j}}\abs{y_j} 
		\le \sum_{j=2}^{n} \abs{a_{1j}} \le \lambda_{SD} < 1.\notag
	\end{align}
	Nehmen wir nun an, dass
	\begin{align}
		\abs{z_1} \le \text{ für } i = 1, \dots, k-1, \notag
	\end{align}
	für ein $k \in \set{2,\dots,n}$ gilt. Dann folgt wegen \eqref{1_2_10} und $\norm{y}_{\infty} = 1$
	\begin{align}
		\abs{z_k} = \frac{1}{\abs{a_{kk}}} \abs{-\sum_{i=1}^{k-1} a_{ki}z_i + \sum_{i=k+1}^{n} a_{ki}y_i} 
		\le \frac{1}{\abs{a{kk}}} \brackets{\sum_{i=1}^{k-1} \abs{a_{ki}} + \sum_{i=k+1}^{n} \abs{a_{ki}}} \le \lambda_{SD}. \notag
	\end{align}
	Somit hat man induktiv $\abs{z_k} \le \lambda_{SD}$ für $k = 1, \dots, n$ und damit
	\begin{align}
		\norm{(L+D)^{-1}Ry}_{\infty} = \norm{z}_{\infty} \le \lambda_{SD} \notag
	\end{align}
	für beliebige $y$ mit $\norm{y}_{\infty} = 1$.
\end{proof}

\subsection{SOR-Verfahren}


\section{\person{Krylov}-Raum-basierte Verfahren}

\subsection{\person{Krylov}-Räume}

Für $A\in\Rnn$, $r\in\Rn$ und $k\in\natur$ ist der $k$-te \begriff{Krylov-Raum} gegeben durch $\mathcal{K}_0=\{0\}$ und
\begin{align}
	\mathcal{K}_k(r,A) = \Span\{r,Ar,A^2r,...,A^{k-1}r\}\quad\text{ für } k>0\notag
\end{align}
Offenbar ist $\dim(\mathcal{K}_k(r,A))\le\min\{k,n\}$ für alle $k\in\natur$.

\begin{lemma}
	\proplbl{lemma_2_6}
	Es seien $A\in\Rnn$, $r\in\Rn\backslash\{0\}$ und $k\in\natur$ gegeben. Dann sind folgende Aussagen äquivalent:
	\begin{enumerate}[label=(\alph*)]
		\item $\dim(\mathcal{K}_{k+1}(r,A))< k+1$
		\item $\mathcal{K}_k(r,A) = \mathcal{K}_{k+1}(r,A)$
	\end{enumerate}
\end{lemma}
\begin{proof}
	\begin{itemize}
		\item (a) $\Rightarrow$ (b): Nach Voraussetzung gibt es $l\in\{1,...,k\}$ und $\alpha_0,...,\alpha_l\in\R$, so dass
		\begin{align}
			A^lr = \sum_{i=0}^{l-1} \alpha_iA^ir \notag
		\end{align}
		Multiplikation mit $A^{k-l}$ liefert
		\begin{align}
			A^kr = \sum_{i=0}^{l-1} \alpha_iA^{k-l+i}r\in\mathcal{K}_k(r,A)\notag
		\end{align}
		Also folgt $\mathcal{K}_k(r,A) = \mathcal{K}_{k+1}(r,A)$.
		\item (b) $\Rightarrow$ (a): Offensichtlich
	\end{itemize}
\end{proof}

\begin{proposition}
	\proplbl{satz_2_7}
	Es seien $A\in\Rnn$ regulär, $x^0\in\Rn$ mit $r^0=b-Ax^0\neq 0$ gegeben. Dann sind folgende Aussagen für $k\in\natur$ äquivalent:
	\begin{enumerate}[label=(\alph*)]
		\item $\mathcal{K}_k(r^0,A) = \mathcal{K}_{k+1}(r^0,A)$
		\item $x^\ast = A^{-1}b\in x^0 + \mathcal{K}_k(r^0,A)$
	\end{enumerate}
\end{proposition}
\begin{proof}
	\begin{itemize}
		\item (a) $\Rightarrow$ (b): Wegen \propref{lemma_2_6} gibt es $l\in\{0,...,k\}$ und $\mu_l,...,\mu_k\in\R$, so dass $\mu_l\neq 0$ und
		\begin{align}
			0 &= \sum_{i=l}^{k} \mu_iA^ir^0 \notag \\
			&= \mu_lA^lr^0 + \sum_{i=l+1}^{k} \mu_iA^ir^0 \notag
		\end{align}
		wobei der Summationsterm auf der rechten Seite entfällt, wenn $l=k$. Wegen der Regularität von $A$ kann man die Gleichung mit $A^{-(l+1)}$ multiplizieren. Für $l=k$ liefert dies $0=A^{-1}r^0 = A^{-1}b-x^0\in\mathcal{K}_k(r^0,A)$. Für $l<k$ folgt
		\begin{align}
			A^{-1}b - x^0 &= A^{-1}r \notag \\
			&= -\frac{1}{\mu_l} \sum_{i=l+1}^k \mu_iA^{i-l-1}r ^0 \notag \\
			&= -\frac{1}{\mu_l}\sum_{i=0}^{k-l-1} \mu_{i+l+1}A^ir^0\in\mathcal{K}_k(r^0,A) \notag
		\end{align}
		Somit gilt Aussage (b).
		\item (b) $\Rightarrow$ (a): Nach Voraussetzung gilt $x^\ast\in x_0 + \mathcal{K}_k(r^0,A)$. Durch Multiplikation mit $A$ folgt
		\begin{align}
			b&\in Ax^0 + A\mathcal{K}_k(r^0,A)\notag \\
			&= Ax^0 + \Span\{Ar^0,A^2r^0,A^3r^0,...,A^kr^0\}\notag
		\end{align}
		Also ist $r^0=b-Ax^0$ eine Linearkombination der Vektoren $Ar^0,A^2r^0,A^3r^0,...,A^kr^0$ und es gilt
		\begin{align}
			\dim(\mathcal{K}_{k+1}(r^0,A)) < k+1\notag
		\end{align}
		 \propref{lemma_2_6} liefert damit die Gültigkeit von Aussage (a).
	\end{itemize}
\end{proof}

\begin{remark}
	Offenbar gibt es $k^\ast\in\{1,...,n\}$, so dass Aussage (a) von \propref{satz_2_7} für $k^\ast$ zutrifft, aber für kein $k<k^\ast$ erfüllt ist. \propref{satz_2_7} zeigt daher, dass die exakte Lösung $x^\ast$ von $Ax=b$ in $x^0 + \mathcal{K}_{k^\ast}(r^0,A)$ liegt. Man kann also nun versuchen, eine Folge $\{x^k\}$ mit $x^k\in x^0 + \mathcal{K}_k(r^0,A)$ zu bestimmen, so dass $x^k$ das Gleichungssystem $Ax=b$ (geeignet) näherungsweise löst. Dazu werden in nächsten Abschnitt zwei grundlegende Ansätze angegeben (Minimum-Residuum und Galerkin).
\end{remark}

\subsection{Basisalgorithmen zur Lösung von $Ax=b$}

\begin{algorithm}[Minimum-Residuum Basisalgorithmus]
	Input: $x^0\in\R$, $A,B\in\Rnn$ regulär, $b\in\Rn$
\begin{lstlisting}
while %$r^k\neq 0$% do
 k = k + 1;
 compute %$x^k\in\Rn\text{ als Lösung von}$%
 %$\quad f_B(x) = \frac{1}{2}\Vert b-Ax\Vert^2_B\to\min\quad\text{bei } x\in x^0+\mathcal{K}_k(r^0,A)$%
 %$r^k$% = b - A%$x^k$%
enddo
\end{lstlisting}
	Output: $x^\ast=x^k$, $k^\ast=k$
\end{algorithm}

\begin{algorithm}[Galerkin Basisalgorithmus]
	Input: $x^0\in\R$, $A,B\in\Rnn$ regulär, $b\in\Rn$
	\begin{lstlisting}
%$r^0$% = b - A%$x^0$%;
k = 0;
while %$r^k\neq 0$% do
 k = k + 1;
 determine %Untervektorraum $\mathcal{L}_k$ von $\Rn$ mit $\dim(\mathcal{L}_k)=k$%;
 compute %$x^k\in x^0+\mathcal{K}_k(r^0,A)$ mit $b-Ax^k\perp\mathcal{L}_k$%;
 %$r^k$% = b - A%$x^k$%;
enddo
	\end{lstlisting}
	Output: $x^\ast=x^k$, $k^\ast=k$
\end{algorithm}

\subsection{Das CG-Verfahren}

\begin{algorithm}[CG-Verfahren]
	Input: $x^0\in\Rn$, $A\in\Rnn$ symmetrisch und positiv definit, $b\in\Rn$
	\begin{lstlisting}
%$d^0$% = %$r^0$% = b - A%$x^0$%;
k = 0;
while %$r^k\neq 0$% do
 %$t_k$% = %$\frac{\Vert r^k\Vert_2^2}{(d^k)^T Ad^k}$%;
 %$x^{k+1}$% = %$x^k$% + %$t_kd^k$%;
 %$r^{k+1}$% = b - A%$x^{k+1}$% = %$r^k$% - %$t_kAd^k$%;
 %$\beta_k$% = %$\frac{\Vert r^{k+1}\Vert_2^2}{\Vert r^k\Vert_2^2}$%;
 %$d^{k+1}$% = %$r^{k+1}$% - %$\beta_kd^k$%;
 k = k + 1;
enddo
	\end{lstlisting}
	Output: $x^\ast=x^k$
\end{algorithm}

\subsection{Fehlerverhalten des CG-Verfahrens}

\subsection{Vorkonditionierung}

\begin{algorithm}[CG-Verfahren für $\tilde{A}\tilde{x}=\tilde{b}$]
	Input: $\tilde{x}^0\in\Rn$, $\tilde{A}\in\Rnn$ symmetrisch und positiv definit, $\tilde{b}\in\Rn$
	\begin{lstlisting}
	%$\tilde{d}^0$% = %$\tilde{r}^0$% = %$\tilde{b} - \tilde{A}\tilde{x}^0$%;
	k = 0;
	while %$\tilde{r}^k\neq 0$% do
	%$\tilde{t}_k$% = %$\frac{\Vert \tilde{r}^k\Vert_2^2}{(\tilde{d}^k)^T \tilde{A}\tilde{d}^k}$%;
	%$\tilde{x}^{k+1}$% = %$\tilde{x}^k$% + %$\tilde{t}_k\tilde{d}^k$%;
	%$\tilde{r}^{k+1}$% = %$\tilde{b} - \tilde{A}\tilde{x}^{k+1}$% = %$\tilde{r}^k$% - %$\tilde{t}_k\tilde{A}\tilde{d}^k$%;
	%$\tilde{\beta}_k$% = %$\frac{\Vert \tilde{r}^{k+1}\Vert_2^2}{\Vert \tilde{r}^k\Vert_2^2}$%;
	%$\tilde{d}^{k+1}$% = %$\tilde{r}^{k+1}$% - %$\tilde{\beta}_k\tilde{d}^k$%;
	k = k + 1;
	enddo
	\end{lstlisting}
	Output: $x^\ast=x^k$
\end{algorithm}

\begin{algorithm}[Vorkonditioniertes CG-Verfahren]
	Input: $x^0\in\Rn$, $A\in\Rnn$ symmetrisch und positiv definit, $b\in\Rn$
	\begin{lstlisting}
%$r^0$% = b - A%$x^0$%;
compute %$z^0$ als Lösung von $Pz = r^0$% 
%$d^0$% = %$z^0$%
k = 0;
while %$r^k\neq 0$% do
 %$t_k$% = %$\frac{(r^k)^T}{(d^k)^T Ad^k}$%;
 %$x^{k+1}$% = %$x^k$% + %$t_kd^k$%;
 %$r^{k+1}$% = %$r^k$% - %$t_kAd^k$%;
 compute %$z^{k+1}$ als Lösung von $Pz = r^{k+1}$%
 %$\beta_k$% = %$\frac{(r^{k+1})^T z^{k+1}}{(r^k)^T z^k}$%;
 %$d^{k+1}$% = %$z^{k+1}$% + %$\beta_kd^k$%;
 k = k + 1;
enddo
	\end{lstlisting}
	Output: $x^\ast=x^k$
\end{algorithm}

\subsection{Ausblick und Anmerkungen}

\chapter{Numerische Behandlung von Anfangswertaufgaben}
\section{Aufgabe und Lösbarkeit}

Es seien $a,b \in \R$ mit $a < b$, eine stetige Funktion $f$: $[a,b] \times \R^m \to \R^m$ und $y^0 \in \R^m$ gegeben. Unter \begriff{Anfangswertaufgabe} (AWA) 1. Ordnung versteht man das Problem, eine stetige Funktion $y$: $[a,b] \to \R^m$ zu ermitteln, so dass $y$ auf $(a,b)$ stetig differenzierbar ist und
\begin{align}
	y'(x) = f(x,y(x)) \quad \mit \quad y(a)=y^0 \notag
\end{align}
für alle $x \in [a,b]$ gilt. Eine solche Funktion wollen wir \begriff[Anfangswertaufgabe!]{Lösung} der AWA nennen. Kürzer schreibt man für die AWA auch
\begin{align}
	\label{3_1_1}
	y'=f(x,y) \quad \mit \quad y(a)=y^0
\end{align}
Die Existenz und Eindeutigkeit einer Lösung einer AWA hängen von den Eingangsinformationen $a,b,f$ und $y^0$ ab. Es gilt folgender Satz zur (globalen) Existenz und Eindeutigkeit einer Lösung auf $[a,b]$:

\begin{proposition}[\person{Picard-Lindelöf}: eine globale Version]
	Es sein $f$: $[a,b] \times \R^m \to \R^m$ stetig und es existiere $L>0$, so dass
	\begin{align}
		\label{3_1_2}
		\norm{f(x,y)-f(x,z)} \le L \norm{y-z} \quad \forall (x,y),(x,z) \in [a,b] \times \R^m
	\end{align}
	Dann besitzt \cref{3_1_1} für jedes $y^0 \in \R^m$ eine eindeutige Lösung.
\end{proposition}

Die Bedingung \cref{3_1_2} ist eine globale Lipschitz-Bedingung an $f$ bezüglich der zweiten Veränderlichen. Es ist leicht, AWA anzugeben, in denen diese Bedingung nicht erfüllt ist und keine Lösung in ganz $[a,b]$ existiert, zum Beispiel
\begin{align}
	y'= y^2 \quad \mit \quad y(0) = 1 \notag
\end{align}
Dafür erhält man für beliebige $x,y,z \in \R$
\begin{align}
	\abs{f(x,y) - f(x,z)} = \abs{y^2 - z^2} = \abs{y+z} \abs{y-z} \notag
\end{align}
das heißt die Bedingung \cref{3_1_2} kann in diesem Beispiel (global) nicht gelten. Die Lösung der AWA lautet $y(x) = \sfrac{-1}{x-1}$ für $x \in [0,1)$. Für Intervalle $[0,b]$ mit $b \ge 1$ existiert keine Lösung. Eine Abschwächung der Lipschitz-Bedingung \cref{3_1_2} gestattet folgender

\begin{proposition}[\person{Picard-Lindelöf}: eine lokale Version]
	Es sei $f$: $[a,b] \times \R^m \to \R^m$ stetig und zu jeder kompakten Menge $\mathcal{Y} \subset \R^m$ existiere $L_Y > 0$, so dass
	\begin{align}
		\norm{f(x,y)-f(x,z)} \le L_y \norm{y-z} \quad \forall (x,y),(x,z) \in [a,b] \times \mathcal{Y} \notag
	\end{align}
	Dann gibt es für jedes $y^0 \in \R^m$ ein Teilintervall $\mathcal{I} \subseteq [a,b]$ mit $a \in \mathcal{I}$, so dass die AWA \cref{3_1_1} auf $\mathcal{I}$ eine eindeutige Lösung besitzt.
\end{proposition}

Seien $g$: $[a,b]\times \R^n \to \R$ stetig und $\eta \in \R^n$. Jede explizite Differentialgleichung $n$-ter Ordnung
\begin{align}
	y^{(n)} = g(x,y,y',y'',...,y^{(n-1)})\notag
\end{align}
mit den Anfangsbedingungen
\begin{align}
	y(a) = \eta_1, \quad y'(a) = \eta_2, \quad y''(a) = \eta_3, \quad \dots \quad y^{(n-1)}(a) = \eta_n\notag
\end{align}
kann mittels Substitution
\begin{align}
	y_1 = y,\quad y_2=y', \quad y_3=y'', \quad \dots \quad y_n = y^{(n-1)}\notag
\end{align}
in eine AWA 1. Ordnung überführt werden:
\begin{align}
	\begin{pmatrix}
		y_1' \\ \vdots \\ y_n'
	\end{pmatrix} = 
	\begin{pmatrix}
		y_2 \\ \vdots \\ y_n \\ g(x,y_1,...,y_n)
	\end{pmatrix} 
	\quad \mit \quad 
	\begin{pmatrix}
	y_1(a) \\ \vdots \\ y_n(a)
	\end{pmatrix} = 
	\begin{pmatrix}
	\eta_1 \\ \vdots \\ \eta_n
	\end{pmatrix} \notag
\end{align}
\section{Einschrittverfahren}

\subsection{Grundlagen}

\subsection{Lokaler Diskretisierungsfehler und Konsistenz}

\subsection{Konvergenz von Einschrittverfahren}

\subsection{Stabilität gegenüber Rundungsfehlern}

\subsection{\person{Runge-Kutta}-Verfahren}
\section{Mehrschrittverfahren}

\subsection{Grundlagen}

Bei Mehrschrittverfahren (MSV) wird eine Näherung $y^{k+l}$ für $y(x_{k+l})$ in bestimmter Weise aus $l$ vorhergehenden Näherungen $y^k,y^{k+1},...,y^{k+l-1}$ bestimmt. Um dies genau zu beschreiben, seien zusätzlich zu $y^0$ (aus AWA) die Startwerte $y^1,...,y^{l-1}\in\R^m$ gegeben. Im Folgenden wollen wir von einem äquidistanten Gitter $G_h=\{x_0,...,x_N\}$ mit Schrittweite $h=\frac{b-a}{N}$ ausgehen. Ein \begriff{lineares Mehrschrittverfahren} mit $l$ Schritten erzeugt dann für $k=0,...,N-l$ die Iterierte $y^{k+l}$ aus $y^k,y^{k+1},...,y^{k+l-1}$ entsprechend
\begin{align}
	\label{3_17}
	\sum_{\nu=0}^{l} \alpha_\nu y^{k+\nu} = h\sum_{\nu=0}^l \beta_\nu f(x_{k+\nu},y^{k+\nu})
\end{align}
wobei $\alpha_\nu$, $\beta_\nu$ ($\nu=0,...,l$) reelle Parameter sind mit $\alpha_l\neq 0$ und $\abs{\alpha_0}+\abs{\beta_0}\neq 0$. Falls $\beta_l=0$, dann spricht man von einem \begriff[lineares Mehrschrittverfahren!]{expliziten} (sonst \begriff[lineares Mehrschrittverfahren!]{impliziten}) linearen MSV. Die MSV \cref{3_17} heißen linear, da die rechte Seite von \cref{3_17} linear von den Funktionswerten $f(x_{k+\nu},y^{k+\nu})$ abhängt. Einem linearen MSV ordnet man sein erstes und zweites \begriff[lineares Mehrschrittverfahren!]{charakteristisches Polynom} $\rho:\comp\to\comp$ und $\sigma:\comp\to\comp$ zu durch
\begin{align}
	\label{3_18}
	\rho(z)=\sum_{\nu=0}^l \alpha_\nu z^\nu\quad\text{und}\quad \sigma(z)=\sum_{\nu=0}^l \beta_\nu z^\nu\quad\forall z\in\comp
\end{align}
Das lineare MSV nach \person{Adams-Bashford} (1883) geht von
\begin{align}
	\label{3_19}
	y(x_{k+l}) - y(x_{k+l-1}) = \int_{x_{k+l-1}}^{x_{k+l}} f(x,y(x))\diff x
\end{align}
aus und approximiert den Integranden $f(x,y(x))$ durch ein Interpolationspolynom, nämlich
\begin{align}
	\label{3_20}
	\sum_{\nu=0}^{l-1} L_\nu(x)f(x_{k+\nu},y(x_{k+\nu}))
\end{align}
Dabei bezeichnen $L_\nu:\R\to\R$ für $\nu=0,...,l-1$ die \person{Langrange}-Polynome mit
\begin{align}
	L\nu(x) = \prod_{\substack{i=k \\ i\neq k+\nu}}^{k+l-1} \frac{x-x_i}{x_{k+\nu} - x_i} \quad\text{für } x\in\R\notag
\end{align}
Definiert man $\beta_\nu$ durch
\begin{align}
	\label{3_21}
	\int_{x_{k+l-1}}^{x_{k+l}} L_\nu(x)\diff x=h\beta_\nu
\end{align}
so liefert die Approximation von \cref{3_19} die Näherungsformel
\begin{align}
	y^{k+l}-y^{k+l-1} &= \sum_{\nu=0}^{l-1} \left(\int_{x_{k+l-1}}^{x_{k+l}} L_\nu(x)\diff x\right)f(x_{k+\nu},y^{k+\nu}) \notag \\
	&= h\sum_{\nu=0}^{l-1} \beta_\nu f(x_{k+\nu},y^{k+\nu}) \notag
\end{align}
also ein explizites $l$-schrittiges lineares MSV mit $\alpha_l=1$, $\alpha_{l-1}=-1$ und den durch \cref{3_21} definierten $\beta_0,...,\beta_{l-1}$ sowie $\beta_l=0$.

Beim linearen MSV nach \person{Adams-Moulton} (1926) wird die Summation in \cref{3_20} von $\nu=0$ bis $\nu=l$ erstreckt und dann analog vorgegangen. Dies ergibt das implizite lineare MSV
\begin{align}
	\label{3_22}
	y^{k+l} - y^{k+l-1} = h\sum_{\nu=0}^l \beta_\nu f(x_{k+\nu},y^{k+\nu})
\end{align}
Es erfolgt die (ggf. näherungsweise) Lösung eines im Allgemeinen nichtlinearen Gleichungssystems für $y^{k+l}$ und kann mit Hilfe des \begriff{Prädiktor-Korrektor-Prinzips} erfolgen. Dabei ermittelt man mit Hilfe eines expliziten linearen MSV (Prädiktor) eine erste Näherung $\zeta^0$ für $y^{k+l}$ und verbessert diese dann mit einem (näherungsweisen) Schritt eines impliziten linearen MSV (Korrektor). Zum Beispiel bestimme man $\zeta^0$ mit \person{Adams-Bashford}, das heißt
\begin{align}
	\zeta^0 = y^{k+l} + h\sum_{\nu=0}^{l-1} \beta_\nu f(x_{k+\nu},y^{k+\nu}) \notag
\end{align}
Danach wird eine Näherungslösung von \cref{3_22} (\person{Adams-Moulton}) etwa mittels Fixpunktiteration ermittelt
\begin{align}
	\zeta^j = y^{k+l-1} + h\beta_l^C f(x_{k+l},\zeta^{j-1}) + h\sum_{\nu=0}^{l-1} \beta_\nu^C f(x_{k+\nu},y^{k+\nu}) \notag
\end{align}
die für ein vorgegebenes $j\ge 1$ abgebrochen wird. Die Bezeichnung $\beta_\nu^C$ dient der Unterscheidung von den im Prädiktor verwendeten Parametern $\beta_\nu$. Für $j=1$ ergibt sich ein nichtlineares MSV (\person{Adams-Bashford-Moulton}-Verfahren). Für $j\to\infty$ kann unter bestimmten Voraussetzungen für hinreichend kleine $h>0$ die Konvergenz der Folge $\{\zeta^j\}$ gegen den eindeutigen Fixpunkt $y^{k+l}$ gezeigt werden.

Eine Klasse von impliziten linearen MSV (sogenannte \begriff{Backward Differentiation Formulas} bzw. \begriff{BDF-Verfahren}) erhält man aus der Idee $y'(x_{k+l}) = f(x_{k+l},y(x_{k+l}))$ durch $\sfrac{1}{h}\sum_{\nu=0}^l \alpha_\nu y(x_{k+\nu})$ (verallgemeinerte Sekantensteigung) zu approximieren. Man hat dann ein lineares MSV der Form
\begin{align}
	\sum_{\nu=0}^l \alpha_\nu y^{k+\nu} = hf(x_{k+l},y^{k+l}) \notag
\end{align}

\subsection{Konsistenz- und Konvergenzordnung für lineare MSV}
\section{A-Stabilität}

Wir betrachten die Test-AWA
\begin{align}
	\label{3_29}
	y'=\lambda y\quad\mit\quad y(0)=1
\end{align}
wobei $\lambda\in\comp$ ein Parameter ist. Die eindeutige Lösung dieser Aufgabe ist gegeben durch $y(x)=\exp(\lambda x)$ und es gilt insbesondere
\begin{align}
	\Re(\lambda)&< 0 \quad\Rightarrow\quad \abs{y(x)}\to 0 \quad\text{für } x\to\infty \notag \\
	\Re(\lambda)&=0 \quad\Rightarrow\quad\abs{y(x)}=1 \quad\text{für alle } x\in[0,\infty) \notag
\end{align}

\begin{definition}[A-Stabilität]
	Ein Verfahren erzeuge zu einem beliebigen Paar $(h,\lambda)\in(0,\infty)\times\comp$ eine Folge $\{y_k\}$. Dann heißt das Verfahren \begriff{A-stabil}, wenn
	\begin{align}
		\abs{y_{k+1}} \le \abs{y_k}\quad\forall k\in\natur\notag
	\end{align}
	für jedes $(h,\lambda)\in (0,\infty)\times\comp$ mit $\Re(\lambda)\le 0$.
\end{definition}

Bei ESV gilt $y_{k+1}=y_k + h\Phi(x_k,y_k,y_{k+1},h)$. Wir nehmen an, dass für $f(x,y)=\lambda y$ eine Darstellung des ESV in der Form
\begin{align}
	y_{k+1} = g(h\lambda)y_k \notag
\end{align}
mit einer Funktion $g$: $\comp\to\comp$ existiert. Die Funktion $g$ heißt dann auch \begriff{Stabilitätsfunktion}. Falls der \begriff{Stabilitätsbereich} (Bereich der absoluten Stabilität)
\begin{align}
	\mathcal{S} = \{z\in\comp\mid \abs{g(z)}\le 1\}\notag
\end{align}
die Halbebene $\comp_-=\{z\in\comp\mid \Re(z)\le 0\}$ enthält, dann ist das ESV A-stabil (und umgekehrt), denn es gilt $\abs{y_{k+1}} = \abs{g(h\lambda)}\abs{y_k}\le \abs{y_k}$ für $k\in\natur$ und beliebige $(h,\lambda)\in (0,\infty)\times\comp$ mit $h\lambda\in\comp_-$. Für die \begriff{Trapenzregel} (ein implizites ESV)
\begin{align}
	y_{k+1} = y_k + \frac{h}{2}\bigg(f(x_k,y_k) + f(x_{k+1},y_{k+1})\bigg) \notag
\end{align}
erhält aus der Test-AWA \cref{3_29} $y_{k+1} = y_k + \frac{h}{2}(\lambda y_k + \lambda y_{k+1})$ und somit
\begin{align}
	\left(1-\frac{h}{2}\lambda\right)y_{k+1} = y_k\left(1+\frac{h}{2}\lambda\right)\notag
\end{align}
das heißt die Stabilitätsfunktion der Trapezregel ist gegeben durch
\begin{align}
	g(z) = \frac{1+\frac{z}{2}}{1-\frac{z}{2}} = \frac{2+z}{2-z}\notag
\end{align}
Falls $\Re(z)\le 0$, so folgt
\begin{align}
	\abs{g(z)}^2 = \frac{(2+\Re(z))^2 + (\Im(z))^2}{(2-\Re(z))^2 + (\Im(z))^2} \le 1\notag
\end{align}
also die A-Stabilität der Trapezregel.

Für das \begriff[\person{Euler}-Verfahren!]{explizite \person{Euler}-Verfahren} ergibt sich (wegen \cref{3_29})
\begin{align}
	y_{k+1} = y_k + h\lambda y_k = (1+h\lambda)y_k\quad\text{und}\quad g(z) = 1+z\notag
\end{align}
Damit gilt
\begin{align}
	\abs{g(z)}^2 = \abs{1+z}^2 = (1+\Re(z))^2 + (\Im(z))^2 \le 1 \notag
\end{align}
genau dann, wenn $z=h\lambda$ im Einheitskreis um $(-1,0)\in\comp$ liegt. Da der Stabilitätsbereich beim expliziten \person{Euler}-Verfahren nicht alle $z\in\comp$ mit $\Re(z)\le 0$ enthält, ist dieses Verfahren nicht A-stabil. Das explizite \person{Euler}-Verfahren hat Konsistenzordnung 1 (vgl. \propref{3_2_3}) und ist mit dieser Ordnung auch konvergent (vgl. \propref{satz_3_8}). Die fehlende A-Stabilität hat zur Folge, dass zur erfolgreichen numerischen Lösung der Test-AWA \cref{3_29} für $\lambda <0$ zumindest $-2\le h\lambda$ gelten muss. Dies erfordert $h\sim \frac{1}{\abs{\lambda}}$, also gegebenenfalls sehr kleine Schrittweiten. Dies ist neben einem hohen Aufwand auch die Gefahr des Überwiegens von Rundungsfehlern verbunden, vgl. Abschnitt 2.4. Verfahren, die A-stabil sind, bzw. einen hinreichend großen Bereich absoluter Stabilität besitzen, haben außerdem Vorteile bei sogenannten steifen AWA, vgl. Abschnitt 5.

Bei RKV kann man den Stabilitätsbereich untersuchen, indem man sich die Stabilitätsfunktion beschafft. Zum Beispiel betrachten wir das 2-stufige explizite RKV
\begin{align}
	y_{k+1} &= y_k + hc_1k_1 + hc_2k_2 \notag \\
	&= y_k + hc_1f(x_k,y_k) + hc_2f(x_k+\alpha_2h,y_k + h\beta_{21}f(x_k,y_k))\notag
\end{align}
Mit der Test-AWA \cref{3_29} folgt
\begin{align}
	y_{k+1} &= y_k + h\lambda c_1y_k + h\lambda c_2(y_k + h\lambda\beta_{21} y_k) \notag \\
	&= y_k(1+h\lambda c_1 + h\lambda c_2 + (h\lambda)^2 c_2\beta_{21}) \notag
\end{align}
Beim Verfahren von \person{Heun} (vgl. Abschnitt 2.5) mit $c_1=c_2=\sfrac{1}{2}$ und $\beta_{21}=1$ ergibt sich
\begin{align}
	y_{k+1} = y_k\left(1+h\lambda + \frac{1}{2}(h\lambda)^2\right)\quad\text{und also}\quad g(z) = 1+z+\frac{1}{2}z^2 \notag
\end{align}
Man sieht schnell, dass dieses Verfahren nicht A-stabil ist (man wähle $z=(a,0)$ mit $a<-2$).

\begin{remark}
	Es gibt kein explizites lineares MSV und kein explizites RKV, dass A-stabil ist und die A-stabilen impliziten MSV haben höchstens Konsistenzordnung 2 (zweite \person{Dahlquist}-Barriere).
\end{remark}
\section{Einblick: Steife Probleme}

Für $A\in\R^{m\times m}$ werde die AWA
\begin{align}
	\label{3_30}
	y' = Ay\quad\mit\quad y(a)=y^0
\end{align}
für $x\in [a,b]$ betrachtet. Wir setzen in diesem Abschnitt voraus, dass $A$ eine diagonalisierbare Matrix ist, das heißt es gibt eine reguläre Matrix $S\in\comp^{m\times m}$ und eine Diagonalmatrix $D\in\comp^{m\times m}$ mit $A=SDS^{-1}$. Dann ist die allgemeine Lösung $y$: $[a,b]\to\R^m$ von $y'=Ay$ gegeben durch
\begin{align}
	y(x) = \sum_{i=1}^m c_i\exp(\lambda_i (x-a))v^i \notag
\end{align}
wobei $\lambda_1,...,\lambda_m\in\comp$ die Eigenwerte von $A$ und $v^1,...,v^m\in\comp^m$ ein zugehöriges System linear unabhängiger Eigenvektoren bezeichnet ($A$ diagonalisierbar!). Die Koeffizienten $c_1,...,c_m$ ergeben sich damit eindeutig aus der Anfangsbedingung $y(a)=c_1v^1+\dots+c_mv^m=y^0$.

Falls $\Re(\lambda_i)<0$ für $i=1,...,m$ wird die Zahl
\begin{align}
	\frac{\max_{1\le i\le m}\abs{\Re(\lambda_i)}}{\min_{1\le i\le m}\abs{\Re(\lambda_i)}}\notag
\end{align}
als \begriff{Steifigkeitsquotient} von $A$ bezeichnet. Ist dieser Quotient groß, dann dient dies als Indikator für ein Phänomen, das bei der Anwendung bestimmter numerischer Verfahren aus \cref{3_30} auftreten kann und als \begriff{Steifheit} (stiffness) der AWA \cref{3_30} bezeichnet wird. Ein solches Phänomen wird im folgenden Beispiel beschrieben und führt bei bestimmten Lösungsverfahren (hier explizites \person{Euler}-Verfahren) zum Erfordernis sehr kleiner Schrittweiten.

\begin{example}
	Für $a=0$ und
	\begin{align}
		A = \begin{pmatrix}
			-80.6 & 119.4 \\ 79.6 & -120.4
		\end{pmatrix} \notag
	\end{align}
	ergibt sich als allgemeine Lösung von $y'=Ay$
	\begin{align}
		y(x) = c_1\exp(-x)v^1 + c_2\exp(-200x)v^2 \quad\mit\quad v^1 = \begin{pmatrix}
			3 \\ 2
		\end{pmatrix} \text{ und } v^2 = \begin{pmatrix}
			-1 \\ 1
		\end{pmatrix} \notag
	\end{align}
	Für $y^0=(2,3)^T$ hat man als exakte Lösung von \cref{3_30} $y(x) = c_1\exp(-x)v^1 + c_2\exp(-200x)v^2$. Das explizite \person{Euler}-Verfahren liefert
	\begin{align}
		y^{k+1} = y^k + hAy^k = (\mathbbm{1} + hA)y^k \notag
	\end{align}
	Da $A$ diagonalisierbar ist, gilt $A=SDS^{-1}$ mit $S=(v^1,v^2)$ und $D=\diag(-1,-200)$ und
	\begin{align}
		S^{-1} = \frac{1}{5}\begin{pmatrix}
			1 & 1 \\ -2 & 3
		\end{pmatrix}\notag
	\end{align}
	Damit folgt
	\begin{align}
		S^{-1}y^{k+1} &= S^{-1}y^k + hS^{-1}ASS^{-1}y^k \notag \\
		&= S^{-1}y^k + hDS^{-1}y^k \notag \\
		&= (\mathbbm{1} + hD)S^{-1}y^k \notag
	\end{align}
	Setzt man $z^k = S^{-1}y^k$ ergibt sich weiter
	\begin{align}
		z^{k+1} = (\mathbbm{1} + hD)u^k \notag
	\end{align}
	für $k=0,...$. Wegen $z^0=S^{-1}y^0 = S^{-1}(v^1+v^2) = (1,0)^T + (0,1)^T = (1,1)^T$ erhält man
	\begin{align}
		z^k = (\mathbbm{1} + hD)^k\begin{pmatrix}
			1 \\ 1
		\end{pmatrix} \notag
	\end{align}
	Für $k\to\infty$ folgt $x_k\to\infty$ und $y(x_k)\to 0$. Um die Konvergenz der Folge $\{z^k\}$ und damit der Folge $\{y^k\}$ gegen 0 zu sichern, müssen
	\begin{align}
		\abs{1+\lambda_1 h} = \abs{1-h} < 1\quad\text{und}\quad \abs{1+\lambda_2 h} = \abs{1-200h} < 1 \notag
	\end{align}
	erfüllt sein. Dies impliziert $h < \sfrac{1}{100}$. Der für die exakte Lösung eigentlich unwesentliche (das heißt sehr schnell abklingende) Anteil $\exp(-200x)v^2$ verursacht beim expliziten \person{Euler}-Verfahren sehr kleine Schrittweiten.
\end{example}

Ähnliche Phänomene können bei der Anwendung anderer Verfahren, die nicht A-stabil sind, bzw. deren Bereich der absoluten Stabilität ungeeignet ist, auftreten. Auch bei allgemeineren AWA als \cref{3_30} treten Phänomene der Steifheit auf und erfordern angepasste Verfahren.
\section{Ausblick}

Die Theorie zur numerischen Behandlung von AWA ist natürlich wesentlich umfangreicher als hier dargestellt werden konnte. Das bedeutet einerseits, dass von den behandelten Themen nur wichtige Grundlagen des vorhandenen Wissens präsentiert wurden. Beispielsweise gibt es eine Reihe von speziellen Verfahren, die nicht oder nur beispielhaft beschrieben und analysiert wurden. Andererseits konnten verschiedene weitere Themen in der Vorlesung gar nicht angesprochen werden. Dazu zählen insbesondere Fragen der Fehlerabschätzung und Schrittweitensteuerung sowie angepasste Stabilitätsbegriffe. Zur Vertiefung stehen neben der Literatur auch weitere Lehrveranstaltungen zur Verfügung.

\part*{Anhang}
\addcontentsline{toc}{part}{Anhang}
\appendix

\nocite{*}
\bibliography{literatur}
\bibliographystyle{acm}

%\printglossary[type=\acronymtype]

\printindex

\end{document}
