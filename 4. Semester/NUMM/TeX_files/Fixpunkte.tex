\section{Fixpunkte}

Seien ein Vektorraum $V$, eine Menge $U \subseteq V$ und eine Abbildung $\Phi: U \to V$ gegeben.
Dann heißt $x^{*} \in U$ \begriff{Fixpunkt} der Abbildung $\Phi$, falls $\Phi(x^{*}) = x^{*}$ gilt.
Die Aufgabe
\begin{align}
\Phi(x) = x\notag
\end{align}
eigentlich die Aufgabe, diese Gleichung zu lösen) wird als \begriff{Fixpunktaufgabe} bezeichnet.
Die Abbildung $\Phi$ heißt \begriff{Fixpunktabbildung}. Im Unterschied zur Fixpunktaufgabe heißt
\begin{align}
F(x) = 0 \notag
\end{align}
\begriff{Nullstellenaufgabe}. 
Zu jeder Nullstellenaufgabe gibt es eine äquivalente Fixpunktaufgabe (z.B. $F(x) = 0 \Leftrightarrow \Phi(x) = x $ mit $\Phi(x) := F(x) + x$) und umgekehrt (z.B.
$\Phi(x) = x \Leftrightarrow F(x) = 0$ mit $F(x) := \Phi(x) -x$).