\section{Basis und Dimension}

\begin{definition}[Basis]
	Eine Familie $(x_i)$ von Elementen von $V$ ist eine \begriff{Basis} von $V$, wenn gilt:
	\begin{itemize}
		\item (B1): Die Familie ist linear unabhängig.
		\item (B2): Die Familie erzeugt $V$, also $\Span_K(x_i) = V$.
	\end{itemize}
\end{definition}

\begin{remark}
	Kurz gesagt ist eine Basis ein linear unabhängiges Erzeugendensystem.
\end{remark}

\begin{proposition}
	Sei $(x_i)$ eine Familie von Elementen von $V$. Genau dann ist $(x_i)$ eine Basis von $V$, 
	wenn sich jedes $x \in V$ auf eindeutige Weise als Linearkombination der $(x_i)$ schreiben lässt.
\end{proposition}
\begin{proof}
	Dies folgt sofort aus \propref{2_2_10}
\end{proof}

\begin{example}
	\begin{itemize}
		\item Die leere Familie ist eine Basis des Nullraums.
		\item Die \begriff{Standardbasis} $(e_1,...,e_n)$ ist eine Basis des Standardraums.
		\item Die Monome $(X^i)$ bilden eine Basis des $K$-Vektorraum $K[X]$.
		\item Die Basis des $\mathbb R$-Vektorraum $\mathbb C$ ist gegeben durch $(1,i)$, eine Basis des $\mathbb C$-
		Vektorraum $\mathbb C$ ist gegeben durch $(1)$
		\item Der $\mathbb C$-Vektorraum $\mathbb C$ hat viele weitere Basen.
	\end{itemize}
\end{example}

\begin{proposition}
	\proplbl{2_3_5}
	Für eine Familie $(x_i)$ von Elementen von $V$ sind äquivalent:
	\begin{itemize}
		\item $B$ ist eine Basis von $V$.
		\item $B$ ist ein minimales Erzeugendensystem.
		\item $B$ ist maximal linear unabhängig, d.h. $B$ ist linear unabhängig, aber wenn Elemente zur Basis 
		hinzugefügt werden, ist diese nicht mehr linear unabhängig.
	\end{itemize}
\end{proposition}
\begin{proof}
	\begin{itemize}
		\item $1 \Rightarrow 2$: Sei $B$ eine Basis von $V$ und $J$ eine echte Teilmenge von $I$. Nach Definition ist $B$ ein 
		Erzeugendensystem. Wähle $i_0 \in I\backslash J$. Da $(x_i)$ linear unabhängig ist, ist $x_{i_0}$ keine Element 
		$\Span_K((x_i)_{i \in I\backslash \{i_0\}}) \supseteq \Span_K((x_i)_{i \in J})$ (\propref{2_2_9}). Insbesondere ist $(x_i)_{i\in J}$ kein 
		Erzeugendensystem von $V$. 
		\item $2 \Rightarrow 3$: Sei $B$ ein minimales Erzeugendensystem und $(x_i)_{i \in J}$ eine Familie mit $J$ echter 
		Obermenge von $I$. Wäre $(x_i)$ linear abhängig, so gäbe es ein $i_0$ mit $\Span_K((x_i)_{i \in I\backslash 
			\{i_0\}}) = \Span_K((x_i)_{i \in I})=V$ im Widerspruch zur Minimalität von $B$. Also ist $B=(x_i)$ linear 
		unabhängig. Wähle $j_0 \in J\backslash I$. Dann ist $x_{j_0} \in V=\Span_K(x_i) \le \Span_K((x_i)_{i \in 
			J\backslash \{j_0\}})$ und somit ist $(x_i)_{i\in J}$ linear abhängig nach \propref{2_2_9}.
		\item $3 \Rightarrow 1$: Sei $B$ nun maximal linear unabhängig. Angenommen $B$ wäre kein Erzeugendensystem. 
		Dann gibt es ein $x\in V \backslash \Span_K(x_i)$. Definiere $J=I \cup \{j_0\}$ mit $j_0 \notin I$ und $x_{j_0}:=x$. 
		Aufgrund der Maximalität von $B$ ist $(x_i)$ linear abhängig, es gibt als Skalare $\lambda$, $(\lambda_i)$, nicht 
		alle gleich 0, mit $\lambda\cdot x+\sum_{i \in I} \lambda_i\cdot x_i=0$. Da $(x_i)$ linear abhängig ist, 
		muss $\lambda \neq 0$ sein, woraus der Widerspruch $x=\lambda^{-1}\cdot\sum_{i \in I} \lambda_i\cdot x_i 
		\in \Span_K(x_i)$. Somit ist $B$ ein Erzeugendensystem.
	\end{itemize}
\end{proof}

\begin{theorem}[Basisauswahlsatz]
	\proplbl{2_3_6}
	Jedes endliche Erzeugendensystem von $V$ besitzt eine Basis  als 
	Teilfamilie: Ist $(x_i)$ ein endliches Erzeugendensystem von $V$, so gibt es eine Teilmenge $J\subseteq I$, 
	für die $(x_i)_{i\in J}$ eine Basis von $V$ ist. 
\end{theorem}
\begin{proof}
	Sei $(x_i)$ ein endliches Erzeugendensystem von $V$. Definiere $\mathcal J:=\{J \subseteq I \mid (x_i)_{i\in J}\; 
	\text{J ist Erzeugendensystem von }V\}$. Da $I$ endlich ist, ist auch $\mathcal J$ endlich. Da $(x_i)$ 
	Erzeugendensystem ist, ist $I\in J$, insbesondere $\mathcal J\neq\emptyset$. Es gibt deshalb ein bezüglich 
	Inklusion minimales $J_0\in \mathcal J$, d.h. $J_1 \in \mathcal J$ so gilt nicht $J_1 \subsetneq J_0$. Deshalb 
	ist $(x_i)_{i\in J_0}$ eine Basis von $V$ (\propref{2_3_5}).
\end{proof}

\begin{conclusion}
	\proplbl{2_3_7}
	Jeder endlich erzeugte $K$-Vektorraum besitzt eine endliche Basis.
\end{conclusion}

\begin{remark}
	Der Beweis von \propref{2_3_6} liefert ein konstruktives Verfahren: Ist $(x_1,...,x_n)$ ein endliches 
	Erzeugendensystem von $V$, so prüfe man, ob es ein $i_0$ mit $x_{i_0} \in \Span_K((x_i)_{i\neq i_0})$ gibt. 
	Falls Nein, ist $(x_1,...,x_n)$ eine Basis von $V$. Falls Ja, macht man mit $(x_1,...,x_{i_{0-1}}, x_{i_{0+1}},
	...,x_n)$ weiter.
\end{remark}

\begin{remark}
	Man kann jedoch zeigen, dass jeder Vektorraum eine Basis besitzt. Die Gültigkeit der Aussage hängt jedoch 
	von bestimmten mengentheoretischen Axiomen ab, auf die wir an dieser Stelle nicht eingehen werden. Siehe dazu 
	LAAG 2. Semester.
\end{remark}

\begin{lemma}[Austauschlemma]
	\proplbl{2_3_9}
	Sei $B=(x_1,...,x_n)$  eine Basis von $V$. Sind $\lambda_1,...,\lambda_n \in K$ und 
	$y=\sum_{i=1}^n \lambda_i\cdot x_i$, so ist für jedes $j\in \{1,2,...,n\}$ mit $\lambda_j\neq 0$ auch 
	$B'=(x_1,...,x_{j-1},y,x_{j+1},...,x_n)$ eine Basis von $V$.
\end{lemma}
\begin{proof}
	oBdA. sei $j=1$, also $B'=(y,x_2,...,x_n)$. Wegen $\lambda_1\neq 0$ ist $x_1=\lambda_1^{-1}\cdot y - \sum
	_{i=2}^n \lambda_i\cdot x_i \in \Span_K(y,x_2,...,x_n)$ und somit ist $B'$ ein Erzeugendensystem. Sind 
	$\mu_1,...,\mu_n \in K$ mit $\mu_1\cdot y - \sum_{i=2}^n \mu_i\cdot x_i=0$, so folgt $0=\mu_1(\sum
	_{i=1}^n \lambda_i\cdot x_i + \sum_{i=2}^n \mu_i\cdot x_i)=\mu_1\cdot \lambda_1\cdot x_1 + \sum
	_{i=2}^n (\mu_1\cdot \lambda_i + \mu_i)x_i$ und aus der linearen Unabhängigkeit von $B$ somit $\mu_1\cdot 
	\lambda_1=0$, $\mu_1\cdot \lambda_2 + \mu_2 =0$, ..., $\mu_1\cdot\lambda_n + \mu_n=0$. Wegen $\lambda_1\neq 0$ folgt 
	$\mu_1=0$ und daraus $\mu_i=0$. Folglich ist $B'$ linear unabhängig.
\end{proof}

\begin{theorem}[\person{Steinitz}'scher Austauschsatz]
	\proplbl{2_3_10}
	Sei $B=(x_1,...,x_n)$ eine Basis von $V$ und $\mathcal F=(y_1,...
	,y_r)$ eine linear unabhängige Familie in $V$. Dann ist $r\le n$ und es gibt $i_1,...,i_{n-r} \in \{1,...,n\}$, für 
	die $B'=(y_1,...,y_r,x_{i_1},...,x_{i_{n-r}})$ eine Basis von $V$ ist. 
\end{theorem}
\begin{proof}
	Induktion nach $r$\\
	Für $r=0$ ist nichts zu zeigen. \\
	Sei nun $r\ge 1$ und gelte die Aussage für $(y_1,...,y_{r-1})$. Insbesondere ist $r-1\le n$ und es gibt $i_1,..,
	i_{n-(r-1)} \in \{1,...,n\}$ für die $B'=(y_1,...,y_r,x_{i_1},...,x_{i_{n-(r-1)}})$ eine Basis von $V$ ist. Da $y_r
	\in V=\Span_K(B')$ ist $y_r=\sum_{i=1}^{r-1} \lambda_i\cdot y_1 + \sum_{j=0}^{n-(r-1)} \mu_j\cdot 
	x_{i_j}$. Da $(y_1,...,y_r)$ linear unabhängig, ist $y_r \notin \Span_K(y_1,...,y_{r-1})$. Folglich gibt es $j_0 \in 
	\{1,...,n-(r-1)\}$ mit $\mu_{j_0}\neq 0$. Insbesondere ist $n-(r-1)\ge 1$, also $r\le n$. oBdA. $j_0=1$, dann 
	ergibt sich mit dem Austauschlemma (\propref{2_3_9}), dass auch $(y_1,...,y_{r-1},y_r,x_{i_2},...,x_{i_{n-(r-1)}})$ eine Basis von 
	$V$ ist.
\end{proof}

\begin{conclusion}[Basisergänzungssatz]
	\proplbl{2_3_12}
	Ist $V$ endlich erzeugt, so lässt sich jede linear unabhängige Familie zu einer Basis ergänzen: 
	Ist $(x_1,...,x_n)$ linear unabhängig, so gibt es $m\ge n$ und $x_{n+1},x_{n+2},...,x_m$ für die $(x_1,...,x_n,
	x_{n+1},...,x_m)$ eine Basis von $V$ ist.
\end{conclusion}
\begin{proof}
	Nach dem Basisauswahlsatz (\propref{2_3_6} und \propref{2_3_7}) besitzt $V$ eine endliche Basis, die Behauptung folgt somit aus dem \person{Steinitz}'schen Austauschsatz (\propref{2_3_10}).
\end{proof}

\begin{conclusion}
	\proplbl{2_3_12}
	Sind $(x_i)$ und $(x_j)$ Basen von $V$ und ist $I$ endlich, so ist $|I|=|J|$.
\end{conclusion}
\begin{proof}
	Da $(y_r)$ linear unabhängig ist, ist $|J|\le |I|$ nach dem \person{Steinitz}'schen Austauschsatz (\propref{2_3_10}). Insbesondere ist $J$ 
	endlich, also $|I|\le |J|$ nach dem Austauschsatz (\propref{2_3_10}).
\end{proof}

\begin{conclusion}
	Ist $V$ endlich erzeugt, so haben alle Basen von $V$ die gleiche Mächtigkeit.
\end{conclusion}
\begin{proof}
	$V$ besitzt eine endliche Basis (\propref{2_3_7}), deshalb folgt die Behauptung aus \propref{2_3_12}.
\end{proof}

\begin{definition}[Dimension]
	Ist $V$ endlich erzeugt, so ist die \begriff{Dimension} des Vektorraum $V$ die Mächtigkeit $\dim_K(V)$ 
	einer Basis von $V$. Andernfalls sagt man, dass $V$ unendliche Dimensionen hat und schreibt $\dim_K(V)= \infty$. 
\end{definition}

\begin{example}
	\begin{itemize}
		\item $\dim_K(K^n)=n$
		\item $\dim_K(K[X])=\infty$
		\item $\dim_K(K[X]_{\le n})=n+1$
		\item $\dim_{\mathbb R}(\mathbb C)=2$
		\item $\dim_{\mathbb C}(\mathbb C)=1$
	\end{itemize}
\end{example}

\begin{remark}
	\begin{itemize}
		\item $V$ ist genau dann endlich erzeugt, wenn $\dim_K(V)< \infty$.
		\item Mit \propref{2_3_5} $\dim_K(V)=\min\{|B| \mid \Span_K(B)=V\}=\max\{|B| \mid B\text{ linear unabhängig}\}$
	\end{itemize}
\end{remark}

\begin{proposition}
	Sei $V$ endlich erzeugt und $W\le V$ ein Untervektorraum.
	\begin{itemize}
		\item Es ist $\dim_K(W)\le \dim_K(V)$. Insbesondere ist $W$ endlich erzeugt.
		\item Ist $\dim_K(W)=\dim_K(V)$, so ist auch $W=V$.
	\end{itemize}
\end{proposition}
\begin{proof}
	\begin{itemize}
		\item Ist $F$ eine linear unabhängige Familie in $W$, so ist auch $F$ linear unabhängig in $V$ und somit $|F|\le 
		\dim_K(V)$. Insbesondere gibt es eine maximal linear unabhängige Familie $B$ in $W$ und es folgt $\dim_K(W)=|B|
		\le \dim_K(V)$.
		\item Sei $B$ eine Basis von $W$. Dann ist $B$ auch in $V$ linear unabhängig. Ist $\dim_K(W)=\dim_K(V)$, so muss 
		auch $B$ in $V$ maximal linear unabhängig sein. Insbesondere ist $W=\Span_K(B)=V$.
	\end{itemize}
\end{proof}