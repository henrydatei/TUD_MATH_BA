\chapter{Metrische und normierte Räume}
\section{Metrische Räume}
\begin{mydefn}[Metrik]
    Sei $X$ Menge und Abbildung $d: X \times X \to \real$ heißt \underline{Metrik} auf $X$ falls $\forall x,y,z \in X$
    \begin{enumerate}[label={\alph*)}]
    \item $d(x,y) = 0 \Leftrightarrow x=y$ 
    \item $d(x,y) = d(y,x)$ (Symmetrie)
    \item $d(x,z) \leq d(x,y) + d(y,z)$ ($\Delta$-Ungleichung)
    \end{enumerate}
    $(X,d)$ heißt metrischer Raum.
\end{mydefn}
Man hat $d(x,y) = 0 \forall x,y \in X$, dann

\begin{align}
    0 &= d(x,x) = d(x,y) + d(y,x) & \text{a), c)}\nonumber\\
    & = 2d(x,y) \forall x,y & \text{b)}\nonumber\\
    \text{nach } & \text{b), c) } &\nonumber\\
    & \vert d(x,y) -d(z,y)\vert \leq d(x,y) \forall x,y,z \in X &
\end{align}

\begin{exmpn}[Standardmetrik]\label{8_1_exmp_metrik}
	$d(x,y) := \vert x-y\vert$ ist Metrik auf $X=\real$ bzw. $X=\comp$
    \begin{align*}
        \text{Eig. a), b), c) klar}& &\\
        \text{c) } \vert x-z\vert& \vert (x+y)-(x-z)\vert &\\
        &\leq \vert x+y\vert + \vert y+z\vert & \Delta\text{-Ungleichung für }\real\text{, }\comp\text{-Betrag}
    \end{align*} 
\end{exmpn}

\begin{exmpn}[diskrete Metrik]
	Diskrete Metrik auf beliebiger Menge $X$.\\
    \[d(x,y) = 
    \begin{dcases*}
        0 & x = y\\
        1 & $x \neq y$
    \end{dcases*}\]
    ist offenbar eine Metrik.
\end{exmpn}

\begin{exmpn}[induzierte Metrik]
	Sei $(X,d)$ metrischer Raum, $Y \subset X$\\
    $\Rightarrow (Y,d)$ ist metrischer Raum mit \emph{induzierter Metrik} $\tilde{d}(x,y):=d(x,y)\forall x,y \in Y$
\end{exmpn}

\section{Normierte Räume}

wichtiger Spezialfall: normierte Vektorraum(VR)

\begin{mydefn}[Norm]
    Sei $X$ Vektorraum über $K=\real$ oder $K=\comp$.\\
    Abbildung $\Vert \cdot \Vert: X \to \real$ heißt \emph{Norm} auf $X$ falls $\forall x,y \in X, \forall \lambda \in \real$ gilt:
    \begin{enumerate}[label={\alph*)}]
    \item $\Vert x\Vert = \Leftrightarrow x=0$
    \item $\Vert \lambda x\Vert = \vert \lambda \vert \Vert x\Vert$ (Homogenität)
    \item $\Vert x+y\Vert \leq \Vert x\Vert + \Vert y\Vert$ ($\Delta$-Ungleichung)
    \end{enumerate}
    $(X,\Vert \cdot\Vert)$ heißt \emph{normierter Raum}.
\end{mydefn}

\begin{align*}
    \text{Metrik} &\leftarrow \text{Norm}&\\
    \text{Abbildung} & \not \rightarrow \text{VR, Abstand } x,0\\
    \text{man hat } \Vert x \Vert &\leq 0 \forall x \in X \text{, denn } 0 = \Vert x-x\Vert \leq \Vert x\Vert + \Vert -x\Vert = 2\Vert x\Vert & \text{a), c), b)}\\   
\end{align*}
Analog Satz 5.5 folgt\\
\begin{align}
    \vert \Vert x \Vert - \Vert y \Vert\vert &\leq \Vert x-y\Vert \forall x,y \in X
\end{align}
$\Vert \cdot\Vert: X \to \real_{\geq0}$ heißt \emph{Halbraum} falls nur b), c) gelten analog Beispiel \ref{8_1_exmp_metrik} folgt.

\begin{satz}
    Sei $(X,\Vert\cdot \Vert)$ normierter Raum, dann $X$ metrischer Raum mit Metrik $d(x,y):=\Vert x-y \Vert\forall x,y \in X$.
\end{satz}

\begin{exmpn}\label{8_5_exmp_Norm}
    $X=\real^n$ ist Vektorraum über $\real$, Elemente in $\real^n$\\ $x=(x_1,\dots,x_n), y=(y_1, \dots, y_n)$,\\ man hat unter anderem folgende Normen auf $\real^n$
    \begin{align*}
        p\text{-Norm}: \vert x \vert_p& := \Bigg( \sum_{i=0}^{n} \vert x_i \vert^p \Bigg)^{\frac{1}{p}} & (1\leq p < \infty)\\
        \text{Maximum-Norm}: \vert x \vert_p& := \max\{\vert x_i \vert \mid i=1,\dots n\} &\\
        \text{a), b) jeweils klar, c) für } & 
        \begin{cases*}
            \vert \cdot \vert_p & \text{ist Minkowski-Ungleichung}\\
            \vert \cdot \vert_{\infty} & \text{wegen } $\vert x_i + y_i \vert \leq \vert x_i \vert + \vert y_i \vert \forall i$
        \end{cases*}
    \end{align*}
    Standardnorm in $\real^n$:
	$\vert \cdot \vert = \vert \cdot \vert_{p=2}$ heißt \emph{eukldische Norm}.\\
\end{exmpn}

\begin{mydefn}[Skalarprodukt]
    $\langle x,y \rangle = \sum_{i=1}^{n}$ heißt \emph{Skalarprodukt} (inneres Produkt) von $x,y \in \real^n$ offenbar $\langle x,y \rangle = \vert x \vert_2 \forall x \in comp$ nur für euklidische Räume gibt es Skalarprodukt (nur für euklische Norm!).\\
    Man hat $\vert \langle x,y\rangle \vert \leq \vert x \vert_2 \cdot \vert y \vert_2 \forall x,y \in \real^n$ Cauchy-Schwarsche Ungleichung (CSU), denn
    \begin{align*}
        \vert \langle x,z \rangle \vert &= \vert \sum_{i=1}^{n} x_i y_i \vert \leq  \sum_{i=1}^{n}\vert x_i y_i\vert & \Delta\text{-Ungleichung in } \real\\
        & \leq \vert x \vert_2 \cdot\vert y \vert_2 & \text{Hölder-Ungleichung mit } p=q=2
    \end{align*}
\end{mydefn}

\begin{exmpn}
	$X=\comp^n$ ist Vektorraum über $\comp$, $x=(x_1,\dots,x_n) \in\comp^n, x_i \in \comp$\\
    analog zum Bsp. \ref{8_5_exmp_Norm} sind $\vert \cdot \vert_{p} \text{ und } \vert \cdot \vert_{\infty}$ Normen auf $\comp^n$\\
    $\langle x,y\rangle = \sum_{i=1}^{n} \bar{x}_i y_i\forall x_i, y_i \in \comp$ heißt \emph{Skalarprodukt} von $x,y \in \comp^n$ (beachte $\langle x,y\rangle \in \comp, \langle x,x \rangle=\vert x \vert^2$) \\
    $\overset{\text{wie oben}}{\Rightarrow} \vert \langle x,y\rangle \vert \leq \vert x \vert\cdot \vert y \vert \forall x,y \in \comp^n$
\end{exmpn}

\begin{mydefn}[Orthogonalität]
    $x,y \in \real^n(\comp^n)$ heißen \emph{orthogonal} falls $\langle x,y\rangle =0$
\end{mydefn}

\begin{exmpn}
    Sei $M$ beliebige Menge, $f: M \to \real$\\
    $\Vert f\Vert:= \sup\{\vert f(x) \vert \mid x\in M\}$. Dann ist \\
    \[\mathcal{B}(M):= \{f: M \to \real \mid \Vert f\Vert < \infty\}\]
    \emph{Menge der beschränkte Funktionen} auf $M$\\
    $\mathcal{B}(M)$ ist Vektorraum auf $\real$
    \begin{enumerate}[label={\alph*)}]
        \item $((f+g)(x) = f(x) + g(x)$
        \item $(\lambda f)(x) = \lambda f(x)$
        \item Nullelement ist Nullfunktion $f(x)=0 \forall x \in M$
    \end{enumerate}
    \begin{align*}
        \shortintertext{$\Vert \cdot\Vert$ ist Norm auf $\mathcal{B}(M)$, denn a), b) klar}\\
        \Vert f+g\Vert&:=\sup\{\vert f(x)+g(x) \vert\mid x \in M\}&\\
        &\leq \sup\{\vert f(x) \vert + \vert g(x) \vert\mid x\in M \} & \Delta\text{-Ungleichung in }\real\\
        &\leq \sup\{\vert f(x) \vert\mid x \in M \} + \sup\{\vert g(x) \vert\mid x \in M \} & \text{Übungsaufgabe}\\
        &=\Vert f\Vert + \Vert g\Vert
    \end{align*}
\end{exmpn}

\begin{exmpn}
	$\Vert x \Vert:=\vert x_1 \vert$ auf $X=\real^n \to$ kein Nullvektor ``nur'' Halbnorm (später wichtige Halbnorm in Integraltheorie). Normen $\Vert \cdot \Vert_1,\;\Vert \cdot \Vert_2$ auf $X$ heißen äquivalent falls
    \[
    \exists \alpha,\beta > 0\;\alpha \vert x \vert_1 \leq \vert x \vert_2 \leq \beta\vert x \vert_1\qquad\forall x \in X
    \]
    (Indizes entsprechen hier keinem p, sondern es sind hier nur beliebige unterschiedliche Normen gemeint.)
\end{exmpn}

\begin{exmpn}
	\[
    \vert x \vert_{\infty} \leq \vert x \vert_p \leq \sqrt[p]{n}\vert x \vert_{\infty}\qquad \forall x \in \real^n,\;p\geq 1\\
    \]
    $\vert \cdot \vert_\infty$ und $\vert \cdot \vert_\infty$ sind äquivalent $\forall p \geq 1$
\end{exmpn}

\begin{proof}
    \begin{align*}
        \vert x \vert_{\infty} &=\big(\max \{ \vert x_j \vert, \vert \dots \}^p\big)^\frac{1}{p} \leq \bigg(\sum_{j=1}^{n} \vert x_j \vert^p \bigg)^\frac{1}{p} = \vert x \vert_p\\
        \vert x \vert_{\infty} &\leq \big( n\cdot \max\{ \vert x_j \vert, \vert \dots \}^p\big)^\frac{1}{p} \leq \bigg(\sum_{j=1}^{n} \vert x_j \vert^p \bigg)^\frac{1}{p} = \sqrt[p]{n}\vert x \vert_{\infty}
    \end{align*}\QEDA
\end{proof}

\begin{folg}
    $\vert \cdot \vert_p,\;\vert \cdot \vert_q$ äquivalent auf $\real^n\;\forall p,q \geq 1$ (siehe Aufgabe 45b))
\end{folg}

\section{Begriffe im metrischen Raum}

\begin{mydefn}[Kugel im metrischen Raum]
    Sei $(X,d)$ metrischer Raum.
    \begin{itemize}
    \item $B_r(a):= \{ a \in X \mid d(a,x) <r \}$ heißt offene \underline{Kugel} um $a$ mit Radius $r>0$
    \item $B_r[a]:= \overline{B}_r(a) = \{ a \in X \mid d(a,x) \leq r \}$ heißt abgeschlossene \emph{Kugel} um $a$ mit Radius $r>0$
    \end{itemize}
\end{mydefn}
Hinweis: muss keine übliche Kugel sein z.B. $\{x\in \real^n \mid d(0,x) < 1\}$ ist Quadrat $B_r(0)$.

\begin{mydefn}
    \begin{itemize}
    \item Menge $M\subset X$ \emph{offen} falls $\forall x \in M\;\exists \epsilon > 0\; B_{\epsilon}(x) \subset M$
    \item Menge $M$ offen falls $X\setminus M$ abgeschlossen
    \item $U \subset X$ Umgebung von $M \subset X$ falls $\exists V \subset X$ offen mit $M \subset V \subset U$
    \item $x \in M$ \emph{innerer Punkt} von $M$ falls $\exists \epsilon >0\colon B_{\epsilon}(x) \subset M$
    \item $x \in M$ \emph{äußerer Punkt} von $M$ falls $\exists \epsilon >0\colon B_{\epsilon}(x) \subset X\setminus M$
    \item $x \in X$ \emph{Randpunkt} von $M$ falls $x$ weder innerer noch äußerer Punkt ist
    \item $\inter M:=$ Menge der \emph{inneren} Punkte von $M$ heißen inneres von $M$
    \item $\ext M:=$ Menge der \emph{äußeren} Punkte von $M$ heißen äußeres von $M$
    \item $\partial M:=$ Menge der Randpunkte von $M$ heißt \emph{Rand} von $M$
    \item $\cl M:= \overline{M}:=\overline{\inter M} \cup \partial M$ heißt Abschluss von $M$ (closure)
    \item $M \subset X$ \emph{beschränkt} falls $\exists a \in X, r >0\; M \subset B_r(a)$
    \item $x \in X$ \emph{Häufungskt (Hp)} von $M$ falls $\forall \epsilon > 0$ enhält \emph{$B_{\epsilon}(x)$ unendlich viele} Elemente aus $M$
    \item $x \in M$ \emph{isolierter} Punkt von $M$ falls $x$ kein Hp von $M$
    \end{itemize}
\end{mydefn}

\begin{exmpn}
    \begin{enumerate}[label={\alph*)}]
        \item Sei $X=\real$ mit $d(x,y)=\vert x-y \vert$
            \begin{itemize}
                \item $(a,b),(-\infty,a)$ offen
                \item $[a,b], (-\infty, b]$ abgeschlossen
                \item $[a,b)$ weder abgeschlossen noch offen, aber beschränkt
            \end{itemize}
        Es gilt:
            \begin{itemize}
                \item $\inter(a,b) = \inter[a,b] = (a,b)$
                \item $\ext(a,b) = \ext[a,b] = (-\infty,a) \cup (b, \infty)$
                \item $\partial(a,b) = \partial[a,b] = \{a,b\}$
                \item $\cl(a,b) = \cl[a,b]=[a,b]$
            \end{itemize}
        Speziell:
            \begin{itemize}
            \item $\ratio$ weder offen noch abgeschlossen in $\real$, da $\inter \ratio =\emptyset, \ext \ratio = \emptyset, \partial \ratio = \real$
            \item $\real \setminus \emptyset$ ist offen
            \item $\natur \text{ in } \real$ abgeschlossen und nicht beschränkt
            \item $[0,3]$ ist Umgebung von $[1,2], B_r(a)$ ist Umgebung von $a$ (eigentlich $\{a\}$)
            \item $a$ ist Hp von $(a,b),[a,b]$ für $a<b$, aber nicht von $[a,a]$ aller $a\in \real$ sind Hp von $\ratio$
            \end{itemize}
        \item für $X=\real$ mit diskreter Metrik: $x\in M \Rightarrow B_{\frac{1}{2}}(x) \{x\} \Rightarrow$ alle $M \subset \real$ offen und abgeschlossen
        \item für $X=\real^n$ mit $\metric$ vgl. Übungsaufgabe
    \end{enumerate}
\end{exmpn}

\begin{lem}
    Sei $(X,d)$ metrischer Raum. Dann
    \begin{enumerate}[label={\arabic*)}]
    \item $B_r(a)$ offene Menge $\forall \epsilon >0,a\in X$
    \item $M\subset X$ beschränkt $\Rightarrow \forall a \in X \exists r>0\colon M\subset B_r(a)$
    \end{enumerate}
\end{lem}

\begin{proof}
    \begin{enumerate}[label={\arabic*)}]
    \item Sei $b \in B_r(a),\epsilon := r - a-d(a,b)>0$, dann gilt für beliebige $x \in B_{\epsilon}(b)$
    \begin{align*}
        d(a,x) &\leq d(a,b) + d(b,x) & \Delta\text{-Ungleichung mit } b\\
        &<d(a,b)+r-d(a,b)&\\
        &=r \Rightarrow B_{\epsilon}(b) \subset B_{\epsilon}(a) \beha &
    \end{align*}
    \item Sei $M\subset B_{\rho}(b),a\in X$ beliebig, $r:=\rho + d(a,b),m\in M$\\
    \begin{align*}
        \Rightarrow d(m,a) &\leq d(m,b)+d(b,a)&\\
        &<\rho + d(b,a) = r \Rightarrow m\in B_{r}(a)
    \end{align*}
    \end{enumerate}\QEDA
\end{proof}

\begin{satz}[Offene Kugeln sind Topologie auf X]\label{8_13_satz_open_topo}
    Sei $(X,d)$ metrische Raum, $\tau:=\{ U \subset X \mid \text{ offen} \}$. Dann
    \begin{enumerate}[label={\arabic*)}]
    \item $X,\emptyset\in\tau$
    \item $\bigcap_{i=1}^{n} U_i \in \tau$ falls $U_i\in \tau \text{ für } i = 1, \dots, n$ (endlich viele) 
    \item $\bigcup_{U\in\tau^{\prime}} U \in \tau$ falls $\tau^{\prime} \subset \tau$ (beliebig viele)
    \end{enumerate}
\end{satz}

\begin{proof}
    \begin{enumerate}[label={\arabic*)}]
    \item $X$ offen, da stets $B_{\epsilon}(x) \subset X$, Definition ``offen'' wahr für $\emptyset$
    \item Sei $X \in \bigcap_{i=0}^{n} U_i \Rightarrow \exists \epsilon_i > 0 \colon B_{\epsilon_i}(x) \subset U_i \forall i, \epsilon = \min\{\epsilon_1, \dots \epsilon_n\}$\\
    $\Rightarrow B_{\epsilon}(x) \in \bigcap_{i=0}^{n} U_i \beha$
    \item Sei $x \in \bigcup_{U\in\tau^{\prime}} U \Rightarrow \exists \tilde{U}\in \tau^{\prime}\colon x \in \tilde{U} \overset{\tilde{U} \text{ offen}}{\Rightarrow}\exists \epsilon > 0 \colon B_{\epsilon}(x) \subset \tilde{U} \in \bigcup_{U\in\tau^{\prime}} U \beha$.
    \end{enumerate}\QEDA
\end{proof}

Hinweis: Durchschnitt beliebiger vieler offener Menge ist nicht offen!

\begin{exmp}
    $\bigcap_{n\in \natur} (-\frac{1}{n},1+\frac{1}{n}) = [0,1]$
\end{exmp}

Komplementbildung im Satz \ref{8_13_satz_open_topo} liefert:

\begin{folg}[Abgeschlossene Kugeln sind Topologie auf X]
    Sei $(X,d)$ metrischer Raum und $\sigma :=\{ V \subset   X \mid V \text{ abgeschlossen}\}$. Dann
    \begin{enumerate}[label={\arabic*)}]
    \item $X,\emptyset\in\sigma$
        \item $\bigcup_{i=1}^{n} U_i \in \sigma$ falls $U_i\in \sigma \text{ für } i = 1, \dots, n$ (endlich viele) 
        \item $\bigcap_{U\in\sigma^{\prime}} U \in \sigma$ falls $\sigma^{\prime} \subset \sigma$ (beliebig viele)
    \end{enumerate}
\end{folg}

\begin{mydefn}[Topologie]
    Sei $X$ Menge und $\tau$ Menge von Teilmengen von $X$ (d.h. $\tau \in \powerset(X)$)\\
    $\tau$ ist Topologie und $(X, \tau)$ topologischer Raum, falls 1), 2), 3) aus Satz \ref{8_13_satz_open_topo} gelten.
\end{mydefn}
\begin{remark}
    Menge $U\in \tau$ heißen dann offen (per Definition!). Folglich oben definierte offene Mengen in metrischen Räumen bilden ein Spezialfall für eine Topologie. Beachte! In metrischem Raum $(X,d)$ ist $\tilde{\tau} = \{\emptyset, X\}$ stets eine Topologie für beliebige Menge $X$).
\end{remark}

\begin{satz}
    Seinen $\Vert \cdot\Vert_1, \Vert \cdot\Vert_2$ äquivalente Normen auf $X$ und $U\subset X$. Dann\\
    $U$ offen bezüglich $\Vert \cdot\Vert_1 \Leftrightarrow U \text{ offen bezüglich } \Vert \cdot\Vert_2$.
\end{satz}

\begin{proof}
    Übungsaufgabe.\QEDA
\end{proof}

\begin{satz}
    Sei $(X,d)$ metrischer Raum und $M\subset X$. Dann
    \begin{enumerate}[label={\arabic*)}]
    \item $\inter M, \ext M$ offen
    \item $\partial M, \inter M$ abgeschlossen
    \item $M \leq \inter M$ falls $M$ offen, \\
    $M= \cl M$ falls $M$ abgeschlossen
    \end{enumerate}
\end{satz}

\begin{proof}
    \begin{enumerate}[label={\arabic*)}]
    \item Seien $x \in \inter M$, d.h. innere Punkte von $M \Rightarrow \exists \epsilon > 0 \colon B_{\epsilon}(x) \subset M$, da $B_{\epsilon}(x)$ offene Menge, ist jedes $y \in B_{\epsilon}(x)$ eine Teilemenge von $\inter M$ $\Rightarrow B_{\epsilon}(x) \subset M \beha$ ($\ext M$ analog)
    \item $\partial X\setminus (\inter M \cup \ext M)$ ist abgeschlossen, $\cl M = X\setminus\ext M$ abgeschlossen
    \item $M$ offen: es ist stets $\int M$ und da $M$ offen $M \subset \inter M \beha$  $\Rightarrow X\setminus M = \inter(X\setminus M) = \ext M = X \setminus \cl M \beha$.
    ($M$ abgeschlossen analog)
    \end{enumerate}\QEDA
\end{proof}