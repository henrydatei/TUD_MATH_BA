\section{Ausdrücke}

\begin{*anmerkung}
	Beliebtes Klausuren-Thema!
\end{*anmerkung}

Ein Ausdruck (= expression) wird in Fortran folgendermaßen ausgewertet:
\begin{enumerate}
	\item Konstanten und Objekte werden ausgewertet
	\item geklammerte Ausdrücke werden von innen nach außen ausgewertet
	\item Funktionen werden aufgerufen
	\item Operatoren höherer Priorität werden vor Operatoren niedrigerer Priorität behandelt
	\item sind die Prioritäten gleich, so wird von links nach rechts gearbeitet (Ausnahme: \texttt{**})
\end{enumerate}

Ein \begriff{Ausdrucksbaum} zeigt auf, wie in Fortran Ausdrücke ausgewertet werden: Für den Ausdruck (auch Infix-Notation)
\begin{align}
	\texttt{logo = i / j / x >= y .OR. - y / z < - x ** 3 ** 2 .AND. char <= 'p' // 'eter'}\notag
\end{align}
sieht der Baum folgendermaßen aus:
\begin{center}
	\begin{tikzpicture}[level/.style={sibling distance=60mm/#1}]
	\node[circle,draw] (root) {\texttt{.OR.}}
	child {node[circle,draw] (a) {\texttt{>=}}
		child {node[circle,draw] (b) {\texttt{/}}
			child {node[circle,draw] (c) {\texttt{/}}
				child {node[] (d) {$i$}}
				child {node[] (e) {$j$}}}
			child {node[] (f) {$x$}}}
		child {node[] (g) {$y$}}}
	child {node[circle,draw] (h) {\texttt{.AND.}}
		child {node[circle,draw] (i) {\texttt{<}}
			child {node[circle,draw] (j) {\texttt{-}}
				child {node[circle,draw] (k) {\texttt{/}}
					child {node[] (l) {$y$}}
					child {node[] (m) {$z$}}}}
			child {node[circle,draw] (q) {\texttt{-}}
				child {node[circle,draw] (p) {\texttt{**}}
					child {node[] (n) {$x$}}
					child {node[circle,draw] (o) {\texttt{**}}
						child {node[] (w) {3}}
						child {node[] (x) {2}}}}}}
		child {node[circle,draw] (r) {\texttt{<=}}
			child {node[] (s) {$char$}}
			child {node[circle,draw] (t) {\texttt{//}}
				child {node[] (u) {'p'}}
				child {node[] (v) {'eter'}}}}};
	\end{tikzpicture}
\end{center}

Es gibt verschiedene Notationen um Ausdrücke aufzuschreiben:
\begin{itemize}
	\item \begriff{Präfix-Notation}: Task $\to$ rekursiv linke Seite $\to$ rekursiv rechte Seite
	\item \begriff{Infix-Notation}: rekursiv linke Seite $\to$ Task $\to$ rekursiv rechte Seite
	\item \begriff{Postfix-Notation}: rekursiv linke Seite $\to$ rekursiv rechte Seite $\to$ Task
\end{itemize}

Für unseren Ausdruck bedeutet das:
\begin{itemize}
	\item Präfix-Notation: \texttt{= logo .OR. >= / / i j x y .AND. < NEG / y z NEG ** x ** 3 2 <= char // ‚p‘ ‚eter‘}
	\item Postfix-Notation: \texttt{logo i j / x / y >= y z / NEG x 3 2 ** ** NEG < char ‚p‘ ‚eter‘ // <= .AND. .OR. =}
\end{itemize}