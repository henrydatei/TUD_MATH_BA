\section{Das Anlageproblem}
\emph{Gegeben:} Vermögen $W$, Anlagegüter $S^1, \dots, S^n$ (Aktien, Anleihen, ...)\\
\emph{Gesucht:} Optimale Verteilung $W = W_1 + \dots + W_n$ auf $S^1 \dots S^n$\\
$S^1 \dots S^n$ weisen unterschiedliche Beträge, Risiken und typischerweise  Korrelationen auf.\\
\emph{Wir unterscheiden:}
\begin{itemize}
	\item \begriff{Einperiodenproblem}: Aufteilung wird heute $(t=0)$ festgelegt und bis zum Zeithorizont ($t=T$) beibehalten
	\item \begriff{Mehrperiodenproblem:} Umschichten zu mehreren Zeitpunkten $\set{t_0, t_1, \dots, t_N}$ möglich
\end{itemize}
\emph{Einfachstes Optimalitätsprinzip}: \begriff{\person{Pareto}-Optimalität}
\begin{itemize}
	\item Bei gleichem Risiko wird Anlage mit größeren Ertrag bevorzugt
	\item Bei gleichem Ertrag wird Anlage mit kleineren Risiko bevorzugt
\end{itemize}
d.h. \begriff{Pareto-Optimal} bedeutet, es gibt keine Anlagestrategie mit größerem Ertrag und kleinerem Risiko.\\
Zum Aufwärmen zwei Toy-Models ($=$ stark vereinfachte Beispiele)
\begin{itemize} %TODO fix structure 
	\item \emph{Toy-Model I}; Einperiodenmodell, eine risikofreie und eine risikobehaftete Anlagenmodel.
	\begin{itemize}
		\item Zeithorizont sei $T=1$
		\item risikofrei: $S_0^0 = 1$, $S_T^0 = (1+r)$
		\item risikobehaftet: $S_0^1 = 1$, $S_T^1 = (1+R)$ mit $R$ stochastisch,
		\begin{align*}
			\mu &= \E[R] \quad \text{Ertrag}\\
			\sigma &= \sqrt{\Var(R)} \quad \text{Risiko}
		\end{align*}
		\item $S=\mu -r$ Überrendite (excess return).
		\item $S \le 0 \implies$ Investiere alles in $S^0$ (Pareto-Optimal)
		\item $S > 0 \implies ?$
		\item Teile $W$ in $(W_0, W-W_0)$ auf $(S^0,S^1)$ auf (Jetzt: $W=1$)
		\begin{align*}
			\begin{cases}
				W-W_0 < 0 &\quad \text{Leerverkauf}\\
				W_0 < 0 &\quad \text{Kredit}
			\end{cases}
		\end{align*}
		\item Endvermögen: $P_T = W_0(1+r) + (1-W_0)(1+R)$
		\item Erweiterte Rendite: 
		\begin{align*}
			\mu_p &= \E[P_T-1] = W_0(1+r)+(W-W_0)(1+\mu)\\
			&= W_0 r + (1-W_0)\mu\\
		\end{align*}
		\item Risiko: $\sigma_p = (1-W_0)\sigma$
		\item Überrendite $S_p = (1-W_0)(\mu -r)$
		\item \emph{Jede} Strategie ist Pareto-Optimal, d.h. Pareto-Prinzip  hilft nicht bei der Auswahl. Insbesondere ist \begriff{Sharp-Ratio} 
		\begin{align*}
			SR(W_0) = \frac{\text{``Überrendite''}}{\text{``Risiko''}} = \frac{S_p}{\sigma_p} = \frac{\mu -r}{\sigma} \quad \text{konstant!}
		\end{align*}
	\end{itemize}
	\item \emph{Alternative} zum Pareto-Prinzip Festlegen von individueller Risikoaversion (mehr dazu später)
	\item \emph{Toy-Model II:} Einperiodenproblem, zwei risikobehaftete Anlagemöglichkeiten
	\begin{itemize}
		\item Zeithorizont $T=1$, Vermögen $W=1$
		\item 
		\begin{align*}
			\begin{matrix}
				S_0^1 = 1 & S_T^1 = (1+R_1) & \text{mit $\E[R_1] = \mu$, $\Var(R_1) = \sigma_1^2 > 0$}\\
				S_0^2 = 1 & S_T^2 = (1+R_1) & \text{mit $\E[R_2] = \mu$, $\Var(R_2) = \sigma_2^2 > 0$}
			\end{matrix}
		\end{align*}
		und $R_1 \upmodels R_2$ (unabhängig)
		\item Portofoliowert: $P_T = W_1(1+R_1) + (1-W_1)(1+R_2)$
		\item Rendite: $\mu_p = \E[P_T -1] = W_1\E[R_1] + (1-W_1)\E[R_2] = \mu$
		\item Risiko:
		\begin{align*}
			\sigma_p^2 = \Var(P_T -1) &= \Var(W_1 R_1 + (1-W_1)R_2)\\
			&= W^2_1\cdot \sigma_1^2 + (1-W_1)^2\sigma_2^2
		\end{align*}
		\item Pareto-Optimales Portofolio: 
		\begin{align*}
			0 &= 2W_1\cdot \sigma_1^2 - 2(1-W_1)\sigma_2^2
			\sigma_2^2\\
			&= W_1(\sigma_1^2 + \sigma_2^2)\\
			&\implies W_{\ast} = \frac{\sigma_2^2}{\sigma_1^2\cdot \sigma_2^2} \in (0,1)
		\end{align*}
		Also existiert genau eine Pareto-Optimale Strategie
		\item Vermögen wird proportional zum Verhältnis der Risiken aufgeteilt
		\item Vermögen wird \emph{nicht} vollständig in risiko-ärmere Anlagen gesteckt, also findet eine \begriff{Diversifikation} statt
		\item $W_{\ast}$ ist auch die Strategie mit maximaler \emph{Sharp-Ratio}
	\end{itemize}
\end{itemize}
Als nächstes: Pareto-Optimale Portofolio mit $n>2$ Anlagegütern
\section{Exkurs: Optimierung mit Nebenbedingung}
\emph{Betrachte Optimierungsproblem:}
\begin{align*}
	\min f_0(x)\quad x \in \R^n\\
	\intertext{unter Nebenbedingungen}
	\begin{cases}
		f_i(x) \le 0 &\quad i = 1, ..., m\\
		h_i(x) = 0 &\quad i = 1, ..., p
	\end{cases}
	\tag{OPT}\label{eq_4_1_opt}
\end{align*}
\begin{itemize}
	\item $x \in \R^n$ welches (NB) erfüllt heißt \begriff{zulässig}
	\item $x_{\ast} \in \R^n$ welches \eqref{eq_4_1_opt} normiert heißt \begriff{Optimallösung}
	\item $p_{\ast} = f_0(x_0)$ heißt \begriff{Minimalwert}
\end{itemize}
\begin{definition}
	\begin{enumerate}
		\item Die Funktion
		\begin{align*}
		\LL(x, \lambda, \nu) = f_0(x) + \sum_{i=1}^{m}f_i(x)\lambda_i + \sum_{i=1}^p h_i(x)\nu_i
		\end{align*}
		mit $\lambda \in \R^m_{\ge 0}, \nu \in \R^p$ heißt \begriff{Lagrange-Zielfunktion}
		\item Die Funktion
		\begin{align*}
			g(\lambda, \nu) = \inf_{x \in \R^m} \LL(x, \lambda, \nu)
		\end{align*}
		heißt \begriff{(Langrange-) duale Funktion} für \eqref{eq_4_1_opt}.
	\end{enumerate}
\end{definition}
\begin{*remark}
	Als Infimum von $(m, \lambda, \nu)$ lineare Funktionen ist $g$ \emph{konkav}.
	\begin{itemize}
		\item Die duale Funktion $g$ erzeugt \emph{untere Schranken} für $p_{\ast}$
		\item Begründung Sei $\overline{x} \in \R^n$ zulässig für \eqref{eq_4_1_opt}, d.h.
		\begin{align*}
			f_i(\overline{x}) &\le 0 \quad \forall i \in [n]\\
			h_i(\overline{x}) &= 0 \quad \forall i \in [p]
		\end{align*}
		\begin{align*}
			\implies \LL(\overline{x}, \lambda, \nu) = f_0(\overline{x}) + \underbrace{\sum_{i=1} f_i(\overline{x})\lambda_i + \sum_{i=1} h_i(\overline{x})\nu_i}_{=0} \le f_0(\overline{x})
		\end{align*}
		Also $g(\lambda, \nu) = \inf_{x \in \R^m} \LL(x, \lambda, \nu) \le \LL(\overline{x}, \lambda, \nu) \le f_0(\overline{x}) \quad \forall \overline{x}$ zulässig und damit folgt
		\begin{align*}
			g(\lambda, \nu) \le p_{\ast} \quad \forall \lambda \in \R^m_{>0}, \nu \in \R^n
		\end{align*}
		Die beste untere Schranke erhalten wir durch \emph{maximieren} über $\lambda, \nu$
	\end{itemize}
\end{*remark}
\begin{definition}
	Das duale Optimierungsproblem zu \eqref{eq_4_1_opt} ist
	\begin{align*}
		\max g(\lambda, \nu) \quad \lambda \in \R^m, \nu \in \R^p
		\intertext{unter Nebenbedingungen}
		\lambda_i \ge 0 \quad i = 1, \dots m \tag{D}\label{eq_4_2_Duality}
	\end{align*}
	Maximalwert: $d_{\ast}$
\end{definition}
\begin{itemize}
	\item Zwischen \eqref{eq_4_1_opt} und \eqref{eq_4_2_Duality} gilt \begriff{schwache Dualität}
	\begin{align*}
		d_{\ast} \le p_{\ast}
	\end{align*}
	\item Unter bestimmten Voraussetzungen gilt auch die \begriff{starke Dualität}
	\begin{align*}
		d_{\ast} = p_{\ast}
	\end{align*}
\end{itemize}
\begin{proof}
	Beweis folgt später.
\end{proof}
\section{Die \person{Markowitz}-Modelle}
\subsection*{Markowitz-Modell I}
(Portofolio-Optimierung \emph{ohne} risikofreie Anlage)\\
Anlagegüter $S = (S^1, \dots, S^n)$ mit stochastische ein-perioden Renditen $R = (R^1, \dots, R^n)$, d.h. $S_T = S_0^i(1+R^i), i \in [n]$ mit auf Analgegüter $S^1, \dots, S^n$ aufteilen $p_i$ Investitionen in $S^i$, d.h. $p_1 + \dots + p_n = W = 1$.
\begin{itemize}
	\item Erwartungswert: $\mu = \E[R] \in \Rn, \mu = \begin{pmatrix}
		\mu_1\ \\ \vdots \\ \mu_n
	\end{pmatrix}$
	\item $\Sigma = \E[(R-\mu)(R-\mu)^T]\quad (n\times 1)(1\times n) = (\Sigma_{ij})_{i,j \in [n]}$
	\begin{align*}
		\Sigma_{ij} = \Var(R^i)\\
		\Sigma_{ij} = \Cov(R^{i},R^j) \mit i \neq j
	\end{align*}
	\begin{itemize}
		\item \emph{Annahme:} $\Sigma$ ist regulär, d.h. $\Sigma^{-1}$ existiert.
		\item \emph{Ziel:} Anlagemengen $W=1$. 
		\item \emph{Erwartete Rendite:} $\mu_p = \E[p^T R] = p^T \mu$
		\item \emph{Risiko (Standardbereich):}
		\begin{align*}
			\sigma_p &= \sqrt{\Var(p^T R)} = \sqrt{\E[(p^T (R-\mu))^2]}\\
			&=\sqrt{\E[p^T (R-\mu)(R-\mu)^Tp]} = \sqrt{p^T \Sigma p}
		\end{align*}
		\item \emph{Optimales Anlageproblem:} Minimiere Risiko, gegeben Zielrendite $\mu_{\ast}$
		\begin{align*}
			\begin{cases}
				\min \half p^T \Sigma p & \quad\text{ über } p \in \R^n\\
				\text{ unter NB} &\quad p^T \mu = \mu_{\ast} (\text{ Zielrendite})\\
				&\quad p^T \indi = \indi (\indi = (1, \dots, 1) \in \Rn)
			\end{cases} \label{eq_Markow_one}\tag{Mark_I}
		\end{align*}
	\end{itemize}
	\item Die Lagrange-Zielfunktion: 
\begin{align*}
	\LL(p,\lambda_1, \lambda_2) = \half p^T \Sigma p + \lambda_1(\mu_{\ast} - p^T\mu) + \lambda_2(1-p^T \indi) \quad \mit \lambda_1, \lambda_2 \in \R
\end{align*}
	\item Die duale Funktion:
\begin{align*}
	g(\lambda_1, \lambda_2) &= \inf_{p \in \Rn} \LL(p, \lambda_1, \lambda_2)\\
	\nabla_p \LL(p, \lambda_1, \lambda_2) = \Sigma - \lambda_1 \mu - \lambda_2 1 = 0\\
	\implies p_{\ast} = \Sigma^{-1}(\lambda_1 \mu + \lambda_2 \mu)
\end{align*}
d.h. $g(\lambda_1, \lambda_2) = \LL(p_{\ast}, \lambda_1, \lambda_2)$
\begin{align*}
	\LL(p_{\ast}, \lambda_1, \lambda_2) &=
	\half (\lambda_1 \mu + \lambda_2 1)^T \Sigma^{-1}\Sigma\Sigma^{-1}(\lambda_1\mu - \lambda_2 1) - (\lambda_1 \mu + \lambda_2 1)^T \Sigma^{-1}(\lambda_1 \mu - \lambda_2 1) + \lambda_1 \mu_{\ast} + \lambda_2\\
	&= -\half (\lambda_1^2 a + 2 \lambda_1 \lambda_2 b + \lambda_2^2 c) + \lambda_1 \mu_{\ast} + \lambda_2
\end{align*}
mit
\begin{align*}
	a = \mu^T \Sigma^{-1}\mu, b = \mu^T\Sigma\indi, c= \indi^T\Sigma\indi
\end{align*}
Es gilt $a \ge 0, c \ge 0$ und (mit Cauchy-Schwarz) und damit $ac \ge b^2$
\item Maximieren von $g$:
	\begin{align*}
		\frac{\partial g}{\partial \lambda_1} &= -a \lambda_1 - b \lambda_2 + \mu_{\ast} = 0 \implies a\lambda_1 + b \lambda_2 = \mu_{\ast} \tag{I}\label{eq_mark_one_I}\\
		\frac{\partial g}{\partial \lambda_2} &= -b \lambda_1 - 1 \lambda_2 - 1 = 0 \implies b\lambda_1 + c \lambda_2 = 1 \tag{II}\label{eq_mark_one_II}
	\end{align*}
	\begin{align*}
		-b\eqref{eq_mark_one_I} + a\eqref{eq_mark_one_II}:\quad (ac - b^2)\lambda_2 = a - b\mu_{\ast} &\implies \lambda^{\ast}_2 = \frac{a - b \mu_{\ast}}{ac - b^2}\\
		c\eqref{eq_mark_one_I} -b\eqref{eq_mark_one_II}:\quad (ac - b^2)\lambda_1 = c\mu_{\ast} - b &\implies \lambda^{\ast}_2 = \frac{c\mu_{\ast} - b }{ac - b^2} \quad (\text{aber nur für } ac > b^2 )
	\end{align*}
\item Minimirer von \eqref{eq_Markow_one}:
\begin{align*}
	p_{\ast} = \lambda_1^{\ast}\Sigma^{-1}\mu + \lambda_2^{\ast}\Sigma^{-1}\indi
\end{align*}
\end{itemize}
\begin{conclusion}[Tobin-Two-Fund Seperation]
	Jeses Pareto-Optimale Portofolio für \eqref{eq_Markow_one} kann (unabhängig von $a$!) als Linearkombination der zwei Portolio
	\begin{align*}
		\underbrace{p^{\ast}_1 = \Sigma^{-1}\mu}_{\text{renditeorientiertes Portofolio}} \und \underbrace{p_2^{\ast} = \Sigma^{-1}\indi}_{\text{sicherheitsorientiertes Portofolio}}
	\end{align*}
	dargestellt werden.
\end{conclusion}
\begin{*remark}
	\begin{itemize}
		\item Gewichtung des Portofolios $p^{\ast}_1 \und p_2^{\ast}$ orientiert sich am Renditeziel $\mu_{\ast}$.
		\item $p^{\ast}_1 \und p_2^{\ast}$ sind breit diversifiziert, d.h. nutzen alle Anlagegüter $S = (S^1, \dots, S^n)$
		\item $p^{\ast}_1 \und p_2^{\ast}$ kann man als Anlagefunds interpretieren welche Vermögen entsprechend der Portfolio $p^{\ast}_1, p^{\ast}_2$ anlegen. Diese zwei Fonds sind ausreichend (unabhängig von $\mu_{\ast}$) um Vermögen Pareto-optimal zu investieren!
	\end{itemize}
\end{*remark}
Zueletzt wollen wir noch Risiko der optimalen Strategie $p_{\ast}$ berechnen:
\begin{align*}
	\sigma_{\ast}_^2 &= \Var(p_{\ast}^T R) = \E[(p_{\ast}^T (R-\mu))^2] = p_{\ast}^T \Sigma p_{\ast}\\
	&= (\lambda_1^{\ast}\mu + \lambda_2{\ast}\indi)^T \Sigma^{-1}\Sigma\Sigma^{-1}(\lambda^{\ast}\mu + \lambda_2^{\ast}\indi)\\
	&= (\lambda_1^{\ast}^2 a) + 2 \lambda_1^{\ast}\lambda_2^{\ast} b + (\lambda_2^{\ast}^2)c\\
	&= \frac{1}{a^2-b^2}^2 (((c\mu_{\ast} -b)^2)a + 2(\mu_{\ast} - b)(a-b \mu_{\ast})b + (a - b\mu_{\ast})^2c)\\
	&= \frac{1}{ac-b^2}(c \mu_{\ast}^2 - 2b\mu_{\ast} +a^2)
\end{align*}
Graph von $(\sigma_{\ast}, \mu_{\ast})$ ist ein Hyperbel-ast:\\
siehe picture phone ...\\
Nennt sich ``Markowitz-Bullet''!
\subsection*{Markowitz-Modell II}
(Optimale Investition \emph{mit} risikofreier Anlage)\\
\begin{itemize}
	\item Anlagegüter $S = (S^1, \dots, S^n)$ mit ein-perioden Rendite $R = (R^1, \dots, R^n)$
	\item Zusätzlich risikofreie Anlage $S^0$ mit Verzinsung $r$. Wegen $W=1$aufgestellt zu $1 = p_0 + p_1 + \dots p_n$. Wir setzen $p = (p_1,\vdots, p_n)^T \in \R^n$
	\item Erwartete Rendite: $\mu = \E[p^T R + (1-pT\indi)r] = p^T(\mu - r\indi) + r$
	\item Risko: $\sigma_{\ast} = \sqrt{\Var(p^T R)} = \sqrt{p^T \Sigma p}$
	\item Anlageproblem
	\begin{align*}
		\begin{cases}
			\min \half p^T \Sigma p & \quad p \in \R^n\\
			\text{ unter NB} &\quad p^T (\mu-r\indi) = \mu_{\ast} -r (\text{ Zielrendite})\\
		\end{cases} \label{eq_Markow_two}\tag{Mark_I}
	\end{align*}
	\item Lagrange ÜA
	\item Optimierer: $p_{\ast} = \lambda_{\ast}\Sigma^{-1}(\mu - r \indi)$ mit $\lambda_{\ast} = \frac{\mu_{\ast} - r}{a^2 - 2br + cr^2}$
\end{itemize}
\begin{conclusion}[Tobin's One-Fund-Theorem]
	Jedes Pareto-Optimale Portfolio für \eqref{eq_Markow_two} kann als Linearkombination der riskofreien Anlage und des Portf.
	\begin{align*}
		\Sigma^{-1}(\mu -r\indi)
	\end{align*}
	dargestellt werden.
\end{conclusion}
Graph von min. Risko $\sigma_{\ast}$ und Zielrendite $\mu_{\ast}$
siehe phone