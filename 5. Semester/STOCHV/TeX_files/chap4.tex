\section{Das Anlageproblem}
\emph{Gegeben:} Vermögen $W$, Anlagegüter $S^1, \dots, S^n$ (Aktien, Anleihen, ...)\\
\emph{Gesucht:} Optimale Verteilung $W = W_1 + \dots + W_n$ auf $S^1 \dots S^n$\\
$S^1 \dots S^n$ weisen unterschiedliche Beträge, Risiken und typischerweise  Korrelationen auf.\\
\emph{Wir unterscheiden:}
\begin{itemize}
	\item \begriff{Einperiodenproblem}: Aufteilung wird heute $(t=0)$ festgelegt und bis zum Zeithorizont ($t=T$) beibehalten
	\item \begriff{Mehrperiodenproblem:} Umschichten zu mehreren Zeitpunkten $\set{t_0, t_1, \dots, t_N}$ möglich
\end{itemize}
\emph{Einfachstes Optimalitätsprinzip}: \begriff{\person{Pareto}-Optimalität}
\begin{itemize}
	\item Bei gleichem Risiko wird Anlage mit größeren Ertrag bevorzugt
	\item Bei gleichem Ertrag wird Anlage mit kleineren Risiko bevorzugt
\end{itemize}
d.h. \begriff{Pareto-Optimal} bedeutet, es gibt keine Anlagestrategie mit größerem Ertrag und kleinerem Risiko.\\
Zum Aufwärmen zwei Toy-Models ($=$ stark vereinfachte Beispiele)
\begin{itemize} %TODO fix structure 
	\item \emph{Toy-Model I}; Einperiodenmodell, eine risikofreie und eine risiko-behaftete Anlagenmodel.
	\begin{itemize}
		\item Zeithorizont sei $T=1$
		\item risikofrei: $S_0^0 = 1$, $S_T^0 = (1+r)$
		\item risikobehaftet: $S_0^1 = 1$, $S_T^1 = (1+R)$ mit $R$ stochastisch,
		\begin{align*}
			\mu &= \E[R] \quad \text{Ertrag}\\
			\sigma &= \sqrt{\Var(R)} \quad \text{Risiko}
		\end{align*}
		\item $S=\mu -r$ Überrendite (excess return).
		\item $S \le 0 \implies$ Investiere alles in $S^0$ (Pareto-Optimal)
		\item $S > 0 \implies ?$
		\item Teile $W$ in $(W_0, W-W_0)$ auf $(S^0,S^1)$ auf (Jetzt: $W=1$)
		\begin{align*}
			\begin{cases}
				W-W_0 < 0 &\quad \text{Leerverkauf}\\
				W_0 < 0 &\quad \text{Kredit}
			\end{cases}
		\end{align*}
		\item Endvermögen: $P_T = W_0(1+r) + (1-W_0)(1+R)$
		\item Erweiterte Rendite: 
		\begin{align*}
			\mu_p &= \E[P_T-1] = W_0(1+r)+(W-W_0)(1+\mu)\\
			&= W_0 r + (1-W_0)\mu\\
		\end{align*}
		\item Risiko: $\sigma_p = (1-W_0)\sigma$
		\item Überrendite $S_p = (1-W_0)(\mu -r)$
		\item \emph{Jede} Strategie ist Pareto-Optimal, d.h. Pareto-Prinzip  hilft nicht bei der Auswahl. Insbesondere ist \begriff{Sharp-Ratio} 
		\begin{align*}
			SR(W_0) = \frac{\text{``Überrendite''}}{\text{``Risiko''}} = \frac{S_p}{\sigma_p} = \frac{\mu -r}{\sigma} \quad \text{konstant!}
		\end{align*}
	\end{itemize}
	\item \emph{Alternative} zum Pareto-Prinzip Festlegen von individueller Risikoaversion (mehr dazu später)
	\item \emph{Toy-Model II:} Einperiodenproblem, zwei risikobehaftete Anlagemöglichkeiten
	\begin{itemize}
		\item Zeithorizont $T=1$, Vermögen $W=1$
		\item 
		\begin{align*}
			\begin{matrix}
				S_0^1 = 1 & S_T^1 = (1+R_1) & \text{mit $\E[R_1] = \mu$, $\Var(R_1) = \sigma_1^2 > 0$}\\
				S_0^2 = 1 & S_T^2 = (1+R_1) & \text{mit $\E[R_2] = \mu$, $\Var(R_2) = \sigma_2^2 > 0$}
			\end{matrix}
		\end{align*}
		und $R_1 \upmodels R_2$ (unabhängig)
		\item Portfoliowert: $P_T = W_1(1+R_1) + (1-W_1)(1+R_2)$
		\item Rendite: $\mu_p = \E[P_T -1] = W_1\E[R_1] + (1-W_1)\E[R_2] = \mu$
		\item Risiko:
		\begin{align*}
			\sigma_p^2 = \Var(P_T -1) &= \Var(W_1 R_1 + (1-W_1)R_2)\\
			&= W^2_1\cdot \sigma_1^2 + (1-W_1)^2\sigma_2^2
		\end{align*}
		\item Pareto-Optimales Portfolio: 
		\begin{align*}
			0 &= 2W_1\cdot \sigma_1^2 - 2(1-W_1)\sigma_2^2
			\sigma_2^2\\
			&= W_1(\sigma_1^2 + \sigma_2^2)\\
			&\implies W_{\ast} = \frac{\sigma_2^2}{\sigma_1^2\cdot \sigma_2^2} \in (0,1)
		\end{align*}
		Also existiert genau eine Pareto-Optimale Strategie
		\item Vermögen wird proportional zum Verhältnis der Risiken aufgeteilt
		\item Vermögen wird \emph{nicht} vollständig in risiko-ärmere Anlagen gesteckt, also findet eine \begriff{Diversifikation} statt
		\item $W_{\ast}$ ist auch die Strategie mit maximaler \emph{Sharp-Ratio}
	\end{itemize}
\end{itemize}
Als nächstes: Pareto-Optimale Portfolio mit $n>2$ Anlagegütern
\section{Exkurs: Optimierung mit Nebenbedingung}
\emph{Betrachte Optimierungsproblem:}
\begin{align*}
	\min f_0(x)\quad x \in \R^n\\
	\intertext{unter Nebenbedingungen}
	\begin{cases}
		f_i(x) \le 0 &\quad i = 1, ..., m\\
		h_i(x) = 0 &\quad i = 1, ..., p
	\end{cases}
	\tag{OPT}\label{eq_4_1_opt}
\end{align*}
\begin{itemize}
	\item $x \in \R^n$ welches (NB) erfüllt heißt \begriff{zulässig}
	\item $x_{\ast} \in \R^n$ welches \eqref{eq_4_1_opt} normiert heißt \begriff{Optimallösung}
	\item $p_{\ast} = f_0(x_0)$ heißt \begriff{Minimalwert}
\end{itemize}
\begin{definition}
	\begin{enumerate}
		\item Die Funktion
		\begin{align*}
		\LL(x, \lambda, \nu) = f_0(x) + \sum_{i=1}^{m}f_i(x)\lambda_i + \sum_{i=1}^p h_i(x)\nu_i
		\end{align*}
		mit $\lambda \in \R^m_{\ge 0}, \nu \in \R^p$ heißt \begriff{Lagrange-Zielfunktion}
		\item Die Funktion
		\begin{align*}
			g(\lambda, \nu) = \inf_{x \in \R^m} \LL(x, \lambda, \nu)
		\end{align*}
		heißt \begriff{(Langrange-) duale Funktion} für \eqref{eq_4_1_opt}.
	\end{enumerate}
\end{definition}
\begin{*remark}
	Als Infimum von $g(\lambda, \nu)$ lineare Funktionen ist $g$ \emph{konkav}.
	\begin{itemize}
		\item Die duale Funktion $g$ erzeugt \emph{untere Schranken} für $p_{\ast}$
		\item Begründung Sei $\overline{x} \in \R^n$ zulässig für \eqref{eq_4_1_opt}, d.h.
		\begin{align*}
			f_i(\overline{x}) &\le 0 \quad \forall i \in [n]\\
			h_i(\overline{x}) &= 0 \quad \forall i \in [p]
		\end{align*}
		\begin{align*}
			\implies \LL(\overline{x}, \lambda, \nu) = f_0(\overline{x}) + \underbrace{\sum_{i=1} f_i(\overline{x})\lambda_i + \sum_{i=1} h_i(\overline{x})\nu_i}_{=0} \le f_0(\overline{x})
		\end{align*}
		Also $g(\lambda, \nu) = \inf_{x \in \R^m} \LL(x, \lambda, \nu) \le \LL(\overline{x}, \lambda, \nu) \le f_0(\overline{x}) \quad \forall \overline{x}$ zulässig und damit folgt
		\begin{align*}
			g(\lambda, \nu) \le p_{\ast} \quad \forall \lambda \in \R^m_{>0}, \nu \in \R^n
		\end{align*}
		Die beste untere Schranke erhalten wir durch \emph{maximieren} über $\lambda, \nu$
	\end{itemize}
\end{*remark}
\begin{definition}
	Das duale Optimierungsproblem zu \eqref{eq_4_1_opt} ist
	\begin{align*}
		\max g(\lambda, \nu) \quad \lambda \in \R^m, \nu \in \R^p
		\intertext{unter Nebenbedingungen}
		\lambda_i \ge 0 \quad i = 1, \dots m \tag{D}\label{eq_4_2_Duality}
	\end{align*}
	Maximalwert: $d_{\ast}$
\end{definition}
\begin{itemize}
	\item Zwischen \eqref{eq_4_1_opt} und \eqref{eq_4_2_Duality} gilt \begriff{schwache Dualität}
	\begin{align*}
		d_{\ast} \le p_{\ast}
	\end{align*}
	\item Unter bestimmten Voraussetzungen gilt auch die \begriff{starke Dualität}
	\begin{align*}
		d_{\ast} = p_{\ast}
	\end{align*}
\end{itemize}
Das duale Problem hat auch eine Lösung $(\lambda_{\ast}, \nu_{\ast})$ und Maximalwert $d_{\ast}$
\begin{lemma}
	Zwischen \eqref{eq_4_1_opt} und \eqref{eq_4_2_Duality} gilt die \begriff{schwache Dualität}
	\begin{align*}
		d_{\ast} \le p_{\ast}\tag{WD}\label{eq_4_3_weak_Duality}
	\end{align*}
\end{lemma}
\begin{proof}
	SeSt.
\end{proof}
\begin{*remark}
	\begin{itemize}
		\item Die Differenz $p_{\ast} - d_{\ast} \ge 0$ heißt \begriff{Dualitätslücke} (duality gap).
		\item Wenn Dualitätslücke verschwindet \begriff{starke Dualität}
		\begin{align*}
			d_{\ast} = p_{\ast}
		\end{align*}
		\item hinreichende Bedingungen für starke Dualität existieren vor allem für \emph{konvexe} Probleme
	\end{itemize}
\end{*remark}
\begin{definition}
	Optimierungsproblem\eqref{eq_4_1_opt} ist konvex wenn $f_0$ konvex ist und die Menge der zulässigen Werte konvex ist. In diesem Fall kann \eqref{eq_4_1_opt} in folgende Form gebracht werden:
	\begin{align*}
		\min f_0(x) \quad x \in \R^n\\
		\intertext{unter NB}
		\begin{cases}
			f_i(x) \le 0 &\quad i \in [m]\\
			Ax = b &\quad
		\end{cases}\tag{K-OPT}\label{eq_4_4_kopt}
	\end{align*}
	mit $f_0, f_1, \dots, f_m$ konvex, $A \in \R^{p\times m}, b \in \R^p$.
\end{definition}
\begin{theorem}[Slaters-Bedingung]
	Betrachte das konvexe Optimierungsprobleme \eqref{eq_4_1_opt}. Wenn $x \in \R^n$ existiert mit
	\begin{align*}
		f_i(x) < 0 \quad \forall i \in [m] \und Ax = b
	\end{align*}
	dann gilt starke Dualität.
\end{theorem}
Für den Beweis verwende wir den Trennungssatz für konvexe Mengen.
\begin{theorem}[Trennungssatz für konvexe Mengen]
	Sei $A,B \subseteq \R^n$ konvex nichtleer und disjunkt, d.h.
	\begin{align*}
		A \cap B = \emptyset
	\end{align*}
	Dann existieren $a \in \R^n \setminus \set{0} \und b \in \R$, sodass
	\begin{align*}
		a^T x \ge b \quad \forall x \in A\\
		a^T x \le b \quad \forall x \in B
	\end{align*}
	Die Hyperebene $h = \set{x \in \R^n \colon a^T x = b}$ heisst \begriff{trennende Hyperbene} für $A \und B$
\end{theorem}
\begin{proof}
	Ohne Beweis.
\end{proof}
Skizze: 
\begin{proof}[Theorem 4.2?]
	Betrachte folgende Teilmengen von $\R^N = \R^{m+p+1}$, 
	\begin{align*}
		\G&=\set{(u,v,t) \in \R^N \colon \exists x \in \R^n \mit \begin{cases}
			f_i(x) = u_i, h_i(x) = v &\quad \forall i \in [m], Ax - b = v\\
			f_0(x) = t &\quad
			\end{cases}}\\
		\G&= \set{(f_1(x), \dots, f_m(x), h_1(x),h_p(x),f_0(x)) \in \R^N \colon x \in \R^n} \subseteq \AAA\\
		\AAA &= \set{(u,v,t) \in \R^N \colon \exists x \in \R^n \mit f_i(x) \le u_i \forall i \in [m], Ax-b = v, f_0(x) \le t} = \G \oplus \R^m_{\ge 0} \times \set{0}^p \times \R_{\ge 0}\\
		\BBB &= \set{(0,0,t)\in \R^N \colon t < p_{\ast}}
	\end{align*}
	Es gilt $\AAA \und \BBB$ sind konvex. Nun folgt die 
	\begin{itemize}
		\item Behauptung: $\AAA \cap \BBB = \emptyset$. Mit Widerspruch: Angenommen es existiert $(u,v,t) \in \AAA \cap \BBB$, dann gilt\\
		wegen $\BBB$
		\begin{align*}
		u=0, v = 0 \und t < p_{\ast}
		\end{align*}
		wegen $\AAA$
		\begin{align*}
		\exists x \in \R^n \mit &f_i(x) \le u_i = 0 \quad i \in [m]\\
		& h_i(x) = v_i = 0 \quad i \in [p]\\
		&f_0(x) \le t < p_{\ast}
		\end{align*}
		d.h. $x$ ist zulässig für \eqref{eq_4_4_kopt} und besser als optimal! $(f_0(x) < p_{\ast})$. Damit ist die Behauptung gezeigt und es folgt $\AAA \cap \BBB = \emptyset$.
		\item Wende Trennungssatz an: $\exists (\lambda, \nu, v) \in \R^N \setminus \set{0}$ und $\alpha \in \R$ mit
		\begin{align*}
			\lambda^T u + \nu^T v + \mu t \ge \alpha \quad (u,v,t) \in \AAA \tag{I}\label{proof_eq_4_2a}\\
			\lambda^T u + \nu^T v + \mu t \le \alpha \quad \forall (u,v,t) \in \BBB \tag{II}\label{proof_eq_4_2b}
		\end{align*}
		\eqref{proof_eq_4_2b} folgt $\mu t \le \alpha \forall t < p_{\ast}$ (da $a = 0, v = 0$) und damit gilt $\mu p_{\ast} \le \alpha$\\
		Aus \eqref{proof_eq_4_2a} bekommt man $\lambda_i \ge 0 \forall i \in [m]$ und $\mu \ge 0$ (sonst Widerspruch!). Dann nimmt man \eqref{proof_eq_4_2a} und \eqref{proof_eq_4_2b} zusammen und hat $\forall x \in \R^n$
		\begin{align*}
			&\sum_{i=1} \lambda_i f_i(x) + \sum_{i=1} \nu_i(Ax-b) + \mu f_0(x)\\
			&\le \lambda^T u + \nu^T v + \mu t \overset{\eqref{proof_eq_4_2a}}{\ge} \alpha \overset{\eqref{proof_eq_4_2b}}{\ge} \mu p_{\ast}\tag{$\ast$}\label{proof_eq_4_2c}
		\end{align*}
		Nun gibt es zwei Fälle:
		\begin{itemize}
			\item Fall: $\mu > 0$ Setze $\tilde{\lambda} = \lambda / mu$, $\tilde{\nu} = \nu /\mu$, damit\\
			$\exists x \in \R^n \mit \sum_{i=1} \tilde{\lambda}_i f_i(x) + \sum_{i=1} \nu_i (Ax-b) + f_0(x) \ge p_{\ast}$ und damit folgt $g(\tilde{\lambda}, \tilde{\nu}) \ge p_{\ast}$, d.h. $d_{\ast} = \max_{(\lambda, \nu)\in \R_{\ge 0}^m \times \R^n} g(\lambda, \nu) \ge p_{\ast}$. Aber mit schwacher Dualität: $d_{\ast} \le p_{\ast}$
			\item Fall: $\mu = 0$ (kann nicht eintreten, weil ...). Aus \eqref{proof_eq_4_2c} bekommen wir
			\begin{align*}
				\sum_{i=1} \lambda_i f_i(x) + \sum_{i=1} \nu_i(Ax-b) \ge 0 \quad \forall x \in \R^n
			\end{align*}
			Slaters-Bedingung $\exists \overline{x} \in \R^n$ mit $f_i(x) < 0, i \in [m]$ und $A\overline{x} - b = 0$ und damit
			\begin{align*}
				\sum_{i=1}\underbrace{\lambda_i}_{\ge 0}\underbrace{f_i(\overline{x})}_{< 0} > 0 \implies \lambda = 0
			\end{align*}
			$(\lambda, \nu, \mu) = (0, \nu, 0)\in \R^N \setminus \set{0}$ impliziert $\nu \neq 0$ und $\nu^T(Ax-b) = 0$, dann existiert nach $\overline{x}$ mit $\nu^T(A\overline{x} - b) < 0$ und das ist der Widerspruch, d.h. $\mu = 0$ tritt nicht ein.
		\end{itemize}
	\end{itemize}
\end{proof}
\section{Die \person{Markowitz}-Modelle}
\subsection*{Markowitz-Modell I}
(Portfolio-Optimierung \emph{ohne} risikofreie Anlage)\\
Anlagegüter $S = (S^1, \dots, S^n)$ mit stochastische ein-perioden Renditen $R = (R^1, \dots, R^n)$, d.h. $S_T = S_0^i(1+R^i), i \in [n]$ mit auf Analgegüter $S^1, \dots, S^n$ aufteilen $p_i$ Investitionen in $S^i$, d.h. $p_1 + \dots + p_n = W = 1$.
\begin{itemize}
	\item Erwartungswert: $\mu = \E[R] \in \Rn, \mu = (\mu_1, dots, \mu_n)^T$
	\item $\Sigma = \E[(R-\mu)(R-\mu)^T]\quad (n\times 1)(1\times n) = (\Sigma_{ij})_{i,j \in [n]}$
	\begin{align*}
		\Sigma_{ij} = \Var(R^i)\\
		\Sigma_{ij} = \Cov(R^{i},R^j) \mit i \neq j
	\end{align*}
	\begin{itemize}
		\item \emph{Annahme:} $\Sigma$ ist regulär, d.h. $\Sigma^{-1}$ existiert.
		\item \emph{Ziel:} Anlagemengen $W=1$. 
		\item \emph{Erwartete Rendite:} $\mu_p = \E[p^T R] = p^T \mu$
		\item \emph{Risiko (Standardbereich):}
		\begin{align*}
			\sigma_p &= \sqrt{\Var(p^T R)} = \sqrt{\E[(p^T (R-\mu))^2]}\\
			&=\sqrt{\E[p^T (R-\mu)(R-\mu)^Tp]} = \sqrt{p^T \Sigma p}
		\end{align*}
		\item \emph{Optimales Anlageproblem:} Minimiere Risiko, gegeben Zielrendite $\mu_{\ast}$
		\begin{align}
			\begin{cases}
				\min \half p^T \Sigma p & \quad\text{über } p \in \R^n\\
				\text{ unter NB} &\quad p^T \mu = \mu_{\ast} \text{ ( Zielrendite)}\\
				&\quad p^T \indi = \indi (\indi = (1, \dots, 1) \in \R^n)
			\end{cases} \label{eq_Markow_one}\tag{Mark I}
		\end{align}
	\end{itemize}
	\item Die Lagrange-Zielfunktion: 
\begin{align*}
	\LL(p,\lambda_1, \lambda_2) = \half p^T \Sigma p + \lambda_1(\mu_{\ast} - p^T\mu) + \lambda_2(1-p^T \indi) \quad \mit \lambda_1, \lambda_2 \in \R
\end{align*}
	\item Die duale Funktion:
\begin{align*}
	g(\lambda_1, \lambda_2) &= \inf_{p \in \Rn} \LL(p, \lambda_1, \lambda_2)\\
	\nabla_p \LL(p, \lambda_1, \lambda_2) &= \Sigma - \lambda_1 \mu - \lambda_2 1 = 0\\
	\implies p_{\ast} &= \Sigma^{-1}(\lambda_1 \mu + \lambda_2 \mu)
\end{align*}
d.h. $g(\lambda_1, \lambda_2) = \LL(p_{\ast}, \lambda_1, \lambda_2)$
\begin{align*}
	\LL(p_{\ast}, \lambda_1, \lambda_2) &=
	\half (\lambda_1 \mu + \lambda_2 1)^T \Sigma^{-1}\Sigma\Sigma^{-1}(\lambda_1\mu - \lambda_2 1) - (\lambda_1 \mu + \lambda_2 1)^T \Sigma^{-1}(\lambda_1 \mu - \lambda_2 1) + \lambda_1 \mu_{\ast} + \lambda_2\\
	&= -\half (\lambda_1^2 a + 2 \lambda_1 \lambda_2 b + \lambda_2^2 c) + \lambda_1 \mu_{\ast} + \lambda_2
\end{align*}
mit
\begin{align*}
	a = \mu^T \Sigma^{-1}\mu, b = \mu^T\Sigma\indi, c= \indi^T\Sigma\indi
\end{align*}
Es gilt $a \ge 0, c \ge 0$ und (mit Cauchy-Schwarz) und damit $ac \ge b^2$
\item Maximieren von $g$:
	\begin{align*}
		\frac{\partial g}{\partial \lambda_1} &= -a \lambda_1 - b \lambda_2 + \mu_{\ast} = 0 \implies a\lambda_1 + b \lambda_2 = \mu_{\ast} \tag{I}\label{eq_mark_one_I}\\
		\frac{\partial g}{\partial \lambda_2} &= -b \lambda_1 - 1 \lambda_2 - 1 = 0 \implies b\lambda_1 + c \lambda_2 = 1 \tag{II}\label{eq_mark_one_II}
	\end{align*}
	\begin{align*}
		-b\eqref{eq_mark_one_I} + a\eqref{eq_mark_one_II}:\quad (ac - b^2)\lambda_2 = a - b\mu_{\ast} &\implies \lambda^{\ast}_2 = \frac{a - b \mu_{\ast}}{ac - b^2}\\
		c\eqref{eq_mark_one_I} -b\eqref{eq_mark_one_II}:\quad (ac - b^2)\lambda_1 = c\mu_{\ast} - b &\implies \lambda^{\ast}_2 = \frac{c\mu_{\ast} - b }{ac - b^2} \quad (\text{aber nur für } ac > b^2 )
	\end{align*}
\item Minimierer von \eqref{eq_Markow_one}:
\begin{align*}
	p_{\ast} = \lambda_1^{\ast}\Sigma^{-1}\mu + \lambda_2^{\ast}\Sigma^{-1}\indi
\end{align*}
\end{itemize}
\begin{conclusion}[Tobin-Two-Fund Seperation]
	Jedes Pareto-Optimale Portfolio für \eqref{eq_Markow_one} kann (unabhängig von $a$!) als Linearkombination der zwei Portfolio
	\begin{align*}
		\underbrace{p^{\ast}_1 = \Sigma^{-1}\mu}_{\text{renditeorientiertes Portofolio}} \und \underbrace{p_2^{\ast} = \Sigma^{-1}\indi}_{\text{sicherheitsorientiertes Portofolio}}
	\end{align*}
	dargestellt werden.
\end{conclusion}
\begin{*remark}
	\begin{itemize}
		\item Gewichtung des Portfolios $p^{\ast}_1 \und p_2^{\ast}$ orientiert sich am Renditeziel $\mu_{\ast}$.
		\item $p^{\ast}_1 \und p_2^{\ast}$ sind breit diversifiziert, d.h. nutzen alle Anlagegüter $S = (S^1, \dots, S^n)$
		\item $p^{\ast}_1 \und p_2^{\ast}$ kann man als Anlagefunds interpretieren welche Vermögen entsprechend der Portfolio $p^{\ast}_1, p^{\ast}_2$ anlegen. Diese zwei Fonds sind ausreichend (unabhängig von $\mu_{\ast}$) um Vermögen Pareto-optimal zu investieren!
	\end{itemize}
\end{*remark}
Zuletzt wollen wir noch Risiko der optimalen Strategie $p_{\ast}$ berechnen:
\begin{align*}
	\sigma_{\ast}^2 &= \Var(p_{\ast}^T R) = \E[(p_{\ast}^T (R-\mu))^2] = p_{\ast}^T \Sigma p_{\ast}\\
	&= (\lambda_1^{\ast}\mu + \lambda_2^{\ast}\indi)^T \Sigma^{-1}\Sigma\Sigma^{-1}(\lambda_1^{\ast}\mu + \lambda_2^{\ast}\indi)\\
	&= (\lambda_1^{\ast 2}2 a) + 2 \lambda_1^{\ast}\lambda_2^{\ast} b + (\lambda_2^{\ast 2})c\\
	&= \frac{1}{a^2-b^2}^2 (((c\mu_{\ast} -b)^2)a + 2(\mu_{\ast} - b)(a-b \mu_{\ast})b + (a - b\mu_{\ast})^2c)\\
	&= \frac{1}{ac-b^2}(c \mu_{\ast}^2 - 2b\mu_{\ast} +a^2)
\end{align*}
Graph von $(\sigma_{\ast}, \mu_{\ast})$ ist ein Hyperbel-ast:\\
siehe picture phone ...\\
Nennt sich ``Markowitz-Bullet''!
\subsection*{Markowitz-Modell II}
(Optimale Investition \emph{mit} risikofreier Anlage)\\
\begin{itemize}
	\item Anlagegüter $S = (S^1, \dots, S^n)$ mit ein-perioden Rendite $R = (R^1, \dots, R^n)$
	\item Zusätzlich risikofreie Anlage $S^0$ mit Verzinsung $r$. Wegen $W=1$aufgestellt zu $1 = p_0 + p_1 + \dots p_n$. Wir setzen $p = (p_1,\dots, p_n)^T \in \R^n$
	\item Erwartete Rendite: $\mu = \E[p^T R + (1-p^T \indi)r] = p^T(\mu - r\indi) + r$
	\item Risiko: $\sigma_{\ast} = \sqrt{\Var(p^T R)} = \sqrt{p^T \Sigma p}$
	\item Anlageproblem:
	\begin{align*}
		\begin{cases}
			\min \half p^T \Sigma p & \quad p \in \R^n\\
			\text{ unter NB} &\quad p^T (\mu-r\indi) = \mu_{\ast} -r (\text{ Zielrendite})\\
		\end{cases}\tag{Mark II}\label{eq_Markow_two}
	\end{align*}
	\item Lagrange ÜA
	\item Optimierer: $p_{\ast} = \lambda_{\ast}\Sigma^{-1}(\mu - r \indi)$ mit $\lambda_{\ast} = \frac{\mu_{\ast} - r}{a^2 - 2br + cr^2}$
\end{itemize}
\begin{conclusion}[Tobin's One-Fund-Theorem]
	Jedes Pareto-Optimale Portfolio für \eqref{eq_Markow_two} kann als Linearkombination der risko-freien Anlage und des Portfolio
	\begin{align*}
		\Sigma^{-1}(\mu -r\indi)
	\end{align*}
	dargestellt werden.
\end{conclusion}
Graph von min. Risiko $\sigma_{\ast}$ und Zielrendite $\mu_{\ast}$
siehe phone
\begin{*remark}[nominales vs. relatives Portfolio]
	\begin{itemize}
		\item $\vartheta = (\vartheta_1, \dots, \vartheta_2) \in \R^n$ mit $\vartheta_i$ Stückzahl von Anlagegut $S^i$ und Portfoliowert
		\begin{align*}
			V_0 &= \vartheta^T S_0 = \sum_{i=1}^n \vartheta_i S_0^i = w \quad \dots \text{ Anfangskapital}\\
			V_T &= \vartheta^T S_T = \sum_{i=1}\vartheta_i S_T^i
		\end{align*}
		\item relatives Portfolio: $p := (p_1, \dots, p_n) \in \R^n$ mit $p_i = \frac{\vartheta_i S_0^i}{W}$ Vermögensanteil in $S^i$
		\begin{align*}
			\sum_{i=1}^n p_i = \frac1w\sum_{i=1}^n \vartheta_i S_0^i = \frac w w = 1
		\end{align*}
		\item Renditen: 
		\begin{itemize}
			\item Einzelnes Anlagegut: $R_i = \frac{S_T^i - S_0^i}{S_0^i}$
			\item Gesamtes Portfolio: 
			\begin{align*}
				R_p &= \frac{V_T - V_0}{V_0} = 1/w (\sum_{i=1}^n \vartheta_i S_0^i - \vartheta_i S_0^i)\\
				&= 1/w \sum_{i=1}^n \vartheta_i(S_T^i - S_0^i) = \sum_{i=1}^n \underbrace{\frac{\vartheta_i S_0^i}{w_{p_i}}}R_i\\
				&= \sum_{i=1}^n p_i R_i = p^T R \quad \dots \text{ linear in }p
			\end{align*}
		\end{itemize}
	\end{itemize}
\end{*remark}
\section{Capital Asset Pricing Model (CAPM)}
\begin{itemize}
	\item Ausgangspunkt: Optimalportfolio in \eqref{eq_Markow_two} $p_{\ast} = \lambda \Sigma^{-1}(\mu - r\indi)$. Normiere $p_{\ast}^T \indi = \indi \implies \lambda_{\ast} = \frac{1}{\indi^T \Sigma^{-1} (\mu - r \indi)} = \frac{1}{b-cr}$ (Marktportfolio)
	\item Wert des Marktportfolios: $M_0 = 1, M_T = (1+ p_{\ast}^T R)$, Rendite $R_M = p_{\ast}^T R$
	\item Zentrale Idee des CAPM:
	\begin{itemize}
		\item Betrachte $M$ als \emph{beobachtbare Größe}
		\item Aktienindex DAX oder S\&P500 sollte gute Näherung für $M$ ergeben.
	\end{itemize}
\end{itemize}
\paragraph*{Wir betrachten folgende Kennzahlen:}
\begin{itemize}
	\item Überschussrendite: [excess return] ($\alpha$)
	\begin{align*}
		\alpha_i &= \E[R_i] - r = \mu_i - r \quad \dots \text{ für Wertpapiere }S^i\\
		\alpha_M &= \E[R_M] - r = p_{\ast}^T\mu -r = \frac{\mu^T \Sigma^{-1}(\mu - r\indi)}{\indi^T\Sigma^{-1}(\mu - r \indi)} - r \\
		&= \frac{a - rb}{b-rc} - r = \frac{a - 2rb +r^2c}{b-cr}
	\end{align*}
	\item \begriff{Beta-Koeffizient}
	\begin{align*}
		\beta_i = \frac{\Cov(R_1,R_M)}{\Var(R_M)}
	\end{align*}
	skalierte Kovarianz zwischen Erträgen von $S^i$ und $M$
	\begin{itemize}
		\item Maß für Korrelation zwischen Wertpapiere $S^i$ und Marktportfolio
		\item Volle Kovarianzmatrix wird \emph{nicht} benötigt
	\end{itemize}
	\item Wir berechnen:
	\begin{align*}
		\beta_i &= \frac{\E[(R_i - \mu_i)(R_M-\mu_M)]}{\E[(R_M - \mu_M)^2]} = \frac{\E[e_i^T(R-\mu)(R-\mu)^Tp_{\ast}]}{\E[p_{\ast}^T(R-\mu)(R-\mu)^Tp_{\ast}]} = \frac{e_i^T \Sigma p_{\ast}}{p_{\ast}^T \Sigma p_{\ast}}\\
		&= \frac{\lambda_{\ast} e_i^T \Sigma \Sigma^{-1}(\mu - r \indi)}{\lambda_{\ast} (\mu - r \indi)^T \Sigma^{-1}\Sigma \Sigma^{-1}(\mu - r \indi)}\\
		&= \frac{\mu_i - r}{\lambda_{\ast}(a-2rb+r^2c)} = \frac{\mu_i -r}{\mu_M - r} \quad \mit \alpha_M := \mu_M - r
	\end{align*}
	d.h. es gilt CAPM-Gleichung
	\begin{align*}
		\beta_i(\mu_M -r) = (\mu_i-r) \quad \forall i \in [n]
	\end{align*}
	$\beta_i$ ist Beta-Koeffizient von $S^i$, $(\mu_M -r)$ ist Überschussrendite Marktportfolio, $(\mu_i - r)$ ist Überschussrendite von $S^i$ (alpha)
	\begin{itemize}
		\item Kann als Regressionsgleichung für $(\alpha_i, \beta_i)_{i \in [n]}$ interpretiert werden
		\item Entscheidend für Attraktivität eines Wertpapieres $S^i$ ist nicht die Überrendite $\alpha_i = \mu_i -r$ \emph{alleine}, sondern \emph{in Relation} zu $\beta_i$
		\item CAPM kann empirisch überprüft werden durch Schätzung $(\hat{\alpha}_i,\hat{\beta}_1)$ und Regression
		\begin{align*}
			\hat{\beta}_i\cdot (\mu_M - r) = \hat{\alpha}_i + \epsilon_i \tag{$\ast$}\label{eq_4_4_beta}
		\end{align*}
		Ideal, wenn $\sum_{i=1}^n \epsilon_i^2$ klein ist
		\item sketch, see phone ...
	\end{itemize}
	\item Kritik am CAPM:
	\begin{itemize}
		\item Regression \eqref{eq_4_4_beta} empirisch im Allgemeinen nicht besonders gut (Fehler $\sum \epsilon_i^2$ groß)
		\item Schätzung von $\mu_i, \mu_M$ schwierig
	\end{itemize}
	\item Erweiterungen:
	\begin{itemize}
		\item Ergänze Schätzer $\hat{\mu}_i \und \hat{\mu}_M$ von Expertenmeinungen und Konfidenzaussagen führt zum \person{Black}-\person{Littermann}-Modell
		\item Erweitere Regressionsgleichung \eqref{eq_4_4_beta} um weitere Variablen und führt zum Beispiel zum \person{Fame}-\person{French}-Modell 
	\end{itemize}
\end{itemize}
\section{Präferenzordnungen und Erwartungsnutzen}
\begin{itemize}
	\item Kritik an Markowitz:
	\begin{itemize}
		\item Ist Standardabweichung $\sqrt{\Var(R)}$ nicht unbedingt gutes Risikomass
		\item Entscheidungen unter Unsicherheit meist komplexer als durch Erwartungs-Varianzprinzip beschrieben
	\end{itemize}
	\item Axiomatischer Zugang: Präferenzordnungen\\
	Sei $(\O,\F,\P)$ Wahrscheinlichkeitsraum, $L_1(\O,\F,\P)$ Raum der integrierbaren Zufallsvariablen.
	\begin{align*}
		\MM = \set{\text{Menge der Verteilungsfunktionen} F_X \text{ von } X \in L_1(\O,\F,\P)}
	\end{align*}
	Seien $X,Y \in L_1(\O)$ Interpretation risikobehafteter Auszahlungen (``\begriff{Lotterie}''). Wir wollen Ordnungsrelation ``$\vartrianglelefteq$'' mit Bedeutung:
	\begin{align*}
		X \vartrianglelefteq Y \Leftrightarrow Y \text{ wird bevorzugt gegenüber }X
	\end{align*}
	Beschränken uns auf ``verteilungsinvariante'' POs welche durch Relation auf $\MM$ erklärt werden können, d.h.
	\begin{align*}
		X \vartrianglelefteq Y \Leftrightarrow F_X \vartrianglelefteq F_Y
	\end{align*}
\end{itemize}
\begin{definition}
	Eine Relation ``$\vartrianglelefteq$'' auf $\MM$ heißt Praferenzordnung (PO), wenn gilt:
	\begin{itemize}
		\item (Reflexiv) $F \vartrianglelefteq F\quad \forall F \in \MM$
		\item (Transitiv) $(F \vartrianglelefteq G)\vee (G \vartrianglelefteq H) \implies (F \vartrianglelefteq H)\quad \forall F,G \in \MM$
		\item (Vollständig) $\forall F,G \in \MM$ gilt: $(F\vartrianglelefteq G)\wedge (G \vartrianglelefteq F)$
	\end{itemize}
\end{definition}
\begin{*remark}
	\begin{itemize}
		\item Menge $\MM$ ist \emph{konvex}, d.h. $\forall F,G \in \MM  \und \alpha \in [0,1]$ gilt
		\begin{align*}
			H - (1-\alpha)F + \alpha G \in \MM \tag{$\oplus$}\label{eq_4_5_convex_PO}
		\end{align*}
		\item \eqref{eq_4_5_convex_PO} lässt sich als ``Mischen'' von $F$ und $G$ interpretieren
		\item Sei $X \sim F_X, Y \sim F_Y$ und $A \upmodels (X,Y)$ mit $\P(A=0) = \alpha, \P(A=1) = 1-\alpha$. Dann gilt:
		\begin{align*}
			(1-A)X + A Y \sim (1-\alpha)F_X + \alpha F_Y
		\end{align*}
		\item Aus gegebenen PO können wir ableiten
		\begin{itemize}
			\item Äquivalenzrelation $F\sim G \Leftrightarrow (F \vartrianglelefteq G)\vee (G \vartrianglelefteq F)$ ``Indifferenz zwischen $F$ und $G$''
			\item strikte Relation $F \lhd G \Leftrightarrow (F \vartrianglelefteq G)\vee (G \vartrianglelefteq F)$ ``$G$ wird strikt gegenüber $F$ bevorzugt''
			\item Für ``deterministische Zufallsvariablen'' $a \in \R$ ist die Verteilungsfunktion $F_a = \indi_{[a, \infty)}$
		\end{itemize}
	\end{itemize}
\end{*remark}
Eine PO kann folgende Eigenschaften besitzen:
\begin{enumerate}
	\item Monotonie: $\forall a,b \in \R$ mit $a\le b$ gilt $F_a \vartrianglelefteq F_b$ (``mehr besser als weniger'')
	\item Risikoaversion: $\forall X \in L_1(X)$ gilt: $F_X F_{\E[X]}$ (``sicher besser als unsicher'')
	\item Mittelwertseigenschaft: Sei $F,G,H \in \MM$ mit $F \vartrianglelefteq G \vartrianglelefteq H$. Dann existiert $\alpha \in [0,1]$ mit $(a-\alpha)F + \alpha H \sim G$
	\item Unabhängigkeitsaxiom: $\forall F,G,H \in \MM$ gilt
	\begin{align*}
		F \vartrianglelefteq G \implies (1-\alpha)F + \alpha H \vartrianglelefteq (1-\alpha) + \alpha H \quad \forall \alpha \in [0,1]
	\end{align*} 
\end{enumerate}