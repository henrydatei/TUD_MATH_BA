\section{Das Anlageproblem}
\emph{Gegeben:} Vermögen $W$, Anlagegüter $S^1, \dots, S^n$ (Aktien, Anleihen, ...)\\
\emph{Gesucht:} Optimale Verteilung $W = W_1 + ... + W_n$ auf $S^1 ... S^n$\\
$S^1 ... S^n$ weisen unterschiedliche Beträge, Risiken und typischerweise  Korrelationen auf.\\
\emph{Wir unterscheiden:}
\begin{itemize}
	\item \begriff{Einperiodenproblem}: Aufteilung wird heute $(t=0)$ festgelegt und bis zum Zeithorizont ($t=T$) beibehalten
	\item \begriff{Mehrperiodenproblem:} Umschichten zu mehreren Zeitpunkten $\set{t_0, t_1, \dots, t_N}$ möglich
\end{itemize}
\emph{Einfachstes Optimalitätsprinzip}: \begriff{\person{Pareto}-Optimalität}
\begin{itemize}
	\item Bei gleichem Risiko wird Anlage mit größeren Ertrag bevorzugt
	\item Bei gleichem Ertrag wird Anlage mit kleineren Risiko bevorzugt
\end{itemize}
d.h. \begriff{Pareto-Optimal} bedeutet, es gibt keine Anlagestrategie mit größerem Ertrag und kleinerem Risiko.\\
Zum Aufwärmen zwei Toy-Models ($=$ stark vereinfachte Beispiele)
\begin{itemize} %TODO fix structure 
	\item \emph{Toy-Model I}; Einperiodenmodell, eine risikofreie und eine risikobehaftete Anlagenmodel.
	\begin{itemize}
		\item Zeithorizont sei $T=1$
		\item risikofrei: $S_0^0 = 1$, $S_T^0 = (1+r)$
		\item risikobehaftet: $S_0^1 = 1$, $S_T^1 = (1+R)$ mit $R$ stochastisch,
		\begin{align*}
			\mu &= \E[R] \quad \text{Ertrag}\\
			\sigma &= \sqrt{\Var(R)} \quad \text{Risiko}
		\end{align*}
		\item $S=\mu -r$ Überrendite (excess return).
		\item $S \le 0 \implies$ Investiere alles in $S^0$ (Pareto-Optimal)
		\item $S > 0 \implies ?$
		\item Teile $W$ in $(W_0, W-W_0)$ auf $(S^0,S^1)$ auf (Jetzt: $W=1$)
		\begin{align*}
			\begin{cases}
				W-W_0 < 0 &\quad \text{Leerverkauf}\\
				W_0 < 0 &\quad \text{Kredit}
			\end{cases}
		\end{align*}
		\item Endvermögen: $P_T = W_0(1+r) + (1-W_0)(1+R)$
		\item Erweiterte Rendite: 
		\begin{align*}
			\mu_p &= \E[P_T-1] = W_0(1+r)+(W-W_0)(1+\mu)\\
			&= W_0 r + (1-W_0)\mu\\
		\end{align*}
		\item Risiko: $\sigma_p &= (1-W_0)\sigma$
		\item Überrendite $S_p = (1-W_0)(\mu -r)$
		\item \emph{Jede} Strategie ist Pareto-Optimal, d.h. Pareto-Prinzip  hilft nicht bei der Auswahl. Insbesondere ist \begriff{Sharp-Ratio} 
		\begin{align*}
			SR(W_0) = \frac{\text{``Überrendite''}}{\text{``Risiko''}} = \frac{S_p}{\sigma_p} = \frac{\mu -r}{\sigma} \quad \text{konstant!}
		\end{align*}
	\end{itemize}
	\item \emph{Alternative} zum Pareto-Prinzip Festlegen von individueller Risikoaversion (mehr dazu später)
	\item \emph{Toy-Model II:} Einperiodenproblem, zwei risikobehaftete Anlagemöglichkeiten
	\begin{itemize}
		\item Zeithorizont $T=1$, Vermögen $W=1$
		\item 
		\begin{align*}
			\begin{matrix}
				S_0^1 = 1 & S_T^1 = (1+R_1) & \text{mit $\E[R_1] = \mu$, $\Var(R_1) = \sigma_1^2 > 0$}\\
				S_0^2 = 1 & S_T^2 = (1+R_1) & \text{mit $\E[R_2] = \mu$, $\Var(R_2) = \sigma_2^2 > 0$}
			\end{matrix}
		\end{align*}
		und $R_1 \upmodels R_2$
		\item Portofoliowert: $P_T = W_1(1+R_1) + (1-W_1)(1+R_2)$
		\item Rendite: $\mu_p = \E[P_T -1] = W_1\E[R_1] + (1-W_1)\E[R_2] = \mu$
		\item Risiko:
		\begin{align*}
			\sigma_p^2 = \Var(P_T -1) &= \Var(W_1 R_1 + (1-W_1)R_2)\\
			&= W^2_1\cdot \sigma_1^2 + (1-W_1)^2\sigma_2^2
		\end{align*}
		\item Pareto-Optimales Portofolio: 
		\begin{align*}
			0 &= 2W_1\cdot \sigma_1^2 - 2(1-W_1)\sigma_2^2
			\sigma_2^2 &= W_1(\sigma_1^2 + \sigma_2^2)\\
			&\implies W_{\ast} = \frac{\sigma_2^2}{\sigma_1^2\cdot \sigma_2^2} \in (0,1)
		\end{align*}
		Also existiert genau eine Pareto-Optimale Strategie
		\item Vermögen wird proportinal zum Verhältnis der Risiken aufgeteilt
		\item Vermögen wird \emph{nicht} vollständig in risikoärmere Anlagen gesteckt, also findet eine \begriff{Diversifikation} statt
		\item $W_{\ast}$ ist auch die Strategie mit maximaler \emph{Sharp-Ratio}
	\end{itemize}
\end{itemize}
Als nächstes: Pareto-Optimale Portofolio mit $n>2$ Anlagegütern
\section{Exkurs: Optimierung mit Nebenbedingung}
\emph{Betrachte Optimierungsproblem:}
\begin{align*}
	\min f_0(x)\quad x \in \R^n\\
	\intertext{unter Nebenbedingungen}
	\begin{cases}
		f_i(x) \le 0 &\quad i = 1, ..., m\\
		h_i(x) = 0 &\quad i = 1, ..., p
	\end{cases}
	\tag{OPT}\label{eq_4_1_opt}
\end{align*}
\begin{itemize}
	\item $x \in \R^n$ welches (NB) erfüllt heißt \begriff{zulässig}
	\item $x_{\ast} \in \R^n$ welches \eqref{eq_4_1_opt} normiert heißt \begriff{Optimallösung}
	\item $p_{\ast} = f_0(x_0)$ heißt \begriff{Minimalwert}
\end{itemize}
\begin{definition}
	\begin{enumerate}
		\item Die Funktion
		\begin{align*}
		\LL(x, \lambda, \nu) = f_0(x) + \sum_{i=1}^{m}\f_i(x)\lambda_i + \sum_{i=1}^p h_i(x)\nu_i
		\end{align*}
		mit $\lambda \in \R^m_{\ge 0}, \nu \in \R^p$ heißt \begriff{Lagrange-Zielfunktion}
		\item Die Funktion
		\begin{align*}
			g(\lambda, \nu) = \inf_{x \in \R^m} \LL(x, \lambda, \nu)
		\end{align*}
		heißt Langrange- duale Funktion für \eqref{eq_4_1_opt}.
	\end{enumerate}
\end{definition}
\begin{*remark}
	Als Infimum von $(m, \lambda, \nu)$ lineare Funktionen ist $g$ \emph{konkav}.
	\begin{itemize}
		\item Die duale Funktion $g$ erzeugt \emph{untere Schranken} für $p_{\ast}$
		\item Begründung Sei $\overline{x} \in \R^n$ zuässig für \eqref{eq_4_1_opt}, d.h.
		\begin{align*}
			f_i(\overline{x}) &\le 0 \quad \forall i \in [n]\\
			h_i(\overline{x}) &= 0 \quad \forall i \in [p]
		\end{align*}
		\begin{align*}
			\implies \LL(\overline{x}, \lambda, \nu) = f_0(\overline{x}) + \underbrace{\sum_{i=1} f_i(\overline{x})\lambda_i + \sum_{i=1} h_i(\overline{x})\nu_i}_{=0} \le f_0(\overline{x})
		\end{align*}
		Also $g(\lambda, \nu) = \inf_{x \in \R^m} \LL(x, \lambda, \nu) \le \LL(\overline{x}, \lambda, \nu) \le f_0(\overline{x}) \quad \forall \overline{x}$ zulässig und damit folgt
		\begin{align*}
			g(\lambda, \nu) \le p_{\ast} \qu \forall \lambda \in \R^m_{>0}, \nu \in \Rn
		\end{align*}
		Die beste untere Schranke erhalten wir durch \emph{maximieren} über $\lambda, \nu$
	\end{itemize}
\end{*remark}
\begin{definition}
	Das duale Optimierungsporblem zu \eqref{eq_4_1_opt} ist
	\begin{align*}
		\max g(\lambda, \nu) \quad \lambda \in \R^m, \nu \in \R^p
		\intertext{unter Nebenbedingungen}
		\lambda_i \ge 0 \quad i = 1, ... m \tag{D}\label{eq_4_2_Duality}
	\end{align*}
	Maximalwert: $d_{\ast}$
\end{definition}
\begin{itemize}
	\item Zwischen \eqref{eq_4_1_opt} und \eqref{eq_4_2_Duality} gilt \begriff{schwache Dualität}
	\begin{align*}
		d_{\ast} \le p_{\ast}
	\end{align*}
	\item Unter bestimmten Vorraussetzungen gilt auch die \begriff{starke Dualität}
	\begin{align*}
		d_{\ast} = p_{\ast}
	\end{align*}
\end{itemize}