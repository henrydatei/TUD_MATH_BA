\chapter{Einführung}
\section{Zentrale Fragestellung der Finanzmathematik}
\subsection*{\begriff{Bewertung}:}
Bewertung von Derivaten und \emph{Absicherung} gegen aus Kauf/Verkauf entstehenden Risiken.

\begin{*definition}[\begriff{Derivat}]
	FInanzprodukt, dessen auszahlungen sich vom Preis einer oder mehrer \begriff{Basisgüter} (underlying) ableitet (ableiten entspricht derivate)
\end{*definition}
\begin{*example}
	\begin{itemize}
		\item Recht, in 3 Monaten 100.000 GBP gegen 125.000 EUR zu erhalten (\begriff{Call-Option}, Underlying: Wechselkurs GBP/EUR)
		\item Recht, innerhalb des nächsten Jahres 100.000 Mwh elektrischer Energie zum Preis von 30EUR/Mwh zu konsumieren mit Mindestabnahme 50.000 Mwh (\begriff{Swing-Option}, Underlying: Strompreis)
		\item Kauf- und Verkaufsoptionen aus Aktien (Underlying: Aktienkurs)
	\end{itemize}
\end{*example}
Fragestellung: Was ist der ``faire'' Preis für solch ein Derivat? (``Pricing''/Bewertung). Wie kann sicher der Verkäufer gegen eingegangenen Risiken absichern? (``Hedging''/Absicherung)
\subsection*{\begriff{Optimale Investition}}
Zusammenstellung von Portofolios, welche nach Risiken/Ertragsgesichtspunkten optimal sind
\begin{itemize}
	\item Wie wäge ich Risiken gegen Ertrag ab?
	\item Was genau bedeutet ``optimal''?
	\item Lösung des resultierenden Optimierungsproblems
\end{itemize}
\subsection*{\begriff{Risikomangement + Risikomessung}}
\begin{itemize}
	\item Gesetzliche Vorschriften (Basel + Solvency) sollen Stabilität des Banken-/Verischerunssystems auch angesichts verschiedener Risiken sicherstellen $\implies$ mathematische Theorie der konvexen und kohärenten Risikomaße
\end{itemize}
Mathematische Werkzeuge: Wtheorie + stochastische Prozesse (Dynamik in der Zeit), etwas lineare Algebra, Optimierung, Maßtheorie
\section{Mathematisches Finanzmodell}
Wir betrachten
\begin{enumerate}
	\item \emph{WRaum} $(\O,\F,\P)$, später auch weitere W-Maße $Q, \dots$ auf demselben Maßraum $(\O,\F), \omega \in \O$ Elementarereignisse bzw. ``Szenarien''
	\item \emph{Zeitachse} $I$ entweder $I=\set{t_0, t_1, \dots, t_N=T}$ $N$-Periode Modell (diskretes Modell) oder $I = [0,T]$ (zeitstetiges Modell), wobei $T = $ Zeithorizont\\
	Ein \begriff{stochasticher Prozess} $S$ ist eine messbare Abbildung $S: (\O,\F) \to \Rd \mit (\omega, t) \mapsto S_t(\omega)$\\
	insbesondere ist 
	\begin{itemize}
		\item $t \mapsto S_t(\omega)$ Funktion $I \to \Rd$ für jedes $\omega \in \O$ (``Pfad'')
		\item $\omega \mapsto S_t(\omega)$ Zufallsvariable $\O \to \Rd$ für jedes $t \in I$
	\end{itemize}
	\item \emph{Filtration} ist Folge von $\omega$-Algebren $(\F_t)_{t \in I}$ mit der Eigenschaft $\F_S \subseteq \F_t \quad \forall s,t \in I, x \le t \und \F_t \subseteq \F\quad \forall t \in I$\\
	Interpretation: $\F_t=$dem Marktteilnehmer zum Zeitpunkt $t$ bekannte/ verfügbare Informationen\\
	Ereignisse $A \in \F_t$ gelten als ``zum Zeitpunkt $t$'' bekannt\\
	Eine $\Rd$-wertige ZV $X$ heißt \begriff{$\F_t$-messbar}, wenn $E = X^{-1}(B) \in \F_t \quad \forall$ Borelmengen $B \subseteq \Rd$ (dabei ist $E$ das Urbild von $B$)

\begin{*example}
	Ein stochastischer Prozess $(S_t)_{t\in I}$ auf $(\O,\F)$ heißt \begriff[stochasticher Prozess]{adaptiert} bezüglich einer Filtration $(\F_t)_{t \in I}$, wenn gilt:
	\begin{align*}
		S_t \text{ ist } \F_t-\text{messbar} \quad \forall t \in I
	\end{align*}
\end{*example}
Interpretation: ``der Wert $S_t$ ist zum Zeitpunkt $t$ bekannt''\\
Warum Filtration in der Finanzmathematik (FiMa)?
\begin{itemize}
	\item Unterscheidung Zukunft / Vergangenheit
	\item unterschiedliche Information (Insider/Outsider) entspricht unterschiedlicher Filtration $(\F_t)_{t \in I}$ bzw. $(\G_t)_{t\in I}$
\end{itemize}
	\item \begriff{Anlagegüter (assets)} $\R^{d+1}$-wertiger stochastischer Prozess mit Komponenten
	\begin{align*}
		S^i: (\O \times I) \to \R\quad (\omega,t) \mapsto S^i_t(\omega) \mit i \in \set{0,1,\dots,d}
	\end{align*} 
	wobei $S^i_t=$ Preis des $i$-ten Anlageguts zum Zeitpunkt $t$\\
	$S^i, i \in \set{1,\dots,d}$ ist typischerweise
	\begin{itemize}
		\item Aktie (Stock), Unternehmensanteil
		\item Währung (currency) bzw. Wechselkurs
		\item Rohstoff (commodity) wie z.B. Öl, Edelmetall, Elektriziät, etc
		\item Anleihe (bond) ... Schuldverschreibung
	\end{itemize}
	Hauptannahme: $S^i$ ist liquide gehandelt (z.B. an Börse), d.h. Kauf/Verkauf zum Preis $S^i_t$ jederzeit möglich\\
	$S^0\dots$ ``Numeraire'' hat Sonderrolle: beschreibt Verzinsung von \emph{nicht} in $(S^1,\dots,S^d)$ angelegten Kapital, wird meist \emph{risikolos} betrachtet
\end{enumerate}
\begin{definition}[Finanzmodell]
	Ein \begriff{Finanzmodell} (FMM) mit Zeitachse $I$ ist gegeben durch
	\begin{enumerate}
		\item einen WRaum $(\O, \F,\P)$ mit Filtration $(\F_t)_{t\in I}$
		\item einen an $(\F_t)_{t \in I}$ adaptieren, $\R^{d+1}$-wertigen stochastischen Prozess $S_t = (S^0_t, S_t^1, \dots, S^d_t),t \in I$
	\end{enumerate} 
\end{definition}
\begin{*example}[\person{Cox}-\person{Rubinstein} (CRR)-Modell (zeitdiskret)]
	\begin{itemize}
		\item $S^0_n = (1+r)^n$, d.h. Verzinsung mit konstanter Rate $r$
		\item $S^1_n = S_0^1 \prod_{k=1}^n(1+Ru)$, wobei $(R_1, R_2, \dots)$ unabhängig ZVen mit zwei möglichen Werten $a < b$\\
		Bild: ``rekonbinierter Baum'' mit Ereignissen $\omega$ entsprechen Pfaden in dem Baum
	\end{itemize}
\end{*example}
\begin{*example}[\person{Block}-\person{Scholes}-Modell (zeitstetig)]
	\begin{itemize}
		\item $S^0_t = e^{rt}$, d.h. Verzinsung mit konstanter Rate $r$
		\item $S_t^1 = S_0^1\cdot \exp((\mu - \frac{\sigma^2}{2}t + \sigma\beta_t) \mit \mu \in \R, \sigma > 0, S^1_0 >0$ und $\beta_t$ entspricht Brownscher Bewegung (stochastischer Prozess in stetiger Zeit) und $\mu - \frac{\omega^2}{2}$ entspricht Trendkomponente
	\end{itemize}
	Bild: Börsenkurve = $S_t(\omega)$, wobei zeitstetiges Modell auf unendlichen W-Raum
\end{*example}
\section{Anleihen und grundeliegende Beispiele für Derivate}
Hier betrachten wir immer nur ein Basisgut $S_t = S^1_t$
\begin{enumerate}
	\item \begriff{Anleihe}(\begriff{bond}): (genauer: Null-Coupon-Anleihe [zero-coupon-bond]) Der \begriff{Emittent} (Herausgeber) einer Anleihe mit Endfälligkeit $T$ [maturity] garantiert dem Käufer zum Zeitpunkt $T$ den Betrag $N$ (EUR/USD/...) zu zahlen.\\
	Typische Emittenten:
	\begin{itemize}
		\item Staaten [government bond]
		\item Unternehmen (als Alternative zur Kreditaufnahme)
	\end{itemize}
	Nach Emission werden Anleihen auf den Sekundärmarkt weiterverkauft, d.h. liquide gehandelte Wertpapiere\\
	Preis bei Emission: $B(0,T)$\\
	Preis bei Weiterverkauf zum Zeitpunkt $t \le T\colon B(t,T)$\\
	Wir normieren stets $N=1 \implies B(T,T) =1$\\
	Anleihen von West/Nord/Mitteleuropäischen Staaten + USA/Kananda werden als risikolos betrachtet (sichere Zahlung).\\
	Sonst: Kreditrisiko\\
	Risikofreie Anleihen können als Numerale $S^0_t = B(t,T)$ genutzt werden\\
	Bild: kann ich gerade nicht beschreiben :/ \\
	\item \begriff{Terminvertrag} [forward contract]\\
	Aus Käufersicht: \emph{Vereinbarung} zu bestimmten, zukünftigen Zeitpunkt $T$ eine Einheit des Basisguts $S$ zum Preis zu kaufen (Kaufverpflichtung)\\
	Beliebt bei Rohstoffen + Elektrizität\\
	Auszahlunsprofil: $F_T = S_T - K$\\
	Bild: ``Eine Gerade mit Schnittpunkt der $x$-Achse bei $K$ und Schnittpkt der $y$-Achse bei $S_T \ge 0$, ist ja nur einer Polynom 1. Ordnung''\\
	Preis zum Zeitpunkt $t$: $F_t$
	\item \begriff{Europäische Put-/Call-Option}:
	Recht zu einem zukünftigen Zeitpunkt $T$ eine Einheit des Basisguts $S$ zum Preis $K$ zu verkaufen (Put) bzw. zu kaufen (Call) \textbf{keine (Ver-)Kaufsverpflichtung}\\
	\begin{itemize}
		\item \emph{Call}:
		\begin{align*}
			C_T := \begin{cases}
				S_T - K &\quad S_T \ge K\\
				0 &\quad S_T < K
			\end{cases} = (S_T - K)_+ %is this really a ``+'' or a t?!
		\end{align*}
		\begin{*remark}
			\begin{align*}
				X_+ &= \max(X,0)\quad X_+ - X__ = X\\
				X__ &= \min(X,0)\quad X_+ + X__ = \abs{X}
			\end{align*}
			Bild: (hockey stick function)
		\end{*remark}
		\item \emph{Put}:
			\begin{align*}
				P_t = \begin{cases}
					0 &\quad S_T \ge K\\
					K-S_T &\quad S_t < K
				\end{cases} = (K-S_T)_+
			\end{align*}
			Bild: ``inversed'' hockey stick function xD
	\end{itemize}
	\item \emph{Amerikanische Put/Call-Option}: Wie Put/Call aber mit Ausübung zu beliebigen Zeitpunkt $t \in [0,T]$\\
	Preis zum Zeitpunkt $t\colon P_t^{AM}, \; C_t^{AM}$\\
	Auszahlungsprofil zum zeitpunkt $\tau\colon (S_{\tau}-K), (K-S_{\tau})_+$\\
	Zeitpunkt $\tau$ muss im Allgemeinen als Lösung eines stochastischen Optimierungsproblems bestimmt werden (``\begriff{Optimales Stopproblem}'')
\end{enumerate}
\section{Elementare Replikations und Arbitrage-Argumente}
Was können wir (mit elementaren Mitteln) über die ``fairen'' Preise $B(t,T), F_t, C_t, P_t$ aussagen?\\
Wir verwenden:
\begin{itemize}
	\item \begriff{Replikationsprinzip}: Zwei identische zukünftige Zahlungsströme haben auch heute denselben Wert. (ein Zahlungstrom ``repliziert'' den anderen) % but what about inflation? Where comes inflation in this whole system in? 
	\item \begriff{No-Arbitage-Prinzip}: ``Ohne Kapiteleinsatz kann sicherer Gewinn ohne Verlustrisiko erzielt werden''
	\item \begriff{Arbitrage}: risikofreier Gewinn\\
	\item Schwächere Form des Replikationsprinzips:\\
	\begriff{Superpositionsprinzip}: Ist ein Zahlungsstrom in jedem Fall größer als ein anderer, so hat er auch heute den größeren Wert
	\begin{align*}
		\begin{matrix}
			\text{stark} & \text{Rep. Prinzip} & \text{eingeschränkt anwendbar}\\
			\downarrow & \text{Superrep. Prinzip} & \uparrow\\
			\text{schwach} & \text{No-Arbitrage-Prinzip} & \text{immer anwendbar}
		\end{matrix}
	\end{align*}
\end{itemize}
\begin{lemma} %1.1 fix numbering later ;)
	Für den preis $C_t$ des europäischen Calls gilt:
	\begin{align*}
		(S_t - K\cdot B(t,T))_+ \le C_t \le S_t
	\end{align*}
\end{lemma}
\begin{proof}
	\begin{itemize}
		\item \emph{untere Schranke}: Für Widerspruch $S_t - K\cdot(B(t,T))-C_t = \epsilon > 0$\\
		\begin{tabular}{l|l|l|l} % jeezzz i fuckin hate tables in latex xD, but i managed it ...
			Portofolio & Wert in $t$ & Wert in $T$, $S_t \le K$ & Wert in $T$, $S_t > K$\\
			Kaufe Call & $C_t$ & 0 & $S_T - K$\\
			Verkaufe Basisgut & $-S_t$ & $-S_T$ & $-S_T$\\
			Kaufe Anleihe & $\epsilon + K\cdot B(t,T)$ & $\frac{\epsilon}{B(t,T)}+K$ & $\frac{\epsilon}{B(t,T)} + K$\\
			$\Sigma$ & 0 & $K - S_T + \frac{\epsilon}{B(t,T)} > 0$ & $\frac{\epsilon}{B(t,T)} > 0$\\
			& keine Anfangskapital & sicherer Gewinn & sicherer Gewinn\\
		\end{tabular}\\
		$\implies$ Widerspruch zu No-Arbitrage\\
		$\implies$ $S_t - K\cdot B(t,T) \le C_t$ und Ausserdem $C_t \ge 0 \implies C_t \ge (S_t - K\cdot B(t,T))_+$
		\item \emph{obere Schranke}: UE
	\end{itemize}
\end{proof}
\begin{lemma}[Put-Call-Parität] % jeezz its 5:48 am im typing this, cant sleep ... :(
	Für Put $P_t$, Call $C_t$ mit demselben Ausübungspreis $K$ und Basisgut $S_t$ gilt
	\begin{align*}
		C_t - P_t = S_t - B(t,T)K
	\end{align*}
	Bild: need to add ..., but should be fast to do ...
\end{lemma}
\begin{proof} %TODO fix tables later, for now it works ..., make them one size for better reading ...
	mit Replikation:\\
	\begin{tabular}{l|l|l|l} % jeezzz i fuckin know now how to make tables in latex xD
		Portofolio 1 & Wert in $t$ & Wert in $T$, $S_t \le K$ & Wert in $T$, $S_t > K$\\
		Kaufe Call & $C_t$ & 0 & $S_T - K$\\
		Kaufe Anleihe & $K \cdot B(t,T)$ & $K$ & $K$\\
		Wert Portofolio 1 & $C_t + K\cdot B(t,T)$ & $K$ & $S_T$\\
	\end{tabular}\\
	\newline
	\begin{tabular}{l|l|l|l} % jeezzz i fuckin hate, but i can copy ... xD
		Portofolio 2 & Wert in $t$ & Wert in $T$, $S_t \le K$ & Wert in $T$, $S_t > K$\\
		Kaufe Put & $P_t$ & $K-S_T$ & 0\\
		Kaufe Basisgut & $S_t$ & $S_T$ & $S_T$\\
		Wert Portofolio 2 & $P_t + S_t$ & $K$ & $S_T$\\
	\end{tabular}\\
	Replikationsprinzip: $C_t + K\cdot B(t,T) = P_t + S_t$\\
	$\implies$ $C_t - P_t = S_t - K\cdot B(t,T)$
\end{proof} % 5:56 am and im done with it, puh ;) hope we can talk about it in english if you like. ;)
\section{Bedingte Erwartungswerte und Martingale} %1.5
\subsection{Bedingte Dichte und bedingter Erwartungswert}
Motivation: Gegeben: Zwei ZVen $(X,Y)$ mit Werten in $\R^m \times \R^n$ und gemeinsame Dichte $f_{XY}(x,y)$. Aus $f_{XY}$ können wir ableiten:
\begin{itemize}
	\item $f_{Y}(y) := \int_{\R^m} f_{XY}(x,y) \d x$ mit Randverteilung von $Y$
	\item $S_Y := \set{y \in \Rn \colon f_Y(y) > 0}$ Träger von $Y$ - Bild?
\end{itemize}
\begin{*definition}[Bedingte Dichte von $X$ bezüglich $Y$]
	Bedingte Dichte von $X$ bezüglich $Y$ ist definiert als
	\begin{align*}
		f_{X\mid Y}(x,y) = \begin{cases}
		\frac{f_{XY}(x,y)}{f_Y(y)} &\quad y \in S_Y\\
		0 &\quad y\notin S_Y
		\end{cases}
	\end{align*}
\end{*definition}
Betrachte folgende Problemstellung:\\
Was ist die beste Vorhersage von $X$ gegeben einer Beobachtung $Y = y$?\\
Kriterium:\\
Minimiere quadratischen Abstand/ zweite Moment/ $L_2$-Norm.\\
Vorhersage:\\
Messbare Funktion $g: \Rn \to \R^m \mit y \mapsto g(y)$, d.h,.
\begin{align*}
	\min\set{\E[(X-g(Y))]^2 \colon g \text{ messbar } \R^n \to \R^m} \tag{min-1}\label{eq_min_1}
\end{align*}
\begin{proposition} %1.3
	Wenn $(X,Y)$ eine gemeinsame Dichte besitzen mit $\E[\abs{X}^2] < \infty$ gilt, dann wird \eqref{eq_min_1} minimiert durch die bedingte Erwartung
	\begin{align*}
		g(y) = \E[X\mid Y=y] := \int_{\R^m} x f_{X\mid Y}(x,y)\d x
	\end{align*}
	(wobei $\E[X\mid Y=y]$ ``Erwartungswert von $X$ bedingt auf $Y=y$'')
\end{proposition}
Allgemeiner gilt:
\begin{theorem} %1.4
	Seien $(X,Y)$ ZVen mit gemeinsamer Dichte auf $\R^m \times \Rn$, $h: \R^m \to \R^n$ messbar mit $\E[h(X,y)^2]$. Dann wird das Minimierungsproblem
	\begin{align*}
		\min\set{\E[(h(X,Y) - g(y))^2]} \quad g\text{messbar von $\Rn$ nach $\R$}
		\intertext{gelöst durch}
		g(y) = \E[h(X,Y) \mid Y=y] = \int_{\R^m} h(X,Y)f_{X\mid Y}(x,y) \d x
	\end{align*}
\end{theorem}
\begin{proof}[nur Prop, Theorem analog, für $n=1$]
	Setze $g(y) = \int_{\R} f_{X\mid Y}(x,y) \d x$. Sei $p: \R \to \R$ beliebige messbare Funktion mit $\E[p(y)^2] < \infty$. Setze $g_{\epsilon}(y) = g(y) + \epsilon p(y)$. Minimiere
	\begin{align*}
		F(\epsilon) &:= \E[(X-g_{\epsilon}(y))^2] = \E[(X-g(y)-\epsilon p(y))^2]\\
		&= \E[(X-g(y))^2] - 2\epsilon\E[(X-g(y))p(y)] + \epsilon^2\E[p(y)^2]\\
		\frac{\partial F}{\partial \epsilon}(\epsilon) &= 2 \epsilon \E[p(y)^2] - 2\E[(X-g(y))p(y)]\\
		&\implies \epsilon_{\ast} :=\frac{\E[(X-g(y))p(y)]}{\E[p(y)^2]} = \frac{A}{B}
		\intertext{wobei}
		A&= \E[Xp(y)] -\E[g(y)p(y)] \\
		&= \int_{\R \times \Rn} xp(y)f_{XY}(x,y)\d x \d y - \int_{S_y}g(y)p(y)f_Y(y) = [\text{Einsetzen von $g$ + Fubini}]\\
		&= \int_{\R \times \Rn} xp(y)f_{XY}(x,y)\d x \d y - \int_{\R\times S_y} xp(y)\underbrace{f_{X\mid Y}(x,y)f_Y(y\d y)}_{=f_{XY}(x,y)} = 0
	\end{align*}
	also $\epsilon^{\ast} = 0$ unabhängig von $p$ $\implies g(y)$ minimiert \eqref{eq_min_1}.
\end{proof}
\begin{*example}
	Seien $(X,Y)$ normalverteilt auf $\R\times \R$ mit 
		\begin{align*}
			\mu = (\mu_x, \mu_y)^T \quad \Sigma = \begin{pmatrix}
			\sigma x^2 \rho\sigma_x \sigma_y\\
			\rho \sigma_x \sigma_y & \sigma_y^2
			\end{pmatrix} = \begin{pmatrix}
				\Var(X) & \Cov(X,Y)\\
				\Cov(X,Y) & \Var(Y)
			\end{pmatrix} \mit \rho \in [-1,1]
		\end{align*}
		Dann ist die beliebige Dichte $f_{X\mid Y}(x,y)$. ($\Sigma$ Kovarianzmatrix). wieder die Dichte einer Normalverteilung mit
		\begin{align*}
			\E[X \mid Y=y] &= \mu_x + \rho \frac{\sigma_x}{\sigma_y}(y-\mu_y)\\
			\Var(X\mid Y=y) &= \sigma_x^2(1-\rho^2)
		\end{align*}
		(ist ÜA!). Die Abbildung $y \mapsto \mu_x + g(y)\frac{\sigma_x}{\sigma_y}(y-\mu_y)$ heißt Regressionsgerade für $X$ gegeben $Y=y$.\\
		Bild: $\mu_x,\mu_y$ sind Werte auf $x,y$-Achse und die $\sigma$'s bilden das Steigungsdreieck (Steigung im Wesentlichen durch $\rho$ bekannt)\\
		Für diskrete ZVen, d.h. wenn $X,Y$ nur endlich viele $\set{x_1,\dots,x_m}$ bzw. $\set{y_1,\dots,y_m}$ annehmen dann erhalten wir mit ähnlichen Überlegungen als Lösung von \eqref{eq_min_1}
		\begin{align*}
			\E[X\mid Y=y_j] = \sum_{i=1}^m X_i \P(X=x_i \mid Y=y_j)
		\end{align*}
		wobei direkt die bedingten Wahrscheinlichkeiten
		\begin{align*}
			\P(X=x_i \mid Y=y_j) = \begin{cases}
			\frac{\P(X=x_i \wedge Y=y_j)}{\P(Y=y_j)} &\quad \text{ wenn } \P(Y=y_j) > 0\\
			0 &\quad \text{ wenn } \P(Y=y_j) = 0 
			\end{cases}
		\end{align*}
\end{*example}
\subsection{Bedingte Erwartung - maßtheoretischer Zugang}
Wir betrachten WRaum $(\O, \F,\P)$. Für ZV $X: \O \to \R$ und $p \in [1,\infty)$ definieren wir die $L_p$-Norm
\begin{align*}
	\norm{X}_p = \E[\abs{X}^p]^{1/p} = \brackets{\int_{\O} \abs{X(\omega)}^p \d \P(\omega)}^{1/p}
\end{align*}
und $L_p$-Raum $L_p(\O,\F,\P):= \set{X: \O \to \R\colon \F-\text{messbar}, \norm{X}_p < \infty}$. Dabei identifzieren wir ZVen, die sich nur auf Nullmengen unterscheiden, d.h. $\P(X \neq X') = 0 \implies X = X'$ (in $L_p$).\\
Aus Maßtheorie bekannt: (?)\\
Die Räume $L_p(\O,\F,\P)$ mit Norm $\norm{\cdot}_p, p \in [1,\infty)$ sind stets \person{Banach}-Räume (lineare, vollständig, normierte Vektorräume). Für $p = 2$ auch Hilbertraum mit inneren Produkt
\begin{align*}
	\scaProd{X}{Y} = \E[XY] = \int_{\O} X(\omega)Y(\omega)\d \P(\omega)
\end{align*}
Für $\G \subseteq \F$ Unter-$\sigma$-Algebra ist $L_p(\O,\F,\P) \subseteq L_p(\O,\F,\P)$ abgeschlossen Unterraum.\\
Wir verallgemeinern ``Vorhersageproblem'' aus dem letzten Abschnitt (1.3?)\\
Gegeben ZVe $X$ aus $L_2(\O,\F,\P)$ ist $\G \subseteq \F$ Unter-$\sigma$-Algebra.\\
Was ist die beste $\G$-messbare Vorhersage für $Y$?
\begin{align*}
	\min\set{\E[(X-G)^2] \colon G \in L_2(\O,\F,\P)} \tag{min-2}\label{eq_min_2}
\end{align*}
wobei $\E[(X-G)^2] = \norm{X-G}^2_2$.\\
Aus Hilbertraumtheorie:\\
\eqref{eq_min_2} besitzt eine eindeutige Lösung $G_{\ast} \in L_2(\F,\G,\P)$. $G_{\ast}$ ist Optimierung (bezüglich $\scaProd{\cdot}{\cdot}$) von $X \in L_2(\O,\F,\P)$ auf abgeschlossenen Unterraum $L_2(\O,\G,\P)$\\
Bild: eventuell von Eric (Orthogonal Projektion auf den Unterraum)\\
Wir bezeichnen mit $G_{\ast}$ mit $\E[X\mid \G]$ bedingte Erwartungswert von $X$ bezüglich $\G$.
\begin{theorem}[Eigenschaften bedingter Erwartungswert] %1.5
	\label{1_5_eigen_bedEW}
	Seien $X,Y \in L_2(\O,\F,\P)$ und $\G \subseteq F$ Unter-$\sigma$-Algebra. Dann gilt
	\begin{enumerate}
		\item (Linearität) $\E[aX+bY] = a\E[X\mid \G] + b\E[Y\mid \G]$
		\item (Turmregel) Für jede weitere $\sigma$-Algebra $\H \subseteq\G$ gilt
		\begin{align*}
			\E[E[X\mid \G \mid \H]] = \E[X\mid \H]
		\end{align*}
		\item (Pullout-Property) $\E[XZ\mid \G] = Z\E[X\mid \G]$, wenn $Z$ beschränkt und $\G$-messbar ist.\\
		zweite Version: Für $Z$ $\G$-messbar mit $\E[\abs{XZ}] < \infty$ gilt:
		\begin{align*}
			\E[XZ\mid \G] = Z \cdot \E[X\mid \G]
		\intertext{insbesondere gilt}
			X \G\text{-messbar }\implies \E[X\mid \G] = X
		\end{align*}
		\item (Monotonie) $X \le Y \implies \E[X\mid \G] \le \E[Y \mid \G]$
		\item ($\Delta$-Ungleichung) $\abs{\E[X\mid \G]} \le \E[\abs{X}\mid \G]$
		\item (Unabhängigkeit) $X$ unabhängig von $G$ $\implies$ $\E[X \mid \G] = \E[X]$
		\item (triviale $\sigma$-Algebra) $\G=\set{\emptyset, \O} \implies \E[X \mid \G] = \E[X]$ 
	\end{enumerate}
\end{theorem}
\begin{proof}
	(ohne Beweis, siehe VL W-Theorie mit Martingalen oder auch STOCH-Skript SS19.)
\end{proof}
\begin{*remark}
	\begin{itemize}
		\item Die für $X \in L_2(\O,\F,\P)$ definierte vedingte Erwartung $\E[X\mid \G]$ lässt sich durch Approximation auf alle $X\in L_1(\O,\F,\P)$ erweitern. Alle Eigenschaften aus Theorem \propref{1_5_eigen_bedEW} bleiben erhalten!
		\item Sei $Y$ eine ZVe und $\G = \sigma(Y)$ die von $Y$ erzeugte $\sigma$-Algebra. Wir schreiben:
		\begin{align*}
		\E[X\mid Y] = \E[X \mid \sigma(Y)] \quad \sigma\text{-messbare ZVe}
		\end{align*}
		\item Maßtheorie: \person{Doob}-\person{Dynkin}-Lemma $\implies \exists$ messbare Funktion $g: \Rn \to \R$ sodass
		\begin{align*}
		\E[X\mid Y] = g(Y)
		\end{align*}
		Dabei ist $g$ genau die Funktion aus \eqref{eq_min_1}.
	\end{itemize}
\end{*remark}
Zusammenfassung:\\
Sei $X,Y$ aus $L_1(\O,\F,\R)$, $\G \subseteq \F$ Unter-$\sigma$-Algebra
\begin{enumerate}
	\item $\E[X\mid Y=y]$ ist messbare Funktion $g: \Rn \to \Rn$. Falls bedingte Dichte existiert, gilt:
	\begin{align*}
		\E[X\mid Y=y] = \int_{\R^m} f_{X\mid Y} (x,y) \d x
	\end{align*}
	\item $\E[X\mid Y]$ ist eine $\sigma(y)$-messbare ZVe, diese kann als $g(Y)$ dargestellt werden. Falls bedingte Dichte existiert, gilt
	\begin{align*}
		\E[X\mid Y](\omega) = \int_{\Rn}xf_{X\mid Y}(x,Y(\omega))\d x
	\end{align*}
	\item $\E[X \mid \G]$ ist eine $\G$-messbare ZVe. Falls $\G = \sigma(y)$ tritt 2) ein.
\end{enumerate}
In allgemeinen Fall kann $\bar{\E[X\mid \cdot]}$ interpretiert werden als \emph{beste Vorhersage} für $X$, gegeben
\begin{enumerate}
	\item punktweise Beobachtung $Y=y$
	\item Beobachtung $Y$
	\item Information $\G$
\end{enumerate}
\subsection{Martingale}
Prototyp eines ``neutralen'' stochastischen Prozesses,der weder Aufwärts- noch Abwärtstrend besitzt. Hier nur in diskrete Zeit $Z = \N_0$.
\begin{*definition}[Martingal ohne Filtration]
	Sei $(X_n)_{n\in \N_0}$ stochastischer Prozess. Wenn gilt
	\begin{enumerate}
		\item $\E[\abs{X_n}] < \infty$ $\forall n \in \N$
		\item $\E[X_{n+1},\dots, X_n] = X_n$ $\forall n \in \N$
	\end{enumerate}
	dann heißt $(X_n)$ \begriff{Martingal}. Wen wir $\F_n^{\ast} = \sigma(X_1,\dots,X_n)$ definieren, können wir 2) schreiben als
	\begin{align*}
		\E[X_{n+1} \mid \F_n^{\ast}] = X_n \quad \forall n \in \N
	\end{align*}
\end{*definition}
Interpretation:\\
\begin{itemize}
	\item Beste Vorhersage für zukünftigen Wert $X_{n+1}$, basierend auf Vergangenheit $\sigma(X_1,\dots,X_n)$ ist der momentane Wert $X_n$.
	\item Aus der Turmregel folgt
	\begin{align*}
		\E[X_{n+k} \mid \F_n^{\ast}] &= X_n \quad n,k \in \N_0
		\intertext{denn}
		\E[X_{n+k}\mid\F_n^{\ast}] &= \E[\E[X_{n+k}\mid \F_{n+k-1}\mid \F_n^{\ast}]] = \E[X_{n+k-1}\mid \F_n^{\ast}] = (k\text{-mal}) = X_n
	\end{align*}
\end{itemize}
Kann von $(\F_{n})_{n \in \N}$ auf beliebige Filtrationen $(\F_n)_{n \in \N_0}$ erweitert werden.
\begin{*definition}[Martingal mit Filtration]
	Sei $(X_n)_{n \in \N_0}$ ein stochastischer Prozess, adaptiert an eine Filtration $(\F_n)_{n \in \N_0}$. Wenn gilt
	\begin{enumerate}
		\item $\E[\abs{X_n}] < \infty$ $\forall n \in \N_0$
		\item $\E[X_{n+1} \mid \F_n] = X_n$ $\forall n \in \N_0$
	\end{enumerate}
	dann heißt $(X_n)_{n \in \N_0}$ \begriff{Martingal bezüglich Filtration} $(\F_n)_{n \in \N_0}$
\end{*definition}
Interpretation:\\
Beste Vorhersage für zukünftige Werte $X_{n+1}$, basierend auf verfügbarer Information $\F_n$ ist momentane Wert $X_n$.
\begin{*definition}[Supermartingal, Submartingal]
	Falls in Punkt 2) statt ``$=$'' die Ungleichung $\le \oder \ge$ gilt, so heißt $(X_n)_{n \in \N}$ \begriff{Supermartingal} bzw. \begriff{Submartingal}.
\end{*definition}
Erste Beobachtung:\\
\begin{itemize}
	\item $X$ Martingal $\implies \E[X_n] = X_0$, d.h. $n \mapsto \E[X_n]$ ist konstant.\\
	Begründung:
	\begin{align*}
		\E[X_{n+1} \mid \F_n] = X_n \implies \E[\E[X_{n+1}\mid \F_n]] = \E[X_n] = \E[X_{n+1}] \implies (n\text{-mal Anwendung } \E[X_n] = X_0)
	\end{align*}
	Bild: Erwartungswert konstant, aber kein Martingal.
	\item $X$ Submartingal $\implies n \mapsto \E[X_n]$ ist monoton steigend
	\item $X$ Supermartingal $\implies n \mapsto \E[X_n]$ ist monoton fallend
\end{itemize}
Um sich den Unterschied zwischen Super- und Submartingal zu merken, hier eine kleine Hilfe:\\
``Das leben ist ein Supermartingal, die Erwartungen fallen mit der Zeit.''
\begin{*example}
	\begin{itemize}
		\item Seien $(Y_n)_{n\in \N}$ unabhängige ZVen in $L_1(\O,\F,\P)$ mit $\E[Y_n] = 0$. Definiere $X_n := \sum_{k=1}^n Y_k \mit X_0 = 0$. Dann ist $(X_n)_{n \in \N_0}$ Martingal, denn
		\begin{enumerate}
			\item $\E[\abs{X_n}] \le \sum_{k=1}^n \E[\abs{Y_k}] < \infty \quad \forall n \in \N$ \checkmark
			\item
			\begin{align*}
				\E[X_{n+1} \mid \F_n^{\ast}] &= \E[Y_{n+1} + X_n \mid \F_n^{\ast}]\\
				&= \E[Y_{n+1} \mid \F_n^{\ast}] = \E[X_n \mid \F_n^{\ast}] \quad (\text{ Turm und $\F_n^{\ast}$-messbar})\\
				&= \underbrace{\E[Y_{n+1}]}_{=0} + X_n = X_n \checkmark
			\end{align*}
		\end{enumerate}
		\item weitere Beispiele auf dem ersten Übungsblatt!
	\end{itemize}
\end{*example}
\begin{*definition}[vorhersehbar]
	Sei $(\F_n)_{n\in \N_0}$ eine Filtration. Ein stochastischer Prozess $(X_n)_{n \in \N}$ heißt \begriff{vorhersehbar} (predictable) bezüglich $(\F_n)_{n \in \N_0}$, wenn gilt:
	\begin{align*}
		H_n \text{ ist } \F_{n-1}\text{-messbar} \quad \forall n \in \N
	\end{align*}
\end{*definition}
\begin{*remark}
	Stärkere Eigenschaft als ``adaptiert''.
\end{*remark}
\begin{*definition}[diskretes stochastische Integral]
	Sei $X$ adaptierter und $H$ ein vorhersehbarer stochastischer Prozess bezüglich $(\F_n)_{n \in \N}$. Dann heißt
	\begin{align*}
		(H \bigcdot X)_n := \sum_{k=1}^n H_k (X_k - X_{k-1}) \tag{$\ast$}\label{eq_pred_stoch_process}
	\end{align*}
	\begriff{diskretes stochastische Integral} von $H$ bezüglich $X$.
\end{*definition}
\begin{*remark}
	Summe \eqref{eq_pred_stoch_process} heissen in der Analysis \person{Riemann}-\person{Stieltjes}-Summen. Werden für Konstruktionen des RS-Integrals $\int h \d \rho$ verwendet.
\end{*remark}
\begin{*definition}[lokal beschränkt]
	Ein stochastischer Prozess $(H_n)_{n \in \N}$ heißt \begriff{lokal beschränkt}, wenn eine (definierte) Folge $c_ \in \R_{\ge 0}$ existiert, sodass
	\begin{align*}
		\abs{H_n} \le c_n \text{ f.s. } \quad \forall n \in \N
	\end{align*}
\end{*definition}
\begin{theorem}
	Sei $X$ adaptiert stochastischer Prozess (bezüglich Filtration $(\F_n)_{n \in \N}$). Dann sind äquivalent:
	\begin{enumerate}
		\item $X$ ist Martingal
		\item $(H \bigcdot X)$ ist Martingale für alle lokal beschränkten, vorhersehbaren $(H_n)_{n \in N}$
	\end{enumerate}
	Das heisst: stochastische Integral erhält die Martingal-Eigenschaft.
\end{theorem}
\begin{proof}
	8.11.2019!
\end{proof}
\begin{*remark}
	Die ZV $H$ wird später die Analagestrategie sein.
\end{*remark}