\documentclass[ngerman,a4paper,order=firstname]{../../texmf/tex/latex/mathscript/mathscript}
%{local_mathscript}
%\usepackage{../../texmf/tex/latex/mathoperators/mathoperators} 
\usepackage{local_mathoperators}  % for ilina xD - local mathoperators

%local packages
\usepackage{cancel}
\usepackage{amssymb}

\title{\textbf{Stochastikvertiefung: Finanzmathematik WS 19/20}}
\author{Dozent: Prof. \person{Martin.Keller-Ressel}}

% local commands
%\renewcommand{\F}{\mathscr F}
\renewcommand{\P}{\mathbb{P}}
\newcommand{\QQ}{\mathbb{Q}}
\renewcommand{\O}{\Omega}
\renewcommand{\G}{\mathscr{G}}
\newcommand{\AAA}{\mathscr{A}}
\newcommand{\MM}{\mathcal M}
\newcommand{\BBB}{\mathscr{B}}
\renewcommand{\H}{\mathscr{H}}
\newcommand{\Gen}{\mathcal{E}}
\newcommand{\E}{\mathbb{E}}
\newcommand{\Var}{\mathbb{V}\text{ar}}   % Varianz
\newcommand{\Cov}{\mathbb{C}\text{ov}}   % Kovarianz
\newcommand{\Corr}{\mathbb{C}\text{orr}}   % Korrelation
%\newcommand{\Rd}{\R^d}
\def\upmodels{\perp\!\!\!\perp}
\newcommand{\distri}{\overset{\text{d}}{=}}
\newcommand{\konverteil}{\xrightarrow[n \to \infty]{\d}}
\newcommand{\normal}{\mathscr N} % Normal distribution
\newcommand{\bigcdot}{\boldsymbol{\cdot}} % predictable stoch process product
\DeclareMathOperator{\dist}{dist}
\DeclareMathOperator{\plim}{plim} % stochastic convergence or convengence in probability
\DeclareMathOperator{\CRR}{CRR}			
\newcommand{\half}{\frac{1}{2}}				% CRR-model sequence
\newcommand{\ZZ}{\mathcal Z }				% 
\newcommand{\NNN}{\mathcal N }				% normal distribution N
\newcommand{\OO}{\mathcal O }				% Big Ohhhh
\newcommand{\LL}{\mathcal L}				% Lagrange L

% % % % % % % % %
% from https://tex.stackexchange.com/questions/319330/notation-for-proper-normal-subgroup
\DeclareFontFamily{U}{matha}{\hyphenchar\font45}
\DeclareFontShape{U}{matha}{m}{n}{ <-6> matha5 <6-7> matha6 <7-8>
	matha7 <8-9> matha8 <9-10> matha9 <10-12> matha10 <12-> matha12 }{}
\DeclareSymbolFont{matha}{U}{matha}{m}{n}
%
\DeclareMathSymbol{\nvartrianglelefteq}{\mathrel}{matha}{"9E}
\DeclareMathSymbol{\vartrianglelefteq}{\mathrel}{matha}{"9C}

\begin{document}
\renewcommand{\F}{\mathscr F}
\pagenumbering{roman}
\pagestyle{plain}

\maketitle

\hypertarget{tocpage}{}
\tableofcontents
\bookmark[dest=tocpage,level=1]{Inhaltsverzeichnis}

\pagebreak
\pagenumbering{arabic}
\pagestyle{fancy}

\chapter*{Vorwort}
%Wir freuen uns, dass du unser Skript für die Vorlesung \textit{Geometrie} bei Prof. Dr. Arno Fehm im WS2018/19 gefunden hast. Da du ja offensichtlich seit einem Jahr Mathematik studierst, kannst du dich glücklich schätzen zu dem einen Drittel zu gehören, dass nicht bis zum zweiten Semester abgebrochen hat.

Wenn du schon das Vorwort zu \textit{Lineare Algebra und analytische Geometrie 1+2} gelesen hast, weißt du sicherlich, dass Prof. Fehm ein Freud der Algebra ist.\footnote{In Zukunft wird sich Prof. Fehm richtig freuen dürfen, denn im Zuge einer neuen Studienordnung, die am 1.4.2019 in Kraft tritt, kommt so gut wie keine Geometrie im \textit{Bachelor Mathematik} vor.} Auf die Frage eines Kommilitonen, wo in seinem Inhaltsverzeichnis (Gruppen, Ringe, Körper) die Geometrie vorkomme, antwortete er:
\begin{quote}
	\textit{Die Frage ist nicht, wieso wir in dieser Vorlesung Algebra statt Geometrie machen, sondern warum hier seit 20 Jahren Geometrie unterrichtet wird.}
\end{quote}

Wie auch im letzten Vorwort können wir dir nur empfehlen die Vorlesung immer zu besuchen, denn dieses Skript ist kein Ersatz dafür. Es soll aber ein Ersatz für deine unleserlichen und (hoffentlich nicht) unvollständigen Mitschriften sein und damit die Prüfungsvorbereitung einfacher machen. Im Gegensatz zu letztem Semester veröffentlicht Prof. Fehm auf seiner Homepage (\url{http://www.math.tu-dresden.de/~afehm/lehre.html}) kein vollständiges Skript mehr, sondern nur noch eine Zusammenfassung.

Der Quelltext dieses Skriptes ist bei Github (\url{https://github.com/henrydatei/TUD_MATH_BA}) gehostet; du kannst ihn dir herunterladen, anschauen, verändern, neu kompilieren, ... Auch wenn wir das Skript immer wieder durchlesen und Fehler beheben, können wir leider keine Garantie auf Richtigkeit geben. Wenn du Fehler finden solltest, wären wir froh, wenn du ein neues Issue auf Github erstellst und dort beschreibst, was falsch ist. Damit wird vielen (und besonders nachfolgenden) Studenten geholfen.

Und jetzt viel Spaß bei \textit{Geometrie}!

\begin{flushright}
	Henry, Pascal und Daniel
\end{flushright}

\chapter{Einführung}
\section{Wiederholung}
\begin{definition}[Halbgruppe, Monoid, Gruppe]
	Eine \begriff{Halbgruppe} ist eine Menge $G$ mit einem \emph{assoziativen} Produkt
	\begin{align*}
		\cdot\colon G \times G \to G.
	\end{align*}
	Ein \begriff{Monoid} ist eine Halbgruppe, in der ein Element $1 \in G$ existiert mit 
	\begin{align*}
		1\cdot x = x\cdot 1 = x \quad\forall x \in G.
	\end{align*}
	Eine \begriff{Gruppe} ist ein Monoid, in dem für jedes $x \in G$ ein $y \in G$ existiert mit
	\begin{align*}
		xy = yx = 1.
	\end{align*}
\end{definition}

\begin{remark}
		1 ist eindeutig, wenn es existiert. $y$ ist durch $x$ eindeutig bestimmt: $x^{-1} = y$.
\end{remark}

\begin{definition}[Morphismus]
	Ein Morphismus zwischen zwei Gruppen $G$ und $H$ ist eine Abbildung
	\begin{align*}
		f\colon G \to H \quad\text{mit}\quad f(xy) = f(x)f(y) \quad\forall x,y \in G.
	\end{align*}
\end{definition}

\begin{proposition}
	Ist $f\colon G \to H$ ein Morphismus von Gruppen, so gilt\begin{itemize}
		\item $f(1) = 1$ und
		\item $f(x^{-1}) = f(x)^{-1}$ $\forall x\in G$.
	\end{itemize}
\end{proposition}
\begin{proof}
	Für alle $x \in G$ gilt \begin{align*}
		f(x) &= f(1\cdot x) = f(1)f(x).
	\end{align*}
	Gilt in einer beliebigen Gruppe jedoch $ab=b$ für zwei Elemente $a,b$, so folgt
	\begin{align*}
		(ab)\cdot b^{-1} = a\big(bb^{-1}\big) = a\cdot 1 = a \quad\text{mit}\quad bb^{-1} = 1.
	\end{align*}
	Ferner gilt \begin{align*}
		f(x)\cdot f\big(x^{-1}\big) = f\big(x \cdot x^{-1}\big) = f(1) = 1
	\end{align*}
	wie schon gezeigt (und analog $f(x^{-1})f(x) = 1$). Also ist $f(x^{-1}) = f(x)^{-1}$).
\end{proof}

\begin{example}
	\begin{enumerate}[label={\arabic*)}]
		\item Sei $X$ eine beliebige Menge. $S_X = \set{f\colon X \to X \mid f \text{ bijektiv}}$ ist eine Gruppe bezüglich Komposition mit $1 = \id_X$. Insbesondere ist $S_n = S_{\set{1,\dots,n}}$ die \begriff{symmetrische Gruppe} und ein Element $f \in S_n$ ist eine \begriff{Permutation}.
		\item $\GL(V) = \set{f \in S_V \mid f \text{ linear}}$, wobei $V$ ein $R$-Modul ist mit kommutativen assoziativen Ring mit 1.
		\item $\Z, \Z_n$ unter Addition 
		\begin{align*}
			U_n = \Z^{\times}_n = \set{ m \in \set{0,\dots,n-1} \mid \ggT(m,n) = 1}
		\end{align*}
		Beide Gruppen sind abelsch, d.h. \begin{align*}
			\forall x,y \in G: xy = yx.
		\end{align*}
		\item $G = U(1) = \set{z \in C \mid \abs{z} = 1} = \set{e^{it}\mid t \in [0,2\pi]}$
		\item $G = U(1) \times \SU(2)\times \SU(3)$, die Eichgruppe im Standardmodell der Elementarteilchen  
	\end{enumerate}
\end{example}

\begin{definition}[Ordnung]
	Ist $G$ endlich, so nennt man $\abs{G}$ die \begriff{Ordnung} von $G$.
\end{definition}

\begin{example}
	$\abs{S_n} = n!$
\end{example}

\begin{definition}[$p$-Gruppe]
	Ist $\abs{G} = p^n$ für eine Primzahl $p$ und ein $n\in\mathbb N$, so nennt man $G$ eine \begriff{$p$-Gruppe}
\end{definition}

\begin{definition}[Untergruppe]
	Sei $G$ Gruppe. Eine Teilmenge $H \le G$ ist eine \begriff{Untergruppe} $H < G$, wenn
	\begin{defenum}
		\item \label{1_1_9_i} Für alle $x,y \in H$: $xy \in H$
		\item \label{1_1_9_ii} $1 \in H$
		\item \label{1_1_9_iii} Für alle $x \in H$: $x^{-1} \in H$ 
	\end{defenum}
\end{definition}

\begin{proposition}
	Ist $\abs{G} < \infty$, so folgen \cref{1_1_9_ii,1_1_9_iii} bereits aus \cref{1_1_9_i} und $H \neq \emptyset$.
\end{proposition}
\begin{proof}
	Sei $x \in H$ ein beliebiges Element. Aus \ref{1_1_9_i} folgt $x^n \in H$ für alle $n \in \N_{+}$. Da $\abs{G} < \infty$ existiert $n\neq m$ mit $x^n = x^m$. O.E. sei $n > m$ \begin{itemize}[label={$\Rightarrow$},topsep=-\parskip]
		 \item $x^{n-m} x^m = x^n$
		 \item $x^{n-m} = 1$
		 \item \ref{1_1_9_ii}
	\end{itemize}
	Ferner impliziert die Existenz der inversen Elemente, dass die Linkstranslation \begin{align*}
		t_x\colon G \to G, y \mapsto xy\quad(x\in G\;\text{fest})
	\end{align*}
	injektiv ist, denn $(t_x)^{-1} = t_{x^{-1}}$. Ist $x \in H$, so heißt \ref{1_1_9_i} gerade $t_x(H) \subseteq H$, sprich $t_x$ kann zu $t_x\big|_H\colon H \to H$ eingeschränkt werden. Die Einschränkung einer injektiven Abbildung ist injektiv. Da $\abs{H} \le \abs{G} < \infty$, folgt $t_x\big|_H\colon H \to H$ ist surjektiv. Also existiert $y \in H$ mit $t_x(y)= 1$. Eindeutigkeit von $x^{-1}$ heißt $y = x^{-1} \in H$.
\end{proof}

\begin{definition}[Erzeugendensystem]
	Ist $X \subseteq G$, so ist
	\begin{align*}
		\langle X \rangle = \bigcap_{\substack{H < G\\X \subset H}} H
	\end{align*}
	die von $X$ erzeugte Untergruppe. Ist $\langle X \rangle = G$, nennen wir $X$ ein Erzeugendensystem.
\end{definition}

\begin{definition}[Konjugation]
	Ist $H < G$ und $x \in G$, so ist\begin{align*}
		x^{-1}Hx = \set{x^{-1}Hx\mid y \in H}
	\end{align*}
	eine Untergruppe ("`$x^{-1}yx$"' $y$ ist konjugiert mit $x$). Wir nennen diese zu $H$ konjugiert.
%	\begin{align*}
%		(x^{-1}yx)^{-1} = x^{-1}y^{-1}x \und x^{-1}yx\cdot x^{-1}zx = x^{-1}yzx
%	\end{align*}
\end{definition}

\begin{definition}[Konjugationsklasse]
	Die Menge $\set{x^{-1}yx \mid x \in G}$ ist i.A. \emph{keine} Untergruppe und diese nennt man \begriff{Konjugationsklassen} von $y$.
\end{definition}

\begin{definition}[Zentralisator, Zentrum]
	Der \begriff{Zentralisator} von $y \in G$ ist
	\begin{align*}
		\set{x \in G \mid xy = yx} =: Z_G(y).
	\end{align*}
	Das \begriff{Zentrum} von G ist
	\begin{align*}
		Z(G) = \bigcap_{y \in G} Z_G(y) = \set{x \in G \mid \forall y \in G xy=yx}.
	\end{align*}
\end{definition}
\begin{example}
	Sei $G = S_n \ni f$ Permutation, z.B.
	\begin{align*} S_6 \in
		\begin{pmatrix}
		1 & 2 &3 &4 & 5 & 6\\
		5 & 4 & 6 & 1 & 2 & 3
		\end{pmatrix}  = (1524)(36)
	\end{align*}
	letzteres nennt man \begriff{Zykelnotation}. 1-Zykeln, d.h. $i \in \set{1,\dots,n}$ mit $f(i)=i$ werden meist nicht notiert, z.B.:
	\begin{align*} S_4 \in 
		\begin{pmatrix}
			1&2&3&4\\
			2&1&3&4
		\end{pmatrix} = (12)
	\end{align*}
\end{example}
\begin{remark}
	Ein $k$-Zykel ist ein Produkt von $k-1$ Transpositionen (2-Zykel), z.B.
	\begin{align*}
		(12345) = (15)(14)(13)(12)
	\end{align*}
	ist das Produkt in $S_5$, d.h. Komposition! Also erzeugt $\set{(i,j)}$ die $S_n$. Jede Permutation kann also als Produkt von Transpositionen geschrieben werden. Diese Darstellung ist nicht eindeutig! (z.B. $(12)(23)(12) = (23)(12)(23)$) (``Braid relation'') % add braid picture?! braids package
	und $(12)(12) = (\,)$. Allerdings kommen in jeder solcher Darstellungen entweder eine gerade oder ungerade Anzahl von Transpositionen vor ($\to$ Fehlstände). Insbesondere bilden gerade Permutationen (gerade Anzahl an Fehlständen $\Leftrightarrow$ Produkte von zu Transpositionen) eine Untergruppe $A_n < S_n$, die sogenannte \begriff{alternierende Gruppe}.
\end{remark}
%Sei $G$ eine endliche Gruppe.
\begin{example}
	Sei $G = \GL(n,R)$ die invertierbare Matrizen mit Einträgen in $R$ (nur endliche, wenn $R^{\times} endlich$!). Untergruppen sind
	\begin{itemize}
		\item $\SL(n,R) = \set{g \in \GL(n,R) \mid \det g = 1}$
		\item $O(n,R) = \set{g \in G \mid gg^T = g^T g = 1}$ mit dem Skalarprodukt $\scaProd{gv}{gw} = \scaProd{v}{w} \quad \forall v,w \in R^n$
		\item $\SO(n,R) = \mathrm{SL}(n, R) \cap \mathrm O(n,R)$.
	\end{itemize}
	Ist $R$ Ring mit Involuten (z.B. $R = \C, z = \bar{z}$)
	\begin{itemize}
		\item $U(n,R) = \set{g \in \GL(n,R) \mid gg^{*} = g^{*}g = 1}$
		\item $\SU(n,R) = \SL(n,R) \cap U(n,R)$
	\end{itemize}
\end{example}

\begin{example}
	Sei $D_n$ definiert durch\begin{align*}
		D_n &= \set{f\colon \R^2 \to \R^2 \; \text{linear, bjektiv} \mid f(P_n) = P_n},
	\end{align*}
	wobei $P_n \subset \R^2$ das regulär $n$-gen ist, z.B. das Hexagon $P_6$. % sketch? 
	Alternativ ist $D_n \subseteq S_n$, wobei $\set{1, \dots,n}$ mit der Menge der Ecken von $P_n$ identifiziert wird und man erhält alle Permutationen, die benachbarte Ecken auf benachbarte abbilden:
	\begin{itemize}
		\item $r$: Rotation um $\sfrac{2\pi}{n}$ im mathematische positiven Sinn
		\item $s$: eine beliebige Spiegelung in $D_n$
	\end{itemize}
	Also hat man
	\begin{align*}
		\langle \set{s,r}\rangle = D_n = \set{s^i r^j \mid i = 0,1, j=0, \dots, n-1}
	\end{align*}
	und der Mächtigkeit $\vert D_n\vert = 2n$.
	
	Für die erzeugenden Elemente $D_n = \langle \lbrace s,r\rbrace \rangle$ gilt:
	\begin{itemize}
		\item $srs = r^{n-1}$,
		\item $r^{n-1} = r^{-1}$,
		\item $r^n = 1$,
		\item $s^2 = 1$.
	\end{itemize}
Im unendlichen Fall $D_\infty \subset S_{\mathbb Z}$ gilt z.B. $r(z) = z+1$, $s(z) = -z$, wobei $r$, $s\colon \mathbb Z\to\mathbb Z$ sind und $D_\infty$ erzeugen: $D_\infty = \langle \lbrace r,s\rbrace\rangle$.
\end{example}

\section{Nebenklassen, Normalteiler, Isomorphiesätze}
\begin{definition}
	$A,B \subseteq G$ Teilmengen (nicht unbedingt Untergruppen!), dann:
	\begin{itemize}
		\item $AB = \set{xy \in G \mid x \in A, y \in B}$
		\item $A^{-1} = \set{x^{-1} \in G \mid x \in A}$
	\end{itemize}
\end{definition}

\begin{remark}
	$\emptyset \neq H \subseteq G$ ist Untergruppe $\Leftrightarrow$ $HH = H$, $H^{-1} = H$
\end{remark}

\begin{definition}
	Ist $x \in G$, so nennen wir
	\begin{align*}
		f_x\colon G \to G \mit y \mapsto x^{-1}yx
	\end{align*}
	den durch $x$ definierten inneren Automorphismus. Ist $H < G$, so nennen wir $f(H) = x^{-1}Hx$ eine zu $H$ konjugierte Untergruppe.
\end{definition}

\begin{proposition}
	\begin{itemize}
		\item $f_x$ ist ein Endomorphismus von $G$ (d.h. ein Morphismus $G \to G$)
		\item Das Bild $\Image f$ eines beliebigen Gruppenmorphismus $f\colon K \to L$ ist eine Untergruppe: $\Image f < L$
	\end{itemize}
\end{proposition}
\begin{proof}\leavevmode
	\begin{itemize}[topsep=-6pt]
		\item $f_x (yz) = x^{-1}yzx = x^{-1}y(xx{^-1})zx = (x^{-1}yx)(x^{-1}zx) = f_x(y)f_x(z)$ $\forall y,z \in G$
		\item Wir untersuchen die drei Eigenschaften:
		\begin{itemize}
			\item $\Image f$ ist abgeschlossen: seien $f(y), f(z) \in \Image f$. Dann gilt:
			\begin{align*}
			f(y)f(z) = f(yz) \in \Image f
			\end{align*}
			\item $f(1) = 1 \implies 1 \in \Image f$
			\item $f(x)^{-1} = f(x^{-1}) \implies (\Image f)^{-1} = \Image f$
		\end{itemize}
	\end{itemize}
\end{proof}

\begin{definition}
	Ist $H < G, x \in G$, so nennt man
	\begin{align*}
	G \supseteq x H &= \set{x}H = \set{xy \in G \mid y \in H} \quad \text{linke Nebenklasse}\\
	G \supseteq H x &= \set{yx \in G \mid y \in H} \quad \text{rechte Nebenklasse}
	\end{align*}
\end{definition}
\begin{example}
	Sei $G = V$ Vektorraum über Körper $K$ mit $+$ als Gruppenstruktur, dann ist $H = W < V$ ein Untervektorraum und $xH = x + W \subseteq V$ affiner Unterraum, Element von $\lnkset{V}{W}$
\end{example}
Dies verallgemeinert sich zu
\begin{definition}
	Sei $H < G$, $\lnkset{G}{H} = \set{x H \mid x \in G} \subseteq \powerset(G)$
\end{definition}
\begin{remark}
	$xH = yH \Leftrightarrow x \sim y$ definiert eine Äquivalenzrelation und das ist äquivalent zu
	\begin{align*}
		\exists h \in H: x = yh \Leftrightarrow y^{-1}y \in H.
	\end{align*}
	Beachte dabei $\lnkset{G}{H} = \lnkset{G}{N}$ ist die Menge aller Äquivalenzklassen $xH = [x]$. Desweiteren gibt es die kanonische Projektion $\pi\colon G \to \lnkset{G}{H},x \mapsto xH$.
	
	Insbesondere ist $G$ die disjunkte Vereinigung aller Äquivalenzklassen. Speziell ist für jedes $x \in G$ definiert:
	\begin{align*}
		t_x\colon G \to G, y \mapsto xy \text{ eine Bijektion,} \quad H = 1H = [x] \to xH = [x].
	\end{align*}
\end{remark}

Alle $xH$ haben also die gleiche Kardinalität und wir erhalten:
\begin{proposition}[Lagrange, Klausur!]
	Sei $\abs{G} < \infty$ und $H < G$. Dann gilt $\abs{G} = \abs{\lnkset{G}{H}}\cdot \abs{H}$. Insbesondere ist $\abs{G}$ durch $\abs{H}$ teilbar.
\end{proposition}
\begin{proof}
	\ul{Beweisskizze:} Äquivalenzrelation und Bijektion $xH \cong yH$.\\
	?, eventuell von Fehm übernehmen?!
\end{proof}

\begin{conclusion}
	Sei $\abs{G} < \infty$, dann gilt $\abs{x} \mid \abs{G}$ für alle $x \in G$. Dabei ist $\abs{x} = \abs{\langle \set{x}\rangle} = \min\set{n \mid x^n = 1}$. Also z.B. $\langle \set{x} \rangle \cong (\Z_{\abs{x}},+)$. Insbesondere gilt für alle $x \in G$:  $x^{\abs{G}} = 1$
\end{conclusion}

\begin{conclusion}[Eulers Theorem]
	$\abs{U_n} = \phi(n) = \abs{\set{m \in \set{1, \dots,n} \mid \ggT(n,m) = 1}} = \abs{\set{(\Z_n^{\times}, \cdot) \mid \ggT(n,m) = 1}}$ mit $n \in \N$. Also ist $m^{\phi(n)} = 1 \mod n$.
\end{conclusion}
\begin{definition}[Index]
	Sei $H < G$, dann ist $[G:H] := \abs{\lnkset{G}{H}}$ der Index von $H$ in $G$.
\end{definition}

\begin{conclusion}
	Sei $K < H < G$ und $\abs{G} < \infty$, dann
	\begin{align*}
		[G:K] = \abs{\lnkset{G}{K}} = \frac{\abs{G}}{\abs{H}}\cdot \frac{\abs{H}}{\abs{K}} = [G:H][H:K].
	\end{align*}
\end{conclusion}

\section{Morphismen}
\begin{definition}
	Ein injektiver Morphismus $f\colon G \to H$ wird auch Einbettung genannt. Ein Isomorphismus ist ein bijektiver Morphismus.
\end{definition}

\begin{*remark}
	Ein injektiver Morphismus wird auch Monomorphismus genannt und ein surjektiver Morphismus Epimorphismus.
\end{*remark}

\begin{example}
	\begin{enumerate}[label={\arabic*)}]
		\item Betrachte die Determinate $\det\colon \GL(n,R) \to R^{\times}$, diese ist ein surjektiver Morphismus von Gruppen mit
		\begin{align*}
		\det(gh) = \det(g)\det(h)
		\end{align*}
		\item Die Wahl einer Basis $B$ in einem endlich erzeugten freien Modul $V$ ist ein Isomorphismus von Moduln $s_B\colon R^{\abs{B}} \to V$. Dieser induziert einen Gruppenisomorphismus
		\begin{align*}
			\GL(n,R) \to \GL(V), g \mapsto s_B \circ M_g \circ s^{-1}_B
		\end{align*}
		\item Die Linkstranslation $t\colon G \to S_G \mit x \mapsto t_x$ (mit $t_x(y) = xy$) ist ein injektiver Gruppenhomomorphismus
		\begin{align*}
			(t_x \circ t_z)(y) &= t_x(t_z (y))=t_x(zy) = xzy = t_{xz}(y)
			\intertext{also}
			t_x \circ t_y &= t_{xz} \quad \forall x,z \in G
		\end{align*}
		Ist $t_x = t_z$, so gilt $t_x(1) = t_z(1)$ und daraus $x1=z1$, also $x=z$
		
		Also kann \emph{jede} endliche Gruppe als Untergruppe der $S_n$ verstanden werden ($n = \abs{G}$)!
	\end{enumerate}
\end{example}
\begin{example}
	\begin{enumerate}[label={\arabic*)}]
		\item $x \mapsto f_{x^{-1}}\colon G \to G$, $y \mapsto xyx^{-1}$ ist ein Morphismus $G \to \mathrm{Aut}\,G$ mit
		\begin{align*}
		f_x(y) = x^{-1}yx, \quad f_z(f_x(y)) = z^{-1}(x^{-1}yx)z = (xz)^{-1}y(xz) = f_{xz},
		\end{align*}
		und ist i.A. nicht injektiv! Denke an $G$ abelsch $\Leftrightarrow f_x = \id_G$ $\forall x \in G$
		\item $\sgn\colon S_n \to \Z_2 = \set{-1,1}$
	\end{enumerate}
\end{example}
\chapter{Cox-Russ-Rubenstein-Modell}
$R$ sie kommutativer Ring mit 1-Element und $V$ ein $R$-Modul.
\begin{erinnerung}
	Ist $X$ eine Menge, so kann man ein
	\begin{align*}
		R^{\times} = \set{f\colon X \to R \mid f(x) = 0, \text{ für fast alle }x \in X}
	\end{align*}
	definieren Modul mit
	\begin{align*}
		+ \colon R^X \times R^X \to R^X \mit (f+g)(x) = f(x)+g(x)\quad x \in X, f,g \in R^X\\
		\cdot\colon R \times R^X \to R^X \mit (rf)(x) = r\cdot f(x)
	\end{align*}
	Ist $V$ ein $R$-Modul und $X \subseteq V$ ($X$ nicht unbedingt Untermodul), so erhält man einen Morphismus von $R$-Moduln ($R$-lineare Abbildungen)
	\begin{align*}
		S_x\colon R^X \to V \mit f \mapsto \sum_{x\in X}f(x)\cdot x
	\end{align*}
	Also ist $\Image(S_X)$ die Menge aller Linearkombinationen von Elementen von $X$, d.h. $\Span_R X$ bzw. der von $X$ erzeugte Untermodul von $V$. $X$ ist ein Erzeugendensystem von $V$, wenn $\Image(S_X) = V$ ist. Ein Modul heißt endlich erzeugt, wenn es ein $X \subseteq V$ mit $\Image(S_X) = V$ und $\abs{X} < \infty$.\\
	$X$ ist linear unabhängig, wenn $S_X$ injektiv ist, also $\ker S_X = 0$. Ist $S_X$ bijektiv, also $X$ ein linearer unabhängiges Erzeugendensystem, so nennt man $X$ Basis von $V$. Ein Modul $V$ der eine Basis enthält ist frei $V \cong R^X$. Also $R$ Körper $\implies$ Jeder Modul (= VR) ist frei und $R^X \cong R^V \Leftrightarrow \abs{X} = \abs{Y} = \dim R^X$
\end{erinnerung}
\begin{example}
	Sei $R = \Z$, dann $\Z$-Moduln sind abelsche Gruppen, also $(4x = x+x+x+x)$ durch $+$ in $V$ festgelegt. $V = \Z_2 = \lnkset{\Z}{2 \Z}, X = \set{1 + 2\Z} = \set{[1]}$ ist Erzeugendensystem (sogar minimal). Aber $X$ ist keine Basis, denn es ist nicht linear unabhängig, da $2 \cdot [1] = [0]$
\end{example}
\begin{definition}
	Ein $R$-Modul $V$ ist \begriff{projektiv} $\Leftrightarrow$ für jede Epimorphismus (surjektiv) $\alpha \colon M \to N$ von $R$-Moduln und jeden Morphismus $\gamma \colon V \to N$ gibt es einen Morphismus $\beta \colon V \to M$ mit $\alpha \circ \beta = \gamma$
	\[
		\begin{tikzcd}
		& V \arrow[ld, "\exists \beta"', dashed] \\
		M \arrow[r, "\alpha"', two heads] & N \arrow[u, "\gamma"', hook]          
		\end{tikzcd}
	\]
\end{definition}
\begin{proposition}
	\proplbl{prop_2_1_1}
	Für einen $R$-Modul $V$ sind äquivalent
	\begin{enumerate}
		\item $V$ ist projektiv
		\item Jeder Epimorphismus $\pi\colon M \to N$ ist \begriff{split}, d.h. $\exists \iota \colon V \to M$ mit $\pi \circ \iota = \id$ ($M$ $R$-Modul)
		\item Es existiert ein $R$-Modul $W$ mit $V \oplus W$, der frei ist
	\end{enumerate}
\end{proposition}
\begin{remark}
	$W \subseteq V$ ist Komplement zu $U \subseteq V \Leftrightarrow \pi \colon V \to \lnkset{V}{U}$ ist isomorph und $\pi_{\mid W}\colon W \to \lnkset{V}{U}$
\end{remark}
\begin{example}
	$\pi \colon \Z \to Z$ splittet nicht als $\Z$-Modulmorphismus. Also ist $\Z_2$ nicht projektiv.
\end{example}
\begin{proof}[\propref{prop_2_1_1}]
	\begin{enumerate}
		\item 1 $\implies$ 2: Betrachte ($N = V$, $\gamma = \id_V$, $\alpha = \pi$)
		\[
			\begin{tikzcd}
			& V \arrow[ld, "\exists \iota"', dashed] \\
			M \arrow[r, "\pi", two heads] & N \arrow[u, "\gamma = \id_V"', hook]  
			\end{tikzcd}
		\]
		\item 2 $\implies$ 3: Jeder Modul $V$ hat ein Erzeugendensystem, z.B. $V = X$ selbst. $S_X\colon R^X \to V$ ist ein Epimorphismus ($R^X$ ist frei! $X$ ist Basis bzw. $\set{\delta_x}_{x \in X}$ ist Basis ($f = \sum_{x\in X} f(x)\delta_x$)). $V$ projektiv, existiert ein Splitting $\iota \colon V \to R^X$ mit $\pi \circ \iota = \id_X$, $\tilde{V} = \Image \iota$ ist dann ein Untermodul von $R^X$, der Isomorphismus zu $V$ ist. Betrachte nun $\epsilon = \iota \circ \pi \colon R^X \to R^X$. Dies ist ein idempotenter Morphismus, d.h. 
		\begin{align*}
			\epsilon \circ \epsilon &= (\iota \circ \pi)(\iota \circ)\\
			&= \iota \circ (\pi \circ \iota)\circ \pi\\
			&= \iota \circ \id \circ \pi = \epsilon.
		\end{align*}
		Somit gilt aber $R^X \cong \ker \epsilon \oplus \Image \epsilon \mit f \mapsto (f - \epsilon(f), \epsilon(f))$ und $\ker \epsilon \cap \Image \epsilon = 0$ % could be \emptyset is meant here ...
		$f = \epsilon(g), \epsilon(f) = 0$ heißt $\epsilon(\epsilon(g)) = \epsilon(g) = f$. Also gilt $\Image \epsilon = \Image \iota = \tilde{V} \cong V$. Denn $\epsilon(f) = (\iota \circ \pi)(f) = \iota(\pi(f))$, also $\Image \epsilon \subseteq \Image \iota$ und da $\pi$ surjektiv ist gilt Gleichheit. $V = \tilde{V}$, da $\iota$ injekti ist, da ($\pi \circ \iota = \id_V$) also $\ker \iota = 0$ und $V \cong \lnkset{V}{\ker \iota} \cong \Image \iota = \tilde{V}$. Also ist
		\begin{align*}
			V \oplus W \cong \tilde{V} \oplus W \mit W = \ker \epsilon
		\end{align*}
		\item 3 $\implies$ 1: Sei $V \oplus W \cong R^X$ frei und ein Diagramm der Form
		\[
			\begin{tikzcd}[ampersand replacement=\&]
			\& V \arrow[d, "\gamma"] \\
			M \arrow[r, "\alpha", two heads] \& N                    
			\end{tikzcd}
		\]
		folgt
		\[
			\begin{tikzcd}[ampersand replacement=\&]
			\& R^X \arrow[ld, "\exists \rho", bend right] \arrow[d, "\pi"] \\
			M \arrow[rd, two heads, bend right] \& V \arrow[d, "\gamma"]                                       \\
			\& N                                                          
			\end{tikzcd}
		\]
		($\alpha$ Epimorphismus $R^X = V \oplus W$, $\pi \colon R^X \to V$, 1. Komponente $V \cong \lnkset{R^X}{W}$)\\
		Für jedes Basiselement $\delta_x \in R^X (x \in X)$ existiert ein $m_x \in M$ mit $\alpha(m_x) = \gamma(\pi(\delta_x))$ (dann ist $\alpha$ surjektiv). Jetzt kommt die Freiheit: Jede Abbildung $\set{\delta_x}_{x \in X} \to M$ kann zu einer linearen Abbildung (eindeutig) $\begin{tikzcd}[cramped, sep=small]
		R^X \arrow[r, "\rho"] & M
		\end{tikzcd}$ festgesetzt werden.
		\begin{align*}
			\Mod_R(R^X, M) \cong \Set(X,M)
		\end{align*}
		Sprich $\exists! \rho \colon R^X \to M$ mit $\rho(\delta_x) = m_x$. Die Einschränkung von $\rho$ auf das Untermodul $V \subseteq R^X$ (bzw. wenn man die Einbettung $V \to R^X$) mit $\iota$ bezeichnet $\rho \circ \iota$)  liefert dies das gewünschte $\beta\colon V \to M$.
	\end{enumerate}
\end{proof}
\begin{example}
	$R \in C^{\infty}(\R, \R)$ stetige Funktion von $\R \to \R$, $S \subseteq \R$ Teilmenge, z.B. $S = [0,1]$,\\ $V = \set{f \in C^{\infty}(\R, \R) \mid f(x) = \text{ für }x \notin S}$ Untermodul von $R$ selbst ist, d.h. $V \lhd R$ ist Ideal und sogar Hauptideal.
	\begin{align*}
		V = p \cdot R, p = \chi_S, p(x) = \begin{cases}
		0 \quad x \notin S\\
		1 \quad x \in S
		\end{cases}\\
		p^2 = p, (1-p)^2 = (1-p) = \chi_{R \setminus S}
	\end{align*}
	damit ergibt sich $R = p \cdot R \oplus (1-p)R = V \oplus W$ 
\end{example}
\begin{erinnerung}
	\begin{itemize}
		\item Morphismus $R^n \to R^n$ ist gegeben durch Matrixmultiplikation. Die Projektion auf $V$ ($\epsilon \colon R^n = V \oplus W \to R^n \with (v,w) \mapsto v$) korrespondiert zu einer idempotenten Matrix. $e \in \Mat(n,R), e^2 = e$ und $V = e\cdot R = \set{e x \mid x \in R} = \Image(\epsilon)$. 
		\item projektiv gdw für alle Endomorphismen $\begin{tikzcd}[cramped, sep=small]
		M \arrow[r, "\pi"] & V
		\end{tikzcd}$ split, also $M \cong V \oplus \ker(\pi)$ ($V$ ist $\Image(\pi)$, Kern-Bild-Formel). Wenn $V$ endlich erzeugt ist $V = \Span\set{x_1, \dots, x_n}$
		\item $\pi \colon V \with (v_1, \dots, v_n)^T \mapsto \sum_{i=1}^n v_i x_i$ surjektiv. Also $V \cong \lnkset{R^n}{ker(\pi)}$
	\end{itemize}
\end{erinnerung}
\begin{example}
	Sei $R = \C$, dann
	\begin{enumerate}
		\item $\Z$-Modul ist abelsche Gruppe
		\item $\C(t)$-Modul ist $\C$-Vektorraum $V$ und $\C$-lineare Abbildung $\pi\colon V \to V$
	\end{enumerate}
	beide sind Hauptidealringe $(\Z, C(t))$ (HIR).
\end{example}
\begin{definition}
	$R$ ist ein HIR $\Leftrightarrow$ $R$ ist Integritätsbereich (eng. integral domain, nullteilerfrei) (ID) und jedes Ideal $I \lhd R$ ist ein Hauptideal, d..h. es existiert $a \in R$ mit
	\begin{align*}
		I = R \cdot a = \set{ra \mid r \in R} = \set{s \in R \mid a \mid s}
	\end{align*}
	($a \mid s$, entspricht $a$ teilt $s$)
\end{definition}
Ziel dabei ist: Endlich erzeugte Moduln über HIR haben eine Zerlegung (decompositon) $V = R \oplus T$ mit $F$ frei und $T$ Torsion.
\begin{definition}
	Sei $R$ ID, $V$ $R$-Modul
	\begin{enumerate}
		\item $x \in V$ ist ein Torsionselement $\Leftrightarrow \exists r \in R, r \neq 0, r\cdot x = 0$
		\item $T(V) = \set{x \in V \mid x \text{ ist Torsionselement}} \le V$
		\item $V$ ist torsionsfrei, wenn $T(V) = 0$
		\item Annihilator $\Ann_R (V) := \set{r \in R \mid rx = 0 \forall x \in V} \lhd R$
	\end{enumerate}
\end{definition}
\begin{example}
	\begin{enumerate}
		\item Da $R$ ID ist, ist ein freier Modul torsionsfrei:\\
		$f \in R^{\times}, r \in R, r \neq 0, r \cdot f = 0$ heißt $(rf)(x) = r \cdot f(x) = 0 \forall x \in X \Leftrightarrow f(x) = 0 \forall x \in X$
		\item Ist $R = \Z$ und $V$ eine endliche abelsche Gruppe, so gilt $\abs{V}\cdot x = 0\quad \forall x \in V$ ($\ord(x) \mid \ord(V)$). Also ist $V = T(V)$.
	\end{enumerate}
\end{example}
\begin{*remark}
	$T(V) \le V$ ist Untermodul (Übung!)
\end{*remark}
\begin{lemma}\proplbl{2_2_1}
	$\lnkset{V}{T(V)}$ ist torsionsfrei. ($R$ ID für den Rest der heutigen VL :S)
\end{lemma}
\begin{proof}
	Sei $X \in \lnkset{V}{T}$ Torsionselement $x$ ist von der Form $X = [v] = v + T$ für ein $v\in V$. $T := T(V)$ $x$ ist Torsionselement heißt: $\exists r \in R \setminus \set{0}\colon r \cdot x = 0$, d.h. $[rv] = [0]$, also $r \cdot v \in T$. Damit ergibt sich $\exists s \in R \setminus \set{0}\colon s(rv) = 0$ also $(sr)v = 0$. Da $R$ ID ist, ist $t:= sr \neq 0$, also ist $v \in T(V) = T$ und da bedeutet $[v] = x = 0$.
\end{proof}
\begin{lemma}
	Sind $N \le M$ Moduln, so gilt $M$ ist Torsions ($(M = T(M))$) $\Leftrightarrow N$ ist Torsion ($N = T(N)$) und $\lnkset{M}{n} = T\brackets{\lnkset{M}{N}}$.
\end{lemma}
\begin{proof}
	Analog zu \propref{2_2_1}.
\end{proof}
\begin{proposition}\proplbl{2_2_2}
	Sei $V$ endlich erzeugter, torsionsfreier $R$-Modul, $V = Span\set{x_1, \dots, x_n}$ und $V \neq 0$.
	\begin{enumerate}
		\item $\exists i_1, \dots, i_k \colon F = R\sum_{j=1}^k x_{i_j}$ ist frei mit Basis $\set{x_{i_1}, \dots, x_{i_k}}$
		\item $\lnkset{V}{F}$ ist Torsionsmodul: $\lnkset{V}{F} = T\brackets{\lnkset{V}{F}}$
		\item Existiert ein Morphismus, der injektiv ist $L \colon \begin{tikzcd}[cramped, sep=small]
		V \arrow[r, hook] & F
		\end{tikzcd}$
	\end{enumerate}
\end{proposition}

\begin{proof}
	Induktion nach $n$: (Beweis die ersten zwei Aussagen)
	\begin{itemize}
		\item $n=1$: $V = R\cdot x \cong R$, da torsionsfrei, $V = Rx$ heißt es existiert ein Endomorphismus $\pi\colon R \to V \mit r \cdot x$ und wenn $x$ kein Torsionselement ist, ist $\ker \pi = 0$ ($\not \exists r \neq 0 \colon rx = 0$, also ist $\pi$ isomorph) und damit gilt \propref{2_2_2}.
		\item $n-1 \to n$: OBdA ist $\set{x_1, \dots, x_n}$ keine Basis, sonst wären wir fertig. (Mit $F = V$). Also existiert eine Linearkombination $0 = \sum_{i=1}^n r_i x_n$ oBdA mit $r \neq 0$, d.h. $\exists r = r_n \in R \setminus{0}$ mit $r\cdot x_n \in \Span\set{x_1, \dots, x_{n-1}} := W$.\\
		Behauptung: $\lnkset{V}{W}$ ist Torsion: Denn ein Element in $\lnkset{V}{F}$ ist von der Form: $[s x_n] = s[x_n]$ und $r[s x_n] = s[r x_n] = 0$, da $r x_n \in W$, also $[r x_n] = 0$ in $\lnkset{V}{W}$. Nach Induktionsannahme existiert Teilmenge $B \subseteq \set{x_1, \dots, x_{n-1}}$, die einen freien Untermodul $G \subseteq W$ aufspannt und $\lnkset{W}{G}$ ist Torsion. Nach dem 3. Isomorphiesatz gilt:
		\begin{align*}
			\lnkset{\lnkset{V}{G}}{\lnkset{W}{G}} \cong \lnkset{V}{W}
		\end{align*}
		Also ist $\lnkset{V}{G}$ ein Torsionmodul ($\lnkset{W}{G}$ Torsion, $\lnkset{V}{W}$ und \propref{2_2_1}) und damit haben wir $F = V$.
		\item Nun zu der 3. Aussage: Da $\lnkset{V}{F}$ Torsion ist existiert für jedes $x_i$ ein $r_i \in R\setminus{0}$ mit $r_i x_i \in F$, $r_i[x_i] = [r_i x_i] = [0] \Leftrightarrow r_i x_i \in F$. Sei $r := \prod_{j=i}^n r_j \neq 0$. Definiere $L\colon V \to F \mit x \mapsto rx$. Da $R$ kommutativ ist, ist $L$ $R$-linear. Da $V$ torsionsfrei ist, ist $L$ injektiv. Das Bild ist in $F$, da $V = \Span_R\set{x_1, \dots, x_n}$ und
		\begin{align*}
			r\brackets{\prod_{j=1}^n r_j x_j} = \prod_{j=1}^n r_j = a_1 r_2\cdots r_n (\underbrace{r_1 x_1}_{\in F}) + a_2 r_1 r_3 \cdots r_n(\underbrace{r_2 x_2}_{\in F}) + \dots \in F.
		\end{align*}
	\end{itemize}
\end{proof}
\begin{conclusion}
	Sei $V$ endlich erzeugter $R$-Modul. Entweder ist $V$ Torsion, $V = T(V)$ oder $V$ enthält einen freien Untermodul $F$ mit $\lnkset{V}{F}$ Torsion.
\end{conclusion}
\begin{proof}
	still TODO.
\end{proof}
three lectures missing :/
\begin{theorem}
	$M(p) = M_1 \oplus \cdots \oplus M_k$ mit $M \cong \lnkset{R}{Rp^n_i}$ zyklisch und $\Ann_R(x) = \set{a \in R \mid ax = 0} \lhd R$
\end{theorem}
\begin{*remark}
	Also sieht das so aus
	\begin{align*}
	M = R^s \oplus \brackets{\bigoplus_{i=1}^n \lnkset{R}{Rp_i^{m_i}}}
	\end{align*}
	mit $p$ prim und $m_i \ge 0$
\end{*remark}
\begin{proof}
	Induktion nach Zahl $t$ der Erzeuger $m_1, \dots, m_t$ von $M=M(p)$. Wir sind im Induktionsschritt und haben $n_1,\dots, n_t$ gewählt mit $\Ann_R(m_i) = Rp^{n_i}$, d.h. $n_i = \min\set{n \in \N \mid p^n m_i = 0, p^{n-1}m_i \neq 0}$, also
	$n = \max\set{n_1, \dots, n_t}$, oBdA $n = n_1$. Dann gilt insbesondere $p^n x = 0$ für alle $x \in M$.\\
	\ul{Trick} Betrachte $\lnkset{M}{Rm_1}$. Dies ist ein $t-1$ erzeugter $p$-primärer $R$-Modul, also zieht die Induktionsannahme. Also ist $\lnkset{M}{R m_i} = R \cdot \bigoplus_{i=1}^k[x_i]$, wobei $[x_i] = x + Rm_i \subset \lnkset{M}{R m_i}, x_i \in M$ und $\Ann_R([x_i]) = Rp^{s_i}$. Sei $a \in \Ann_R([x_i]) \Leftrightarrow a[x_i] = 0$ in $\lnkset{M}{Rm_i} \Leftrightarrow [ax_i = 0]$ in $\lnkset{M}{Rm_i} \Leftrightarrow ax_i \in Rm_i$. Also haben wir $p^{s_i}x_i = r_i m_1$ ($\ast$) für irgendwelche $r_1, \dots, r_k \in R$.\\
	Erinnerung: $p^n x_i = 0$ (denn $p^n \in \Ann_R(M)$), daraus folgt $r_i = r'_i \cdot p^{s_i}$. Setze $y_i := x_i - r'_i m_1$. Dann ist $[x_i] = [y_i]$. Dann gilt $p^{s_i}y_i = p^{s_i}x_i - p^{s_i}r'_i m_1 = p^{s_i}x_i - r_i m_1 \overset{(\ast)}{=} p^{s_i} x_i - r_i m_1 = 0$. Betrachte $p^{s_i -1}y_i = p^{s_i - 1}x_i - p^{s_i - 1}r'_i m_1 \neq 0$, denn ansonsten wäre $p^{s_i - 1}x_i \in Rm_1$, also $p^{s_i - 1} \in \Ann_R([x_i]) = Rp^{s_i}$. Also ist $\Ann_R(y_i) = \Ann_R([y_i]) = Rp^{s_i}$, wobei $\Ann_R(y_i)$ in $M$ und $\Ann_R([y_i])$ in $\lnkset{M}{Rm_1}$ betrachtet wird. Es gilt nun
	\begin{align*}
	\lnkset{M}{Rm_1} = R \bigoplus_{i=1}^n [y_i]
	\intertext{also}
	M = Rm_1 + Ry_1 + \dots + Ry_k
	\end{align*}
	Es blibt noch zu zeigen, das die Summe direkt ist! Setze $y_0 := m_1$. Sei $x \in Ry_i \cap \sum_{j\neq i} Ry_j$, dann ist zu zeigen, dass $x = 0$ ist. Also existiert $s_0, \dots, s_k \in R$, sodass $x = (-s_i y_i) = \sum_{j\neq i}s_j y_j$, daraus folgt $\sum_{j=0}^k s_j y_j = 0$. Insbesondere gilt dies modulo $Rm_1$:
	\begin{align*}
	s_1[y_1] + \cdots + s_k [y_k] &= 0 \quad ([y_0] = [m_1] = 0)\\
	[s_1 y_1 + \cdots + s_k y_k] &= 0
	\end{align*}
	Aber:
	\begin{align*}
	\lnkset{M}{R m_1} = R\bigoplus_{i=1}^k [y_i]
	\end{align*}
	hier eine direkte Summe. Also gilt schon $[s_j y_j] = 0$ für jedes $j = 1, \dots, k$ allein. Also $s_j \in \Ann_R([b_j]) = \Ann_R(b_j)$ (``Das war der Witz!'' $\implies x = 0$) und $s_j y_j = 0$ für $j \ge 1$.
\end{proof}
\begin{proposition}
	Ist $M$ ein endlich erzeugter Modul über einem HIR $R$, so existieren $s, s_1, \dots, s_k \in \N$ und $p_1, \dots, p_k \in R$ prim mie 
	\begin{align*}
	M \cong R^s \oplus \brackets{\bigoplus_{i=1}^k \lnkset{R}{Rp_i^{s_i}}}
	\end{align*}
	Anders gesagt: Es existiert Erzeuger $e_1, \dots, e_{s+k}$ in $M$ mit $\sum_{i=1}^{s+k} r_i e_i = 0 \Leftrightarrow r_i e_i = 0$ für alle $i$ genau, dann wenn $r_1 = \dots = r_s = 0$ und $r_{s+j} \in Rp_j^{s_j}$ für $j = 1, \dots, k$. 
\end{proposition}
\begin{example}
	\begin{itemize}
		\item $R = \Z$, dann folgt $R$-Moduln sind dasselbe wie abelsche Gruppen. Also erhalten wir eine Klassifikation der endlich erzeugten abelschen Gruppen. Hier ist $R^s = \Z^s, \lnkset{R}{Rp_j^{s_j}} = \Z_{p_j^{s_j}}$. Also $\Z_p \times \Z_q \cong \Z_{pq}$ (Chinesischer Restsatz!). Insbesondere haben wir alle endlichen abelschen Gruppen klassifiziert! (endlich $\Leftrightarrow s = 0 \Leftrightarrow M$ ist Torsion)
		\item Abelsche Gruppe der Ordnung 6. Gibt es nur eine $\Z_6 \cong \Z_2 \times \Z_2$, andererseits Ab. Gruppe der Ordnung 4. Gibt es zwei! $\Z_4 \not\cong \Z_2 \times \Z_2$. Also
		\begin{align*}
		\abs{\bigtimes_{i=1}^k\Z_{p_i^{s_i}}} = \abs{\Z_{p_1^{s_1}}} \cdots \abs{\Z_{p_n^{s_n}}} = p_1^{s_1} \cdots p_k^{s_k}
		\end{align*}
	\end{itemize}
\end{example}
\section*{Polynomring als $R$-Modul}
\begin{example}
	Sei $R = K[t]$, $K$ Körper, $K[t] = \set{f \colon \N \to K \mid f(n) = 0 \text{ für fast alle }n}$. Wir schreiben $f$ auch als $f = \sum_{i=0}^{\infty} f(i)t^{i}$. Mit folgenden Eigenschaften
	\begin{itemize}
		\item $(f\cdot g)(n) = \sum_{i=0}^n f(i)g(n-i)$
		\item $(f+g)(n) = f(n) + g(n)$
	\end{itemize}
	Haben gezeigt, dass $R$ HIR ist. $R$-Moduln sind die dieselben wie $K$-Vektorräume $M$ mit einem gewählten Endomorphismus ($K$-linearen Abbildungen) $\tau \colon M \to M$.Denn:\\
	$K \le R$ (eingebettet als die $f$ mit $f(n) = 0, n > 0$, d.h. die ``konstanten'' Polynome vom Grad $\deg f = \max\set{n \mid f(n) \neq 0} = 0$)
	\begin{align*}
		f = f(0)t^0 + 0t^1 + 0t^2 + \dots 
	\end{align*} 
	Also wir ein $R$-Modul $M$ uch ein $K$-Modul=$K$-Vektorraum und Multiplikation mit $t$ definiert eine $K$-lineare Abbildung $\tau \tau \colon M \to M \mit m \mapsto tm$ und umgekehrt liegt der Vektorraum und $\tau$ den Modul fest:
	\begin{align*}
		fm &:= f(\tau)(m) \mit f(\tau) = \sum_{i=0}^{\infty}f(i)\tau' \subset \Vect(M,M)
	\end{align*}
\end{example}
\begin{proposition}
	Ein endlich erzeugter $R$-Modul ist Torsionsmodul $\Leftrightarrow \dim_K M < \infty$
\end{proposition}
\begin{proof}
	Sei $M$ endlich erzeugt, dann haben wir
	\begin{align*}
		M \cong R^{n_0} \oplus \lnkset{R}{Rp_{1}^{n_1}} \oplus \cdots \oplus \lnkset{R}{Rp_{d}^{n_d}}
	\end{align*}
	$\dim_K R^{n_0} = \infty$ für $n_0 > 0$, dann $\dim_K R = \infty$, da $\set{t^0, t^1, t^2, \dots}$ ist Basis und $\dim_K M = \infty$, falls $n_0 > 0$. Umgekehrt ist $\dim_K M < \infty$, so ist $\chi_{\tau} = 0$ in $\Vect(M,M)$ (``$\tau$'' erfüllt sein eigenes charakteristisches Polynom $\chi_{\tau} = \det(\tau - t\id_M)$. Abstrakter: Für den $R$-Modul $M$ gilt
	\begin{align*}
		\set{a \in R \mid ax = 0 \forall x \in M} = \Ann M = R \cdot \chi_{\tau}
	\end{align*}
	Also ist jedes $x \in M$ Torsion, denn $\chi_{\tau} x = 0$. Also Lineare Algebra ist also Theorie der endlichen Torsionsmoduln über $R = K[t]$.
	\begin{align*}
		M \cong \lnkset{R}{Rp_{1}^{n_1}} \oplus \cdots \plus \lnkset{R}{Rp_{d}^{n_d}}
	\end{align*}
	\begin{itemize}
		\item Wann ist $p \in R = K[t]$ prim = irreduzibel? Allgemein: Schwer!!! Polynome vom Grad 1 sind immer $f = t-\lambda$, $\lambda \in K$, aber $t^2 +1$ ist irreduzibel in $\Q[t]$ oder $\R[t]$, $t^2+2$ ist irreduzibel in $\Q[t]$, reduzibel in $\R[t]$. Leicht ist aber dagegen, wenn $K = \bar{K}$ algebraisch abgeschlossen (z.B. $K = \C$), so ist $f \in K[t]$ irreduzibel genau dann, wenn $f = t - \lambda$ mit $\lambda \in K$. Also jetzt $K = \bar{K}$
		\begin{align*}
			\lnkset{R}{Rp_{j}^{n_j}} \cong \lnkset{K[t]}{(t-\lambda_j)^{n_j}K[t]} \quad \text{ für ein }\lambda_j \in K
		\end{align*}
		Als Vektorraum über $K$ hat dieser Modul die Basis
		\begin{align*}
			e_r &:= (t-\lambda_j)^r + (t-\lambda_j)^{n_j}K[t] = [(t-\lambda_j)^r]_{r = 0, \dots, n_j -1}\quad e_r := (t-\lambda_j)^r + (t-\lambda_j)^{n_j}K[t]\\
			&= [(t-\lambda_j)^r]_{r = 0, \dots, n_j -1}$ (\person{Taylor}).
		\end{align*} 
		Nun $\tau(e_r) = te_r = \lambda_j e_r + e_{r+1}$ ($r < n_j -1$) bzw. $t e_{n_j - 1} = \lambda_j e_{n_j -1}$
		\begin{align*}
			(t-\lambda_j)e_r = \begin{cases}
			e_{r+1} \quad & r<n_j -1\\
			0 \quad & r = n_j -1
			\end{cases}
		\end{align*}
		denn $e_r = [(t-\lambda_j)^r]$ also $(t - \lambda_j)[(t-\lambda_j)^r] = [(t-\lambda_j)^{r+1}]$. Bezüglich der Basis $e_0, \dots, e{n_j -1}$ hat $\tau_{\mid S}$ mit $S = \lnkset{R}{Rp_j^{n_j}}$ also die darstellende Matrix
		\begin{align*} %TODO fix matrix!!!
			\begin{pmatrix}
			\lambda_j & & & & & \\
			1 &\lambda_j & & & &\\
			0 &1 &\lambda_j & & &\\
			& & & 1&\lambda_j\\
			\end{pmatrix} = \diag(\lambda_j, \dots, \lambda_j)
		\end{align*}
		und Nebendiagonale mit 1en ... Die zyklischen primären Komponenten sind also die Jordanblöcke von $\tau$.
	\end{itemize}
\end{proof}
\chapter{Das Black-Scholes-Modell}
THe goal is to transition from CRR-model (in discrete time) to \person{Block}-\person{Scholes} (BS-)model (in continious time) through fomation of limit.
\begin{itemize}
	\item Derivation of \person{Block}-\person{Scholes}-formula for price of european put- and call-options. 
\end{itemize}
Consider the time interval $[0,T]$, for every $N \in \N$ divided in steps of length $\Delta_n = \frac{T}{N}$. Choose a parameter $r \in \R, \mu \in \R$ (trend parameter), $\sigma > 0$ (violatility). Define a sequence of CRR-models $(S^N)_{N \in \N}$ embedded in $[0,T]$ with parameters
\begin{align*}
	r_N = r \cdot \Delta_n \quad b_N = \mu \Delta_n + \sigma \sqrt{\Delta_n}\quad a_N = \mu \Delta_n - \sigma \sqrt{\Delta_n}\;p \in (0,1),\;s> 0
\end{align*}
i.e. $S^N_0 = s$, $S^N_{t_k} = s \cdot \prod_{i=1}^k (1+R_i^N)$ with $t_k = k \cdot \Delta_n$, or $\tilde{S}_0^N = s$ and hence $\tilde{S}^N_{t_k}= s \cdot \prod_{i=1}^k \frac{1+R_i^N}{1+r_N}$, where $\P(R^N_i = n_N) = p, \P(R^N_i = a_N) = 1-p$.
Denote the sequence with $\CRR_N$. If its necessasry, we interpolate between the grid points with
\begin{align*}
	S_t^N = S^N_{t_k} \quad t \in [t_k,t_{k+1}]
\end{align*}
Calculate the risk-neutral probabilities
\begin{align*}
	q_N = \QQ_N(R_i^N = b_N) = \frac{r_N - a_N}{b_N - a_N} = \frac{(r-\mu)\Delta_n + \sigma \sqrt{\Delta_n}}{2\sigma\sqrt{\Delta_n}} = \frac{1}{2} - \frac{\lambda}{2}\sqrt{\Delta_n}
\end{align*}
with $\lambda := \frac{\mu - r}{\sigma}$
\begin{*remark}
	\begin{itemize}
		\item If $\mu = r$, then $q_N = \frac{1}{2}$ and in generall $\lim_{k \to \infty}a_N = \half$
		\item $\lambda := \frac{\mu - r}{\sigma}$ is called ``Sharp-ratio'' or market risk price 
	\end{itemize}
\end{*remark}
Question: convergence of the distribution of $S^N_T$ under $\QQ_N$ for $N \to \infty$?\\
Transition to logarithm:
\begin{align*}
	\ZZ_N := \log(\frac{S^N_T}{S_0}) = \sum_{k=1}^N \underbrace{\log(1+R_k^N)}_{L^N_k}
\end{align*}
Sum the independent, identically distributed random variables, then use the central convergence theorem?\\
There exists a so-called \emph{Triangle-scheme}
\begin{align*}
	\begin{matrix}
	\ZZ_1 &= L_1^1 & &\\
	\ZZ_2 &= L_2^1 &+L^2_2 &\\
	\ZZ_3 &= L_3^1 &+L^3_2 &+L^3_3\\
	\end{matrix} \text{Random variables in a row are stochastically independent.}
\end{align*}
\begin{theorem}[Central convergence theorem for triangle-scheme]
	Let $L^N := (L^N_1, L^N_2, \dots, L^N_N)$ be a vector of random variables for every $N \in \N$ (``triangle-scheme'') with the following properties:
	\begin{enumerate}
		\item $\forall N \in \N$, $(L^n_1, \dots, L_N^N)$ are independent with identicall distribution
		\item $\exists$ Sequence of (deterministic) constants $C_N \to 0$, such that
		\begin{align*}
			\abs{L_k^N} \le C_N \quad \forall k \in [N]
		\end{align*}
		\item With $\ZZ_N = L^N_1 + \dots + L^N_N$ it holds
		\begin{align*}
			\E[\ZZ_N] \to m \in \R\\
			\Var(\ZZ_N) \to s^2 > 0 \text{ für }N \to \infty
		\end{align*}
	\end{enumerate}
	Then $(\ZZ_N)_{N \in \N}$ converges in distribution towards the normally distributed random variable $\ZZ$ with $\E[\ZZ] = m \und \Var(\ZZ) = s^2$
\end{theorem}
\begin{proof}
	Without a proof, see eg Probability theory with martingale.
\end{proof}
\begin{*remark}
	Compare 2nd exercise sheet/ 1st exercise.
\end{*remark}
\begin{erinnerung} % should not count :/
	The density of the standard normal distribution is:
	\begin{align*}
		\phi(x) = \frac{1}{\sqrt{2\pi}} e^{-x^2/2}
	\end{align*}
	and the distribution function:
	\begin{align*}
		\Phi(x) = \int_{-\infty}^x \phi(y) \d y = \int_{-\infty}^x \frac{1}{\sqrt{2\pi}} e^{-y^2/2} \d y
	\end{align*}
	The normal distribution with expecte value $m$ and variance $s^2$ has distribution function $\Phi(\frac{x-m}{s})$
\end{erinnerung}
\begin{definition}
	A strict positive random variable $X$ is called \begriff{log-normally distributed} with parameter $m, s^2$, if it holds
	\begin{align*}
		\log(X) \sim \NNN(m,s^2)
	\end{align*}
\end{definition}
\begin{theorem}
	Consider the sequence $(S^N)_{N \in \N}$ of CRR-models, as described in $\CRR_N$. Then $S_T^N$ converges in distribution under $\QQ_N$ towards a random variable $S_T$ and $S_T/S_0$ is log-normally distributed with paraeters $n = T(r - \sigma^2/2)$ and $s^2 = T\sigma^2$. Equivalently it holds $\ZZ_N = \log(S_T^N / S_0)$
	\begin{align*}
		\QQ_N(\ZZ_N \le x) \xrightarrow{N \to \infty} \Phi\brackets{\frac{x-T(r-\sigma^2/2)}{\sigma\sqrt{T}}}
	\end{align*}
\end{theorem}
\begin{proof}
	The triangle-scheme $L^N = (L^N_1, \dots, L_N^N)$ with $L^N_k = \log(1+R^N_k)$ obviously satisfies the condition 1. and 2. from theorem 3.1, (under $\QQ_N$). %TODO add references
	Choose as eg.: 
	\begin{align*}
		C_N = \max(\abs{\log(1+\mu\Delta_n + \sigma\sqrt{\Delta_n})}, \abs{\log(1+\mu \Delta_n - \sigma\sqrt{\Delta_n})})
	\end{align*}
	We calculate the expected value and variance of $L^N_k$ or $\ZZ_N$. Use the Taylor expansion:
	\begin{align*}
		\log(1+x) = x - x^2/2 + x^3/3 + \OO(x^4) \quad (x \to 0)
	\end{align*}
	I.e.
	\begin{align*}
		\log(1+ \underbrace{\mu \Delta_n \pm \sigma\sqrt{\Delta_n}}_{b_N \text{ or }a_N}) = \pm \sigma \sqrt{\Delta_n} + \mu \Delta_n - \sigma^2/2 \Delta_n + \OO(\Delta_n^{3/2})
	\end{align*}
	The risk-neutral probabilities are
	\begin{align*}
		q_N = \half - \frac{\lambda}{2}\sqrt{\Delta_n}\quad 1-q_N = \half + \frac{\lambda}{2}\sqrt{\Delta_n}
	\end{align*}
	\begin{align*}
		\E^{\QQ_N}[L^N_k] &= \E^{\QQ_N}[\log(1+R^N_k)] = q_N\log(1+b_N) + (1+p_N)\log(1+a_N)\\
		&= (\mu - \sigma^2/2)\Delta_n - \lambda\sigma\Delta_n + \OO(\Delta_n^{3/2}) \quad \mit \lambda = \frac{\mu -r}{\sigma}\\
		&= (\mu - (\mu - r) - \sigma^2/2) \Delta_n + \OO(\Delta_n^{3/2})\\
		&= (r-\sigma^2/2)\Delta_n + \OO(\Delta_n^{3/2})\\
		\E^{\QQ_N}[(L^N_k)^2] &= q_N\log^2(1+b_N) + (1-q_N)\log^2(1+a_N)\\
		&= \sigma^2\Delta_n + \OO(\Delta_n^{3/2})\\
		\Var^{\QQ_N}(L^N_k) &= \E^{\QQ_N}[(L^N_k)^2]-\E^{\QQ_N}[L^N_k]^2 = \sigma^2\Delta_n + \OO(\Delta_n^{3/2})
	\end{align*}
	So, it holds
	\begin{align*}
		\E^{\QQ_N}[\ZZ_N] &= N \cdot \E^{\QQ_N}[L_k^N] = (r-\sigma^2/2)T + \OO(N^{-1/2}) \xrightarrow{N \to \infty} (r-\sigma^2/2)T =: m\\
		\Var^{\QQ_N}[\ZZ_N] &= N \cdot \Var^{\QQ_N}[L^N_k] = \sigma^2 T + \OO(N^{-1/2}) \xrightarrow{N \to \infty} \sigma^2T =:s^2
	\end{align*}
	The result follows with the central limit theorem (Theorem 3.1).
\end{proof}
\subsection*{Asymptotics of put- and call-option}
Fix the duration $T$ and the strike price $K$ and write:\\
\begin{itemize}
	\item $C_N(t, S_t^N)$ ... price of a european call-option in $\CRR_N$ model, dependant of time $t$ and basic good $S^N_t$
	\item $P_N(t, S_t^N)$ ... analogously for put
\end{itemize}
\begin{theorem}[\person{Block}-\person{Scholes}-formula]
	The prices $C_N$, $P_N$ converge for $N \to \infty$ towards a BS-price
	\begin{align*}
		C_{BS}(t,S_t) &= \lim_{N \to \infty} C_N(t, S_t^N)\\
		P_{BS}(t,S_t) &= \lim_{N \to \infty} P_N(t, S_t^N)
	\end{align*}
	and the following \begriff{\person{Block}-\person{Scholes}-formula} holds:
	\begin{align*}
		C_{BS}(t,S_t) &= S_t \Phi(d_1) - e^{-r(T-t)} K \Phi(d_2)\\
		P_{BS}(t,S_t) &= S_t \Phi(-d_1) - e^{-r(T-t)} K \Phi(-d_2)
		\intertext{where}
		d_1 &= d_1(t, S_t) = \frac{\log(S_t/K) + (r+\sigma^2/2)(T-t)}{\sigma \sqrt{T-t}}\\
		d_2 &= d_2(t, S_t) = \frac{\log(S_t/K) + (r-\sigma^2/2)(T-t)}{\sigma \sqrt{T-t}} = d_1 - \sigma\sqrt{T-t}
	\end{align*}
\end{theorem}
\chapter{Optimale Investition}
\section{Das Anlageproblem}
\emph{Gegeben:} Vermögen $W$, Anlagegüter $S^1, \dots, S^n$ (Aktien, Anleihen, ...)\\
\emph{Gesucht:} Optimale Verteilung $W = W_1 + \dots + W_n$ auf $S^1 \dots S^n$\\
$S^1 \dots S^n$ weisen unterschiedliche Beträge, Risiken und typischerweise  Korrelationen auf.\\
\emph{Wir unterscheiden:}
\begin{itemize}
	\item \begriff{Einperiodenproblem}: Aufteilung wird heute $(t=0)$ festgelegt und bis zum Zeithorizont ($t=T$) beibehalten
	\item \begriff{Mehrperiodenproblem:} Umschichten zu mehreren Zeitpunkten $\set{t_0, t_1, \dots, t_N}$ möglich
\end{itemize}
\emph{Einfachstes Optimalitätsprinzip}: \begriff{\person{Pareto}-Optimalität}
\begin{itemize}
	\item Bei gleichem Risiko wird Anlage mit größeren Ertrag bevorzugt
	\item Bei gleichem Ertrag wird Anlage mit kleineren Risiko bevorzugt
\end{itemize}
d.h. \begriff{Pareto-Optimal} bedeutet, es gibt keine Anlagestrategie mit größerem Ertrag und kleinerem Risiko.\\
Zum Aufwärmen zwei Toy-Models ($=$ stark vereinfachte Beispiele)
\begin{itemize} %TODO fix structure 
	\item \emph{Toy-Model I}; Einperiodenmodell, eine risikofreie und eine risikobehaftete Anlagenmodel.
	\begin{itemize}
		\item Zeithorizont sei $T=1$
		\item risikofrei: $S_0^0 = 1$, $S_T^0 = (1+r)$
		\item risikobehaftet: $S_0^1 = 1$, $S_T^1 = (1+R)$ mit $R$ stochastisch,
		\begin{align*}
			\mu &= \E[R] \quad \text{Ertrag}\\
			\sigma &= \sqrt{\Var(R)} \quad \text{Risiko}
		\end{align*}
		\item $S=\mu -r$ Überrendite (excess return).
		\item $S \le 0 \implies$ Investiere alles in $S^0$ (Pareto-Optimal)
		\item $S > 0 \implies ?$
		\item Teile $W$ in $(W_0, W-W_0)$ auf $(S^0,S^1)$ auf (Jetzt: $W=1$)
		\begin{align*}
			\begin{cases}
				W-W_0 < 0 &\quad \text{Leerverkauf}\\
				W_0 < 0 &\quad \text{Kredit}
			\end{cases}
		\end{align*}
		\item Endvermögen: $P_T = W_0(1+r) + (1-W_0)(1+R)$
		\item Erweiterte Rendite: 
		\begin{align*}
			\mu_p &= \E[P_T-1] = W_0(1+r)+(W-W_0)(1+\mu)\\
			&= W_0 r + (1-W_0)\mu\\
		\end{align*}
		\item Risiko: $\sigma_p = (1-W_0)\sigma$
		\item Überrendite $S_p = (1-W_0)(\mu -r)$
		\item \emph{Jede} Strategie ist Pareto-Optimal, d.h. Pareto-Prinzip  hilft nicht bei der Auswahl. Insbesondere ist \begriff{Sharp-Ratio} 
		\begin{align*}
			SR(W_0) = \frac{\text{``Überrendite''}}{\text{``Risiko''}} = \frac{S_p}{\sigma_p} = \frac{\mu -r}{\sigma} \quad \text{konstant!}
		\end{align*}
	\end{itemize}
	\item \emph{Alternative} zum Pareto-Prinzip Festlegen von individueller Risikoaversion (mehr dazu später)
	\item \emph{Toy-Model II:} Einperiodenproblem, zwei risikobehaftete Anlagemöglichkeiten
	\begin{itemize}
		\item Zeithorizont $T=1$, Vermögen $W=1$
		\item 
		\begin{align*}
			\begin{matrix}
				S_0^1 = 1 & S_T^1 = (1+R_1) & \text{mit $\E[R_1] = \mu$, $\Var(R_1) = \sigma_1^2 > 0$}\\
				S_0^2 = 1 & S_T^2 = (1+R_1) & \text{mit $\E[R_2] = \mu$, $\Var(R_2) = \sigma_2^2 > 0$}
			\end{matrix}
		\end{align*}
		und $R_1 \upmodels R_2$ (unabhängig)
		\item Portofoliowert: $P_T = W_1(1+R_1) + (1-W_1)(1+R_2)$
		\item Rendite: $\mu_p = \E[P_T -1] = W_1\E[R_1] + (1-W_1)\E[R_2] = \mu$
		\item Risiko:
		\begin{align*}
			\sigma_p^2 = \Var(P_T -1) &= \Var(W_1 R_1 + (1-W_1)R_2)\\
			&= W^2_1\cdot \sigma_1^2 + (1-W_1)^2\sigma_2^2
		\end{align*}
		\item Pareto-Optimales Portofolio: 
		\begin{align*}
			0 &= 2W_1\cdot \sigma_1^2 - 2(1-W_1)\sigma_2^2
			\sigma_2^2\\
			&= W_1(\sigma_1^2 + \sigma_2^2)\\
			&\implies W_{\ast} = \frac{\sigma_2^2}{\sigma_1^2\cdot \sigma_2^2} \in (0,1)
		\end{align*}
		Also existiert genau eine Pareto-Optimale Strategie
		\item Vermögen wird proportional zum Verhältnis der Risiken aufgeteilt
		\item Vermögen wird \emph{nicht} vollständig in risiko-ärmere Anlagen gesteckt, also findet eine \begriff{Diversifikation} statt
		\item $W_{\ast}$ ist auch die Strategie mit maximaler \emph{Sharp-Ratio}
	\end{itemize}
\end{itemize}
Als nächstes: Pareto-Optimale Portofolio mit $n>2$ Anlagegütern
\section{Exkurs: Optimierung mit Nebenbedingung}
\emph{Betrachte Optimierungsproblem:}
\begin{align*}
	\min f_0(x)\quad x \in \R^n\\
	\intertext{unter Nebenbedingungen}
	\begin{cases}
		f_i(x) \le 0 &\quad i = 1, ..., m\\
		h_i(x) = 0 &\quad i = 1, ..., p
	\end{cases}
	\tag{OPT}\label{eq_4_1_opt}
\end{align*}
\begin{itemize}
	\item $x \in \R^n$ welches (NB) erfüllt heißt \begriff{zulässig}
	\item $x_{\ast} \in \R^n$ welches \eqref{eq_4_1_opt} normiert heißt \begriff{Optimallösung}
	\item $p_{\ast} = f_0(x_0)$ heißt \begriff{Minimalwert}
\end{itemize}
\begin{definition}
	\begin{enumerate}
		\item Die Funktion
		\begin{align*}
		\LL(x, \lambda, \nu) = f_0(x) + \sum_{i=1}^{m}f_i(x)\lambda_i + \sum_{i=1}^p h_i(x)\nu_i
		\end{align*}
		mit $\lambda \in \R^m_{\ge 0}, \nu \in \R^p$ heißt \begriff{Lagrange-Zielfunktion}
		\item Die Funktion
		\begin{align*}
			g(\lambda, \nu) = \inf_{x \in \R^m} \LL(x, \lambda, \nu)
		\end{align*}
		heißt \begriff{(Langrange-) duale Funktion} für \eqref{eq_4_1_opt}.
	\end{enumerate}
\end{definition}
\begin{*remark}
	Als Infimum von $g(\lambda, \nu)$ lineare Funktionen ist $g$ \emph{konkav}.
	\begin{itemize}
		\item Die duale Funktion $g$ erzeugt \emph{untere Schranken} für $p_{\ast}$
		\item Begründung Sei $\overline{x} \in \R^n$ zulässig für \eqref{eq_4_1_opt}, d.h.
		\begin{align*}
			f_i(\overline{x}) &\le 0 \quad \forall i \in [n]\\
			h_i(\overline{x}) &= 0 \quad \forall i \in [p]
		\end{align*}
		\begin{align*}
			\implies \LL(\overline{x}, \lambda, \nu) = f_0(\overline{x}) + \underbrace{\sum_{i=1} f_i(\overline{x})\lambda_i + \sum_{i=1} h_i(\overline{x})\nu_i}_{=0} \le f_0(\overline{x})
		\end{align*}
		Also $g(\lambda, \nu) = \inf_{x \in \R^m} \LL(x, \lambda, \nu) \le \LL(\overline{x}, \lambda, \nu) \le f_0(\overline{x}) \quad \forall \overline{x}$ zulässig und damit folgt
		\begin{align*}
			g(\lambda, \nu) \le p_{\ast} \quad \forall \lambda \in \R^m_{>0}, \nu \in \R^n
		\end{align*}
		Die beste untere Schranke erhalten wir durch \emph{maximieren} über $\lambda, \nu$
	\end{itemize}
\end{*remark}
\begin{definition}
	Das duale Optimierungsproblem zu \eqref{eq_4_1_opt} ist
	\begin{align*}
		\max g(\lambda, \nu) \quad \lambda \in \R^m, \nu \in \R^p
		\intertext{unter Nebenbedingungen}
		\lambda_i \ge 0 \quad i = 1, \dots m \tag{D}\label{eq_4_2_Duality}
	\end{align*}
	Maximalwert: $d_{\ast}$
\end{definition}
\begin{itemize}
	\item Zwischen \eqref{eq_4_1_opt} und \eqref{eq_4_2_Duality} gilt \begriff{schwache Dualität}
	\begin{align*}
		d_{\ast} \le p_{\ast}
	\end{align*}
	\item Unter bestimmten Voraussetzungen gilt auch die \begriff{starke Dualität}
	\begin{align*}
		d_{\ast} = p_{\ast}
	\end{align*}
\end{itemize}
Das duale Problem hat auch eine Lösung $(\lambda_{\ast}, \nu_{\ast})$ und Maximalwert $d_{\ast}$
\begin{lemma}
	Zwischen \eqref{eq_4_1_opt} und \eqref{eq_4_2_Duality} gilt die \begriff{schwache Dualität}
	\begin{align*}
		d_{\ast} \le p_{\ast}\tag{WD}\label{eq_4_3_weak_Duality}
	\end{align*}
\end{lemma}
\begin{proof}
	SeSt.
\end{proof}
\begin{*remark}
	\begin{itemize}
		\item Die Differenz $p_{\ast} - d_{\ast} \ge 0$ heißt \begriff{Dualitätslücke} (duality gap).
		\item Wenn Dualitätslücke verschwindet \begriff{starke Dualität}
		\begin{align*}
			d_{\ast} = p_{\ast}
		\end{align*}
		\item hinreichende Bedingungen für starke Dualität existieren vor allem für \emph{konvexe} Probleme
	\end{itemize}
\end{*remark}
\begin{definition}
	Optimierungsproblem\eqref{eq_4_1_opt} ist konvex wenn $f_0$ konvex ist und die Menge der zulässigen Werte konvex ist. In diesem Fall kann \eqref{eq_4_1_opt} in folgende Form gebracht werden:
	\begin{align*}
		\min f_0(x) \quad x \in \R^n\\
		\intertext{unter NB}
		\begin{cases}
			f_i(x) \le 0 &\quad i \in [m]\\
			Ax = b &\quad
		\end{cases}\tag{K-OPT}\label{eq_4_4_k-opt}
	\end{align*}
	mit $f_0, f_1, \dots, f_m$ konvex, $A \in \R^{p\times m}, b \in \R^p$.
\end{definition}
\begin{theorem}[Slaters-Bedingung]
	Betrachte das konvexe Optimierungsprobleme \eqref{eq_4_1_opt}. Wenn $x \in \R^n$ existiert mit
	\begin{align*}
		f_i(x) < 0 \quad \forall i \in [m] \und Ax = b
	\end{align*}
	dann gilt starke Dualität.
\end{theorem}
Für den Beweis verwende wir den Trennungssatz für konvexe Mengen.
\begin{theorem}[Trennungssatz für konvexe Mengen]
	Sei $A,B \subseteq \R^n$ konvex nichtleer und disjunkt, d.h.
	\begin{align*}
		A \cap B = \emptyset
	\end{align*}
	Dann existieren $a \in \R^n \setminus \set{0} \und b \in \R$, sodass
	\begin{align*}
		a^T x \ge b \quad \forall x \in A\\
		a^T x \le b \quad \forall x \in B
	\end{align*}
	Die Hyperebene $h = \set{x \in \R^n \colon a^T x = b}$ heisst \begriff{trennende Hyperbene} für $A \und B$
\end{theorem}
\begin{proof}
	Ohne Beweis.
\end{proof}
Skizze: 
\begin{proof}[Theorem 4.2?]
	Betrachte folgende Teilmengen von $\R^N = \R^{m+p+1}$, 
	\begin{align*}
		\G&=\set{(u,v,t) \in \R^N \colon \exists x \in \R^n \mit \begin{cases}
			f_i(x) = u_i &\quad \forall i \in [m], Ax - b = v\\
			f_0(x) = t &\quad
			\end{cases}}\\
		\AAA &= \set{(u,v,t) \in \R^N \colon \exists x \in \R^n \mit f_i(x) \le u_i \forall i \in [m], Ax-b = v, f_0(x) \le t} = \G \oplus \R^m_{\ge 0} \times \set{0}^p \times \R_{\ge 0}\\
		\BBB &= \set{(0,0,t)\in \R^N \colon t < p_{\ast}}
	\end{align*}
	Es gilt $\AAA \und \BBB$ sind konvex. Nun folgt die 
	\begin{itemize}
		\item Behauptung: $\AAA \cap \BBB = \emptyset$. Mit Widerspruch: Angenommen es existiert $(u,v,t) \in \AAA \cap \BBB$, dann gilt\\
		wegen $\BBB$
		\begin{align*}
		u=0, v = 0 \und t < p_{\ast}
		\end{align*}
		wegen $\AAA$
		\begin{align*}
		\exists x \in \R^n \mit &f_i(x) \le u_i = 0 \quad i \in [m]\\
		& h_i(x) = v_i = 0 \quad i \in [p]\\
		&f_0(x) \le t < p_{\ast}
		\end{align*}
		d.h. $x$ ist zulässig für \eqref{eq_4_4_kopt} und besser als optimal! $(f_0(x) < p_{\ast})$. Damit ist die Behauptung gezeigt und es folgt $\AAA \cap \BBB = \emptyset$.
		\item Wende Trennungssatz an: $\exists (\lambda, \nu, v) \in \R^N \setminus \set{0}$ und $\alpha \in \R$ mit
		\begin{align*}
			\lambda^T u + \nu^T v + \mu t \ge \alpha \quad (u,v,t) \in \AAA \tag{I}\label{proof_eq_4_2a}\\
			\lambda^T u + \nu^T v + \mu t \le \alpha \quad \forall (u,v,t) \in \BBB \tag{II}\label{proof_eq_4_2b}
		\end{align*}
		\eqref{proof_eq_4_2b} folgt $\mu t \le \alpha \forall t < p_{\ast}$ (da $a = 0, v = 0$) und damit gilt $\mu p_{\ast} \le \alpha$\\
		Aus \eqref{proof_eq_4_2a} bekommt man $\lambda_i \ge 0 \forall i \in [m]$ und $\mu \ge 0$ (sonst Widerspruch!). Dann nimmt man \eqref{proof_eq_4_2a} und \eqref{proof_eq_4_2b} zusammen und hat $\forall x \in \R^n$
		\begin{align*}
			&\sum_{i=1} \lambda_i f_i(x) + \sum_{i=1} \nu_i(Ax-b) + \mu f_0(x)\\
			&\le \lambda^T u + \nu^T v + \mu t \overset{\eqref{proof_eq_4_2a}}{\ge} \alpha \overset{\eqref{proof_eq_4_2b}}{\ge} \mu p_{\ast}\tag{$\ast$}\label{proof_eq_4_2c}
		\end{align*}
		Nun gibt es zwei Fälle:
		\begin{itemize}
			\item Fall: $\mu > 0$ Setze $\tilde{\lambda} = \lambda / mu$, $\tilde{\nu} = \nu /\mu$, damit\\
			$\exists x \in \R^n \mit \sum_{i=1} \tilde{\lambda}_i f_i(x) + \sum_{i=1} \nu_i (Ax-b) + f_0(x) \ge p_{\ast}$ und damit folgt $g(\tilde{\lambda}, \tilde{\nu}) \ge p_{\ast}$, d.h. $d_{\ast} = \max_{(\lambda, \nu)\in \R_{\ge 0}^m \times \R^n} g(\lambda, \nu) \ge p_{\ast}$. Aber mit schwacher Dualität: $d_{\ast} \le p_{\ast}$
			\item Fall: $\mu = 0$ (kann nicht eintreten, weil ...). Aus \eqref{proof_eq_4_2c} bekommen wir
			\begin{align*}
				\sum_{i=1} \lambda_i f_i(x) + \sum_{i=1} \nu_i(Ax-b) \ge 0 \quad \forall x \in \R^n
			\end{align*}
			Slaters-Bedingung $\exists \overline{x} \in \R^n$ mit $f_i(x) < 0, i \in [m]$ und $A\overline{x} - b = 0$ und damit
			\begin{align*}
				\sum_{i=1}\underbrace{\lambda_i}_{\ge 0}\underbrace{f_i(\overline{x})}_{< 0} > 0 \implies \lambda = 0
			\end{align*}
			$(\lambda, \nu, \mu) = (0, \nu, 0)\in \R^N \setminus \set{0}$ impliziert $\nu \neq 0$ und $\nu^T(Ax-b) = 0$, dann existiert nach $\overline{x}$ mit $\nu^T(A\overline{x} - b) < 0$ und das ist der Widerspruch, d.h. $\mu = 0$ tritt nicht ein.
		\end{itemize}
	\end{itemize}
\end{proof}
\section{Die \person{Markowitz}-Modelle}
\subsection*{Markowitz-Modell I}
(Portofolio-Optimierung \emph{ohne} risikofreie Anlage)\\
Anlagegüter $S = (S^1, \dots, S^n)$ mit stochastische ein-perioden Renditen $R = (R^1, \dots, R^n)$, d.h. $S_T = S_0^i(1+R^i), i \in [n]$ mit auf Analgegüter $S^1, \dots, S^n$ aufteilen $p_i$ Investitionen in $S^i$, d.h. $p_1 + \dots + p_n = W = 1$.
\begin{itemize}
	\item Erwartungswert: $\mu = \E[R] \in \Rn, \mu = (\mu_1, dots, \mu_n)^T$
	\item $\Sigma = \E[(R-\mu)(R-\mu)^T]\quad (n\times 1)(1\times n) = (\Sigma_{ij})_{i,j \in [n]}$
	\begin{align*}
		\Sigma_{ij} = \Var(R^i)\\
		\Sigma_{ij} = \Cov(R^{i},R^j) \mit i \neq j
	\end{align*}
	\begin{itemize}
		\item \emph{Annahme:} $\Sigma$ ist regulär, d.h. $\Sigma^{-1}$ existiert.
		\item \emph{Ziel:} Anlagemengen $W=1$. 
		\item \emph{Erwartete Rendite:} $\mu_p = \E[p^T R] = p^T \mu$
		\item \emph{Risiko (Standardbereich):}
		\begin{align*}
			\sigma_p &= \sqrt{\Var(p^T R)} = \sqrt{\E[(p^T (R-\mu))^2]}\\
			&=\sqrt{\E[p^T (R-\mu)(R-\mu)^Tp]} = \sqrt{p^T \Sigma p}
		\end{align*}
		\item \emph{Optimales Anlageproblem:} Minimiere Risiko, gegeben Zielrendite $\mu_{\ast}$
		\begin{align}
			\begin{cases}
				\min \half p^T \Sigma p & \quad\text{über } p \in \R^n\\
				\text{ unter NB} &\quad p^T \mu = \mu_{\ast} \text{ ( Zielrendite)}\\
				&\quad p^T \indi = \indi (\indi = (1, \dots, 1) \in \R^n)
			\end{cases} \label{eq_Markow_one}\tag{Mark I}
		\end{align}
	\end{itemize}
	\item Die Lagrange-Zielfunktion: 
\begin{align*}
	\LL(p,\lambda_1, \lambda_2) = \half p^T \Sigma p + \lambda_1(\mu_{\ast} - p^T\mu) + \lambda_2(1-p^T \indi) \quad \mit \lambda_1, \lambda_2 \in \R
\end{align*}
	\item Die duale Funktion:
\begin{align*}
	g(\lambda_1, \lambda_2) &= \inf_{p \in \Rn} \LL(p, \lambda_1, \lambda_2)\\
	\nabla_p \LL(p, \lambda_1, \lambda_2) &= \Sigma - \lambda_1 \mu - \lambda_2 1 = 0\\
	\implies p_{\ast} &= \Sigma^{-1}(\lambda_1 \mu + \lambda_2 \mu)
\end{align*}
d.h. $g(\lambda_1, \lambda_2) = \LL(p_{\ast}, \lambda_1, \lambda_2)$
\begin{align*}
	\LL(p_{\ast}, \lambda_1, \lambda_2) &=
	\half (\lambda_1 \mu + \lambda_2 1)^T \Sigma^{-1}\Sigma\Sigma^{-1}(\lambda_1\mu - \lambda_2 1) - (\lambda_1 \mu + \lambda_2 1)^T \Sigma^{-1}(\lambda_1 \mu - \lambda_2 1) + \lambda_1 \mu_{\ast} + \lambda_2\\
	&= -\half (\lambda_1^2 a + 2 \lambda_1 \lambda_2 b + \lambda_2^2 c) + \lambda_1 \mu_{\ast} + \lambda_2
\end{align*}
mit
\begin{align*}
	a = \mu^T \Sigma^{-1}\mu, b = \mu^T\Sigma\indi, c= \indi^T\Sigma\indi
\end{align*}
Es gilt $a \ge 0, c \ge 0$ und (mit Cauchy-Schwarz) und damit $ac \ge b^2$
\item Maximieren von $g$:
	\begin{align*}
		\frac{\partial g}{\partial \lambda_1} &= -a \lambda_1 - b \lambda_2 + \mu_{\ast} = 0 \implies a\lambda_1 + b \lambda_2 = \mu_{\ast} \tag{I}\label{eq_mark_one_I}\\
		\frac{\partial g}{\partial \lambda_2} &= -b \lambda_1 - 1 \lambda_2 - 1 = 0 \implies b\lambda_1 + c \lambda_2 = 1 \tag{II}\label{eq_mark_one_II}
	\end{align*}
	\begin{align*}
		-b\eqref{eq_mark_one_I} + a\eqref{eq_mark_one_II}:\quad (ac - b^2)\lambda_2 = a - b\mu_{\ast} &\implies \lambda^{\ast}_2 = \frac{a - b \mu_{\ast}}{ac - b^2}\\
		c\eqref{eq_mark_one_I} -b\eqref{eq_mark_one_II}:\quad (ac - b^2)\lambda_1 = c\mu_{\ast} - b &\implies \lambda^{\ast}_2 = \frac{c\mu_{\ast} - b }{ac - b^2} \quad (\text{aber nur für } ac > b^2 )
	\end{align*}
\item Minimirer von \eqref{eq_Markow_one}:
\begin{align*}
	p_{\ast} = \lambda_1^{\ast}\Sigma^{-1}\mu + \lambda_2^{\ast}\Sigma^{-1}\indi
\end{align*}
\end{itemize}
\begin{conclusion}[Tobin-Two-Fund Seperation]
	Jeses Pareto-Optimale Portofolio für \eqref{eq_Markow_one} kann (unabhängig von $a$!) als Linearkombination der zwei Portolio
	\begin{align*}
		\underbrace{p^{\ast}_1 = \Sigma^{-1}\mu}_{\text{renditeorientiertes Portofolio}} \und \underbrace{p_2^{\ast} = \Sigma^{-1}\indi}_{\text{sicherheitsorientiertes Portofolio}}
	\end{align*}
	dargestellt werden.
\end{conclusion}
\begin{*remark}
	\begin{itemize}
		\item Gewichtung des Portofolios $p^{\ast}_1 \und p_2^{\ast}$ orientiert sich am Renditeziel $\mu_{\ast}$.
		\item $p^{\ast}_1 \und p_2^{\ast}$ sind breit diversifiziert, d.h. nutzen alle Anlagegüter $S = (S^1, \dots, S^n)$
		\item $p^{\ast}_1 \und p_2^{\ast}$ kann man als Anlagefunds interpretieren welche Vermögen entsprechend der Portfolio $p^{\ast}_1, p^{\ast}_2$ anlegen. Diese zwei Fonds sind ausreichend (unabhängig von $\mu_{\ast}$) um Vermögen Pareto-optimal zu investieren!
	\end{itemize}
\end{*remark}
Zueletzt wollen wir noch Risiko der optimalen Strategie $p_{\ast}$ berechnen:
\begin{align*}
	\sigma_{\ast}^2 &= \Var(p_{\ast}^T R) = \E[(p_{\ast}^T (R-\mu))^2] = p_{\ast}^T \Sigma p_{\ast}\\
	&= (\lambda_1^{\ast}\mu + \lambda_2^{\ast}\indi)^T \Sigma^{-1}\Sigma\Sigma^{-1}(\lambda_1^{\ast}\mu + \lambda_2^{\ast}\indi)\\
	&= (\lambda_1^{\ast 2}2 a) + 2 \lambda_1^{\ast}\lambda_2^{\ast} b + (\lambda_2^{\ast 2})c\\
	&= \frac{1}{a^2-b^2}^2 (((c\mu_{\ast} -b)^2)a + 2(\mu_{\ast} - b)(a-b \mu_{\ast})b + (a - b\mu_{\ast})^2c)\\
	&= \frac{1}{ac-b^2}(c \mu_{\ast}^2 - 2b\mu_{\ast} +a^2)
\end{align*}
Graph von $(\sigma_{\ast}, \mu_{\ast})$ ist ein Hyperbel-ast:\\
siehe picture phone ...\\
Nennt sich ``Markowitz-Bullet''!
\subsection*{Markowitz-Modell II}
(Optimale Investition \emph{mit} risikofreier Anlage)\\
\begin{itemize}
	\item Anlagegüter $S = (S^1, \dots, S^n)$ mit ein-perioden Rendite $R = (R^1, \dots, R^n)$
	\item Zusätzlich risikofreie Anlage $S^0$ mit Verzinsung $r$. Wegen $W=1$aufgestellt zu $1 = p_0 + p_1 + \dots p_n$. Wir setzen $p = (p_1,\dots, p_n)^T \in \R^n$
	\item Erwartete Rendite: $\mu = \E[p^T R + (1-p^T \indi)r] = p^T(\mu - r\indi) + r$
	\item Risiko: $\sigma_{\ast} = \sqrt{\Var(p^T R)} = \sqrt{p^T \Sigma p}$
	\item Anlageproblem:
	\begin{align*}
		\begin{cases}
			\min \half p^T \Sigma p & \quad p \in \R^n\\
			\text{ unter NB} &\quad p^T (\mu-r\indi) = \mu_{\ast} -r (\text{ Zielrendite})\\
		\end{cases}\tag{Mark II}\label{eq_Markow_two}
	\end{align*}
	\item Lagrange ÜA
	\item Optimierer: $p_{\ast} = \lambda_{\ast}\Sigma^{-1}(\mu - r \indi)$ mit $\lambda_{\ast} = \frac{\mu_{\ast} - r}{a^2 - 2br + cr^2}$
\end{itemize}
\begin{conclusion}[Tobin's One-Fund-Theorem]
	Jedes Pareto-Optimale Portfolio für \eqref{eq_Markow_two} kann als Linearkombination der risko-freien Anlage und des Portfolio
	\begin{align*}
		\Sigma^{-1}(\mu -r\indi)
	\end{align*}
	dargestellt werden.
\end{conclusion}
Graph von min. Risko $\sigma_{\ast}$ und Zielrendite $\mu_{\ast}$
siehe phone
\begin{*remark}[nominales vs. relatives Portfolio]
	\begin{itemize}
		\item $\vartheta = (\vartheta_1, \dots, \vartheta_2) \in \R^n$ mit $\vartheta_i$ Stückzahl von Anlagegut $S^i$ und Portofoliowert
		\begin{align*}
			V_0 = \vartheta^T S_0 = \sum_{i=1}^n \vartheta_i S_0^i = w \quad \dots \text{Anfangskapital}\\
			V_T = \vartheta^T S_T = \sum_{i=1}\vartheta_i S_T^i
		\end{align*}
		\item relatives Portfolio: $p := (p_1, \dots, p_n) \in \R^n$ mit $p_i = \frac{\vartheta_i S_0^i}{W}$ Vermögensanteil in $S^i$
		\begin{align*}
			\sum_{i=1}^n p_i = \frac{1}{w}\sum_{i=1}^n \vartheta_i S_0^i = w/w = 1
		\end{align*}
		\item Renditen: 
		\begin{itemize}
			\item Einzelnes Anlagegut: $R_i = \frac{S_T^i - S_0^i}{S_0^i}$
			\item Gesamtes Portfolio: 
			\begin{align*}
				R_p &= \frac{V_T - V_0}{V_0} = 1/w (\sum_{i=1}^n \vartheta_i S_0^i - \vartheta_i S_0^i)\\
				&= 1/w \sum_{i=1}^n \vartheta_i(S_T^i - S_0^i) = \sum_{i=1}^n \underbrace{\frac{\vartheta_i S_0^i}{w}_{p_i}}R_i\\
				&= \sum_{i=1}^n p_i R_i = p^T R \quad \text{... linear in }p
			\end{align*}
		\end{itemize}
	\end{itemize}
\end{*remark}

\part*{Anhang}
\addcontentsline{toc}{part}{Anhang}
\appendix

\nocite{*}
\bibliography{literatur}
\bibliographystyle{acm}

%\printglossary[type=\acronymtype]

\printindex

\end{document}