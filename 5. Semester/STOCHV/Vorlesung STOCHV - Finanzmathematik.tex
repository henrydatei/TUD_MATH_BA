\documentclass[ngerman,a4paper,order=firstname]{../../texmf/tex/latex/mathscript/mathscript}
%{local_mathscript}
%\usepackage{../../texmf/tex/latex/mathoperators/mathoperators} 
\usepackage{local_mathoperators}  % for ilina xD - local mathoperators

%local packages
\usepackage{cancel}
\usepackage{amssymb}

\title{\textbf{Stochastikvertiefung: Finanzmathematik WS 19/20}}
\author{Dozent: Prof. \person{Martin.Keller-Ressel}}

% local commands
%\renewcommand{\F}{\mathscr F}
\renewcommand{\P}{\mathbb{P}}
\newcommand{\QQ}{\mathbb{Q}}
\renewcommand{\O}{\Omega}
\renewcommand{\G}{\mathscr{G}}
\newcommand{\AAA}{\mathscr{A}}
\newcommand{\MM}{\mathcal M}
\newcommand{\BBB}{\mathscr{B}}
\renewcommand{\H}{\mathscr{H}}
\newcommand{\Gen}{\mathcal{E}}
\newcommand{\E}{\mathbb{E}}
\newcommand{\Var}{\mathbb{V}\text{ar}}   % Varianz
\newcommand{\Cov}{\mathbb{C}\text{ov}}   % Kovarianz
\newcommand{\Corr}{\mathbb{C}\text{orr}}   % Korrelation
%\newcommand{\Rd}{\R^d}
\def\upmodels{\perp\!\!\!\perp}
\newcommand{\distri}{\overset{\text{d}}{=}}
\newcommand{\konverteil}{\xrightarrow[n \to \infty]{\d}}
\newcommand{\normal}{\mathscr N} % Normal distribution
\newcommand{\bigcdot}{\boldsymbol{\cdot}} % predictable stoch process product
\DeclareMathOperator{\dist}{dist}
\DeclareMathOperator{\plim}{plim} % stochastic convergence or convengence in probability
\DeclareMathOperator{\CRR}{CRR}			
\newcommand{\half}{\frac{1}{2}}				% CRR-model sequence
\newcommand{\ZZ}{\mathcal Z }				% 
\newcommand{\NNN}{\mathcal N }				% normal distribution N
\newcommand{\OO}{\mathcal O }				% Big Ohhhh
\newcommand{\LL}{\mathcal L}				% Lagrange L

% % % % % % % % %
% from https://tex.stackexchange.com/questions/319330/notation-for-proper-normal-subgroup
\DeclareFontFamily{U}{matha}{\hyphenchar\font45}
\DeclareFontShape{U}{matha}{m}{n}{ <-6> matha5 <6-7> matha6 <7-8>
	matha7 <8-9> matha8 <9-10> matha9 <10-12> matha10 <12-> matha12 }{}
\DeclareSymbolFont{matha}{U}{matha}{m}{n}
%
\DeclareMathSymbol{\nvartrianglelefteq}{\mathrel}{matha}{"9E}
\DeclareMathSymbol{\vartrianglelefteq}{\mathrel}{matha}{"9C}

\begin{document}
\renewcommand{\F}{\mathscr F}
\pagenumbering{roman}
\pagestyle{plain}

\maketitle

\hypertarget{tocpage}{}
\tableofcontents
\bookmark[dest=tocpage,level=1]{Inhaltsverzeichnis}

\pagebreak
\pagenumbering{arabic}
\pagestyle{fancy}

\chapter*{Vorwort}
%Due to the coronavirus crisis, this is reading course only with online chats. Lecture notes were supplied from an older course, which we want to summarize here and add additional remarks to them. Proofs will be sketched with added remarks.

\chapter{Einführung}
\section{Zentrale Fragestellung der Finanzmathematik}
\subsection*{\begriff{Bewertung}:}
Bewertung von Derivaten und \emph{Absicherung} gegen aus Kauf/Verkauf entstehenden Risiken.

\begin{*definition}[\begriff{Derivat}]
	FInanzprodukt, dessen auszahlungen sich vom Preis einer oder mehrer \begriff{Basisgüter} (underlying) ableitet (ableiten entspricht derivate)
\end{*definition}
\begin{*example}
	\begin{itemize}
		\item Recht, in 3 Monaten 100.000 GBP gegen 125.000 EUR zu erhalten (\begriff{Call-Option}, Underlying: Wechselkurs GBP/EUR)
		\item Recht, innerhalb des nächsten Jahres 100.000 Mwh elektrischer Energie zum Preis von 30EUR/Mwh zu konsumieren mit Mindestabnahme 50.000 Mwh (\begriff{Swing-Option}, Underlying: Strompreis)
		\item Kauf- und Verkaufsoptionen aus Aktien (Underlying: Aktienkurs)
	\end{itemize}
\end{*example}
Fragestellung: Was ist der ``faire'' Preis für solch ein Derivat? (``Pricing''/Bewertung). Wie kann sicher der Verkäufer gegen eingegangenen Risiken absichern? (``Hedging''/Absicherung)
\subsection*{\begriff{Optimale Investition}}
Zusammenstellung von Portofolios, welche nach Risiken/Ertragsgesichtspunkten optimal sind
\begin{itemize}
	\item Wie wäge ich Risiken gegen Ertrag ab?
	\item Was genau bedeutet ``optimal''?
	\item Lösung des resultierenden Optimierungsproblems
\end{itemize}
\subsection*{\begriff{Risikomangement + Risikomessung}}
\begin{itemize}
	\item Gesetzliche Vorschriften (Basel + Solvency) sollen Stabilität des Banken-/Verischerunssystems auch angesichts verschiedener Risiken sicherstellen $\implies$ mathematische Theorie der konvexen und kohärenten Risikomaße
\end{itemize}
Mathematische Werkzeuge: Wtheorie + stochastische Prozesse (Dynamik in der Zeit), etwas lineare Algebra, Optimierung, Maßtheorie
\section{Mathematisches Finanzmodell}
Wir betrachten
\begin{enumerate}
	\item \emph{WRaum} $(\O,\F,\P)$, später auch weitere W-Maße $Q, \dots$ auf demselben Maßraum $(\O,\F), \omega \in \O$ Elementarereignisse bzw. ``Szenarien''
	\item \emph{Zeitachse} $I$ entweder $I=\set{t_0, t_1, \dots, t_N=T}$ $N$-Periode Modell (diskretes Modell) oder $I = [0,T]$ (zeitstetiges Modell), wobei $T = $ Zeithorizont\\
	Ein \begriff{stochasticher Prozess} $S$ ist eine messbare Abbildung $S: (\O,\F) \to \Rd \mit (\omega, t) \mapsto S_t(\omega)$\\
	insbesondere ist 
	\begin{itemize}
		\item $t \mapsto S_t(\omega)$ Funktion $I \to \Rd$ für jedes $\omega \in \O$ (``Pfad'')
		\item $\omega \mapsto S_t(\omega)$ Zufallsvariable $\O \to \Rd$ für jedes $t \in I$
	\end{itemize}
	\item \emph{Filtration} ist Folge von $\omega$-Algebren $(\F_t)_{t \in I}$ mit der Eigenschaft $\F_S \subseteq \F_t \quad \forall s,t \in I, x \le t \und \F_t \subseteq \F\quad \forall t \in I$\\
	Interpretation: $\F_t=$dem Marktteilnehmer zum Zeitpunkt $t$ bekannte/ verfügbare Informationen\\
	Ereignisse $A \in \F_t$ gelten als ``zum Zeitpunkt $t$'' bekannt\\
	Eine $\Rd$-wertige ZV $X$ heißt \begriff{$\F_t$-messbar}, wenn $E = X^{-1}(B) \in \F_t \quad \forall$ Borelmengen $B \subseteq \Rd$ (dabei ist $E$ das Urbild von $B$)

\begin{*example}
	Ein stochastischer Prozess $(S_t)_{t\in I}$ auf $(\O,\F)$ heißt \begriff[stochasticher Prozess]{adaptiert} bezüglich einer Filtration $(\F_t)_{t \in I}$, wenn gilt:
	\begin{align*}
		S_t \text{ ist } \F_t-\text{messbar} \quad \forall t \in I
	\end{align*}
\end{*example}
Interpretation: ``der Wert $S_t$ ist zum Zeitpunkt $t$ bekannt''\\
Warum Filtration in der Finanzmathematik (FiMa)?
\begin{itemize}
	\item Unterscheidung Zukunft / Vergangenheit
	\item unterschiedliche Information (Insider/Outsider) entspricht unterschiedlicher Filtration $(\F_t)_{t \in I}$ bzw. $(\G_t)_{t\in I}$
\end{itemize}
	\item \begriff{Anlagegüter (assets)} $\R^{d+1}$-wertiger stochastischer Prozess mit Komponenten
	\begin{align*}
		S^i: (\O \times I) \to \R\quad (\omega,t) \mapsto S^i_t(\omega) \mit i \in \set{0,1,\dots,d}
	\end{align*} 
	wobei $S^i_t=$ Preis des $i$-ten Anlageguts zum Zeitpunkt $t$\\
	$S^i, i \in \set{1,\dots,d}$ ist typischerweise
	\begin{itemize}
		\item Aktie (Stock), Unternehmensanteil
		\item Währung (currency) bzw. Wechselkurs
		\item Rohstoff (commodity) wie z.B. Öl, Edelmetall, Elektriziät, etc
		\item Anleihe (bond) ... Schuldverschreibung
	\end{itemize}
	Hauptannahme: $S^i$ ist liquide gehandelt (z.B. an Börse), d.h. Kauf/Verkauf zum Preis $S^i_t$ jederzeit möglich\\
	$S^0\dots$ ``Numeraire'' hat Sonderrolle: beschreibt Verzinsung von \emph{nicht} in $(S^1,\dots,S^d)$ angelegten Kapital, wird meist \emph{risikolos} betrachtet
\end{enumerate}
\begin{definition}[Finanzmodell]
	Ein \begriff{Finanzmodell} (FMM) mit Zeitachse $I$ ist gegeben durch
	\begin{enumerate}
		\item einen WRaum $(\O, \F,\P)$ mit Filtration $(\F_t)_{t\in I}$
		\item einen an $(\F_t)_{t \in I}$ adaptieren, $\R^{d+1}$-wertigen stochastischen Prozess $S_t = (S^0_t, S_t^1, \dots, S^d_t),t \in I$
	\end{enumerate} 
\end{definition}
\begin{*example}[\person{Cox}-\person{Rubinstein} (CRR)-Modell (zeitdiskret)]
	\begin{itemize}
		\item $S^0_n = (1+r)^n$, d.h. Verzinsung mit konstanter Rate $r$
		\item $S^1_n = S_0^1 \prod_{k=1}^n(1+Ru)$, wobei $(R_1, R_2, \dots)$ unabhängig ZVen mit zwei möglichen Werten $a < b$\\
		Bild: ``rekonbinierter Baum'' mit Ereignissen $\omega$ entsprechen Pfaden in dem Baum
	\end{itemize}
\end{*example}
\begin{*example}[\person{Block}-\person{Scholes}-Modell (zeitstetig)]
	\begin{itemize}
		\item $S^0_t = e^{rt}$, d.h. Verzinsung mit konstanter Rate $r$
		\item $S_t^1 = S_0^1\cdot \exp((\mu - \frac{\sigma^2}{2}t + \sigma\beta_t) \mit \mu \in \R, \sigma > 0, S^1_0 >0$ und $\beta_t$ entspricht Brownscher Bewegung (stochastischer Prozess in stetiger Zeit) und $\mu - \frac{\omega^2}{2}$ entspricht Trendkomponente
	\end{itemize}
	Bild: Börsenkurve = $S_t(\omega)$, wobei zeitstetiges Modell auf unendlichen W-Raum
\end{*example}
\section{Anleihen und grundeliegende Beispiele für Derivate}
Hier betrachten wir immer nur ein Basisgut $S_t = S^1_t$
\begin{enumerate}
	\item \begriff{Anleihe}(\begriff{bond}): (genauer: Null-Coupon-Anleihe [zero-coupon-bond]) Der \begriff{Emittent} (Herausgeber) einer Anleihe mit Endfälligkeit $T$ [maturity] garantiert dem Käufer zum Zeitpunkt $T$ den Betrag $N$ (EUR/USD/...) zu zahlen.\\
	Typische Emittenten:
	\begin{itemize}
		\item Staaten [government bond]
		\item Unternehmen (als Alternative zur Kreditaufnahme)
	\end{itemize}
	Nach Emission werden Anleihen auf den Sekundärmarkt weiterverkauft, d.h. liquide gehandelte Wertpapiere\\
	Preis bei Emission: $B(0,T)$\\
	Preis bei Weiterverkauf zum Zeitpunkt $t \le T\colon B(t,T)$\\
	Wir normieren stets $N=1 \implies B(T,T) =1$\\
	Anleihen von West/Nord/Mitteleuropäischen Staaten + USA/Kananda werden als risikolos betrachtet (sichere Zahlung).\\
	Sonst: Kreditrisiko\\
	Risikofreie Anleihen können als Numerale $S^0_t = B(t,T)$ genutzt werden\\
	Bild: kann ich gerade nicht beschreiben :/ \\
	\item \begriff{Terminvertrag} [forward contract]\\
	Aus Käufersicht: \emph{Vereinbarung} zu bestimmten, zukünftigen Zeitpunkt $T$ eine Einheit des Basisguts $S$ zum Preis zu kaufen (Kaufverpflichtung)\\
	Beliebt bei Rohstoffen + Elektrizität\\
	Auszahlunsprofil: $F_T = S_T - K$\\
	Bild: ``Eine Gerade mit Schnittpunkt der $x$-Achse bei $K$ und Schnittpkt der $y$-Achse bei $S_T \ge 0$, ist ja nur einer Polynom 1. Ordnung''\\
	Preis zum Zeitpunkt $t$: $F_t$
	\item \begriff{Europäische Put-/Call-Option}:
	Recht zu einem zukünftigen Zeitpunkt $T$ eine Einheit des Basisguts $S$ zum Preis $K$ zu verkaufen (Put) bzw. zu kaufen (Call) \textbf{keine (Ver-)Kaufsverpflichtung}\\
	\begin{itemize}
		\item \emph{Call}:
		\begin{align*}
			C_T := \begin{cases}
				S_T - K &\quad S_T \ge K\\
				0 &\quad S_T < K
			\end{cases} = (S_T - K)_+ %is this really a ``+'' or a t?!
		\end{align*}
		\begin{*remark}
			\begin{align*}
				X_+ &= \max(X,0)\quad X_+ - X__ = X\\
				X__ &= \min(X,0)\quad X_+ + X__ = \abs{X}
			\end{align*}
		\end{*remark}
			Bild: (hockey stick function)
		\item \emph{Put}:
			\begin{align*}
				P_t = \begin{cases}
					0 &\quad S_T \ge K\\
					K-S_T &\quad S_t < K
				\end{cases} = (K-S_T)_+
			\end{align*}
			Bild: ``inversed'' hockey stick function xD
	\end{itemize}
	\item \emph{Amerikanische Put/Call-Option}: Wie Put/Call aber mit Ausübung zu beliebigen Zeitpunkt $t \in [0,T]$\\
	Preis zum Zeitpunkt $t\colon P_t^{AM}, \; C_t^{AM}$\\
	Auszahlungsprofil zum zeitpunkt $\tau\colon (S_{\tau}-K), (K-S_{\tau})_+$\\
	Zeitpunkt $\tau$ muss im Allgemeinen als Lösung eines stochastischen Optimierungsproblems bestimmt werden (``\begriff{Optimales Stopproblem}'')
\end{enumerate}
\section{Elementare Replikations und Arbitrage-Argumente}
Was können wir (mit elementaren Mitteln) über die ``fairen'' Preise $B(t,T), F_t, C_t, P_t$ aussagen?\\
Wir verwenden:
\begin{itemize}
	\item \begriff{Replikationsprinzip}: Zwei identische zukünftige Zahlungsströme haben auch heute denselben Wert. (ein Zahlungstrom ``repliziert'' den anderen) % but what about inflation? Where comes inflation in this whole system in? 
	\item \begriff{No-Arbitage-Prinzip}: ``Ohne Kapiteleinsatz kann sicherer Gewinn ohne Verlustrisiko erzielt werden''
	\item \begriff{Arbitrage}: risikofreier Gewinn\\
	\item Schwächere Form des Replikationsprinzips:\\
	\begriff{Superpositionsprinzip}: Ist ein Zahlungsstrom in jedem Fall größer als ein anderer, so hat er auch heute den größeren Wert
	\begin{align*}
		\begin{matrix}
			\text{stark} & \text{Rep. Prinzip} & \text{eingeschränkt anwendbar}\\
			\downarrow & \text{Superrep. Prinzip} & \uparrow\\
			\text{schwach} & \text{No-Arbitrage-Prinzip} & \text{immer anwendbar}
		\end{matrix}
	\end{align*}
\end{itemize}
\begin{lemma} %1.1 fix numbering later ;)
	Für den preis $C_t$ des europäischen Calls gilt:
	\begin{align*}
		(S_t - K\cdot B(t,T))_+ \le C_t \le S_t
	\end{align*}
\end{lemma}
\begin{proof}
	\begin{itemize}
		\item \emph{untere Schranke}: Für Widerspruch $S_t - K\cdot(B(t,T))-C_t = \epsilon > 0$\\
		\begin{tabular}{l|l|l|l} % jeezzz i fuckin hate tables in latex xD, but i managed it ...
			Portofolio & Wert in $t$ & Wert in $T$, $S_t \le K$ & Wert in $T$, $S_t > K$\\
			Kaufe Call & $C_t$ & 0 & $S_T - K$\\
			Verkaufe Basisgut & $-S_t$ & $-S_T$ & $-S_T$\\
			Kaufe Anleihe & $\epsilon + K\cdot B(t,T)$ & $\frac{\epsilon}{B(t,T)}+K$ & $\frac{\epsilon}{B(t,T)} + K$\\
			$\Sigma$ & 0 & $K - S_T + \frac{\epsilon}{B(t,T)} > 0$ & $\frac{\epsilon}{B(t,T)} > 0$\\
			& keine Anfangskapital & sicherer Gewinn & sicherer Gewinn\\
		\end{tabular}\\
		$\implies$ Widerspruch zu No-Arbitrage\\
		$\implies$ $S_t - K\cdot B(t,T) \le C_t$ und Ausserdem $C_t \ge 0 \implies C_t \ge (S_t - K\cdot B(t,T))_+$
		\item \emph{obere Schranke}: UE
	\end{itemize}
\end{proof}
\begin{lemma}[Put-Call-Parität] % jeezz its 5:48 am im typing this, cant sleep ... :(
	Für Put $P_t$, Call $C_t$ mit demselben Ausübungspreis $K$ und Basisgut $S_t$ gilt
	\begin{align*}
		C_t - P_t = S_t - B(t,T)K
	\end{align*}
	Bild: need to add ..., but should be fast to do ...
\end{lemma}
\begin{proof} %TODO fix tables later, for now it works ..., make them one size for better reading ...
	mit Replikation:\\
	\begin{tabular}{l|l|l|l} % jeezzz i fuckin know now how to make tables in latex xD
		Portofolio 1 & Wert in $t$ & Wert in $T$, $S_t \le K$ & Wert in $T$, $S_t > K$\\
		Kaufe Call & $C_t$ & 0 & $S_T - K$\\
		Kaufe Anleihe & $K \cdot B(t,T)$ & $K$ & $K$\\
		Wert Portofolio 1 & $C_t + K\cdot B(t,T)$ & $K$ & $S_T$\\
	\end{tabular}\\
	\newline
	\begin{tabular}{l|l|l|l} % jeezzz i fuckin hate, but i can copy ... xD
		Portofolio 2 & Wert in $t$ & Wert in $T$, $S_t \le K$ & Wert in $T$, $S_t > K$\\
		Kaufe Put & $P_t$ & $K-S_T$ & 0\\
		Kaufe Basisgut & $S_t$ & $S_T$ & $S_T$\\
		Wert Portofolio 2 & $P_t + S_t$ & $K$ & $S_T$\\
	\end{tabular}\\
	Replikationsprinzip: $C_t + K\cdot B(t,T) = P_t + S_t$\\
	$\implies$ $C_t - P_t = S_t - K\cdot B(t,T)$
\end{proof} % 5:56 am and im done with it, puh ;) hope we can talk about it in english if you like. ;)
\section{Bedingte Erwartungswerte und Martingale} %1.5
\subsection{Bedingte Dichte und bedingter Erwartungswert}
Motivation: Gegeben: Zwei ZVen $(X,Y)$ mit Werten in $\R^m \times \R^n$ und gemeinsame Dichte $f_{XY}(x,y)$. Aus $f_{XY}$ können wir ableiten:
\begin{itemize}
	\item $f_{Y}(y) := \int_{\R^m} f_{XY}(x,y) \d x$ mit Randverteilung von $Y$
	\item $S_Y := \set{y \in \Rn \colon f_Y(y) > 0}$ Träger von $Y$ - Bild?
\end{itemize}
\begin{*definition}[Bedingte Dichte von $X$ bezüglich $Y$]
	Bedingte Dichte von $X$ bezüglich $Y$ ist definiert als
	\begin{align*}
		f_{X\mid Y}(x,y) = \begin{cases}
		\frac{f_{XY}(x,y)}{f_Y(y)} &\quad y \in S_Y\\
		0 &\quad y\notin S_Y
		\end{cases}
	\end{align*}
\end{*definition}
Betrachte folgende Problemstellung:\\
Was ist die beste Vorhersage von $X$ gegeben einer Beobachtung $Y = y$?\\
Kriterium:\\
Minimiere quadratischen Abstand/ zweite Moment/ $L_2$-Norm.\\
Vorhersage:\\
Messbare Funktion $g: \Rn \to \R^m \mit y \mapsto g(y)$, d.h,.
\begin{align*}
	\min\set{\E[(X-g(Y))]^2 \colon g \text{ messbar } \R^n \to \R^m} \tag{min-1}\label{eq_min_1}
\end{align*}
\begin{proposition} %1.3
	Wenn $(X,Y)$ eine gemeinsame Dichte besitzen mit $\E[\abs{X}^2] < \infty$ gilt, dann wird \eqref{eq_min_1} minimiert durch die bedingte Erwartung
	\begin{align*}
		g(y) = \E[X\mid Y=y] := \int_{\R^m} x f_{X\mid Y}(x,y)\d x
	\end{align*}
	(wobei $\E[X\mid Y=y]$ ``Erwartungswert von $X$ bedingt auf $Y=y$'')
\end{proposition}
Allgemeiner gilt:
\begin{theorem} %1.4
	Seien $(X,Y)$ ZVen mit gemeinsamer Dichte auf $\R^m \times \Rn$, $h: \R^m \to \R^n$ messbar mit $\E[h(X,y)^2]$. Dann wird das Minimierungsproblem
	\begin{align*}
		\min\set{\E[(h(X,Y) - g(y))^2]} \quad g\text{messbar von $\Rn$ nach $\R$}
		\intertext{gelöst durch}
		g(y) = \E[h(X,Y) \mid Y=y] = \int_{\R^m} h(X,Y)f_{X\mid Y}(x,y) \d x
	\end{align*}
\end{theorem}
\begin{proof}[nur Prop, Theorem analog, für $n=1$]
	Setze $g(y) = \int_{\R} f_{X\mid Y}(x,y) \d x$. Sei $p: \R \to \R$ beliebige messbare Funktion mit $\E[p(y)^2] < \infty$. Setze $g_{\epsilon}(y) = g(y) + \epsilon p(y)$. Minimiere
	\begin{align*}
		F(\epsilon) &:= \E[(X-g_{\epsilon}(y))^2] = \E[(X-g(y)-\epsilon p(y))^2]\\
		&= \E[(X-g(y))^2] - 2\epsilon\E[(X-g(y))p(y)] + \epsilon^2\E[p(y)^2]\\
		\frac{\partial F}{\partial \epsilon}(\epsilon) &= 2 \epsilon \E[p(y)^2] - 2\E[(X-g(y))p(y)]\\
		&\implies \epsilon_{\ast} :=\frac{\E[(X-g(y))p(y)]}{\E[p(y)^2]} = \frac{A}{B}
		\intertext{wobei}
		A&= \E[Xp(y)] -\E[g(y)p(y)] \\
		&= \int_{\R \times \Rn} xp(y)f_{XY}(x,y)\d x \d y - \int_{S_y}g(y)p(y)f_Y(y) = [\text{Einsetzen von $g$ + Fubini}]\\
		&= \int_{\R \times \Rn} xp(y)f_{XY}(x,y)\d x \d y - \int_{\R\times S_y} xp(y)\underbrace{f_{X\mid Y}(x,y)f_Y(y\d y)}_{=f_{XY}(x,y)} = 0
	\end{align*}
	also $\epsilon^{\ast} = 0$ unabhängig von $p$ $\implies g(y)$ minimiert \eqref{eq_min_1}.
\end{proof}
\begin{*example}
	Seien $(X,Y)$ normalverteilt auf $\R\times \R$ mit 
		\begin{align*}
			\mu = (\mu_x, \mu_y)^T \quad \Sigma = \begin{pmatrix}
			\sigma x^2 \rho\sigma_x \sigma_y\\
			\rho \sigma_x \sigma_y & \sigma_y^2
			\end{pmatrix} = \begin{pmatrix}
				\Var(X) & \Cov(X,Y)\\
				\Cov(X,Y) & \Var(Y)
			\end{pmatrix} \mit \rho \in [-1,1]
		\end{align*}
		Dann ist die beliebige Dichte $f_{X\mid Y}(x,y)$. ($\Sigma$ Kovarianzmatrix). wieder die Dichte einer Normalverteilung mit
		\begin{align*}
			\E[X \mid Y=y] &= \mu_x + \rho \frac{\sigma_x}{\sigma_y}(y-\mu_y)\\
			\Var(X\mid Y=y) &= \sigma_x^2(1-\rho^2)
		\end{align*}
		(ist ÜA!). Die Abbildung $y \mapsto \mu_x + g(y)\frac{\sigma_x}{\sigma_y}(y-\mu_y)$ heißt Regressionsgerade für $X$ gegeben $Y=y$.\\
		Bild: $\mu_x,\mu_y$ sind Werte auf $x,y$-Achse und die $\sigma$'s bilden das Steigungsdreieck (Steigung im Wesentlichen durch $\rho$ bekannt)\\
		Für diskrete ZVen, d.h. wenn $X,Y$ nur endlich viele $\set{x_1,\dots,x_m}$ bzw. $\set{y_1,\dots,y_m}$ annehmen dann erhalten wir mit ähnlichen Überlegungen als Lösung von \eqref{eq_min_1}
		\begin{align*}
			\E[X\mid Y=y_j] = \sum_{i=1}^m X_i \P(X=x_i \mid Y=y_j)
		\end{align*}
		wobei direkt die bedingten Wahrscheinlichkeiten
		\begin{align*}
			\P(X=x_i \mid Y=y_j) = \begin{cases}
			\frac{\P(X=x_i \wedge Y=y_j)}{\P(Y=y_j)} &\quad \text{ wenn } \P(Y=y_j) > 0\\
			0 &\quad \text{ wenn } \P(Y=y_j) = 0 
			\end{cases}
		\end{align*}
\end{*example}
\subsection{Bedingte Erwartung - maßtheoretischer Zugang}
Wir betrachten WRaum $(\O, \F,\P)$. Für ZV $X: \O \to \R$ und $p \in [1,\infty)$ definieren wir die $L_p$-Norm
\begin{align*}
	\norm{X}_p = \E[\abs{X}^p]^{1/p} = \brackets{\int_{\O} \abs{X(\omega)}^p \d \P(\omega)}^{1/p}
\end{align*}
und $L_p$-Raum $L_p(\O,\F,\P):= \set{X: \O \to \R\colon \F-\text{messbar}, \norm{X}_p < \infty}$. Dabei identifzieren wir ZVen, die sich nur auf Nullmengen unterscheiden, d.h. $\P(X \neq X') = 0 \implies X = X'$ (in $L_p$).\\
Aus Maßtheorie bekannt: (?)\\
Die Räume $L_p(\O,\F,\P)$ mit Norm $\norm{\cdot}_p, p \in [1,\infty)$ sind stets \person{Banach}-Räume (lineare, vollständig, normierte Vektorräume). Für $p = 2$ auch Hilbertraum mit inneren Produkt
\begin{align*}
	\scaProd{X}{Y} = \E[XY] = \int_{\O} X(\omega)Y(\omega)\d \P(\omega)
\end{align*}
Für $\G \subseteq \F$ Unter-$\sigma$-Algebra ist $L_p(\O,\F,\P) \subseteq L_p(\O,\F,\P)$ abgeschlossen Unterraum.\\
Wir verallgemeinern ``Vorhersageproblem'' aus dem letzten Abschnitt (1.3?)\\
Gegeben ZVe $X$ aus $L_2(\O,\F,\P)$ ist $\G \subseteq \F$ Unter-$\sigma$-Algebra.\\
Was ist die beste $\G$-messbare Vorhersage für $Y$?
\begin{align*}
	\min\set{\E[(X-G)^2] \colon G \in L_2(\O,\F,\P)} \tag{min-2}\label{eq_min_2}
\end{align*}
wobei $\E[(X-G)^2] = \norm{X-G}^2_2$.\\
Aus Hilbertraumtheorie:\\
\eqref{eq_min_2} besitzt eine eindeutige Lösung $G_{\ast} \in L_2(\F,\G,\P)$. $G_{\ast}$ ist Optimierung (bezüglich $\scaProd{\cdot}{\cdot}$) von $X \in L_2(\O,\F,\P)$ auf abgeschlossenen Unterraum $L_2(\O,\G,\P)$\\
Bild: eventuell von Eric (Orthogonal Projektion auf den Unterraum)\\
Wir bezeichnen mit $G_{\ast}$ mit $\E[X\mid \G]$ bedingte Erwartungswert von $X$ bezüglich $\G$.
\begin{theorem}[Eigenschaften bedingter Erwartungswert] %1.5
	\label{1_5_eigen_bedEW}
	Seien $X,Y \in L_2(\O,\F,\P)$ und $\G \subseteq F$ Unter-$\sigma$-Algebra. Dann gilt
	\begin{enumerate}
		\item (Linearität) $\E[aX+bY] = a\E[X\mid \G] + b\E[Y\mid \G]$
		\item (Turmregel) Für jede weitere $\sigma$-Algebra $\H \subseteq\G$ gilt
		\begin{align*}
			\E[E[X\mid \G \mid \H]] = \E[X\mid \H]
		\end{align*}
		\item (Pullout-Property) $\E[XZ\mid \G] = Z\E[X\mid \G]$, wenn $Z$ beschränkt und $\G$-messbar ist.\\
		zweite Version: Für $Z$ $\G$-messbar mit $\E[\abs{XZ}] < \infty$ gilt:
		\begin{align*}
			\E[XZ\mid \G] = Z \cdot \E[X\mid \G]
		\intertext{insbesondere gilt}
			X \G\text{-messbar }\implies \E[X\mid \G] = X
		\end{align*}
		\item (Monotonie) $X \le Y \implies \E[X\mid \G] \le \E[Y \mid \G]$
		\item ($\Delta$-Ungleichung) $\abs{\E[X\mid \G]} \le \E[\abs{X}\mid \G]$
		\item (Unabhängigkeit) $X$ unabhängig von $G$ $\implies$ $\E[X \mid \G] = \E[X]$
		\item (triviale $\sigma$-Algebra) $\G=\set{\emptyset, \O} \implies \E[X \mid \G] = \E[X]$ 
	\end{enumerate}
\end{theorem}
\begin{proof}
	(ohne Beweis, siehe VL W-Theorie mit Martingalen oder auch STOCH-Skript SS19.)
\end{proof}
\begin{*remark}
	\begin{itemize}
		\item Die für $X \in L_2(\O,\F,\P)$ definierte vedingte Erwartung $\E[X\mid \G]$ lässt sich durch Approximation auf alle $X\in L_1(\O,\F,\P)$ erweitern. Alle Eigenschaften aus Theorem \propref{1_5_eigen_bedEW} bleiben erhalten!
		\item Sei $Y$ eine ZVe und $\G = \sigma(Y)$ die von $Y$ erzeugte $\sigma$-Algebra. Wir schreiben:
		\begin{align*}
		\E[X\mid Y] = \E[X \mid \sigma(Y)] \quad \sigma\text{-messbare ZVe}
		\end{align*}
		\item Maßtheorie: \person{Doob}-\person{Dynkin}-Lemma $\implies \exists$ messbare Funktion $g: \Rn \to \R$ sodass
		\begin{align*}
		\E[X\mid Y] = g(Y)
		\end{align*}
		Dabei ist $g$ genau die Funktion aus \eqref{eq_min_1}.
	\end{itemize}
\end{*remark}
Zusammenfassung:\\
Sei $X,Y$ aus $L_1(\O,\F,\R)$, $\G \subseteq \F$ Unter-$\sigma$-Algebra
\begin{enumerate}
	\item $\E[X\mid Y=y]$ ist messbare Funktion $g: \Rn \to \Rn$. Falls bedingte Dichte existiert, gilt:
	\begin{align*}
		\E[X\mid Y=y] = \int_{\R^m} f_{X\mid Y} (x,y) \d x
	\end{align*}
	\item $\E[X\mid Y]$ ist eine $\sigma(y)$-messbare ZVe, diese kann als $g(Y)$ dargestellt werden. Falls bedingte Dichte existiert, gilt
	\begin{align*}
		\E[X\mid Y](\omega) = \int_{\Rn}xf_{X\mid Y}(x,Y(\omega))\d x
	\end{align*}
	\item $\E[X \mid \G]$ ist eine $\G$-messbare ZVe. Falls $\G = \sigma(y)$ tritt 2) ein.
\end{enumerate}
In allgemeinen Fall kann $\bar{\E[X\mid \cdot]}$ interpretiert werden als \emph{beste Vorhersage} für $X$, gegeben
\begin{enumerate}
	\item punktweise Beobachtung $Y=y$
	\item Beobachtung $Y$
	\item Information $\G$
\end{enumerate}
\subsection{Martingale}
Prototyp eines ``neutralen'' stochastischen Prozesses,der weder Aufwärts- noch Abwärtstrend besitzt. Hier nur in diskrete Zeit $Z = \N_0$.
\begin{*definition}[Martingal ohne Filtration]
	Sei $(X_n)_{n\in \N_0}$ stochastischer Prozess. Wenn gilt
	\begin{enumerate}
		\item $\E[\abs{X_n}] < \infty$ $\forall n \in \N$
		\item $\E[X_{n+1},\dots, X_n] = X_n$ $\forall n \in \N$
	\end{enumerate}
	dann heißt $(X_n)$ \begriff{Martingal}. Wen wir $\F_n^{\ast} = \sigma(X_1,\dots,X_n)$ definieren, können wir 2) schreiben als
	\begin{align*}
		\E[X_{n+1} \mid \F_n^{\ast}] = X_n \quad \forall n \in \N
	\end{align*}
\end{*definition}
Interpretation:\\
\begin{itemize}
	\item Beste Vorhersage für zukünftigen Wert $X_{n+1}$, basierend auf Vergangenheit $\sigma(X_1,\dots,X_n)$ ist der momentane Wert $X_n$.
	\item Aus der Turmregel folgt
	\begin{align*}
		\E[X_{n+k} \mid \F_n^{\ast}] &= X_n \quad n,k \in \N_0
		\intertext{denn}
		\E[X_{n+k}\mid\F_n^{\ast}] &= \E[\E[X_{n+k}\mid \F_{n+k-1}\mid \F_n^{\ast}]] = \E[X_{n+k-1}\mid \F_n^{\ast}] = (k\text{-mal}) = X_n
	\end{align*}
\end{itemize}
Kann von $(\F_{n})_{n \in \N}$ auf beliebige Filtrationen $(\F_n)_{n \in \N_0}$ erweitert werden.
\begin{*definition}[Martingal mit Filtration]
	Sei $(X_n)_{n \in \N_0}$ ein stochastischer Prozess, adaptiert an eine Filtration $(\F_n)_{n \in \N_0}$. Wenn gilt
	\begin{enumerate}
		\item $\E[\abs{X_n}] < \infty$ $\forall n \in \N_0$
		\item $\E[X_{n+1} \mid \F_n] = X_n$ $\forall n \in \N_0$
	\end{enumerate}
	dann heißt $(X_n)_{n \in \N_0}$ \begriff{Martingal bezüglich Filtration} $(\F_n)_{n \in \N_0}$
\end{*definition}
Interpretation:\\
Beste Vorhersage für zukünftige Werte $X_{n+1}$, basierend auf verfügbarer Information $\F_n$ ist momentane Wert $X_n$.
\begin{*definition}[Supermartingal, Submartingal]
	Falls in Punkt 2) statt ``$=$'' die Ungleichung $\le \oder \ge$ gilt, so heißt $(X_n)_{n \in \N}$ \begriff{Supermartingal} bzw. \begriff{Submartingal}.
\end{*definition}
Erste Beobachtung:\\
\begin{itemize}
	\item $X$ Martingal $\implies \E[X_n] = X_0$, d.h. $n \mapsto \E[X_n]$ ist konstant.\\
	Begründung:
	\begin{align*}
		\E[X_{n+1} \mid \F_n] = X_n \implies \E[\E[X_{n+1}\mid \F_n]] = \E[X_n] = \E[X_{n+1}] \implies (n\text{-mal Anwendung } \E[X_n] = X_0)
	\end{align*}
	Bild: Erwartungswert konstant, aber kein Martingal.
	\item $X$ Submartingal $\implies n \mapsto \E[X_n]$ ist monoton steigend
	\item $X$ Supermartingal $\implies n \mapsto \E[X_n]$ ist monoton fallend
\end{itemize}
Um sich den Unterschied zwischen Super- und Submartingal zu merken, hier eine kleine Hilfe:\\
``Das leben ist ein Supermartingal, die Erwartungen fallen mit der Zeit.''
\begin{*example}
	\begin{itemize}
		\item Seien $(Y_n)_{n\in \N}$ unabhängige ZVen in $L_1(\O,\F,\P)$ mit $\E[Y_n] = 0$. Definiere $X_n := \sum_{k=1}^n Y_k \mit X_0 = 0$. Dann ist $(X_n)_{n \in \N_0}$ Martingal, denn
		\begin{enumerate}
			\item $\E[\abs{X_n}] \le \sum_{k=1}^n \E[\abs{Y_k}] < \infty \quad \forall n \in \N$ \checkmark
			\item
			\begin{align*}
				\E[X_{n+1} \mid \F_n^{\ast}] &= \E[Y_{n+1} + X_n \mid \F_n^{\ast}]\\
				&= \E[Y_{n+1} \mid \F_n^{\ast}] = \E[X_n \mid \F_n^{\ast}] \quad (\text{ Turm und $\F_n^{\ast}$-messbar})\\
				&= \underbrace{\E[Y_{n+1}]}_{=0} + X_n = X_n \checkmark
			\end{align*}
		\end{enumerate}
		\item weitere Beispiele auf dem ersten Übungsblatt!
	\end{itemize}
\end{*example}
\begin{*definition}[vorhersehbar]
	Sei $(\F_n)_{n\in \N_0}$ eine Filtration. Ein stochastischer Prozess $(X_n)_{n \in \N}$ heißt \begriff{vorhersehbar} (predictable) bezüglich $(\F_n)_{n \in \N_0}$, wenn gilt:
	\begin{align*}
		H_n \text{ ist } \F_{n-1}\text{-messbar} \quad \forall n \in \N
	\end{align*}
\end{*definition}
\begin{*remark}
	Stärkere Eigenschaft als ``adaptiert''.
\end{*remark}
\begin{*definition}[diskretes stochastische Integral]
	Sei $X$ adaptierter und $H$ ein vorhersehbarer stochastischer Prozess bezüglich $(\F_n)_{n \in \N}$. Dann heißt
	\begin{align*}
		(H \bigcdot X)_n := \sum_{k=1}^n H_k (X_k - X_{k-1}) \tag{$\ast$}\label{eq_pred_stoch_process}
	\end{align*}
	\begriff{diskretes stochastische Integral} von $H$ bezüglich $X$.
\end{*definition}
\begin{*remark}
	Summe \eqref{eq_pred_stoch_process} heissen in der Analysis \person{Riemann}-\person{Stieltjes}-Summen. Werden für Konstruktionen des RS-Integrals $\int h \d \rho$ verwendet.
\end{*remark}
\begin{*definition}[lokal beschränkt]
	Ein stochastischer Prozess $(H_n)_{n \in \N}$ heißt \begriff{lokal beschränkt}, wenn eine (definierte) Folge $c_ \in \R_{\ge 0}$ existiert, sodass
	\begin{align*}
		\abs{H_n} \le c_n \text{ f.s. } \quad \forall n \in \N
	\end{align*}
\end{*definition}
\begin{theorem}
	Sei $X$ adaptiert stochastischer Prozess (bezüglich Filtration $(\F_n)_{n \in \N}$). Dann sind äquivalent:
	\begin{enumerate}
		\item $X$ ist Martingal
		\item $(H \bigcdot X)$ ist Martingale für alle lokal beschränkten, vorhersehbaren $(H_n)_{n \in N}$
	\end{enumerate}
	Das heisst: stochastische Integral erhält die Martingal-Eigenschaft.
\end{theorem}
\begin{*remark}
	Die ZV $H$ wird später die Anlagestrategie sein.
\end{*remark}
%\begin{proof}
%	\begin{itemize}
%		\item $\Rightarrow$:
%		\begin{itemize}
%			\item Adaptiertheit: klar
%			\item Integrierbarkeit: $H$ lokal beschränkt, d.h. $\abs{H_k} \le c_k < \infty$ für alle $k$.
%			\begin{align*}
%			\EW[\abs{H_k (X_k - X_{k-1})}] \le c_k * \brackets{\EW[\abs{x_k}] + \EW[\abs{X_{k+1}}]} < \infty
%			\end{align*}
%			Mit der Dreiecksungleichung folgt daraus $\EW[\abs{(H \bigcdot X)_n}] < \infty$.
%			\item Martingaleigenschaft: 
%			\begin{align*}
%			\EW[(H \bigcdot X)_n \mid \F_{n-1}] &= (H \bigcdot X)_{n-1} + \EW[H_n (X_n - X{n-1}) \mid \F_{n-1}]\\
%			&=  (H \bigcdot X)_{n-1} + H_n * \underbrace{\brackets{\EW[X_n \mid \F_{n-1}] - X{n-1}}}_{=0}\\ 
%			&=(H \bigcdot X)_{n-1} \quad \forall n \in \N
%			\end{align*}
%			Damit ist also auch $(H \bigcdot X)$ ein Martingal.		
%		\end{itemize}
%	\end{itemize} $(H \bigcdot X)_n = \sum_{k=1}^n H_k (X_k - X_{k-1})$.
%	\item $\Leftarrow$: Fixiere $N \in \N$. Setze $H_n := \indi_{n = N}$, dieser ist lokal beschränkt und deterministisch (also auch vorhersehbar). Man stellt fest, dass $(H \bigcdot X)_n = 0$ für alle $n \le N-1$. Für alle $n \ge N$ gilt dagegen $(H \bigcdot X)_n = X_N - X_{N-1}$. Wir überprüfen nur die Martingaleigenschaft (Integrierbarkeit folgt aus Dreiecksungleichung). Wir wissen, dass $(H \bigcdot X)$ ein Martingal ist. 
%	\begin{align*}
%	0 &= (H \bigcdot X)_{N-1} = \E[(H \bigcdot X)_N \mid \F_{N-1}]\\
%	 &= \E[x_N - X_{N-1} \mid \F_{N-1}]\\ 
%	 &= \E[X_N \mid \F_{N-1}] - X_{N-1}\\
%	&\implies X_{N-1} = \E[X_N \mid \F_{N-1}] \mit N \in \N \text{ beliebig}
%	\end{align*}
%	Somit ist $X$ ein Martingal.
%\end{proof}
%\begin{conclusion} %1.7
%	Sei $X = \set{X_n}{n=1 , \dots, N}$ ein adaptierter stochastischer Prozess bezüglich einer Filtration \\$\set{\F_n}{n=1 , \dots, N}$. Wenn $\E[(H \bigcdot X)_N] = 0$ für alle lokal beschränkten vorhersehbaren Prozesse $H$, dann ist $X$ ein Martingal bezüglich $\set{\F_n}{}$.
%\end{conclusion}
%\begin{proof}
%	Fixiere $K \in  [N] := \set{1, 2, \dots , N}$ und eine Menge $A \in \F_{K-1}$. Definiere $H_n(\omega) = \indi_A (\omega) * \indi_{\set{n=K}}$, dieser ist lokale beschränkt und vorhersehbar.
%	Es ist $(H \bigcdot X)_n = 0$ für alle $n \le K-1$. Für alle $n \ge K$ gilt $(H \bigcdot X)_n = \indi_A * (X_K - X_{K-1})$. 
%	\begin{align*}
%		0 &= \E[(H \bigcdot X)_N] = \E[\indi_A (X_K - X_{K-1})]\\ 
%		\overset{Turm}&{=} \E[\E[\indi_A (X_K - X_{K-1}) \mid \F_{K-1}]]\\ 
%		&= \E[\indi_A * \brackets{\underbrace{\E[X_K \mid \F_{K-1] - X_{K-1}}]}_{ Y_{K-1}}} \quad \forall A \in \F_{K-1}\\
%		&\implies \int_A  Y_{K-1}(\omega) \d{\P(\omega)}\\ 
%		&= \int_A X_{K-1}(\omega) \d{\P(\omega)} \quad \forall A \in \F_{K-1}\\
%		&\implies Y_{K-1} = X_{K-1} \text{ fast sicher }\\
%		&\implies \E[X_K \mid \F_{K-1] - X_{K-1}}] = X_{K-1}
%	\end{align*}
%	für beliebige $K$. Somit ist $X$ ein Martingal.
%\end{proof}
%\begin{*remark}
%	Wir schreiben $[N] := \set{1, 2, \dots, N}$ und $[N]_0 := \set{0, 1, 2 , \dots , N}$.
%\end{*remark}
\chapter{Cox-Russ-Rubenstein-Modell}
\section{Rings}
\begin{definition}
	A \begriff{ring} is a set $R$ together with two binary operators $+,\cdot \colon R^2 \to R$ addition and multiplication and an element $1 \in R$, which satisfies the conditions
	\begin{enumerate}
		\item $(R,+)$ is abelian group
		\item $(a\cdot b)\cdot c = a \cdot (b\cdot c)$ for all $a,b,c \in R$. (Associativity)
		\item $1\cdot a = a = a \cdot 1$ for all $a \in R$. (Unital)
		\item We have
		\begin{align*}
			(a+b)\cdot c = (a\cdot c) + (b \cdot c) \nd c\cdot (a+b) = (c\cdot a) + (c \cdot b) \quad \forall a,b,c \in R.
		\end{align*}
	\end{enumerate}
\end{definition}
\begin{*remark}
	\begin{itemize}
		\item write $ab$ instead of $a\cdot b$
		\item unit element of $(R,+)$ is denoted by 0.
		\item We do not require $1\neq 0$ and note $1=0 \iff R = 0$, this means that $R$ consists of exactly one element.
	\end{itemize}
\end{*remark}
\begin{definition}
	$R,S$ rings, a ring homomorphism is a homormorphism $f\colon R \to S$ such that $f(ab)$
\end{definition}
\chapter{Das Black-Scholes-Modell}
Ziel ist der Übergang vom CRR-Modell (zeit-diskret) zum \person{Black}-\person{Scholes} (BS-)Modell (zeit-stetig) durch Grenzwertbildung.
\begin{itemize}
	\item Herleitung \person{Block}-\person{Scholes}-Formel für Preise von europäischen Put- und Call-Optionen
\end{itemize}
Beachte Zeitintervall $[0,T]$, für jedes $N \in \N$ geteilt in Schritte der Länge $\Delta_n = \frac{T}{N}$. Wähle Parameter $r \in \R, \mu \in \R$ (Trendparameter), $\sigma > 0$ (Volatitität). Definiere Folge von CRR-Modelle $(S^N)_{N \in \N}$ eingebettet in $[0,T]$ mit Parametern
\begin{align*}
	r_N = r \cdot \Delta_n \quad b_N = \mu \Delta_n + \sigma \sqrt{\Delta_n}\quad a_N = \mu \Delta_n - \sigma \sqrt{\Delta_n},\;p \in (0,1),\;s> 0
\end{align*}
d.h. $S^N_0 = s$, $S^N_{t_k} = s \cdot \prod_{i=1}^k (1+R_i^N)$ mit $t_k = k \cdot \Delta_n$, bzw. $\tilde{S}_0^N = s$ und damit $\tilde{S}^N_{t_k}= s \cdot \prod_{i=1}^k \frac{1+R_i^N}{1+r_N}$, wobei $\P(R^N_i = b_N) = p, \P(R^N_i = a_N) = 1-p$.
Bezeichne diese Folge mit $\CRR_N$. Falls notwendig, interpolieren wir zwischen den Gitterpunkten mit
\begin{align*}
	S_t^N = S^N_{t_k} \quad t \in [t_k,t_{k+1}]
\end{align*}
Berechne risko-neutrale Wahrscheinlichkeiten
\begin{align*}
	q_N = \QQ_N(R_i^N = b_N) = \frac{r_N - a_N}{b_N - a_N} = \frac{(r-\mu)\Delta_n + \sigma \sqrt{\Delta_n}}{2\sigma\sqrt{\Delta_n}} = \frac{1}{2} - \frac{\lambda}{2}\sqrt{\Delta_n}
\end{align*}
mit $\lambda := \frac{\mu - r}{\sigma}$
\begin{*remark}
	\begin{itemize}
		\item Wenn $\mu = r$, dann $q_N = \frac{1}{2}$ und im Allgemeinen $\lim_{k \to \infty}a_N = \half$
		\item $\lambda := \frac{\mu - r}{\sigma}$ heißt ``Sharp-ratio'' oder Marktrisikopreis
	\end{itemize}
\end{*remark}
Frage: Konvergenz der Verteilung von $S^N_T$ unter $\QQ_N$ für $N \to \infty$?\\
Übergang zum Logarithmus:
\begin{align*}
	\ZZ_N := \log(\frac{S^N_T}{S_0}) = \sum_{k=1}^N \underbrace{\log(1+R_k^N)}_{L^N_k}
\end{align*}
Summe von unabhängige identisch verteilte Zufallsvariablen, dann Zentraler Grenzwertsatz(ZGS)?\\
Es liegt ein sogenanntes \emph{Dreiecksschema} vor
\begin{align*}
	\begin{matrix}
	\ZZ_1 &= L_1^1 & &\\
	\ZZ_2 &= L_2^1 &+L^2_2 &\\
	\ZZ_3 &= L_3^1 &+L^3_2 &+L^3_3\\
	\end{matrix} \text{ Zufallsvariablen in einer Zeile sind stoch. unabhängig.}
\end{align*}
\begin{theorem}[ZGS für Dreiecksschemata]
	Sei für jedes $N \in \N$ ein Vektor $L^N := (L^N_1, L^N_2, \dots, L^N_N)$ von Zufallsvariablen gegeben (``Dreiecksschema'') mit folgenden Eigenschaften:
	\begin{enumerate}
		\item $\forall N \in \N$ sind $(L^n_1, \dots, L_N^N)$ unabhängig mit identischer Verteilung
		\item $\exists$ Folge von (deterministischen) Konstanten $C_N \to 0$, sodass
		\begin{align*}
			\abs{L_k^N} \le C_N \quad \forall k \in [N]
		\end{align*}
		\item Mit $\ZZ_N = L^N_1 + \dots + L^N_N$ gilt
		\begin{align*}
			\begin{matrix}
				\E[\ZZ_N] \to m \in \R\\
			\Var(\ZZ_N) \to s^2 > 0 
			\end{matrix}\text{ für }N \to \infty			
		\end{align*}
	\end{enumerate}
	Dann konvergiert $(\ZZ_N)_{N \in \N}$ in Verteilung gegen normalverteilte Zufallsvariable $\ZZ$ mit $\E[\ZZ] = m \und \Var(\ZZ) = s^2$
\end{theorem}
\begin{proof}
	Ohne Beweis, siehe z.B. Wahrscheinlichkeitstheorie mit Martingale.
\end{proof}
\begin{*remark}
	Vergleiche 2. Übung erste Aufgabe.
\end{*remark}
\begin{erinnerung} % should not count :/
	Dichte der Standardnormalverteilung heißt hier
	\begin{align*}
		\phi(x) = \frac{1}{\sqrt{2\pi}} e^{-x^2/2}
	\end{align*}
	und die Verteilungsfunktion
	\begin{align*}
		\Phi(x) = \int_{-\infty}^x \phi(y) \d y = \int_{-\infty}^x \frac{1}{\sqrt{2\pi}} e^{-y^2/2} \d y
	\end{align*}
	Normalverteilung mit Erwartungswert $m$ und Varianz $s^2$ hat Verteilungsfunktion $\Phi(\frac{x-m}{s})$
\end{erinnerung}
\begin{definition}
	Eine strikt positive Zufallsvariable $X$ heißt \begriff{lognormalverteilt} mit Parameter $m, s^2$, wenn gilt
	\begin{align*}
		\log(X) \sim \NNN(m,s^2)
	\end{align*}
\end{definition}
\begin{theorem}
	\proplbl{theo_3_1}
	Betrachte Folge $(S^N)_{N \in \N}$ von CRR-Modellen wie in $\CRR_N$ beschrieben. Dann konvergiert $S_T^N$ unter $\QQ_N$ in Verteilung gegen eine Zufallsvariable $S_T$ und $S_T/S_0$ ist lognormalverteilt mit Parameters $n = T(r - \sigma^2/2)$ und $s^2 = T\sigma^2$. Äquivalent dazu gilt mit $\ZZ_N = \log(S_T^N / S_0)$
	\begin{align*}
		\QQ_N(\ZZ_N \le x) \xrightarrow{N \to \infty} \Phi\brackets{\frac{x-T(r-\sigma^2/2)}{\sigma\sqrt{T}}}
	\end{align*}
\end{theorem}
\begin{proof}
	Das Dreiecksschema $L^N = (L^N_1, \dots, L_N^N)$ mit $L^N_k = \log(1+R^N_k)$ erfüllt (unter $\QQ_N$) offensichtlich Bedingung 1. und 2. aus Theorem 3.1 %TODO add references
	wähle z.B.: 
	\begin{align*}
		C_N = \max(\abs{\log(1+\mu\Delta_n + \sigma\sqrt{\Delta_n})}, \abs{\log(1+\mu \Delta_n - \sigma\sqrt{\Delta_n})})
	\end{align*}
	. Wir berechnen Erwartungswert und Varianz von $L^N_k$ bzw. $\ZZ_N$. Verwende die Taylorentwicklung:
	\begin{align*}
		\log(1+x) = x - x^2/2 + x^3/3 + \OO(x^4) \quad (x \to 0)
	\end{align*}
	Das heißt
	\begin{align*}
		\log(1+ \underbrace{\mu \Delta_n \pm \sigma\sqrt{\Delta_n}}_{b_N \text{ bzw. }a_N}) = \pm \sigma \sqrt{\Delta_n} + \mu \Delta_n - \sigma^2/2 \Delta_n + \OO(\Delta_n^{3/2})
	\end{align*}
	Risiko-neutralen Wahrscheinlichkeiten sind
	\begin{align*}
		q_N = \half - \frac{\lambda}{2}\sqrt{\Delta_n}\quad 1-q_N = \half + \frac{\lambda}{2}\sqrt{\Delta_n}
	\end{align*}
	\begin{align*}
		\E^{\QQ_N}[L^N_k] &= \E^{\QQ_N}[\log(1+R^N_k)] = q_N\log(1+b_N) + (1+p_N)\log(1+a_N)\\
		&= (\mu - \sigma^2/2)\Delta_n - \lambda\sigma\Delta_n + \OO(\Delta_n^{3/2}) \quad \mit \lambda = \frac{\mu -r}{\sigma}\\
		&= (\mu - (\mu - r) - \sigma^2/2) \Delta_n + \OO(\Delta_n^{3/2})\\
		&= (r-\sigma^2/2)\Delta_n + \OO(\Delta_n^{3/2})\\
		\E^{\QQ_N}[(L^N_k)^2] &= q_N\log^2(1+b_N) + (1-q_N)\log^2(1+a_N)\\
		&= \sigma^2\Delta_n + \OO(\Delta_n^{3/2})\\
		\Var^{\QQ_N}(L^N_k) &= \E^{\QQ_N}[(L^N_k)^2]-\E^{\QQ_N}[L^N_k]^2 = \sigma^2\Delta_n + \OO(\Delta_n^{3/2})
	\end{align*}
	Also gilt
	\begin{align*}
		\E^{\QQ_N}[\ZZ_N] &= N \cdot \E^{\QQ_N}[L_k^N] = (r-\sigma^2/2)T + \OO(N^{-1/2}) \xrightarrow{N \to \infty} (r-\sigma^2/2)T =: m\\
		\Var^{\QQ_N}[\ZZ_N] &= N \cdot \Var^{\QQ_N}[L^N_k] = \sigma^2 T + \OO(N^{-1/2}) \xrightarrow{N \to \infty} \sigma^2T =:s^2
	\end{align*}
	Resultat folgt nun mir ZGS (Theorem 3.1).
\end{proof}
\subsection*{Asymptotik von Put- und Call-Option}
Fixiere Laufzeit $T$ und Ausübungspreis $K$ und schreiben:\\
\begin{itemize}
	\item $C_N(t, S_t^N)$ ... Preis einer europäischen Call-Option im $\CRR_N$ Modell in Abhängigkeit von Zeit $t$ und Basisgut $S^N_t$
	\item $P_N(t, S_t^N)$ ... Analog für Put
\end{itemize}
\begin{theorem}[\person{Black}-\person{Scholes}-Formel]
	\proplbl{theo_BS_eq}
	Die Preise $C_N$, $P_N$ konvergieren für $N \to \infty$ gegen BS-Preis
	\begin{align*}
		C_{BS}(t,S_t) &= \lim_{N \to \infty} C_N(t, S_t^N)\\
		P_{BS}(t,S_t) &= \lim_{N \to \infty} P_N(t, S_t^N)
	\end{align*}
	und es gilt die \begriff{\person{Black}-\person{Scholes}-Formel}:
	\begin{align*}
		C_{BS}(t,S_t) &= S_t \Phi(d_1) - e^{-r(T-t)} K \Phi(d_2)\\
		P_{BS}(t,S_t) &= S_t \Phi(-d_1) - e^{-r(T-t)} K \Phi(-d_2)
		\intertext{wobei}
		d_1 &= d_1(t, S_t) = \frac{\log(S_t/K) + (r+\sigma^2/2)(T-t)}{\sigma \sqrt{T-t}}\\
		d_2 &= d_2(t, S_t) = \frac{\log(S_t/K) + (r-\sigma^2/2)(T-t)}{\sigma \sqrt{T-t}} = d_1(t, S_t) - \sigma\sqrt{T-t}
	\end{align*}
\end{theorem}
\begin{*remark}
	\begin{itemize}
		\item Geschlossener Ausdruck für Bewertung von europäischen Put- und Call-Optionen
		\item Herleitung als Grenzwert aus dem CRR-Modell entspricht nicht der ursrünglichen Herleitung von \person{Black} \& \person{Scholes} mittels stochastische Analysis ($\implies$ VL stoch. Calculus)
		\item Für Entwicklung von BS-Formel und BS-Modell erhielten \person{Scholes} \& \person{Merton} dem Wirtschaftsnobel(gedenk)preis 1997
		\item Der Parameter $\sigma$ heißt Voliatitität und entspricht der Schwankungsbreite der Preisänderung
	\end{itemize}
\end{*remark}
Skizze vom BS-Call-Preis:
\begin{center}
	\begin{tikzpicture}
	\begin{axis}[
	xmin=0, xmax=1, xlabel=\scriptsize$S_t$ Preis Basisgut,
	ymin=0, ymax=1, ylabel={\scriptsize$C_{BS}(t,S_t)$},
	title=BS-Call-Preis,
	samples=400,
	axis x line=middle,
	axis y line=middle,
	domain=0:1,
	yticklabels={,,},
	xticklabels={,,},
	xtick style={draw=none},
	ytick style={draw=none},
	]
	\draw[fill=red,opacity=0.1] (axis cs: 0,0) rectangle (axis cs: 0.4,1) node[pos=0.5,black,yshift=2cm] {\scriptsize OTM};
	\draw[fill=lime!70!black,opacity=0.1] (axis cs: 0.4,0) rectangle (axis cs: 1,1) node[pos=0.5,black,yshift=2cm] {\scriptsize ITM};
	
	\draw[red] (axis cs: 0,0) -- (axis cs: 0.4,0) -- (axis cs: 1,0.9);
	\draw[blue] (axis cs: 0,0) to[out=5,in=215] (axis cs: 0.4,0.15) to[out=35,in=226] (axis cs: 1,0.92);
	\draw[cyan] (axis cs: 0,0) to[bend right=10] (axis cs: 1,0.95);
	
	\node[red,align=center] at (axis cs: 0.83,0.35) {\scriptsize Auszahlungsprofil\\\scriptsize$(S_T-K)_+$};
	\node[cyan] at (axis cs: 0.83,0.90) {\scriptsize $C(t_0,S_0)$};
	\node[blue] at (axis cs: 0.83,0.85) {\scriptsize $C(t,S_t)$};
	
	\draw (axis cs: 0.4,0) -- (axis cs: 0.4,0.05);
	\node at (axis cs: 0.4,0.07) {\scriptsize $K$};
	
	\draw[dotted] (axis cs: 0.5,0) -- (axis cs: 0.5,0.38);
	\node at (axis cs: 0.62,0.12) {\scriptsize innerer Wert};
	
	\node[blue] at (axis cs: 0.5,0.7) {\scriptsize Zeitwert};
	\draw[blue,->] (axis cs: 0.58,0.68) -- (axis cs: 0.7,0.5);
	\end{axis}
	\end{tikzpicture}
\end{center}
\begin{itemize}
	\item \begriff{innere Wert}: $(S_t -K)_+$ konvergiert gegen Auszahlungsprofil: $(S_T - K)_+$, für $t \to T$
	\item \begriff{Zeitwert}: $C_{BS}(t,S_t) - (S_t - K)_+ \ge 0$ konvergiert gegen Null für $t \to T$
	\item 
	\begin{itemize}
		\item \begriff{out of the money} (OTM): Innere Wert $=0$ bei $S_t < K$
		\item \begriff{in the money} (ITM): Innere Wert $>0$ bzw. $S_t > K$
		\item \begriff{at the money} (ATM): Grenzfall $S_t = K$
	\end{itemize}
	\item Zeitwert ist am größten für ATM-Optionen
	\item $t \mapsto C_{BS}(t,S_t)$ ist streng monoton fallend bzw.
	\begin{align*}
		\frac{\partial C_{BS}(t,S_t)}{\partial t} < 0
	\end{align*}
	\item $S_t \mapsto C_{BS}(t,S_t)$ ist streng monoton steigend und konvex bzw
	\begin{align*}
		\frac{\partial C_{BS}(t,S_t)}{\partial S}(t,S_t) > 0 \und \frac{\partial^2 C_{BS}}{\partial S^2}(t,S_t) > 0
	\end{align*}
	\item Für den Put ist das ganze Symmetrisch
\end{itemize}
Skizze vom BS-Put-Preis:
\begin{center}
	\begin{tikzpicture}
	\begin{axis}[
	xmin=0, xmax=1, xlabel=\scriptsize$S_t$ Preis Basisgut,
	ymin=0, ymax=1, ylabel={\scriptsize$P_{BS}(t,S_t)$},
	title=BS-Put-Preis,
	samples=400,
	axis x line=middle,
	axis y line=middle,
	domain=0:1,
	yticklabels={,,},
	xticklabels={,,},
	xtick style={draw=none},
	ytick style={draw=none},
	]
	\draw[fill=red,opacity=0.1] (axis cs: 0.6,0) rectangle (axis cs: 1,1) node[pos=0.5,black,yshift=2cm] {\scriptsize OTM};
	\draw[fill=lime!70!black,opacity=0.1] (axis cs: 0,0) rectangle (axis cs: 0.6,1) node[pos=0.5,black,yshift=2cm] {\scriptsize ITM};
	
	\draw[red] (axis cs: 0,0.9) -- (axis cs: 0.6,0) -- (axis cs: 1,0);
	\draw[blue] (axis cs: 0,0.92) to[out=-45,in=140] (axis cs: 0.6,0.15) to[out=-40,in=170] (axis cs: 1,0);
	\draw[cyan] (axis cs: 0,0.95) to[bend right=10] (axis cs: 1,0);
	
	\node[red,align=center] at (axis cs: 0.17,0.32) {\scriptsize Auszahlungsprofil\\\scriptsize$(K-S_T)_+$};
	\node[cyan] at (axis cs: 0.3,0.80) {\scriptsize $P_{BS}(t_0,S_0)$};
	\node[blue] at (axis cs: 0.3,0.75) {\scriptsize $P_{BS}(t,S_t)$};
	
	\draw (axis cs: 0.6,0) -- (axis cs: 0.6,0.05);
	\node at (axis cs: 0.6,0.07) {\scriptsize $K$};
	
	\draw (axis cs: 0.4,0) -- (axis cs: 0.4,0.3);
	\draw[dotted] (axis cs: 0.4,0.3) -- (axis cs: 0.4,0.47);
	\node at (axis cs: 0.27,0.1) {\scriptsize innerer Wert};
	
	\node[blue] at (axis cs: 0.6,0.5) {\scriptsize Zeitwert};
	\draw[blue,->] (axis cs: 0.52,0.5) -- (axis cs: 0.4,0.4);
	\end{axis}
	\end{tikzpicture}
\end{center}
\begin{itemize}
	\item \begriff{innere Wert}: $(K -S_t)_+$ konvergiert gegen Auszahlungsprofil: $(K - S_T)_+$, für $t \to T$
	\item \begriff{Zeitwert}: $P_{BS}(t,S_t) - (K - S_t)_+ \ge 0$ konvergiert gegen Null für $t \to T$
	\item 
	\begin{itemize}
		\item \begriff{out of the money} (OTM): Innere Wert $=0$ bei $S_t > K$
		\item \begriff{in the money} (ITM): Innere Wert $>0$ bzw. $S_t < K$
		\item \begriff{at the money} (ATM): Grenzfall $S_t = K$
	\end{itemize}
	\item Zeitwert ist am größten für ATM-Optionen
	\item $t \mapsto C_{BS}(t,S_t)$ ist streng monoton fallend bzw.
	\begin{align*}
	\frac{\partial C_{BS}(t,S_t)}{\partial t} < 0
	\end{align*}
	\item $S_t \mapsto C_{BS}(t,S_t)$ ist streng monoton fallend und konvex bzw
	\begin{align*}
	\frac{\partial C_{BS}(t,S_t)}{\partial S}(t,S_t) < 0 \und \frac{\partial^2 C_{BS}}{\partial S^2}(t,S_t) > 0
	\end{align*}
\end{itemize}
\begin{proof}[\propref{theo_BS_eq}]
	Wir beweisen das Resultat für $t = 0$: andere Zeitpunkte $t \in [0,T]$ können analog behandelt werden. 
	\begin{itemize}
		\item Nach \propref{theo_2_8}, gilt für Preis der Put-Option im $\CRR_N$-Modell
		\begin{align*}
		P^N(0, S_0^N) &= (1+r\Delta_n)^{-N}\cdot \E^{\QQ}[(K-S_T^N)_+]\\
		&= (1+r \Delta_n)^{-N}\cdot \E^{\QQ}[(K-S_0 e^{\ZZ_N = \log(\frac{S_T^N}{S_0})})]\\
		&= (1+ r \Delta_n)^{-N}\cdot \E^{\QQ}[f(\ZZ_N)]
		\end{align*}
		mit $f(z)= (K-S_0 e^z)_+$ stetig und beschränkt. Aus Stochastik ist bekannt $\ZZ_N \to \ZZ$ in Verteilung, dann folgt $\E[f(\ZZ_N)] \to \E[f(\ZZ)] \quad \forall f \in C_b(\R)$.
		\begin{itemize}
			\item $\lim_{N \to \infty}(1+r\Delta_n)^{-N} = \lim_{N \to \infty}(1+rT/N)^{-N} = e^{-rT}$
			\item $\lim_{N \to \infty} \E^{\QQ}[f(\ZZ_N)] = \E[f(Z)]$ mit $\ZZ \sim \NNN((r-\sigma^2/2)T, \sigma^2 T) = \NNN(mT,\sigma^2 T)$
			\begin{align*}
			\E[f(\ZZ)] &= \frac{1}{\sqrt{2\pi}}\frac{1}{\sigma\sqrt{T}}\int_{-\infty}^{\infty}(K-S_0 e^{\ZZ})_+ \exp(- \frac{(\ZZ - mt)^2}{2 \sigma^2 T})\d z\\
			&= \frac{1}{\sqrt{2 \pi}} \int_{-\infty}^{\log(K/S_0)}(K - S_0 e^{\ZZ})\exp(-\half (\frac{\ZZ-mT}{\sigma \sqrt{T}})^2) \d z\\
			&= \begin{pmatrix}
			y = \frac{\ZZ - mT}{\sigma \sqrt{T}}\\
			\d y = \frac{\d \ZZ}{\sigma \sqrt{T}}
			\end{pmatrix}\\
			&= \frac{1}{\sqrt{2 \pi}} \int_{-\infty}^{-d_2} (K - S_0 \exp(y \sigma \sqrt{T} + m T) e^{y^2/2} \d y\\
			&= K\Phi(-d_2)-S_0 \frac{1}{\sqrt{2 \pi}} \int_{-\infty}^{-d_2}\exp(-y^2/2 + y \sigma\sqrt{T}+mT)\d y
			\intertext{Nebenrechung:}
			& y^2/2 + y \sigma\sqrt{T} (+ mT = rT - \half(y^2-2y\sigma\sqrt{T}+\sigma^2 T) = rT - \half(y-\sigma\sqrt{T})\\
			&= K \Phi(-d_2) = S_0 e^{rT}\underbrace{\frac{1}{\sqrt{2 \pi}} \int_{-\infty}^{-d_2} e^{(y-\sigma\sqrt{T})/2}\d y}_{\Phi(-d_2-\sigma\sqrt{T})}\\
			&= K\Phi(-d_2) - S_0 e^{rT}\Phi(-d_1)\\
			\end{align*}
			Dann folgt $\lim_{N \to \infty} P_N(0, S_0^N) = e^{-rT}K\Phi(-d_2) - S_0\Phi(-d_1)$ und das ist die Formel für den Put \checkmark
		\end{itemize}
		\item Für Call: Nutze Put-Call-Parität
		\begin{align*}
			C^N(0,S_0) - \underbrace{P^N(0,S_0)}_{P_{BS}(0,S_0)} = \underbrace{S_0}_{S_0} - \underbrace{(1+ r\Delta_n)^{-N} K}_{\to e^{-rT}K}
		\end{align*} 
		\begin{align*}
			C_{BS}(0,S_0) &= \lim_{N \to \infty}C^N(0,S_0)\\
			&= P_{BS}(0,S_0) + S_0 - e^{rT}K\\
			&= e^{-rT}K(\underbrace{\Phi(-d_2)-1}_{-\Phi(d_2)}) - S_0 (\underbrace{\Phi(-d_1)-1}_{-\Phi(d_1)})\\
			&= S_0\Phi(d_1)-e^{-rT}K\Phi(d_2)
		\end{align*}
		wobei wir die Symmetrie der Normalverteilung: $\Phi(-x) = 1- \Phi(x)$ genutzt haben. Damit ist auch die BS-Formel für den Call gezeigt \checkmark.
	\end{itemize}
\end{proof}
\emph{Wir haben gezeigt:} $\CRR_N$-Preise konvergieren gegen BS-Preise.\\
\emph{Frage:} Was gilt für die Replikationsstrategie? Konvergiert diese auch?
\begin{theorem} %3.4
	Für die Replikationsstrategie $\xi_{t_N}^N$ der Put- bzw. Call-Optionen in $\CRR_N$-Modell gilt:
	\begin{itemize}
		\item \emph{Put:} $\lim_{N \to \infty} \xi_{t_N}^N = \frac{\partial P_{BS}}{\partial S}(t,S_t) = - \Phi(-d_1)$
		\item \emph{Call:} $\lim_{N \to \infty} \xi_{t_N}^N = \frac{\partial C_{BS}}{\partial S}(t,S_t) = \Phi(d_1)$
	\end{itemize}
	Diese partielle Ableitungen heißen auch ``Delta'' des Put- bzw. Call-Preisen. 
\end{theorem}
\begin{proof}
	Betrachte nur $t=0$, $t \in[0,T]$ kann analog behandelt werden. Nach \propref{th_2_3} ist $\xi_0^N$ für Put gegeben durch
	\begin{align*}
		\xi_0^N &= \frac{P_N(\Delta_N, S_0(1+b_N)) - P_N(\Delta_N, S_0(1+a_N))}{S_0(b_N-A_N)}\\
		&= \frac{P_N(\Delta_N, S_0(1+\mu\Delta_n + \sigma\sqrt{\Delta_n})) - P_N(\Delta_N, S_0(1+\mu\Delta_N + \sigma\sqrt{\Delta_N}))}{2 S_0 \sigma \sqrt{\Delta_N}}
	\end{align*}
	Es gilt $\lim_{N \to \infty} P_N(\Delta_N, S_0(1+\mu\Delta_N)) = P_{BS}(0,S_0)$. Unter geeigneten Annahmen an gleichmäßige Konvergenz folgt
	\begin{align*}
		\lim_{N \to \infty} \xi_0^N = \frac{\partial P_{BS}}{\partial S}(t,S_t)
	\end{align*}
	und analog für Call. Wir berechnen explizit:
	\begin{align*}
		\frac{\partial C_{BS}}{\partial S}(t,S) &= \Phi(d_1) + S\phi(d_1)\cdot \frac{\partial d_1}{\partial S} - e^{-r(T-t)}K\phi(d_2)\frac{\partial d_2}{\partial S}\\
		&= \Phi(d_1) + \frac{\partial d_1}{\partial S}(S\phi(d_1)-e^{-r(T-t)}K\phi(d_2))
		\intertext{Nebenrechung:}
		e^{-rt}K/S\phi(d_2) &= e^{-r\tau}\frac{1}{\sqrt{2\pi}}K/S \exp(-\half \frac{\log(S/K) + r\tau - \sigma^2 r/\tau}{\sigma \sqrt{\tau}})\\
		&= \frac{1}{\sqrt{2\pi}} e^{-r\tau}K/S \exp(-\half \frac{(\log(S/K) + r\tau)^2}{\sigma^2 \tau} - 2 (\log(S/K) + r\tau) + \sigma^2\tau/4)\\
		&= \frac{1}{\sqrt{2\pi}}\exp(-\half \frac{(\log(S/K) + r\tau)^2}{\sigma^2 \tau} + (\log(S/K) + r \tau + \sigma^2\tau/4)\\
		&= \phi(d_1)
	\end{align*}
	also 
	\begin{align*}
		e^{-r(T-t)}K \phi(d_2) = S\phi(d_1)
	\end{align*}
	Das heißt: $\frac{\partial C_{BS}}{\partial S}(t,S) = \Phi(d_1)$. Put folgt analog oder mit Put-Call-Parität.
\end{proof}
\begin{*remark}
	\begin{itemize}
		\item $\frac{\partial C_{BS}}{\partial S}$ bzw. $\frac{\partial P_{BS}}{\partial S}$ lassen sich auch interpretieren als \begriff{Sensitivität} des Call- bzw. Put-Preises gegenüber Preisänderungen des Basisguts.
	\end{itemize}
\end{*remark}
Analog lassen sich die Sensitivitäten (``\begriff{Greeks}'') nach den weiteren Parametern berechnen.
\begin{definition}
	Die ``Greeks'' des BS-Preises sind folgende partielle Ableitungen\\
	\begin{tabular}{|c|c|c|c|c|} % TODO fix the table :/
		Bezeichg. & part. Abl. & Call & Put & Bemerkungen\\
		\hline
		Delta & $\frac{\partial}{\partial S}$ & $\Phi(d_1)$ & $-\Phi(-d_1)$ & Bestimmt die Replikations- bzw. Hedgingstrat.\\
		Gamma & $\frac{\partial^2}{\partial S^2}$ & $\frac{\phi(d_1)}{S\sigma\sqrt{T-t}}$ & $\frac{\phi(d_1)}{S\sigma\sqrt{T-t}}$ & Sensitivität von Delta ggü Basisgut, ``wie oft'' muss Strategie anpassen: Konvexität\\
		Vega & $\frac{\partial}{\partial \sigma}$ & $S_t \sqrt{T-t}\phi(d_1)$ & $S_t \sqrt{T-t}\phi(d_1)$ & Sensitivität gegenüber Änderg Volatitität ($\nu >0$)\\
		Theta & $\frac{\partial}{\partial t}$ & siehe ÜA & siehe ÜA & Änderung in der Zeit\\
		Rho & $\frac{\partial}{\partial r}$ & $K(T-t)(e^{-r(T-t)}) \Phi(d_2)$ & $-K(T-t)(e^{-r(T-t)}) \Phi(-d_2)$ & Sensitivität ggü Änderung Zinsrate
	\end{tabular}
\end{definition}
\begin{*remark}
	``Vega'' ist kein Buchstabe des griechischen Alphabets :/
\end{*remark}
\begin{conclusion}
	Der BS-Preis $C_{BS}(t,S)$ erfüllt folgende partielle DGL
	\begin{align*}
		\frac{\partial C_{BS}}{\partial t} + rS\frac{\partial C_{BS}}{\partial S} + \frac{\sigma^2}{2}S^2 \frac{\partial^2 C_{BS}}{\partial S^2}+rC_{BS} = 0 \tag{BS-PDE}\label{eq_3_1_BS_PDE}
	\end{align*}
	wobei $(t,s) \in [0,T]\times \R_{\ge 0}$. Mit Endwertbedingung
	\begin{align*}
		\lim_{t \to T} C_{BS}(t,S) = (S-K)_+
	\end{align*}
	Für $P_{BS}$ gilt die gleiche PDE mit Endwertbedingung
	\begin{align*}
		\lim_{t \to T} P_{BS}(t,S) = (K-S)_+
	\end{align*}
\end{conclusion}
\begin{proof}
	Siehe Übung 3.0.
\end{proof}
\begin{*remark}
	In Erweiterungen des BS-Modells gibt es \emph{keine} geschlossene Ausdrücke für Put/Call-Preise, aber eine PDE ähnlich zu \eqref{eq_3_1_BS_PDE} gilt weiterhin.
\end{*remark}
\section{Implizite Volatilität/ Grenzen des BS-Modells}
Wir schreiben etwas ausführlicher
\begin{align*}
	C_{BS}(t,S_t, T, K, \sigma) := C_{BS}(t, S_t)
\end{align*}
eine Abhängigkeit von $(T,K,\sigma)$ zu verdeutlichen.
\begin{theorem}[Implizite Volatlität] %3.6
	Sei $C_{\ast}(0,S_0,T, K)$ ein vorgegebener (beobachtbarer) Preis einer Call-Option mit Fälligkeit $T$, Ausübungspreis $K$ welcher innerhalb der Arbitragegrenzen liegt
	\begin{align*}
		(S_0 - e^{-rT}K)_+ < C_{\ast}(0,S_0, T,K) < S_0
	\end{align*}
	Dann existiert ein eindeutiges $\sigma_{\ast}(T, K) \in (0,\infty)$, die \begriff{implizite Volativilität} von $C_{\ast}$ sodass
	\begin{align*}
		C_{\ast}(0,S_0, T,K) = C_{BS}(0,S_0,T,K, \sigma_{\ast}(T,K))
	\end{align*}
	gilt.
\end{theorem}
\begin{*remark}
	$\sigma_{\ast}(T,K)$ ist Lösung eines inversen Problems.
	\begin{align*}
		\begin{matrix}
			\text{Vorwärtsproblem: } & \text{ Parameter }\to \text{ Call-Preis}\\
			\text{inverses Problem: } & \text{ Call-Preis }\to \text{ Parameter}
		\end{matrix}
	\end{align*}
	Kann zur impirischen Überpürfung des BS-Modells verwendet werden:
	\begin{itemize}
		\item BS-Modell passt gut zu Daten: $(T,K) \mapsto \sigma_{\ast}(T,K)$ ist annähernd konstant
		\item BS-Modell passt nicht gut zu Daten: $(T,K) \mapsto \sigma_{\ast}(T,K)$ variiert stark mit $(T,K)$
	\end{itemize}
\end{*remark}
Typische tatsächliche Beobachtung:\\
\begin{center}
	\begin{tikzpicture}
	\begin{axis}[
	xmin=0, xmax=1, xlabel=$K$,
	ymin=0, ymax=1, ylabel={$C_{BS}(T,h)$},
	title=Volativitäts-Smile,
	samples=400,
	axis x line=middle,
	axis y line=middle,
	domain=0:1,
	yticklabels={,,},
	xticklabels={,,},
	xtick style={draw=none},
	ytick style={draw=none},
	]
	\addplot[mark=none,smooth,blue] {0.3*(x-0.5)*(x-0.5)+0.2};
	\addplot[mark=none,smooth,red] {0.8*(x-0.5)*(x-0.5)+0.3};
	\node[blue] at (axis cs: 0.7,0.15) {$T$ groß};
	\node[red] at (axis cs: 0.7,0.4) {$T$ klein};
	
	\draw (axis cs: 0.5,0) -- (axis cs: 0.5,0.02);
	\node[align=center] at (axis cs: 0.5,0.09) {\scriptsize$K=S_0$\\ \scriptsize ATM};
	\node[align=center] at (axis cs: 0.25,0.05) {\scriptsize ITM};
	\node[align=center] at (axis cs: 0.75,0.05) {\scriptsize OTM};
	\end{axis}
	\end{tikzpicture}
\end{center}
Eigenschaften:
\begin{itemize}
	\item konvex
	\item assymetrisch (höher für große K)
	\item Minimum bei ATM
	\item flacher für lange Laufzeiten, steiler für kurze Laufzeiten
\end{itemize}
Form weist daraufhin, dass BS-Modell große Preissprünge des Basisguts \emph{unterschätzt}. Form des \person{Vola}-\person{Smiles} in Modellen jeweils von BS $\implies$ \emph{aktuelles Forschungsthema}.
\chapter{Optimale Investition}
\section{Das Anlageproblem}
\emph{Gegeben:} Vermögen $W$, Anlagegüter $S^1, \dots, S^n$ (Aktien, Anleihen, ...)\\
\emph{Gesucht:} Optimale Verteilung $W = W_1 + \dots + W_n$ auf $S^1 \dots S^n$\\
$S^1 \dots S^n$ weisen unterschiedliche Beträge, Risiken und typischerweise  Korrelationen auf.\\
\emph{Wir unterscheiden:}
\begin{itemize}
	\item \begriff{Einperiodenproblem}: Aufteilung wird heute $(t=0)$ festgelegt und bis zum Zeithorizont ($t=T$) beibehalten
	\item \begriff{Mehrperiodenproblem:} Umschichten zu mehreren Zeitpunkten $\set{t_0, t_1, \dots, t_N}$ möglich
\end{itemize}
\emph{Einfachstes Optimalitätsprinzip}: \begriff{\person{Pareto}-Optimalität}
\begin{itemize}
	\item Bei gleichem Risiko wird Anlage mit größeren Ertrag bevorzugt
	\item Bei gleichem Ertrag wird Anlage mit kleineren Risiko bevorzugt
\end{itemize}
d.h. \begriff{Pareto-Optimal} bedeutet, es gibt keine Anlagestrategie mit größerem Ertrag und kleinerem Risiko.\\
Zum Aufwärmen zwei Toy-Models ($=$ stark vereinfachte Beispiele)
\begin{itemize} %TODO fix structure 
	\item \emph{Toy-Model I}; Einperiodenmodell, eine risikofreie und eine risiko-behaftete Anlagenmodel.
	\begin{itemize}
		\item Zeithorizont sei $T=1$
		\item risikofrei: $S_0^0 = 1$, $S_T^0 = (1+r)$
		\item risikobehaftet: $S_0^1 = 1$, $S_T^1 = (1+R)$ mit $R$ stochastisch,
		\begin{align*}
			\mu &= \E[R] \quad \text{Ertrag}\\
			\sigma &= \sqrt{\Var(R)} \quad \text{Risiko}
		\end{align*}
		\item $S=\mu -r$ Überrendite (excess return).
		\item $S \le 0 \implies$ Investiere alles in $S^0$ (Pareto-Optimal)
		\item $S > 0 \implies ?$
		\item Teile $W$ in $(W_0, W-W_0)$ auf $(S^0,S^1)$ auf (Jetzt: $W=1$)
		\begin{align*}
			\begin{cases}
				W-W_0 < 0 &\quad \text{Leerverkauf}\\
				W_0 < 0 &\quad \text{Kredit}
			\end{cases}
		\end{align*}
		\item Endvermögen: $P_T = W_0(1+r) + (1-W_0)(1+R)$
		\item Erweiterte Rendite: 
		\begin{align*}
			\mu_p &= \E[P_T-1] = W_0(1+r)+(W-W_0)(1+\mu)\\
			&= W_0 r + (1-W_0)\mu\\
		\end{align*}
		\item Risiko: $\sigma_p = (1-W_0)\sigma$
		\item Überrendite $S_p = (1-W_0)(\mu -r)$
		\item \emph{Jede} Strategie ist Pareto-Optimal, d.h. Pareto-Prinzip  hilft nicht bei der Auswahl. Insbesondere ist \begriff{Sharp-Ratio} 
		\begin{align*}
			SR(W_0) = \frac{\text{``Überrendite''}}{\text{``Risiko''}} = \frac{S_p}{\sigma_p} = \frac{\mu -r}{\sigma} \quad \text{konstant!}
		\end{align*}
	\end{itemize}
	\item \emph{Alternative} zum Pareto-Prinzip Festlegen von individueller Risikoaversion (mehr dazu später)
	\item \emph{Toy-Model II:} Einperiodenproblem, zwei risikobehaftete Anlagemöglichkeiten
	\begin{itemize}
		\item Zeithorizont $T=1$, Vermögen $W=1$
		\item 
		\begin{align*}
			\begin{matrix}
				S_0^1 = 1 & S_T^1 = (1+R_1) & \text{mit $\E[R_1] = \mu$, $\Var(R_1) = \sigma_1^2 > 0$}\\
				S_0^2 = 1 & S_T^2 = (1+R_1) & \text{mit $\E[R_2] = \mu$, $\Var(R_2) = \sigma_2^2 > 0$}
			\end{matrix}
		\end{align*}
		und $R_1 \upmodels R_2$ (unabhängig)
		\item Portfoliowert: $P_T = W_1(1+R_1) + (1-W_1)(1+R_2)$
		\item Rendite: $\mu_p = \E[P_T -1] = W_1\E[R_1] + (1-W_1)\E[R_2] = \mu$
		\item Risiko:
		\begin{align*}
			\sigma_p^2 = \Var(P_T -1) &= \Var(W_1 R_1 + (1-W_1)R_2)\\
			&= W^2_1\cdot \sigma_1^2 + (1-W_1)^2\sigma_2^2
		\end{align*}
		\item Pareto-Optimales Portfolio: 
		\begin{align*}
			0 &= 2W_1\cdot \sigma_1^2 - 2(1-W_1)\sigma_2^2
			\sigma_2^2\\
			&= W_1(\sigma_1^2 + \sigma_2^2)\\
			&\implies W_{\ast} = \frac{\sigma_2^2}{\sigma_1^2\cdot \sigma_2^2} \in (0,1)
		\end{align*}
		Also existiert genau eine Pareto-Optimale Strategie
		\item Vermögen wird proportional zum Verhältnis der Risiken aufgeteilt
		\item Vermögen wird \emph{nicht} vollständig in risiko-ärmere Anlagen gesteckt, also findet eine \begriff{Diversifikation} statt
		\item $W_{\ast}$ ist auch die Strategie mit maximaler \emph{Sharp-Ratio}
	\end{itemize}
\end{itemize}
Als nächstes: Pareto-Optimale Portfolio mit $n>2$ Anlagegütern
\section{Exkurs: Optimierung mit Nebenbedingung}
\emph{Betrachte Optimierungsproblem:}
\begin{align*}
	\min f_0(x)\quad x \in \R^n\\
	\intertext{unter Nebenbedingungen}
	\begin{cases}
		f_i(x) \le 0 &\quad i = 1, ..., m\\
		h_i(x) = 0 &\quad i = 1, ..., p
	\end{cases}
	\tag{OPT}\label{eq_4_1_opt}
\end{align*}
\begin{itemize}
	\item $x \in \R^n$ welches (NB) erfüllt heißt \begriff{zulässig}
	\item $x_{\ast} \in \R^n$ welches \eqref{eq_4_1_opt} normiert heißt \begriff{Optimallösung}
	\item $p_{\ast} = f_0(x_0)$ heißt \begriff{Minimalwert}
\end{itemize}
\begin{definition}
	\begin{enumerate}
		\item Die Funktion
		\begin{align*}
		\LL(x, \lambda, \nu) = f_0(x) + \sum_{i=1}^{m}f_i(x)\lambda_i + \sum_{i=1}^p h_i(x)\nu_i
		\end{align*}
		mit $\lambda \in \R^m_{\ge 0}, \nu \in \R^p$ heißt \begriff{Lagrange-Zielfunktion}
		\item Die Funktion
		\begin{align*}
			g(\lambda, \nu) = \inf_{x \in \R^m} \LL(x, \lambda, \nu)
		\end{align*}
		heißt \begriff{(Langrange-) duale Funktion} für \eqref{eq_4_1_opt}.
	\end{enumerate}
\end{definition}
\begin{*remark}
	Als Infimum von $g(\lambda, \nu)$ lineare Funktionen ist $g$ \emph{konkav}.
	\begin{itemize}
		\item Die duale Funktion $g$ erzeugt \emph{untere Schranken} für $p_{\ast}$
		\item Begründung Sei $\overline{x} \in \R^n$ zulässig für \eqref{eq_4_1_opt}, d.h.
		\begin{align*}
			f_i(\overline{x}) &\le 0 \quad \forall i \in [n]\\
			h_i(\overline{x}) &= 0 \quad \forall i \in [p]
		\end{align*}
		\begin{align*}
			\implies \LL(\overline{x}, \lambda, \nu) = f_0(\overline{x}) + \underbrace{\sum_{i=1} f_i(\overline{x})\lambda_i + \sum_{i=1} h_i(\overline{x})\nu_i}_{=0} \le f_0(\overline{x})
		\end{align*}
		Also $g(\lambda, \nu) = \inf_{x \in \R^m} \LL(x, \lambda, \nu) \le \LL(\overline{x}, \lambda, \nu) \le f_0(\overline{x}) \quad \forall \overline{x}$ zulässig und damit folgt
		\begin{align*}
			g(\lambda, \nu) \le p_{\ast} \quad \forall \lambda \in \R^m_{>0}, \nu \in \R^n
		\end{align*}
		Die beste untere Schranke erhalten wir durch \emph{maximieren} über $\lambda, \nu$
	\end{itemize}
\end{*remark}
\begin{definition}
	Das duale Optimierungsproblem zu \eqref{eq_4_1_opt} ist
	\begin{align*}
		\max g(\lambda, \nu) \quad \lambda \in \R^m, \nu \in \R^p
		\intertext{unter Nebenbedingungen}
		\lambda_i \ge 0 \quad i = 1, \dots m \tag{D}\label{eq_4_2_Duality}
	\end{align*}
	Maximalwert: $d_{\ast}$
\end{definition}
\begin{itemize}
	\item Zwischen \eqref{eq_4_1_opt} und \eqref{eq_4_2_Duality} gilt \begriff{schwache Dualität}
	\begin{align*}
		d_{\ast} \le p_{\ast}
	\end{align*}
	\item Unter bestimmten Voraussetzungen gilt auch die \begriff{starke Dualität}
	\begin{align*}
		d_{\ast} = p_{\ast}
	\end{align*}
\end{itemize}
Das duale Problem hat auch eine Lösung $(\lambda_{\ast}, \nu_{\ast})$ und Maximalwert $d_{\ast}$
\begin{lemma}
	Zwischen \eqref{eq_4_1_opt} und \eqref{eq_4_2_Duality} gilt die \begriff{schwache Dualität}
	\begin{align*}
		d_{\ast} \le p_{\ast}\tag{WD}\label{eq_4_3_weak_Duality}
	\end{align*}
\end{lemma}
\begin{proof}
	SeSt.
\end{proof}
\begin{*remark}
	\begin{itemize}
		\item Die Differenz $p_{\ast} - d_{\ast} \ge 0$ heißt \begriff{Dualitätslücke} (duality gap).
		\item Wenn Dualitätslücke verschwindet \begriff{starke Dualität}
		\begin{align*}
			d_{\ast} = p_{\ast}
		\end{align*}
		\item hinreichende Bedingungen für starke Dualität existieren vor allem für \emph{konvexe} Probleme
	\end{itemize}
\end{*remark}
\begin{definition}
	Optimierungsproblem\eqref{eq_4_1_opt} ist konvex wenn $f_0$ konvex ist und die Menge der zulässigen Werte konvex ist. In diesem Fall kann \eqref{eq_4_1_opt} in folgende Form gebracht werden:
	\begin{align*}
		\min f_0(x) \quad x \in \R^n\\
		\intertext{unter NB}
		\begin{cases}
			f_i(x) \le 0 &\quad i \in [m]\\
			Ax = b &\quad
		\end{cases}\tag{K-OPT}\label{eq_4_4_kopt}
	\end{align*}
	mit $f_0, f_1, \dots, f_m$ konvex, $A \in \R^{p\times m}, b \in \R^p$.
\end{definition}
\begin{theorem}[Slaters-Bedingung]
	Betrachte das konvexe Optimierungsprobleme \eqref{eq_4_1_opt}. Wenn $x \in \R^n$ existiert mit
	\begin{align*}
		f_i(x) < 0 \quad \forall i \in [m] \und Ax = b
	\end{align*}
	dann gilt starke Dualität.
\end{theorem}
Für den Beweis verwende wir den Trennungssatz für konvexe Mengen.
\begin{theorem}[Trennungssatz für konvexe Mengen]
	Sei $A,B \subseteq \R^n$ konvex nichtleer und disjunkt, d.h.
	\begin{align*}
		A \cap B = \emptyset
	\end{align*}
	Dann existieren $a \in \R^n \setminus \set{0} \und b \in \R$, sodass
	\begin{align*}
		a^T x \ge b \quad \forall x \in A\\
		a^T x \le b \quad \forall x \in B
	\end{align*}
	Die Hyperebene $h = \set{x \in \R^n \colon a^T x = b}$ heisst \begriff{trennende Hyperbene} für $A \und B$
\end{theorem}
\begin{proof}
	Ohne Beweis.
\end{proof}
Skizze: 
\begin{proof}[Theorem 4.2?]
	Betrachte folgende Teilmengen von $\R^N = \R^{m+p+1}$, 
	\begin{align*}
		\G&=\set{(u,v,t) \in \R^N \colon \exists x \in \R^n \mit \begin{cases}
			f_i(x) = u_i, h_i(x) = v &\quad \forall i \in [m], Ax - b = v\\
			f_0(x) = t &\quad
			\end{cases}}\\
		\G&= \set{(f_1(x), \dots, f_m(x), h_1(x),h_p(x),f_0(x)) \in \R^N \colon x \in \R^n} \subseteq \AAA\\
		\AAA &= \set{(u,v,t) \in \R^N \colon \exists x \in \R^n \mit f_i(x) \le u_i \forall i \in [m], Ax-b = v, f_0(x) \le t} = \G \oplus \R^m_{\ge 0} \times \set{0}^p \times \R_{\ge 0}\\
		\BBB &= \set{(0,0,t)\in \R^N \colon t < p_{\ast}}
	\end{align*}
	Es gilt $\AAA \und \BBB$ sind konvex. Nun folgt die 
	\begin{itemize}
		\item Behauptung: $\AAA \cap \BBB = \emptyset$. Mit Widerspruch: Angenommen es existiert $(u,v,t) \in \AAA \cap \BBB$, dann gilt\\
		wegen $\BBB$
		\begin{align*}
		u=0, v = 0 \und t < p_{\ast}
		\end{align*}
		wegen $\AAA$
		\begin{align*}
		\exists x \in \R^n \mit &f_i(x) \le u_i = 0 \quad i \in [m]\\
		& h_i(x) = v_i = 0 \quad i \in [p]\\
		&f_0(x) \le t < p_{\ast}
		\end{align*}
		d.h. $x$ ist zulässig für \eqref{eq_4_4_kopt} und besser als optimal! $(f_0(x) < p_{\ast})$. Damit ist die Behauptung gezeigt und es folgt $\AAA \cap \BBB = \emptyset$.
		\item Wende Trennungssatz an: $\exists (\lambda, \nu, v) \in \R^N \setminus \set{0}$ und $\alpha \in \R$ mit
		\begin{align*}
			\lambda^T u + \nu^T v + \mu t \ge \alpha \quad (u,v,t) \in \AAA \tag{I}\label{proof_eq_4_2a}\\
			\lambda^T u + \nu^T v + \mu t \le \alpha \quad \forall (u,v,t) \in \BBB \tag{II}\label{proof_eq_4_2b}
		\end{align*}
		\eqref{proof_eq_4_2b} folgt $\mu t \le \alpha \forall t < p_{\ast}$ (da $a = 0, v = 0$) und damit gilt $\mu p_{\ast} \le \alpha$\\
		Aus \eqref{proof_eq_4_2a} bekommt man $\lambda_i \ge 0 \forall i \in [m]$ und $\mu \ge 0$ (sonst Widerspruch!). Dann nimmt man \eqref{proof_eq_4_2a} und \eqref{proof_eq_4_2b} zusammen und hat $\forall x \in \R^n$
		\begin{align*}
			&\sum_{i=1} \lambda_i f_i(x) + \sum_{i=1} \nu_i(Ax-b) + \mu f_0(x)\\
			&\le \lambda^T u + \nu^T v + \mu t \overset{\eqref{proof_eq_4_2a}}{\ge} \alpha \overset{\eqref{proof_eq_4_2b}}{\ge} \mu p_{\ast}\tag{$\ast$}\label{proof_eq_4_2c}
		\end{align*}
		Nun gibt es zwei Fälle:
		\begin{itemize}
			\item Fall: $\mu > 0$ Setze $\tilde{\lambda} = \lambda / mu$, $\tilde{\nu} = \nu /\mu$, damit\\
			$\exists x \in \R^n \mit \sum_{i=1} \tilde{\lambda}_i f_i(x) + \sum_{i=1} \nu_i (Ax-b) + f_0(x) \ge p_{\ast}$ und damit folgt $g(\tilde{\lambda}, \tilde{\nu}) \ge p_{\ast}$, d.h. $d_{\ast} = \max_{(\lambda, \nu)\in \R_{\ge 0}^m \times \R^n} g(\lambda, \nu) \ge p_{\ast}$. Aber mit schwacher Dualität: $d_{\ast} \le p_{\ast}$
			\item Fall: $\mu = 0$ (kann nicht eintreten, weil ...). Aus \eqref{proof_eq_4_2c} bekommen wir
			\begin{align*}
				\sum_{i=1} \lambda_i f_i(x) + \sum_{i=1} \nu_i(Ax-b) \ge 0 \quad \forall x \in \R^n
			\end{align*}
			Slaters-Bedingung $\exists \overline{x} \in \R^n$ mit $f_i(x) < 0, i \in [m]$ und $A\overline{x} - b = 0$ und damit
			\begin{align*}
				\sum_{i=1}\underbrace{\lambda_i}_{\ge 0}\underbrace{f_i(\overline{x})}_{< 0} > 0 \implies \lambda = 0
			\end{align*}
			$(\lambda, \nu, \mu) = (0, \nu, 0)\in \R^N \setminus \set{0}$ impliziert $\nu \neq 0$ und $\nu^T(Ax-b) = 0$, dann existiert nach $\overline{x}$ mit $\nu^T(A\overline{x} - b) < 0$ und das ist der Widerspruch, d.h. $\mu = 0$ tritt nicht ein.
		\end{itemize}
	\end{itemize}
\end{proof}
\section{Die \person{Markowitz}-Modelle}
\subsection*{Markowitz-Modell I}
(Portfolio-Optimierung \emph{ohne} risikofreie Anlage)\\
Anlagegüter $S = (S^1, \dots, S^n)$ mit stochastische ein-perioden Renditen $R = (R^1, \dots, R^n)$, d.h. $S_T = S_0^i(1+R^i), i \in [n]$ mit auf Analgegüter $S^1, \dots, S^n$ aufteilen $p_i$ Investitionen in $S^i$, d.h. $p_1 + \dots + p_n = W = 1$.
\begin{itemize}
	\item Erwartungswert: $\mu = \E[R] \in \Rn, \mu = (\mu_1, dots, \mu_n)^T$
	\item $\Sigma = \E[(R-\mu)(R-\mu)^T]\quad (n\times 1)(1\times n) = (\Sigma_{ij})_{i,j \in [n]}$
	\begin{align*}
		\Sigma_{ij} = \Var(R^i)\\
		\Sigma_{ij} = \Cov(R^{i},R^j) \mit i \neq j
	\end{align*}
	\begin{itemize}
		\item \emph{Annahme:} $\Sigma$ ist regulär, d.h. $\Sigma^{-1}$ existiert.
		\item \emph{Ziel:} Anlagemengen $W=1$. 
		\item \emph{Erwartete Rendite:} $\mu_p = \E[p^T R] = p^T \mu$
		\item \emph{Risiko (Standardbereich):}
		\begin{align*}
			\sigma_p &= \sqrt{\Var(p^T R)} = \sqrt{\E[(p^T (R-\mu))^2]}\\
			&=\sqrt{\E[p^T (R-\mu)(R-\mu)^Tp]} = \sqrt{p^T \Sigma p}
		\end{align*}
		\item \emph{Optimales Anlageproblem:} Minimiere Risiko, gegeben Zielrendite $\mu_{\ast}$
		\begin{align}
			\begin{cases}
				\min \half p^T \Sigma p & \quad\text{über } p \in \R^n\\
				\text{ unter NB} &\quad p^T \mu = \mu_{\ast} \text{ ( Zielrendite)}\\
				&\quad p^T \indi = \indi (\indi = (1, \dots, 1) \in \R^n)
			\end{cases} \label{eq_Markow_one}\tag{Mark I}
		\end{align}
	\end{itemize}
	\item Die Lagrange-Zielfunktion: 
\begin{align*}
	\LL(p,\lambda_1, \lambda_2) = \half p^T \Sigma p + \lambda_1(\mu_{\ast} - p^T\mu) + \lambda_2(1-p^T \indi) \quad \mit \lambda_1, \lambda_2 \in \R
\end{align*}
	\item Die duale Funktion:
\begin{align*}
	g(\lambda_1, \lambda_2) &= \inf_{p \in \Rn} \LL(p, \lambda_1, \lambda_2)\\
	\nabla_p \LL(p, \lambda_1, \lambda_2) &= \Sigma - \lambda_1 \mu - \lambda_2 1 = 0\\
	\implies p_{\ast} &= \Sigma^{-1}(\lambda_1 \mu + \lambda_2 \mu)
\end{align*}
d.h. $g(\lambda_1, \lambda_2) = \LL(p_{\ast}, \lambda_1, \lambda_2)$
\begin{align*}
	\LL(p_{\ast}, \lambda_1, \lambda_2) &=
	\half (\lambda_1 \mu + \lambda_2 1)^T \Sigma^{-1}\Sigma\Sigma^{-1}(\lambda_1\mu - \lambda_2 1) - (\lambda_1 \mu + \lambda_2 1)^T \Sigma^{-1}(\lambda_1 \mu - \lambda_2 1) + \lambda_1 \mu_{\ast} + \lambda_2\\
	&= -\half (\lambda_1^2 a + 2 \lambda_1 \lambda_2 b + \lambda_2^2 c) + \lambda_1 \mu_{\ast} + \lambda_2
\end{align*}
mit
\begin{align*}
	a = \mu^T \Sigma^{-1}\mu, b = \mu^T\Sigma\indi, c= \indi^T\Sigma\indi
\end{align*}
Es gilt $a \ge 0, c \ge 0$ und (mit Cauchy-Schwarz) und damit $ac \ge b^2$
\item Maximieren von $g$:
	\begin{align*}
		\frac{\partial g}{\partial \lambda_1} &= -a \lambda_1 - b \lambda_2 + \mu_{\ast} = 0 \implies a\lambda_1 + b \lambda_2 = \mu_{\ast} \tag{I}\label{eq_mark_one_I}\\
		\frac{\partial g}{\partial \lambda_2} &= -b \lambda_1 - 1 \lambda_2 - 1 = 0 \implies b\lambda_1 + c \lambda_2 = 1 \tag{II}\label{eq_mark_one_II}
	\end{align*}
	\begin{align*}
		-b\eqref{eq_mark_one_I} + a\eqref{eq_mark_one_II}:\quad (ac - b^2)\lambda_2 = a - b\mu_{\ast} &\implies \lambda^{\ast}_2 = \frac{a - b \mu_{\ast}}{ac - b^2}\\
		c\eqref{eq_mark_one_I} -b\eqref{eq_mark_one_II}:\quad (ac - b^2)\lambda_1 = c\mu_{\ast} - b &\implies \lambda^{\ast}_2 = \frac{c\mu_{\ast} - b }{ac - b^2} \quad (\text{aber nur für } ac > b^2 )
	\end{align*}
\item Minimierer von \eqref{eq_Markow_one}:
\begin{align*}
	p_{\ast} = \lambda_1^{\ast}\Sigma^{-1}\mu + \lambda_2^{\ast}\Sigma^{-1}\indi
\end{align*}
\end{itemize}
\begin{conclusion}[Tobin-Two-Fund Seperation]
	Jedes Pareto-Optimale Portfolio für \eqref{eq_Markow_one} kann (unabhängig von $a$!) als Linearkombination der zwei Portfolio
	\begin{align*}
		\underbrace{p^{\ast}_1 = \Sigma^{-1}\mu}_{\text{renditeorientiertes Portofolio}} \und \underbrace{p_2^{\ast} = \Sigma^{-1}\indi}_{\text{sicherheitsorientiertes Portofolio}}
	\end{align*}
	dargestellt werden.
\end{conclusion}
\begin{*remark}
	\begin{itemize}
		\item Gewichtung des Portfolios $p^{\ast}_1 \und p_2^{\ast}$ orientiert sich am Renditeziel $\mu_{\ast}$.
		\item $p^{\ast}_1 \und p_2^{\ast}$ sind breit diversifiziert, d.h. nutzen alle Anlagegüter $S = (S^1, \dots, S^n)$
		\item $p^{\ast}_1 \und p_2^{\ast}$ kann man als Anlagefunds interpretieren welche Vermögen entsprechend der Portfolio $p^{\ast}_1, p^{\ast}_2$ anlegen. Diese zwei Fonds sind ausreichend (unabhängig von $\mu_{\ast}$) um Vermögen Pareto-optimal zu investieren!
	\end{itemize}
\end{*remark}
Zuletzt wollen wir noch Risiko der optimalen Strategie $p_{\ast}$ berechnen:
\begin{align*}
	\sigma_{\ast}^2 &= \Var(p_{\ast}^T R) = \E[(p_{\ast}^T (R-\mu))^2] = p_{\ast}^T \Sigma p_{\ast}\\
	&= (\lambda_1^{\ast}\mu + \lambda_2^{\ast}\indi)^T \Sigma^{-1}\Sigma\Sigma^{-1}(\lambda_1^{\ast}\mu + \lambda_2^{\ast}\indi)\\
	&= (\lambda_1^{\ast 2}2 a) + 2 \lambda_1^{\ast}\lambda_2^{\ast} b + (\lambda_2^{\ast 2})c\\
	&= \frac{1}{a^2-b^2}^2 (((c\mu_{\ast} -b)^2)a + 2(\mu_{\ast} - b)(a-b \mu_{\ast})b + (a - b\mu_{\ast})^2c)\\
	&= \frac{1}{ac-b^2}(c \mu_{\ast}^2 - 2b\mu_{\ast} +a^2)
\end{align*}
Graph von $(\sigma_{\ast}, \mu_{\ast})$ ist ein Hyperbel-ast:\\
siehe picture phone ...\\
Nennt sich ``Markowitz-Bullet''!
\subsection*{Markowitz-Modell II}
(Optimale Investition \emph{mit} risikofreier Anlage)\\
\begin{itemize}
	\item Anlagegüter $S = (S^1, \dots, S^n)$ mit ein-perioden Rendite $R = (R^1, \dots, R^n)$
	\item Zusätzlich risikofreie Anlage $S^0$ mit Verzinsung $r$. Wegen $W=1$aufgestellt zu $1 = p_0 + p_1 + \dots p_n$. Wir setzen $p = (p_1,\dots, p_n)^T \in \R^n$
	\item Erwartete Rendite: $\mu = \E[p^T R + (1-p^T \indi)r] = p^T(\mu - r\indi) + r$
	\item Risiko: $\sigma_{\ast} = \sqrt{\Var(p^T R)} = \sqrt{p^T \Sigma p}$
	\item Anlageproblem:
	\begin{align*}
		\begin{cases}
			\min \half p^T \Sigma p & \quad p \in \R^n\\
			\text{ unter NB} &\quad p^T (\mu-r\indi) = \mu_{\ast} -r (\text{ Zielrendite})\\
		\end{cases}\tag{Mark II}\label{eq_Markow_two}
	\end{align*}
	\item Lagrange ÜA
	\item Optimierer: $p_{\ast} = \lambda_{\ast}\Sigma^{-1}(\mu - r \indi)$ mit $\lambda_{\ast} = \frac{\mu_{\ast} - r}{a^2 - 2br + cr^2}$
\end{itemize}
\begin{conclusion}[Tobin's One-Fund-Theorem]
	Jedes Pareto-Optimale Portfolio für \eqref{eq_Markow_two} kann als Linearkombination der risko-freien Anlage und des Portfolio
	\begin{align*}
		\Sigma^{-1}(\mu -r\indi)
	\end{align*}
	dargestellt werden.
\end{conclusion}
Graph von min. Risiko $\sigma_{\ast}$ und Zielrendite $\mu_{\ast}$
siehe phone
\begin{*remark}[nominales vs. relatives Portfolio]
	\begin{itemize}
		\item $\vartheta = (\vartheta_1, \dots, \vartheta_2) \in \R^n$ mit $\vartheta_i$ Stückzahl von Anlagegut $S^i$ und Portfoliowert
		\begin{align*}
			V_0 &= \vartheta^T S_0 = \sum_{i=1}^n \vartheta_i S_0^i = w \quad \dots \text{ Anfangskapital}\\
			V_T &= \vartheta^T S_T = \sum_{i=1}\vartheta_i S_T^i
		\end{align*}
		\item relatives Portfolio: $p := (p_1, \dots, p_n) \in \R^n$ mit $p_i = \frac{\vartheta_i S_0^i}{W}$ Vermögensanteil in $S^i$
		\begin{align*}
			\sum_{i=1}^n p_i = \frac1w\sum_{i=1}^n \vartheta_i S_0^i = \frac w w = 1
		\end{align*}
		\item Renditen: 
		\begin{itemize}
			\item Einzelnes Anlagegut: $R_i = \frac{S_T^i - S_0^i}{S_0^i}$
			\item Gesamtes Portfolio: 
			\begin{align*}
				R_p &= \frac{V_T - V_0}{V_0} = 1/w (\sum_{i=1}^n \vartheta_i S_0^i - \vartheta_i S_0^i)\\
				&= 1/w \sum_{i=1}^n \vartheta_i(S_T^i - S_0^i) = \sum_{i=1}^n \underbrace{\frac{\vartheta_i S_0^i}{w_{p_i}}}R_i\\
				&= \sum_{i=1}^n p_i R_i = p^T R \quad \dots \text{ linear in }p
			\end{align*}
		\end{itemize}
	\end{itemize}
\end{*remark}
\section{Capital Asset Pricing Model (CAPM)}
\begin{itemize}
	\item Ausgangspunkt: Optimalportfolio in \eqref{eq_Markow_two} $p_{\ast} = \lambda \Sigma^{-1}(\mu - r\indi)$. Normiere $p_{\ast}^T \indi = \indi \implies \lambda_{\ast} = \frac{1}{\indi^T \Sigma^{-1} (\mu - r \indi)} = \frac{1}{b-cr}$ (Marktportfolio)
	\item Wert des Marktportfolios: $M_0 = 1, M_T = (1+ p_{\ast}^T R)$, Rendite $R_M = p_{\ast}^T R$
	\item Zentrale Idee des CAPM:
	\begin{itemize}
		\item Betrachte $M$ als \emph{beobachtbare Größe}
		\item Aktienindex DAX oder S\&P500 sollte gute Näherung für $M$ ergeben.
	\end{itemize}
\end{itemize}
\paragraph*{Wir betrachten folgende Kennzahlen:}
\begin{itemize}
	\item Überschussrendite: [excess return] ($\alpha$)
	\begin{align*}
		\alpha_i &= \E[R_i] - r = \mu_i - r \quad \dots \text{ für Wertpapiere }S^i\\
		\alpha_M &= \E[R_M] - r = p_{\ast}^T\mu -r = \frac{\mu^T \Sigma^{-1}(\mu - r\indi)}{\indi^T\Sigma^{-1}(\mu - r \indi)} - r \\
		&= \frac{a - rb}{b-rc} - r = \frac{a - 2rb +r^2c}{b-cr}
	\end{align*}
	\item \begriff{Beta-Koeffizient}
	\begin{align*}
		\beta_i = \frac{\Cov(R_1,R_M)}{\Var(R_M)}
	\end{align*}
	skalierte Kovarianz zwischen Erträgen von $S^i$ und $M$
	\begin{itemize}
		\item Maß für Korrelation zwischen Wertpapiere $S^i$ und Marktportfolio
		\item Volle Kovarianzmatrix wird \emph{nicht} benötigt
	\end{itemize}
	\item Wir berechnen:
	\begin{align*}
		\beta_i &= \frac{\E[(R_i - \mu_i)(R_M-\mu_M)]}{\E[(R_M - \mu_M)^2]} = \frac{\E[e_i^T(R-\mu)(R-\mu)^Tp_{\ast}]}{\E[p_{\ast}^T(R-\mu)(R-\mu)^Tp_{\ast}]} = \frac{e_i^T \Sigma p_{\ast}}{p_{\ast}^T \Sigma p_{\ast}}\\
		&= \frac{\lambda_{\ast} e_i^T \Sigma \Sigma^{-1}(\mu - r \indi)}{\lambda_{\ast} (\mu - r \indi)^T \Sigma^{-1}\Sigma \Sigma^{-1}(\mu - r \indi)}\\
		&= \frac{\mu_i - r}{\lambda_{\ast}(a-2rb+r^2c)} = \frac{\mu_i -r}{\mu_M - r} \quad \mit \alpha_M := \mu_M - r
	\end{align*}
	d.h. es gilt CAPM-Gleichung
	\begin{align*}
		\beta_i(\mu_M -r) = (\mu_i-r) \quad \forall i \in [n]
	\end{align*}
	$\beta_i$ ist Beta-Koeffizient von $S^i$, $(\mu_M -r)$ ist Überschussrendite Marktportfolio, $(\mu_i - r)$ ist Überschussrendite von $S^i$ (alpha)
	\begin{itemize}
		\item Kann als Regressionsgleichung für $(\alpha_i, \beta_i)_{i \in [n]}$ interpretiert werden
		\item Entscheidend für Attraktivität eines Wertpapieres $S^i$ ist nicht die Überrendite $\alpha_i = \mu_i -r$ \emph{alleine}, sondern \emph{in Relation} zu $\beta_i$
		\item CAPM kann empirisch überprüft werden durch Schätzung $(\hat{\alpha}_i,\hat{\beta}_1)$ und Regression
		\begin{align*}
			\hat{\beta}_i\cdot (\mu_M - r) = \hat{\alpha}_i + \epsilon_i \tag{$\ast$}\label{eq_4_4_beta}
		\end{align*}
		Ideal, wenn $\sum_{i=1}^n \epsilon_i^2$ klein ist
		\item sketch, see phone ...
	\end{itemize}
	\item Kritik am CAPM:
	\begin{itemize}
		\item Regression \eqref{eq_4_4_beta} empirisch im Allgemeinen nicht besonders gut (Fehler $\sum \epsilon_i^2$ groß)
		\item Schätzung von $\mu_i, \mu_M$ schwierig
	\end{itemize}
	\item Erweiterungen:
	\begin{itemize}
		\item Ergänze Schätzer $\hat{\mu}_i \und \hat{\mu}_M$ von Expertenmeinungen und Konfidenzaussagen führt zum \person{Black}-\person{Littermann}-Modell
		\item Erweitere Regressionsgleichung \eqref{eq_4_4_beta} um weitere Variablen und führt zum Beispiel zum \person{Fame}-\person{French}-Modell 
	\end{itemize}
\end{itemize}
\section{Präferenzordnungen und Erwartungsnutzen}
\begin{itemize}
	\item Kritik an Markowitz:
	\begin{itemize}
		\item Ist Standardabweichung $\sqrt{\Var(R)}$ nicht unbedingt gutes Risikomass
		\item Entscheidungen unter Unsicherheit meist komplexer als durch Erwartungs-Varianzprinzip beschrieben
	\end{itemize}
	\item Axiomatischer Zugang: Präferenzordnungen\\
	Sei $(\O,\F,\P)$ Wahrscheinlichkeitsraum, $L_1(\O,\F,\P)$ Raum der integrierbaren Zufallsvariablen.
	\begin{align*}
		\MM = \set{\text{Menge der Verteilungsfunktionen} F_X \text{ von } X \in L_1(\O,\F,\P)}
	\end{align*}
	Seien $X,Y \in L_1(\O)$ Interpretation risikobehafteter Auszahlungen (``\begriff{Lotterie}''). Wir wollen Ordnungsrelation ``$\vartrianglelefteq$'' mit Bedeutung:
	\begin{align*}
		X \vartrianglelefteq Y \Leftrightarrow Y \text{ wird bevorzugt gegenüber }X
	\end{align*}
	Beschränken uns auf ``verteilungsinvariante'' POs welche durch Relation auf $\MM$ erklärt werden können, d.h.
	\begin{align*}
		X \vartrianglelefteq Y \Leftrightarrow F_X \vartrianglelefteq F_Y
	\end{align*}
\end{itemize}
\begin{definition}
	Eine Relation ``$\vartrianglelefteq$'' auf $\MM$ heißt Praferenzordnung (PO), wenn gilt:
	\begin{itemize}
		\item (Reflexiv) $F \vartrianglelefteq F\quad \forall F \in \MM$
		\item (Transitiv) $(F \vartrianglelefteq G)\vee (G \vartrianglelefteq H) \implies (F \vartrianglelefteq H)\quad \forall F,G \in \MM$
		\item (Vollständig) $\forall F,G \in \MM$ gilt: $(F\vartrianglelefteq G)\wedge (G \vartrianglelefteq F)$
	\end{itemize}
\end{definition}
\begin{*remark}
	\begin{itemize}
		\item Menge $\MM$ ist \emph{konvex}, d.h. $\forall F,G \in \MM  \und \alpha \in [0,1]$ gilt
		\begin{align*}
			H - (1-\alpha)F + \alpha G \in \MM \tag{$\oplus$}\label{eq_4_5_convex_PO}
		\end{align*}
		\item \eqref{eq_4_5_convex_PO} lässt sich als ``Mischen'' von $F$ und $G$ interpretieren
		\item Sei $X \sim F_X, Y \sim F_Y$ und $A \upmodels (X,Y)$ mit $\P(A=0) = \alpha, \P(A=1) = 1-\alpha$. Dann gilt:
		\begin{align*}
			(1-A)X + A Y \sim (1-\alpha)F_X + \alpha F_Y
		\end{align*}
		\item Aus gegebenen PO können wir ableiten
		\begin{itemize}
			\item Äquivalenzrelation $F\sim G \Leftrightarrow (F \vartrianglelefteq G)\vee (G \vartrianglelefteq F)$ ``Indifferenz zwischen $F$ und $G$''
			\item strikte Relation $F \lhd G \Leftrightarrow (F \vartrianglelefteq G)\vee (G \vartrianglelefteq F)$ ``$G$ wird strikt gegenüber $F$ bevorzugt''
			\item Für ``deterministische Zufallsvariablen'' $a \in \R$ ist die Verteilungsfunktion $F_a = \indi_{[a, \infty)}$
		\end{itemize}
	\end{itemize}
\end{*remark}
Eine PO kann folgende Eigenschaften besitzen:
\begin{enumerate}
	\item Monotonie: $\forall a,b \in \R$ mit $a\le b$ gilt $F_a \vartrianglelefteq F_b$ (``mehr besser als weniger'')
	\item Risikoaversion: $\forall X \in L_1(X)$ gilt: $F_X F_{\E[X]}$ (``sicher besser als unsicher'')
	\item Mittelwertseigenschaft: Sei $F,G,H \in \MM$ mit $F \vartrianglelefteq G \vartrianglelefteq H$. Dann existiert $\alpha \in [0,1]$ mit $(a-\alpha)F + \alpha H \sim G$
	\item Unabhängigkeitsaxiom: $\forall F,G,H \in \MM$ gilt
	\begin{align*}
		F \vartrianglelefteq G \implies (1-\alpha)F + \alpha H \vartrianglelefteq (1-\alpha) + \alpha H \quad \forall \alpha \in [0,1]
	\end{align*} 
\end{enumerate}

\part*{Anhang}
\addcontentsline{toc}{part}{Anhang}
\appendix

\nocite{*}
\bibliography{literatur}
\bibliographystyle{acm}

%\printglossary[type=\acronymtype]

\printindex

\end{document}