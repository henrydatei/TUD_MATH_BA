\section{Permutationsdarstellungen und Gruppenoperationen}
\subsection*{Allgemeines}
Motivation: $H < G \implies \lnkset{G}{H}$\\
Notation: $x,y \in G$ konjugiert $\Leftrightarrow \exists g \in G \colon g^{-1}yg$ $\Leftrightarrow \exists h \in G\colon x = hyh^{-1}$
\begin{definition}
	Sei $G$ Gruppe, $X \neq \emptyset$ Menge, $S_X$ Permutationsgruppe von $X$. Dann
	\begin{enumerate}
		\item Eine \begriff{Permutationsdarstellung} ist ein Gruppenmorphismus
		\begin{align*}
			\theta\colon G \to S_X
		\end{align*}
		\item Eine \begriff{(linke) Operation} von $G$ auf $X$ ist eine Abbildung
		\begin{align*}
			G\times X \to X \mit (g,x) \mapsto g\cdot x
		\end{align*}
		so dass: ($\forall x \in X, \forall g,h \in G$)
		\begin{itemize}
			\item $g \cdot x = x$
			\item $(g\cdot h)\cdot x = g \cdot (h\cdot x)$
		\end{itemize}
	\end{enumerate}
\end{definition}
\begin{proposition}
	Es gibt eine bijektive Korrespondenz zwischen den Operationen von $G$ auf $X$ und den Darstellungen von $G$ als Permutationen von $X$.
\end{proposition}
\begin{proof}
	\begin{itemize}
		\item Sei $\theta\colon G \to S_X$ eine Permutationsdarstellung. Definiere $G\times G \to X$, wobei $g \cdot x := \theta(g)(x)$
		\begin{align*}
			1 \cdot x = \theta(1)(x) = \id_X(x) = x\\
			(gh)\cdot x = \theta(gh)(x) = \theta(g)\cdot\theta(h)(x) = \theta(g)(h\cdot x) = g(h\cdot x) \quad g,h \in G
		\end{align*}
		ist Operation von $G$ auf $X$.
		\item Sei $G \times X \to X \mit (g,x) \mapsto g x$ eine Operation. Für jedes $g \in G$: $\theta(g):= x \mapsto g \cdot x$ und damit haben wir $\theta\colon G \to \Set(X,X)$. Sei $g,h \in G$ und $x \in X$
		\begin{align*}
			\theta(gh)(x) &= (gh)\cdot x\\
			&= g \cdot (h \cdot x)\\
			&= g\cdot (\theta(g)(x))\\
			&= \theta(g)(\theta(h)(x))\\
			&= \theta(g) \cdot \theta(h)(x)
		\end{align*}
		also gilt $\theta(gh) = \theta(g)\cdot \theta(h)$.
		\begin{align*}
			\theta(1)(x) = 1 \cdot x = x \quad \forall x \in X \implies \theta(1) = \id_X
		\end{align*}
		also $\theta$ Morphismus von Monoide.
		\begin{align*}
			\forall g \in G \colon \theta(g)\cdot \theta(g^{-1}) = \theta(g\cdot g^{-1}) = \theta(1) = \id_X = \theta(g^{-1}) \cdot \theta(g^{-1})
		\end{align*}
		und wir haben $\theta(g)$ bijektiv mit Inverse $\theta(g^{-1})$ und damit $\theta\colon G \to S_X$
	\end{itemize}
\end{proof}
\begin{*example}
	Setze Notation: $G \Circlearrowleft X$ $G$ operiert auf $X$.
	\begin{enumerate}
		\item $X \neq \emptyset$ Menge $\forall G < S_X \implies G$ operiert natürlich auf $X$.
		\item $D_n$ operiert auf $P_n$ (reguläre Polygone mit $n$ Seiten)
		\item $V$ Vektorraum $\implies$ $\GL(V)$ operiert auf $V$
		\item $G \Circlearrowleft X \implies$
		\begin{itemize}
			\item $G \Circlearrowleft X^n \mit (x_1, \dots, x_n) \in X^n$ und $g\colon (x_1, \dots, x_n) := (g\cdot x_1, \dots, g \cdot x_n)$
			\item $G \Circlearrowleft \powerset(X)$ und $A \subseteq X$, sowie $g \cdot A = \set{g\cdot a \mid a \in A}$
		\end{itemize}
		\item $H <G \implies G \Circlearrowleft \lnkset{G}{H}$ und $aH$ mit $g(aH) = (ga)H$
	\end{enumerate}
\end{*example}
\subsection*{Morphismen}
\begin{definition}
	Ein Morphismus zwischen zwei Operationen $(G,X)$ und $(H,Y)$ ist ein Paar $(\phi, \alpha)$, wobei
	\begin{itemize}
		\item $\phi\colon G \to H$ Gruppenhomomorphismus
		\item $\alpha\colon X \to Y$ Abbildung
		\item $\forall g \in G \und x \in X\colon$ $\alpha(g \cdot x) = \phi(g)\cdot \alpha(x)$
		\[
			\begin{tikzcd}
			X \arrow[r, "\alpha"] \arrow[d, "g"] & Y \arrow[d, "\phi(g)"] \\
			X \arrow[r, "\alpha"]                & Y                     
			\end{tikzcd}
		\] 
	\end{itemize}
\end{definition}
\begin{*example}
	Sei $G=\Z \Circlearrowleft \Z_n=X, n \neq 0, \forall g \in \Z, \overline{x} \in \Z_n\colon g \cdot \overline{x} = \overline{g+x}$ ist \emph{nicht treu}, da $\forall g \in \Z \colon (g+n)\cdot \overline{x} = \overline{g+n+x} = g \cdot \overline{x}$
	\[
		\begin{tikzcd}
		G \arrow[r, "\theta"] \arrow[d]                       & S_X \\
		G/\ker \theta \arrow[ru, "\overline{\theta}", dashed] &    
		\end{tikzcd}
	\]
	also $\overline{\theta}$ injektiv $\implies \lnkset{G}{\ker \theta} \Circlearrowleft X$ treu.\\
	Für $\Z \Circlearrowleft \Z_n \colon \ker \theta = n\Z \implies \Z_n \Circlearrowleft \Z_n$ treu, da
	\begin{align*}
		\overline{x}, \overline{y} \in \Z_n\quad k\in \Z, \text{ so dass } \overline{k} = \overline{x-y} \in \Z_n\\
		k \cdot \overline{y} = \overline{k+y} = \overline{x-y+y} = \overline{x}
	\end{align*}
	Sei $n \in \N, n \ge 1$ und $\Z \Circlearrowleft \Z_n := \set{\overline{0}, \overline{1}, \overline{2}, ..., \overline{n-1}}$, also $D_n \Circlearrowleft P_n$.
	\begin{align*}
		x \in \Z \colon x \cdot \overline{y} := \overline{x+y}\\
		\alpha\colon \Z_n \to P_n\\
		\phi\colon \Z \to D_n
	\end{align*}
	(Als Beispiel kann man sich die $P_5$ nehmen und aufmalen ;))
	Definiere $\phi(1)$. Drehung von Zentrum $Z$ um den Winkel $\sfrac{2\pi}{n}$.
	\begin{align*}
		\alpha(1 \cdot \overline{0}) = \alpha(\overline{1})\\
		\phi(1)\cdot \alpha(\overline{0}) = \alpha(\overline{1})\\
		\implies (\phi,\alpha)\colon \text{ Morphismus}
	\end{align*}
\end{*example}
\begin{definition}
	\begin{itemize}
		\item $H = G, \phi = \id_G$, also ist $G$-Morphismus
		\item $(\phi, \alpha)$ ist ein Isomorphismus, wenn $\phi$ Gruppenmorphismus und $\alpha$ Bijektion
	\end{itemize}
\end{definition}
\subsection*{Bahnen}
\begin{lemma}
	Sei $G \Circlearrowleft X$. Definiere eine Relation $\sim$ auf $X$:
	\begin{align*}
		\forall x,y \in X \colon x \sim y \Leftrightarrow \exists g \in G\colon y = y \cdot x
	\end{align*}
	Dann ist $\sim$ eine Äquivalenzrelation.
\end{lemma}
\begin{proof}
	\begin{itemize}
		\item $x \sim x$, da $x = 1 \cdot x$
		\item $x \sim y \implies y \sim x$, da
		\begin{align*}
			y = y\cdot x &\implies g^{-1}y = y^{-1}(gx)\\
			&\implies g^{-1}y = (g^{-1}y)x\\
			& \implies g^{-1}y = x
		\end{align*}
		\item $x \sim y$ und $y \sim z \implies x \sim z$, da
		\begin{align*}
			y = g\cdot x \und z = h \cdot y \implies z = h(g\cdot x) = (hg)\cdot x
		\end{align*}
	\end{itemize}
\end{proof}
\begin{definition}
	\begin{itemize}
		\item Für alle $x \in X$ \begriff{Äquivalenzklassen} von $X$: $G\cdot x := \set{gx \mid g \in G}$.
		\item $Gx$ wird \begriff{Bahn von $x$} genannt und $\abs{Gx}$ die \begriff{Länge} von $Gx$.
		\item $\forall x,y \in X$ entweder $Gx = Gy$ oder $Gx \cap Gy = \emptyset$:
		\begin{align*}
			X = \bigcup_{x \in X}G \cdot x
		\end{align*}
	\end{itemize}
\end{definition}
\begin{*example}
	Sei $G=(\R,+)$ und $X = \C$:
	\begin{enumerate} % picture from Florian
		\item Translation: $a \in \C\setminus \set{0}$ und $\forall \lambda \in \R, z \in \C\colon \lambda \cdot z := z + \lambda a$ (nicht transitiv). Ist \emph{treu}, da
		\begin{align*}
			\forall \lambda_1, \lambda_2 \in \R, z \in \C\colon &z + \lambda_1 a = z + \lambda_2 a\\
			&\Leftrightarrow z + \lambda_1 a = z + \lambda_2 a\\
			&\Leftrightarrow (\lambda_1-\lambda_2)a = 0\\
			&\Leftrightarrow \lambda_1 = \lambda_2
		\end{align*}
		\item Drehungen: $\forall \lambda \in \R, z \in \C$ und damit $\lambda \cdot z = e^{2\pi\ii \lambda}z$ (nicht transitiv). Ist \emph{nicht treu}, da 
		\begin{align*}
			\forall \lambda \in \R, \forall k \in \Z\colon \lambda \cdot z = (\lambda + k) \cdot z,\\
			\ker\theta = \Z \implies S^1 = \lnkset{\R}{\Z} \Circlearrowleft \C
		\end{align*}
	\end{enumerate}
\end{*example}
\subsection*{Stabilisator}
Sei $G \Circlearrowleft X$ und $x \in X$.
\begin{definition}
	Definiere $G_x := \set{g \in G \mid g \cdot x = x}$ als \begriff{Stabilisator} von $x$.
\end{definition}
\begin{lemma}
	Es gilt $G_x < G$.
\end{lemma}
\begin{proof}
	\begin{itemize}
		\item $1 \in G_x\colon 1 \cdot x = x$
		\item $\forall g,h \in G_x \implies h \in G_x$ und $(gh)x = g(hx) = gx = x$
		\item Sei $g \in G_x \implies g^{-1} \in G_x$: $gx = x \implies g^{-1}(gx) = g^{-1}x$ und damit $x = 1 x = (g^{-1}g)x$
	\end{itemize}
\end{proof}
... additions to previous example is missing, get from florian :(
\begin{lemma}
	$G \Circlearrowleft X$ und $\theta\colon G \to G_X$ assozierte Permutationsdarstellung. Dann
	\begin{align*}
	\ker \theta = \bigcup_{x \in X}G_x
	\end{align*}
\end{lemma}
\begin{proof}
	\begin{align*}
		g \in \ker \theta &\Leftrightarrow \theta(x) = \id_X\\
		&\Leftrightarrow g x = x \quad \forall x \in X\\
		&\Leftrightarrow g \in G_x \quad \forall x \in X
	\end{align*}
\end{proof}
\begin{definition}
	$G \Circlearrowleft X$ ist \begriff{treu} genau dann, wenn
	\begin{align*}
	\theta\colon G \to S_X \text{ injektiv}\\
	\forall g,h \in G \; (\forall x \in X \colon gx = hx) \implies g = h
	\end{align*}
\end{definition}
\subsection*{Transitive Operationen}
\begin{definition}
	\begin{align*}
		G \Circlearrowleft X \text{ \begriff{transitiv} } \Leftrightarrow \text{ gibt genau eine Bahn}\\
		&\Leftrightarrow x_0 \in X\quad X = G \cdot x_0\\
		&\Leftrightarrow \forall x,y \in X\colon \exists g \in G \colon y = gx
	\end{align*}
\end{definition}
\begin{*example}
	Betrachte 
	\begin{align*}
		O(n) = \set{A \in \Mat(n,\R) \mid A^T A = 1_n} = \set{A \in \Mat(n\R)\mid \norm{Ax} = \norm{x} \forall x \in \R^n}
	\end{align*}
	also $O(n) \Circlearrowleft S^{n-1} = \set{x \in \R^n \mid \norm{x} = 1}$ (Drehungen und Spiegelungen in der $S^2$ zum Beispiel) ist transitiv.
\end{*example}
\begin{lemma}
	$G \Circlearrowleft \lnkset{G}{H}$ transitiv
\end{lemma}
\begin{proof}
	$g,h \in G$, dann $gH = gh^{-1}\cdot h H$
\end{proof}
\begin{theorem}[Die Struktur von Gruppenoperationen]
	\begin{enumerate}
		\item $G \Circlearrowleft X \implies \exists H < G$ und ein $G$-Isomorphismus durch $(G,X) \cong (G, \lnkset{G}{H})$ ist \emph{transitiv}. ($H$ muss nicht eindeutig sein)
		\item $H,K < G$, dann
		\begin{align*}
			(G, \lnkset{G}{H}) \cong (G, \lnkset{G}{K}) \Leftrightarrow H \und K \text{ konjugiert} 
		\end{align*}
		also linke Seite $G$-Isomorph und bei der rechten Seite: $\exists g_0 \in G\colon H = g_0 Kg_0^{-1}$.
	\end{enumerate}
\end{theorem}
\begin{proof}
	\begin{enumerate}
		\item 
		\begin{itemize}
			\item $\Leftarrow$: $H < G \colon$ $G \Circlearrowleft \lnkset{G}{H}$ transitiv.
			\item $\Rightarrow$: Sei $G \Circlearrowleft X$ transitiv $x \in X$ beliebig und $X = Gx$. Definiere
			\begin{align*}
				H:= G_x \text{ und } \alpha: X \to \lnkset{G}{H} \mit gx \mapsto gH
			\end{align*} 
			\begin{itemize}
				\item Ist $\alpha$ wohldefiniert? Ja, da $\forall g,h \in G$ haben wir 
				\begin{align*}
				gx = hx &\Leftrightarrow h^{-1}g x = x\\
				&\Leftrightarrow h^{-1}g \in G_x = H \text{ (siehe HA1.1 gilt)}\\
				&\Leftrightarrow gH = hH.
				\end{align*}
				\item $\alpha$ injektiv? $\forall g,h \in G\colon \alpha(gx) = \alpha(hx)$. (Gehe die Wohldefiniertheit rückwärts).
				\item $\alpha$ surjektiv? $\forall g \in G \colon gH = \alpha(gx)$
				\item Betrachte
				\[
					\begin{tikzcd}
					X \arrow[r, "\alpha"] \arrow[d, "g"] & G/H \arrow[d, "g"] \\
					X \arrow[r, "\alpha"]                & G/H               
					\end{tikzcd}
				\]
				\begin{align*}
					\forall g\in G, y \in X g \alpha(y) \overset{?}&{=} \alpha(gy)\exists h \in G, y = hx\\
					\alpha(gy) = \alpha(g(hx)) = \alpha(ghx)\\
					&=ghH = ghH = g\alpha(hx)\\
					&= g \alpha(y)
				\end{align*}
				Also ist $\alpha$ $G$-Isomorphismus.
			\end{itemize}
			\item Wir müssen zuerst ein Lemma zeigen: 
			\begin{lemma}
				Sei $\alpha\colon X \to Y$ $G$-Isomorphismus, dann $\forall x \in X\colon G_x = G_{\alpha(x)}$
			\end{lemma}
			\begin{proof}
				$\alpha$ $G$-Isomorphismus gdw $\alpha$ bijektiv und $\forall x \in X\colon g\alpha(x) = \alpha(gx)$ und $\forall g \in G$
				\begin{align*}
					gx = x \Leftrightarrow g\alpha(x) = \alpha(gx) = \alpha(x)\\
					\implies g \in G_x \Leftrightarrow g \in G_{\alpha(x)}
				\end{align*}
			\end{proof}
			\begin{itemize}
				\item $\Rightarrow$: Sei $\alpha\colon \lnkset{G}{H} \to \lnkset{G}{K}$ ein $G$-Isomorphismus. Sei $g_0 \in G$, so dass $\alpha(H) = g_0 K$. Wir haben Stabilisator von $H$ in $G$
				\begin{align*}
					\set{g \in G \mid gH = H} = H
				\end{align*}
				und der Stabilisator von $gK$ in $G$
				\begin{align*}
					\set{g\in G \mid g \cdot g_0 K = g_0 K} = g_0 K g_0^{-1}
				\end{align*}
				dann haben wir
				\begin{align*}
					 g g_0 K = g_0 K &\Leftrightarrow g^{-1}_0 g g_0 K = K\\
					 &\Leftrightarrow g^{-1}g g_0 \in K\\
					 &\Leftrightarrow g \in g_0 K g^{-1}_0
				\end{align*}
				nun nutze das Lemma und es folgt $H = g_0 K g^{-1}_0$.
				\item $\Leftarrow$: Sei $g_0 \in G$ und nehme an $H = g_0 K g^{-1}$. Definiere $\alpha: \lnkset{G}{H} \to \lnkset{G}{K} \mit gH \mapsto gg_0 K$.
				\begin{itemize}
					\item $\alpha$ wohldefiniert: $\forall h, g \in G$
					\begin{align*}
						gH = gH &\Leftrightarrow h^{-1}g \in H = g_0 K g_0^{-1}\\
						&\Leftrightarrow g^{-1}_0 h^{.1} g g_0 \in K\\
						&\Leftrightarrow (hg_0)^{-1}g g_0 \in K\\
						&\Leftrightarrow gg_0 K = hg_0 K
					\end{align*}
				\item $\alpha$ injektiv
				\item $\alpha$ surjektiv $\forall g \in G\colon$ $gK = \alpha(g g_0^{-1}H)$.
				\item $\forall g,h \in G\colon h \alpha(gH) = h g g_0 K = \alpha(hgH)$.
				\end{itemize}
				Also haben wir, dass $\alpha$ $G$-Isomorphismus ist.
			\end{itemize}
		\end{itemize}
		\item 
	\end{enumerate}
\end{proof}
\begin{theorem}[Bahnen.Stabilisator-Satz]
	\begin{align*}
		G \Circlearrowleft X \implies \forall x \in X\colon \abs{Gx} = [G \colon G_x]
	\end{align*}
\end{theorem}
\begin{proof}
	Für alle $x \in X$ gilt $G \Circlearrowleft Gx$ transitiv
	\begin{align*}
		(G,G_x) \cong (G,\lnkset{G}{G_x}) \quad G\text{-Isomorph}\\
		\implies \abs{Gx} = \abs{\lnkset{G}{G_x}} = [G \colon G_x].
	\end{align*}
\end{proof}
Ausblick:\\
\begin{enumerate}
	\item $G \Circlearrowleft G$ durch Linksmultiplikation gegeben $\forall g,h \in G\colon g\cdot h = gh$
	\item $H< G$, $G \Circlearrowleft \lnkset{G}{H}$ durch Linksmultiplikation
	\item $G \Circlearrowleft G$ durch Konjugation $\forall g,x \in G \colon x^g := gxg^{-1}$
	\item $G \Circlearrowleft \powerset(G)$ durch Konjugation
\end{enumerate}
\begin{theorem}
	As $G$-Space, we have $G_x \cong \lnkset{G}{G_x}$.
\end{theorem}
\begin{*example}
	\begin{itemize}
		\item $G = \SO(3)$ Rotation im $\R^3$, dann $A \in \SO(3) \ni \Mat(3,\R)$, $AA^T = 1$, $\det A = 1$, dann kann man sich das Skalarprodukt anschaun % siehe Bild
		\begin{itemize}
			\item (Wirkung durch Multiplikation) und nehme $x = (0 0 1)^T, G_x = S^2$ (Bahn, Orbit dieses Punkte ist $S^2$), Stabilisator $G_x = \SO(2)$
			\item Allgemeiner: eingebettet in $\SO(3)$ durch
			\begin{align*}
			\gamma\colon \SO(2) \to \SO(3) \mit B \mapsto \begin{pmatrix}
			B & 0\\
			0 & 1
			\end{pmatrix}			 
			\end{align*}
			ist Gruppenhomo. (also $S^2 \cong \lnkset{\SO(3)}{\SO(2)}$ und Erlangen Programm ...)
		\end{itemize}
		\item \person{Cayley}s Satz $X = G$, $\alpha$ ist Gruppenstruktur. Hier gilt:
		\begin{align*}
			G = \set{1}\forall x \in X = G \mit gx = x \implies g = 1
		\end{align*}
		durch Multiplikation von rechts mit $x^{-1}$. (Spezielle Situation, i.A. gibts \emph{kein} $xy$ für $x,y \in X$!) Insbesondere ist die Wirkung treu, also $\rho$ injektiv und wir erhalten
		\begin{align*}
			G \cong \rho(G) < S_G
		\end{align*}
		\item $G$ Gruppe, $H < G$, $X = \lnkset{G}{H}$ mit Wirkung $g(hH) := (gh)H, g,h \in G$, transitiv, da $x = hH, y= tH$ ($x,y \in X$ beliebig), dann wähle $g:= t H^{-1}$ meine Wahl ($\exists g \in G$ $t h^{-1}hH = gx = y = tH$)
		\begin{align*}
			G_x = \set{g \in G \mid gx = x} \und x = hH \quad gx = x \Leftrightarrow ghH = hH \implies h^{-1}gh \in H \implies g \in hHH^{-1} \in H\\
			\text{Insbesondere }\ker \rho = \bigcap_{x \in X} G_x = \bigcap_{H \in H} hHh^{-1} \lhd G 
		\end{align*}
		Dies ist $= H \Leftrightarrow H \lhd G$
		\item Adjungierte Wirkung: $X = G$ (wie oben) \emph{aber}
		\begin{align*}
			g \rhd x\cdot gxg^{-1}
		\end{align*}
		($\rhd$ neues Symbol zum unterscheiden). $G_x = $ Konjugationsklassen von $x \in X = C(x)$ und
		\begin{align*}
			G_x = \set{g \in G\colon gxg^{-1} = x \Leftrightarrow gx = xg} = Z(x) \text{ Zentralisator}
		\end{align*}
		Nutze Orbit-Stabiliser-Theorem:
		\begin{align*}
			\lnkset{G}{Z(x)} \cong C(x) \und \abs{C(x)} = \frac{\abs{G}}{\abs{Z(x)}}
		\end{align*}
		Die Konjugationsklassen mit nur einem Element bilden das Zentrum von $G$. Also gilt 
		\begin{align*}
			G = Z(G) \cup C(x_1) \cup \dots C(x_d)
			\intertext{für geeignete $x_1, \dots, x_d \in G$}
			\abs{G} = \abs{Z(G)} + \sum_{i=1}^d \frac{\abs{G}}{\abs{Z(x_i)}} \tag{class-equation}\label{eq_1_6_1_class_eq}
		\end{align*}
		(wobei die Vereinigung disjunkt sind) % find disjoint union symbol :S MINT?!
	\end{itemize}
\end{*example}
\begin{proposition}
	Eine endliche $p$-Gruppe hat nichttriviales Zentrum.
\end{proposition}
\begin{proof}
	$\abs{G} = p^n$ mit $p=$prim und dann
	\begin{align*}
		p^n &= \abs{Z(G)} + \sum_{i=1}^d \frac{\abs{G}}{\abs{Z(x_i)}}\\
		&= \abs{Z(G)} + p^{n_1} + \dots + p^{n_d} \quad n > 0
	\end{align*}
	Also ist $\abs{Z(G)} \ge 1$ denn $1 \in Z(G)$ und $\abs{Z(G)}$ ist teilbar durch $p$ und damit ist $\abs{Z(G)} \ge p$
\end{proof}
Insbesondere ist $G$ nicht einfach! ($Z(G) \lhd G$)
\section{Die \person{Sylow}-Sätze}
Sei $G$ eine endliche Gruppe.
\begin{definition}
	Eine Untergruppe von $G$ ist eine maximale $p$-Untergruppe (für eine Primzahl $p$), d.h. $H < G$ und $\exists p$ prim, $n \in \N\colon \abs{H} = p^{n+1} \nmid G$ (teil nicht).
\end{definition}
Also $\abs{G} = p^n \cdot m$ und $p \nmid m$.
\begin{theorem}[alle Sylow-Sätze]
	Mit der wie oben gilt:
	\begin{enumerate}
		\item Die Zahl $r$ der Sylowschen Untergruppen von $G$ ist 1 modulo $p$ ($\exists s\colon r = 1+sp$). Insbesondere ist $r \neq 0$!
		\item Jede $p$-Untergruppe von $G$ ist in einer Sylowschen enthalten
		\item Alle Sylowschen Untegruppen sind konjugiert zueinander. Insbesondere gilt: $r \mid m$. Also
		\begin{align*}
			H < G\quad \abs{H} = p^n \quad N_G = N := \set{g\in G \mid gHg^{-1} = H} \quad \abs{\lnkset{G}{N}} = r = \frac{p^n \cdot m}{p^n \cdot \lnkset{N}{H}}
		\end{align*}
		da $m = r \cdot \abs{\lnkset{N}{H}}$ und $\abs{N} = \abs{\lnkset{N}{H}} \cdot p^n$
	\end{enumerate}
\end{theorem}
\begin{proof}
	\begin{enumerate}
		\item Sei $X := \set{P \subseteq G \mid \abs{P} = p^n} \ni \set{x_1, \dots, x_{p^n} \mid x_j \in G}$. $G$ wirkt auf $X$ durch Multiplikation von links.
		\begin{align*}
		g \set{x_1, \dots, x_{p^n}} = \set{gx_1, \dots, gx_{p^n}}
		\end{align*}
		Gesucht ist $H \subset X$ mit $H < G$. Für ein solches $H$ ist dann $\lnkset{G}{H} \subseteq X$ eine Bahn der $G$-Wirkung mit der Länge $m = \abs{\lnkset{G}{H}} = \frac{\abs{G}}{\abs{H}} = \frac{p^n m}{p^n}$\\
		\emph{Behauptung 1:} Alle Bahnen in $X$ der Länge $m$ sind von dieser Form.
		\begin{proof}[Beweis der Behauptung]
			Sei $GP \subseteq X$ eine Bahn mit $\abs{GP} = m$. WLOG ist $1 \in P$ (wenn nicht, wähle $x \in P$ beliebig und ersetze $P$ durch $x^{-1}P$)
			\begin{align*}
			G_P = \set{g \in G \mid gP = P}\quad P = \set{1, x_2, \dots, x_{p^n}}\\
			\und gP = \set{g1, gx_2, \dots, gx_{p^n}}
			\end{align*}
			Also $1 \in P \implies G_P \subset P$ und daraus $\abs{G_P} \le \abs{P} = p^n$. Aber wir wissen auch:
			\begin{align*}
			m = \abs{GP} = \abs{\lnkset{G}{G_P}} = \frac{\abs{G}}{\abs{G_P}} = \frac{p^n m}{G_P} \implies \abs{G_P} = p^n.\tag{$\ast$}\label{eq_1_7_sylow_1}
			\end{align*}
			Also ist $G_P = P$. Damit folgt die Behauptung $G_p < G$.\\
			\emph{Behauptung \eqref{eq_1_7_sylow_1}:} Alle Bahnen in $X$, deren Länge nicht durch $p$ teilbar ist, sind von der Form $\lnkset{G}{H}$
		\end{proof}
		Damit folgt 1.\\
		\emph{Behauptung 2:} Die einzige Nebenklasse $gH$, die eine Untergruppe von $G$ ist, ist $H$ selbst ($H$ wie oben Sylowsche Untergruppe)
		\begin{proof}
			$gH$ Untergruppe $\implies 1 \in gH \implies g^{-1} \in H$ und daraus $g \in H$ und schließlich $gH = H$
		\end{proof}
		Also existiert eine Bijektion zwischen den Sylowschen $p$-Untergruppen $H < G$ und den Bahnen $GP \subseteq X$ mit $\abs{GP}$ nicht durch $p$ teilbar.\\
		\emph{Behauptung:} Ist $[n]\in \Z_p$ die Klasse von $n \in \Z$ in $\Z_p$, so gilt
		\begin{align*}
		[r] = \frac{1}{[m]}\binom{\abs{G}}{p^n}
		\end{align*}
		\begin{*remark}
			$\abs{X} = \binom{\abs{G}}{p^n}$ per Definition von $X$ als $\set{P \subseteq G \mid \abs{P} = p^n}$
		\end{*remark}
		Nach Behauptung 1 und Behauptung 2 gilt $rm = \binom{\abs{G}}{p^n}$ modulo $p$, denn $X$ zerfällt in Bahnen, da $G$-Wirkung und die Bahnen deren Länge nicht durch $p$ teilbar sind. Die Bahnen sind von der Form $\lnkset{G}{H}$ für eine eindeutige, bestimmte Sylow $p$-Untergruppe $H$, haben die Länge $m$ und es gibt $r$ davon. Damit ist der erste Sylow-Satz bewiesen. \\
		Yeah its scrambled again :/
		\begin{*remark}
			Only $\abs{G}$ enters here, we can compute $r$ modulo $p$ using any group $G$ of the size $\abs{G}$, z.B. können wir $G = \Z_{\abs{G}}$, und in $\Z_{p^{i} m}$ gibt es genau eine Sylow $p$-Untergruppe, nämlich $\langle [m]\rangle = \set{[e], [m], [2m], \dots, [(p^{i} - 1)m]}$ (Erinnerung 1. Semester: $H \le \Z \Leftrightarrow H = a\Z \dots $) $\implies$ 1.
		\end{*remark}
	\item Betrachte die Konjugationswirkung von $G$ auf $X$
	\begin{align*}
		\Ad(g)P := \set{gx^{-1}g, \dots, gx_{p^{i}}g^{-1}}
	\end{align*}
	\emph{Behauptung 3:} Ist $P \le G$ Sylow $p$-Untergruppe und $Q \le P$ eine $p$-Untergruppe, so existiert $R = gPg^{-1}$ mit $aRa^{-1}\quad \forall a \in G$ ($N_G R \ge Q$)
	\begin{proof}
		Betrachte den Orbit von $P$ unter der adjungierten Wirkung
		\begin{align*}
			\Ad(G)P := \set{gPg^{-1} \mid g \in G} \cong \lnkset{G}{N_G P} \quad \text{ orbit stabilizer}\\
			(g \in N_G P \Leftrightarrow gPg^{-1} = P)
		\end{align*}
		Dann gilt noch 
		\begin{align*}
			p^i n = \abs{N_G P} \implies \abs{\Ad(G)P} = \frac{m}{n}\tag{$\ast \ast$}\label{eq_sylow_proof_2}
		\end{align*}
		\emph{nicht} durch $p$ teilbar ($P \le N_G P$). Jetzt wirken nur mit $Q$ durch Konjugation auf diesen einen Orbit $\Ad(G)P$. $Q \le G$, der eine Orbit zerfällt gegebenenfalls in mehrere. Jeder Teil ist von der Form
		\begin{align*}
			\set{aRa^{-1} \mid a \in Q} \quad \text{ für ein $R$ von der Form} \quad R = gPg^{-1}\tag{$\ast\ast\ast$}\label{eq_sylow_proof_3}
		\end{align*}
		für ein $g \in G$
		\begin{align*}
			\Ad(G)R \cong \lnkset{Q}{Q} \cap N_G R
		\end{align*}
		$\cong$ folgt aus orbit-stabilizer wieder und $N_G R$ sind die $a \in Q$ mit $a R a^{-1} = R$
		\begin{align*}
			\abs{\Ad(Q)R} = \frac{\abs{Q}}{\abs{Q \cap N_G R}} = p^s
		\end{align*}
		für ein $s$, denn $Q$ war $p$-Untergruppe nach Annahme. Wegen \eqref{eq_sylow_proof_2} muss ein $P$ existieren mit $s = 0$, d.h. $Q \subseteq N_G R$
	\end{proof}
	\emph{Behauptung 5:} $R$ wie im Behauptung 4 ist eine Sylowsche Untergruppe und $Q \le R$
	\begin{proof}
		$R$ ist Bild von $P$ unter dem (inneren) Automorphismus $x \mapsto gxg^{-1}$ mit $g$ aus \eqref{eq_sylow_proof_3} also ist $R$ eine Sylowuntergruppe, da $Q \le N_G R$ ist
		\begin{align*}
			\set{ab \in G \mid a \in Q,b \in R} = QR \le G \text{ eine Untergruppe}\\
			a,a' \in Q,b,b' \in R\; ab \cdot a'b' = ... \text{ need to add still!}
		\end{align*}
		Dann $\abs{QR} = \lnkset{\abs{Q}\abs{R}}{\abs{Q \cap R}} = \frac{p^j p^i}{Q \cap R}$, d.h. $QR$ ist $p$-Untergruppe wegen $R \le QR$ und $R$ Sylow, also maximal, so folgt $QR = R \implies Q \le R$ und das gibt uns den zweiten Sylow Satz.
	\end{proof}
	\item Ist der Spezialfall, in dem $Q$ selbst eine Sylowuntergruppe war! Dann ist $Q = R$ im obigen Beweis, aber $R = gPg^{-1}$. 
 	\end{enumerate}
\end{proof}
\subsection*{Anwendungen}
\begin{enumerate}
	\item Cauchy's Satz: $G$ endlich Gruppe $p \mid \abs{G} \implies \exists g \in G$ mit $\ord(g) = p$ (p ist prim!)
	\begin{proof}
		Sei $P \le G$ eine Sylow $p$-Untergruppe, $\abs{P} = p^i$. Für $x \in P$ gilt $\ord(x) \mid p^{i}$, d.h. $\ord(x) = p^s$, also hat $x^{p^{s-1}}$ die Ordnung $p$
	\end{proof}
	\item $\abs{G} = pg$, $p\neq q$, $p,q$ prim $\implies G$ \emph{nicht} einfach.
	\begin{proof}
		Sei $q < p$. Wieviele Sylow $p$-Untergruppen gibt es? Ist $r$ diese Zahl, so ist $r=1 \mod p$. jede davon hat $p$ Elemente, d.h. ist isomorphic zu $\Z_p$. Sind $P_1, P_2$ zwei solche, so folgt $P_1 \cap P_2 = \set{1}$ (denn dies ist die einzige Untergruppe in $\Z_p$). Gäbe es mehr als eine Sylow $p$-Untergruppe, also mindestens $p+1$ Stück, wären deren Vereinigung eine Menge mit 1 $\in \set{1} + (p+1)(p-1) = p^2$ Elementen und das ist ein Widerspruch also $\abs{G} = pq < p^2$.
	\end{proof}
\end{enumerate}
\begin{proposition}
	Annahme wie oben, aber $q \nmid p-1$. Dann ist $G \cong \Z_{pq}$.
\end{proposition}
\begin{proof}
	Denke zusätzlich über Sylow $q$-Gruppen nach. Deren Zahl $s$ ist 1 modulo $q$. Ausserdem teilt $\gamma, p\cdot q$, d.h. $\gamma = 1, \gamma = p, \gamma = q$ oder $s = pq$, damit die Sylowgruppen alle konjugiert sind zueinander sind. D.h. die Menge $S_q$ der Sylow $p$-Untergruppe von $G$ ist Orbit (eine Bahn) in $X = \set{P \subseteq G \mid \abs{p} = q}$ unter Konjugation. Nach dem Stabilisator-Bahnen-Satz ist die Menge der Sylow $q$-Gruppen in einer Gruppe $G$ also im Bijektion mit $\lnkset{G}{N_G H}$, wobei $H \le G$ eine beliebige Sylow $q$-Untergruppe ist und $N_G H = \set{x \in G \mid x H x^{-1} = H}$ der Normalisator (Stabilisator von $H$ unter der adjungierten Wirkung) von $H$ in $G$ ist. D.h. $\lnkset{G}{N_G H}$ ist die Menge der zu $H$ konjugierten Untergruppen.
	\begin{align*}
		s = \abs{\lnkset{G}{N_G H}} = \frac{\abs{G}}{\abs{N_G H}}
	\end{align*}
	ist also die Zahl der zu $H$ konjugierten Untergruppen. Insbesondere ist $H \le N_G H$, also $\abs{H}\abs{N_G H}$ und 
	\begin{align*}
		s = \frac{\lnkset{G}{H}}{\lnkset{\abs{N_G H}}{\abs{H}}}
	\end{align*}
	Wir haben in den Sylowsätzen 3. hinzugefügt, dass $\gamma \mid m$. Im unserer speziellen Situation heisst dies: $s$ teilt $p$
	\begin{align*}
		\abs{G} = pq \quad s = \#\text{ Sylow-}p\text{-Untergruppen}
	\end{align*}
	$H \le G$ beliebige solche Untegruppe, d.h. $\abs{H} = q$, d.h. $H \cong \Z_q$ und $H \le N_G H$
	\begin{align*}
		s = \frac{\lnkset{\abs{G}}{\abs{H}}}{\lnkset{\abs{N_G H}}{\abs{H}}} = \frac{(pq)/q}{n/q} = \frac{p}{n/q}
	\end{align*}
	(n könnte $pq$ ($s=1$) oder $q$ ($s=p$ sein). $s = \#$ Sylow $q$-Untergruppen in $G$ mit $\abs{G} = pq$ mit $p>q$. $S \mid p$ (nach Sylow 3.) folgt dann $s =1 \oder s =p$. $s = 1 \mod q$ (sylow 1.) folgt
	\begin{align*}
		s=1,s=q+1, s= 2q+1, \dots \tag{$\ast$}\label{eq_sylow_1_not proof}
	\end{align*}
	Wenn wir also noch annehmen $q \nmid p -1$, so fällt $s = p$ als Fall weg, wegen \eqref{eq_sylow_1_not proof}. Also ist $s =1$ und auch die Sylow $q$-Untergruppe ist eindeutig und somit normal.
	\begin{align*}
		\Z_p \cong \le G \ge Q \cong \Z_q
	\end{align*}
	Wir haben $H \cap Q = \set{1}$, also gilt $G \cong H \times Q \cong \Z_p \times \Z_q \cong \Z_{pq}$ $G \cong \lnkset{R}{I_1 \cap ... \cap I_d} \cong \bigtimes_{i=1}^d\lnkset{R}{I_i}$.
\end{proof}
\begin{proposition}
	Sei $H$ $p$-Sylow Untergruppe von $G$, $H$ eindeutig, dann folgt damit $H$ normal.
\end{proposition}
\begin{proof}
		Sei
	\begin{align*}
	\alpha \colon G \to G \mit y \mapsto y x y^{-1}
	\end{align*}
	Automorphismus und $\alpha(H) \le G$, $\abs{\alpha(H)} = \abs{H}$, dann folgt $\alpha(H) = H$
\end{proof}
\begin{proposition}
	Ist $n = pq, p>q$ und $q \mid p-1$, so gibt es bis auf Isomorphie genau zwei Gruppen $G$ der Ordnung $n$, die zyklische Gruppe $\Z_{pq}$ und eine nichtabelsche Gruppe, die ein semidirektes Produkt $\Z_p \rtimes\Z_q$ ist.
\end{proposition}
\begin{proof}
	Wie oben gibt es eine eindeutige Sylow $p$-Untergruppe $H \cong \Z_p$. Wir haben oben gesehen, dass es entweder $s=1$ oder $s = p$ Sylow $q$-Untergruppe gibt. Im Fall $s=1$, gibt es eine eindeutige Sylow $q$-Untergruppe $Q = \Z_q$ und $G = \Z_p \times \Z_q$ (wie oben). Im Fall $s=p$ gibt es Sylow $q$-Untergruppen $Q_1, \dots, Q_p$ und $Q_i \cong \Z_q \forall i \in [1,p]$ aber keine ist normal.\\
	$Q_1 \cong \Z_q$ wirkt auf $H \cong \Z_p$ durch Konjugation und wie oben ist $Q_1 \cap H = \set{1}$. Also ist $HQ \le G$ Untegruppe mit $pq$ Elementen, also $HQ = G$ und dies ist ein internes semidirektes Produkt, $H \rtimes Q_1 = G$
	\begin{align*}
		G \cong \Z_p \times \Z_q = \Z_{pq}
	\end{align*}
	Noch zu zeigen: 1. dieser Fall tritt auf, egal welche $p,q$ wir betrachten.\\
	2. und zwar auf eindeutige Weise bis auf Isomorphie.
\end{proof}
\begin{proposition}[\person{Burnside}]
	Sei $\abs{G} = p^i q^j$ ($p,q$ Primzahlen). Dann $G$ auflösbar (Auf jeden Fall nicht einfach!!!)
\end{proposition}
\begin{proof}
	to be done ...
\end{proof}
\begin{definition}
	Ist $G$ eine Gruppe und sind $(a,b) \in G^2$, so ist deren \begriff{Kommutator}
	\begin{align*}
		[a,b] = aba^{-1}b^{-1}
	\end{align*}
	und die derivierte Gruppe $G'$ also die von allem Kommutatoren erzeugte Untergruppe $G' \lhd G$
	\begin{align*}
		G \rhd G' \rhd G'' \rhd ...
	\end{align*}
	$G$ auflösbar wenn $n$ mit $G^{(n)} = \set{1}$
\end{definition}