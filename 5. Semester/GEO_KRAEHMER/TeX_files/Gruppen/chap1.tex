\section{Wiederholung}
\begin{definition}[Halbgruppe, Monoid, Gruppe]
	Eine \begriff{Halbgruppe} ist eine Menge $G$ mit einem \emph{assoziativen} Produkt
	\begin{align*}
		\cdot\colon G \times G \to G.
	\end{align*}
	Ein \begriff{Monoid} ist eine Halbgruppe, in der ein Element $1 \in G$ existiert mit 
	\begin{align*}
		1\cdot x = x\cdot 1 = x \quad\forall x \in G.
	\end{align*}
	Eine \begriff{Gruppe} ist ein Monoid, in dem für jedes $x \in G$ ein $y \in G$ existiert mit
	\begin{align*}
		xy = yx = 1.
	\end{align*}
\end{definition}

\begin{remark}
		1 ist eindeutig, wenn es existiert. $y$ ist durch $x$ eindeutig bestimmt: $x^{-1} = y$.
\end{remark}

\begin{definition}[Morphismus]
	Ein Morphismus zwischen zwei Gruppen $G$ und $H$ ist eine Abbildung
	\begin{align*}
		f\colon G \to H \quad\text{mit}\quad f(xy) = f(x)f(y) \quad\forall x,y \in G.
	\end{align*}
\end{definition}

\begin{proposition}
	Ist $f\colon G \to H$ ein Morphismus von Gruppen, so gilt\begin{itemize}
		\item $f(1) = 1$ und
		\item $f(x^{-1}) = f(x)^{-1}$ $\forall x\in G$.
	\end{itemize}
\end{proposition}
\begin{proof}
	Für alle $x \in G$ gilt \begin{align*}
		f(x) &= f(1\cdot x) = f(1)f(x).
	\end{align*}
	Gilt in einer beliebigen Gruppe jedoch $ab=b$ für zwei Elemente $a,b$, so folgt
	\begin{align*}
		(ab)\cdot b^{-1} = a\big(bb^{-1}\big) = a\cdot 1 = a \quad\text{mit}\quad bb^{-1} = 1.
	\end{align*}
	Ferner gilt \begin{align*}
		f(x)\cdot f\big(x^{-1}\big) = f\big(x \cdot x^{-1}\big) = f(1) = 1
	\end{align*}
	wie schon gezeigt (und analog $f(x^{-1})f(x) = 1$). Also ist $f(x^{-1}) = f(x)^{-1}$).
\end{proof}

\begin{example}
	\begin{enumerate}[label={\arabic*)}]
		\item Sei $X$ eine beliebige Menge. $S_X = \set{f\colon X \to X \mid f \text{ bijektiv}}$ ist eine Gruppe bezüglich Komposition mit $1 = \id_X$. Insbesondere ist $S_n = S_{\set{1,\dots,n}}$ die \begriff{symmetrische Gruppe} und ein Element $f \in S_n$ ist eine \begriff{Permutation}.
		\item $\GL(V) = \set{f \in S_V \mid f \text{ linear}}$, wobei $V$ ein $R$-Modul ist mit kommutativen assoziativen Ring mit 1.
		\item $\Z, \Z_n$ unter Addition 
		\begin{align*}
			U_n = \Z^{\times}_n = \set{ m \in \set{0,\dots,n-1} \mid \ggT(m,n) = 1}
		\end{align*}
		Beide Gruppen sind abelsch, d.h. \begin{align*}
			\forall x,y \in G: xy = yx.
		\end{align*}
		\item $G = U(1) = \set{z \in C \mid \abs{z} = 1} = \set{e^{it}\mid t \in [0,2\pi]}$
		\item $G = U(1) \times \SU(2)\times \SU(3)$, die Eichgruppe im Standardmodell der Elementarteilchen  
	\end{enumerate}
\end{example}

\begin{definition}[Ordnung]
	Ist $G$ endlich, so nennt man $\abs{G}$ die \begriff{Ordnung} von $G$.
\end{definition}

\begin{example}
	$\abs{S_n} = n!$
\end{example}

\begin{definition}[$p$-Gruppe]
	Ist $\abs{G} = p^n$ für eine Primzahl $p$ und ein $n\in\mathbb N$, so nennt man $G$ eine \begriff{$p$-Gruppe}
\end{definition}

\begin{definition}[Untergruppe]
	Sei $G$ Gruppe. Eine Teilmenge $H \le G$ ist eine \begriff{Untergruppe} $H < G$, wenn
	\begin{defenum}
		\item \label{1_1_9_i} Für alle $x,y \in H$: $xy \in H$
		\item \label{1_1_9_ii} $1 \in H$
		\item \label{1_1_9_iii} Für alle $x \in H$: $x^{-1} \in H$ 
	\end{defenum}
\end{definition}

\begin{proposition}
	Ist $\abs{G} < \infty$, so folgen \cref{1_1_9_ii,1_1_9_iii} bereits aus \cref{1_1_9_i} und $H \neq \emptyset$.
\end{proposition}
\begin{proof}
	Sei $x \in H$ ein beliebiges Element. Aus \ref{1_1_9_i} folgt $x^n \in H$ für alle $n \in \N_{+}$. Da $\abs{G} < \infty$ existiert $n\neq m$ mit $x^n = x^m$. O.E. sei $n > m$ \begin{itemize}[label={$\Rightarrow$},topsep=-\parskip]
		 \item $x^{n-m} x^m = x^n$
		 \item $x^{n-m} = 1$
		 \item \ref{1_1_9_ii}
	\end{itemize}
	Ferner impliziert die Existenz der inversen Elemente, dass die Linkstranslation \begin{align*}
		t_x\colon G \to G, y \mapsto xy\quad(x\in G\;\text{fest})
	\end{align*}
	injektiv ist, denn $(t_x)^{-1} = t_{x^{-1}}$. Ist $x \in H$, so heißt \ref{1_1_9_i} gerade $t_x(H) \subseteq H$, sprich $t_x$ kann zu $t_x\big|_H\colon H \to H$ eingeschränkt werden. Die Einschränkung einer injektiven Abbildung ist injektiv. Da $\abs{H} \le \abs{G} < \infty$, folgt $t_x\big|_H\colon H \to H$ ist surjektiv. Also existiert $y \in H$ mit $t_x(y)= 1$. Eindeutigkeit von $x^{-1}$ heißt $y = x^{-1} \in H$.
\end{proof}

\begin{definition}[Erzeugendensystem]
	Ist $X \subseteq G$, so ist
	\begin{align*}
		\langle X \rangle = \bigcap_{\substack{H < G\\X \subset H}} H
	\end{align*}
	die von $X$ erzeugte Untergruppe. Ist $\langle X \rangle = G$, nennen wir $X$ ein Erzeugendensystem.
\end{definition}

\begin{definition}[Konjugation]
	Ist $H < G$ und $x \in G$, so ist\begin{align*}
		x^{-1}Hx = \set{x^{-1}Hx\mid y \in H}
	\end{align*}
	eine Untergruppe ("`$x^{-1}yx$"' $y$ ist konjugiert mit $x$). Wir nennen diese zu $H$ konjugiert.
%	\begin{align*}
%		(x^{-1}yx)^{-1} = x^{-1}y^{-1}x \und x^{-1}yx\cdot x^{-1}zx = x^{-1}yzx
%	\end{align*}
\end{definition}

\begin{definition}[Konjugationsklasse]
	Die Menge $\set{x^{-1}yx \mid x \in G}$ ist i.A. \emph{keine} Untergruppe und diese nennt man \begriff{Konjugationsklassen} von $y$.
\end{definition}

\begin{definition}[Zentralisator, Zentrum]
	Der \begriff{Zentralisator} von $y \in G$ ist
	\begin{align*}
		\set{x \in G \mid xy = yx} =: Z_G(y).
	\end{align*}
	Das \begriff{Zentrum} von G ist
	\begin{align*}
		Z(G) = \bigcap_{y \in G} Z_G(y) = \set{x \in G \mid \forall y \in G xy=yx}.
	\end{align*}
\end{definition}
\begin{example}
	Sei $G = S_n \ni f$ Permutation, z.B.
	\begin{align*} S_6 \in
		\begin{pmatrix}
		1 & 2 &3 &4 & 5 & 6\\
		5 & 4 & 6 & 1 & 2 & 3
		\end{pmatrix}  = (1524)(36)
	\end{align*}
	letzteres nennt man \begriff{Zykelnotation}. 1-Zykeln, d.h. $i \in \set{1,\dots,n}$ mit $f(i)=i$ werden meist nicht notiert, z.B.:
	\begin{align*} S_4 \in 
		\begin{pmatrix}
			1&2&3&4\\
			2&1&3&4
		\end{pmatrix} = (12)
	\end{align*}
\end{example}
\begin{remark}
	Ein $k$-Zykel ist ein Produkt von $k-1$ Transpositionen (2-Zykel), z.B.
	\begin{align*}
		(12345) = (15)(14)(13)(12)
	\end{align*}
	ist das Produkt in $S_5$, d.h. Komposition! Also erzeugt $\set{(i,j)}$ die $S_n$. Jede Permutation kann also als Produkt von Transpositionen geschrieben werden. Diese Darstellung ist nicht eindeutig! (z.B. $(12)(23)(12) = (23)(12)(23)$) (``Braid relation'') % add braid picture?! braids package
	und $(12)(12) = (\,)$. Allerdings kommen in jeder solcher Darstellungen entweder eine gerade oder ungerade Anzahl von Transpositionen vor ($\to$ Fehlstände). Insbesondere bilden gerade Permutationen (gerade Anzahl an Fehlständen $\Leftrightarrow$ Produkte von zu Transpositionen) eine Untergruppe $A_n < S_n$, die sogenannte \begriff{alternierende Gruppe}.
\end{remark}
%Sei $G$ eine endliche Gruppe.
\begin{example}
	Sei $G = \GL(n,R)$ die invertierbare Matrizen mit Einträgen in $R$ (nur endliche, wenn $R^{\times} endlich$!). Untergruppen sind
	\begin{itemize}
		\item $\SL(n,R) = \set{g \in \GL(n,R) \mid \det g = 1}$
		\item $O(n,R) = \set{g \in G \mid gg^T = g^T g = 1}$ mit dem Skalarprodukt $\scaProd{gv}{gw} = \scaProd{v}{w} \quad \forall v,w \in R^n$
		\item $\SO(n,R) = \mathrm{SL}(n, R) \cap \mathrm O(n,R)$.
	\end{itemize}
	Ist $R$ Ring mit Involuten (z.B. $R = \C, z = \bar{z}$)
	\begin{itemize}
		\item $U(n,R) = \set{g \in \GL(n,R) \mid gg^{*} = g^{*}g = 1}$
		\item $\SU(n,R) = \SL(n,R) \cap U(n,R)$
	\end{itemize}
\end{example}

\begin{example}
	Sei $D_n$ definiert durch\begin{align*}
		D_n &= \set{f\colon \R^2 \to \R^2 \; \text{linear, bjektiv} \mid f(P_n) = P_n},
	\end{align*}
	wobei $P_n \subset \R^2$ das regulär $n$-gen ist, z.B. das Hexagon $P_6$. % sketch? 
	Alternativ ist $D_n \subseteq S_n$, wobei $\set{1, \dots,n}$ mit der Menge der Ecken von $P_n$ identifiziert wird und man erhält alle Permutationen, die benachbarte Ecken auf benachbarte abbilden:
	\begin{itemize}
		\item $r$: Rotation um $\sfrac{2\pi}{n}$ im mathematische positiven Sinn
		\item $s$: eine beliebige Spiegelung in $D_n$
	\end{itemize}
	Also hat man
	\begin{align*}
		\langle \set{s,r}\rangle = D_n = \set{s^i r^j \mid i = 0,1, j=0, \dots, n-1}
	\end{align*}
	und der Mächtigkeit $\vert D_n\vert = 2n$.
	
	Für die erzeugenden Elemente $D_n = \langle \lbrace s,r\rbrace \rangle$ gilt:
	\begin{itemize}
		\item $srs = r^{n-1}$,
		\item $r^{n-1} = r^{-1}$,
		\item $r^n = 1$,
		\item $s^2 = 1$.
	\end{itemize}
Im unendlichen Fall $D_\infty \subset S_{\mathbb Z}$ gilt z.B. $r(z) = z+1$, $s(z) = -z$, wobei $r$, $s\colon \mathbb Z\to\mathbb Z$ sind und $D_\infty$ erzeugen: $D_\infty = \langle \lbrace r,s\rbrace\rangle$.
\end{example}

\section{Nebenklassen, Normalteiler, Isomorphiesätze}
\begin{definition}
	$A,B \subseteq G$ Teilmengen (nicht unbedingt Untergruppen!), dann:
	\begin{itemize}
		\item $AB = \set{xy \in G \mid x \in A, y \in B}$
		\item $A^{-1} = \set{x^{-1} \in G \mid x \in A}$
	\end{itemize}
\end{definition}

\begin{remark}
	$\emptyset \neq H \subseteq G$ ist Untergruppe $\Leftrightarrow$ $HH = H$, $H^{-1} = H$
\end{remark}

\begin{definition}
	Ist $x \in G$, so nennen wir
	\begin{align*}
		f_x\colon G \to G \mit y \mapsto x^{-1}yx
	\end{align*}
	den durch $x$ definierten inneren Automorphismus. Ist $H < G$, so nennen wir $f(H) = x^{-1}Hx$ eine zu $H$ konjugierte Untergruppe.
\end{definition}

\begin{proposition}
	\begin{itemize}
		\item $f_x$ ist ein Endomorphismus von $G$ (d.h. ein Morphismus $G \to G$)
		\item Das Bild $\Image f$ eines beliebigen Gruppenmorphismus $f\colon K \to L$ ist eine Untergruppe: $\Image f < L$
	\end{itemize}
\end{proposition}
\begin{proof}\leavevmode
	\begin{itemize}[topsep=-6pt]
		\item $f_x (yz) = x^{-1}yzx = x^{-1}y(xx{^-1})zx = (x^{-1}yx)(x^{-1}zx) = f_x(y)f_x(z)$ $\forall y,z \in G$
		\item Wir untersuchen die drei Eigenschaften:
		\begin{itemize}
			\item $\Image f$ ist abgeschlossen: seien $f(y), f(z) \in \Image f$. Dann gilt:
			\begin{align*}
			f(y)f(z) = f(yz) \in \Image f
			\end{align*}
			\item $f(1) = 1 \implies 1 \in \Image f$
			\item $f(x)^{-1} = f(x^{-1}) \implies (\Image f)^{-1} = \Image f$
		\end{itemize}
	\end{itemize}
\end{proof}

\begin{definition}
	Ist $H < G, x \in G$, so nennt man
	\begin{align*}
	G \supseteq x H &= \set{x}H = \set{xy \in G \mid y \in H} \quad \text{linke Nebenklasse}\\
	G \supseteq H x &= \set{yx \in G \mid y \in H} \quad \text{rechte Nebenklasse}
	\end{align*}
\end{definition}
\begin{example}
	Sei $G = V$ Vektorraum über Körper $K$ mit $+$ als Gruppenstruktur, dann ist $H = W < V$ ein Untervektorraum und $xH = x + W \subseteq V$ affiner Unterraum, Element von $\lnkset{V}{W}$
\end{example}
Dies verallgemeinert sich zu
\begin{definition}
	Sei $H < G$, $\lnkset{G}{H} = \set{x H \mid x \in G} \subseteq \powerset(G)$
\end{definition}
\begin{remark}
	$xH = yH \Leftrightarrow x \sim y$ definiert eine Äquivalenzrelation und das ist äquivalent zu
	\begin{align*}
		\exists h \in H: x = yh \Leftrightarrow y^{-1}y \in H.
	\end{align*}
	Beachte dabei $\lnkset{G}{H} = \lnkset{G}{N}$ ist die Menge aller Äquivalenzklassen $xH = [x]$. Desweiteren gibt es die kanonische Projektion $\pi\colon G \to \lnkset{G}{H},x \mapsto xH$.
	
	Insbesondere ist $G$ die disjunkte Vereinigung aller Äquivalenzklassen. Speziell ist für jedes $x \in G$ definiert:
	\begin{align*}
		t_x\colon G \to G, y \mapsto xy \text{ eine Bijektion,} \quad H = 1H = [x] \to xH = [x].
	\end{align*}
\end{remark}

Alle $xH$ haben also die gleiche Kardinalität und wir erhalten:
\begin{proposition}[Lagrange, Klausur!]
	\label{1_2_9_lagrange}
	Sei $\abs{G} < \infty$ und $H < G$. Dann gilt $\abs{G} = \abs{\lnkset{G}{H}}\cdot \abs{H}$. Insbesondere ist $\abs{G}$ durch $\abs{H}$ teilbar.
\end{proposition}
\begin{proof}
	\ul{Beweisskizze:} Äquivalenzrelation und Bijektion $xH \cong yH$.\\
	Let $H < G$ a subgroup of $G$ and $xH := \set{xy \mid y \in H} \le G$ the coset of $H$ in $G$. Then we have $\lnkset{G}{H} = \set{xH \mid x \in G} \subseteq \powerset(G)$. Then
	\begin{align*}
		xH = \tilde{x}H \Leftrightarrow \exists y \in H \quad x = \tilde{x}y \Leftrightarrow \tilde{x}^{-1}x \in H\\
		G = \bigcup_{xH \in \lnkset{G}{H}} xH \implies \abs{G} = \abs{\lnkset{G}{H}}\cdot \abs{H}\\
		\abs{xH} = \abs{H} \quad \forall x \in G.
	\end{align*}
\end{proof}

\begin{conclusion}
	Sei $\abs{G} < \infty$, dann gilt $\abs{x} \mid \abs{G}$ für alle $x \in G$. Dabei ist $\abs{x} = \abs{\langle \set{x}\rangle} = \min\set{n \mid x^n = 1}$. Also z.B. $\langle \set{x} \rangle \cong (\Z_{\abs{x}},+)$. Insbesondere gilt für alle $x \in G$:  $x^{\abs{G}} = 1$
\end{conclusion}

\begin{conclusion}[Eulers Theorem]
	$\abs{U_n} = \phi(n) = \abs{\set{m \in \set{1, \dots,n} \mid \ggT(n,m) = 1}} = \abs{\set{(\Z_n^{\times}, \cdot) \mid \ggT(n,m) = 1}}$ mit $n \in \N$. Also ist $m^{\phi(n)} = 1 \mod n$.
\end{conclusion}
\begin{definition}[Index]
	Sei $H < G$, dann ist $[G:H] := \abs{\lnkset{G}{H}}$ der Index von $H$ in $G$.
\end{definition}

\begin{conclusion}
	Sei $K < H < G$ und $\abs{G} < \infty$, dann
	\begin{align*}
		[G:K] = \abs{\lnkset{G}{K}} = \frac{\abs{G}}{\abs{H}}\cdot \frac{\abs{H}}{\abs{K}} = [G:H][H:K].
	\end{align*}
\end{conclusion}

\section{Morphismen}
\begin{definition}
	Ein injektiver Morphismus $f\colon G \to H$ wird auch Einbettung genannt. Ein Isomorphismus ist ein bijektiver Morphismus.
\end{definition}

\begin{*remark}
	Ein injektiver Morphismus wird auch Monomorphismus genannt und ein surjektiver Morphismus Epimorphismus.
\end{*remark}

\begin{example}
	\begin{enumerate}[label={\arabic*)}]
		\item Betrachte die Determinate $\det\colon \GL(n,R) \to R^{\times}$, diese ist ein surjektiver Morphismus von Gruppen mit
		\begin{align*}
		\det(gh) = \det(g)\det(h)
		\end{align*}
		\item Die Wahl einer Basis $B$ in einem endlich erzeugten freien Modul $V$ ist ein Isomorphismus von Moduln $s_B\colon R^{\abs{B}} \to V$. Dieser induziert einen Gruppenisomorphismus
		\begin{align*}
			\GL(n,R) \to \GL(V), g \mapsto s_B \circ M_g \circ s^{-1}_B
		\end{align*}
		\item Die Linkstranslation $t\colon G \to S_G \mit x \mapsto t_x$ (mit $t_x(y) = xy$) ist ein injektiver Gruppenhomomorphismus
		\begin{align*}
			(t_x \circ t_z)(y) &= t_x(t_z (y))=t_x(zy) = xzy = t_{xz}(y)
			\intertext{also}
			t_x \circ t_y &= t_{xy} \quad \forall x,y \in G
		\end{align*}
		Ist $t_x = t_z$, so gilt $t_x(1) = t_z(1)$ und daraus $x1=z1$, also $x=z$
		
		Also kann \emph{jede} endliche Gruppe als Untergruppe der $S_n$ verstanden werden ($n = \abs{G}$)!
	\end{enumerate}
\end{example}
\begin{example}
	\begin{enumerate}[label={\arabic*)}]
		\item $y \mapsto f_{x^{-1}}\colon G \to G$, $y \mapsto xyx^{-1}$ ist ein Morphismus $G \to \mathrm{Aut}\,G$ mit
		\begin{align*}
		f_x(y) = x^{-1}yx, \quad f_z(f_x(y)) = z^{-1}(x^{-1}yx)z = (xz)^{-1}y(xz) = f_{xz},
		\end{align*}
		und ist i.A. nicht injektiv! Denke an $G$ abelsch $\Leftrightarrow f_x = \id_G$ $\forall x \in G$
		\item $\sgn\colon S_n \to \Z_2 = \set{-1,1}$
	\end{enumerate}
\end{example}
\section{Normalteiler}
% fix \lhd ==> GEO Fehm nachschaun!
\begin{definition}[normale Untergruppe]
	Eine Untergruppe $H < G$ ist \begriff[Gruppe]{normal} $\Leftrightarrow$ $\forall x \in G\colon xH = Hx$.
\end{definition}
\begin{proposition}
	\begin{align*}
		H < G \text{ ist normal } &\Leftrightarrow \forall x \in G x^{-1}Hx = H \Leftrightarrow AH = HA \quad \forall A \subseteq G
	\end{align*}
\end{proposition}
\begin{proof}
	Sei $H$ normal und $x \in G$. Dann gilt $xH = Hx \implies x^{-1}xH = x^{-1}Hx$. Hier verwende $A,B \subseteq G, AB = \set{ab \mid a \in A, b \in B}$ definiert ein assoziatives Produkt $u \in \powerset(G)$, angewendet auf $A(BH) = (AB)H, A = \set{x^{-1}}, B=\set{x}$. Umgekehrt genauso $xH = Hx \implies x^{-1}xH = x^{-1}Hx$. Weiter gilt:
	\begin{align*}
		AH = \bigcup_{x \in A} x H, \quad HA = \bigcup_{x \in A} Hx
	\end{align*}
	Also $xH = Hx\quad \forall x \in G \implies AH=HA \forall A \subseteq G$. Umgekehrt: Nimm $A = \set{x}$.
\end{proof}
Warum relevant?\\
Weil $\lnkset{G}{H}$ eine Gruppenstruktur von $G$ erbt $\Leftrightarrow H \lhd G$
\begin{proposition}
	\label{1_4_3_prop}
	Sei $H \lhd G$. Dann definiert
	\begin{align*}
		xH \cdot yH = (xy)H
	\end{align*}
	eine Gruppenstruktur auf $\lnkset{G}{H}$ und $\pi: G \to G \mit x \mapsto xH$ ist eine surjektiver Morphismus von Gruppen.
\end{proposition}
\begin{proof}
	$xH \cdot yH$ kann ich in $\powerset(G)$ immer bilden. Ist $H \lhd G$, gilt $xH = Hx$, also $xH\cdot yH = HxyH = xyHH = xyH$. (oder $A=\set{x}, B = \set{y}, C,D=H$). Anders gedacht: $H \lhd G$ heißt: $H \in Z(\powerset(G))$. Sprich: $\lnkset{G}{H} \subseteq \powerset(G)$ ist eine Unterhalbgruppe, d.h. abgeschlossen unter $\cdot$. Ferner gilt: $H = 1H$ ist ein Einselement $(xH)H = x(HH) = xH, H(xH) = (Hx)H = xH=(Hx)H=(xH)H$. Ausserdem ist $\lnkset{G}{H}$ Gruppe 
	\begin{align*}
		xH \cdot x^{-1}H = Hx\cdot x^{-1}H = H1H = HH = H
	\end{align*}
	und genauso $x^{-1}HxH = H$. Sei $\pi: G \to \lnkset{G}{H} \mit x \mapsto xH$ also $\pi(xy) = \pi(x)\pi(y) \mit xyH = xH \cdot yH$.
\end{proof}
\begin{definition}[Kern einer Gruppe]
	Ist $f: G \to K$ ein Morphismus von Gruppen, so definieren wir den \begriff[Gruppe]{Kern} $\ker f = \set{x \in G \mid f(x) = 1_K}$
\end{definition}
\begin{*remark}
	Ist $K$ eine abelsche Gruppe, so schreibt man die Gruppenoperation oft als $+$ und 1 oft als 0.
\end{*remark}
\begin{proposition}
	Es gilt $\ker(f) \lhd G$.
\end{proposition}
\begin{proof} %make more beautiful!!!
	\begin{enumerate}
		\item Sind $x,y \in \ker f$, so gilt $f(x)f(y) = f(xy)$ (und 11 = 1), also ist $ker f$ abgeschlossen unter $\cdot$.
		\item Ferner $f(1) = 1$, da $f$ Morphismus $\implies$ $G \ni 1 \in \ker f$.
		\item 	Zuletzt: $f(x)^{-1} = f(x^{-1})$, dann
		\begin{align*}
		x \in \ker f \implies f(x) = 1 \implies f(x^{-1}) = f(x)^{-1} \implies x^{-1} \in \ker f
		\end{align*}
		\item Ferner gilt für $x \in G, y \in \ker f$:
		\begin{align*}
		f(x^{-1}yx) = f(x^{-1}f(y)f(x)) = f(x^{-1})\cdot 1 f(x) = f(x^{-1}x) = f(1) = 1 \text{ also}\\
		x^{-1} \in \ker f \subseteq \ker f \text{ also }\\
		(x^{-1})^{-1} \ker f (x^{-1}) \subseteq \ker f \implies \ker f \subseteq x^{-1} \ker f x
		\end{align*}
		\item $\ker f \lhd G$.
	\end{enumerate}
	also 1. 2. 3. : $\ker f < G$. 
\end{proof}
\begin{*remark}
	Ist $H \lhd G$, so gilt: $H = \ker \pi \mit \pi: G \to \lnkset{G}{H}$. (denn $\pi(x) = xH$, also $\ker \pi = \set{x \mid xH = H} = H$). Normalerteiler sind also die Kerne von Morphismen.
\end{*remark}
\begin{proposition}[1. Isomorphiesatz, Klausur]
	Ein Morphismus $f: G \to K$ von Gruppen induziert einen Isomorphismus:
	\begin{align*}
		\bar{f}: \lnkset{G}{\ker f} \to \Image f \mit [x] = x \ker f \mapsto f(x)
	\end{align*}
\end{proposition}
\begin{proof} % fix label!!!
	Der einzig schwere Teil ist $\bar{f}$ ist wohldefiniert. $f(xH) := f(x) \in \Image f$, also landet $\bar{f}$ in $\Image f$. Ferner gilt: Ist $x \ker f = y \ker f$, so ist $y^{-1}x \in \ker f$ (allgemein: $xH = yH \Leftrightarrow y^{-1}xH$) $\Leftrightarrow f(y^{-1}x) = 1 \Leftrightarrow f(y^{-1})f(x) = 1 \Leftrightarrow f(y)^{-1}f(x) = 1$ $\Leftrightarrow f(x) = f(y)$. Also ist $\bar{f}$ wohldefiniert und injektiv. Surjektiv ist $\bar{f}$ per Definition.\\
	Letzter Schritt: $\bar{f}(x \ker f \cdot y \ker f) = \bar{f}(xy \cdot \ker f)$ (\propref{1_4_3_prop}). $\bar{f}(x \ker f)\bar{f}(y\ker f) = f(x)f(y) = f(xy)$. Also ist $\bar{f}$ ein Morphismus von Gruppen.
\end{proof}
\begin{example}
	Sei $G = (\R, +)$, $K = (\C\setminus \set{0},\cdot)$ und $f(t) = \exp(2\pi \ii t)$, dann $f(x+y) = \exp(x) + \exp(y)$ also ein Gruppenmorphismus $G \to K$. Dann $\Image f = U(1) = S^1 = \set{z \in \C \mid \abs{z} = 1}$ und $\ker f = \Z$. Also ist $\lnkset{\R}{\Z} \cong S^1$ ($\lnkset{\R^2}{\Z^2} \cong T$ ist dann der Torus $T$, $U(n) = \set{A \in \Mat(n, \C) \mid AA^T = 1}$ unitäre Gruppe)
\end{example}
\begin{definition}[einfache Gruppe]
	Eine Gruppe $G$ ist \begriff[Gruppe]{einfach}, wenn $H \lhd G \implies H = G \und H = \set{1}$.
\end{definition}
\begin{*remark}
	D.h.: Ist $f: G \to K$ irgendein Morphismus, so ist $G \cong \Image f$ ($\ker f = \set{1}$) oder $\Image f = \set{1}$ ($\ker f = G$).
\end{*remark}
Die endlichen einfachen Gruppen sind klassifiziert!
\begin{example}
	\begin{enumerate}
		\item Sei $G = \Z_p, p$ prim hat nach \person{Lagrange} (\cref{1_2_9_lagrange}) noch nicht mal irgendeine \emph{echte} Untergruppe, ist also einfach.
		\item $\lnkset{\SL(n, \Z_p)}{Z(\SL(n, \Z_p))} = \PSL(n, \Z_p)$ (projective linear group)
	\end{enumerate}
\end{example}
\begin{proposition}[Korrespondenztheorem]
	Ist $H \lhd G$, so induziert (definiert) $\pi: G \to \lnkset{G}{H}$ einen Isomorphismus von teilgeordneten Mengen (partial ordered sets)
	\begin{align*}
		\set{L < G \mid H < L} \to \set{K < \lnkset{G}{H}} \mit L \mapsto \pi(L)
	\end{align*}
	Dieser erhält und reflektiert Normalität und auch Unterquotienten. 
\end{proposition}
%\begin{erinnerung}
%	$N \lhd G$ normal subgroups $\Leftrightarrow N < G$ is a subgroup $(NN = N, 1 \in N, N^{-1} = N)$ and the left cosets of $N$ conincides with the right cosets $(xN = Nx \forall x \in G)$ then we have $\lnkset{G}{N}$ inherits a group structure $\pi: G \to \lnkset{G}{N} \with \ker \pi = N$, so $xNyN = xyN$ and then we had the 1. Iso theorem
%\end{erinnerung}
%\begin{theorem}[correspondence theorem]
%	Let $H \lhd G$
%	\begin{align*}
%		\set{L <G \mid H < L} \xrightarrow[iso. of posets]{\cong} \set{K < \lnkset{G}{H}} \with
%		L \mapsto \pi(L)
%	\end{align*}
%	respects and reflects normality.
%\end{theorem}
\begin{proposition}[2. Isomorphiesatz, Klausur]
	Let $H < G$, $K \lhd G \implies H\cap K \lhd H, K\lhd H,K < G$ und 
	\begin{align*}
		\lnkset{H}{H \cap K} \to \lnkset{HK}{K} \with x(H\cap K) \mapsto xK
	\end{align*}
	ist ein Isomorphismus.
\end{proposition}
\begin{proof}
	\begin{enumerate}
		\item Durchschnitte von Untergruppen sind Untergruppen, also $H \cap K < H$. Ist ferner $x \in H$ und $y \in H\cap K$, so gilt $xyx^{-1} \in H$ (da $H$ Untergruppe) und $xyx^{-1} \in K$, da $x \in G$ und $y \in K$ und auch $K \lhd G$. Also gilt $xyx^{-1} \in H \cap K$.
		\item Auch klar, da $(HK)(HK) = H(KH)K = H(HK)K = (HH)(KK) = HK$ (wobei $K \lhd G$), Bemerkung: $H,K < G$ reicht nicht, $HK$ ist i.A. \emph{keine} Untergruppe von $G$. Klar das $1 = 1\cdot 1 \in HK, 1 \in H, 1\in K$ (da $H,K < G$).
		\begin{align*}
			x \in H, y\in K\quad (xy)^{-1} = y^{-1}x^{-1} \implies (HK)^{-1} = K^{-1}H^{-1} = KH = KH\text{, da } K\lhd G.
		\end{align*}
		Also gilt $HK < G$. $K \lhd HK$, wie im ersten Punkt des Beweises.
		\item \begin{align*}
		\phi: \lnkset{H}{H \cap K} \to \lnkset{HK}{K} \mit (H \cap K) \mapsto xK
		\end{align*}
		wohldefiniert? Natürlich: Ist $x(H \cap K) = y (H\cap K)$ $x,y \in H$, so folgt $x = yz$ mit $z \in H \cap K$, also $yK = xK$, da 2. eben insbesondere in $K$ ist. Gruppenhomomorphismus auch klar, da
		\begin{align*}
			\phi(x(H \cap K) \cdot y(H \cap K) = \phi(xy(H \cap K)) = xyK\\
			\phi(x(H\cap K))\phi(y(H\cap K)) = xKyK
		\end{align*}
	\end{enumerate}
\end{proof}
\begin{lemma}
	$\phi: G \to H$ ist injektiv genau, dann wenn $\ker \phi = \set{1}$. 
\end{lemma}
\begin{proof}
	Auf diesen Fall angewendet:
	\begin{align*}
	\ker \phi = \set{x(H \cap K) \subset \lnkset{H}{H \cap K} \mid xK = K}
	\end{align*}
	$xK = K$ heisst aber nichts anderes als $x \in K$, d.h.
	\begin{align*}
	\ker \phi = \set{x(H \cap K) \mid x \in H \cap K} = H\cap K = 1
	\end{align*}
	(in der Gruppe $\lnkset{H}{H \cap K}$) und surjektiv ist klar, da $xyK = xK$ für $x \in H, y \in K$.
\end{proof}
\begin{proposition}[3. Isomorphiesatz]
	Sei $H \lhd G, K \lhd G, K < H$ impliziert $\lnkset{H}{K} \lhd \lnkset{G}{H}$ und es gilt
	\begin{align*}
		\lnkset{G}{K} \to \lnkset{\lnkset{H}{H}}{\lnkset{H}{K}} \mit xK \mapsto (xH)\cdot \lnkset{H}{K}
	\end{align*}
	ist ein Isomorphismus.
\end{proposition}
\section{Einfache Gruppen}
\begin{definition}
	$G$ ist einfach, genau dann wenn $\set{1}, G$ sind die einzigen Normalteiler.
	\begin{align*}
		\phi: G \to H \mit \Image \phi = \lnkset{G}{\lnkset{G}{\ker \phi}}\\
		\Image \phi \cong G \oder \Image \phi = \set{1}
	\end{align*}
\end{definition}
\begin{*example}
	$G = \Z_p,p$ prim.
\end{*example}
\begin{proposition}
	\label{prop_a_n_simple}
	$A_n$ ist einfach für $n > 4$.
\end{proposition}
\begin{*remark}
	$A_4 \rhd \set{(), (12)(34),(13)(24),(14)(23)}$, d.h. $A_4$ ist nicht einfach.
\end{*remark}
\begin{lemma}
	\begin{enumerate}
		\item $S_n = \langle (i i+1)\rangle$ für $i = 1, \dots, n -1$
		\item $A_n = \langle (ijk) \rangle$
	\end{enumerate}
\end{lemma}
\begin{proof}
	\begin{enumerate}
		\item 	Betrachte $(23)(12)(23) = (13)$ und $(34)(13)(34) = (14)$, per Induktion folgt 
		\begin{align*}
		(1i) \in \langle (i i+1)\rangle \und (1i)(1j)(1i) = (ij) \quad (i \neq j)
		\implies (ij) = \langle (r r+1) \rangle \quad r = 1,\dots n-1
		\end{align*}
		damit folgt 1.
		\item 
		\begin{align*}
			(ij)(jk) = (ijk) \und (ij)(kr) = (ijk)(jkr)
		\end{align*}
		also folgt 2.
	\end{enumerate}
\end{proof}
\begin{proof}[\cref{prop_a_n_simple}]
	Sei $N \neq \set{1}$ Normalteiler von $A_n, n> 4$
	\begin{enumerate}
		\item Ist $(yk) \in N$, so gilt $(abc) \in N$ für alle $a,b,c$. Ist $\sigma \in S_n$, so gilt $\sigma(ijk)\sigma^{-1} = (\sigma(i)\sigma(j)\sigma(k))$ (siehe ÜA 5, 1. Blatt) Sei $\sigma \in S_n$so, dass $\sigma(i) = a$, $\sigma(j) = b$ und natürlich $\sigma(k) = c$. Ist $\sigma \in A_n$ so folgt (da $N \lhd A_n$, $(abc) \in N$). Wenn nicht wähle $r,s$ ungleich und setze $\tilde{\sigma}:= \sigma(rs) \in A_n$. % maybe inclusion?!
		Dann ist $\tilde{\sigma}(ijk)\tilde{\sigma}^{-1} = \sigma(rs)(ijk)(rs)\sigma^{-1} = (abc)$.
		\item Bleibt zu zeigen: $\exists(ijk) \in N$. Sei
		\begin{align*}
			1 \neq \sigma=\gamma_1 \gamma_2 \dots \gamma_r \in N \text{ beliebig,}
		\end{align*} 
		wobei $\gamma_i$ Zykel ist und die Länge der Zykel $\gamma_i$ nicht wachse also $l(\gamma_i) \ge l(\gamma_{i+1})$. Fallunterscheidung: (Ziel ist in jedem Fall gibt es ein $(ijk) \in N$.)
		\begin{enumerate} %TODO make nicer!
			\item $l(\gamma_i) \ge 4$:\\
				Sei $\gamma_1 = (i_1, \dots, i_k)$ und $\pi(i_1 i_2 i_3)$ 
				\begin{align*}
					\implies \pi_{\gamma_j} = \gamma_j \pi \quad \forall j > 1				\sigma^{-1}\underbrace{\pi^{-1}\sigma \pi}_{\in N\text{, da } N \lhd A_n} = (i_1 i_2 i_4) \subset N
				\end{align*}
				also einfach Nachrechnen.
			\item $l(\gamma_1) - l(\gamma_2) = 3$:\\
			Sei $\gamma_1=(ijk), \gamma_2 = (pqs)$. Nimm $\pi = (kpq)$ und daraus folgt $\sigma^{-1}\pi^{-1}\sigma\pi = (isk)$
			\item $l(\gamma_1) = 3, l(\gamma_2) = 2$:\\
			$\gamma_1 = (ijk) \implies \sigma^{2} = (ikj)$
			\item $l(\gamma_1) = l(\gamma_2) = 2, r = 2$:\\
			Sei $\gamma_1 = (ij)$, $\gamma_2 = (kl)$ $n \ge 5$ $\implies \exists m \notin \set{i,j,k,l}$ und $\pi = (ijm)$\\
			$\sigma \pi^{-1}\sigma\pi = (imj) \in N$
			\item $l(\gamma_i) = 2 \forall i, r > 2$:\\
			$\sigma \in A_n$, $\gamma_1 =(ij)$, $\gamma_2 = (kl)$, sowie $\gamma_3 = (pq)$, $\gamma_4 = (st)$, setze $\pi = (ip)(jk)$ und $\sigma\pi \sigma \pi = (ipl)(jkq)$ und benutze Fall 2.
		\end{enumerate}
	\end{enumerate}
\end{proof}
\begin{definition}
	Eine kurze Sequence von Gruppen ist ein Paar von Morphismen $f: H \to G \mit g: G \to K$ und dann
	\begin{enumerate}
		\item $f$ ist injectiv
		\item $g$ ist surjektiv
		\item $\Image f = \ker g$
	\end{enumerate}
	Man schreibt auch:
		\[
			\begin{tikzcd}
			\set{1} \arrow[r] & H \arrow[r, "f"] & G \arrow[r, "g"] & K \arrow[r] & \set{1}
			\end{tikzcd}
		\]
		Sprich: $H$ ist (isomorph zu einer) normalen Untergruppe von $G$ und $K$ ist (isomorph zu) $\lnkset{G}{H}$\\
		Allgemeiner: exakte Folgen:\\
		\[
			\begin{tikzcd}
			... \arrow[r, "f_i"] & G_{i-1} \arrow[r, "f_{i-1}"] & G_{i-2} \arrow[r] & ...
			\end{tikzcd} \quad \Image f_i = \ker f_{i-1}
		\]
\end{definition}
Einfachster Fall ($\implies$ langweiliger) Fall: \begriff{Direkte Produkte}
\begin{definition}[äußeres direktes Produkt]
	Seien $H,K$ Gruppen. Auf der Menge $G:= H \times K$ (Wenn sowas im Buch steht ist es direktes Produkt gemeint) erhalten von einer Gruppenstruktur durch
	\begin{align*}
		(a,b)(x,y) := (ax,by) \mit a,b \in H,x,y \in K
	\end{align*}
\end{definition}
\begin{definition}[\begriff{inneres direktes Produkt}]
	Sei $G$ Gruppe und $H,K \lhd G$ mit
	\begin{enumerate}
		\item $H \cap K = \set{1}$ %diese für den Beweis unten!
		\item $HK = G$
	\end{enumerate}
	Dann nennen wir $G$ das innere direkte Produkt von $H \und K$.
\end{definition}
\begin{proposition}
	Ist $G$ das innere direkte Produkt von $H,K \lhd G$, so gilt 
	\begin{align*}
		G \cong H \times K
	\end{align*}
	als Gruppe.
\end{proposition}
\begin{proof}
	Wir zeigen, dass $\phi: H \times K \to G \mit (a,b) \mapsto ab$ ein Isomorphismus von Gruppen ist. Es gilt für alle $a,x \in H, b,y \in K$.
	\begin{align*}
		\phi((a,b))\cdot \phi((x,y)) = abxy = axx^{-1}bxb^{-1}by = axby = \phi((ax,by))\\
	\end{align*}
	denn der Kommutator $x^{-1}bxb^{-1}$ liegt in $H \cap K = \set{1}$. (denn $x^{-1}bx \in K$, da $K\lhd G$, also $x^{-1}bxb^{-1} \in K$, genauso $bxb^{-1} \in H$, da $H \lhd G$, also $x^{-1}bxb^{-1} \in H$) Nach Annahme 1. ist $\phi$ surjektiv. Die Abbildung ist injektiv, denn
	\begin{align*}
		ab = xy \implies x^{-1}a = yb^{-1} \in G \cap K = \set{1} \implies x=a, y=b \quad \forall y,x\in H,b,y \in K.
	\end{align*}
\end{proof}
\begin{*remark}
	In diesem Fall ist $\lnkset{G}{H} \cong K, \lnkset{G}{K} \cong H$
	\[
		\begin{tikzcd}
		1 \arrow[r] & H \arrow[r] & G \arrow[r] & K \arrow[r] & 1
		\end{tikzcd}
	\]
	\[
		\begin{tikzcd}
		1 \arrow[r] & K \arrow[r] & G \arrow[r] & H \arrow[r] & 1
		\end{tikzcd}
	\]
\end{*remark}
\begin{*example}
	Sei $G = D_6$, also die Sachen, die man mit einem Hexagon machen kann.
	\begin{align*}
		H = \set{1, r^3} \cong \Z_2 \und K = \set{s^jr^{2i} \mid i =0,1,2, j=0,1} \cong D_3\\
		D_6 \cong \Z_2 \times D_3
	\end{align*}
\end{*example}
Kompositionsreihen und \person{Jordan}-\person{Hölder}.
\begin{definition}[\begriff{Kette},\begriff{Subnormale Reihe},\begriff{einfache Kompositionsreihe}]
	Eine Reihe in $G$ ist eine Kette von Untergruppen
	\begin{align*}
		G = G_0 > G_1 > G_2 > \dots > G_d = \set{1} \mit G_{i+1} \neq G,
	\end{align*}
	Eine subnormale Reihe ist eine in der $G_{i+1} \lhd G_i$ gilt ($G_i \lhd G$ für alle $i$ $\Leftrightarrow$ ``normale Reihe''). Eine Kompositionsreihe ist eine solche mit $\lnkset{G_i}{G_{i+1}}$ einfach.
\end{definition}
\begin{*remark}
	Nach dem Korrespondenztheorem heisst $\lnkset{G_i}{G_{i+1}}$ einfach genau, dass $G_{i+1} \lhd G_i$ eine maximale normale Untergruppe ist.
	\begin{align*}
		\set{L < G_i \mid G_{i+1} < L} \xrightarrow{\pi} \set{P < \lnkset{G_i}{G_{i+1}}}
	\end{align*}
\end{*remark}
\begin{example}
	Sei $G=G_0=S_4, G_1 = A_4, \lnkset{S_4}{A_4} \cong \Z_2$ (nach 1. Isomorphiesatz, $A_4 = \ker \sgn \mit \sgn: S_4 \to \Z_2$). Und die $G_2 = N = \set{(12)(34),(13)(24),(14)(23),1}$ \person{Klein}sche Vierergruppe. Dann
	\begin{align*}
		\abs{\lnkset{A_4}{N}} = \frac{\abs{A_4}}{\abs{N}} = \frac{12}{4} = 3 \quad \text{ Lagrange Theorem}\\
		\implies \lnkset{A_4}{N} \cong \Z_3 = \lnkset{\Z}{3\Z} \text{ einfach.}
	\end{align*}
	\begin{align*}
		G_3:= \set{1}\\
		\set{1} \lhd H \lhd A_4 \lhd S_4 \text{ ist Kompositionsreihe}
	\end{align*}
	\[ %how to get two tikzcd in one \[\] under each other?
		\begin{tikzcd}
			\set{1} \arrow[r] & N \arrow[r] & A_4 \arrow[r] & \Z_3 \arrow[r] & \set{1}
		\end{tikzcd}
	\]
	\[
		\begin{tikzcd}
			\set{1} \arrow[r] & A_4 \arrow[r] & S_4 \arrow[r] & \Z_2 \arrow[r] & \set{1}
		\end{tikzcd}
	\]
	wobei $N, \Z_3, \Z_2$ einfach ist und $A_4$ gerade gebaut.
\end{example}
\begin{proposition}
	Ist $G$ endliche Gruppe, so besitzt $G$ eine Kompositionsreihe.
\end{proposition}
\begin{proof}
	Induktion nach $\abs{G}$. Also $G = G_0$ gegeben
	\begin{enumerate}
		\item $G$ einfach, dann $G_1 = \set{1}$ und \checkmark $G=G_0 \triangleright \set{1} = G_1$, also $\lnkset{G_0}{G_1} = G$
		\item $G$ nicht einfach dann $G_1$ maximale normale Untergruppe, gibts da $\abs{G} < \infty$. Dann $\abs{G_1}< \abs{G}$, also existiert nach Induktion Kompositionsreihe
		\begin{align*}
			G_1 \triangleright G_2 \triangleright G_3 \triangleright \dots \triangleright G_d \triangleright \set{1}
		\end{align*}
		Also $G_0 \triangleright G_1 \triangleright \dots$ ``tuts''.
	\end{enumerate}
\end{proof}
\begin{*example}
	$G = \Z$ hat keine Kompositionsreihe, denn
	\begin{align*}
		\Z = G_0 \triangleright G_1 \implies G_1 \cong \Z	
	\end{align*}
\end{*example}
\begin{proposition}[\person{Jordan}-\person{Hölder}]
	Sei $G$ endliche Gruppe und seien $\set{H_i}_{i=0,\dots,p}$ und $\set{G_j}_{j=0,\dots,n}$ zwei Kompositionsreihe der Länge $p$. Dann gilt $p = n$ und ex existiert $\sigma \in S_{0,\dots, n-1}$ mit
	\begin{align*}
		\lnkset{G_i}{G_{i+1}} \cong \lnkset{H_{\sigma(i)}}{H_{\sigma(i)+1}}
	\end{align*}
\end{proposition}
\begin{proof}
	\begin{align*}
		G = G_0 \triangleright G_1 \triangleright \dots \triangleright G_n = \set{1}\\
		G = H_0 \triangleright H_1 \triangleright \dots \triangleright H_p = \set{1}
	\end{align*}
	Beweis erfolgt durch Induktion nach $m = \min(n,p)$\\
	Beweise zuerst folgende Behauptung:\\
		Ist $G = G_0 \rhd G_1 \rhd \dots \rhd G_1 = 1$ Kompositionsreihe und $N \lhd G$, so ist $N = N \cap G_0 \rhd N \cap N_1 \rhd \dots, \rhd N \cap 1 = 1$ Kompositionsreihe (gegebenenfalls nach Auslassen von $N \cap G_1 = N ßcap G_{i+1}$)
	\begin{proof}
		\begin{align*}
			\lnkset{N \cap G_i}{N \cap G_{i+1}} = \lnkset{N \cap G_i}{(N \cap G_i)\cap G_{i+1}} \mit (G_{i+1} < G_1!)\\
			\cong \lnkset{(N \cap G_1)G_{i+1}}{G_{i+1}} \lhd \lnkset{G_i}{G_{i+1}} \text{ einfach }\quad \cong\text{ da 2. Isomorphiesatz}
		\end{align*}
		Also $\lnkset{N \cap G_i}{N \cap G_{i+1}} \cong \lnkset{G_i}{G_{i+1}}$ oder $\lnkset{N \cap G_i}{N \cap G_{i+1}} \Leftrightarrow N \cap G_i = N \cap G_{i+1}$.
	\end{proof}
\begin{enumerate}
	\item Fall 1: $G_1 = H_1$ folgt mit Induktion 
	\begin{align*}
		G_i = H_1 \rhd G_2 \rhd \dots \rhd G_n = \set{1}\\
		G_i = H_1 \rhd G_2 \rhd \dots \rhd H_p = \set{1}
	\end{align*}
	\item Fall 2: $G_1 \neq H_1$. Dann ist $G \cap H_1 \lhd G_1$ (nicht gleich!). Beachte Nach Korrespondenztheorem ist $H_1 \le$ ... %TODO
\end{enumerate}
\end{proof}