\section{Wiederholung}
\begin{definition}[Halbgruppe, Monoid, Gruppe]
	Eine \begriff{Halbgruppe} ist eine Menge $G$ mit einem \emph{assoziativen} Produkt
	\begin{align*}
		\cdot\colon G \times G \to G.
	\end{align*}
	Ein \begriff{Monoid} ist eine Halbgruppe, in der ein Element $1 \in G$ existiert mit 
	\begin{align*}
		1\cdot x = x\cdot 1 \quad\forall x \in G.
	\end{align*}
	Eine \begriff{Gruppe} ist ein Monoid, in dem für jedes $x \in G$ ein $y \in G$ existiert mit
	\begin{align*}
		xy = yx = 1.
	\end{align*}
\end{definition}
\begin{remark} % Bem 1
		1 ist eindeutig, wenn existiert $y$ ist durch $x$ eindeutig bestimmt $x^{-1} :=y$. % correct sentence
\end{remark}
\begin{definition}[Morphismus]
	Ein Morphismus zwischen Gruppen $G,H$ ist eine Abbildung
	\begin{align*}
		f: G \to H \mit f(xy) = f(x)f(y) \quad\forall x,y \in G.
	\end{align*}
\end{definition}
\begin{proposition}
	Ist $f: G \to H$ ein Morphismus von Gruppen, so gilt
	\begin{align*}
		f(1_G) = 1_H \und f(x^{-1}) = f(x)^{-1} \quad \forall x \in G
	\end{align*}
\end{proposition}
\begin{proof}
	Für alle $x \in G$ gilt $f(x) = f(1\cdot x) = f(1)f(x)$. Gilt in einer beliebigen Gruppe jedoch $ab=b$ für zwei Elemente $a,b$, so folgt % add which prop is used here?
	\begin{align*}
		(ab)\cdot b^{-1} = a(bb^{-1}) = a\cdot 1 = a \mit bb^{-1} = 1
	\end{align*}
	Ferner gilt $f(x)\cdot f(x^{-1}) = f(x \cdot x^{-1}) = f(1) = 1$ wie schon gezeigt (und genauer $f(x^{-1})f(x) = 1$. Also $f(x^{-1}) = f(x)^{-1}$).
\end{proof}
\begin{example}
	\begin{enumerate}
		\item Sei $X$ beliebige Menge und $S_X = \set{f: X \to X \mid f \text{ bijektiv}}$ ist eine Gruppe bezüglich Komposition mit $1 = \id_X$. Insbesondere ist $S_n = \set{S_{\set{1,\dots,n}}}$ \begriff{symmetrische Gruppe} und ein Element $f \in S_n$ ist eine \begriff{Permutation}.
		\item $\GL(V) = \set{f \in S_V \mid f \text{ linear}}$, wobei $V$ ein $R$-Modul ist mit kommutativen assoziativen Ring mit 1.
		\item $\Z, \Z_n$ unter Addition 
		\begin{align*}
			U_n = \Z^{\times}_n = \set{ m \in \set{0,\dots,n-1} \mid \ggT(m,n) = 1}\quad \forall x,y \in G \colon xy = yx
		\end{align*}
		wobei beide Gruppen abelsch sind.
		\item $G = U(1) = \set{z \in C \mid \abs{z} = 1} = \set{e^{it}\mid f \in [0,2\pi]}$
		\item $G = U(1) \times \SU(2)\times \SU(3)$ Eichgruppe im Standardmodell der Elementarteilchen  
	\end{enumerate}
\end{example}
\begin{definition}[Ordnung]
	Ist $G$ endlich, so nennt man $\abs{G}$ die \begriff{Ordnung} von $G$.
\end{definition}
\begin{example}
	$\abs{S_n} = n!$
\end{example}
\begin{definition}[$p$-Gruppe]
	Ist $\abs{G} = p^n$ für eine Primzahl $p$, so nennt man $G$ eine \begriff{$p$-Gruppe}. ($n \in \N$)
\end{definition}
\begin{definition}[Untergruppe]
	Sei $G$ Gruppe. Eine Teilmenge $H \le G$ ist eine \begriff{Untergruppe} $H < G$, wenn
	\begin{enumerate}
		\item Für alle $x,y \in H$ $xy \in H$
		\item $1 \in H$
		\item Für alle $x \in H$ $\exists x^{-1} \in H$ 
	\end{enumerate}
\end{definition}
\begin{proposition}
	Ist $\abs{G} \le \infty$, so folgt 2) aus 1)  und $H \neq \emptyset$. % link previous definition here please.
\end{proposition}
\begin{proof}
	Sei $x \in H$ ein beliebiges Element. Aus 1) folgt $x^n \in H$ für alle $n \in \N_{+}$. Da $\abs{G}< \infty$ existiert $n\neq m$ mit $x^n = x^m$. O.b.d.A.
	\begin{align*}
		 n > m &\Longleftrightarrow x^{n-m} x^m = x^n\\
		 &\Longleftrightarrow x^{n-m} = 1
	\end{align*}
	Also gilt 2). Ferner impliziert die Existenz der inversen Elemente, dass die Linkstranslation $t_x: G \to G \mit y \mapsto xy$ ($x \in G$ fest) injektiv ist, denn $(t_x)^{-1} = t_{x^{-1}}$. Ist $x \in H$, so heißt 1) gerade $t_x(H) \subseteq H$, sprich $t_x$ kann zu $t_{x\mid_H}. H \to H$ eingeschränkt werden. Die Einschränkung einer injektiven Abbildung ist injektiv. Da $\abs{H} \le \abs{G} < \infty$, folgt $t_{x\mid_H}: H \to H$ surjektiv. Aber existiert $y \in H: t_x(y)= 1$. Eindeutigkeit von $x^{-1}$ heißt $y = x^{-1} \in H$.
\end{proof}
\begin{definition}[Erzeugendensystem]
	Ist $X \subseteq G$, so ist
	\begin{align*}
		\langle X \rangle = \bigcap_{\substack{H < G\\X \subset H}} H \text{ die von $X$ erzeugte Untergruppe}.
	\end{align*}
	Ist $\langle X \rangle = G$ nennen wir $X$ ein Erzeugendensystem.
\end{definition}
\begin{definition}[Konjugation]
	Ist $H < G$ und $x \in G$, so ist $X^{-1}Hx = \set{x^{-1}Hx\mid y \in H}$ eine Untergruppe (``$x^{-1}yx$'' $y$ ist konjugiert mit $x$). Wir nennen diese zu $H$ konjugiert.
	\begin{align*}
		(x^{-1}yx)^{-1} = x^{-1}y^{-1}x \und x^{-1}yx\cdot x^{-1}zx = x^{-1}yzx
	\end{align*}
\end{definition}
\begin{definition}[Konjugationsklasse]
	Die Menge $\set{x^{-1}yx \mid x \in G}$ ist i.A. \emph{keine} Untergruppe und diese nennt man \begriff{Kojugationsklasse} von $y$.
\end{definition}
\begin{definition}[Zentralisator, Zentrum]
	Der \begriff{Zentralisator} von $y \in G$ ist $\set{x \in G \mid xy = yx} =: Z_G(y)$. Das \begriff{Zentrum} von G ist
	\begin{align*}
		Z(G) = \bigcap_{y \in G} Z_G(y) = \set{x \in G \mid \forall y \in G xy=yx}.
	\end{align*}
\end{definition}
\begin{example}
	Sei $G = S_n \ni f$ Permutation, z.B.
	\begin{align*} S_6 \in
		\begin{pmatrix}
		1 & 2 &3 &4 & 5 & 6\\
		5 & 4 & 6 & 1 & 2 & 3
		\end{pmatrix}  = (1524)(36)
	\end{align*}
	letzteres nennt man \begriff{Zykelnotation}. 1-Zykeln, d.h. $i \in \set{1,\dots,n}$ mit $f(i)=i$ werden meist nicht notiert, z.B.:
	\begin{align*} S_4 \in 
		\begin{pmatrix}
			1&2&3&4\\
			2&1&3&4
		\end{pmatrix} = (12)
	\end{align*}
\end{example}
\begin{remark}
	Ein $k$-Zykel ist ein Produkt von $k-1$ Transpositionen (2-Zykel), z.B.
	\begin{align*}
		(12345) = (15)(14)(13)(12)
	\end{align*}
	ist das Produkt in $S_5$, d.h. Komposition! Also erzeugt $\set{(y)}$ die $S_n$. Jede Permutation kann also als Produkt von Transpositionen geschrieben werden. Diese Darstellung ist nicht eindeutig! (z.B. $(12)(23)(12) = (23)(12)(23)$) (``Braid relation'') % add braid picture?! braids package
	und $(12)(12) = ()$. Allerdings kommen in jeder solcher Darstellungen entweder eine gerade oder ungerade Anzahl von Transpositionen vor. ($\to$ Fehlstände). Insbesondere bilden gerade Permutationen (gerade Anzahl an Fehlständen $\Leftrightarrow$ Produkte von zu Transpositionen) eine Untergruppe $A_n < S_n$, die sogenannte \begriff{alternierende Gruppe}.
\end{remark}
Sei $G$ eine endliche Gruppe.
\begin{example}
	Also $G = \GL(n,R) = $ invertierbare Matrizen mit Einträgen in $R$ (nur endliche, wenn $R^{\times} endlich$!). Untergruppen sind
	\begin{itemize}
		\item $\SL(n,R) = \set{g \in \GL(n,R) \mid \det g = 1}$
		\item $O(n,R) = \set{g \in G \mid gg^T = g^T g = 1}$ mit dem Skalarprodukt $\scaProd{gv}{gw} = \scaProd{v}{w} \quad \forall v,w \in R^n$
		\item $\SO(n,R) = \set{g \in G \mid \dots}$ % TODO
	\end{itemize}
	Ist $R$ Ring mit Involuten (z.B. $R = \C, z = \bar{z}$)
	\begin{itemize}
		\item $U(n,R) = \set{g \in \GL(n,R) \mid gg^{*} = g^{*}g = 1}$
		\item $\SU(n,R) = \SL(n,R) \cap U(n,R)$
	\end{itemize}
\end{example}
\begin{example}
	$D_n = \set{f: \R^2 \to \R^2 \text{ linear, bjektiv} \mid f(P_n) = P_n}$, wobei $P_n \subset \R^2$ das regulär $n$-gen ist. $P_6$ ist das Hexagon % sketch? 
	Alternativ ist $D_n \subseteq S_n$, wobei $\set{1, \dots,n}$ mit der Menge der Ecken von $P_n$ identifiziert wird und man erhält alle Permutationen, die benachbarte Ecken auf benachbarte abbilden
	\begin{itemize}
		\item $r = $ Rotation um $\sfrac{2\pi}{n}$ im mathematische positiven Sinn
		\item $s = $ eine beliebige Spiegelung in $D_n$
		Also hat man
		\begin{align*}
			\langle \set{s,r}\rangle = D_n = \set{s^i r^j \mid i = 0,1, j=0, \dots, n-1} \quad \abs{D_n} = 2n\\
			srs = r^{n-1} = r^{-1} \quad r^n = 1 \quad s^2 = 1\\
			\langle \set{r,s}\rangle= D_{\alpha} \subset S_{\Z} \quad r(z) = z+1 \quad s(z) = -z \quad r,s : \Z \to \R\\
		\end{align*}
	\end{itemize}
\end{example}
\section{Nebenklassen, Normalteiler, Isomorphiesätze}
\begin{definition}
	$A,B \subseteq G$ Teilmengen (nicht unbedingt Untergruppen!)
	\begin{align*}
		AB = \set{xy \in G \mid x \in A, y \in B} \quad A^{-1} = \set{x^{-1} \in G \mid x \in A}
	\end{align*}
\end{definition}
\begin{remark}
	$\emptyset \neq H \subseteq G$ ist Untergruppe $\Leftrightarrow HH = H, H^{-1} = H$
\end{remark}
\begin{definition}
	Ist $x \in G$, so nennen wir $f_x : G \to G \mit y \mapsto x^{-1}yx$ den durch $x$ definierer inneren Automorphismus. Ist $H < G$, so nennen wir $f(H) = x^{-1}Hx$ eine zu $H$ konjugierte Untegruppe.
\end{definition}
\begin{proposition}
	\begin{enumerate}
		\item $f_x$ ist ein Endomorphismus von $G$ (d.h. ein Morphismus $G \to G$)
		\item Das Bild $\Image f$ eines beliebigen Gruppenmorphismus $f: K \to L$ ist eine Untergruppe: $\Image f < L$
	\end{enumerate}
\end{proposition}
\begin{proof}
	\begin{enumerate}
		\item $f_x (yz) = x^{-1}yzx = x^{-1}y(xx{^-1})zx = (x^{-1}yx)(x^{-1}zx) = f_x(y)f_x(z)$ $\forall y,z \in G$
		\item 
		\begin{itemize}
			\item $\Image f$ ist abgeschlossen. Seien $f(y), f(z) \in \Image f$. Dann gilt:
			\begin{align*}
			f(y)f(z) = f(yz) \in \Image f
			\end{align*}
			\item $f(1) = 1 \implies 1 \in \Image f$
			\item $f(x)^{-1} = f(x^{-1}) \implies (\Image f)^{-1} = \Image f$
		\end{itemize}
	\end{enumerate}
\end{proof}
\begin{definition}
	Ist $H < G, x \in G$, so nennt man
	\begin{align*}
	G \supseteq x H &= \set{x}H = \set{xy \in G \mid y \in H} \quad \text{linke Nebenklasse}\\
	G \supseteq H x &= \set{yx \in G \mid y \in H} \quad \text{rechte Nebenklasse}
	\end{align*}
\end{definition}
\begin{example}
	Sei $G = V$ Vektorraum über Körper $K$ mit $+$ als Gruppenstruktur, dann ist $H = W < V$ ein Untervektorraum und $xH = x + W \subseteq V$ affiner Unterraum, Element von $\lnkset{V}{W}$
\end{example}
Dies verallgemeinert sich zu
\begin{definition}
	Sei $H < G$, $\lnkset{G}{H} = \set{x H \mid x \in G} \subseteq \powerset(G)$
\end{definition}
\begin{remark}
	$xH = yH \Leftrightarrow x \sim y$ definiert eine äquivalenzrelation und das ist äquivalent zu $\exists h \in H: x = yh \Leftrightarrow y^{-1}y \in H$. Beachte dabei $\lnkset{G}{H} = \lnkset{G}{N}$ ist die Menge aller Äquivalenzklassen $xH = [x]$. Desweiteren gibt es die kanonische Projektion $\pi : G \to \lnkset{G}{H} \mit x \mapsto xH$.
\end{remark}
Insbesondere ist $G$ die disjunkte Vereinigung aller Äquivalenzklassen. Speziell ist für jedes $x \in G$ definiert:
\begin{align*}
	t_x : G \to G \mit y \mapsto xy \text{ eine Bijektion } \quad H = 1H = [x] \to xH = [x].
\end{align*}
Alle $xH$ haben also die gleiche Kardinalität und wir erhalten:
\begin{proposition}[Lagrange, Klausur!]
	Sei $\abs{G} < \infty$ und $H < G$. Dann gilt $\abs{G} = \abs{\lnkset{G}{H}}\cdot \abs{H}$. Insbesondere ist $\abs{G}$ durch $\abs{H}$ teilbar.
\end{proposition}
\begin{conclusion}
	Sei $\abs{G} < \infty$, dann $\abs{x} \mid \abs{G}$ für alle $x \in G$. Dabei ist $\abs{x} = \abs{\langle \set{x}\rangle} = \min\set{n \mid x^n = 1}$. Also z.B. $\langle \set{x} \rangle \cong (\Z_{\abs{x}},+)$. Insbesondere gilt für alle $x \in G$ $x^{\abs{G}} = 1$
\end{conclusion}
\begin{conclusion}[Eulers Theorem]
	$\abs{U_n} = \phi(n) = \abs{\set{m \in \set{1, \dots,n} \mid \ggT(n,m) = 1}} = \abs{\set{(\Z_n^{\times}, \cdot) \mid \ggT(n,m) = 1}}$ mit $n \in \N$. Also ist $m^{\phi(n)} = 1 \mod n$.
\end{conclusion}
\begin{definition}[Index]
	Sei $H < G$, dann $\abs{[G:H]} := \abs{\lnkset{G}{H}}$ Index vo $H$ in $G$.
\end{definition}
\begin{conclusion}
	Sei $K < H < G$ und $\abs{G} < \infty$, dann
	\begin{align*}
		[G:K] = \abs{\lnkset{G}{K}} = \frac{\abs{G}}{\abs{H}}\cdot \frac{\abs{H}}{\abs{K}} = [G:H][H:K].
	\end{align*}
\end{conclusion}
\section{Morphismen}
\begin{definition}
	Ein injektiver Morphismus $f: G \to H$ wird auch Einbettung genannt. Ein Isomorphismus ist ein bijektiver Morphismus.
\end{definition}
\begin{*remark}
	Ein injektiver Morphismus wird auch Monomorphismus genannt und ein surjektiver Morphismus Epimorphismus.
\end{*remark}
\begin{example}
	\begin{enumerate}
		\item Betrachte die Determinate $\det: \GL(n,R) \to R^{\times}$, diese ist ein surjektiver Morphismus von Gruppen mit
		\begin{align*}
		\det(gh) = \det(g)\det(h)
		\end{align*}
		\item Die Wahl einer Basis $B$ in einem endlich erzeugten freien Modul $V$ ist ein Isomorphismus von Moduln $s_B: R^{\abs{B}} \to V$. Dieser induziert einen Gruppenisomorphismus
		\begin{align*}
			\GL(n,R) \to \GL(V) \mit g \mapsto s_B \circ M_g \circ s^{-1}_B
		\end{align*}
		\item Die Linkstranslation $t: G \to S_G \mit x \mapsto t_x$ (mit $t_x(x) = xy$) ist ein injektiver Gruppenhomomorphismus
		\begin{align*}
			(t_x \circ t_z)(y) = t_x(t_z (y))=t_x(zy) = xzy = t_{xz}(y)\\
			\text{also} t_x \circ t_y = t_{xz} \quad \forall x,z \in G
		\end{align*}
		Ist $t_x = t_z$, so gilt $t_x(1) = t_z(1)$ und daraus $x1=z1$, also $x=z$
	\end{enumerate}
	Also kann \emph{jede} endliche Gruppe als Untergruppe der $S_n$ verstanden werden ($n = \abs{G}$)!
\end{example}
\begin{example}
	\begin{enumerate}
		\item Setze $x \mapsto f_{x^{-1}}: G \to G$, $y \mapsto xyx^{-1}$ ist ein Morphismus $G \to \Ad(G)$ mit
		\begin{align*}
		f_x(y) = x^{-1}yx \quad f_z(f_x(y)) = z^{-1}(x^{-1}yx)z = (xz)^{-1}y(xz) = f
		\end{align*}
		\item $A$ nicht injektiv! Denke an $G$ abelsch $\Leftrightarrow f_x = \id_G \forall x \in G$
		\item $\sgn: S_n \to \Z_2 = \set{-1,1}$
	\end{enumerate}
\end{example}