\documentclass[ngerman,a4paper,order=firstname]{../../texmf/tex/latex/mathscript/mathscript}
\usepackage{../../texmf/tex/latex/mathoperators/mathoperators}

% % % local commands
\DeclareMathOperator{\Ad}{Ad}				% Adjoint
\DeclareMathOperator{\PSL}{PSL} 			% projective linear group 
%\newcommand{\with}{\text{ with }}
\renewcommand{\rhd}{\triangleright}
\renewcommand{\lhd}{\triangleleft} 			% normal subgroups
\DeclareMathOperator{\Set}{Set}				% Category of sets
\DeclareMathOperator{\Vect}{Vect}			% Category of vector spaces
\DeclareMathOperator{\UU}{U}				% unitary group
\DeclareMathOperator{\Grass}{Grass}			% Grassmanian
\DeclareMathOperator{\Grp}{Grp}				% Category of groups
\DeclareMathOperator{\Mod}{Mod}				% Cat of moduls
\DeclareMathOperator{\Ann}{Ann}				% annihilator
\newcommand{\Circlearrowleft}{\mathbin{\rotatebox[origin=c]{180}{$\circlearrowright$}}}
\DeclareMathOperator{\Cl}{Cl}				% conjugation class of something.
%\DeclareMathOperator{\Span}{span}

% get this stupid arrows:
%\usepackage{mathabx,graphicx}  % ---> add to mathoperators
%\def\Circlearrowleft{\ensuremath{%
%		\rotatebox[origin=c]{180}{$\circlearrowleft$}}}
%\def\Circlearrowright{\ensuremath{%
%		\rotatebox[origin=c]{180}{$\circlearrowright$}}}
%\def\CircleArrowleft{\ensuremath{%
%		\reflectbox{\rotatebox[origin=c]{180}{$\circlearrowleft$}}}}
%\def\CircleArrowright{\ensuremath{%
%		\reflectbox{\rotatebox[origin=c]{180}{$\circlearrowright$}}}}
%\begin{document}
%	\Huge
%	$\circlearrowleft \circlearrowright $
%	
%	$\Circlearrowleft \Circlearrowright $
%	
%	$\CircleArrowleft \CircleArrowright $

% % % local packages
\usepackage{braids}

\newlist{remarkenum}{enumerate}{1}
\setlist[remarkenum]{label=(\alph*),ref=\theremark~(\alph*)}
\crefalias{remarkenumi}{remark}

\newlist{propenum}{enumerate}{1}
\setlist[propenum]{label=(\alph*),ref=\theproposition~(\alph*)}
\crefalias{propenumi}{proposition}

\newlist{expenum}{enumerate}{1}
\setlist[expenum]{label=(\alph*),ref=\theexample~(\alph*)}
\crefalias{expenumi}{example}

\newlist{lemmaenum}{enumerate}{1}
\setlist[lemmaenum]{label=(\alph*),ref=\thelemma~(\alph*)}
\crefalias{lemmaenumi}{lemma}

\newlist{defenum}{enumerate}{1}
\setlist[defenum]{label=(\roman*),ref=\thedefinition~(\roman*)}
\crefalias{defenumi}{definition}

\title{\textbf{Geometrie WS 19/20}}
\author{Dozent: Prof. Dr. \person{Ulrich Krähmer}}

\begin{document}
\pagenumbering{roman}
\pagestyle{plain}

\maketitle

\hypertarget{tocpage}{}
\tableofcontents
\bookmark[dest=tocpage,level=1]{Inhaltsverzeichnis}

\pagebreak
\pagenumbering{arabic}
\pagestyle{fancy}

\chapter*{Vorwort}
Due to the coronavirus crisis, this is reading course only with online chats. Lecture notes were supplied from an older course, which we want to summarize here and add additional remarks to them. Proofs will be sketched with added remarks.
\chapter{Gruppen}
\section{Zentrale Fragestellung der Finanzmathematik}
\subsection*{\begriff{Bewertung}:}
Bewertung von Derivaten und \emph{Absicherung} gegen aus Kauf/Verkauf entstehenden Risiken.

\begin{*definition}[\begriff{Derivat}]
	FInanzprodukt, dessen auszahlungen sich vom Preis einer oder mehrer \begriff{Basisgüter} (underlying) ableitet (ableiten entspricht derivate)
\end{*definition}
\begin{*example}
	\begin{itemize}
		\item Recht, in 3 Monaten 100.000 GBP gegen 125.000 EUR zu erhalten (\begriff{Call-Option}, Underlying: Wechselkurs GBP/EUR)
		\item Recht, innerhalb des nächsten Jahres 100.000 Mwh elektrischer Energie zum Preis von 30EUR/Mwh zu konsumieren mit Mindestabnahme 50.000 Mwh (\begriff{Swing-Option}, Underlying: Strompreis)
		\item Kauf- und Verkaufsoptionen aus Aktien (Underlying: Aktienkurs)
	\end{itemize}
\end{*example}
Fragestellung: Was ist der ``faire'' Preis für solch ein Derivat? (``Pricing''/Bewertung). Wie kann sicher der Verkäufer gegen eingegangenen Risiken absichern? (``Hedging''/Absicherung)
\subsection*{\begriff{Optimale Investition}}
Zusammenstellung von Portofolios, welche nach Risiken/Ertragsgesichtspunkten optimal sind
\begin{itemize}
	\item Wie wäge ich Risiken gegen Ertrag ab?
	\item Was genau bedeutet ``optimal''?
	\item Lösung des resultierenden Optimierungsproblems
\end{itemize}
\subsection*{\begriff{Risikomangement + Risikomessung}}
\begin{itemize}
	\item Gesetzliche Vorschriften (Basel + Solvency) sollen Stabilität des Banken-/Verischerunssystems auch angesichts verschiedener Risiken sicherstellen $\implies$ mathematische Theorie der konvexen und kohärenten Risikomaße
\end{itemize}
Mathematische Werkzeuge: Wtheorie + stochastische Prozesse (Dynamik in der Zeit), etwas lineare Algebra, Optimierung, Maßtheorie
\section{Mathematisches Finanzmodell}
Wir betrachten
\begin{enumerate}
	\item \emph{WRaum} $(\O,\F,\P)$, später auch weitere W-Maße $Q, \dots$ auf demselben Maßraum $(\O,\F), \omega \in \O$ Elementarereignisse bzw. ``Szenarien''
	\item \emph{Zeitachse} $I$ entweder $I=\set{t_0, t_1, \dots, t_N=T}$ $N$-Periode Modell (diskretes Modell) oder $I = [0,T]$ (zeitstetiges Modell), wobei $T = $ Zeithorizont\\
	Ein \begriff{stochasticher Prozess} $S$ ist eine messbare Abbildung $S: (\O,\F) \to \Rd \mit (\omega, t) \mapsto S_t(\omega)$\\
	insbesondere ist 
	\begin{itemize}
		\item $t \mapsto S_t(\omega)$ Funktion $I \to \Rd$ für jedes $\omega \in \O$ (``Pfad'')
		\item $\omega \mapsto S_t(\omega)$ Zufallsvariable $\O \to \Rd$ für jedes $t \in I$
	\end{itemize}
	\item \emph{Filtration} ist Folge von $\omega$-Algebren $(\F_t)_{t \in I}$ mit der Eigenschaft $\F_S \subseteq \F_t \quad \forall s,t \in I, x \le t \und \F_t \subseteq \F\quad \forall t \in I$\\
	Interpretation: $\F_t=$dem Marktteilnehmer zum Zeitpunkt $t$ bekannte/ verfügbare Informationen\\
	Ereignisse $A \in \F_t$ gelten als ``zum Zeitpunkt $t$'' bekannt\\
	Eine $\Rd$-wertige ZV $X$ heißt \begriff{$\F_t$-messbar}, wenn $E = X^{-1}(B) \in \F_t \quad \forall$ Borelmengen $B \subseteq \Rd$ (dabei ist $E$ das Urbild von $B$)

\begin{*example}
	Ein stochastischer Prozess $(S_t)_{t\in I}$ auf $(\O,\F)$ heißt \begriff[stochasticher Prozess]{adaptiert} bezüglich einer Filtration $(\F_t)_{t \in I}$, wenn gilt:
	\begin{align*}
		S_t \text{ ist } \F_t-\text{messbar} \quad \forall t \in I
	\end{align*}
\end{*example}
Interpretation: ``der Wert $S_t$ ist zum Zeitpunkt $t$ bekannt''\\
Warum Filtration in der Finanzmathematik (FiMa)?
\begin{itemize}
	\item Unterscheidung Zukunft / Vergangenheit
	\item unterschiedliche Information (Insider/Outsider) entspricht unterschiedlicher Filtration $(\F_t)_{t \in I}$ bzw. $(\G_t)_{t\in I}$
\end{itemize}
	\item \begriff{Anlagegüter (assets)} $\R^{d+1}$-wertiger stochastischer Prozess mit Komponenten
	\begin{align*}
		S^i: (\O \times I) \to \R\quad (\omega,t) \mapsto S^i_t(\omega) \mit i \in \set{0,1,\dots,d}
	\end{align*} 
	wobei $S^i_t=$ Preis des $i$-ten Anlageguts zum Zeitpunkt $t$\\
	$S^i, i \in \set{1,\dots,d}$ ist typischerweise
	\begin{itemize}
		\item Aktie (Stock), Unternehmensanteil
		\item Währung (currency) bzw. Wechselkurs
		\item Rohstoff (commodity) wie z.B. Öl, Edelmetall, Elektriziät, etc
		\item Anleihe (bond) ... Schuldverschreibung
	\end{itemize}
	Hauptannahme: $S^i$ ist liquide gehandelt (z.B. an Börse), d.h. Kauf/Verkauf zum Preis $S^i_t$ jederzeit möglich\\
	$S^0\dots$ ``Numeraire'' hat Sonderrolle: beschreibt Verzinsung von \emph{nicht} in $(S^1,\dots,S^d)$ angelegten Kapital, wird meist \emph{risikolos} betrachtet
\end{enumerate}
\begin{definition}[Finanzmodell]
	Ein \begriff{Finanzmodell} (FMM) mit Zeitachse $I$ ist gegeben durch
	\begin{enumerate}
		\item einen WRaum $(\O, \F,\P)$ mit Filtration $(\F_t)_{t\in I}$
		\item einen an $(\F_t)_{t \in I}$ adaptieren, $\R^{d+1}$-wertigen stochastischen Prozess $S_t = (S^0_t, S_t^1, \dots, S^d_t),t \in I$
	\end{enumerate} 
\end{definition}
\begin{*example}[\person{Cox}-\person{Rubinstein} (CRR)-Modell (zeitdiskret)]
	\begin{itemize}
		\item $S^0_n = (1+r)^n$, d.h. Verzinsung mit konstanter Rate $r$
		\item $S^1_n = S_0^1 \prod_{k=1}^n(1+Ru)$, wobei $(R_1, R_2, \dots)$ unabhängig ZVen mit zwei möglichen Werten $a < b$\\
		Bild: ``rekonbinierter Baum'' mit Ereignissen $\omega$ entsprechen Pfaden in dem Baum
	\end{itemize}
\end{*example}
\begin{*example}[\person{Block}-\person{Scholes}-Modell (zeitstetig)]
	\begin{itemize}
		\item $S^0_t = e^{rt}$, d.h. Verzinsung mit konstanter Rate $r$
		\item $S_t^1 = S_0^1\cdot \exp((\mu - \frac{\sigma^2}{2}t + \sigma\beta_t) \mit \mu \in \R, \sigma > 0, S^1_0 >0$ und $\beta_t$ entspricht Brownscher Bewegung (stochastischer Prozess in stetiger Zeit) und $\mu - \frac{\omega^2}{2}$ entspricht Trendkomponente
	\end{itemize}
	Bild: Börsenkurve = $S_t(\omega)$, wobei zeitstetiges Modell auf unendlichen W-Raum
\end{*example}
\section{Anleihen und grundeliegende Beispiele für Derivate}
Hier betrachten wir immer nur ein Basisgut $S_t = S^1_t$
\begin{enumerate}
	\item \begriff{Anleihe}(\begriff{bond}): (genauer: Null-Coupon-Anleihe [zero-coupon-bond]) Der \begriff{Emittent} (Herausgeber) einer Anleihe mit Endfälligkeit $T$ [maturity] garantiert dem Käufer zum Zeitpunkt $T$ den Betrag $N$ (EUR/USD/...) zu zahlen.\\
	Typische Emittenten:
	\begin{itemize}
		\item Staaten [government bond]
		\item Unternehmen (als Alternative zur Kreditaufnahme)
	\end{itemize}
	Nach Emission werden Anleihen auf den Sekundärmarkt weiterverkauft, d.h. liquide gehandelte Wertpapiere\\
	Preis bei Emission: $B(0,T)$\\
	Preis bei Weiterverkauf zum Zeitpunkt $t \le T\colon B(t,T)$\\
	Wir normieren stets $N=1 \implies B(T,T) =1$\\
	Anleihen von West/Nord/Mitteleuropäischen Staaten + USA/Kananda werden als risikolos betrachtet (sichere Zahlung).\\
	Sonst: Kreditrisiko\\
	Risikofreie Anleihen können als Numerale $S^0_t = B(t,T)$ genutzt werden\\
	Bild: kann ich gerade nicht beschreiben :/ \\
	\item \begriff{Terminvertrag} [forward contract]\\
	Aus Käufersicht: \emph{Vereinbarung} zu bestimmten, zukünftigen Zeitpunkt $T$ eine Einheit des Basisguts $S$ zum Preis zu kaufen (Kaufverpflichtung)\\
	Beliebt bei Rohstoffen + Elektrizität\\
	Auszahlunsprofil: $F_T = S_T - K$\\
	Bild: ``Eine Gerade mit Schnittpunkt der $x$-Achse bei $K$ und Schnittpkt der $y$-Achse bei $S_T \ge 0$, ist ja nur einer Polynom 1. Ordnung''\\
	Preis zum Zeitpunkt $t$: $F_t$
	\item \begriff{Europäische Put-/Call-Option}:
	Recht zu einem zukünftigen Zeitpunkt $T$ eine Einheit des Basisguts $S$ zum Preis $K$ zu verkaufen (Put) bzw. zu kaufen (Call) \textbf{keine (Ver-)Kaufsverpflichtung}\\
	\begin{itemize}
		\item \emph{Call}:
		\begin{align*}
			C_T := \begin{cases}
				S_T - K &\quad S_T \ge K\\
				0 &\quad S_T < K
			\end{cases} = (S_T - K)_+ %is this really a ``+'' or a t?!
		\end{align*}
		\begin{*remark}
			\begin{align*}
				X_+ &= \max(X,0)\quad X_+ - X__ = X\\
				X__ &= \min(X,0)\quad X_+ + X__ = \abs{X}
			\end{align*}
		\end{*remark}
			Bild: (hockey stick function)
		\item \emph{Put}:
			\begin{align*}
				P_t = \begin{cases}
					0 &\quad S_T \ge K\\
					K-S_T &\quad S_t < K
				\end{cases} = (K-S_T)_+
			\end{align*}
			Bild: ``inversed'' hockey stick function xD
	\end{itemize}
	\item \emph{Amerikanische Put/Call-Option}: Wie Put/Call aber mit Ausübung zu beliebigen Zeitpunkt $t \in [0,T]$\\
	Preis zum Zeitpunkt $t\colon P_t^{AM}, \; C_t^{AM}$\\
	Auszahlungsprofil zum zeitpunkt $\tau\colon (S_{\tau}-K), (K-S_{\tau})_+$\\
	Zeitpunkt $\tau$ muss im Allgemeinen als Lösung eines stochastischen Optimierungsproblems bestimmt werden (``\begriff{Optimales Stopproblem}'')
\end{enumerate}
\section{Elementare Replikations und Arbitrage-Argumente}
Was können wir (mit elementaren Mitteln) über die ``fairen'' Preise $B(t,T), F_t, C_t, P_t$ aussagen?\\
Wir verwenden:
\begin{itemize}
	\item \begriff{Replikationsprinzip}: Zwei identische zukünftige Zahlungsströme haben auch heute denselben Wert. (ein Zahlungstrom ``repliziert'' den anderen) % but what about inflation? Where comes inflation in this whole system in? 
	\item \begriff{No-Arbitage-Prinzip}: ``Ohne Kapiteleinsatz kann sicherer Gewinn ohne Verlustrisiko erzielt werden''
	\item \begriff{Arbitrage}: risikofreier Gewinn\\
	\item Schwächere Form des Replikationsprinzips:\\
	\begriff{Superpositionsprinzip}: Ist ein Zahlungsstrom in jedem Fall größer als ein anderer, so hat er auch heute den größeren Wert
	\begin{align*}
		\begin{matrix}
			\text{stark} & \text{Rep. Prinzip} & \text{eingeschränkt anwendbar}\\
			\downarrow & \text{Superrep. Prinzip} & \uparrow\\
			\text{schwach} & \text{No-Arbitrage-Prinzip} & \text{immer anwendbar}
		\end{matrix}
	\end{align*}
\end{itemize}
\begin{lemma} %1.1 fix numbering later ;)
	Für den preis $C_t$ des europäischen Calls gilt:
	\begin{align*}
		(S_t - K\cdot B(t,T))_+ \le C_t \le S_t
	\end{align*}
\end{lemma}
\begin{proof}
	\begin{itemize}
		\item \emph{untere Schranke}: Für Widerspruch $S_t - K\cdot(B(t,T))-C_t = \epsilon > 0$\\
		\begin{tabular}{l|l|l|l} % jeezzz i fuckin hate tables in latex xD, but i managed it ...
			Portofolio & Wert in $t$ & Wert in $T$, $S_t \le K$ & Wert in $T$, $S_t > K$\\
			Kaufe Call & $C_t$ & 0 & $S_T - K$\\
			Verkaufe Basisgut & $-S_t$ & $-S_T$ & $-S_T$\\
			Kaufe Anleihe & $\epsilon + K\cdot B(t,T)$ & $\frac{\epsilon}{B(t,T)}+K$ & $\frac{\epsilon}{B(t,T)} + K$\\
			$\Sigma$ & 0 & $K - S_T + \frac{\epsilon}{B(t,T)} > 0$ & $\frac{\epsilon}{B(t,T)} > 0$\\
			& keine Anfangskapital & sicherer Gewinn & sicherer Gewinn\\
		\end{tabular}\\
		$\implies$ Widerspruch zu No-Arbitrage\\
		$\implies$ $S_t - K\cdot B(t,T) \le C_t$ und Ausserdem $C_t \ge 0 \implies C_t \ge (S_t - K\cdot B(t,T))_+$
		\item \emph{obere Schranke}: UE
	\end{itemize}
\end{proof}
\begin{lemma}[Put-Call-Parität] % jeezz its 5:48 am im typing this, cant sleep ... :(
	Für Put $P_t$, Call $C_t$ mit demselben Ausübungspreis $K$ und Basisgut $S_t$ gilt
	\begin{align*}
		C_t - P_t = S_t - B(t,T)K
	\end{align*}
	Bild: need to add ..., but should be fast to do ...
\end{lemma}
\begin{proof} %TODO fix tables later, for now it works ..., make them one size for better reading ...
	mit Replikation:\\
	\begin{tabular}{l|l|l|l} % jeezzz i fuckin know now how to make tables in latex xD
		Portofolio 1 & Wert in $t$ & Wert in $T$, $S_t \le K$ & Wert in $T$, $S_t > K$\\
		Kaufe Call & $C_t$ & 0 & $S_T - K$\\
		Kaufe Anleihe & $K \cdot B(t,T)$ & $K$ & $K$\\
		Wert Portofolio 1 & $C_t + K\cdot B(t,T)$ & $K$ & $S_T$\\
	\end{tabular}\\
	\newline
	\begin{tabular}{l|l|l|l} % jeezzz i fuckin hate, but i can copy ... xD
		Portofolio 2 & Wert in $t$ & Wert in $T$, $S_t \le K$ & Wert in $T$, $S_t > K$\\
		Kaufe Put & $P_t$ & $K-S_T$ & 0\\
		Kaufe Basisgut & $S_t$ & $S_T$ & $S_T$\\
		Wert Portofolio 2 & $P_t + S_t$ & $K$ & $S_T$\\
	\end{tabular}\\
	Replikationsprinzip: $C_t + K\cdot B(t,T) = P_t + S_t$\\
	$\implies$ $C_t - P_t = S_t - K\cdot B(t,T)$
\end{proof} % 5:56 am and im done with it, puh ;) hope we can talk about it in english if you like. ;)
\section{Bedingte Erwartungswerte und Martingale} %1.5
\subsection{Bedingte Dichte und bedingter Erwartungswert}
Motivation: Gegeben: Zwei ZVen $(X,Y)$ mit Werten in $\R^m \times \R^n$ und gemeinsame Dichte $f_{XY}(x,y)$. Aus $f_{XY}$ können wir ableiten:
\begin{itemize}
	\item $f_{Y}(y) := \int_{\R^m} f_{XY}(x,y) \d x$ mit Randverteilung von $Y$
	\item $S_Y := \set{y \in \Rn \colon f_Y(y) > 0}$ Träger von $Y$ - Bild?
\end{itemize}
\begin{*definition}[Bedingte Dichte von $X$ bezüglich $Y$]
	Bedingte Dichte von $X$ bezüglich $Y$ ist definiert als
	\begin{align*}
		f_{X\mid Y}(x,y) = \begin{cases}
		\frac{f_{XY}(x,y)}{f_Y(y)} &\quad y \in S_Y\\
		0 &\quad y\notin S_Y
		\end{cases}
	\end{align*}
\end{*definition}
Betrachte folgende Problemstellung:\\
Was ist die beste Vorhersage von $X$ gegeben einer Beobachtung $Y = y$?\\
Kriterium:\\
Minimiere quadratischen Abstand/ zweite Moment/ $L_2$-Norm.\\
Vorhersage:\\
Messbare Funktion $g: \Rn \to \R^m \mit y \mapsto g(y)$, d.h,.
\begin{align*}
	\min\set{\E[(X-g(Y))]^2 \colon g \text{ messbar } \R^n \to \R^m} \tag{min-1}\label{eq_min_1}
\end{align*}
\begin{proposition} %1.3
	Wenn $(X,Y)$ eine gemeinsame Dichte besitzen mit $\E[\abs{X}^2] < \infty$ gilt, dann wird \eqref{eq_min_1} minimiert durch die bedingte Erwartung
	\begin{align*}
		g(y) = \E[X\mid Y=y] := \int_{\R^m} x f_{X\mid Y}(x,y)\d x
	\end{align*}
	(wobei $\E[X\mid Y=y]$ ``Erwartungswert von $X$ bedingt auf $Y=y$'')
\end{proposition}
Allgemeiner gilt:
\begin{theorem} %1.4
	Seien $(X,Y)$ ZVen mit gemeinsamer Dichte auf $\R^m \times \Rn$, $h: \R^m \to \R^n$ messbar mit $\E[h(X,y)^2]$. Dann wird das Minimierungsproblem
	\begin{align*}
		\min\set{\E[(h(X,Y) - g(y))^2]} \quad g\text{messbar von $\Rn$ nach $\R$}
		\intertext{gelöst durch}
		g(y) = \E[h(X,Y) \mid Y=y] = \int_{\R^m} h(X,Y)f_{X\mid Y}(x,y) \d x
	\end{align*}
\end{theorem}
\begin{proof}[nur Prop, Theorem analog, für $n=1$]
	Setze $g(y) = \int_{\R} f_{X\mid Y}(x,y) \d x$. Sei $p: \R \to \R$ beliebige messbare Funktion mit $\E[p(y)^2] < \infty$. Setze $g_{\epsilon}(y) = g(y) + \epsilon p(y)$. Minimiere
	\begin{align*}
		F(\epsilon) &:= \E[(X-g_{\epsilon}(y))^2] = \E[(X-g(y)-\epsilon p(y))^2]\\
		&= \E[(X-g(y))^2] - 2\epsilon\E[(X-g(y))p(y)] + \epsilon^2\E[p(y)^2]\\
		\frac{\partial F}{\partial \epsilon}(\epsilon) &= 2 \epsilon \E[p(y)^2] - 2\E[(X-g(y))p(y)]\\
		&\implies \epsilon_{\ast} :=\frac{\E[(X-g(y))p(y)]}{\E[p(y)^2]} = \frac{A}{B}
		\intertext{wobei}
		A&= \E[Xp(y)] -\E[g(y)p(y)] \\
		&= \int_{\R \times \Rn} xp(y)f_{XY}(x,y)\d x \d y - \int_{S_y}g(y)p(y)f_Y(y) = [\text{Einsetzen von $g$ + Fubini}]\\
		&= \int_{\R \times \Rn} xp(y)f_{XY}(x,y)\d x \d y - \int_{\R\times S_y} xp(y)\underbrace{f_{X\mid Y}(x,y)f_Y(y\d y)}_{=f_{XY}(x,y)} = 0
	\end{align*}
	also $\epsilon^{\ast} = 0$ unabhängig von $p$ $\implies g(y)$ minimiert \eqref{eq_min_1}.
\end{proof}
\begin{*example}
	Seien $(X,Y)$ normalverteilt auf $\R\times \R$ mit 
		\begin{align*}
			\mu = (\mu_x, \mu_y)^T \quad \Sigma = \begin{pmatrix}
			\sigma x^2 \rho\sigma_x \sigma_y\\
			\rho \sigma_x \sigma_y & \sigma_y^2
			\end{pmatrix} = \begin{pmatrix}
				\Var(X) & \Cov(X,Y)\\
				\Cov(X,Y) & \Var(Y)
			\end{pmatrix} \mit \rho \in [-1,1]
		\end{align*}
		Dann ist die beliebige Dichte $f_{X\mid Y}(x,y)$. ($\Sigma$ Kovarianzmatrix). wieder die Dichte einer Normalverteilung mit
		\begin{align*}
			\E[X \mid Y=y] &= \mu_x + \rho \frac{\sigma_x}{\sigma_y}(y-\mu_y)\\
			\Var(X\mid Y=y) &= \sigma_x^2(1-\rho^2)
		\end{align*}
		(ist ÜA!). Die Abbildung $y \mapsto \mu_x + g(y)\frac{\sigma_x}{\sigma_y}(y-\mu_y)$ heißt Regressionsgerade für $X$ gegeben $Y=y$.\\
		Bild: $\mu_x,\mu_y$ sind Werte auf $x,y$-Achse und die $\sigma$'s bilden das Steigungsdreieck (Steigung im Wesentlichen durch $\rho$ bekannt)\\
		Für diskrete ZVen, d.h. wenn $X,Y$ nur endlich viele $\set{x_1,\dots,x_m}$ bzw. $\set{y_1,\dots,y_m}$ annehmen dann erhalten wir mit ähnlichen Überlegungen als Lösung von \eqref{eq_min_1}
		\begin{align*}
			\E[X\mid Y=y_j] = \sum_{i=1}^m X_i \P(X=x_i \mid Y=y_j)
		\end{align*}
		wobei direkt die bedingten Wahrscheinlichkeiten
		\begin{align*}
			\P(X=x_i \mid Y=y_j) = \begin{cases}
			\frac{\P(X=x_i \wedge Y=y_j)}{\P(Y=y_j)} &\quad \text{ wenn } \P(Y=y_j) > 0\\
			0 &\quad \text{ wenn } \P(Y=y_j) = 0 
			\end{cases}
		\end{align*}
\end{*example}
\subsection{Bedingte Erwartung - maßtheoretischer Zugang}
Wir betrachten WRaum $(\O, \F,\P)$. Für ZV $X: \O \to \R$ und $p \in [1,\infty)$ definieren wir die $L_p$-Norm
\begin{align*}
	\norm{X}_p = \E[\abs{X}^p]^{1/p} = \brackets{\int_{\O} \abs{X(\omega)}^p \d \P(\omega)}^{1/p}
\end{align*}
und $L_p$-Raum $L_p(\O,\F,\P):= \set{X: \O \to \R\colon \F-\text{messbar}, \norm{X}_p < \infty}$. Dabei identifzieren wir ZVen, die sich nur auf Nullmengen unterscheiden, d.h. $\P(X \neq X') = 0 \implies X = X'$ (in $L_p$).\\
Aus Maßtheorie bekannt: (?)\\
Die Räume $L_p(\O,\F,\P)$ mit Norm $\norm{\cdot}_p, p \in [1,\infty)$ sind stets \person{Banach}-Räume (lineare, vollständig, normierte Vektorräume). Für $p = 2$ auch Hilbertraum mit inneren Produkt
\begin{align*}
	\scaProd{X}{Y} = \E[XY] = \int_{\O} X(\omega)Y(\omega)\d \P(\omega)
\end{align*}
Für $\G \subseteq \F$ Unter-$\sigma$-Algebra ist $L_p(\O,\F,\P) \subseteq L_p(\O,\F,\P)$ abgeschlossen Unterraum.\\
Wir verallgemeinern ``Vorhersageproblem'' aus dem letzten Abschnitt (1.3?)\\
Gegeben ZVe $X$ aus $L_2(\O,\F,\P)$ ist $\G \subseteq \F$ Unter-$\sigma$-Algebra.\\
Was ist die beste $\G$-messbare Vorhersage für $Y$?
\begin{align*}
	\min\set{\E[(X-G)^2] \colon G \in L_2(\O,\F,\P)} \tag{min-2}\label{eq_min_2}
\end{align*}
wobei $\E[(X-G)^2] = \norm{X-G}^2_2$.\\
Aus Hilbertraumtheorie:\\
\eqref{eq_min_2} besitzt eine eindeutige Lösung $G_{\ast} \in L_2(\F,\G,\P)$. $G_{\ast}$ ist Optimierung (bezüglich $\scaProd{\cdot}{\cdot}$) von $X \in L_2(\O,\F,\P)$ auf abgeschlossenen Unterraum $L_2(\O,\G,\P)$\\
Bild: eventuell von Eric (Orthogonal Projektion auf den Unterraum)\\
Wir bezeichnen mit $G_{\ast}$ mit $\E[X\mid \G]$ bedingte Erwartungswert von $X$ bezüglich $\G$.
\begin{theorem}[Eigenschaften bedingter Erwartungswert] %1.5
	\label{1_5_eigen_bedEW}
	Seien $X,Y \in L_2(\O,\F,\P)$ und $\G \subseteq F$ Unter-$\sigma$-Algebra. Dann gilt
	\begin{enumerate}
		\item (Linearität) $\E[aX+bY] = a\E[X\mid \G] + b\E[Y\mid \G]$
		\item (Turmregel) Für jede weitere $\sigma$-Algebra $\H \subseteq\G$ gilt
		\begin{align*}
			\E[E[X\mid \G \mid \H]] = \E[X\mid \H]
		\end{align*}
		\item (Pullout-Property) $\E[XZ\mid \G] = Z\E[X\mid \G]$, wenn $Z$ beschränkt und $\G$-messbar ist.\\
		zweite Version: Für $Z$ $\G$-messbar mit $\E[\abs{XZ}] < \infty$ gilt:
		\begin{align*}
			\E[XZ\mid \G] = Z \cdot \E[X\mid \G]
		\intertext{insbesondere gilt}
			X \G\text{-messbar }\implies \E[X\mid \G] = X
		\end{align*}
		\item (Monotonie) $X \le Y \implies \E[X\mid \G] \le \E[Y \mid \G]$
		\item ($\Delta$-Ungleichung) $\abs{\E[X\mid \G]} \le \E[\abs{X}\mid \G]$
		\item (Unabhängigkeit) $X$ unabhängig von $G$ $\implies$ $\E[X \mid \G] = \E[X]$
		\item (triviale $\sigma$-Algebra) $\G=\set{\emptyset, \O} \implies \E[X \mid \G] = \E[X]$ 
	\end{enumerate}
\end{theorem}
\begin{proof}
	(ohne Beweis, siehe VL W-Theorie mit Martingalen oder auch STOCH-Skript SS19.)
\end{proof}
\begin{*remark}
	\begin{itemize}
		\item Die für $X \in L_2(\O,\F,\P)$ definierte vedingte Erwartung $\E[X\mid \G]$ lässt sich durch Approximation auf alle $X\in L_1(\O,\F,\P)$ erweitern. Alle Eigenschaften aus Theorem \propref{1_5_eigen_bedEW} bleiben erhalten!
		\item Sei $Y$ eine ZVe und $\G = \sigma(Y)$ die von $Y$ erzeugte $\sigma$-Algebra. Wir schreiben:
		\begin{align*}
		\E[X\mid Y] = \E[X \mid \sigma(Y)] \quad \sigma\text{-messbare ZVe}
		\end{align*}
		\item Maßtheorie: \person{Doob}-\person{Dynkin}-Lemma $\implies \exists$ messbare Funktion $g: \Rn \to \R$ sodass
		\begin{align*}
		\E[X\mid Y] = g(Y)
		\end{align*}
		Dabei ist $g$ genau die Funktion aus \eqref{eq_min_1}.
	\end{itemize}
\end{*remark}
Zusammenfassung:\\
Sei $X,Y$ aus $L_1(\O,\F,\R)$, $\G \subseteq \F$ Unter-$\sigma$-Algebra
\begin{enumerate}
	\item $\E[X\mid Y=y]$ ist messbare Funktion $g: \Rn \to \Rn$. Falls bedingte Dichte existiert, gilt:
	\begin{align*}
		\E[X\mid Y=y] = \int_{\R^m} f_{X\mid Y} (x,y) \d x
	\end{align*}
	\item $\E[X\mid Y]$ ist eine $\sigma(y)$-messbare ZVe, diese kann als $g(Y)$ dargestellt werden. Falls bedingte Dichte existiert, gilt
	\begin{align*}
		\E[X\mid Y](\omega) = \int_{\Rn}xf_{X\mid Y}(x,Y(\omega))\d x
	\end{align*}
	\item $\E[X \mid \G]$ ist eine $\G$-messbare ZVe. Falls $\G = \sigma(y)$ tritt 2) ein.
\end{enumerate}
In allgemeinen Fall kann $\bar{\E[X\mid \cdot]}$ interpretiert werden als \emph{beste Vorhersage} für $X$, gegeben
\begin{enumerate}
	\item punktweise Beobachtung $Y=y$
	\item Beobachtung $Y$
	\item Information $\G$
\end{enumerate}
\subsection{Martingale}
Prototyp eines ``neutralen'' stochastischen Prozesses,der weder Aufwärts- noch Abwärtstrend besitzt. Hier nur in diskrete Zeit $Z = \N_0$.
\begin{*definition}[Martingal ohne Filtration]
	Sei $(X_n)_{n\in \N_0}$ stochastischer Prozess. Wenn gilt
	\begin{enumerate}
		\item $\E[\abs{X_n}] < \infty$ $\forall n \in \N$
		\item $\E[X_{n+1},\dots, X_n] = X_n$ $\forall n \in \N$
	\end{enumerate}
	dann heißt $(X_n)$ \begriff{Martingal}. Wen wir $\F_n^{\ast} = \sigma(X_1,\dots,X_n)$ definieren, können wir 2) schreiben als
	\begin{align*}
		\E[X_{n+1} \mid \F_n^{\ast}] = X_n \quad \forall n \in \N
	\end{align*}
\end{*definition}
Interpretation:\\
\begin{itemize}
	\item Beste Vorhersage für zukünftigen Wert $X_{n+1}$, basierend auf Vergangenheit $\sigma(X_1,\dots,X_n)$ ist der momentane Wert $X_n$.
	\item Aus der Turmregel folgt
	\begin{align*}
		\E[X_{n+k} \mid \F_n^{\ast}] &= X_n \quad n,k \in \N_0
		\intertext{denn}
		\E[X_{n+k}\mid\F_n^{\ast}] &= \E[\E[X_{n+k}\mid \F_{n+k-1}\mid \F_n^{\ast}]] = \E[X_{n+k-1}\mid \F_n^{\ast}] = (k\text{-mal}) = X_n
	\end{align*}
\end{itemize}
Kann von $(\F_{n})_{n \in \N}$ auf beliebige Filtrationen $(\F_n)_{n \in \N_0}$ erweitert werden.
\begin{*definition}[Martingal mit Filtration]
	Sei $(X_n)_{n \in \N_0}$ ein stochastischer Prozess, adaptiert an eine Filtration $(\F_n)_{n \in \N_0}$. Wenn gilt
	\begin{enumerate}
		\item $\E[\abs{X_n}] < \infty$ $\forall n \in \N_0$
		\item $\E[X_{n+1} \mid \F_n] = X_n$ $\forall n \in \N_0$
	\end{enumerate}
	dann heißt $(X_n)_{n \in \N_0}$ \begriff{Martingal bezüglich Filtration} $(\F_n)_{n \in \N_0}$
\end{*definition}
Interpretation:\\
Beste Vorhersage für zukünftige Werte $X_{n+1}$, basierend auf verfügbarer Information $\F_n$ ist momentane Wert $X_n$.
\begin{*definition}[Supermartingal, Submartingal]
	Falls in Punkt 2) statt ``$=$'' die Ungleichung $\le \oder \ge$ gilt, so heißt $(X_n)_{n \in \N}$ \begriff{Supermartingal} bzw. \begriff{Submartingal}.
\end{*definition}
Erste Beobachtung:\\
\begin{itemize}
	\item $X$ Martingal $\implies \E[X_n] = X_0$, d.h. $n \mapsto \E[X_n]$ ist konstant.\\
	Begründung:
	\begin{align*}
		\E[X_{n+1} \mid \F_n] = X_n \implies \E[\E[X_{n+1}\mid \F_n]] = \E[X_n] = \E[X_{n+1}] \implies (n\text{-mal Anwendung } \E[X_n] = X_0)
	\end{align*}
	Bild: Erwartungswert konstant, aber kein Martingal.
	\item $X$ Submartingal $\implies n \mapsto \E[X_n]$ ist monoton steigend
	\item $X$ Supermartingal $\implies n \mapsto \E[X_n]$ ist monoton fallend
\end{itemize}
Um sich den Unterschied zwischen Super- und Submartingal zu merken, hier eine kleine Hilfe:\\
``Das leben ist ein Supermartingal, die Erwartungen fallen mit der Zeit.''
\begin{*example}
	\begin{itemize}
		\item Seien $(Y_n)_{n\in \N}$ unabhängige ZVen in $L_1(\O,\F,\P)$ mit $\E[Y_n] = 0$. Definiere $X_n := \sum_{k=1}^n Y_k \mit X_0 = 0$. Dann ist $(X_n)_{n \in \N_0}$ Martingal, denn
		\begin{enumerate}
			\item $\E[\abs{X_n}] \le \sum_{k=1}^n \E[\abs{Y_k}] < \infty \quad \forall n \in \N$ \checkmark
			\item
			\begin{align*}
				\E[X_{n+1} \mid \F_n^{\ast}] &= \E[Y_{n+1} + X_n \mid \F_n^{\ast}]\\
				&= \E[Y_{n+1} \mid \F_n^{\ast}] = \E[X_n \mid \F_n^{\ast}] \quad (\text{ Turm und $\F_n^{\ast}$-messbar})\\
				&= \underbrace{\E[Y_{n+1}]}_{=0} + X_n = X_n \checkmark
			\end{align*}
		\end{enumerate}
		\item weitere Beispiele auf dem ersten Übungsblatt!
	\end{itemize}
\end{*example}
\begin{*definition}[vorhersehbar]
	Sei $(\F_n)_{n\in \N_0}$ eine Filtration. Ein stochastischer Prozess $(X_n)_{n \in \N}$ heißt \begriff{vorhersehbar} (predictable) bezüglich $(\F_n)_{n \in \N_0}$, wenn gilt:
	\begin{align*}
		H_n \text{ ist } \F_{n-1}\text{-messbar} \quad \forall n \in \N
	\end{align*}
\end{*definition}
\begin{*remark}
	Stärkere Eigenschaft als ``adaptiert''.
\end{*remark}
\begin{*definition}[diskretes stochastische Integral]
	Sei $X$ adaptierter und $H$ ein vorhersehbarer stochastischer Prozess bezüglich $(\F_n)_{n \in \N}$. Dann heißt
	\begin{align*}
		(H \bigcdot X)_n := \sum_{k=1}^n H_k (X_k - X_{k-1}) \tag{$\ast$}\label{eq_pred_stoch_process}
	\end{align*}
	\begriff{diskretes stochastische Integral} von $H$ bezüglich $X$.
\end{*definition}
\begin{*remark}
	Summe \eqref{eq_pred_stoch_process} heissen in der Analysis \person{Riemann}-\person{Stieltjes}-Summen. Werden für Konstruktionen des RS-Integrals $\int h \d \rho$ verwendet.
\end{*remark}
\begin{*definition}[lokal beschränkt]
	Ein stochastischer Prozess $(H_n)_{n \in \N}$ heißt \begriff{lokal beschränkt}, wenn eine (definierte) Folge $c_ \in \R_{\ge 0}$ existiert, sodass
	\begin{align*}
		\abs{H_n} \le c_n \text{ f.s. } \quad \forall n \in \N
	\end{align*}
\end{*definition}
\begin{theorem}
	Sei $X$ adaptiert stochastischer Prozess (bezüglich Filtration $(\F_n)_{n \in \N}$). Dann sind äquivalent:
	\begin{enumerate}
		\item $X$ ist Martingal
		\item $(H \bigcdot X)$ ist Martingale für alle lokal beschränkten, vorhersehbaren $(H_n)_{n \in N}$
	\end{enumerate}
	Das heisst: stochastische Integral erhält die Martingal-Eigenschaft.
\end{theorem}
\begin{*remark}
	Die ZV $H$ wird später die Anlagestrategie sein.
\end{*remark}
%\begin{proof}
%	\begin{itemize}
%		\item $\Rightarrow$:
%		\begin{itemize}
%			\item Adaptiertheit: klar
%			\item Integrierbarkeit: $H$ lokal beschränkt, d.h. $\abs{H_k} \le c_k < \infty$ für alle $k$.
%			\begin{align*}
%			\EW[\abs{H_k (X_k - X_{k-1})}] \le c_k * \brackets{\EW[\abs{x_k}] + \EW[\abs{X_{k+1}}]} < \infty
%			\end{align*}
%			Mit der Dreiecksungleichung folgt daraus $\EW[\abs{(H \bigcdot X)_n}] < \infty$.
%			\item Martingaleigenschaft: 
%			\begin{align*}
%			\EW[(H \bigcdot X)_n \mid \F_{n-1}] &= (H \bigcdot X)_{n-1} + \EW[H_n (X_n - X{n-1}) \mid \F_{n-1}]\\
%			&=  (H \bigcdot X)_{n-1} + H_n * \underbrace{\brackets{\EW[X_n \mid \F_{n-1}] - X{n-1}}}_{=0}\\ 
%			&=(H \bigcdot X)_{n-1} \quad \forall n \in \N
%			\end{align*}
%			Damit ist also auch $(H \bigcdot X)$ ein Martingal.		
%		\end{itemize}
%	\end{itemize} $(H \bigcdot X)_n = \sum_{k=1}^n H_k (X_k - X_{k-1})$.
%	\item $\Leftarrow$: Fixiere $N \in \N$. Setze $H_n := \indi_{n = N}$, dieser ist lokal beschränkt und deterministisch (also auch vorhersehbar). Man stellt fest, dass $(H \bigcdot X)_n = 0$ für alle $n \le N-1$. Für alle $n \ge N$ gilt dagegen $(H \bigcdot X)_n = X_N - X_{N-1}$. Wir überprüfen nur die Martingaleigenschaft (Integrierbarkeit folgt aus Dreiecksungleichung). Wir wissen, dass $(H \bigcdot X)$ ein Martingal ist. 
%	\begin{align*}
%	0 &= (H \bigcdot X)_{N-1} = \E[(H \bigcdot X)_N \mid \F_{N-1}]\\
%	 &= \E[x_N - X_{N-1} \mid \F_{N-1}]\\ 
%	 &= \E[X_N \mid \F_{N-1}] - X_{N-1}\\
%	&\implies X_{N-1} = \E[X_N \mid \F_{N-1}] \mit N \in \N \text{ beliebig}
%	\end{align*}
%	Somit ist $X$ ein Martingal.
%\end{proof}
%\begin{conclusion} %1.7
%	Sei $X = \set{X_n}{n=1 , \dots, N}$ ein adaptierter stochastischer Prozess bezüglich einer Filtration \\$\set{\F_n}{n=1 , \dots, N}$. Wenn $\E[(H \bigcdot X)_N] = 0$ für alle lokal beschränkten vorhersehbaren Prozesse $H$, dann ist $X$ ein Martingal bezüglich $\set{\F_n}{}$.
%\end{conclusion}
%\begin{proof}
%	Fixiere $K \in  [N] := \set{1, 2, \dots , N}$ und eine Menge $A \in \F_{K-1}$. Definiere $H_n(\omega) = \indi_A (\omega) * \indi_{\set{n=K}}$, dieser ist lokale beschränkt und vorhersehbar.
%	Es ist $(H \bigcdot X)_n = 0$ für alle $n \le K-1$. Für alle $n \ge K$ gilt $(H \bigcdot X)_n = \indi_A * (X_K - X_{K-1})$. 
%	\begin{align*}
%		0 &= \E[(H \bigcdot X)_N] = \E[\indi_A (X_K - X_{K-1})]\\ 
%		\overset{Turm}&{=} \E[\E[\indi_A (X_K - X_{K-1}) \mid \F_{K-1}]]\\ 
%		&= \E[\indi_A * \brackets{\underbrace{\E[X_K \mid \F_{K-1] - X_{K-1}}]}_{ Y_{K-1}}} \quad \forall A \in \F_{K-1}\\
%		&\implies \int_A  Y_{K-1}(\omega) \d{\P(\omega)}\\ 
%		&= \int_A X_{K-1}(\omega) \d{\P(\omega)} \quad \forall A \in \F_{K-1}\\
%		&\implies Y_{K-1} = X_{K-1} \text{ fast sicher }\\
%		&\implies \E[X_K \mid \F_{K-1] - X_{K-1}}] = X_{K-1}
%	\end{align*}
%	für beliebige $K$. Somit ist $X$ ein Martingal.
%\end{proof}
%\begin{*remark}
%	Wir schreiben $[N] := \set{1, 2, \dots, N}$ und $[N]_0 := \set{0, 1, 2 , \dots , N}$.
%\end{*remark}
\section{Permutationsdarstellungen und Gruppenoperationen}
\subsection*{Allgemeines}
Motivation: $H < G \implies \lnkset{G}{H}$\\
Notation: $x,y \in G$ konjugiert $\Leftrightarrow \exists g \in G \colon g^{-1}yg$ $\Leftrightarrow \exists h \in G\colon x = hyh^{-1}$
\begin{definition}
	Sei $G$ Gruppe, $X \neq \emptyset$ Menge, $S_X$ Permutationsgruppe von $X$. Dann
	\begin{enumerate}
		\item Eine \begriff{Permutationsdarstellung} ist ein Gruppenmorphismus
		\begin{align*}
			\theta\colon G \to S_X
		\end{align*}
		\item Eine \begriff{(linke) Operation} von $G$ auf $X$ ist eine Abbildung
		\begin{align*}
			G\times X \to X \mit (g,x) \mapsto g\cdot x
		\end{align*}
		so dass: ($\forall x \in X, \forall g,h \in G$)
		\begin{itemize}
			\item $g \cdot x = x$
			\item $(g\cdot h)\cdot x = g \cdot (h\cdot x)$
		\end{itemize}
	\end{enumerate}
\end{definition}
\begin{proposition}
	Es gibt eine bijektive Korrespondenz zwischen den Operationen von $G$ auf $X$ und den Darstellungen von $G$ als Permutationen von $X$.
\end{proposition}
\begin{proof}
	\begin{itemize}
		\item Sei $\theta\colon G \to S_X$ eine Permutationsdarstellung. Definiere $G\times G \to X$, wobei $g \cdot x := \theta(g)(x)$
		\begin{align*}
			1 \cdot x = \theta(1)(x) = \id_X(x) = x\\
			(gh)\cdot x = \theta(gh)(x) = \theta(g)\cdot\theta(h)(x) = \theta(g)(h\cdot x) = g(h\cdot x) \quad g,h \in G
		\end{align*}
		ist Operation von $G$ auf $X$.
		\item Sei $G \times X \to X \mit (g,x) \mapsto g x$ eine Operation. Für jedes $g \in G$: $\theta(g):= x \mapsto g \cdot x$ und damit haben wir $\theta\colon G \to \Set(X,X)$. Sei $g,h \in G$ und $x \in X$
		\begin{align*}
			\theta(gh)(x) &= (gh)\cdot x\\
			&= g \cdot (h \cdot x)\\
			&= g\cdot (\theta(g)(x))\\
			&= \theta(g)(\theta(h)(x))\\
			&= \theta(g) \cdot \theta(h)(x)
		\end{align*}
		also gilt $\theta(gh) = \theta(g)\cdot \theta(h)$.
		\begin{align*}
			\theta(1)(x) = 1 \cdot x = x \quad \forall x \in X \implies \theta(1) = \id_X
		\end{align*}
		also $\theta$ Morphismus von Monoide.
		\begin{align*}
			\forall g \in G \colon \theta(g)\cdot \theta(g^{-1}) = \theta(g\cdot g^{-1}) = \theta(1) = \id_X = \theta(g^{-1}) \cdot \theta(g^{-1})
		\end{align*}
		und wir haben $\theta(g)$ bijektiv mit Inverse $\theta(g^{-1})$ und damit $\theta\colon G \to S_X$
	\end{itemize}
\end{proof}
\begin{*example}
	Setze Notation: $G \Circlearrowleft X$ $G$ operiert auf $X$.
	\begin{enumerate}
		\item $X \neq \emptyset$ Menge $\forall G < S_X \implies G$ operiert natürlich auf $X$.
		\item $D_n$ operiert auf $P_n$ (reguläre Polygone mit $n$ Seiten)
		\item $V$ Vektorraum $\implies$ $\GL(V)$ operiert auf $V$
		\item $G \Circlearrowleft X \implies$
		\begin{itemize}
			\item $G \Circlearrowleft X^n \mit (x_1, \dots, x_n) \in X^n$ und $g\colon (x_1, \dots, x_n) := (g\cdot x_1, \dots, g \cdot x_n)$
			\item $G \Circlearrowleft \powerset(X)$ und $A \subseteq X$, sowie $g \cdot A = \set{g\cdot a \mid a \in A}$
		\end{itemize}
		\item $H <G \implies G \Circlearrowleft \lnkset{G}{H}$ und $aH$ mit $g(aH) = (ga)H$
	\end{enumerate}
\end{*example}
\subsection*{Morphismen}
\begin{definition}
	Ein Morphismus zwischen zwei Operationen $(G,X)$ und $(H,Y)$ ist ein Paar $(\phi, \alpha)$, wobei
	\begin{itemize}
		\item $\phi\colon G \to H$ Gruppenhomomorphismus
		\item $\alpha\colon X \to Y$ Abbildung
		\item $\forall g \in G \und x \in X\colon$ $\alpha(g \cdot x) = \phi(g)\cdot \alpha(x)$
		\[
			\begin{tikzcd}
			X \arrow[r, "\alpha"] \arrow[d, "g"] & Y \arrow[d, "\phi(g)"] \\
			X \arrow[r, "\alpha"]                & Y                     
			\end{tikzcd}
		\] 
	\end{itemize}
\end{definition}
\begin{*example}
	Sei $G=\Z \Circlearrowleft \Z_n=X, n \neq 0, \forall g \in \Z, \overline{x} \in \Z_n\colon g \cdot \overline{x} = \overline{g+x}$ ist \emph{nicht treu}, da $\forall g \in \Z \colon (g+n)\cdot \overline{x} = \overline{g+n+x} = g \cdot \overline{x}$
	\[
		\begin{tikzcd}
		G \arrow[r, "\theta"] \arrow[d]                       & S_X \\
		G/\ker \theta \arrow[ru, "\overline{\theta}", dashed] &    
		\end{tikzcd}
	\]
	also $\overline{\theta}$ injektiv $\implies \lnkset{G}{\ker \theta} \Circlearrowleft X$ treu.\\
	Für $\Z \Circlearrowleft \Z_n \colon \ker \theta = n\Z \implies \Z_n \Circlearrowleft \Z_n$ treu, da
	\begin{align*}
		\overline{x}, \overline{y} \in \Z_n\quad k\in \Z, \text{ so dass } \overline{k} = \overline{x-y} \in \Z_n\\
		k \cdot \overline{y} = \overline{k+y} = \overline{x-y+y} = \overline{x}
	\end{align*}
	Sei $n \in \N, n \ge 1$ und $\Z \Circlearrowleft \Z_n := \set{\overline{0}, \overline{1}, \overline{2}, ..., \overline{n-1}}$, also $D_n \Circlearrowleft P_n$.
	\begin{align*}
		x \in \Z \colon x \cdot \overline{y} := \overline{x+y}\\
		\alpha\colon \Z_n \to P_n\\
		\phi\colon \Z \to D_n
	\end{align*}
	(Als Beispiel kann man sich die $P_5$ nehmen und aufmalen ;))
	Definiere $\phi(1)$. Drehung von Zentrum $Z$ um den Winkel $\sfrac{2\pi}{n}$.
	\begin{align*}
		\alpha(1 \cdot \overline{0}) = \alpha(\overline{1})\\
		\phi(1)\cdot \alpha(\overline{0}) = \alpha(\overline{1})\\
		\implies (\phi,\alpha)\colon \text{ Morphismus}
	\end{align*}
\end{*example}
\begin{definition}
	\begin{itemize}
		\item $H = G, \phi = \id_G$, also ist $G$-Morphismus
		\item $(\phi, \alpha)$ ist ein Isomorphismus, wenn $\phi$ Gruppenmorphismus und $\alpha$ Bijektion
	\end{itemize}
\end{definition}
\subsection*{Bahnen}
\begin{lemma}
	Sei $G \Circlearrowleft X$. Definiere eine Relation $\sim$ auf $X$:
	\begin{align*}
		\forall x,y \in X \colon x \sim y \Leftrightarrow \exists g \in G\colon y = y \cdot x
	\end{align*}
	Dann ist $\sim$ eine Äquivalenzrelation.
\end{lemma}
\begin{proof}
	\begin{itemize}
		\item $x \sim x$, da $x = 1 \cdot x$
		\item $x \sim y \implies y \sim x$, da
		\begin{align*}
			y = y\cdot x &\implies g^{-1}y = y^{-1}(gx)\\
			&\implies g^{-1}y = (g^{-1}y)x\\
			& \implies g^{-1}y = x
		\end{align*}
		\item $x \sim y$ und $y \sim z \implies x \sim z$, da
		\begin{align*}
			y = g\cdot x \und z = h \cdot y \implies z = h(g\cdot x) = (hg)\cdot x
		\end{align*}
	\end{itemize}
\end{proof}
\begin{definition}
	\begin{itemize}
		\item Für alle $x \in X$ \begriff{Äquivalenzklassen} von $X$: $G\cdot x := \set{gx \mid g \in G}$.
		\item $Gx$ wird \begriff{Bahn von $x$} genannt und $\abs{Gx}$ die \begriff{Länge} von $Gx$.
		\item $\forall x,y \in X$ entweder $Gx = Gy$ oder $Gx \cap Gy = \emptyset$:
		\begin{align*}
			X = \bigcup_{x \in X}G \cdot x
		\end{align*}
	\end{itemize}
\end{definition}
\begin{*example}
	Sei $G=(\R,+)$ und $X = \C$:
	\begin{enumerate} % picture from Florian
		\item Translation: $a \in \C\setminus \set{0}$ und $\forall \lambda \in \R, z \in \C\colon \lambda \cdot z := z + \lambda a$ (nicht transitiv). Ist \emph{treu}, da
		\begin{align*}
			\forall \lambda_1, \lambda_2 \in \R, z \in \C\colon &z + \lambda_1 a = z + \lambda_2 a\\
			&\Leftrightarrow z + \lambda_1 a = z + \lambda_2 a\\
			&\Leftrightarrow (\lambda_1-\lambda_2)a = 0\\
			&\Leftrightarrow \lambda_1 = \lambda_2
		\end{align*}
		\item Drehungen: $\forall \lambda \in \R, z \in \C$ und damit $\lambda \cdot z = e^{2\pi\ii \lambda}z$ (nicht transitiv). Ist \emph{nicht treu}, da 
		\begin{align*}
			\forall \lambda \in \R, \forall k \in \Z\colon \lambda \cdot z = (\lambda + k) \cdot z,\\
			\ker\theta = \Z \implies S^1 = \lnkset{\R}{\Z} \Circlearrowleft \C
		\end{align*}
	\end{enumerate}
\end{*example}
\subsection*{Stabilisator}
Sei $G \Circlearrowleft X$ und $x \in X$.
\begin{definition}
	Definiere $G_x := \set{g \in G \mid g \cdot x = x}$ als \begriff{Stabilisator} von $x$.
\end{definition}
\begin{lemma}
	Es gilt $G_x < G$.
\end{lemma}
\begin{proof}
	\begin{itemize}
		\item $1 \in G_x\colon 1 \cdot x = x$
		\item $\forall g,h \in G_x \implies h \in G_x$ und $(gh)x = g(hx) = gx = x$
		\item Sei $g \in G_x \implies g^{-1} \in G_x$: $gx = x \implies g^{-1}(gx) = g^{-1}x$ und damit $x = 1 x = (g^{-1}g)x$
	\end{itemize}
\end{proof}
... additions to previous example is missing, get from florian :(
\begin{lemma}
	$G \Circlearrowleft X$ und $\theta\colon G \to G_X$ assozierte Permutationsdarstellung. Dann
	\begin{align*}
	\ker \theta = \bigcup_{x \in X}G_x
	\end{align*}
\end{lemma}
\begin{proof}
	\begin{align*}
		g \in \ker \theta &\Leftrightarrow \theta(x) = \id_X\\
		&\Leftrightarrow g x = x \quad \forall x \in X\\
		&\Leftrightarrow g \in G_x \quad \forall x \in X
	\end{align*}
\end{proof}
\begin{definition}
	$G \Circlearrowleft X$ ist \begriff{treu} genau dann, wenn
	\begin{align*}
	\theta\colon G \to S_X \text{ injektiv}\\
	\forall g,h \in G \; (\forall x \in X \colon gx = hx) \implies g = h
	\end{align*}
\end{definition}
\subsection*{Transitive Operationen}
\begin{definition}
	\begin{align*}
		G \Circlearrowleft X \text{ \begriff{transitiv} } \Leftrightarrow \text{ gibt genau eine Bahn}\\
		&\Leftrightarrow x_0 \in X\quad X = G \cdot x_0\\
		&\Leftrightarrow \forall x,y \in X\colon \exists g \in G \colon y = gx
	\end{align*}
\end{definition}
\begin{*example}
	Betrachte 
	\begin{align*}
		O(n) = \set{A \in \Mat(n,\R) \mid A^T A = 1_n} = \set{A \in \Mat(n\R)\mid \norm{Ax} = \norm{x} \forall x \in \R^n}
	\end{align*}
	also $O(n) \Circlearrowleft S^{n-1} = \set{x \in \R^n \mid \norm{x} = 1}$ (Drehungen und Spiegelungen in der $S^2$ zum Beispiel) ist transitiv.
\end{*example}
\begin{lemma}
	$G \Circlearrowleft \lnkset{G}{H}$ transitiv
\end{lemma}
\begin{proof}
	$g,h \in G$, dann $gH = gh^{-1}\cdot h H$
\end{proof}
\begin{theorem}[Die Struktur von Gruppenoperationen]
	\begin{enumerate}
		\item $G \Circlearrowleft X \implies \exists H < G$ und ein $G$-Isomorphismus durch $(G,X) \cong (G, \lnkset{G}{H})$ ist \emph{transitiv}. ($H$ muss nicht eindeutig sein)
		\item $H,K < G$, dann
		\begin{align*}
			(G, \lnkset{G}{H}) \cong (G, \lnkset{G}{K}) \Leftrightarrow H \und K \text{ konjugiert} 
		\end{align*}
		also linke Seite $G$-Isomorph und bei der rechten Seite: $\exists g_0 \in G\colon H = g_0 Kg_0^{-1}$.
	\end{enumerate}
\end{theorem}
\begin{proof}
	\begin{enumerate}
		\item 
		\begin{itemize}
			\item $\Leftarrow$: $H < G \colon$ $G \Circlearrowleft \lnkset{G}{H}$ transitiv.
			\item $\Rightarrow$: Sei $G \Circlearrowleft X$ transitiv $x \in X$ beliebig und $X = Gx$. Definiere
			\begin{align*}
				H:= G_x \text{ und } \alpha: X \to \lnkset{G}{H} \mit gx \mapsto gH
			\end{align*} 
			\begin{itemize}
				\item Ist $\alpha$ wohldefiniert? Ja, da $\forall g,h \in G$ haben wir 
				\begin{align*}
				gx = hx &\Leftrightarrow h^{-1}g x = x\\
				&\Leftrightarrow h^{-1}g \in G_x = H \text{ (siehe HA1.1 gilt)}\\
				&\Leftrightarrow gH = hH.
				\end{align*}
				\item $\alpha$ injektiv? $\forall g,h \in G\colon \alpha(gx) = \alpha(hx)$. (Gehe die Wohldefiniertheit rückwärts).
				\item $\alpha$ surjektiv? $\forall g \in G \colon gH = \alpha(gx)$
				\item Betrachte
				\[
					\begin{tikzcd}
					X \arrow[r, "\alpha"] \arrow[d, "g"] & G/H \arrow[d, "g"] \\
					X \arrow[r, "\alpha"]                & G/H               
					\end{tikzcd}
				\]
				\begin{align*}
					\forall g\in G, y \in X g \alpha(y) \overset{?}&{=} \alpha(gy)\exists h \in G, y = hx\\
					\alpha(gy) = \alpha(g(hx)) = \alpha(ghx)\\
					&=ghH = ghH = g\alpha(hx)\\
					&= g \alpha(y)
				\end{align*}
				Also ist $\alpha$ $G$-Isomorphismus.
			\end{itemize}
			\item Wir müssen zuerst ein Lemma zeigen: 
			\begin{lemma}
				Sei $\alpha\colon X \to Y$ $G$-Isomorphismus, dann $\forall x \in X\colon G_x = G_{\alpha(x)}$
			\end{lemma}
			\begin{proof}
				$\alpha$ $G$-Isomorphismus gdw $\alpha$ bijektiv und $\forall x \in X\colon g\alpha(x) = \alpha(gx)$ und $\forall g \in G$
				\begin{align*}
					gx = x \Leftrightarrow g\alpha(x) = \alpha(gx) = \alpha(x)\\
					\implies g \in G_x \Leftrightarrow g \in G_{\alpha(x)}
				\end{align*}
			\end{proof}
			\begin{itemize}
				\item $\Rightarrow$: Sei $\alpha\colon \lnkset{G}{H} \to \lnkset{G}{K}$ ein $G$-Isomorphismus. Sei $g_0 \in G$, so dass $\alpha(H) = g_0 K$. Wir haben Stabilisator von $H$ in $G$
				\begin{align*}
					\set{g \in G \mid gH = H} = H
				\end{align*}
				und der Stabilisator von $gK$ in $G$
				\begin{align*}
					\set{g\in G \mid g \cdot g_0 K = g_0 K} = g_0 K g_0^{-1}
				\end{align*}
				dann haben wir
				\begin{align*}
					 g g_0 K = g_0 K &\Leftrightarrow g^{-1}_0 g g_0 K = K\\
					 &\Leftrightarrow g^{-1}g g_0 \in K\\
					 &\Leftrightarrow g \in g_0 K g^{-1}_0
				\end{align*}
				nun nutze das Lemma und es folgt $H = g_0 K g^{-1}_0$.
				\item $\Leftarrow$: Sei $g_0 \in G$ und nehme an $H = g_0 K g^{-1}$. Definiere $\alpha: \lnkset{G}{H} \to \lnkset{G}{K} \mit gH \mapsto gg_0 K$.
				\begin{itemize}
					\item $\alpha$ wohldefiniert: $\forall h, g \in G$
					\begin{align*}
						gH = gH &\Leftrightarrow h^{-1}g \in H = g_0 K g_0^{-1}\\
						&\Leftrightarrow g^{-1}_0 h^{.1} g g_0 \in K\\
						&\Leftrightarrow (hg_0)^{-1}g g_0 \in K\\
						&\Leftrightarrow gg_0 K = hg_0 K
					\end{align*}
				\item $\alpha$ injektiv
				\item $\alpha$ surjektiv $\forall g \in G\colon$ $gK = \alpha(g g_0^{-1}H)$.
				\item $\forall g,h \in G\colon h \alpha(gH) = h g g_0 K = \alpha(hgH)$.
				\end{itemize}
				Also haben wir, dass $\alpha$ $G$-Isomorphismus ist.
			\end{itemize}
		\end{itemize}
		\item 
	\end{enumerate}
\end{proof}
\begin{theorem}[Bahnen.Stabilisator-Satz]
	\begin{align*}
		G \Circlearrowleft X \implies \forall x \in X\colon \abs{Gx} = [G \colon G_x]
	\end{align*}
\end{theorem}
\begin{proof}
	Für alle $x \in X$ gilt $G \Circlearrowleft Gx$ transitiv
	\begin{align*}
		(G,G_x) \cong (G,\lnkset{G}{G_x}) \quad G\text{-Isomorph}\\
		\implies \abs{Gx} = \abs{\lnkset{G}{G_x}} = [G \colon G_x].
	\end{align*}
\end{proof}
Ausblick:\\
\begin{enumerate}
	\item $G \Circlearrowleft G$ durch Linksmultiplikation gegeben $\forall g,h \in G\colon g\cdot h = gh$
	\item $H< G$, $G \Circlearrowleft \lnkset{G}{H}$ durch Linksmultiplikation
	\item $G \Circlearrowleft G$ durch Konjugation $\forall g,x \in G \colon x^g := gxg^{-1}$
	\item $G \Circlearrowleft \powerset(G)$ durch Konjugation
\end{enumerate}
\begin{theorem}
	As $G$-Space, we have $G_x \cong \lnkset{G}{G_x}$.
\end{theorem}
\begin{*example}
	\begin{itemize}
		\item $G = \SO(3)$ Rotation im $\R^3$, dann $A \in \SO(3) \ni \Mat(3,\R)$, $AA^T = 1$, $\det A = 1$, dann kann man sich das Skalarprodukt anschaun % siehe Bild
		\begin{itemize}
			\item (Wirkung durch Multiplikation) und nehme $x = (0 0 1)^T, G_x = S^2$ (Bahn, Orbit dieses Punkte ist $S^2$), Stabilisator $G_x = \SO(2)$
			\item Allgemeiner: eingebettet in $\SO(3)$ durch
			\begin{align*}
			\gamma\colon \SO(2) \to \SO(3) \mit B \mapsto \begin{pmatrix}
			B & 0\\
			0 & 1
			\end{pmatrix}			 
			\end{align*}
			ist Gruppenhomo. (also $S^2 \cong \lnkset{\SO(3)}{\SO(2)}$ und Erlangen Programm ...)
		\end{itemize}
		\item \person{Cayley}s Satz $X = G$, $\alpha$ ist Gruppenstruktur. Hier gilt:
		\begin{align*}
			G = \set{1}\forall x \in X = G \mit gx = x \implies g = 1
		\end{align*}
		durch Multiplikation von rechts mit $x^{-1}$. (Spezielle Situation, i.A. gibts \emph{kein} $xy$ für $x,y \in X$!) Insbesondere ist die Wirkung treu, also $\rho$ injektiv und wir erhalten
		\begin{align*}
			G \cong \rho(G) < S_G
		\end{align*}
		\item $G$ Gruppe, $H < G$, $X = \lnkset{G}{H}$ mit Wirkung $g(hH) := (gh)H, g,h \in G$, transitiv, da $x = hH, y= tH$ ($x,y \in X$ beliebig), dann wähle $g:= t H^{-1}$ meine Wahl ($\exists g \in G$ $t h^{-1}hH = gx = y = tH$)
		\begin{align*}
			G_x = \set{g \in G \mid gx = x} \und x = hH \quad gx = x \Leftrightarrow ghH = hH \implies h^{-1}gh \in H \implies g \in hHH^{-1} \in H\\
			\text{Insbesondere }\ker \rho = \bigcap_{x \in X} G_x = \bigcap_{H \in H} hHh^{-1} \lhd G 
		\end{align*}
		Dies ist $= H \Leftrightarrow H \lhd G$
		\item Adjungierte Wirkung: $X = G$ (wie oben) \emph{aber}
		\begin{align*}
			g \rhd x\cdot gxg^{-1}
		\end{align*}
		($\rhd$ neues Symbol zum unterscheiden). $G_x = $ Konjugationsklassen von $x \in X = C(x)$ und
		\begin{align*}
			G_x = \set{g \in G\colon gxg^{-1} = x \Leftrightarrow gx = xg} = Z(x) \text{ Zentralisator}
		\end{align*}
		Nutze Orbit-Stabiliser-Theorem:
		\begin{align*}
			\lnkset{G}{Z(x)} \cong C(x) \und \abs{C(x)} = \frac{\abs{G}}{\abs{Z(x)}}
		\end{align*}
		Die Konjugationsklassen mit nur einem Element bilden das Zentrum von $G$. Also gilt 
		\begin{align*}
			G = Z(G) \cup C(x_1) \cup \dots C(x_d)
			\intertext{für geeignete $x_1, \dots, x_d \in G$}
			\abs{G} = \abs{Z(G)} + \sum_{i=1}^d \frac{\abs{G}}{\abs{Z(x_i)}} \tag{class-equation}\label{eq_1_6_1_class_eq}
		\end{align*}
		(wobei die Vereinigung disjunkt sind) % find disjoint union symbol :S MINT?!
	\end{itemize}
\end{*example}
\begin{proposition}
	Eine endliche $p$-Gruppe hat nichttriviales Zentrum.
\end{proposition}
\begin{proof}
	$\abs{G} = p^n$ mit $p=$prim und dann
	\begin{align*}
		p^n &= \abs{Z(G)} + \sum_{i=1}^d \frac{\abs{G}}{\abs{Z(x_i)}}\\
		&= \abs{Z(G)} + p^{n_1} + \dots + p^{n_d} \quad n > 0
	\end{align*}
	Also ist $\abs{Z(G)} \ge 1$ denn $1 \in Z(G)$ und $\abs{Z(G)}$ ist teilbar durch $p$ und damit ist $\abs{Z(G)} \ge p$
\end{proof}
Insbesondere ist $G$ nicht einfach! ($Z(G) \lhd G$)
\section{Die \person{Sylow}-Sätze}
Sei $G$ eine endliche Gruppe.
\begin{definition}
	Eine Untergruppe von $G$ ist eine maximale $p$-Untergruppe (für eine Primzahl $p$), d.h. $H < G$ und $\exists p$ prim, $n \in \N\colon \abs{H} = p^{n+1} \nmid G$ (teil nicht).
\end{definition}
Also $\abs{G} = p^n \cdot m$ und $p \nmid m$.
\begin{theorem}[alle Sylow-Sätze]
	Mit der wie oben gilt:
	\begin{enumerate}
		\item Die Zahl $r$ der Sylowschen Untergruppen von $G$ ist 1 modulo $p$ ($\exists s\colon r = 1+sp$). Insbesondere ist $r \neq 0$!
		\item Jede $p$-Untergruppe von $G$ ist in einer Sylowschen enthalten
		\item Alle Sylowschen Untegruppen sind konjugiert zueinander. Insbesondere gilt: $r \mid m$. Also
		\begin{align*}
			H < G\quad \abs{H} = p^n \quad N_G = N := \set{g\in G \mid gHg^{-1} = H} \quad \abs{\lnkset{G}{N}} = r = \frac{p^n \cdot m}{p^n \cdot \lnkset{N}{H}}
		\end{align*}
		da $m = r \cdot \abs{\lnkset{N}{H}}$ und $\abs{N} = \abs{\lnkset{N}{H}} \cdot p^n$
	\end{enumerate}
\end{theorem}
\begin{proof}
	\begin{enumerate}
		\item Sei $X := \set{P \subseteq G \mid \abs{P} = p^n} \ni \set{x_1, \dots, x_{p^n} \mid x_j \in G}$. $G$ wirkt auf $X$ durch Multiplikation von links.
		\begin{align*}
		g \set{x_1, \dots, x_{p^n}} = \set{gx_1, \dots, gx_{p^n}}
		\end{align*}
		Gesucht ist $H \subset X$ mit $H < G$. Für ein solches $H$ ist dann $\lnkset{G}{H} \subseteq X$ eine Bahn der $G$-Wirkung mit der Länge $m = \abs{\lnkset{G}{H}} = \frac{\abs{G}}{\abs{H}} = \frac{p^n m}{p^n}$\\
		\emph{Behauptung 1:} Alle Bahnen in $X$ der Länge $m$ sind von dieser Form.
		\begin{proof}[Beweis der Behauptung]
			Sei $GP \subseteq X$ eine Bahn mit $\abs{GP} = m$. WLOG ist $1 \in P$ (wenn nicht, wähle $x \in P$ beliebig und ersetze $P$ durch $x^{-1}P$)
			\begin{align*}
			G_P = \set{g \in G \mid gP = P}\quad P = \set{1, x_2, \dots, x_{p^n}}\\
			\und gP = \set{g1, gx_2, \dots, gx_{p^n}}
			\end{align*}
			Also $1 \in P \implies G_P \subset P$ und daraus $\abs{G_P} \le \abs{P} = p^n$. Aber wir wissen auch:
			\begin{align*}
			m = \abs{GP} = \abs{\lnkset{G}{G_P}} = \frac{\abs{G}}{\abs{G_P}} = \frac{p^n m}{G_P} \implies \abs{G_P} = p^n.\tag{$\ast$}\label{eq_1_7_sylow_1}
			\end{align*}
			Also ist $G_P = P$. Damit folgt die Behauptung $G_p < G$.\\
			\emph{Behauptung \eqref{eq_1_7_sylow_1}:} Alle Bahnen in $X$, deren Länge nicht durch $p$ teilbar ist, sind von der Form $\lnkset{G}{H}$
		\end{proof}
		Damit folgt 1.\\
		\emph{Behauptung 2:} Die einzige Nebenklasse $gH$, die eine Untergruppe von $G$ ist, ist $H$ selbst ($H$ wie oben Sylowsche Untergruppe)
		\begin{proof}
			$gH$ Untergruppe $\implies 1 \in gH \implies g^{-1} \in H$ und daraus $g \in H$ und schließlich $gH = H$
		\end{proof}
		Also existiert eine Bijektion zwischen den Sylowschen $p$-Untergruppen $H < G$ und den Bahnen $GP \subseteq X$ mit $\abs{GP}$ nicht durch $p$ teilbar.\\
		\emph{Behauptung:} Ist $[n]\in \Z_p$ die Klasse von $n \in \Z$ in $\Z_p$, so gilt
		\begin{align*}
		[r] = \frac{1}{[m]}\binom{\abs{G}}{p^n}
		\end{align*}
		\begin{*remark}
			$\abs{X} = \binom{\abs{G}}{p^n}$ per Definition von $X$ als $\set{P \subseteq G \mid \abs{P} = p^n}$
		\end{*remark}
		Nach Behauptung 1 und Behauptung 2 gilt $rm = \binom{\abs{G}}{p^n}$ modulo $p$, denn $X$ zerfällt in Bahnen, da $G$-Wirkung und die Bahnen deren Länge nicht durch $p$ teilbar sind. Die Bahnen sind von der Form $\lnkset{G}{H}$ für eine eindeutige, bestimmte Sylow $p$-Untergruppe $H$, haben die Länge $m$ und es gibt $r$ davon. Damit ist der erste Sylow-Satz bewiesen. \\
		Yeah its scrambled again :/
		\begin{*remark}
			Only $\abs{G}$ enters here, we can compute $r$ modulo $p$ using any group $G$ of the size $\abs{G}$, z.B. können wir $G = \Z_{\abs{G}}$, und in $\Z_{p^{i} m}$ gibt es genau eine Sylow $p$-Untergruppe, nämlich $\langle [m]\rangle = \set{[e], [m], [2m], \dots, [(p^{i} - 1)m]}$ (Erinnerung 1. Semester: $H \le \Z \Leftrightarrow H = a\Z \dots $) $\implies$ 1.
		\end{*remark}
	\item Betrachte die Konjugationswirkung von $G$ auf $X$
	\begin{align*}
		\Ad(g)P := \set{gx^{-1}g, \dots, gx_{p^{i}}g^{-1}}
	\end{align*}
	\emph{Behauptung 3:} Ist $P \le G$ Sylow $p$-Untergruppe und $Q \le P$ eine $p$-Untergruppe, so existiert $R = gPg^{-1}$ mit $aRa^{-1}\quad \forall a \in G$ ($N_G R \ge Q$)
	\begin{proof}
		Betrachte den Orbit von $P$ unter der adjungierten Wirkung
		\begin{align*}
			\Ad(G)P := \set{gPg^{-1} \mid g \in G} \cong \lnkset{G}{N_G P} \quad \text{ orbit stabilizer}\\
			(g \in N_G P \Leftrightarrow gPg^{-1} = P)
		\end{align*}
		Dann gilt noch 
		\begin{align*}
			p^i n = \abs{N_G P} \implies \abs{\Ad(G)P} = \frac{m}{n}\tag{$\ast \ast$}\label{eq_sylow_proof_2}
		\end{align*}
		\emph{nicht} durch $p$ teilbar ($P \le N_G P$). Jetzt wirken nur mit $Q$ durch Konjugation auf diesen einen Orbit $\Ad(G)P$. $Q \le G$, der eine Orbit zerfällt gegebenenfalls in mehrere. Jeder Teil ist von der Form
		\begin{align*}
			\set{aRa^{-1} \mid a \in Q} \quad \text{ für ein $R$ von der Form} \quad R = gPg^{-1}\tag{$\ast\ast\ast$}\label{eq_sylow_proof_3}
		\end{align*}
		für ein $g \in G$
		\begin{align*}
			\Ad(G)R \cong \lnkset{Q}{Q} \cap N_G R
		\end{align*}
		$\cong$ folgt aus orbit-stabilizer wieder und $N_G R$ sind die $a \in Q$ mit $a R a^{-1} = R$
		\begin{align*}
			\abs{\Ad(Q)R} = \frac{\abs{Q}}{\abs{Q \cap N_G R}} = p^s
		\end{align*}
		für ein $s$, denn $Q$ war $p$-Untergruppe nach Annahme. Wegen \eqref{eq_sylow_proof_2} muss ein $P$ existieren mit $s = 0$, d.h. $Q \subseteq N_G R$
	\end{proof}
	\emph{Behauptung 5:} $R$ wie im Behauptung 4 ist eine Sylowsche Untergruppe und $Q \le R$
	\begin{proof}
		$R$ ist Bild von $P$ unter dem (inneren) Automorphismus $x \mapsto gxg^{-1}$ mit $g$ aus \eqref{eq_sylow_proof_3} also ist $R$ eine Sylowuntergruppe, da $Q \le N_G R$ ist
		\begin{align*}
			\set{ab \in G \mid a \in Q,b \in R} = QR \le G \text{ eine Untergruppe}\\
			a,a' \in Q,b,b' \in R\; ab \cdot a'b' = ... \text{ need to add still!}
		\end{align*}
		Dann $\abs{QR} = \lnkset{\abs{Q}\abs{R}}{\abs{Q \cap R}} = \frac{p^j p^i}{Q \cap R}$, d.h. $QR$ ist $p$-Untergruppe wegen $R \le QR$ und $R$ Sylow, also maximal, so folgt $QR = R \implies Q \le R$ und das gibt uns den zweiten Sylow Satz.
	\end{proof}
	\item Ist der Spezialfall, in dem $Q$ selbst eine Sylowuntergruppe war! Dann ist $Q = R$ im obigen Beweis, aber $R = gPg^{-1}$. 
 	\end{enumerate}
\end{proof}
\subsection*{Anwendungen}
\begin{enumerate}
	\item Cauchy's Satz: $G$ endlich Gruppe $p \mid \abs{G} \implies \exists g \in G$ mit $\ord(g) = p$ (p ist prim!)
	\begin{proof}
		Sei $P \le G$ eine Sylow $p$-Untergruppe, $\abs{P} = p^i$. Für $x \in P$ gilt $\ord(x) \mid p^{i}$, d.h. $\ord(x) = p^s$, also hat $x^{p^{s-1}}$ die Ordnung $p$
	\end{proof}
	\item $\abs{G} = pg$, $p\neq q$, $p,q$ prim $\implies G$ \emph{nicht} einfach.
	\begin{proof}
		Sei $q < p$. Wieviele Sylow $p$-Untergruppen gibt es? Ist $r$ diese Zahl, so ist $r=1 \mod p$. jede davon hat $p$ Elemente, d.h. ist isomorphic zu $\Z_p$. Sind $P_1, P_2$ zwei solche, so folgt $P_1 \cap P_2 = \set{1}$ (denn dies ist die einzige Untergruppe in $\Z_p$). Gäbe es mehr als eine Sylow $p$-Untergruppe, also mindestens $p+1$ Stück, wären deren Vereinigung eine Menge mit 1 $\in \set{1} + (p+1)(p-1) = p^2$ Elementen und das ist ein Widerspruch also $\abs{G} = pq < p^2$.
	\end{proof}
\end{enumerate}
\begin{proposition}
	Annahme wie oben, aber $q \nmid p-1$. Dann ist $G \cong \Z_{pq}$.
\end{proposition}
\begin{proof}
	Denke zusätzlich über Sylow $q$-Gruppen nach. Deren Zahl $s$ ist 1 modulo $q$. Ausserdem teilt $\gamma, p\cdot q$, d.h. $\gamma = 1, \gamma = p, \gamma = q$ oder $s = pq$, damit die Sylowgruppen alle konjugiert sind zueinander sind. D.h. die Menge $S_q$ der Sylow $p$-Untergruppe von $G$ ist Orbit (eine Bahn) in $X = \set{P \subseteq G \mid \abs{p} = q}$ unter Konjugation. Nach dem Stabilisator-Bahnen-Satz ist die Menge der Sylow $q$-Gruppen in einer Gruppe $G$ also im Bijektion mit $\lnkset{G}{N_G H}$, wobei $H \le G$ eine beliebige Sylow $q$-Untergruppe ist und $N_G H = \set{x \in G \mid x H x^{-1} = H}$ der Normalisator (Stabilisator von $H$ unter der adjungierten Wirkung) von $H$ in $G$ ist. D.h. $\lnkset{G}{N_G H}$ ist die Menge der zu $H$ konjugierten Untergruppen.
	\begin{align*}
		s = \abs{\lnkset{G}{N_G H}} = \frac{\abs{G}}{\abs{N_G H}}
	\end{align*}
	ist also die Zahl der zu $H$ konjugierten Untergruppen. Insbesondere ist $H \le N_G H$, also $\abs{H}\abs{N_G H}$ und 
	\begin{align*}
		s = \frac{\lnkset{G}{H}}{\lnkset{\abs{N_G H}}{\abs{H}}}
	\end{align*}
	Wir haben in den Sylowsätzen 3. hinzugefügt, dass $\gamma \mid m$. Im unserer speziellen Situation heisst dies: $s$ teilt $p$
	\begin{align*}
		\abs{G} = pq \quad s = \#\text{ Sylow-}p\text{-Untergruppen}
	\end{align*}
	$H \le G$ beliebige solche Untegruppe, d.h. $\abs{H} = q$, d.h. $H \cong \Z_q$ und $H \le N_G H$
	\begin{align*}
		s = \frac{\lnkset{\abs{G}}{\abs{H}}}{\lnkset{\abs{N_G H}}{\abs{H}}} = \frac{(pq)/q}{n/q} = \frac{p}{n/q}
	\end{align*}
	(n könnte $pq$ ($s=1$) oder $q$ ($s=p$ sein). $s = \#$ Sylow $q$-Untergruppen in $G$ mit $\abs{G} = pq$ mit $p>q$. $S \mid p$ (nach Sylow 3.) folgt dann $s =1 \oder s =p$. $s = 1 \mod q$ (sylow 1.) folgt
	\begin{align*}
		s=1,s=q+1, s= 2q+1, \dots \tag{$\ast$}\label{eq_sylow_1_not proof}
	\end{align*}
	Wenn wir also noch annehmen $q \nmid p -1$, so fällt $s = p$ als Fall weg, wegen \eqref{eq_sylow_1_not proof}. Also ist $s =1$ und auch die Sylow $q$-Untergruppe ist eindeutig und somit normal.
	\begin{align*}
		\Z_p \cong \le G \ge Q \cong \Z_q
	\end{align*}
	Wir haben $H \cap Q = \set{1}$, also gilt $G \cong H \times Q \cong \Z_p \times \Z_q \cong \Z_{pq}$ $G \cong \lnkset{R}{I_1 \cap ... \cap I_d} \cong \bigtimes_{i=1}^d\lnkset{R}{I_i}$.
\end{proof}
\begin{proposition}
	Sei $H$ $p$-Sylow Untergruppe von $G$, $H$ eindeutig, dann folgt damit $H$ normal.
\end{proposition}
\begin{proof}
		Sei
	\begin{align*}
	\alpha \colon G \to G \mit y \mapsto y x y^{-1}
	\end{align*}
	Automorphismus und $\alpha(H) \le G$, $\abs{\alpha(H)} = \abs{H}$, dann folgt $\alpha(H) = H$
\end{proof}
\begin{proposition}
	Ist $n = pq, p>q$ und $q \mid p-1$, so gibt es bis auf Isomorphie genau zwei Gruppen $G$ der Ordnung $n$, die zyklische Gruppe $\Z_{pq}$ und eine nichtabelsche Gruppe, die ein semidirektes Produkt $\Z_p \rtimes\Z_q$ ist.
\end{proposition}
\begin{proof}
	Wie oben gibt es eine eindeutige Sylow $p$-Untergruppe $H \cong \Z_p$. Wir haben oben gesehen, dass es entweder $s=1$ oder $s = p$ Sylow $q$-Untergruppe gibt. Im Fall $s=1$, gibt es eine eindeutige Sylow $q$-Untergruppe $Q = \Z_q$ und $G = \Z_p \times \Z_q$ (wie oben). Im Fall $s=p$ gibt es Sylow $q$-Untergruppen $Q_1, \dots, Q_p$ und $Q_i \cong \Z_q \forall i \in [1,p]$ aber keine ist normal.\\
	$Q_1 \cong \Z_q$ wirkt auf $H \cong \Z_p$ durch Konjugation und wie oben ist $Q_1 \cap H = \set{1}$. Also ist $HQ \le G$ Untegruppe mit $pq$ Elementen, also $HQ = G$ und dies ist ein internes semidirektes Produkt, $H \rtimes Q_1 = G$
	\begin{align*}
		G \cong \Z_p \times \Z_q = \Z_{pq}
	\end{align*}
	Noch zu zeigen: 1. dieser Fall tritt auf, egal welche $p,q$ wir betrachten.\\
	2. und zwar auf eindeutige Weise bis auf Isomorphie.
\end{proof}
\begin{proposition}[\person{Burnside}]
	Sei $\abs{G} = p^i q^j$ ($p,q$ Primzahlen). Dann $G$ auflösbar (Auf jeden Fall nicht einfach!!!)
\end{proposition}
\begin{proof}
	to be done ...
\end{proof}
\begin{definition}
	Ist $G$ eine Gruppe und sind $(a,b) \in G^2$, so ist deren \begriff{Kommutator}
	\begin{align*}
		[a,b] = aba^{-1}b^{-1}
	\end{align*}
	und die derivierte Gruppe $G'$ also die von allem Kommutatoren erzeugte Untergruppe $G' \lhd G$
	\begin{align*}
		G \rhd G' \rhd G'' \rhd ...
	\end{align*}
	$G$ auflösbar wenn $n$ mit $G^{(n)} = \set{1}$
\end{definition}
\begin{*remark}
	Sei $\alpha\colon G \to H$ Homomorphismus, also
	\begin{align*}
		\alpha([a,b]) = \alpha(aba^{-1}b^{-1}) = \alpha(a)\alpha(b)\alpha(a)^{-1}\alpha(b)^{-1} = [\alpha(a),\alpha(b)]
	\end{align*}
	Damit gilt $\alpha(G') \subseteq H'$. Insbesondere $(G=H)$, $G'$ ist eine charakteristische Untergruppe (invariant unter allen $\alpha \in \Aut(G)$) und $G' \lhd G$ $(\alpha(a) = xax^{-1})$
\end{*remark}
\begin{*example}
	Wenn $G$ einfach ist, dann $G' = G$ oder $G' = 1 \Leftrightarrow G$ abelsch. Also einfach und äuflösbar $\Leftrightarrow G = \Z_p$. $\Aut(G) = \set{\alpha \colon G \to G \mid \alpha \text{ bijektiv}, \alpha(ab) = \alpha(a)\alpha(b)\forall (a,b)\in G}$. Ist $x \in G$, so ist $\alpha_2(a) = x a x^{-1}$ ein Automorphismus. Die Automorphismen dieser Form nennt man \begriff{innere Automorphismen}.
	\begin{align*}
		G \to \Aut(G) \mit x \mapsto \alpha_x 
	\end{align*}
	ist ein Gruppenhomomorphismus und es gilt
	\begin{align*}
		(\alpha_x \circ \alpha_y)(a) &= \alpha_x(\alpha_y(a)) = \alpha_x(g a g^{-1})\\
		&= x(yay^{-1})x^{-1} = (xy)a(xy)^{-1}\\
		&= \alpha_{xy}(a)
	\end{align*}
	für alle $a \in G$, also hat man $\alpha_x \circ \alpha_y = \alpha_{xy}$. Der Kern von $x \mapsto \alpha_x$ ist das Zentrum $Z(G)$ ($\alpha_x = \id \Leftrightarrow xa = ax \forall a \in G$). Das Bild von $x \mapsto \alpha_x$ ist die Gruppe $\Inn(G)$ der inneren Automorphismen. Diese ist normal: Ist $\beta \in \Aut(G)$, so gilt:
	\begin{align*}
		(\beta \circ \alpha_x \circ \beta^{-1})(a) &= \beta(\alpha_x(\beta^{-1}(a))) = \beta(x\beta^{-1}(a)x^{-1})\\
		&= \beta(x)a\beta^{-1}(x)\\
		&= \alpha_{\beta(x)}(a)
	\end{align*}
	also $\beta \circ \alpha_x \circ \beta = \alpha_{\beta(x)}$ und es gilt $\Inn(G) \lhd \Aut(G)$.
\end{*example}
\begin{definition}
	$\Out(G) = \lnkset{\Aut(G)}{\Inn(G)}$ \begriff{äussere Automorphismen}.
\end{definition}
\begin{proposition}[Klausur!!!]
	$G'$ ist die kleinste normale Untergruppe $N \lhd G$, für die $\lnkset{G}{N}$ ($\lnkset{G}{G'}$ abelsch) abelsch ist.
\end{proposition}
\begin{proof}
	\begin{itemize}
		\item $G \lhd G$ haben wir gesehen.
		\item $\lnkset{G}{G'}$ abelsch
		\begin{align*}
			ab' bG'\cdot (aG')^{-1}(bG)^{-1} &= [aG',bG']_{\lnkset{G}{G'}}\\
			aba^{-1}b^{-1} = 1G'
		\end{align*}
		nach Definition $(aba^{-1}b^{-1} \in G')$ ($G$ abelsch $\Leftrightarrow \forall (ab) \in G^2\colon aba^{-1}b^{-1} = 1$). Hier $a = aG', b = bG'$ und $1 = G'$.
		\item Ist umgekehrt $\lnkset{G}{N}$ abelsch, so folgt $[aN, bN]_{\lnkset{G}{N}} = 1_{\lnkset{G}{N}} = 1N$, also $aba^{-1}b^{-1} \in N;\ \forall (a,b) \in G^2$. Also gilt $G' \le N$.
	\end{itemize}
\end{proof}
\begin{*example}
	Sei $G = A_4$ und es gilt:
	\begin{align*}
		[(123),(234)] &= (123)(234)(123)^{-1}(234)^{-1}\\
		&= (123)(234)(321)(432)\\
		&= (14)(23)\in A'_4
	\end{align*}
	Ähnlich sieht man $(12)(34), (13)(24) \in A'_4$, d.h. $V \lhd A'_4$ und $\lnkset{A_4}{V} \cong \Z_3$ abelsch. Also ist $V = A'_4$, wobei $V = \set{\id, (12)(34),(13)(24),(14)(23)}$ die \begriff{kleinsche Vierergruppe} ist.
\end{*example}
\begin{*remark}
	$[a,b]\cdot [c,d]$ ist im Allgemeinen kein Kommutator, man muss $G'$ schon echt erzeugen, passiert aber erst do ab $\abs{G} = 90$.
\end{*remark}
\begin{*example}
	$S'_4 = A_4$, denn $A_4 \lhd S_4$, denn $A_4 = \ker \delta$ und $\delta\colon S_4 \to \Z_2 \mit \delta(a) = \pm 1$ und $\lnkset{S_4}{A_4} \cong \Z_2$ (mit ersten Isomorphiesatz $\Inn \delta = \Z_2$) ist abelsch:
	\begin{align*}
		S''_4 = A'_4 = V \cong \Z_2 \times \Z_2\\
		S'''_4 = A''_4 = V' = 1
	\end{align*}
	Also ist $S_4$ aufösbar. Aber: $n > 4\colon S'_n = A_n$, doch $A_n$ ist einfach, also $A'_n = A_n$, $S_n$ also nicht auflösbar.
\end{*example}
\begin{*remark}
	\begin{itemize}
		\item $\abs{G} = p^i q^j \implies G$ auflösbar (\person{Burnside})
		\item $\abs{G}$ ungerage $\implies G$ auflösbar (\person{Feit}-\person{Thompson})
	\end{itemize}
\end{*remark}
\begin{proposition}[Klausur!!!]
	\begin{enumerate}
		\item $G$ auflösbar, $H \le G \implies H$ auflösbar.
		\item $G$ auflösbar, $N \lhd G \implies \lnkset{G}{N}$ auflösbar
		\item $N \lhd G$, $N$ auflösbar, $\lnkset{G}{N}$ auflösbar $\implies G$ auflösbar
	\end{enumerate}
\end{proposition}
\begin{proof}
	\begin{enumerate}
		\item $H' \le G' \implies H'' \le G'' \dots H^{(n)} \le 1 \implies H^{(n)} = 1$
		\item $\pi \colon G \to \lnkset{G}{N} \mit a \mapsto aN$ surjektiver Gruppenhomomorphismus, da $\pi([a,b]) = [\pi(a), \pi(b)]$ $\implies \pi(G') = \pi (\lnkset{G}{N})'$. Nun verwende Induktion $\pi(G'') = (\lnkset{G}{N})'' \dots$
		\item Umgekehrt: Wie in 2. projeziert die Reihe von $G$ uf die von $\lnkset{G}{N}$, also
		\begin{align*}
			\begin{matrix}
				G \rhd & G' \rhd & G'' \rhd & G''' \rhd & \dots & G^{(n)}\\
				\downarrow & \downarrow & \downarrow & \downarrow & \dots & \downarrow\\
				\lnkset{G}{N} \rhd & (\lnkset{G}{N})' \rhd & (\lnkset{G}{N})'' \rhd & (\lnkset{G}{N})''' \rhd & \dots & (\lnkset{G}{N})^{n} = 1
			\end{matrix}
		\end{align*}
		Also $(\lnkset{G}{N})' = \lnkset{G' N}{N}$ (haben wir in 2.) ausgerechnet, nun anders geschrieben. Induktion:\\
		$\lnkset{G}{N}^{(n)} = \lnkset{G^{(n)}N}{N}$. Also $(\lnkset{G}{N})^{(n)} = 1 \Leftrightarrow G^{(n)} \subset N$. Damit folgt aber, dass $G^{(n+1)} < N$ (wie in 1.) und $G^{(n+m)}< N^{(m)}$. Da $N$ auflösbar, erhält man irgendwann 1.
	\end{enumerate}
\end{proof}
\begin{proposition}
	$G$ auflösbar und endlich $\Leftarrow$ Alle Faktoren in einer Kompositionsreihe isnd abelsch und in der Form $\Z_n$ (zyklisch)
	\begin{align*}
		G = G_0 \rhd G_1 \rhd G_2 \rhd \dots \rhd G_n = 1 \und \lnkset{G_i}{G_{i+1}} \text{ einfach}
	\end{align*}
\end{proposition}
\begin{proof}
	\begin{itemize}
		\item ``$\Rightarrow$'': Induktion nach der derivaten Länge $l := \min\set{n \mid G^{(n)} = 1}$
		\begin{itemize}
			\item $l=1$: $G$ abelsch $\implies$ Behauptung
			\item $l \to l+1$: Nach Korrespondenztheorem sind die Kompositionsfaktoren von $G$ die von $\lnkset{G}{G'}$ zusammen mit denen von $G'$
		\end{itemize}
		\item Wieder Induktion, diesmal nach der Länge $c$ einer Kompositionsreihe von $G$
		\begin{itemize}
			\item $c=1$: $G$ abelsch und damit auflösbar
			\item $c \to c+1$: Sei $G = H_1 \rhd H_2 \rhd \dots \rhd H_c =1$ eine Kompositionsreihe. Dann ist $\lnkset{G}{H_2}$ abelsch nach Annahme (denn $\cong \Z_n$) und $H_2 \rhd \dots \rhd H_c =1$ ist eine kürzere Kompositionsreihe, also per Induktion $H_2$ auflösbar. Nach dem Satz ist $G$ auflösbar ($H \lhd G$ auflösbar $\implies G$ auflösbar) $\lnkset{G}{H_2}$ auflösbar.
		\end{itemize}
	\end{itemize}
\end{proof}
\section{nilpotente Gruppen }
\begin{definition}
	Zu einer Gruppe $G$ definiert man die \begriff{untere zentrale Reihe} induktiv wie folgt
	\begin{align*}
		G = \Gamma_0(G) \rhd \Gamma_1(G) \rhd\Gamma_2(G)\rhd ...
	\end{align*}
	durch $G = \Gamma_0$. Wobei $\Gamma_{i+1}(G)$ ist die durch alle Kommutatoren $[a,b]$, $a \in \Gamma_1(G), b \in G$ erzeugte Untergruppe.
\end{definition}
\begin{*remark}
	$\Gamma_1(G) = G'$. $G''$ ist nur durch die $aba^{-1}b^{-1}$ mit $a,b\in G'$ erzeugt $\Gamma_2(G)$ enthält durch $aba^{-1}b^{-1}$ mit $b \notin G'$. Durch Induktion sehen wir: $G^{(n)} < \Gamma_n(G)$.
\end{*remark}
\begin{definition}
	$G$ nilpotent, genau dann wenn $\exists n\colon\Gamma_n(G) = 1$.
\end{definition}
\begin{remark}
	Es gilt die Hierarchie: Abelsche Gruppen $<$ nilpotente Gruppen $<$ auflösbare Gruppen
\end{remark}
\begin{*example}
	\begin{itemize}
		\item $S_3$ ist auflösbar, aber nicht nilpotent, da
		\begin{align*}
			S_3' = A_3 = \set{(),(123),(132)} \cong \Z_3 \quad \text{abelsch}\quad S_3'' = A_3' = 1
		\end{align*}
		Aber in $\Gamma_2(S_3)$ gibt es $(123)(12)(123)(12)^{-1}= (123)(12)(132)(132)(12) = (132)$ und somit $\Gamma_2(S_3) = \Gamma_1(S_3) = A_3$ und daraus folgt $\Gamma_n(S_3)\forall n \ge 1$, damit ist $S_3$ \emph{nicht} nilpotent.
		\item $\Z_3 \times \Z_3,S_3 \cong D_3 \cong \Z_3 \times \Z_2$
	\end{itemize}
\end{*example}
\begin{proposition}
	Untergruppen und Quotienten von nilpotenten Gruppen sind wieder nilpotent.
\end{proposition}
\begin{proof}
	Siehe Beweis bei auflösbaren Gruppen.
\end{proof}
\begin{*remark}
	Es gilt \emph{NICHT}: $N \rhd G$ nilpotent, $\lnkset{G}{N}$ nilpotent impliziert $G$ nilpotent
\end{*remark}
\begin{proposition}
	$\Gamma_i(G\times H) = \Gamma_i(G)\times \Gamma_i(H)$ (impliziert $G,H$ nilpotent $\implies G \times H$ nilpotent)
\end{proposition}
\begin{proof}
	Induktion nach $\abs{G} = p^i$. Wir haben gesehen: Eine $p$-Gruppe hat $Z(G)\neq 1$. $\abs{G} = 1$ trivial. Per Induktion ist $\lnkset{G}{Z(G)}$ nilpotent, da $\lnkset{G}{Z(G)} = \frac{\abs{G}}{\abs{Z(G)}} = \frac{p^i}{p^j} = p^s$. Also existiert $n \in \N$ mit $\Gamma_n(\lnkset{G}{Z(G)}) = 1$, d.h. $\Gamma_n(G) \subseteq Z(G)$. Schlussendlich haben wir $\Gamma_{n+1}(G) = 1$ und $G$ ist nilpotent.
\end{proof}
\begin{lemma}
	$G$ nilpotent, $H \lneq G \implies H \lneq N_G(H)$.
\end{lemma}
\begin{proof}
	Sei $i$ minimal mit $\Gamma_i(G) \le H$. Da $H \neq G$, ist $i \ge 1$. Es gilt 
	\begin{align*}
		[\Gamma_{i-1}(G),H] \le [\Gamma_{i-1},G] = \Gamma_i(G) \le H
	\end{align*}
	zum ersten Term in der Gleichung: von allen $aba^{-1}b^{-1}$ mit $a\in \Gamma_{i-1}(G),b \in H$ erzeugte Untergruppe, also ist $aba^{-1}b^{-1} \in H$ für $a \in \Gamma_{i-1}(G), b \in H$, d.h. $aba^{-1} \in H$ und $\Gamma_{i-1}(G) \le N_G(H)$, $\Gamma_{i-1}(G) \gneq H$.
\end{proof}
\begin{*example}
	Sei $B = \set{
		\begin{pmatrix}
			x & y\\ 0 & z
		\end{pmatrix} \mid x,y,z \in K}$, wobei $xz \neq 0$
	\begin{align*}
		\begin{pmatrix}
			x & y\\ 0 & z
		\end{pmatrix}
		\begin{pmatrix}
			u & v\\ 0 w
		\end{pmatrix}
		\begin{pmatrix}
			x & y\\ 0 & z
		\end{pmatrix}^{-1}
		\begin{pmatrix}
			u & v\\ 0 & w
		\end{pmatrix}^{-1}\\
		= \begin{pmatrix}
		x & y\\ 0 & z
		\end{pmatrix}
		\begin{pmatrix}
		u & v\\ 0 w
		\end{pmatrix}
		\frac{1}{xz}
		\begin{pmatrix}
			z & -y\\ 0 & x
		\end{pmatrix}
		\frac{1}{uw}
		\begin{pmatrix}
			w &-v\\0 & v
		\end{pmatrix}\\
		= \begin{pmatrix}
			1 & \frac{1}{xzuw}(xv + yw - zv - yu)\\
			0 & 1
		\end{pmatrix}\\
	\end{align*}
	damit ist dann
	\begin{align*}
		B' = \set{
			\begin{pmatrix}
				1 & s\\ 0 & 1
			\end{pmatrix}\mid s \in K}
	\end{align*}
	und $B'' = 1$, so $B$ ist solvable. However ... if u need the pictures let me know, i have them, to bored to type it.
	
\end{*example}
\begin{erinnerung}
	\begin{enumerate}
		\item $H,G$ nilpotent $\implies G \times H$ nilpotent
		\item $p$-Gruppen nilpotent
		\item $G$ nilpotent, $H \le G \implies H \lneq N_G(H) = \set{x \in G \mid xyx^{-1}\in H \forall y \in H} (H \lhd G \implies N_G(H) = G)$
	\end{enumerate}
\end{erinnerung}
\begin{lemma}
	Sei $G$ endliche Gruppe, $H \le G$ Sylowgruppe. Dann gilt
	\begin{align*}
		N_G(N_G(H)) = N_G(H).
	\end{align*}
\end{lemma}
\begin{proof}
	Sei $x \in N_G(N_G(H))$, d.h. $xN_G(H)x^{-1} = N_G(H)$. Also gilt insbesondere
	\begin{align*}
		\tilde{H} := xHx^{.1} \subseteq N_G(H)
	\end{align*}
	dann $H, \tilde{H}$ sind dann beides Sylowgruppen von $N_G(H)$. Also existiert nach Sylowsätzen ein $n \in N_G(H)$ mit $nHn^{-1} = \tilde{H}$. Aber $n \in N_G(H)$ heisst ja $nHn^{-1} = H$, also $H = \tilde{H}$. Also $x \in N_G(H)$.
\end{proof}
\begin{proposition}
	Eine endliche Gruppe ist nilpotent, genau dann wenn $G = \bigtimes_{i=1}^d H_i$, wobei die $H_d$ ihre Syllowuntergruppe sind.
\end{proposition}
\begin{proof}
	\begin{itemize}
		\item $\Leftarrow$: folgt aus den Sätzen von VL 9.12.2019, bzw. aus den ersten beiden punkten der letzten Erinnerung. %TODO add reference?
		\item $\Rightarrow$ Sei $H \le G$ eine Sylowuntergruppe. Ein Lemma sagt: $N_G(N_G(H)) = N_G(H)$. Also sagt dritte Punkt in der letzten Erinnerung, dass $N_G(H) = G$ gilt. Also gibt es für jedes $p \mid \abs{G}$ genau eine Sylow $p$-Untergruppe, die normal ist. Sind $H_1,H_2$ zwei Sylowgruppen für verschiedene Primzahlen $p_1,p_2 \mid \abs{G}$, folgt
		\begin{align*}
			H_1 \cap H_2 = \set{1}
		\end{align*}
		denn $H_1 \cap H_2 \le H_1$, also (Lagrange) ($H_1 \cap H_2 = p^j_1$), genau aber $H_1 \cap H_2 \le H_2$, also $\abs{H_1 \cap H_2} = p_2^k$, also $j = k = 0$, d.h. $H_1H_2 = H_1 \times H_2 \lhd G$ und jetzt Induktion nutzen.
	\end{itemize}
\end{proof}
\[
	\begin{tikzcd}
		& \text{Unendliche Gruppen} \arrow[rrd] \arrow[rd] \arrow[ld] &                               &                     \\
		\text{geo. Gruppentheo.} \arrow[rd] &                                                             & \text{LIE-Gruppen} \arrow[ld] & \text{alg. Gruppen} \\
		& \text{top. Gruppen}                                         &                               &                    
	\end{tikzcd}
\]
\begin{itemize}
	\item Geometrische Gruppentheorie (Thom): Zunächst sind die Gruppen selbst nur Gruppen, aber man studiert durch ihre Wirkungen auf geometrische Objekte, z.B. Graphen, topologische Räume, HILBERT-Räume, Mannigfaltigkeiten, ... \\
	Die Gruppe $G$ sind meist durch eine Repräsentation durch Erzeuger und Relatoren gegeben.
\end{itemize}
\begin{enumerate}
	\item Freie Gruppen: Bei einem Vektorraum der Form $F(B):= k^B$ (\begriff{frei} über $B$) galt
	\begin{align*}
		\Hom_{\Vect}(F(B),V) \cong \Hom_{\Set}(B,U(V))
	\end{align*}
	dann haben wir $F \colon \Set \to \Vect$ \begriff{Freie Funktor} und $U \colon \Vect \to \Set$ \begriff{Vergissfunktor}, $U(v)$ ist einfach die Menge $V$. Also bilden eine Adjunktion (siehe CAT-VL) $U \dashv F$.\\
	Genauer gibt es:
	\begin{align*}
		F\colon \Set \to \Grp\\
		U \colon \Grp \to \Set
	\end{align*}
	mit $\Hom_{\Grp}(F(X),G) \cong \Hom_{\Set}(X,U(G))$. Die freie Gruppe $F(X)$ ist hierbei explizit konstruiert als die Menge aller Wörter $x_1, \dots, x_d$ mit Buchstaben aus dem Alphabet $\set{a,a^{-1}\mid a \in X}$, die dann noch auf offenkundige Weise reduziert (gekürzt) werden.
	\begin{*example}
		$X = \set{a,b}$ $F(X)$ enthält z.B. $bab, b^3a, b^{-1}a^2, ... a^{-1}ab$ würde zu $b$ gekürzt
	\end{*example}
	Relationen sind vorgegebene Gleichung, die man festlegt um kleinere Gruppen als Quotienten von $F(X)$ zu definieren, also Präsentation einer Gruppe
	\begin{align*}
		\langle a,b \mid a^2, b^2, ababab\rangle
	\end{align*}
	(Bemerkung: lässt man $ababab$ weg erhält man die Gruppe der Zöpfe (Braid groups))
	$\lnkset{F(\set{a,b})}{N}$ wobei die kleinste normale Untergruppe ist, die $a^2,b^2,ababab$ enthält $a^2 = 1$ in $\lnkset{F(\set{a,b})}{N}$.\\
	Übung $\langle a,b \mid a^2, b^2, ababab\rangle \cong S_3$
\end{enumerate}
\section{Exkurs: LIE-Algebra}
%TODO see anna pics
i will add this, when there is some time left, but it was really extra and relevant for the exam ...
\chapter{Ringe}
\section{Rings}
\begin{definition}
	A \begriff{ring} is a set $R$ together with two binary operators $+,\cdot \colon R^2 \to R$ addition and multiplication and an element $1 \in R$, which satisfies the conditions
	\begin{enumerate}
		\item $(R,+)$ is abelian group
		\item $(a\cdot b)\cdot c = a \cdot (b\cdot c)$ for all $a,b,c \in R$. (Associativity)
		\item $1\cdot a = a = a \cdot 1$ for all $a \in R$. (Unital)
		\item We have
		\begin{align*}
			(a+b)\cdot c = (a\cdot c) + (b \cdot c) \nd c\cdot (a+b) = (c\cdot a) + (c \cdot b) \quad \forall a,b,c \in R.
		\end{align*}
	\end{enumerate}
\end{definition}
\begin{*remark}
	\begin{itemize}
		\item write $ab$ instead of $a\cdot b$
		\item unit element of $(R,+)$ is denoted by 0.
		\item We do not require $1\neq 0$ and note $1=0 \iff R = 0$, this means that $R$ consists of exactly one element.
	\end{itemize}
\end{*remark}
\begin{definition}
	$R,S$ rings, a ring homomorphism is a homormorphism $f\colon R \to S$ such that $f(ab)$
\end{definition}

\part*{Anhang}
\addcontentsline{toc}{part}{Anhang}
\appendix

\nocite{*}
\bibliography{literatur}
\bibliographystyle{acm}

%\printglossary[type=\acronymtype]

\printindex

\end{document}
