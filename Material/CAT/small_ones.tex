\documentclass{standalone}


\usepackage{tikz, amsmath}
	\usepackage{tikz-qtree}
	\usetikzlibrary{cd}
	\usetikzlibrary{arrows}
	\usetikzlibrary{automata}
	\usetikzlibrary{babel}
	\usetikzlibrary{calc}
	\usetikzlibrary{fit}
	\usetikzlibrary{matrix}
	\usetikzlibrary{positioning}
	\usetikzlibrary{shapes.geometric}
	\usetikzlibrary{arrows.meta,bending}

\begin{document}
	\begin{tikzcd}
		[execute at end picture={
			\draw[{Hooks[width=+0pt 10.8,length=+0pt 3.6,harpoon,line cap=round]}->] 
			(\tikzcdmatrixname-1-2) to[out=175,in=185,looseness=3] 
			node[midway,left]{$i$} (\tikzcdmatrixname-3-2);
			\draw[-{Latex[bend]},thick] ([xshift=-0.35cm,yshift=0.45cm]\tikzcdmatrixname-2-2)
			arc(0:340:0.2);
			\draw[-{Latex[bend]},thick] ([xshift=-0.45cm,yshift=-0.45cm]\tikzcdmatrixname-2-2)
			arc(20:360:0.2);
			\draw[-{Latex[bend]},thick] ([xshift=0.55cm,yshift=-0.45cm]\tikzcdmatrixname-2-2)
			arc(20:360:0.2);
		}]
		& R \arrow[d, "f"] \arrow[ld, "\phi"', hook'] \arrow[rd, "\phi", hook] &  \\
		R\left\lbrack\frac{1}{r}\right\rbrack \arrow[r, "\exists!F", two heads, dashed, hook] & {\frac{R[x]}{(rx-1)}} \arrow[r, "\exists!G", two heads, dashed, hook] & R\left\lbrack\frac{1}{r}\right\rbrack \\
		& {R[x]} \arrow[u, "\pi", two heads] \arrow[ru, "g"'] & 
	\end{tikzcd}
\end{document}