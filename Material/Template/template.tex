\RequirePackage{ifluatex}
\documentclass[ngerman,a4paper]{report}
%\usepackage[top=2cm,bottom=2cm,left=1cm,right=1cm]{geometry}
\usepackage[ngerman]{babel}
%\usepackage[utf8]{inputenc} %not recommended with lualatex

\usepackage{zref-base}
\usepackage{etoolbox}
\usepackage{xparse}%better macros
\usepackage{chngcntr}

\usepackage[T1]{fontenc}
\ifluatex
\usepackage{fontspec}
\fi

\usepackage[texindy]{imakeidx}
\makeindex[name=semester1,title={Symbolverzeichnis (1. Semester)}]
\makeindex[name=semester2,title={Symbolverzeichnis (2. Semester)}]
\makeindex[name=symbols,title=Symbolverzeichnis]

\usepackage[xindy,acronym]{glossaries}
\makeglossaries

\usepackage{amsmath}
\usepackage{amssymb}
\usepackage{amsfonts}
\usepackage{latexsym}
\usepackage{marvosym} %lighning
\usepackage{bbm} %unitary matrix 1
\usepackage{cancel}
\usepackage{xfrac}%sfrac -> fractions e.g. 3/4

\usepackage[table]{xcolor}
\usepackage{graphicx}

\usepackage{enumerate}
\usepackage{enumitem} %customize label

\usepackage{tabularx}
\usepackage{multirow}
\usepackage{booktabs}

\usepackage{ulem} %better underlines

\usepackage{parskip}%split paragraphs by vspace instead of intendations
\usepackage{tocloft}
\usepackage{fancyhdr}
\usepackage{titlesec}%customize titles
\usepackage{marginnote}

\usepackage{tikz}
\usetikzlibrary{patterns,arrows,calc,decorations.pathmorphing}

\usepackage[amsthm,thmmarks,hyperref]{ntheorem}%customize theorem-environments more effectively
\usepackage[ntheorem,framemethod=TikZ]{mdframed}

\usepackage[bookmarks=true]{hyperref}
\hypersetup{
	colorlinks,
	citecolor=green,
	filecolor=green,
	linkcolor=blue,
	urlcolor=green
}
\usepackage{cleveref}
\usepackage{bookmark}

\definecolor{lightgrey}{gray}{0.91}
\definecolor{lightred}{rgb}{1,0.6,0.6}
\definecolor{darkgrey}{gray}{0.6}
\definecolor{darkgreen}{rgb}{0,0.6,0}

\newlength{\blacktrianglewidth}
\settowidth{\blacktrianglewidth}{$\blacktriangleright$}

%numbered theorems
\theoremstyle{break}
\theorembodyfont{}

\mdfdefinestyle{boxedtheorem}{%
	outerlinewidth=3pt,%
	skipabove=5pt,%
	skipbelow=10pt,%
	frametitlefont=\normalfont\bfseries\color{black},%
}

\newmdtheoremenv[%
	style=boxedtheorem,%
	innertopmargin=\topskip,%
	innerbottommargin=\topskip,%
	linecolor=darkgrey,%
	backgroundcolor=lightgrey,%
]{theorem}{Theorem}[section]
\newmdtheoremenv[%
	style=boxedtheorem,%
	linecolor=darkgrey,%
	topline=false,%
	rightline=false,%
	bottomline=false,%
	innertopmargin=\topskip,%
	innerbottommargin=\topskip,%
	backgroundcolor=lightgrey,%
]{proposition}[theorem]{Satz}
\newmdtheoremenv[%
	style=boxedtheorem,%
	linecolor=darkgrey,%
	topline=false,%
	rightline=false,%
	bottomline=false,%
	backgroundcolor=lightgrey,%
	innertopmargin=\topskip,%
	innerbottommargin=\topskip,%
]{lemma}[theorem]{Lemma}
\newmdtheoremenv[%
	style=boxedtheorem,%
	linecolor=red,%
	topline=false,%
	rightline=false,%
	bottomline=false,%
	innertopmargin=0,%
	innerbottommargin=-3pt,%
]{definition}[theorem]{Definition}
\newmdtheoremenv[%
	outerlinewidth=3pt,%
	linecolor=black,%
	topline=false,%
	rightline=false,%
	bottomline=false,%
	innertopmargin=0pt,%
	innerbottommargin=-3pt,%
	frametitlefont=\normalfont\bfseries\color{black},%
	skipabove=5pt,%
	skipbelow=10pt,%
]{conclusion}[theorem]{Folgerung}
\newmdtheoremenv[%
	hidealllines=true,%
	frametitlefont=\normalfont\bfseries\color{black},%
	innerleftmargin=0pt,%
	skipabove=5pt,%
	innerleftmargin=10pt,%
]{remark}[theorem]{\hspace*{-10pt}$\blacktriangleright$\hspace*{\dimexpr \mdflength{innerleftmargin} - \blacktrianglewidth\relax}Bemerkung}
\newmdtheoremenv[%
	hidealllines=true,%
	frametitlefont=\normalfont\bfseries\color{black},%
	innerleftmargin=10pt,%
]{example}[theorem]{\hspace*{-\mdflength{innerleftmargin}}\rule{5pt}{5pt}\hspace*{5pt}Beispiel}

%unnumbered theorems
\theoremstyle{nonumberbreak}
\theoremindent0cm
\newmdtheoremenv[%
	style=boxedtheorem,%
	linecolor=red,%
	topline=false,%
	rightline=false,%
	bottomline=false,%
	innertopmargin=0,%
	innerbottommargin=-3pt,%
]{*definition}{Definition}
\newmdtheoremenv[%
	hidealllines=true,%
	frametitlefont=\normalfont\bfseries\color{black},%
	innerleftmargin=0pt,%
	skipabove=5pt,%
	innerleftmargin=10pt,%
]{*remark}{\hspace*{-10pt}$\blacktriangleright$\hspace*{\dimexpr \mdflength{innerleftmargin} - \blacktrianglewidth\relax}Bemerkung}
\newmdtheoremenv[%
	hidealllines=true,%
	innerleftmargin=10pt,%
]{*example}{\hspace*{-\mdflength{innerleftmargin}}\rule{5pt}{5pt}\hspace*{5pt}Beispiel}

%Hinweis-Theoremstyle and environment
%To get rid of the parentheses, a new theorem style is neccessary (definition of nonumberbreak from ntheorem.sty)
%to achieve the underlining, this needed to put in the theoremstyle definition
\theoremheaderfont{\mdseries}
\theoremseparator{:}
\theorempostskip{0pt}
\makeatletter
\newtheoremstyle{noparentheses}%
{\item[\rlap{\vbox{\hbox{\hskip\labelsep \theorem@headerfont
				\uline{##1}\theorem@separator}\hbox{\strut}}}]}%
{\item[\rlap{\vbox{\hbox{\hskip\labelsep \theorem@headerfont
				\uline{##1\ ##3\theorem@separator}}\hbox{\strut}}}]}
\newtheoremstyle{underlinedPlain}%
	{\item[\hskip\labelsep \uline{\theorem@headerfont ##1\theorem@separator}]}%
	{\item[\hskip\labelsep \uline{\theorem@headerfont ##1\ (##3)\theorem@separator}]}
\makeatother

\theoremstyle{noparentheses}
\newmdtheoremenv[%
	hidealllines=true,%
	innerleftmargin=1em,%
]{interpretation}{\hspace*{\dimexpr - \mdflength{innerleftmargin}\relax}Interpretation}
\theoremstyle{underlinedPlain}
\newmdtheoremenv[%
hidealllines=true,%
innerleftmargin=1em,%
innerrightmargin=0,%
skipbelow=0pt,%
]{hint}{\hspace*{\dimexpr - \mdflength{innerleftmargin}\relax}Hinweis}

\theoremheaderfont{\normalfont}
\theorembodyfont{\itshape\small}
\theoremseparator{:}
\theoremsymbol{$\blacksquare$}
\renewtheorem{proof}{Beweis}

%for \cref: printed environment names
\crefname{theorem}{Theorem}{Theoreme}
\crefname{proposition}{Satz}{Sätze}
\crefname{lemma}{Lemma}{Lemmata}
\crefname{conclusion}{Folgerung}{Folgerungen}
\crefname{definition}{Definition}{Definitionen}
\crefname{remark}{Bemerkung}{Bemerkungen}
\crefname{example}{Beispiel}{Beispiele}
\crefname{*definition}{Definition}{Definitionen}
\crefname{*remark}{Bemerkung}{Bemerkungen}
\crefname{*example}{Beispiel}{Beispiele}

\makeatletter
\newcommand*{\rom}[1]{\expandafter\@slowromancap\romannumeral #1@}
\newcommand*{\proplbl}[1]{%
	\@bsphack
	\begingroup
		\label{#1}%
		\zref@setcurrent{default}{\arabic{chapter}}%
		\zref@wrapper@immediate{%
			\zref@labelbyprops{#1@chapter}{default}
		}
	\endgroup
	\@esphack
}
\newcommand*{\propref}[1]{%
	\zref@refused{#1@chapter}%
	\ifnumcomp{\value{chapter}}{=}{\zref@extractdefault{#1@chapter}{default}{1}}%
		{%same chapter
			\cref{#1}%
		}%
		{%otherwise
			\ifcsdef{r@#1}%in first compilation the label may not be defined yet
				{%
					\def\propositionref@current@type{}%
					\cref@gettype{#1}{\propositionref@current@type}%get the environment's name
		%example for following line:
		%\crefformat{truetheorem}{\cref@truetheorem@name~##2\rom{\zref@extractdefault{#1}{#1chapter}{1}}.##1##3}
		%this changes the format used by \cref to <environtment name> <chapter-number>.<section-number>.<theorem number>
					\crefformat{\propositionref@current@type}{%
						\csname cref@\propositionref@current@type @name\endcsname ~##2\rom{\zref@extractdefault{#1@chapter}{default}{1}}.##1##3%
					}%
					\cref{#1}%
					\crefformat{\propositionref@current@type}{%
						\csname cref@\propositionref@current@type @name\endcsname~##2##1##3%
					}%
				}%
				{??}%similar to \ref\cref: question marks in case of undefined labels
		}%
}
\makeatother

\NewDocumentCommand{\begriff}{s O{} m O{}}{%
	\IfBooleanTF{#1}%
	{\index{#2#3#4}}%
	{%
		\uline{#3}%
		\ifnumcomp{\value{chapter}}{<}{16}%
			{\index[semester1]{#2#3#4}}%
			{\index[semester2]{#2#3#4}}%
	}%
}
\NewDocumentCommand{\mathsymbol}{s O{} m m O{}}{%
	\IfBooleanTF{#1}%
	{\index[symbols]{#2#3@\detokenize{#4}#5}}%
	{#4\index[symbols]{#2#3@\detokenize{#4}#5}}%
}

\NewDocumentCommand{\zeroAmsmathAlignVSpaces}{s s O{0 pt} O{0 pt}}{%
	\IfBooleanTF{#1}%
	{%
		\IfBooleanTF{#2}%
		{\setlength{\belowdisplayskip}{#4}}%
		{\setlength{\abovedisplayskip}{#3}}%
	}%
	{%
		\setlength{\abovedisplayskip}{#3}%
		\setlength{\belowdisplayskip}{#4}%
	}%
}

\NewDocumentCommand{\itemEq}{s m}{%
	\begingroup%
	\setlength{\abovedisplayskip}{\dimexpr -\parskip + 1pt\relax}%
	\setlength{\belowdisplayskip}{0pt}%
	\IfBooleanTF{#1}%
		{\parbox[c]{\linewidth}{\begin{flalign*}#2&&\end{flalign*}}}%
		{\parbox[c]{\linewidth}{\begin{flalign}#2&&\end{flalign}}}%
	\endgroup%
}

\newcommand{\person}[1]{\textsc{#1}}
\newcommand{\highlight}[1]{\emph{#1}}

\newcommand{\realz}{\mathfrak{Re}}
\newcommand{\imagz}{\mathfrak{Im}}

\renewcommand{\epsilon}{\varepsilon}

%change headings:
\titlelabel{\thetitle.\quad}%. behind section/sub... (3. instead of 3)
\counterwithout{section}{chapter}
\renewcommand{\thechapter}{\Roman{chapter}}
%italic chapters (due to titlesec package some more stuff)
\titleformat{\chapter}[display]{\huge\bfseries\itshape}{\chaptername\;\thechapter:}{-5pt}{\huge\bfseries}
\titlespacing{\chapter}{0pt}{0pt}{10pt}
\titleformat{\section}[hang]{\bfseries\Large}{\thesection.}{8pt}{\Large\bfseries}

%change header:
\renewcommand{\headrulewidth}{0.75pt}
\renewcommand{\footrulewidth}{0.3pt}
\lhead{\rightmark}%left: section-number. section-title
\rhead{\leftmark}%right: chapter chapternumber: chapter-title

%change appearence of heading of toc: 0 space above, bold, italic huge toc-heading
\renewcommand{\cftbeforetoctitleskip}{0pt}
\renewcommand{\cfttoctitlefont}{\itshape\Huge\bfseries}

\pagestyle{fancy}
\pagenumbering{arabic}
%remember chapter-title in \leftmark and \rightmark
\renewcommand{\chaptermark}[1]{%
	\markboth{\chaptername
		\ \thechapter:\ #1}{}}
%remember section title in \leftmark
\renewcommand{\sectionmark}[1]{%
	\markright{\thesection.\ #1}{}}

%change numbering of equations to be section by section
\counterwithout{equation}{section}

\author{Dozent: Prof. F. Schuricht}
\title{Analysis Grundlagen 1 \& 2} % add WS & SS
\date{Updated: \today}

\begin{document}
\pagenumbering{roman}
\pagestyle{plain}

\maketitle
\tableofcontents

\pagebreak
\pagestyle{fancy}
\pagenumbering{arabic}
%remember chapter-title in \leftmark and \rightmark
%\renewcommand{\chaptermark}[1]{%
%	\markboth{\chaptername
%		\ \thechapter:\ #1}{}}
%remember section title in \leftmark
%\renewcommand{\sectionmark}[1]{%
%	\markright{\thesection.\ #1}{}}

\chapter{Funktionen und Stetigkeit}

\section{Anwendungen}

\begin{theorem}[Mit Titel]
    test
\end{theorem}

\begin{proposition}[Incl. Titel]
    Und Inhalt.
\end{proposition}

\begin{interpretation}[von Gleichung (1)]
	Und hier der spannende Text...
\end{interpretation}

\subsection{Partialbruchzerlegung}
	
\begin{theorem}[Mit Titel]
	\proplbl{eintheorem}
	test
\end{theorem}
	
\begin{proposition}[Incl. Titel]
	Und Inhalt.\propref{eintheorem}
\end{proposition}

\section{Noch mehr Anwendungen}

\begin{definition}[Und Name]
	Tada
\end{definition}

\begin{example}[Einschließlich Überschrift]
	Verständnisfördernd...
\end{example}

\begin{conclusion}[Ergänzung]
	Einfache Schlussfolgerung
\end{conclusion}

\begin{lemma}
	nur Müll
\end{lemma}

\begin{lemma}[Kurze, aber wichtige, Aussage]
	Klein aber fein.
\end{lemma}

\begin{remark}[Hossa]
	Dumdidum
\end{remark}

\begin{example}[Test]
	LOhne Nummer
\end{example}

\begin{proof}
	Ein Beweis unter Ausnutzung von \propref{einzweitestheorem}
	
	\begin{align}
		f = f(x) \text{ (mit Formelnnummer)}
	\end{align}
	und inline-math $\hbar$.
\end{proof}

\chapter{Its you}

\section{Or me?}

\begin{theorem}[Mit Titel]
	\proplbl{einzweitestheorem}
    test
\end{theorem}

\begin{proposition}[Incl. Titel]
    Und Inhalt.
\end{proposition}

\begin{definition}[Und Name]
    Tada
\end{definition}

\begin{example}[Einschließlich Überschrift]
    Verständnisfördernd...
\end{example}

\begin{conclusion}[Ergänzung]
    Einfache Schlussfolgerung
\end{conclusion}

\begin{lemma}[Kurze Aussage]
    Klein aber fein.
\end{lemma}

\begin{remark}[Hossa]
    Dumdidum
\end{remark}

\begin{*example}[Test]
    Ohne Nummer
\end{*example}

\begin{proof}
    Ein Beweis. Siehe auch \propref{eintheorem} und \propref{einzweitestheorem}.
    \begin{align}
    f = f(x) \text{ (mit Formelnnummer)}
    \end{align}
    und inline-math $\hbar$.
    
    \begin{itemize}
    	\item \itemEq{f(x) = m\cdot x + n}
    	Eine Formel auf Höhe des Itemize-Punktes \emph{mit} Numerierung (\verb|\itemEq*{\dotsc}| ohne Nummerierung)
    \end{itemize}

	Eventuell ist es mal praktisch, \zeroAmsmathAlignVSpaces \begin{align*}
		f(x) = m\cdot x + n
	\end{align*}
	\verb|align| ohne Abstände einzubauen. Das geht auch einseitig mit \verb|\zeroAmsmathAlignVSpaces*| (nur oben Ändern) bzw. \verb|\zeroAmsmathAlignVSpaces**| (nur unten ändern).\marginnote{Hinweis am Rand!}
\end{proof}

\begin{*definition}[Aber Name]
	Ohne Nummer
\end{*definition}

\begin{*remark}
	Bemerkung sowieso
\end{*remark}

\renewcommand*\textcircled[1]{\tikz[baseline=(char.base)]{
		\node [shape=circle,draw,inner sep=1pt] (char) {#1};}}
\begin{tikzpicture}
% variables for pn-junction diagram:
% all parameters are in tikz scale
% p-side of the junction is here on the right

\def\V{1.5}   % junction polarisation (0=flat band)
\def\EG{3}    % band gap of semiconductor
\def\EF{1.5}  % vertical Fermi level position
\def\EFn{3.3} % pseudo fermi level for electrons
\def\EFp{1.8} % pseudo fermi level for holes
\def\DZCE{4}  % start position on the left for space charge region (SCR)
\def\LZCE{2}  % SCR width
\def\PN{10}   % total lentgh of the junction

% calculations
\pgfmathsetmacro\EC{\EG+\V};% conduction band heigth (without polarisation)
\pgfmathsetmacro\FZCE{\DZCE+\LZCE};% SCR end position

% valence and conduction band drawing: 
\draw (0,0) node [left]{$E_V$} -- (\DZCE,0)
to[out=0, in=180, looseness=0.75] (\FZCE,\V) -- (\PN,\V); % EV
\draw (0,\EG) node [left] {$E_C$} -- (\DZCE,\EG)
to[out=0, in=180, looseness=0.75] (\FZCE,\EC) -- (\PN,\EC); % EC

% fermi level drawing (if needed):		
%	\draw [dashed](0,\EF) -- ({\PN-0.5},\EF) node [right]{$E_F$}; % EF

% quasi fermi levels drawing (if needed) :	
\draw [dashed] (0,\EFn)  -- (\PN,\EFn)
node [right] {$E_{Fn}$}; % EFn for electron
\draw [dashed] (0,\EFp)  -- ({\PN},\EFp)
node [right] {$E_{Fp}$}; % EFp for holes

% electric field in SCR drawing :
\draw [->] (\DZCE, {\V+\EG+1}) --
node [above] {$\vec{E}_{ZCE}$} (\FZCE, {\V+\EG+1}) ; % E vector

% excess carriers
\foreach \x in {1,2,...,7}
\draw ({\FZCE+1+\x/3},{\EC+0.2}) node {$\bullet$}; % p side : electrons
\foreach \x in {1,2,...,7}
\draw ({1+\x/3},{-0.2}) node {$\circ$}; % n side : holes

% photon injection and carrier generation
% p side : carrier generation:
\draw [->, loosely dashed] ({\FZCE+3}, \V) --
node [right] {\textcircled{1}}({\FZCE+3}, \EC);
% the textcircled{number} option is used in several places
% to describe the physical mechanisms.
% It can be safely removed if not needed
% photon wave injection in the bandgap on p-side :
\draw [decorate, decoration={snake}, ->] ({\PN+1},{\EC+1}) --
node [below,sloped]{$h\nu$} ({\FZCE+3}, \EG); 

% excess carriers diffusion, with diffusion length :
% electrons on p side :
\draw [->] ({\FZCE+1},{\EC+0.2}) -- node [above] {$L_{Dn}$}
node [below=6pt] {\textcircled{2}}({\FZCE},{\EC+0.2})
node [left] {$\bullet$} ;
\draw [->] ({\FZCE-0.3},{\EC+0.2}) to [out=200, in=0, looseness=0.75]
node [above left] {\textcircled{3}} ({\DZCE},{\EG+0.2})
node [left] {$\bullet$} node [above left=3pt] {\textcircled{4}};
% holes on n side :
\draw [->] ({1.2+7/3},{-0.2}) -- node [below] {$L_{Dp}$} ({\DZCE},{-0.2})
node [right]{$\circ$} ;
\draw [->] ({\DZCE+0.3},-0.2) to [out=20, in=180, looseness=0.75]
({\FZCE},{\V-0.2}) node [right]{$\circ$};
\end{tikzpicture}

\theoremlisttype{allname}
\listtheorems{theorem}

\end{document}