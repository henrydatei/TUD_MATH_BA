\RequirePackage{ifluatex,ifpdf}
\documentclass[ngerman,a4paper]{report}
\usepackage[left=2.1cm,right=3.1cm,bottom=3cm]{geometry}
\usepackage[ngerman]{babel}
\ifpdf
\usepackage[utf8]{inputenc} %not recommended with lualatex
\usepackage[T1]{fontenc}
\fi

\usepackage{zref-base}
\usepackage{etoolbox}
\usepackage{xparse}%better macros
\usepackage{chngcntr}
\usepackage{calc}

\usepackage{scalerel,stackengine}
\usepackage{tocloft}

\ifluatex
\usepackage{fontspec}
%\usepackage{luacode}
\fi

\usepackage[texindy]{imakeidx}
\indexsetup{
	level=\chapter*
}
\makeindex[intoc]
\makeindex[name=semester1,title={Symbolverzeichnis (1. Semester)}]
\makeindex[name=semester2,title={Symbolverzeichnis (2. Semester)}]
\makeindex[name=symbols,title=Symbolverzeichnis,intoc]

\usepackage[xindy,acronym]{glossaries}
\makeglossaries

\usepackage[title,titletoc]{appendix}

\usepackage{amsmath}
\usepackage{amssymb}
\usepackage{amsfonts}
\usepackage{mathtools}
\usepackage{latexsym}
\usepackage{marvosym} %lighning
\usepackage{bbm} %unitary matrix 1
\usepackage{cancel}
\usepackage{xfrac}%sfrac -> fractions e.g. 3/4

\usepackage[table]{xcolor}
\usepackage{graphicx}
\usepackage{pgfplots}
\pgfplotsset{compat=1.10}
\usepgfplotslibrary{fillbetween}
\usepackage{pgf}
\usepackage{tikz}
\usetikzlibrary{patterns,arrows,calc,decorations.pathmorphing}
\usetikzlibrary{matrix}
\usepackage{color}
\usepackage{wasysym}

\usepackage{enumerate}
\usepackage{enumitem} %customize label
\usepackage{stmaryrd} % Lightning symbol

\usepackage{tabularx}
\usepackage{multirow}
\usepackage{booktabs}

\usepackage{ulem} %better underlines

\usepackage{parskip}%split paragraphs by vspace instead of intendations
\usepackage{fancyhdr}
\usepackage{titlesec}%customize titles
\usepackage{marginnote}

\usepackage[amsmath,amsthm,thmmarks,hyperref]{ntheorem}%customize theorem-environments more effectively
\usepackage[ntheorem,framemethod=TikZ]{mdframed}

\usepackage[unicode,bookmarks=true]{hyperref}
\hypersetup{
	colorlinks,
	citecolor=green,
	filecolor=green,
	linkcolor=blue,
	urlcolor=green
}
\usepackage{cleveref}
\usepackage{bookmark}

\newcommand{\coloredRule}[3][black]{\textcolor{#1}{\rule{#2}{#3}}}
\newlength{\blacktrianglewidth}
\settowidth{\blacktrianglewidth}{$\blacktriangleright$}

\definecolor{lightgrey}{gray}{0.91}
\definecolor{lightred}{rgb}{1,0.6,0.6}
\definecolor{darkgrey}{gray}{0.6}
\definecolor{darkgreen}{rgb}{0,0.6,0}

%numbered theorems
\theoremstyle{break}
\theorembodyfont{}

\mdfdefinestyle{boxedtheorem}{%
	outerlinewidth=3pt,%
	skipabove=5pt,%
	skipbelow=10pt,%
	frametitlefont=\normalfont\bfseries\color{black},%
}

\newmdtheoremenv[%
	style=boxedtheorem,%
	innertopmargin=\topskip,%
	innerbottommargin=\topskip,%
	linecolor=darkgrey,%
	backgroundcolor=lightgrey,%
]{theorem}{Theorem}[section]

\newmdtheoremenv[%
	style=boxedtheorem,%
	linecolor=darkgrey,%
	topline=false,%
	rightline=false,%
	bottomline=false,%
	innertopmargin=\topskip,%
	innerbottommargin=\topskip,%
	backgroundcolor=lightgrey,%
]{proposition}[theorem]{Satz}

\newmdtheoremenv[%
	style=boxedtheorem,%
	linecolor=darkgrey,%
	topline=false,%
	rightline=false,%
	bottomline=false,%
	backgroundcolor=lightgrey,%
	innertopmargin=\topskip,%
	innerbottommargin=\topskip,%
]{lemma}[theorem]{Lemma}

\newmdtheoremenv[%
	style=boxedtheorem,%
	linecolor=red,%
	topline=false,%
	rightline=false,%
	bottomline=false,%
	innertopmargin=0,%
	innerbottommargin=-3pt,%
]{definition}[theorem]{Definition}

\newmdtheoremenv[%
	outerlinewidth=3pt,%
	linecolor=black,%
	topline=false,%
	rightline=false,%
	bottomline=false,%
	innertopmargin=0pt,%
	innerbottommargin=-0pt,%
	frametitlefont=\normalfont\bfseries\color{black},%
	skipabove=5pt,%	
	skipbelow=10pt,%
]{conclusion}[theorem]{Folgerung}

\newmdtheoremenv[%
	hidealllines=true,%
	frametitlefont=\normalfont\bfseries\color{black},%
	innerleftmargin=0pt,%
	skipabove=5pt,%
	innerleftmargin=10pt,%
]{remark}[theorem]{\hspace*{-10pt}$\blacktriangleright$\hspace*{\dimexpr 10pt - \blacktrianglewidth\relax}Bemerkung}

\newmdtheoremenv[%
	hidealllines=true,%
	frametitlefont=\normalfont\bfseries\color{black},%
	innerleftmargin=10pt,%
]{example}[theorem]{\hspace*{-10pt}\rule{5pt}{5pt}\hspace*{5pt}Beispiel}

%unnumbered theorems
\theoremstyle{nonumberbreak}
\theoremindent0cm
\newmdtheoremenv[%
	style=boxedtheorem,%
	linecolor=red,%
	topline=false,%
	rightline=false,%
	bottomline=false,%
	innertopmargin=1pt,%
	innerbottommargin=1pt,%
]{*definition}{Definition}

\newmdtheoremenv[%
	hidealllines=true,%
	frametitlefont=\normalfont\bfseries\color{black},%
	skipabove=5pt,%
	innerleftmargin=10pt,%
]{*remark}{\hspace*{-10pt}$\blacktriangleright$\hspace*{\dimexpr 10pt - \blacktrianglewidth\relax}Bemerkung}

\newmdtheoremenv[%
	hidealllines=true,%
	innerleftmargin=10pt,%
]{*example}{\hspace*{-10pt}\rule{5pt}{5pt}\hspace*{5pt}Beispiel}
\newtheorem{overview}[theorem]{Überblick}

\newmdtheoremenv[%
	style=boxedtheorem,%
	topline=false,%
	rightline=false,%
	leftline=false,
	bottomline=false,%
	innertopmargin=\topskip,%
	innerbottommargin=\topskip,%
	backgroundcolor=lightgrey,%
]{*anmerkung}{Anmerkung}

%Hinweis-Theoremstyle and environment
%To get rid of the parentheses, a new theorem style is neccessary (definition of nonumberbreak from ntheorem.sty)
%to achieve the underlining, this needed to put in the theoremstyle definition
\theoremheaderfont{\mdseries}
\theoremseparator{:}
\theorempostskip{0pt}
\makeatletter
\newtheoremstyle{noparentheses}%
	{\item[\rlap{\vbox{\hbox{\hskip\labelsep \theorem@headerfont
					\underline{##1}\theorem@separator}\hbox{\strut}}}]}%
	{\item[\rlap{\vbox{\hbox{\hskip\labelsep \theorem@headerfont
					\underline{##1\ ##3\theorem@separator}}\hbox{\strut}}}]}
\newtheoremstyle{underlinedPlain}%
	{\item[\hskip\labelsep \uline{\theorem@headerfont ##1\theorem@separator}]}%
	{\item[\hskip\labelsep \uline{\theorem@headerfont ##1\ \theorem@headerfont(##3)\theorem@separator}]}
\newtheoremstyle{underlinedEnvironment}{}%
{\item[\hskip\labelsep \uline{##1\theorem@headerfont ##3\theorem@separator}]}
\newtheoremstyle{boldEnvironment}{}%
{\item[\hskip\labelsep \textbf{##1\theorem@headerfont ##3\theorem@separator}]}
\newtheoremstyle{proofstyle}%
{\item[\hskip\labelsep {\theorem@headerfont ##1}\theorem@separator]}%
{\item[\hskip\labelsep {\theorem@headerfont ##1}\ (##3)\theorem@separator]}
\makeatother

\theoremstyle{noparentheses}
\newmdtheoremenv[%
	hidealllines=true,%
	innerleftmargin=1em,%
	innerbottommargin=0pt,%
	innerrightmargin=0,%
	skipbelow=0pt,%
]{interpretation}{\hspace*{\dimexpr - \mdflength{innerleftmargin}\relax}Interpretation}
\theoremstyle{underlinedPlain}
\newmdtheoremenv[%
	hidealllines=true,%
	innerleftmargin=1em,%
	innerrightmargin=0,%
	skipbelow=0pt,%
]{hint}{\hspace*{\dimexpr - \mdflength{innerleftmargin}\relax}Hinweis}

\theoremstyle{underlinedEnvironment}
\newmdtheoremenv[%
	hidealllines=true,%
	innerleftmargin=1em,%
	innerrightmargin=0,%
	skipbelow=0pt,%
]{underlinedenvironment}{\hspace*{\dimexpr -\mdflength{innerleftmargin}\relax}}
\theoremheaderfont{\bfseries}
\theoremstyle{boldEnvironment}
\newmdtheoremenv[%
	hidealllines=true,%
	innerleftmargin=1em,%
	innerrightmargin=0,%
	skipbelow=0pt,%
]{boldenvironment}{\hspace*{\dimexpr -\mdflength{innerleftmargin}\relax}}

\theoremstyle{proofstyle}
\theoremheaderfont{\normalfont\normalsize\itshape}
\theorembodyfont{\normalfont\small}
\theoremseparator{.}
\theorempreskip{5pt}
\theorempostskip{5pt}
\theoremsymbol{$\square$}
\renewtheorem{proof}{Beweis}

%for \cref: printed environment names
\crefname{theorem}{Theorem}{Theoreme}
\crefname{proposition}{Satz}{Sätze}
\crefname{lemma}{Lemma}{Lemmata}
\crefname{conclusion}{Folgerung}{Folgerungen}
\crefname{definition}{Definition}{Definitionen}
\crefname{remark}{Bemerkung}{Bemerkungen}
\crefname{example}{Beispiel}{Beispiele}
\crefname{*definition}{Definition}{Definitionen}
\crefname{*remark}{Bemerkung}{Bemerkungen}
\crefname{*example}{Beispiel}{Beispiele}

\makeatletter
\newcommand*{\rom}[1]{\expandafter\@slowromancap\romannumeral #1@}
\newcommand*{\proplbl}[1]{%
	\@bsphack
	\begingroup
	\label{#1}%
	\zref@setcurrent{default}{\arabic{chapter}}%
%		\zref@wrapper@immediate{%
		\zref@labelbyprops{#1@chapter}{default}
%		}
	\endgroup
	\@esphack
}
\newcommand*{\propref}[1]{%
	\ifcsdef{r@#1}%in first compilation the label may not be defined yet
	{%
		\zref@refused{#1@chapter}%
		\ifnumcomp{\value{chapter}}{=}{\zref@extractdefault{#1@chapter}{default}{0}}%
		{%same chapter
			\ifmmode 
				\cref{#1}%
			\else
				\mbox{\cref{#1}}%
			\fi
		}%
		{%otherwise
			\def\propositionref@current@type{}%
			\cref@gettype{#1}{\propositionref@current@type}%get the environment's name
			%example for following line:
			%\crefformat{truetheorem}{\cref@truetheorem@name~##2\rom{\zref@extractdefault{#1}{#1chapter}{1}}.##1##3}
			%this changes the format used by \cref to <environtment name> <chapter-number>.<section-number>.<theorem number>
			\crefformat{\propositionref@current@type}{%
				\csname cref@\propositionref@current@type @name\endcsname ~##2\rom{\zref@extractdefault{#1@chapter}{default}{1}}.##1##3%
			}%
			\ifmmode 
				\cref{#1}%
			\else
				\mbox{\cref{#1}}%
			\fi
			\crefformat{\propositionref@current@type}{%
				\csname cref@\propositionref@current@type @name\endcsname~##2##1##3%
			}%
		}%
	}%
	{??}%similar to \ref\cref: question marks in case of undefined labels
}
\makeatother

\NewDocumentCommand{\begriff}{s O{} m O{}}{%
	\IfBooleanTF{#1}%
	{\index{#2#3#4}}%
	{%
		\uline{#3}%
		\ifnumcomp{\value{section}}{<}{16}%
		{\index[semester1]{#2#3#4}}%
		{\index[semester2]{#2#3#4}}%
		\index{#2#3#4}%
	}%
}
\NewDocumentCommand{\mathsymbol}{s O{} m m O{}}{%
	\IfBooleanTF{#1}%
	{\index[symbols]{#2#3@\detokenize{#4}#5}}%
	{#4\index[symbols]{#2#3@\detokenize{#4}#5}}%
}
\NewDocumentCommand{\zeroAmsmathAlignVSpaces}{s s O{0 pt} O{0 pt}}{%
	\IfBooleanTF{#1}%
	{%
		\IfBooleanTF{#2}%
			{\setlength{\belowdisplayskip}{#4}}%
			{\setlength{\abovedisplayskip}{#3}}%
	}%
	{%
		\setlength{\abovedisplayskip}{#3}%
		\setlength{\belowdisplayskip}{#4}%
	}%
}

\NewDocumentCommand{\transpose}{m}{\ensuremath{#1^\mathsf{T}}}

\NewDocumentCommand{\itemEq}{s m}{%
	\begingroup%
	\setlength{\abovedisplayskip}{\dimexpr -\parskip + 1pt\relax}%
	\setlength{\belowdisplayskip}{0pt}%
	\IfBooleanTF{#1}%
		{\parbox[c]{\linewidth}{\begin{flalign*}#2&&\end{flalign*}}}%}
		{\parbox[c]{\linewidth}{\begin{flalign}#2&&\end{flalign}}}%}
	\endgroup% 
}
\newcommand\equalhat{\mathrel{\stackon[1.5pt]{=}{\stretchto{%
	\scalerel*[\widthof{=}]{\wedge}{\rule{1ex}{3ex}}}{0.5ex}}}}

\makeatletter
\newcommand{\leqnos}{\tagsleft@true\let\veqno\@@leqno}
\newcommand{\reqnos}{\tagsleft@false\let\veqno\@@eqno}
\reqnos

\pdfstringdefDisableCommands{%
	\def\\{}%
	\def\texttt#1{<#1>}%
	\def\mathbb#1{<#1>}%
}
\makeatother

%General newcommands!
\newcommand{\comp}{\mathbb{C}} % complex set C
\newcommand{\real}{\mathbb{R}} % real set R
\newcommand{\whole}{\mathbb{Z}} % whole number Symbol
\newcommand{\natur}{\mathbb{N}} % natural number Symbol
\newcommand{\ratio}{\mathbb{Q}} % rational number symbol
\newcommand{\field}{\mathbb{K}} % general field for the others above!
\newcommand{\diff}{\mathrm{d}} % differential d
\newcommand{\s}{\,\,}     % space after the function in the intergral
\newcommand{\cont}{\mathcal{C}} % Contour C
\newcommand{\fuk}{f(z) \s\diff z} % f(z) dz
\newcommand{\diffz}{\s\diff z}
\newcommand{\subint}{\int\limits} % lower boundaries for the integral
\newcommand{\poly}{\mathcal{P}} % special P - polygon
\newcommand{\defi}{\mathcal{D}} % D for the domain of a function
\newcommand{\cover}{\mathcal{U}} % cover for a set
\newcommand{\setsys}{\mathcal{M}} % set system M
\newcommand{\setnys}{\mathcal{N}} % set system N
\newcommand{\zetafunk}{f(\zeta)\s\diff \zeta} %f(zeta) d zeta
\newcommand{\ztfunk}{f(\zeta)} % f(zeta)
\newcommand{\bocirc}{S_r(z)}
\newcommand{\prop}{\,|\,}
\newcommand*{\QEDA}{\hfill\ensuremath{\blacksquare}} %tombstone
\newcommand{\emptybra}{\{\varnothing\}} % empty set with set-bracket
\newcommand{\realpos}{\real_{>0}}
\newcommand{\realposr}{\real_{\geq0}}
\newcommand{\naturpos}{\natur_{>0}}
\newcommand{\Imag}{\operatorname{Im}} % Imaginary symbol
\newcommand{\Realz}{\operatorname{Re}} % Real symbol
\newcommand{\norm}{\Vert \cdot \Vert}
\newcommand{\metric}{\vert \cdot \vert}
\newcommand{\foralln}{\forall n} %all n
\newcommand{\forallnset}{\forall n \in \natur} %all n € |N
\newcommand{\forallnz}{\forall n \geq _0} % all n >= n_0
\newcommand{\conjz}{\overline{z}} % conjugated z
\newcommand{\tildz}{\tilde{z}} % different z
\newcommand{\lproofar}{"`$ \Leftarrow $"'} % "`<="'
\newcommand{\rproofar}{"`$ \Rightarrow $"'} % "`=>"'
\newcommand{\beha}{\Rightarrow \text{ Behauptung}}
\newcommand{\powerset}{\mathcal{P}}
\newcommand{\person}[1]{\textsc{#1}}
\newcommand{\highlight}[1]{\emph{#1}}
\newcommand{\realz}{\mathfrak{Re}}
\newcommand{\imagz}{\mathfrak{Im}}
\renewcommand{\epsilon}{\varepsilon}
\renewcommand{\phi}{\varphi}
\newcommand{\lebesque}{\person{Lebesgue}}
\renewcommand{\Re}{\mathfrak{Re}}
\renewcommand{\Im}{\mathfrak{Im}}
\renewcommand*{\arraystretch}{1.4}

% Math Operators
\DeclareMathOperator{\inn}{int} % Set of inner points
\DeclareMathOperator{\ext}{ext} % Set of outer points
\DeclareMathOperator{\cl}{cl} % Closure
\DeclareMathOperator{\grad}{grad}
\DeclareMathOperator{\D}{d}
\DeclareMathOperator{\id}{id}
\DeclareMathOperator{\graph}{graph}
\DeclareMathOperator{\Int}{int}
\DeclareMathOperator{\Ext}{ext}
\DeclareMathOperator{\diam}{diam}

%change headings:
\titlelabel{\thetitle.\quad}%. behind section/sub... (3. instead of 3)
\counterwithout{section}{chapter}
\renewcommand{\thechapter}{\Roman{chapter}}
\renewcommand{\thepart}{\Alph{part}}
%italic chapters (due to titlesec package some more stuff)
%\titleformat{command}[shape]{format}{label}{sep}{before-code}[after-code]
\titleformat{\chapter}[display]{\bfseries}{\Large\chaptername\;\thechapter}{-5pt}{\huge\bfseries\itshape}
\titlespacing{\chapter}{0pt}{0pt}{10pt}
\titleformat{\section}[hang]{\bfseries\Large}{\thesection.}{8pt}{\Large\bfseries}
%\titlespacing{command}{left}{before-sep}{after-sep}
\titlespacing{\subsection}{0pt}{0pt}{5pt}

%change appearence of heading of toc: 0 space above, bold, italic huge toc-heading
\renewcommand{\cftbeforetoctitleskip}{0pt}
\renewcommand{\cfttoctitlefont}{\itshape\Huge\bfseries}
%change indentations due to width of capital roman numbers
\renewcommand{\cftchapnumwidth}{2.5em}
\renewcommand{\cftsecindent}{2.5em}
%\renewcommand{\cftsecnumwidth}{3.3em}
\renewcommand{\cftsubsecindent}{4.8em}
%\renewcommand{\cftsubsecnumwidth}{4.2em}

%change header:
\renewcommand{\headrulewidth}{0.75pt}
\renewcommand{\footrulewidth}{0.3pt}
\lhead{\rightmark}%left: section-number. section-title
\rhead{\leftmark}%right: chapter chapternumber: chapter-title

% Add new page-style (just footer), patch \chapter command to use this page style
\fancypagestyle{plainChapter}{%
	\fancyhf{}%
	\fancyfoot[C]{\thepage}%
	\renewcommand{\headrulewidth}{0pt}% Line at the header invisible
	\renewcommand{\footrulewidth}{0.4pt}% Line at the footer visible
}
\patchcmd{\chapter}{\thispagestyle{plain}}{\thispagestyle{plainChapter}}{}{}

\pagestyle{fancy}
\pagenumbering{arabic}
%remember chapter-title in \leftmark and \rightmark
\renewcommand{\chaptermark}[1]{%
	\markboth{\chaptername
		\ \thechapter:\ #1}{}}
%remember section title in \leftmark
\renewcommand{\sectionmark}[1]{%
	\markright{\thesection.\ #1}{}}

%change numbering of equations to be section by section
\counterwithout{equation}{section}

\newacronym{gdw}{gdw.}{genau dann wenn}
\newacronym{fa}{fa.}{fast alle}
\newacronym{fü}{f.ü.}{fast überall}
\newacronym{obda}{oBdA}{ohne Beschränkung der Allgemeinheit}
\newacronym{tf}{TF}{\begriff{Teilfolge}}
\newacronym{hw}{Hw}{\begriff{Häufungswert}}
\newacronym{cf}{CF}{\begriff{\person{Cauchy}-Folge}}
\newacronym{hp}{HP}{\begriff{Häufungspunkt}}
\newacronym{vr}{VR}{Vektorraum}
\newacronym{diffbar}{diffbar}{differenzierbar}
\newacronym{mws}{MWS}{Mittelwertsatz}

\title{\textbf{Analysis (WS2017/18 + SS2018)}}
\author{Dozent: Prof. Dr. Friedemann Schuricht\\
	Kursassistenz: Moritz Schönherr}

%remove page number from part{}-pages
\makeatletter
\let\sv@endpart\@endpart
\def\@endpart{\thispagestyle{empty}\sv@endpart}
\makeatother

\begin{document}
\pagenumbering{roman}
\pagestyle{plain}

\maketitle

\hypertarget{tocpage}{}
\tableofcontents
\bookmark[dest=tocpage,level=1]{Inhaltsverzeichnis}

\pagebreak
\pagestyle{fancy}
\pagenumbering{arabic}

\part{1. Semester}
\pagestyle{fancy}
\chapter{Grundlagen der Mathematik}
\section{Grundbegriffe aus Logik und Mengenlehre}

\textbf{Mengenlehre:} Universalität von Aussagen, Verwendung von Mengen \\
\textbf{Logik:} Regeln des Folgerns, wahre und falsche Aussagen \\
$\to$ hier werden einige Aspekte etwas vereinfacht, aber ausreichend genug behandelt

\begin{definition}[Aussage]
	\begriff{Aussage} ist ein Schverhalt, dem man entweder den Warheitswert wahr ($w$) oder falsch ($f$) zuordnen kann (und nichts anderes).
\end{definition}

\begin{example}
	\begin{itemize}
		\item 5 ist eine Quadratzahl (Aussage) $\to$ falsch
		\item Die Elbe fließt durch Dresden (Aussage) $\to$ wahr
		\item Mathematik ist rot (keine Aussage)
	\end{itemize}
\end{example}
	
\begin{definition}[Menge]
	\begriff{Menge} ist (nach Cantor 1877) eine Zusammenfassung von bestimmten, wohlunterschiedenen Objekten der Anschauung oder des Denkens, welche die \begriff{Elemente} der Menge genannt werden, zu einem Ganzen.
\end{definition}

\begin{example}
	\begin{itemize}
		\item $M_1=$ Menge aller Städte in Deutschland
		\item $M_2=\{1,2,3\}$
	\end{itemize}
\end{example}

\begin{definition}
	\begin{itemize}
		\item $M=N$, falls dieselben Elemente enthalten sind
		\item $N$\mathsymbol{c}{$\subset$}$M$ (\begriff{Teilmenge}), falls $n\in M$ für jedes $n\in\mathbb{N}$
		\item $N$\mathsymbol{c=}{$\subsetneqq$}$M$ (\begriff{echte Teilmenge}), falls zusätzlich $N\neq M$.
		\item \begriff{Aussageform}: Sachverhalt mit Variablen, der durch geeignete Ersetzung der Variablen zur Aussage führt
	\end{itemize}
\end{definition}

\begin{example}
	\begin{itemize}
		\item $A(X)=$ Die Elbe fließt durch $X$
		\item $B(X,Y,Z)=X+Y=Z$
		\item $\to$ $A(\text{Dresden})$ und $B(2,3,4)$ sind Aussagen
		\item $\to$ $A(\text{Mathematik})$ ist keine Aussage
		\item $\to$ $A(X)$ ist Aussage für jedes $X\in M_1$
	\end{itemize}
\end{example}

\begin{center}\begin{tabular}{|c c|c c c c c|}
	\hline 
	$A$& $B$ & $\neg A$ & $A\land B$ & $A\lor B$ & $A\Rightarrow B$ & $A\iff B$ \\ 
	\hline 
	W& W & F & W & W & W & W \\ 
	W& F & F & F & W & F &  F\\ 
	F& W & W & F & W & W & F \\ 
	F& F & W & F & F & W &  W\\ 
	\hline 
\end{tabular}\end{center}

\begin{example}
	\begin{itemize}
		\item $\neg$(3 ist gerade) - wahr
		\item (4 ist gerade) $\land$ (4 ist Primzahl) - falsch
		\item (3 ist gerade) $\lor$ (3 ist Primzahl) - wahr
		\item (Sonne ist heißt) $\Rightarrow$ (Es gibt Primzahlen) - w
		\item (3 ist gerade) $\iff$ ($\pi\in\natur$) - w
		\item Ausschließendes oder wird realisiert durch $\neg(A\iff B)$
	\end{itemize}
\end{example}

\begin{definition}[Quantoren]
	Neue Aussagen können mittels \begriff{Quantoren} gebildet werden:
	\begin{itemize}
		\item $\forall x\in M: A(x)$ wahr \gls{gdw} $A(x)$ wahr für jedes $x\in M$
		\item $\exists x\in M: A(x)$ wahr \gls{gdw} $A(x)$ wahr für mindestens ein $x\in M$
	\end{itemize}
\end{definition}

\begin{example}
	\begin{itemize}
		\item $\foralln\in\natur:n$ ist gerade - f
		\item $\exists n\in\natur:n$ ist gerade - w
	\end{itemize}
\end{example}

\begin{definition}[Tautologie, Kontraduktion]
	\begriff{Tautologie} bzw. \begriff{Kontradiktion}\slash\begriff{Widerspruch} ($\lightning$) ist zusätzlich gesetzte Aussage, die unabhängig vom Wahrheitswert der Teilaussagen stets wahr bzw. falsch ist.
\end{definition}

\begin{example}
	\begin{itemize}
		\item Tautologien: $A\lor\neg A$, $\neg(A\land\neg A)$, $(A\land B)\Rightarrow A$
		\item Widerspruch: $A\land\neg A$, $A\iff\neg A$
		\item besondere Tautologie: $(A\Rightarrow B)\iff (\neg B\Rightarrow \neg A)$
	\end{itemize}
\end{example}

\begin{proposition}[\person{de Morgan}'sche Regeln]
	Folgende Aussagen sind stets Tautologien
	\begin{enumerate}[label={\alph*)}]
		\item $\neg(A\land B) \Leftrightarrow \neg A \lor \neg B$
		\item $\neg(A\lor B) \Leftrightarrow \neg A\land \neg B$
		\item $\neg (\forall x\in M: A(x)) \;\Leftrightarrow \; \exists x\in M:\neg A(x)$
		\item $\neg (\exists x\in M: A(x)) \;\Leftrightarrow \;\forall x\in M:\neg A(x)$
	\end{enumerate}
\end{proposition}

\begin{definition}
	\begin{itemize}
		\item \begriff{leere Menge} \mathsymbol{o}{$\emptyset$}$=:$ Menge, die kein Element enthält
		\item $M,N$ sind \begriff[Menge!]{disjunkt}, falls $M\cap N = \emptyset$
		\item Sei $\mathcal{M}$ \begriff{Mengensystem}, d.h. Mengen von Mengen, dann
		\begin{itemize}
			\item $\bigcup_{M\in\mathcal{M}} M := \{x \mid \exists M\in\mathcal{M}: x\in M\}$
			\item $\bigcap_{M\in\mathcal{M}} M:= \{ x\mid\forall M\in\mathcal{M}: x\in M \}$
		\end{itemize}
		\item \begriff{Potenzmenge}: \mathsymbol{p}{$\mathcal{P}$}$(XM):=\{\tilde{M} | \tilde{M}\in M\}$
		\item \begriff{\person{de Morgan}'sche Regeln} (für $\mathcal{N}\subset\mathcal{P}(M)$)
		\begin{itemize}
			\item $\left(\bigcup_{N\in\mathcal{N}} N\right)^C = \bigcap_{N\in\mathcal{N}} N^C$
			\item $\left(\bigcap_{N\in\mathcal{N}} N\right)^C = \bigcup_{N\in\mathcal{N}} N^C$
		\end{itemize}
		\item \begriff{kartesisches Produkt} $M$\mathsymbol{x}{$\times$}$N:=\{(m,n) | m\in M \text{ und } n\in N\}$
		\item $(m_1, \dotsc, m_n)$ ist \begriff{n-Tupel}
		\item \begriff{Auswahlaxiom} (AC / axiom of choice)
		
		Sei $\mathcal{M}$ Menge nichtleerer, paarweise disjunkter Mengen $M$\\
		$\Rightarrow$ es gibt immer (Auswahl-) Menge $\tilde{M}$, die mit jedem $M\in\mathcal{M}$ genau ein Element gemein hat.
	\end{itemize}
\end{definition}

\subsection{Aufbau einer mathematischen Theorie}

Axiome (als wahr angenommene Aussagen) $\to$ Beweise $\to$ Sätze ("'neue"' wahre Aussagen) \\
$\Rightarrow$ ergibt Ansammlung (Menge) wahrer Aussagen \\
$\newline$

Formulierung mathematischer Aussagen:
\begin{itemize}
	\item typische Form eines mathematischen Satzes: $\underbrace{\text{Wenn A gilt,}}_\text{Vorraussetzung}\underbrace{\text{dann folgt B}}_\text{Behauptung}$
	\item formal: $A\Rightarrow B$
\end{itemize}

\begin{example}
	\begin{itemize}
		\item $n\in\natur$ ist durch 4 teilbar $\Rightarrow$ $n$ ist durch 2 teilbar
		\item genauer meint man sogar $A\land C\Rightarrow B$, wobei $C$ aus allen bekannten 
		wahren Aussagen besteht
		\item $B$ ist \begriff[Bedingung!]{notwendig} für $A$
		\item $A$ ist \begriff[Bedingung!]{hinreichend} für $B$
	\end{itemize}
\end{example}

\begin{*anmerkung}
	Aus dem Wikipedia-Artikel zu notwendiger und hinreichender Bedingung:
	\begin{itemize}
		\item notwendige Bedingung: Wenn $B$ wahr ist, dann muss auch $A$ wahr sein. Es kann nicht sein, dass $B$ wahr ist, ohne dass $A$ wahr ist. 
		\item Beispiel: Für jede Primzahl $>2$ gilt: Sie ist ungerade. Also: ist die Eigenschaft "'Primzahl"' 
		notwendig für die Eigenschaft "'ist ungerade"', denn es gibt keine Primzahl, die gerade ist.
		\item hinreichende Bedingung: Eine hinreichende Bedingung sorgt für das Eintreten des Ereignisses. Wenn die Bedingung nicht notwendig, sondern nur hinreichend ist, dann gibt es andere hinreichende Bedingungen, die zum Eintreten des Ereignisses führen.
		\item Cola trinken ist nicht notwendig zum überleben, da man auch Wasser trinken kann.
	\end{itemize}
\end{*anmerkung}

\begin{definition}[direkter Beweis, indirekter Beweis]
	\begin{itemize}
		\item \begriff[Beweis!]{direkt}\highlight{er Beweis}: $(A\Rightarrow A_1)\land(A_1\Rightarrow A_2)\land\dotsc\land(A_n\Rightarrow B)$ wahr für $A\Rightarrow B$
		\item \begriff[Beweis!]{indirekt}\highlight{er Beweis} durch Tautologie $(A\Rightarrow B)\Leftrightarrow (\neg B\rightarrow \neg A)$
	\end{itemize}
\end{definition}

\subsection{Relation und Funktion}
\begin{definition}[Relation]
	\begin{itemize}
		\item \begriff{Relation} ist Teilmenge $R\subset M\times N$. $(x,y)\in R$ heißt: $x$ und $y$ stehen in Relation zueinander.
		\item Relation $R\subset M\times N$ heißt \begriff{Ordnungsrelation} (kurz \begriff{Ordnung}) auf $M$, falls $\forall a,b,c\in M$:
		\begin{enumerate}[label={\alph*)}]
			\item $(a,a)\in R$ (\begriff[Ordnung!]{reflexiv})
			\item $(a,b),(b,a)\in R \rightarrow a=b$ (\begriff[Ordnung!]{antisymmetrisch})
			\item $(a,b),(b,c)\in R \rightarrow (a,c)\in R$ (\begriff[Ordnung!]{transitiv})
		\end{enumerate}
		\item Ordnungsrelation $R$ auf $M$ heißt \begriff{Totalordnung}, falls $\forall a,b\in M: (a,b)\in R \lor (b,a)\in R$
		\item Relation auf $M$ heißt \begriff{Äquivalenzrelation}, falls $\forall a,b,c\in M$:
		\begin{enumerate}[label={\alph*)}]
			\item $(a,a)\in R$ (\begriff[Ordnung!]{reflexiv})
			\item $(a,b)\in R \Rightarrow (b,a)\in R$ (\begriff[Ordnung!]{symmetrisch})
			\item $(a,b),(b,c)\in R \Rightarrow (a,c)\in R$ (\begriff[Ordnung!]{transitiv})
		\end{enumerate}
		\item \mathsymbol{[a]}{$[a]$}$:=\{b\in M\mid (a,b)\in R\}$ heißt \begriff{Äquivalenzklasse} von $a\in M$ bzgl. $R$
		
		Jedes $b\in [a]$ ist ein \begriff{Repräsentant} von $[a]$
	\end{itemize}
\end{definition}

\begin{example}
	\proplbl{beispiel_brueche_aequivalenzklassen}
	$B=\left\lbrace \frac{m}{n}\mid m,n\in\whole,n\neq 0 \right\rbrace$ Menge der Brüche \\
	man hat Äquivalenzrelation auf $B$ mit $R=\left\lbrace\left( \frac{m}{n},\frac{p}{q}\right)\in B\times B \mid mq=np\right\rbrace$ \\
	beachte: Menge der Äquivalenzklassen $\left\lbrace\left[ \frac{m}{n}\right]\mid \frac{m}{n}\in B \right\rbrace$ ist die Menge der rationalen Zahlen
\end{example}

\begin{*anmerkung}
	\begin{itemize}
		\item Mit einer Ordungsrelation kann man eigentlich unordenbare Dinge wie Funktionen (gilt $x^2 < x^3$ oder $x^2 > x^3$?) ordnen.
		\item Eine Äquivalenzrelation ist eine Art Gleichheitszeichen, nur eben für mathematische Objekte, die 
		keine Zahen sind.
		\item zu \propref{beispiel_brueche_aequivalenzklassen}: Zwei Brüche $\frac{m}{n}$ und $\frac p q$ 
		sind gleich, wenn $mq=np$, d.h. diese zwei Brüche gehören zu einer Äquivalenzklasse. So 
		gehören die Brüche $\frac 2 3$ und $\frac 4 6$ zu einer Äquivalenzklasse, nämlich zu $\left[ \frac 
		2 3\right]$, da $2\cdot 6=12=3\cdot 4$. Alle Äquivalenzklassen, also alle nicht mehr kürzbaren Brüche 
		ergeben dann die rationalen Zahlen $\ratio$.
	\end{itemize}
\end{*anmerkung}

\begin{definition}[Abbildung]
	\begriff{Abbildung}/\begriff{Funktion} von $M$ nach $N$, kurz: $F:M\rightarrow N$ ist Vorschrift, die jedem \begriff{Argument} / \begriff{Urbild} $m\in M$ genau einen \begriff{Wert} / \begriff{Bild} $F(m)\in N$ zuordnet.
	
	\begin{itemize}
		\item \mathsymbol{D}{$\mathcal{D}$}$(F):=M$ heißt \begriff{Definitionsbereich} / \begriff{Urbildmenge}
		\item $N$ heißt \begriff{Zielbereich}
		\item $F(M'):=\{n\in N \mid n=F(m)$ für ein $m\in M'\}$ ist \begriff{Bild}\highlight{ von $M'$}$\subset M$
		\item $F^{-1}(N'):=\{ m\in M\mid n=F(m)$ für ein $N' \}$ ist \begriff{Urbild}\highlight{ von $N'$}$\subset N$
		\item \mathsymbol{R}{$\mathcal{R}$}$(F):= F(M)$ heißt \begriff{Wertebereich} / \begriff{Bildmenge}
		\item \mathsymbol{graph}{$\graph$}$(F) :=\{ (mn,)\in M\times N | n = F(m)\}$ heißt \begriff{Graph}\highlight{von $F$}
		\item \mathsymbol{fm}{$F|_{M'}$} ist \begriff{Einschränkung}\highlight{ der Funktion} von $F$ auf $M'\subset M$
		\item Zwei Funktionen $F$ und $G$ sind gleich, wenn
		\begin{itemize}
			\item $\mathcal{D}(F)=\mathcal{D}(G)$
			\item $F(m)=G(m)\quad\forall m\in\mathcal{D}(F)$
		\end{itemize}
		\item \begriff{Komposition} von $F:M\rightarrow N$ und $G:N\rightarrow P$ ist Abbildung $G$\mathsymbol{o}{$\circ$}$F:M\rightarrow P$ mit $(G\circ F)(m):=G(F(m))$
		\item Abbildung $F:M\rightarrow N$ heißt
		\begin{itemize}
			\item \begriff[Abbildung!]{injektiv}, falls eineindeutig (d.h. $F(m_1) = F(m_2) \Rightarrow m_1 = m_2$)
			\item \begriff[Abbildung!]{surjektiv}, falls $F(M) = N$, d.h. $\forall n\in N\,\exists m\in M: F(m) = n$
			\item \begriff[Abbildung!]{bijektiv}, falls injektiv und surjektiv
		\end{itemize}
		\item Für bijektive Abb. $F:M\rightarrow N$ ist \begriff{Umkehrabbildung} / \begriff{inverse Abbildung} \mathsymbol{f-1}{$F^{-1}$}$:N\rightarrow M$ definiert durch $F^{-1}(n) = m \Leftrightarrow F(m) = n$
	\end{itemize}
\end{definition}

\begin{example}
	betrachte $f:\real\to\real$ und $f(x)=\sin(x)$ \\
	Zielmenge: $\real$, aber Wertebereich $[-1,1]$!
\end{example}

\stepcounter{theorem}
\begin{proposition}
	Sei $F:M\rightarrow N$ surjektiv. Dann existiert Abbildung $G:N\rightarrow M$, sodass $F\circ G = \id_N$ (d.h. $F(G(n)) = n\,\forall n\in N$)
\end{proposition}

\begin{definition}[Verknüpfung]
	Eine \begriff{Rechenoperation} / \begriff{Verknüpfung} auf $M$ ist Abb. $*:M\times M\rightarrow M$, d.h. $m,n\in M$ wird \begriff{Ergebnis} $m*n\in M$
	
	Rechenoperation
	\begin{itemize}
		\item hat \begriff[Verknüpfung!]{neutrales Element} $e\in M$, falls $m*e = e*m = m\,\forall m\in M$
		\item ist \begriff[Verknüpfung!]{kommutativ}, falls $m*n = n*m$
		\item ist \begriff[Verknüpfung!]{assoziativ}, falls $k*(m*n) = (k*m)*n\,\forall k,m,n\in M$
		\item hat \begriff[Verknüpfung!]{inverses Element} $m'\in M$ zu $m\in M$, falls $m*m' = m'*m = e$
	\end{itemize}
\end{definition}

\begin{example}
	\begin{itemize}
		\item \begriff{Addition}: $(m,n)\mapsto: m+n$ \begriff{Summe},
		\begin{itemize}
			\item neutrales Element heißt \begriff{Null} / \begriff{Nullelement}
			\item Inverses Element: \mathsymbol{-}{$-m$}
		\end{itemize}
		\item \begriff{Multiplikation} $\cdot:(m,n)\mapsto: m\cdot n$ \begriff{Produkt}
		\begin{itemize}
			\item neutrales Element heißt \begriff{Eins} / \begriff{Einselement}
			\item Inverses Element:\mathsymbol{-1}{$m^{-1}$}
		\end{itemize}
	\end{itemize}
\end{example}

\begin{definition}[distributiv]
	Addition und Multiplikation heißen \begriff{distributiv}, falls $k\cdot(m+n) = k\cdot m + k\cdot n\,\forall k,m,n\in M$
\end{definition}

\begin{definition}[Körper]
	Menge $K$ heißt \begriff{Körper}, falls auf $K$ eine Addition und Multiplikation existiert mit
	\begin{itemize}
		\item es existieren neutrale Elemente $0\in K$ und $1\in K_{\neg 0}$
		\item Addition und Multiplikation sind distributiv
		\item Es gibt Inverse
	\end{itemize}
\end{definition}

\begin{definition}
	Menge $M$ habe Ordnung "`$\le$"', sowie Addition und Multiplikation.
	Ordnung ist \begriff[Ordnung!]{verträglich}\highlight{ mit Addition und Multiplikation}, wenn $\forall a,b,c\in M$
	\begin{itemize}
		\item $a\le b \Leftrightarrow a+c \le b+c$
		\item $a\le b \Leftrightarrow a\cdot c \le b\cdot c$ mit $c > 0$
	\end{itemize}
\end{definition}

\begin{definition}[angeordnet]
	Körper $K$ heißt \begriff[Körper!]{angeordnet}, falls mit Addition und Multiplikation verträgliche Totalordnung existiert.
\end{definition}

\begin{definition}[Isomorphismus]
	\begriff{Isomorphismus} bezüglich einer Struktur ist bijektive Abbildung $I:M_1\rightarrow M_2$, die auf $M_1$ und $M_2$ vorhandene Struktur erhält, z.B. 
	\begin{itemize}
		\item Ordnung: $a\le b\iff I(a)\le I(b)$
		\item Rechenoperationen: $I(a*b)=I(a)*I(b)$
	\end{itemize}
	Mengen $M_1$ und $M_2$ heißen \begriff[Menge!]{isomorph}.
\end{definition}

\begin{*anmerkung}
	Mit einem Isomorphismus kann man die Elemente einer Menge, z.B. ganze Zahlen, den Elementen einer anderen Menge, z.B. den natürlichen Zahlen, zuordnen. Konkret würde das dann so aussehen: \\
	$0\mapsto 0$, $1\mapsto 1$, $-1\mapsto 2$, $2\mapsto 3$, $-2\mapsto 4$, ... \\
	Insbesondere wenn es darum geht, ob die ganzen Zahlen abzählbar sind, also ob ich diese mit den natürlichen Zahlen neu durchnummerieren kann, ist ein solcher Isomorphismus (denn dieses "'neunummerieren"' ist ein Isomorphismus) notwendig. Alle Aussagen, die die Struktur betreffen, z.B. die Kommutativität, bleiben erhalten und müssen nicht neu bewiesen werden.
\end{*anmerkung}

\begin{example}
	$M_1=\natur$, $M_2=\{\text{gerade Zahlen}\}$ jeweils mit Addition, Multiplikation, Ordnung \\
	$\Rightarrow I: M_1\to M_2$ mit $I(n)=2n$ ist ein Isomorphismus, denn alle geraden Zahlen werden 
	einfach nur neu durchgezählt \\
	$\Rightarrow$ Isomorphismus erhält Addtion, Ordnung und die 0, aber nicht die Multiplikation, da
	$I(a)*I(b)=2a*2b=4ab$ aber $I(a*b)=2(a*b)=2ab$, also $I(a)*I(b)\neq I(a*b)$
\end{example}

\subsection{Bemerkungen zum Fundament der Mathematik}

Forderungen an eine mathematische Theorie
\begin{itemize}
	\item widerspruchsfrei: Satz und seine Negation sind nicht gleichzeitig herleitbar
	\item vollständig: alle Aussagen innerhalb einer Theorie sind als wahr oder falsch beweisbar
\end{itemize}

2 Unvollständigkeitssätze
\begin{itemize}
	\item jedes System ist nicht gleichzeitig widerspruchsfrei und vollständig
	\item in einem System kann man nicht die eigenen Widerspruchsfreiheit zeigen
\end{itemize} % +Fundament der Mathematik
\chapter{Zahlenbereiche}
\section{Natürliche Zahlen}
\begin{*definition}[Peano Axiome]
$\mathbb{N}$ sei Menge, die die \begriff{\person{Peano}-Axiome} erfüllen, d.h.
\begin{enumerate}[label={P\arabic*)}]
	\item $\mathbb{N}$ sei indutkiv, d.h. es ex.
	\begin{itemize}
		\item Nullelement $0\in \mathbb{N}$ und
		\item injektive (Nachfolger-) Abb. \mathsymbol{nu}{$\nu$}$:\mathbb{N}\rightarrow\mathbb{N}$ mit $\nu(n)\neq 0\,\forall n\in \mathbb{N}$
	\end{itemize}
	\item (Induktionsaxiom)
	
	Falls $N\subset\mathbb{N}$ induktiv in $\mathbb{N}$ (d.h. $0,\nu(n)\in\mathbb{N}$ falls $n\in\mathbb{N}$)\\
	$\Rightarrow N=\mathbb{N}$ ($N$ ist die kleinste indutkive Menge)
\end{enumerate}

Nach Mengenlehre ZF existiert eine Solche Menge der \begriff{natürliche Zahlen} mit üblichen Symbolen.
\end{*definition}

\begin{theorem}
	\proplbl{naturliche_zahlen_isomorph}
	Falls $\mathbb{N}$ und $\mathbb{N}*$ \person{Peano}-Axiome erfüllen, dann sind sie isomorph bezüglich Nachfolger-Abbildung und Nullelement (Anfangselement).
\end{theorem}

\begin{proposition}[Prinzip der vollständigen Induktion] \proplbl{prin_voll_induktion}
	\begriff*{vollständigen Induktion}
	Sei $\{A_n \mid n\in\mathbb{N}\}$ Aussagenmenge mit d. Eigenschaften
	\begin{itemize}
		\item[(IA)] $A_0$ ist wahr (\begriff{Induktionsanfang})
		\item[(IS)] $\forall n\in\mathbb{N}$ gilt: $A_n$ (wahr) $\Rightarrow A_{n+1}$
	\end{itemize}
	$\Rightarrow A_n$ ist wahr $\forall n\in\mathbb{N}$
\end{proposition}

\begin{proof}
	Sei $N := \{n \in \mathbb{N} \mid A_n \text{ ist wahr}\} \subset \mathbb{N}$, offenbar $0 \in \mathbb{N}$ und $\nu(n) \in \mathbb{N}$, falls $n \in \mathbb{N} \Rightarrow \mathbb{N}$ induktiv in $\mathbb{N} \overset{\text{P2)}}{\Rightarrow} N = \mathbb{N}$
\end{proof}

\begin{lemma}
	\proplbl{lemma_nachfolgerabb}
	Es gilt:
	\begin{enumerate}[label={\alph*)}]
		\item $\nu(\mathbb{N})\cup \{0\}=\mathbb{N}$
		\item $\nu(n)\neq n\,\forall n\in\mathbb{N}$
	\end{enumerate}
\end{lemma}

\begin{proof}
	\begin{itemize}
		\item[a)] $N := \{n \in \mathbb{N} \mid n = \nu(m) \text{ für } n\in \mathbb{N} \} \cup \{ 0 \}$ ist induktiv in $\mathbb{N} \overset{\text{P2)}}{\Rightarrow} N = \mathbb{N}$
		\item[b)] Beweis mittels vollständiger Induktion
		\begin{itemize}
			\item[(IA)] $\nu(0) \neq 0$ nach P1)
			\item[(IS)] Zeige: $(\nu(n) \overset{\text{IV)}}{\neq} n \Rightarrow \nu(\nu(n)) \neq \nu(n) \forall n \in \mathbb{N}$
		indirekter Beweis: \\
		Angenommen $\nu(\nu(n)) = \nu(n) \overset{\nu \text{ inj.}}{\Rightarrow} \nu(n) = n \overset{\text{IV)}}{\Rightarrow} \lightning \Rightarrow$ (1) $\Rightarrow$ b) nach Prinzip der vollst. Induktion (vgl.\propref{prin_voll_induktion})
		\end{itemize}
	\end{itemize}
\end{proof}

\begin{proposition}[Rekursive Definition / Rekursion]
	\proplbl{rekursive_def_abb}
	\begriff*{Rekursion}
	Sei b$B$ Menge, $b\in B$ u. $F:B\times\mathbb{N}\rightarrow B$ Abbildung. Dann liefert die Vorschrift \begin{align}\label{rekur_**definition}
		f(0) &:= b,\\f(n+1)&:=F(f(n),n)\quad\forall n\in \mathbb{N}
	\end{align}
	genau eine Abbildung für $f:\mathbb{N}\rightarrow B$ (d.h. solche Abbildung ist eindeutig)
\end{proposition}

\begin{proof}
	mittels vollständiger Induktion:
	\begin{itemize}
		\item[IA] $f(0) = b$ eindeutig definiert 
		\item[IS] angenommen $f(n)$ eindeutig definiert $\overset{\text{1)}}{\Rightarrow}$ 
		$f(n+1) \overset{\text{\propref{prin_voll_induktion}}}{\Rightarrow}$ Behauptung gilt nach Prinzip der vollständigen Induktion
	\end{itemize}
\end{proof}

\begin{proof}[\propref{naturliche_zahlen_isomorph}]
	$\mathbb{N}$ und $\mathbb{N}^{*}$ mögen \person{Peano}-Axiome erfüllen mit $(\nu, 0)$ bzw. $(\nu^{*}, 0^{*})$. Betrachte rekursive eindeutige definierte Abbildung: $I: \mathbb{N} \to \mathbb{N}^{*}$ (\propref{rekursive_def_abb} $B=\mathbb{N}^{*}$, $F(n^{*}, n) = \nu^{*}(n^{*})$)
	$I(0) = 0^{*}, I(\nu(n)) = \nu^{*}(I(n)) \forall n \in \mathbb{N}$ $I$ enthält Nullelement und Nachfolgerabbildung. Falls $I$ bijektiv, dann ist $I$ ein Isomorphismus und Behauptung folgt.\\
		Zeige $I$ surjektiv: offenbar $0^{*} \in I(\mathbb{N})$, falls $n^{*} \in I(\mathbb{N}) \Rightarrow \exists n \in \mathbb{N}:n^{*} = I(n) \Rightarrow \nu^{*}(n^{*}) = \nu^{*}(I(n)) = I(\nu(n)) \in I(\mathbb{N})$ (Bild). Folglich ist $I(\mathbb{N}) \subset \mathbb{N}^{*}$ induktiv in $\mathbb{N}^{*} \overset{\text{P2)}}{\Rightarrow} I(\mathbb{N}) = \mathbb{N}^{*}$.\\
		Zeige $I$ injektiv: $I(n) \neq I(m)\forall n\neq m$ (*) vollständige Induktion nach $m$ (jeweils $\forall n \neq m$)
		\begin{itemize}
			\item[IA)] $m=0: \forall n \neq 0 \exists n \in \mathbb{N} \colon n = \nu(n^{'})$ (vgl. \propref{lemma_nachfolgerabb}) $\Rightarrow I(n) = I(\nu(n^{'})) = \nu^{*}(I(n^{'})) \overset{\text{P1)}}{\neq} 0^{*} = I(0) \forall n \neq 0$ (ist gerade (*)))
			\item[IS)] IV: Sei $I(n) \neq I(m) \forall n \neq m$, dann 
			für $n = 0$, $n = \nu(m) \text{ mit } I(0) = 0^{*} \neq \nu^{*}(I(m)) = I(\nu(m))$.\\
			für $n \neq 0$, $n \overset{\text{\propref{lemma_nachfolgerabb}}}{=} \nu(n^{'}) \neq \nu(m) \overset{\nu \text{ inj.}}{\Rightarrow} n^{'} \overset{\text{IV)}}{\neq} m$ und $I(n) = I(\nu(n{'})) = \nu^{*}(I(n^{'})) \neq \nu^{*}(I(m)) = I(\nu(m))$\\
			$\Rightarrow$ in der Behauptung $I(n) \neq I(\nu(m)) \forall n \neq \nu(m) \Rightarrow$(*) mittels vollständiger Induktion, \\ d.h. $I$ ist injektiv
		\end{itemize}
\end{proof}

\subsection{Rechenoperationen}
\begin{*definition}[Rechenoperation auf $\boldsymbol{\mathbb{N}}$]
	Definiere \begriff{Addition}[!natürliche Zahlen] $+:\mathbb{N}\times\mathbb{N}\rightarrow \mathbb{N}$ auf $\mathbb{N}$ durch $n+0:=n, n+\nu(m) :=\nu(n+m)\,\forall n,m\in\mathbb{N}$
	
	Definiere \begriff{Multiplikation}[!natürliche Zahlen] $\cdot:\mathbb{N}\times\mathbb{N} \rightarrow\mathbb{N}$ auf $\mathbb{N}$ durch $n\cdot 0 = 0, n\cdot\nu(m) = n\cdot m+n\,\forall m,n\in\mathbb{N}$
\end{*definition}

\begin{proposition}
	Addition und Multiplikation haben folgende Eigenschaften, d.h. $\forall k,m,n\in\mathbb{N}$ gilt:
	
	\begin{tabular}{clll}
		\toprule
		&& Addition & Multiplikation\\
		\midrule
		a)& $\exists$ neutrales Element & $n+0=n$ & $n\cdot 1 =  n$\\
		b)& kommutativ & $m+n=n+m$ & $m\cdot n = n\cdot m$ \\
		c)& assoziativ & $(k+m)+n = k+(m+n)$ & $(k\cdot m)\cdot n = k\cdot (m\cdot n)$ \\
		d)&distributiv & \multicolumn{2}{c}{$k(m+n) = k\cdot m + k\cdot n$} \\
		\bottomrule
	\end{tabular}
\end{proposition}

\begin{proof}
	\begin{itemize}
		\item[a)] $n+0 = n$ klar, $n \cdot 1 = n\nu(0) = n \cdot 0 + n = 0 + n \overset{\text{Add. kommutativ}}{=} n$
		\item[b)] ÜA
		\item[c)] assoziativ für Addition (vollst. Induktion nach $n$)
		\begin{itemize}
			\item[IA)] $n=0$: $k+(m+0) = k + m = (k+m)+0 \forall k,m \in \mathbb{N}$
			\item[IV)] Sei $k+(m+n) = (k+m)+n \forall k,m \in \mathbb{N}$
			\item[IS)] 
				IV) $\Rightarrow k + (m + \nu(n)) = k+\nu(m+n) =\nu(k+(m+n)) \overset{\text{IV)}}{=} \nu((k+m)+n)
				=(k+m)+\nu(n) \forall k,m \in \mathbb{N}
				\Rightarrow$ Indunktionbehauptung
			$\overset{\text{voll. Ind.}}{\Rightarrow}$ Addition assoziativ: Beweis für Multiplikation analog
		\end{itemize}
		\item[d)] distributiv (vollst. Ind. nach $k$)
			\begin{itemize}
				\item[IA)] $k=0$ $0\cdot(m+n) = km + kn \forall m,n \in \mathbb{N}$
				\item[IS)] Sei $k(m+n) = km + kn \forall m,n \in \mathbb{N}$
				$\Rightarrow \nu(k)\cdot (m+n) \overset{\text{Def M.}}{=} k \cdot (m+n) + (m+n) \overset{\text{IV)}}{=}k\cdot m + k \cdot m + m+n \overset{\text{Def M.}}{=} \nu(k)\cdot m + \nu(k)\cdot n \forall m,n \in \mathbb{N} \overset{\text{voll. Ind.}}{\Rightarrow}$ Behauptung
			\end{itemize}
	\end{itemize}
\end{proof}

\begin{conclusion}
	\proplbl{kurzungsregel}
	Es gilt $\forall k,m,n\in\mathbb{N}$:
	\begin{enumerate}[label={\alph*)}]
		\item $m\neq 0 \Rightarrow m+n \neq 0$
		\item $m\cdot n = 0 \Leftrightarrow m = 0 \lor n = 0$
		\item $m + k = n + k \Leftrightarrow m = n$ (Kürzungsregel(KR) Addition)
		\item $k\neq 0: m\cdot k = n\cdot k \Leftrightarrow m = n$ (KR Multiplikation)
	\end{enumerate}
\end{conclusion}

\begin{proof}
	\begin{itemize}
		\item[a)] $m \neq 0 \Rightarrow \exists n \in \mathbb{N} \colon m = \nu(m^{'}) \Rightarrow n + m = n + \nu(m^{'}) \overset{\text{Def Add.}}{=} \nu(n+m^{'}) \neq 0 \forall n \in \mathbb{N}$
		\item[b)] ``$\Leftarrow$'': folgt nach Def M. \\ ``$\Rightarrow$'': SeSt
		\item[c)] ``$\Leftarrow$'': Wegen Eindeutigkeit der Addition\\
		``$\Rightarrow$'': vollst. Induktion nach $k$
			\begin{itemize}
				\item[IA)] $n = 0$ klar
				\item[IS)] Behauptung gelte für $k$, sei nun $m+(k+1) = n + (k+1) \Rightarrow \nu(n+k) \overset{\nu \text{ inj.}}{\Rightarrow} m+k = n+k \overset{\text{IV)}}{\Rightarrow} m = n \Rightarrow$
			\end{itemize}
		\item[d)]ÜA/SeSt (kann erst nach \propref{ordnung_nat_zahlen} beweisen werden!)
	\end{itemize}
\end{proof}

\subsection{Ordnung auf \texorpdfstring{$\boldsymbol{\mathbb{N}}$}{N}}
\begin{*definition}[Ordnung auf $\boldsymbol{\mathbb{N}}$]
	Betr. Relation $R:=\{(m,n) \in\mathbb{N}\times\mathbb{N} \mid m \le n\}$
\end{*definition}
\begin{proposition}
	\proplbl{ordnung_nat_zahlen}
	Es gilt auf $\mathbb{N}$:
	\begin{enumerate}[label={\arabic*)}]
		\item $m\le n \;\Rightarrow \;\exists!k\in\mathbb{N}: n = m + k$, nenne $n - m=:k$ \begriff{Differenz}
		\item Relation $R$ (bzw. "`$\le$"') ist Totalordnung auf $\mathbb{N}$
		\item Ordnung "`$\leq$"' ist verträglich mit Addition und Multiplikation
	\end{enumerate}
\end{proposition}

\begin{proof}
	\begin{itemize}
		\item[1)] Sei $n=m+k=m+k^{'} \overset{\text{KR}}{\Rightarrow} k = k^{'}$
		\item[2)] $n=n+0 \Rightarrow n \leq n \Rightarrow$  reflexiv \\
		Sei $k \leq m , m \leq n \Rightarrow \exists l,j \colon m = k+l, n = m+j = (k+l)+j = k + (l+j) \Rightarrow k \leq n \Rightarrow$ transitiv\\
		Sei $m \leq n, n \leq m \overset{\text{transitiv} k = n}{\Rightarrow} n = m + j = n + l + j \overset{\text{KR}}{\Rightarrow} 0 = l+j \overset{\text{\propref{kurzungsregel}}}{\Rightarrow} j =0 \Rightarrow n = m \Rightarrow$ antisymmetrisch\\
		$\Rightarrow R$ ist eine Ordnung auf $\mathbb{N}$\\
		Zeige R Totalordnung, d.h. $\forall m,n \in \mathbb{N}\colon m \leq n \text{ oder } n \leq m$ (\propref{kurzungsregel})\\
		vollst. Induktion nach $m$:
		\begin{itemize}
			\item[IA)] $m=0$: wegen $n = 0 + n$ folgt $0 \leq n \forall n$
			\item[IS)] gelte \propref{kurzungsregel} für festes $m$ und $\forall n \in \mathbb{N}$, dann \\
			falls $n \leq m \overset{m \leq m + 1 \text{ transitiv}}{\Rightarrow} n \leq m + 1$\\
			falls $m \leq n \Rightarrow \exists k \in \mathbb{N} \colon n = m + (k+1) = (m+1) + k \Rightarrow m +1 \leq n \Rightarrow \text{ \propref{kurzungsregel}}$ gilt für $m +1$ und $\forall n \in \mathbb{N} \overset{\text{voll. Ind.}}{\Rightarrow} \text{ \propref{kurzungsregel}}$
		\end{itemize}
		\item[3)] Sei $m \leq n \Rightarrow \exists j\colon n = m + j \overset{\text{KR}}{\Rightarrow} n + k = m + j + k \Rightarrow m + k \leq n + k$ und Rest analog
	\end{itemize}
\end{proof}
\section{Ganze und rationale Zahlen}
\begin{definition}
	Definiere Äquivalenzrelation $Q:=\{ ((n_1,n_1'),(n_2,n_2'))\in((\mathbb{N}\times\mathbb{N})\times(\mathbb{N}\times\mathbb{N})) | n_1+n_2' = n_1' + n_2 \}$
\end{definition}
\begin{proposition}
	$Q$ ist Äquivalenzrelation auf $\mathbb{N}\times\mathbb{N}$.
\end{proposition}

\begin{proposition}
	Sei $[(n,n')]\in\overline{\mathbb{Z}}$. Dann ex. eindeutige $n^{*}\in\mathbb{N}:(n^{*},0)\in[(n,n')]$ falls $n\geq n'$ bzw. $(0,n^{*})\in[(n,n')]$ falls $n\leq n'$.
\end{proposition}

\subsection*{Rechenoperationen}
\begin{definition}
	\begriff{Addition}[!ganze Zahlen]: $\overline{m}+\overline{n} = [(m,n')] + [(n,n')] :=[(m+n,m'+n')]$
	
	\begriff{Multiplikation}[!ganze Zahlen]: $\overline{m}\cdot\overline{n} = \overline{m}\overline{n} = [(m,m')]\cdot[(n,n')]:=[(mn+m'n',mn'+m'n)]$
\end{definition}

\begin{proposition}
	Addition und Multiplikation sind eindeutig definiert, d.h. unabhängig vom Repräsentanten bzgl. $Q$.
\end{proposition}

\begin{proposition}
	Für Addition und Multiplikation auf $Z$ gilt $\forall \overline{m},\overline{n}\in\overline{\mathbb{Z}}$:
	\begin{enumerate}[label={\arabic*)}]
		\item Es ex. neutrales Element $0:=[(0,0)]$ (Add.), $1:=[(1,0)]$ (Mult., $=[(k,k)]$)
		\item Jeweils kommutativ, assoziativ und gemeinsam distributiv
		\item $-\overline{n} := [(n',n)]\in\overline{\mathbb{Z}}$ ist Inverses bzgl. Addition von $[(n,n')]=\overline{n}$
		\item $(-1)\cdot \overline{n} = -\overline{n}$
		\item $\overline{m}\cdot\overline{n} = 0 \Leftrightarrow \overline{m} = 0 \lor \overline{n} = 0$
	\end{enumerate}
\end{proposition}

\begin{proposition}
	Für $\overline{m},\overline{n}\in\overline{\mathbb{Z}}$ hat Gleichung $\overline{m} = \overline{n} + \overline{x}$ eindeutige Lösung $\overline{x} = \overline{m} + (-\overline{n}) = [(m+n'),(m'+n)]$.
\end{proposition}

\subsection*{Ordnung auf $\overline{\mathbb{Z}}$}
\begin{definition}
	Betr. Relation $R:=\{(\overline{m},\overline{n})\in\overline{\mathbb{Z}}\times\overline{\mathbb{Z}} | \overline{m} \le \overline{n}\}$, wobei $\overline{m} = [(m,m')] \le [(n,n')]$ \gls{gdw} $(m+n'\le m'+n)$
\end{definition}

\begin{proposition}
	$R$ ist Totalordnung auf $\overline{\mathbb{Z}}$, die verträglich ist mit Addition und Multiplikation.
\end{proposition}

\begin{definition}
	Betr. $\mathbb{Z} = \mathbb{Z}\cup\{ (-k) | k\in\mathbb{N}_{>0} \}$ mit üblicher Addition, Multiplikation und Ordnung "`$\ge$"'.
\end{definition}
\begin{proposition}
	$\mathbb{Z},\overline{\mathbb{Z}}$ sind isomorph bzgl. Addition, Multiplikation, Ordnung.
\end{proposition}

\subsection*{Rationale Zahlen}
\begin{definition}
	Betr. Relation $Q:=\left\lbrace \left. \left( \frac{n_1}{n_1'},\frac{n_2}{n_2'}\right) \in \left( \mathbb{Z}\times\mathbb{Z}_{\neq 0}\right)\times\left(\mathbb{Z}\times\mathbb{Z}_{\neq 0}\right) \right| n_1n_2' = n_1'n_2\right\rbrace$
	
	Setzte $\mathbb{Q} := \left\lbrace \left[ \left. \frac{n}{n'}\right] \right| (n,n')\in\mathbb{Z}\times\mathbb{Z}_{\neq 0}\right\rbrace$ Menge der \begriff{rationale Zahlen}.
	
	Offenbar gilt \begriff{Kürzungsregel}[!rationale Zahlen] $\left[ \frac{n}{n'}\right] = \left[ \frac{k\cdot n}{k\cdot n'}\right]\quad\forall k\in\mathbb{Z}_{\neq 0}$.
\end{definition}

\subsection*{Rechenoperationen auf $\mathbb{Q}$}
\begin{definition}
	\begriff{Addition}[!rationale Zahlen]: $\left[ \frac{m}{m'}\right] + \left[ \frac{n}{n'}\right] := \left[ \frac{mn' + m'n}{m'+n'}\right]$
	
	\begriff{Multiplikation}[!rationale Zahlen]: $\left[\frac{m}{m'}\right]\cdot\left[\frac{n}{n'}\right]:=\left[\frac{m\cdot n}{m'\cdot n'}\right]$
	
	Addition und Multiplikation sind unabhängig vom Repräsentanten bzgl. $Q$ $\Rightarrow$ Operationen auf $Q$ eindeutig definiert.
\end{definition}

\begin{proposition}
	Mit Addition und Multiplikation ist $\mathbb{Q}$ Körper mit
	\begin{itemize}
	\item neutralem Element $0:=\left[\frac{0_\mathbb{Z}}{1_\mathbb{Z}}\right] = \left[\frac{0_\mathbb{Z}}{n}\right], 1 :=\left[\frac{1_\mathbb{Z}}{1_\mathbb{Z}}\right] = \left[ \frac{n}{n}\right] \neq 0\;n\neq 0$
	\item Inverse Elemente $-\left[\frac{n}{n'}\right] = \left[ \frac{-n}{n'}\right], \left[\frac{n}{n'}\right]^{-1} = \left[\frac{n'}{n}\right]$
	\end{itemize}
\end{proposition}

\subsection*{Ordnung auf $\mathbb{Q}$}
\begin{definition}
	Relation $R:=\left\lbrace \left. \left( \left[\frac{m}{m'}\right],\left[\frac{n}{n'}\right]\right)\in\mathbb{Q}\times\mathbb{Q} \right| mn'\le m'n'; m',n'>0\right\rbrace$ gibt Ordnung "`$\le$"'.
\end{definition}

\begin{proposition}
	$\mathbb{Q}$ ist angeordneter Körper ("`$\leq$"') ist Totalordnung verträglich mit Addition und Multiplikation).
\end{proposition}
\begin{conclusion}
	Körper $\mathbb{Q}$ ist \begriff[Körper!]{archimedisch angeordnet}, d.h. $\forall q\in\mathbb{Q} \, \exists n\in\mathbb{N}: q < n$.
\end{conclusion}
\section{Reelle Zahlen}
\stepcounter{theorem}%Example 1 is missing (not important)
\subsection*{Struktur von archimedisch angeordneten Körpern}
\begin{proposition}
	Sei $K$ Körper. Dann gilt $\forall a,b\in K$:
	\begin{enumerate}[label={\arabic*)}]
		\item $0,1,(-a),b^{-1} (b\neq 0)$ sind eindeutig bestimmt
		\item $(-0) = 0, 1^{-1} = 1$
		\item $-(-a) = a, (b^{-1})^{-1} = b (b\neq 0)$
		\item $-(a+b) = (-a) + (-b), (ab)^{-1} = a^{-1}b^{-1} (a,b\neq 0)$
		\item $-a = (-1) a, (-a)(-b) = ab,\;a\cdot 0 = 0$
		\item $ab = 0 \Leftrightarrow a=0\lor b = 0$
		\item $a+x = b$ hat eindeutige Lösung $x = b+(-a) =: b-a$ \begriff{Differenz}
		
		$ax=b (a\neq 0)$ hat eindeutige Lösung $x=a^{-1}b =:\frac{b}{a}$ \begriff{Quotient}
	\end{enumerate}
\end{proposition}

\begin{definition}
	\begin{itemize}
	\item \begriff{Vielfache}: $na := \sum_{k=1}^{n}a$
	
	Damit:
	\begin{itemize}
		\item $(-n)a := n(-a), 0_\mathbb{N} a := a_K$ für $n\in\mathbb{N}_{\ge 1}$
		\item $ma + na = (m+n)a, na + nb = n(a+b)$
		\item $(ma)\cdot(na) = (mn)a^2, (-n)a = -(na)$
	\end{itemize}
	\item \begriff{Potenz}: $a^n$ von $a\in K, n\in\mathbb{Z}:=\prod_{k=1}^{n} a$
	
	Damit
	\begin{itemize}
		\item $a^{-n} :=(a^{-1})^n, a^{0_K}:=1_K$ für $n\in\mathbb{N}_{\ge 1}, a\neq 0$
		\item $a^m a^n = a^{m+n}, (a^m)^n = a^{mn}, a^nb^n = (ab)^n, a^{-n} = (a^n)^{-1}$
	\end{itemize}
	\item \begriff{Fakkultät} für $n\in\mathbb{N}:$\mathsymbol*{n}{$n"!$} $n!:=\prod_{k=1}^n k, 0!=1$
	\item \begriff{Binomialkoeffizient} \mathsymbol{noverm}{$\binom{n}{k}$}$:=\frac{n!}{k!(n-k)!}\in\mathbb{N}$ $\forall k,n\in\mathbb{N}, 0\le k\le n$
	\begin{itemize}
		\item $\binom{k+1}{n+1} = \binom{n}{k} + \binom{n}{k+1}$
		\item Rechenregel führt auf \begriff{\person{Pascal}'sches Dreieck}
	\end{itemize}
	\end{itemize}
\end{definition}

\begin{proposition}[Binomischer Satz]
	In Körper $K$ gilt: $(a+b)^n = \sum_{k=0}^n\binom{n}{k}a^n b^{n-k}, ,b\in K, n\in\mathbb{N}$
\end{proposition}
\begin{proposition}
	Sei $K$ angeordneter Körper. Dann gilt $\forall a,b,c,d\in K$:
	\begin{enumerate}[label={\alph*)}]
		\item $a < b \Leftrightarrow 0 < b-a$
		\item $a < b, c < d \Leftrightarrow a+c < b+d$
		
		$0 \le a < b, 0 \le c < d \Leftrightarrow a\cdot c < b\cdot d$
		\item $a < b \Leftrightarrow -b < -a$ (insbes. $a > 0 \Leftrightarrow -a < 0$)
		
		$a < b, c < 0 \Leftrightarrow a\cdot c > b \cdot c$
		\item $a\neq 0 \Leftrightarrow a^2 > 0$ (insbes. 1 > 0)
		\item $a > 0 \Leftrightarrow a^{-1} > 0$
		\item $0 < a < b \Leftrightarrow b^{-1} < a^{-1}$
	\end{enumerate}
\end{proposition}

\begin{definition}
	\begriff{Absolutbetrag} $\vert\cdot\vert:K\rightarrow K$ (auf angeordneten Körper $K$) \[\vert a \vert:=\begin{cases}
	a&\text{für }a \ge 0 \\ -a& \text{für }a < 0\end{cases}\]
\end{definition}

\begin{proposition}
	Sei $K$ angeordneter Körper. Dann gilt $\forall a,b\in K$:
	\begin{enumerate}[label={\arabic*)}]
		\item $\vert a\vert\ge 0, \vert a\vert\ge a$
		\item $\vert a\vert = 0$ \gls{gdw} $a=0$
		\item $\vert a\vert = \vert -a\vert$
		\item $\vert a\vert\cdot\vert b\vert = \vert a\cdot b\vert$
		\item $\left\vert \frac{a}{b}\right\vert = \frac{\vert a\vert}{\vert b\vert} (b\neq 0)$
		\item \begriff{Dreiecksungleichung}
		
		$\vert a+b\vert \le \vert a\vert + \vert b\vert$ ($\vert a-b\vert = \vert a+(-b)\vert \le \vert a\vert + \vert b\vert$)
		\item $\left\vert a\vert - \vert b\right\vert \le \vert a+b\vert$
		\item \begriff{\person{Bernoulli}-Ungleichung}
		
		$(1+a)^n \ge 1 + n\cdot a \,\forall a\ge -1, n\in\mathbb{N} (a\neq -1 \text{ bei }n = 0)$
		
		(Gleichheit \gls{gdw} $n=0,1$ oder $a=0$)
	\end{enumerate}
\end{proposition}
\begin{definition}
	Betr. $f:\mathbb{Q}\rightarrow K$ mit $f\left(\frac{m}{n}\right):= \frac{m\cdot 1_K}{n\cdot 1_K}=(m 1_k)(n 1_K)^{-1}\,\forall m\in\mathbb{Z},k\in\mathbb{Z}_{\neq 0}$
\end{definition}
\begin{proposition}
	Sei $K$ angeordneter Körper\\
	$\Rightarrow$ $f:\mathbb{Q}\rightarrow K$ ist injektiv und $f$ erhält die Körperstruktur und Ordnung, d.h. $\forall p,q\in\mathbb{Q}$:
	\begin{itemize}
		\item $f(p+q) = f(p) + f(q), f(0) = 0_K, f(-p) = -f(p)$
		\item $f(p\cdot q) = f(p)\cdot f(q), f(1) = 1_K, f(p^{-1}) = f(p)^{-1} (p\neq 0)$
		\item $p \le_\mathbb{Q} q \Leftrightarrow f(p) \le_K f(q)$
	\end{itemize}
\end{proposition}

\begin{conclusion}
	Es gilt im angeordneten Körper:
	\begin{enumerate}[label={\arabic*)}]
		\item $\mathbb{Q}_K = f(\mathbb{Q})$ ist mit Addition, Multiplikation und Ordnung von $K$ selbst angeordneter Körper
		\item $\mathbb{Q}_K$ ist isomorph zu $\mathbb{Q}$ bzgl. Körperstruktur und Ordnung.
	\end{enumerate}
\end{conclusion}

\begin{definition}
	Angeordneter Körper heißt \begriff[Körper!]{archimedisch}, falls $\forall a\in K\,\exists n\in\mathbb{N}\subset K: a < n$.
\end{definition}
\begin{proposition}
	Sei $K$ archimedisch angeordneter Körper. Dann\begin{enumerate}[label={\arabic*)}]
		\item $\forall a,b\in K$ mit $a,b>0\,\exists n\in\mathbb{N}: n\cdot a > b$
		\item $\forall a\in K\,\exists!\,[a]\in\mathbb{Z}: [a]\le a \le [a] +1$, \mathsymbol{a}{$[a]$} heißt \begriff{ganzer Anteil} von $a$
		\item $\forall \epsilon \in K$ mit $\epsilon > 0\,\exists n\in\mathbb{N}_{\neq 0}: \frac{1}{n}< \epsilon$ (beachte: $0 < \frac{1}{n}$)
		\item $\forall a,b\in K$ mit $a>1\,\exists n\in\mathbb{N}: a^n > b$
		\item $\forall a,\epsilon > 0\,\exists p,q\in\mathbb{Q}: p \le a  q$ und $q - p < \epsilon$
		
		(d.h. $a\in K$ kann auch rationale Zahlen beliebig genau approximiert werden, $\mathbb{Q}$ "`dicht"' in $K$)
		\item $\forall a,b\in K, a < b\,\exists q\in\mathbb{Q}:a < q < b$.
	\end{enumerate}
\end{proposition}

\begin{definition}[Intervall]
	\begriff{Intervall} für angeordneten Körper $K$: Sei $a,b\in K$:
	\begin{itemize}
		\item \begriff{beschränktes Intervall}
		\begin{itemize}
			\item $[a,b]:=\{ x\in K | a \le x \le b \}$ \begriff[Intervall!]{abgeschlossen}
			\item $(a,b):=\{a < x < b\}$ \begriff[Intervall!]{offen}
			\item $[a,b) := \{a \le x < b\}, (a,b]:=\{a < x \le b\}$ \begriff[Intervall!]{halboffen}
		\end{itemize}
		\item \begriff{unbeschränktes Intervall}
		\begin{itemize}
			\item $[a,\infty]:=\{x\in K\mid a \le x\}$
			\item $(a,\infty):=\{x\in K\mid a > x\}$
            \item $(-\infty, b]:= \{x \in K \mid x< a\}$
            \item $(-\infty, b) := \{x\in K\mid x \leq b\}$
		\end{itemize}
	\end{itemize}
\end{definition}

\begin{definition}[Folge]
    Eine \begriff{Folge} in Menge $M$ ist eine Abbildung $\alpha:\mathbb{N}\rightarrow M$ (evtl. $\alpha:\mathbb{N}_{\ge n}\rightarrow M$), $\alpha_n := \alpha(n)$ heißen \begriff{Folgenglieder}, und \begriff{Folgenindex}.
    
	Notation: $\{a_n\}_{n\in\mathbb{N}}, \{\alpha_n\}_{k=1}^\infty$ bzw. $\alpha_0, \alpha_1, \dotsc$\\
	kurz: $\{\alpha_n\}_n, \{\alpha_n \}$
		
	Hinweis: $\{x\}_n$ ist \begriff{konstante Folge}, d.h. $\alpha_n = \alpha\,\forall n$
\end{definition}

Aussage gilt für \gls{fa} $n\in\mathbb{N}$, wenn höchstens für endlich viele $n$ falsch.

\begin{definition}[Intervallschachtelung]
	Folge $\{x_n\}_{n\in\mathbb{N}} =:\mathcal{X}$ von abgeschlossenen Intervallen $X_n=[x_n, x_n']\subset K$ $(x_n, x_n'\in K)$ heißt \begriff{Intervallschachtelung} (im angeordneten Körper K), falls
	\begin{enumerate}[label={\alph*)}]
		\item $X_n\neq \emptyset$ und $X_{n+1}\subset X_n\,\forall n\in\mathbb{N}$
		\item $\forall\epsilon > 0$ in $K$ existiert $n\in\mathbb{N}: l(X_n):= x_n' - x_n < \epsilon$, mit $l$ \begriff{Intervalllänge}
	\end{enumerate}
\end{definition}

\begin{lemma}
	Sei $\mathcal{X} = \{X_n\}_{n\in\mathbb{N}}$ Intervallschachtelung im angeordneten Körper $K$\\
	$\Rightarrow \bigcap_{n\in\mathbb{N}} X_n$ enthält höchstens ein Element.
\end{lemma}

\begin{definition}
	Archimedisch angeordneter Körper heißt \begriff[Körper]{vollständig}, falls $\bigcap_{n\in\mathbb{N}} X_n\neq \emptyset$ für jede Intervallschachtelung $\mathcal{X} = \{x_n\}$ in $K$.
\end{definition}

\begin{definition}
	$Q:=\{ (\{x_n\}, \{y_n\})\in I_\mathbb{Q}\times I_\mathbb{Q} \}$ ist Relation auf $I_\mathbb{Q}$, $I_\mathbb{Q}:=$ Menge aller Intervallschachtelungen $\mathcal{X}=\{x_n\} \in \mathbb{Q}$.
\end{definition}

\begin{proposition}
	$Q$ ist Äquivalenzrelation auf $I_\mathbb{Q}$.
\end{proposition}

\begin{definition}
	setze $\mathbb{R} := \{ [\mathcal{X}] \mid \mathcal{X}\in I_\mathbb{Q} \}$ Menge der \begriff{reellen Zahlen}.
	
	\begin{itemize}
		\item $\bigcap_{n\in\mathbb{N}} X_n \neq 0 \rightarrow [\mathcal{X}]$ ist "`neue"' sog. \begriff{irrationale Zahl}
	\end{itemize}
\end{definition}

\subsection*{Rechenoperationen}
\begin{definition}
	Für Intervalle $X=[x,x'], Y=[y,y']$ in $\mathbb{Q}$ defineren wir Intervall in $\mathbb{Q}$:
	\begin{itemize}
		\item $X + Y := \{\xi + \eta \mid \xi \in X, \eta\in Y\} = [x + y, x' + y']$
		\item $X\cdot Y :=\{\xi \cdot \eta \mid \xi \in X, \eta\in Y\} = [\tilde{x}\tilde{y}, \tilde{x}'\tilde{y}']$, wobei $\tilde{x},\tilde{x}'\in\{x,x'\},\tilde{y},\tilde{y}'\in\{y,y'\}$
		\item $-X := [-x,-x']$, $X^{-1}:=[\frac{1}{x'}, \frac{1}{x}]$ falls $0\in X$
	\end{itemize}

	Für relle Zahl $[\mathcal{X}] = [\{x_n\}], [\mathcal{Y}]=[\{y_n\}]$ sei
	\begin{itemize}
		\item $[\mathcal{X}]+\mathcal{Y} :=[\{x_n + y_n\}]$
		\item $[\mathcal{X}]\cdot[\mathcal{Y}] :=[\{x_n\cdot y_n\}]$
		\item $-[\mathcal{X}]:=[\{-x_n\}]$
			
			$[\mathcal{X}]^{-1} := [\{x_n^{-1}\}]$ falls $[\mathcal{X}]\neq 0_\mathbb{R}$
	\end{itemize}
\end{definition}

\begin{proposition}
	\begin{enumerate}[label={\arabic*)}]
		\item Addition, Multiplikation und Inverse sind in $\mathbb{R}$ eindeutig definiert
		\item $\mathbb{R}$ ist damit und neutralen Elementen ein Körper.
	\end{enumerate}
\end{proposition}

\subsection*{Ordnung auf $\mathbb{R}$}
\begin{definition}
	Betr. Relation "`$\le$"': $R:=\{ ([\{x_n\}],[\{y_n\}])\in\mathbb{R}\times\mathbb{R} | x_n \le y_n\,\forall n\in\mathbb{N}\}$
\end{definition}
\begin{proposition}
	$\mathbb{R}$ ist mit "`$\le$"' angeordneter Körper.
\end{proposition}
\begin{proposition}
	$\mathbb{R}$ ist archimedisch angeordneter Körper.
\end{proposition}
\begin{theorem}
	$\mathbb{R}$ ist vollständiger, archimedisch angeordneter Körper.
\end{theorem}
\begin{theorem}
	Sei $K$ vollständiger, archimedisch angeordneter Körper\\
	$\Rightarrow K$ ist isomorph zu $\mathbb{R}$ bzgl. Körperstruktur und Ordnung.
\end{theorem}

\begin{definition}
	Sei $M\subset K$, $K$ angeordneter Körper.
	\begin{itemize}
		\item $s\in K$ ist \begriff[Schranke!]{obere} / \begriff[Schranke!]{untere} \begriff{Schranke} von $M$, falls $x \le s (x \ge s)\,\forall x\in M$
		
		$M$ ist nach \begriff[beschränkt!]{oben} / \begriff[beschränkt!]{unten} \highlight{beschränkt}, falls obere ( untere ) Schranke existiert.
		\item $M$ \begriff{beschränkt}[!Menge im Körper], falls $M$ nach oben und unten beschränkt.
		\item kleinste obere (größte untere) Schranke $\tilde{s}$ von $M$ ist \begriff{Supremum} (\begriff{Infimum}) von $M$, d.h. \\
		\mathsymbol{sup}{$\sup$}$ M:= \tilde{s} \le s ($\mathsymbol{inf}{$\inf$}$ M = s \ge \tilde{s}) \;$ obere (untere) Schranken $s\in M$.
		\item Falls $\sup M \in M (\inf M\in M)$ nennt man dies auch \begriff{Maximum} (\begriff{Minimum}) von $M$.
		
		kurz: \mathsymbol{max}{$\max$}$M = \sup M ($\mathsymbol{min}{$\min$}$M = \inf M)$
		\item falls $M$ nach oben (unten) \begriff[Menge!]{unbeschränkt}, d.h. nicht beschränkt, schreibt man auch $\sup M = \infty (\inf M = -\infty)$
	\end{itemize}

	Man hat
	\begin{align*}
	\sup M &= \min\{s \mid s \text{ obere Schranke von } M\}\\
	\inf M &= \max\{s \mid s \text{ untere Schranke von } M\}
	\end{align*}
\end{definition}
\stepcounter{theorem}
\begin{proposition}
	Sei $K$ angeordneter Körper, $M\subset K$. Falls $\sup M\;(\inf M)$ existiert, dann
	\begin{enumerate}[label={\arabic*)}]
		\item $\sup M\;(\inf M)$ eindeutig
		\item $\forall \epsilon > 0\,\exists y\in M: \sup M < y + \epsilon\;(\inf M > y - \epsilon)$
	\end{enumerate}
\end{proposition}

\begin{theorem}
	Sei $K$ archimedisch angeordneter Körper. Dann
	\[ K \text{ vollständig } \Leftrightarrow \sup M \slash \inf M \text{ ex. }\forall M\in K, M\neq \emptyset \text{ nach oben \slash unten beschränkt} \]
\end{theorem}

\subsection*{Anwendung: Wurzeln, Potenzen, Logarithmen in $\mathbb{R}$}
\begin{proposition}[Wurzeln]
	Sei $a\in\mathbb{R}_{>0}, k\in\mathbb{N}_{>0} \;\Rightarrow \; \exists ! x\in \mathbb{R}_{>0}: x^k = a, \sqrt[k]{a}:=a^{\frac{1}{k}} = x$ heißt \highlight{k-te} \begriff{Wurzel} von $a$.
\end{proposition}
\begin{definition}[Potenz]
	$n$-te \begriff{Potenz} von $a\in\mathbb{R}_{>0}, r\in\mathbb{R}$:
	
	Zunächst $r=\frac{m}{n}\in\mathbb{Q}$ (\gls{obda}) $n\in\mathbb{N}_{>0}$): $ a^{\frac{m}{n}}:= (a^m)^{\frac{1}{n}}$
	Allgemein für $a\ge 0, a > : a^r := \sup \{ a^q \mid 0 \le q \le r,q\in\mathbb{Q} \}$
	offenbar eindeutig definiert und allgemeine Definition konsistent mit Definition für $\frac{m}{n}\in\mathbb{Q}$.
	Damit: \begriff{Exponentialfunktion}
\end{definition}
\begin{proposition}\label{proposition_potenz_r}
	Seien $a,b\in\mathbb{R}_{>0}, r,s\in\mathbb{R}. Dann$
	\begin{enumerate}[label={\arabic*)}]
		\item $a^r b^r = (ab)^r, (a^r)^s = a^{rs}, a^ra^s = a^{r+s}$
		\item f. $r > 0: a < b \Leftrightarrow a^r < b^r$
		\item für $a > 1: r < s \Leftrightarrow a^r < a^s$
	\end{enumerate}
\end{proposition}

\begin{definition}[Logarithmus]
	Sei $a,b\in\mathbb{R}_{<0}, a\neq 1$: \begriff{Logarithmus}\highlight{von $b$ zur Basis $a$} ist \begin{align*}
	 \log_a b :=\begin{cases}
	 \sup \{ r \in \mathbb{R} \mid a^r \le b\}& a > 1\\
 	\sup \{r\in\mathbb{R}\mid a^r \ge b\}& 0 < a < 1
	 \end{cases}
	\end{align*}
\end{definition}
\begin{proposition}\label{proposition_logarithmus_r}
	Se $a,b,c\in\mathbb{R}_{>0}, a\neq 1$. Dann
	\begin{enumerate}[label={\arabic*)}]
		\item $log_a b$ ist eindeutige Lösung von $a^x = b$, d.h. $a^{log_a b} = b$
		\item $\log_a a = 1, log_a 1 = 0$
		\item $\log_a b^\gamma = \gamma \log_a b \,\forall \gamma\in\mathbb{R}$
		\item $\log_a(bc) = \log_a b + \log_a c, \log_a \frac{b}{c} = \log_a b - \log_a c$
		\item $\log_a b = \frac{\log_\alpha b}{\log_\alpha a}\,\forall \alpha\in\mathbb{R}_{>0},\alpha\neq 1$
	\end{enumerate}
\end{proposition}

\subsection*{Mächtigkeit von Mengen}
\begin{definition}
	$M$ \begriff[Mächtigkeit!]{endlich}, falls $M$ endlich viele Elemente hat, sonst \begriff[Mächtigkeit!]{unendlich}.
	
	Unendliches $M$ ist \begriff[Mächtigkeit!]{abzählbar}, falls bijektive Abbildung $f:\mathbb{N}\rightarrow M$ existiert, sonst ist $M$ \begriff[Mächtigkeit!]{überabzählbar}.
\end{definition}
\begin{proposition}
	Es gilt:
	\begin{enumerate}[label={\arabic*)}]
		\item $\mathbb{Z},\mathbb{Q}$ abzählbar
		\item $M$ abzählbar, $n\in\mathbb{N}_{>0} \Rightarrow M^n$ abzählbar ($\Rightarrow \mathbb{Z}^n, \mathbb{Q}^n$ abzählbar)
		\item Ein offenes Intervall $I\in\mathbb{R}\neq \emptyset $ ist überabzählbar
		\item $\mathcal{P}(\mathbb{N})$ ist überabzählbar.
	\end{enumerate}
\end{proposition}
\section{Komplexe Zahlen (kurzer Überblick)}

\textbf{Frage:} Hat $x^2 = -1$  eine Lösung in $\real$? \\
\textbf{Antwort:} keine Lösung $\Rightarrow$ Körpererweiterung $\real \to \comp$

\begin{definition}[komplexe Zahlen]
	betrachte Menge der komplexen Zahlen: $\comp := \real \times \real = \real^2$ mit Addition und Multiplikation:
	\begin{itemize}
		\item $(x,x^{'}) + (y,y^{'}) = (x+y, x^{'} + y^{'})$
		\item $(x,x^{'}) \cdot (y,y^{'}) = (xy - x^{'}y^{'}, xy^{'}+x^{'}y)$
	\end{itemize}
\end{definition}

$\comp$ ist ein Körper mit (vgl. lin Algebra):\\
$0_{\comp} = (0,0)$,  $1_{\comp} = (1,0)$, $-(x,y) = (-x,-y)$ and $(x,y)^{-1} = \bigg(\frac{x}{x^2+y^2},\frac{-y}{x^2+y^2}\bigg)$\\
mit imaginärer Einheit $\iota=(0,1)$\\
$z=x+\iota y$ statt $z=(x,y)$ mit $x:=\Realz(z)$ Realteil von $z$, $y:= \Imag(z)$ Imaginärteil von $z$\\
komplexe Zahl $z=x + \iota y$ wird mit reeller Zahl $x \in \real$ identifiziert\\
offenbar $\iota^2=(-1,0)=-1$, d.h. $z=\iota \in \comp$ und löst die Gleichung $z^2=-1$ (nicht eindeutig, auch $(-\iota)^2 = -1$)\\
Betrag $|\cdot|: \comp \to \real_{> 0}$ mit $|z|:= \sqrt{x^2+y^2}$ (ist Betrag/Länge des Vektors $(x,y)$)\\

\begin{proposition}
	Es gilt:
	\begin{enumerate}[label={\alph*)}]
		\item $\Realz(z) = \frac{z+\overline{z}}{2}, \Imag(z) = \frac{z+\overline{z}}{2\iota}$
		\item $\overline{z_1 + z_2} = \overline{z_1} + \overline{z_2}$, $\overline{z_1 \cdot z_2} = \overline{z_1} \cdot \overline{z_2}$
		\item $|z| = 0 \iff z=0$
		\item $|\overline{z}| = |z|$
		\item $|z_1 \cdot z_2| = |z_1| \cdot |z_2|$
		\item $|z_1 + z_2| \leq |z_1| + |z_2|$ (Dreiecks-Ungleichung: Mikoswski-Ungleichung)
	\end{enumerate}
\end{proposition}
\begin{proof}
	SeSt \QEDA
\end{proof}

\chapter{Metrische Räume und Konvergenz}
\section{Grundlegende Ungleichungen}
\begin{proposition}[geoemtrisches / arithemtisches Mittel]
	\proplbl{ungleichung_arithmetisches_geometrisches_mittel}
	Seien $x_1, \dotsc, x_n\in\mathbb{R}_{>0}$.\\
	\[\Rightarrow \underbrace{\sqrt[n]{x_1\cdot x_2\cdot \dotsc \cdot x_n}}_{\text{\begriff{geometrisches Mittel}}} \le \underbrace{\frac{x_1 + \dotsc + x_n}{n}}_{\text{\begriff{arithmetisches Mittel}}}\]
\end{proposition}
\begin{proof}
	Zeige zunächst mit vollständiger Induktion\\
	\begin{align} %% add /nonumber to have no numbering
	\prod_{i=1}^{n}x_i=1 \Rightarrow \sum\limits_{i=1}^{n} x_i \geq n \text{, mit } x_1=\dots=x_n \label{7_1_ind}
	\end{align}
	\begin{itemize}
		\item (IA) $n = 1$ klar
		\item (IS) (\ref{7_1_ind}) gelte für $n$, zeige (\ref{7_1_ind}) für $n+1$ d.h. $\prod_{i=1}^{n+1} = 1$, falls alle $x_i=1 \beha$. Sonst oBdA $x_n < 1$, $x_{n+1} > 1:$\\ mit $y_n:=x_n x_{n+1}$ gilt $x_1\cdot\dots\cdot x_{n-1}\cdot y_n=1$
		\begin{align*}
		\Rightarrow x_1 + \dots + x_{n+1} &= \underbrace{x_1+\dots+x_{n-1}}_{\geq \text{ (IV)}} + y_n - y_n + x_n+x_{n+1}\\ 
		&\geq n + \underbrace{(x_{n+1} -1)}_{>n}\underbrace{(1-x_n)}_{>n}\\ 
		&\Rightarrow (\ref{7_1_ind}) \quad\forall n \in \natur\\ 
		\shortintertext{allgemein sei nun $g:=\big( \prod_{i=1}^{n} x_i \big)^{\frac{1}{n}} \Rightarrow \prod_{i=1}^{n} \frac{x_i}{g} = 1$}
		&\Rightarrow \sum\limits_{i=1}^{n} \frac{x_i}{g} \geq n \beha\\ 
		\shortintertext{Aussage über Gleichheit nach nochmaliger Durchsicht.}
		\end{align*}
	\end{itemize} 
\end{proof}

\begin{proposition}[allgemeine \person{Bernoulli}-Ungleichung]
	\proplbl{bernoulli_ungleichung}
	Seien $\alpha,x\in\mathbb{R}$. Dann
	\begin{enumerate}[label={\arabic*)}]
		\item $(1+x)^\alpha \ge 1 + \alpha x\,\forall x\ge -1, \alpha > 1$
		\item $(1+x)^\alpha \le 1+\alpha x \,\forall x\ge -1, 0 < \alpha < 1$
	\end{enumerate}
\end{proposition}
\begin{proof} % fix alignment
	\begin{enumerate}
		\item[2)] Sei $\alpha =\frac{m}{n} \in \ratio_{<1}\text{, d.h. } m\leq n$
		\begin{align*}
		&\Rightarrow (1+x)^\frac{m}{n} \overset{Definition}{=} \sqrt[n]{(1+x)^m\cdot1^{n-m}} \\
		&\leq \frac{m(1+x)+(n-m)\cdot1}{n}&\\ 
		&=\frac{n + mx}{n} = 1 + \frac{m}{n}x \text{, für } \alpha \in \ratio \beha&
		\shortintertext{Sei $\alpha \in \real$ angenommen $(1+x)^{\alpha} > 1 + \alpha x$ ($x\neq 0$ sonst klar!)}
		& \overset{\text{\propref{k_archimedisch_angeordneter_körper}}}{\Rightarrow} \exists \in \ratio_{<1} 
		\begin{cases*}
		x > 0&$\alpha<q< \frac{(1+x)^{\alpha}-1}{x}$\\
		x < 0&$\alpha < q$
		\end{cases*} \\
		&\Rightarrow 1+qx < (1+x)^{\alpha} \leq (1+x)^q \overset{\text{\propref{satz_potenz_r}}}{\Rightarrow} \lightning \beha 
		\end{align*}
		\item[1)] Sei $1+\alpha x \geq 0$, sonst klar
		\begin{align*}
		&\Rightarrow \alpha x \geq -1 \overset{2)}{\Rightarrow} (1+\alpha x)^{\frac{1}{\alpha}}\\
		&\geq 1 +\frac{1}{\alpha}\alpha x = 1 +x &\\
		&\Rightarrow \text{ Behauptung und Gleichheit ist Selbststudium.}
		\end{align*}
	\end{enumerate}   
\end{proof}

\begin{proposition}[\person{Young}'sche Ungleichung]
	\proplbl{youngsche_ungleichung}
	Seien $p,q\in\mathbb{R}, p,q > 1$ mit $\frac{1}{p}+\frac{1}{q}=1$.\\
	$\Rightarrow a\cdot b \le \frac{a^p}{p} + \frac{b^q}{q}\,\forall a,b\ge 0$
	
	\uline{Spezialfall:} $p=q=2: ab \le \frac{a^2+b^2}{2} \,\forall a,b\in \mathbb{R}$
\end{proposition}
\begin{proof} %fix formating
	\begin{align*}
	\shortintertext{Sei $a,b > 0$ (sonst klar!)} \\
	&\Rightarrow \big(\frac{b^q}{a^p}\big)^{\frac{p}{q}} = \big(1+\big(\frac{b^q}{a^p} -1\big)\big)^{\frac{p}{q}}\\ 
	&\overset{\text{Bernoulli}}{\leq} 1+ \frac{1}{q}\big(\frac{b^q}{a^p} -1\big)\\ 
	&=\frac{1}{p}+\frac{1}{q}+\frac{1}{q}\frac{b^q}{a^p}-\frac{1}{q}\\
	&\overset{\cdot a^p}{\Rightarrow} a^p\frac{b}a^{\frac{p}{q}} = a^{p(1-\frac{1}{q})}b = ab \leq \frac{a^p}{p} + \frac{b^q}{q} 
	\end{align*}
\end{proof}

\begin{proposition}[\person{Hölder}'sche Ungleichung]
	\proplbl{hoeldersche_ungleichung}
	Sei $p,q\in\mathbb{R}, p,q > 1$ mit $\frac{1}{p} + \frac{1}{q} = 1$\\
	$\Rightarrow \sum\limits_{i=1}^{n} |x_i y_i| \le \left(\sum\limits_{i=1}^n |x_i|^p \right)^{\frac{1}{p}}\left(\sum\limits_{i=1}^n |y_i|^q\right)^{\frac{1}{q}}\,\forall x,y\in\mathbb{R}$
\end{proposition}
\begin{proof}
	Faktoren rechts seien $\mathcal{X} \text{ und } \mathcal{Y}$ d.h.
	\begin{align*}
	\mathcal{X}^p &= \sum\limits_{i=1}^{n} \vert x_i \vert^{\frac{1}{p}}, \mathcal{Y}^p = \sum\limits\limits_{i=1}^{n} \vert y_i \vert^{\frac{1}{q}}\text{, falls } \mathcal{X}=0\\ 
	&\Rightarrow x_i = 0\;\forall i \beha \text{, analog für } \mathcal{Y} =0\\
	\shortintertext{Seien $\mathcal{X}, \mathcal{Y} > 0$} \\
    &\overset{\text{Young}}{\Rightarrow} 
	\frac{\vert x_i y_i \vert}{\mathcal{XY}} \leq \frac{1}{p}\frac{\vert x_i \vert^p}{\mathcal{X}^p}+ \frac{1}{q}\frac{\vert y_i \vert^q}{\mathcal{Y}^p} \forall i\\
	&\Rightarrow \frac{1}{\mathcal{XY}}\sum\limits\limits_{i=1}^{n}\vert x_i y_i \vert \leq \frac{1}{p}\frac{\mathcal{X}^p}{\mathcal{X}^p}+\frac{1}{q}\frac{\mathcal{Y}^p}{\mathcal{Y}^p} = 1 \beha
	\end{align*}
\end{proof}

\begin{remark}
	\begin{itemize}
		\item Ungleichung gilt auch für $x_i,y_i \in \comp$ (nur Beträge gehen ein)
		\item für $p=q=2$ heißt Ungleichung \person{Cauchy-Schwarz}-Ungleichung (Gleichheit gdw. $\exists x \in \real x_i = \alpha y_i \text{ oder } y_i = \alpha x_i\;\forall i$)
	\end{itemize}
\end{remark}

\begin{proposition}[\person{Minkowski}-Ungleichung]
	\proplbl{minkowski_ungleichung}
	Sei $p\in\mathbb{R}, p>1$\\
    $\Rightarrow \big(\sum\limits_{i=1}^{n} \vert x_i + y_i \vert^p \big)^\frac{1}{p} \leq \big(\sum\limits_{i=1}^{n} \vert x_i \vert^p \big)^\frac{1}{p} + \big(\sum\limits_{i=1}^{n} \vert y_i \vert^p \big)^\frac{1}{p}\,\forall x,y\in \mathbb{R}$
\end{proposition}
\begin{proof}
	$p=1$ Beh. folgt aus $\Delta$-Ungleichung $\vert x_i + y_i\vert \overset{\text{\propref{k_angeordneter_körper}}}{\leq} \vert x_i \vert + \vert y_i \vert \forall i$\\ $p>1$ sei $\frac{1}{p} + \frac{1}{q} = 1\Rightarrow q = \frac{p}{p-1}$, $z_i:=\vert x_i + y_i\vert^{p-1}\forall i$
	\begin{align*}
	\mathcal{S}^{p\cdot q} &= \sum_{i=1}^{n} \vert z_i \vert^q\\
	& = \sum_{i=1}^{n} \vert x_i+y_i \vert\cdot\vert z_i \vert^q\\
	& \overset{\Delta\text{-Ungleichung}}{=} \sum_{i=1}^{n} \vert x_i\cdot z_i \vert + \sum_{i=1}^{n} \vert y_i\cdot z_i \vert\\
	& \overset{\text{Hölder}}{\leq} \big(\mathcal{X+Y}\big)\big(\sum_{i=1}^{n} \vert z_i\vert^q \big)^\frac{1}{q}\\
	& = \big(\mathcal{X+Y}\big)\mathcal{S}^\frac{p}{q}\\
	&\Rightarrow S\leq \mathcal{X}+\mathcal{Y}\beha
	\end{align*}
\end{proof}

\begin{remark}
	\begin{itemize}
    \item Ungleichung gilt auch für $x_i, y_i \in \mathbb{C}$
    \item ist $\Delta$-Ungleichung für $p$-Normen
    \end{itemize}
\end{remark}
\section{Metrische Räume}
\begin{*definition}[Metrik]
	Sei $X$ Menge, Abbildung $d:X\times X\rightarrow \mathbb{R}$ heißt \begriff{Metrik} auf $X$, falls $\forall x,y,z\in X$:
	\begin{enumerate}[label={\alph*)}]
		\item $d(x,y) = 0 \Leftrightarrow x=y$
		\item $d(x,y) = d(y,x)$ \begriff[Metrik!]{Symmetrie}\index{Symmetrie!Metrik}
		\item $d(x,z)\le d(x,y) + d(y,z)$ \begriff{Dreiecksungleichung}[!Metrik]
	\end{enumerate}

	$(X,d)$ heißt \begriff{metrischer Raum}.
\end{*definition}
\stepcounter{theorem}
\begin{example}
	\begriff{Diskrete Metrik} auf bel. Menge $X$ ist \[ d(x,y) = \begin{cases}0& x=y \\ 1 & x\neq y \end{cases} \] ist offenbar Metrik.
\end{example}
\begin{example}
	Sei $(X,d)$ metrischer Raum, $Y\subset X$\\
	$\Rightarrow (Y,\tilde{d})$ ist metrischer Raum mit \begriff{induzierte Metrik} $\tilde{d}(x,y) := d(x,y)\,\forall x,y\in X$.
\end{example}

\begin{*definition}[Norm]
	Sei $X$ Vektorraum über $K=\mathbb{R}$ bzw. $K=\mathbb{C}$.
	
	Abbildung \mathsymbol{.}{$\Vert.\Vert$}$: X\rightarrow\mathbb{R}$ heißt \begriff{Norm} auf $X$, falls $\forall x,y\in X$
	\begin{enumerate}[label={\alph*)}]
		\item $\Vert x\Vert = 0$ \gls{gdw} $x = 0$
		\item \label{norm_2} $\Vert \lambda\cdot x\Vert = |\lambda| \cdot \Vert x \Vert\,\forall \lambda\in K$ (\begriff{Homogenität})
		\item \label{norm_3} $\Vert x + y\Vert \le \Vert x \Vert + \Vert y \Vert$ \begriff{Dreiecksungleichung}[!Vektorraum]
	\end{enumerate}

	$(X,\Vert . \Vert)$ heißt \begriff{normierter Raum}
\end{*definition}
\begin{*definition}[Halbnorm]
	$\Vert . \Vert:X\rightarrow\mathbb{R}_{\ge0}$ heißt \begriff{Halbnorm}, falls nur \ref{norm_2} und \ref{norm_3} gelten.
\end{*definition}

\begin{conclusion}
	\begin{itemize}
		\item $\Vert x\Vert\ge 0$
		\item $\vert \; \Vert x\Vert - \Vert y\Vert \; \vert \leq \Vert x-y\Vert$
	\end{itemize}
\end{conclusion}

\begin{proposition}
	Sei $(X,\Vert .\Vert)$ normierter Raum.\\
	$\Rightarrow X$ ist metrischer Raum mit Metrik $d(x,y):=\Vert x - y \Vert\,\forall x,y\in X$.
\end{proposition}
\begin{example}
	\label{norm_r}
	Man hat u.a. folgende Normen auf $\mathbb{R}^n$:
	\begin{description}
		\item[\begriff{$p$-Norm}] $\vert x\vert_p:=\left(\sum_{i=1}^n |x_i|^p\right)^\frac{1}{p}\;(1\le p<\infty)$
		\item[\begriff{Maximum-Norm}] $|x|_\infty :=\max\{|x_i| \mid i=1,\dots,n\}$
	\end{description}

	Standardnorm im $\mathbb{R}^n: \vert \cdot \vert:=\vert \cdot \vert_{p=2}$ heißt \begriff{euklidische Norm}
\end{example}
\begin{*definition}[Skalarprodukt]
	$\langle x,y\rangle:=\sum_{i=1}^n x_i y_i$ heißt \begriff{Skalarprodukt}[!$\mathbb{R}$] (\begriff{inneres Produkt}) von $x,y\in\mathbb{R}^n$.
	
	Offenbar ist $\langle x,x\rangle = |x|^2\,\forall x\in\mathbb{R}^n$ (\highlight{ausschließlich für Euklidische Norm})\\
	Man hat $|\langle x,y\rangle | \le |x|\cdot |y|\,\forall x,y\in\mathbb{R}^n$ (\begriff{\person{Cauchy}-\person{Schwarz}'sche Ungleichung})
\end{*definition}
\begin{example}
	$X=\mathbb{C}^n$ ist Vektorraum über $\mathbb{C}$, $x=(x_1,\dotsc,x_n)\in\mathbb{C}^n, x_i\in\mathbb{C}$.
	
	Analog zu \ref{norm_r} sind $\vert\cdot\vert_p$ und $\vert\cdot\vert_\infty$ Normen auf $\mathbb{C}^n$
	
	$\langle x,y\rangle :=\sum_{i=1}^n \overline{x_i} y_i \,\forall x,y\in\mathbb{C}$ heißt \begriff{Skalarprodukt}[!$\mathbb{C}$] von $x,y\in\mathbb{C}^n$.
	
	$x,y\in\mathbb{R}^n (\mathbb{C}^n)$ heißen \begriff{orthogonal}, falls $\langle x,y\rangle = 0$.
\end{example}
\begin{example}
	Sei $M$ beliebige Menge, $f:M\rightarrow \mathbb{R}$.
	\begin{itemize}
		\item $\Vert f \Vert :=\sup\{ \vert f(x)\vert \mid x\in M\}$ \begriff{Supremumsnorm}
		\item \mathsymbol{B}{$B$}$(M):=\{ f:M\rightarrow \mathbb{R} \mid\; \Vert f \Vert < \infty \}$ \begriff{Menge der beschränkten Funktionen}
	\end{itemize}
\end{example}

\begin{example}
	$\Vert x\Vert:=\vert x_1\vert$ ist Halbnorm auf $X=\real^n$, da $\Vert(0,1)\Vert=0$, aber $(0,1)\neq 0$
\end{example}

\begin{*definition}
	Normen $\Vert .\Vert_1, \Vert .\Vert_2$ auf $X$ heißen \begriff[Norm!]{äquivalent}, falls $\exists \alpha,\beta > 0:\alpha \Vert x \Vert_1 \le \Vert x\Vert_2 \le \beta \Vert x\Vert_1 \,\forall x\in X$
\end{*definition}

\begin{example}
	$\vert x \vert_\infty \le \vert x\vert_p \le \sqrt[p]{n}\cdot\vert x\vert_\infty$, d.h. $\vert\cdot\vert_\infty$ und $\vert\cdot\vert_p$ sind äquivalent für alle $p\ge 1$
\end{example}
\begin{proof}
	$\vert x\vert_\infty=(\max\{\vert x_1\vert,...\}^p)^{\frac{1}{p}}\le \left(\sum_{j=1}^n \vert x_j\vert^p \right)^{\frac{1}{p}}=\vert x\vert_p\le (n\cdot\max\{\vert x_1\vert,...\}^p)^{\frac{1}{p}}=\sqrt[p]{n}\cdot \vert x\vert_\infty$
\end{proof}

\begin{conclusion}
	$\vert\cdot\vert_p, \vert\cdot\vert_q$ sind äquivalent auf $\mathbb{R}^n\,\forall p,q\ge 1$.
\end{conclusion}

\begin{*definition}
    \begin{itemize}
    \item $B_r(a):=\{ x\in X \mid d(a,x) < r \}$ heißt (offene) \begriff{Kugel} um $a$ mit Radius $r > 0$
    \item $B_r[a]:=\bar{B}_r(a):=\{ x\in X \mid d(a,x) \le r \}$ heißt (abgeschlossene) \begriff{Kugel} um $a$ mit Radius $r > 0$
    \end{itemize}
    Hinweis: muss keine "`übliche"' Kugel sein, zum Beispiel $\{ x\in \mathbb{R}^n \mid d(0,x) = \Vert x\Vert_{\infty} < 1 \}$ hat die Form eines "`üblichen"' Quadrats.
    \begin{itemize}
        \item Menge $M\subset X$ heißt \begriff[Menge!]{offen}, falls $\forall x\in M\,\exists \epsilon > 0: B_\epsilon(x) \subset M$
        \item Menge $M\subset X$ ist \begriff[Menge!]{abgeschlossen}, falls $X\setminus M$ offen
        \item $U\subset X$ \begriff{Umgebung} von $M$, falls $\exists V\subset X$ offen mit $M\subset V\subset U$
        \item $x\in M$ \begriff{innerer Punkt}, von $M$, falls $\exists \epsilon > 0: B_\epsilon(x)\subset M$
        \item $x\in X\setminus M$ \begriff{äußerer Punkt} von $M$, falls $\exists \epsilon > 0: B_\epsilon(x)\subset X\setminus M$
        \item $x\in X$ heißt \begriff{Randpunkt}, von $M$, wenn $x$ weder innerer noch äußerer Punkt
        \item \mathsymbol{int}{$\Int$}$ M:=$ Menge aller inneren Punkte von $M$, heißt \begriff{Inneres} von $M$
        \item \mathsymbol{ext}{$\Ext$}$M:=$ Menge aller äußeren Punkte von $M$, heißt \begriff{Äußeres} von $M$.
        \item \mathsymbol{p}{$\partial$}$M:=$ Menge der Randpunkte von $M$, heißt \begriff{Rand} von $M$
        \item \mathsymbol{cl}{$\cl$}$:=\overline{M} = \Int M \cup \partial M$ heißt \begriff{Abschluss} von $M$
        \item $M\subset X$ heißt \begriff{beschränkt}[!Menge], falls $\exists a\in X, r>0: M\subset B_r(a)$
        \item $x\in X$ heißt \gls{hp} von $M$, falls $\forall \epsilon > 0$ enthält $B_\epsilon(x)$ unendlich viele Elemente aus $M$
        \item $x\in M$ heißt \begriff{isolierter Punkt} von $M$, falls $x$ kein Häufungspunkt
        \end{itemize}
\end{*definition}

\begin{example}
	\begin{enumerate}
		\item $X=\real$ mit $d(x,y)=\vert x-y\vert$
		\begin{itemize}
			\item $(a,b)$ offen, $[a,b]$ abgeschlossen
			\item $[a,b)$ halboffen, aber beschränkt
			\item $\inn (a,b)=\inn [a,b]=(a,b)$
			\item $\ext (a,b)=\ext [a,b]=(-\infty,a)\cup(b,\infty)$
			\item $\partial (a,b)=\partial [a,b]=\{a,b\}$
			\item $\cl (a,b)=\cl [a,b]=[a,b]$
			\item $\ratio$ weder offen noch abgeschlossen in $\real$, $\inn\ratio=\emptyset$, $\partial\ratio=\real$
			\item $\real\backslash\{0\}$ offen, $\natur$ in $\real$ abgeschlossen und nicht beschränkt
			\item $[0,3]$ ist Umgebung von $[1,2]$, $B_r(a)$ ist Umgebung von $a$
			\item $a$ ist HP von $(a,b)$ und $[a,b]$, wenn $a<b$, aber nicht von $[a,a]$
			\item alle $a\in\real$ sind HP von $\ratio$
		\end{itemize}
	\item für $X=\real$ mit diskreter Metrik: $x\in M\Rightarrow B_{0,5}(x)=\{x\}$ \\
	$\Rightarrow$ alle $M\subset\real$ sind offen und abgeschlossen
	\end{enumerate}
\end{example}

\begin{lemma}
	Sei $(X,d)$ metrischer Raum. Dann
	\begin{enumerate}[label={\arabic*)}]
		\item $B_r(a)$ offene Menge $\forall r>0,a\in X$
		\item $M\subset X$ beschränkt $\Rightarrow\; \forall a\in X\,\exists r>0: M\subset B_r(a)$
	\end{enumerate}
\end{lemma}
\begin{proof}
	\begin{enumerate}[label={\arabic*)}]
		\item Sei $b \in B_r(a),\epsilon := r - a-d(a,b)>0$, dann gilt für beliebige $x \in B_{\epsilon}(b)$
		\begin{align*}
		d(a,x) &\overset{\Delta\text{-Ungl.}}{\leq} d(a,b) + d(b,x)\\
		&<d(a,b)+r-d(a,b)&\\
		&=r \Rightarrow B_{\epsilon}(b) \subset B_{\epsilon}(a) \beha &
		\end{align*}
		\item Sei $M\subset B_{\rho}(b),a\in X$ beliebig, $r:=\rho + d(a,b),m\in M$\\
		\begin{align*}
		\Rightarrow d(m,a) &\leq d(m,b)+d(b,a)&\\
		&<\rho + d(b,a) = r \Rightarrow m\in B_{r}(a)
		\end{align*}
	\end{enumerate}
\end{proof}

\begin{proposition}\label{proposition_topologie}
	Sei $(X,d)$ metrischer Raum, $\tau:=\{U\subset X \mid U \text{ offen}\}$. Dann
	\begin{enumerate}[label={\arabic*)}]
		\item \label{topologie_1} $X,\emptyset\in \tau$ offen
		\item \label{topologie_2} $\bigcap_{i=1}^n U_i\subset \tau$ falls $U_i\in\tau$ für $i=1,\dotsc,n$
		\item \label{topologie_3} $\bigcup_{U\in\tau'} U\in\tau$ falls $\tau'\in\tau$ 
	\end{enumerate}
\end{proposition}
\begin{proof}
	\begin{enumerate}[label={\arabic*)}]
		\item $X$ offen, da stets $B_{\epsilon}(x) \subset X$, Definition ``offen'' wahr für $\emptyset$
		\item Sei $X \in \bigcap_{i=0}^{n} U_i \Rightarrow \exists \epsilon_i > 0 \colon B_{\epsilon_i}(x) \subset U_i \forall i, \epsilon = \min\{\epsilon_1, \dots \epsilon_n\}$\\
		$\Rightarrow B_{\epsilon}(x) \in \bigcap_{i=0}^{n} U_i \beha$
		\item Sei $x \in \bigcup_{U\in\tau^{\prime}} U \Rightarrow \exists \tilde{U}\in \tau^{\prime}\colon x \in \tilde{U} \overset{\tilde{U} \text{ offen}}{\Rightarrow}\exists \epsilon > 0 \colon B_{\epsilon}(x) \subset \tilde{U} \in \bigcup_{U\in\tau^{\prime}} U \beha$.
	\end{enumerate}
\end{proof}

\begin{underlinedenvironment}[Hinweis]
	Durchschnitt beliebig vieler offener Mengen im Allgemeinen nicht offen \\
\end{underlinedenvironment}

\begin{example}
	$\bigcap (-\frac{1}{n},1+\frac{1}{n})=[0,1]$
\end{example}

\begin{conclusion}
	Sei $(X,d)$ metrischer Raum, $\sigma :=\{ V\subset X \mid  V \text{ abgeschlossen}\}$. Dann
	\begin{enumerate}[label={\arabic*)}]
		\item $X,\emptyset \in \sigma$ abgeschlossen
		\item $\bigcup_{i=1}^n V_i\subset\sigma$ falls $V_i\in\sigma_i$ für $i=1,\dotsc, n$
		\item $\bigcap_{V\in\sigma'} V\in\sigma$ falls $\sigma'\subset\sigma$
	\end{enumerate}
\end{conclusion}

\begin{*definition}[Topologie]
	Sei $X$ Menge, und $\tau$ Menge von Teilmengen von $X$, d.h. $\tau\subset\mathcal{P}(X)$.\\
	$\tau$ ist \begriff{Topologie} und $(X,\tau)$ \begriff{topologischer Raum}, falls \ref{topologie_1},\ref{topologie_2},\ref{topologie_3} aus \ref{proposition_topologie} gelten. \\
	Mengen $U\subset\tau$ heißen dann (per Definition) offene Mengen, folglich in metrischen Räumen definierte offene Mengen sind ein Spezialfall einer Topologie. \\
	\begin{underlinedenvironment}[beachte]
		$\tilde{\tau}=\{\emptyset, X\}$ ist stets Topologie für beliebige Menge $X$
	\end{underlinedenvironment}
\end{*definition}

\begin{proposition}
	Seien $\Vert.\Vert_1, \Vert.\Vert_2$ äquivalente Normen in $X$ und $U\subset X$. Dann \[ U\text{ offen bezüglich } \Vert .\Vert_1\; \Leftrightarrow\; U\text{ offen bzgl. } \Vert .\Vert_2 \]
\end{proposition}
\begin{proof}
	Übungsaufgabe
\end{proof}

\begin{proposition}
	Sei $(X,d)$ metrischer Raum und $M\subset X$: Dann
	\begin{enumerate}[label={\arabic*)}]
		\item $\Int M, \Ext M$ offen
		\item $\partial M, \cl M$ abgeschlossen
		\item $M = \Int M$, falls $M$ offen, $M=\cl M$ falls $M$ abgeschlossen
	\end{enumerate}
\end{proposition}
\begin{proof}
	\begin{enumerate}[label={\arabic*)}]
		\item Seien $x \in \inn M$, d.h. innere Punkte von $M \Rightarrow \exists \epsilon > 0 \colon B_{\epsilon}(x) \subset M$, da $B_{\epsilon}(x)$ offene Menge, ist jedes $y \in B_{\epsilon}(x)$ eine Teilemenge von $\inn M$ $\Rightarrow B_{\epsilon}(x) \subset M \beha$ ($\ext M$ analog)
		\item $\partial X\setminus (\inn M \cup \ext M)$ ist abgeschlossen, $\cl M = X\setminus\ext M$ abgeschlossen
		\item $M$ offen: es ist stets $\int M$ und da $M$ offen $M \subset \inn M \beha$  $\Rightarrow X\setminus M = \inn(X\setminus M) = \ext M = X \setminus \cl M \beha$.
		($M$ abgeschlossen analog)
	\end{enumerate}
\end{proof}
\section{Konvergenz}\setcounter{theorem}{0}
\begin{definition}[konvergent]
	Sei $(X,d)$ metrischer Raum. Folge $\{x_n\}_{n\in\mathbb{N}}$ in $X$, (d.h. $x_n\in X\,\forall n$) heißt \begriff[Folge!]{konvergent}, falls $x\in X$ existiert mit \[\forall \epsilon > 0 \,\exists n_0=n_0(\epsilon)\in\mathbb{N}: d(x_n, x) < \epsilon\quad \forall n\ge n_0\]
	
	$x$ heißt dann \begriff{Grenzwert} (auch Limes) der Folge.
	
	Notation: $x=$\mathsymbol{lim}{$\lim\limits_{n\rightarrow\infty}$}, $x_n\rightarrow x$ für $n\rightarrow\infty$, $x_n \overset{n\rightarrow\infty}{\longrightarrow}x$
	
	Folge heißt \begriff[Folge!]{divergent}, falls nicht konvergent.
\end{definition}

\begin{conclusion}
	Für Folge $\{x_n\}$ gilt: \[ x=\lim\limits_{n\rightarrow\infty}x_n \;\Leftrightarrow \text{Jede Kugel $B_\epsilon(x)$ enthält fast alle $x_n$} \]
\end{conclusion}

\begin{example}
	\begin{itemize}
		\item konstante Folge: Sei $\{x_n\}_n=\{x\}_n\in\natur$, d.h. $x=x_n$
		\item $X=\real$: Folge $\{\frac{1}{n}\}$ konvergent, Grenzwert 0
		\item $X=\real$: $\lim\limits_{n\to\infty} \sqrt[n]{x}=1$
		\item $X=\real$: $\{-1\}^n$ ist divergent
	\end{itemize}
\end{example}

\begin{proposition}[Eindeutigkeit des Grenzwertes]
	Sei $(X,d)$ metr. Raum, $\{x_n\}$ Folge in $X$. Dann \[ x,x' \text{ Grenzwert von $\{x_n\}$} \;\Rightarrow\; x = x' \]
\end{proposition}
\begin{proof}
	Sei $\varepsilon:=\frac{1}{3}$, $d(x,x')>0\Rightarrow\exists m\in\natur:d(x_m,x)<\varepsilon$, $d(x_m,x')<\varepsilon$\\
	$3\varepsilon=d(x,x')\le d(x_m,x)+d(x_m,x')< 2\varepsilon\Rightarrow\lightning\Rightarrow d(x,x')=0$
\end{proof}

\begin{proposition}
	Sei $(X,d)$ metrischer Raum, $\{x_n\}$ konvergente Folge in $X$\\
    $\Rightarrow$ $\{x_n\}$ ist beschränkt.
\end{proposition}
\begin{proof}
	Sei $\lim\limits_{n\to\infty} x_n=x\Rightarrow$ für $\varepsilon=1\exists n_0:d(x_n,x)<1$ mit $r=\max\{d(x,x_n)\}+1$ folgt: $x_n\in B_r(x)\Rightarrow$ beschränkt
\end{proof}

\begin{example}
	$X=\real$ mit diskreter Metrik: betrachte $\{x_n\}$ \\
	angenommen $\lim\limits_{n\to\infty} x_n=x\Rightarrow$ für $\varepsilon=\frac{1}{2}\exists n_0: x_n\in B_{0,5}(x)=\{x\}$ \\
	$\Rightarrow$ fast alle $x_n$ sind gleich $x$ bei Konvergenz $\Rightarrow \left\lbrace \frac{1}{n}\right\rbrace$ ist divergent $\Rightarrow$ Konvergenz ist abhängig von Metrik
\end{example}

\begin{example}
	$X=\comp$ mit $\vert\cdot\vert$. betrachte $\{z^n\}$ für $z\in\comp$
	\begin{itemize}
		\item $\vert z \vert < 1$: $\forall\varepsilon >0\exists n_0:\vert z^n-n_0\vert<\varepsilon\Rightarrow\lim\limits_{n\to\infty} z^n=0$
		\item $\vert z \vert > 1$: $\forall r>0\exists n_0:\vert z^{n_0}-0\vert=\vert z\vert^{n_0}>r\Rightarrow$ es gibt also kein $r>0\Rightarrow \{z^n\}$ ist nicht beschränkt $\Rightarrow$ divergent
		\item $z=1$ offenbar $\lim\limits_{n\to\infty} 1^n=1$
		\item $\vert z\vert =1$, aber $z\neq 1$: angenommen $\lim\limits_{n\to\infty} z^n=\tilde z\Rightarrow \varepsilon=\frac{1}{2}\vert z-1\vert\Rightarrow \vert z-\tilde{z}\vert<\varepsilon\Rightarrow 2\varepsilon=\vert z-1\vert=\vert z^{n_0}\vert\cdot\vert z-1\vert=\vert z^{n_0}+1-\tilde{z}+\tilde{z}-z^{n_0}\vert\le\vert z^{n_0}+1-\tilde{z}\vert+\vert\tilde{z}-z^{n_0}\vert< 2\varepsilon\Rightarrow\lightning\Rightarrow \{z^n\}$ divergent
	\end{itemize}
\end{example}

\begin{example}
	$\lim\limits_{n\to\infty} \sqrt[n]{n}=1$, denn: \\
	\begin{align*}
		x_n:=\sqrt[n]{n}-1&\ge 0 \\
		n=(1+x_n)^n&\ge 1+\binom{n}{2}\cdot x_n^2 \\
		n-1 &\ge n\frac{n-1}{2x_n^2} \\
		x_n=\sqrt[n]{n}-1 &\le \sqrt{\frac{2}{n}}\le \varepsilon
	\end{align*}
\end{example}

\begin{example}
	$\lim\limits_{n\to\infty} \frac{\log_a n}{n}=0$ für $a>1$, denn \\
	$1<\sqrt[n]{n}< a^{\varepsilon}\Rightarrow 0 < \frac{\log_a n}{n} < \varepsilon$
\end{example}

\begin{definition}[Teilfolge, Häufungswert]
	Sei $\{x_n\}$ beliebige Folge in $X$, $\{n_k\}_{k\in\mathbb{N}}$ Folge in $\mathbb{N}$ mit $n_{k+1} > n_k\,\forall k\in\mathbb{N}$. Dann heißt $\{x_{n_k}\}_{k\in\mathbb{N}}$ \gls{tf} von $\{x_n\}_{n\in\mathbb{N}}$.
	
	$\gamma\in X$ heißt \gls{hw} (auch Häufungspunkt) der Folge $\{x_n\}$, falls $\forall \epsilon > 0$ enthält $B_\epsilon(\gamma)$ unendlich viele $x_n$.
	\begin{underlinedenvironment}[beachte]
		HP der Folge muss nicht HP der Menge $\{x_n\}$ sein, z.B. konstante Folge
	\end{underlinedenvironment}
\end{definition}

\begin{proposition}\label{tfprinzip}
	\proplbl{tfprinzip}
	Sei $\{x_n\}$ Folge im metrischen Raum $(X,d)$. Dann
	\begin{enumerate}[label={\arabic*)}]
		\item $x_n\rightarrow x \;\Rightarrow\; x_{n_k} \overset{n\rightarrow\infty}{\longrightarrow} x$ für jede \gls{tf} $\{x_{n_k}\}_k$
		\item $\gamma$ ist \gls{hw} der Folge $\{x_n\}$ $\Leftrightarrow$ $\exists$\gls{tf} $\{x_{n_k}\}: x_{n_k} \overset{n\rightarrow\infty}{\longrightarrow} \gamma$
		\item \begriff{Teilfolgenprinzip}: Jede \gls{tf} $\{x_{k'}\}$ von $\{x_n\}$ hat \gls{tf} $\{x_{k''}\}$ mit $x_{n''}\rightarrow x$ $\Rightarrow$ $x_n \rightarrow x$
	\end{enumerate}
\end{proposition}
\begin{proof}
	\begin{enumerate}
		\item folgt aus Definition
		\item $(\Rightarrow):\exists n_k:x_{n_k}\in B_{\frac{1}{k}}(x), n_{k+1}>n_k\Rightarrow \{x_{n_k}\}$ ist TF mit $x_{n_k}\to x$ \\
		$(\Leftarrow):x_{n_k}\to x\Rightarrow B_{\varepsilon}(x)$ fast alle $x\beha$
		\item Übungsaufgabe
	\end{enumerate}
\end{proof}

\begin{example}
	$\{(-1)^n\}$ hat TF $\{(-1)^{2k}\}$ und $\{(-1)^{2k+1}\}$ mit Grenzwert +1 und -1 $\Rightarrow \{(-1)^n\}$ ist divergent, da es 2 HW gibt.
\end{example}

\begin{proposition}
	Sei $(X,d)$ metrischer Raum, $M\subset X$ Teilmenge. Dann
	\[ M\text{ abgeschlossen} \quad\Leftrightarrow\quad \text{für jede konv. Folge $\{x_n\}$ in $M$ gilt: }\lim\limits_{n\rightarrow\infty} x_n\in M \]
\end{proposition}
\begin{proof}
	$(\Rightarrow):$ sei $\{x_n\}\in M$ mit $x_n\to x\notin M\Rightarrow\exists\varepsilon:B_{\varepsilon}\subset X\backslash M\Rightarrow x_n\not\to x\lightning\beha$ \\
	$(\Leftarrow):$ sei $X\backslash M$ nicht offen, also abgeschlossen $\Rightarrow\exists x\in X\backslash M:B_{\varepsilon}(x)\cap M\neq\emptyset\Rightarrow\exists x_n\in B_{\frac{1}{n}}(x)\cap M\Rightarrow x_n\to x\in M\lightning\Rightarrow X\backslash M$ offen
\end{proof}

\subsection{Konvergenz im normierten Raum $X$}
$x_n\to x$ in $(X,\Vert .\Vert)$ und $\lambda_n\to\lambda$ in $(\real,\vert\cdot\vert)$

\begin{proposition}
	Sei $X$ normierter Raum, $\{x_n\}, \{y_n\}$ in $X$, $\{\lambda_n\}$ in $K$ mit $\lim x_n = x, \lim y_n = y$. Dann
	\begin{enumerate}[label={\arabic*)}]
		\item $\{x_n \pm y_n\}$ konvergiert und $\lim\limits_{n\rightarrow\infty}x_n \pm y_n = \lim\limits_{n\rightarrow\infty} x_n \pm \lim\limits_{n\rightarrow\infty} y_n$
		\item $\{\lambda_n x_n\}$ konvergiert und $\lim\limits_{n\rightarrow\infty} \lambda_n x_n = \lim\limits_{n\rightarrow\infty} \lambda_n \cdot \lim\limits_{n\rightarrow\infty}x_n$
		\item $\lambda\neq 0 \;\Rightarrow\;\lim\limits_{n\rightarrow\infty} \frac{1}{\lambda_n} = \frac{1}{\lambda}$ (in $K$) für $\{\frac{1}{\lambda_n}\}_{n\ge\tilde{n}}$ ($\lambda_n\neq 0\,\forall n\ge\tilde{n}$)
	\end{enumerate}
\end{proposition}
\begin{proof}
	\begin{enumerate}
		\item Übungsaufgabe
		\item $\{x_n\}$ beschränkt $\Rightarrow\exists r>\vert\lambda\vert>0:\Vert rx_n\Vert\le r$\\
		$\varepsilon>0\Rightarrow\exists n_0:\vert \lambda_n-\lambda\vert<\frac{\varepsilon}{2r},\Vert x_n-x\Vert < \frac{\varepsilon}{2r}$ \\
		$\Rightarrow \Vert \lambda_{x_n}-\lambda_n\Vert\le \Vert\lambda_nx_n-\lambda x_n\Vert+\Vert\lambda x_n\lambda x\Vert=\vert\lambda_n-\lambda\cdot\Vert x_n\Vert+\vert\lambda\vert\cdot\Vert x_n-x\Vert\le \frac{\varepsilon}{2r}\cdot r+r\cdot\frac{\varepsilon}{2r}=\varepsilon\beha$
		\item offenbar: $\exists\tilde n:\lambda_n\neq 0$ für $\varepsilon>0\exists n_0:\vert\lambda-\lambda_n\vert<m\cdot n\cdot \left\lbrace \left(\frac{\vert x\vert}{2}\right),\left(\frac{\varepsilon\cdot\vert\lambda\vert^2}{2}\right)\right\rbrace\Rightarrow\frac{1}{2}\cdot\vert\lambda\vert\le \vert\lambda\vert-\vert\lambda+\lambda_n\vert\le \lambda_n\Rightarrow ... \Rightarrow$ Behauptung
	\end{enumerate}
\end{proof}

\begin{conclusion}
	Seien $\{\lambda_n\}, \{\mu_n\}$ Folgen in $K$ mit $\lambda_n\rightarrow\lambda,\mu_n\rightarrow\mu$. Dann
	\begin{enumerate}[label={\arabic*)}]
		\item $\lambda_n + \mu_n\rightarrow \lambda + \mu, \lambda_n \mu_n\rightarrow\lambda \mu$
		\item falls $\lambda\neq 0$ (\gls{obda} $\lambda_n\neq 0$): $\frac{\mu_n}{\lambda_n}\rightarrow\frac{\mu}{\lambda}$
	\end{enumerate}
\end{conclusion}

\begin{definition}[Nullfolge]
	$\{x_n\}$ im normierten Raum heißt \begriff{Nullfolge}, falls $x_n\to 0$
\end{definition}

\begin{lemma}
	\proplbl{sandwich_lemma}
	\begin{enumerate}[label={\arabic*)}]
		\item Im metrischen Raum $X$ gilt:$x_n\rightarrow x$ in $X$ $\Leftrightarrow\;d(x_n,x)\rightarrow 0$ in $\mathbb{R}$
		\item Sei $0\le \alpha_n\le\beta_n\,\forall n\in\mathbb{N}, \alpha_n, \beta_n\in\mathbb{R}, \beta_n\rightarrow 0$\\
		$\Rightarrow \alpha_n\rightarrow 0$ \begriff{Sandwich-Prinzip}
	\end{enumerate}
\end{lemma}
\begin{proof}
	\begin{enumerate}
		\item benutze $d(x_n,x)<\varepsilon\iff\vert d(x_n,x)-0\vert<\varepsilon$
		\item $\varepsilon>0\Rightarrow\exists n:\beta_n=\vert\beta_n-0\beta_n\vert<\varepsilon\Rightarrow \alpha_n=\vert\alpha_n-0\vert\le \beta_n<\varepsilon\beha$
	\end{enumerate}
\end{proof}

\begin{proposition}
	Sei $X$ normierter Raum, $\{x_n\}$ in $X$. Dann\\
	$x_n\rightarrow x$ in $X$ $\Rightarrow$ $\Vert x_n\Vert \rightarrow\Vert x\Vert$ in $\mathbb{R}$
\end{proposition}
\begin{proof}
	$0\le \vert \;\Vert x_n\Vert-\Vert x\Vert\;\vert\le \Vert x_n-x\Vert\to 0\overset{\text{\propref{sandwich_lemma}}}{\Rightarrow}$ Behauptung
\end{proof}

\begin{proposition}
	Seien $(X,\Vert .\Vert_1)$, $(X,\Vert.\Vert_2)$ normierte Räume mit äquivalenten Normen. Dann
	
	$x_n\rightarrow x$ in $(X,\Vert.\Vert_1)$ $\Leftrightarrow$ $x_n\rightarrow x$ in $(X,\Vert.\Vert_2)$
\end{proposition}
\begin{proof}
	Es gibt $a,b>0:a\cdot\Vert y\Vert_1\le \Vert y\Vert_2\le b\cdot\Vert y\Vert_1$ \\
	$(\Rightarrow)$: es ist $0\le \Vert x_n-x\Vert_2\le b\cdot\Vert x_n-x\Vert_1\to 0\beha$ \\
	$(\Leftarrow)$: analog
\end{proof}

\begin{example}
	$X=\real^n$ bzw. $\comp^n$: $x_n\to x$ bezüglich $\Vert .\Vert_1\iff x_n\to x$ bezüglich $\Vert .\Vert_2$, somit Konvergenz in $\real^n$ bzw. $\comp^n$ unabhängig von Norm.
\end{example}

\begin{proposition}[Konvergenz in $\mathbb{R}^n$/$\mathbb{C}^n$ bzgl. Norm]
	Sei $\{x_n\}$ Folge mit $x_n = (x_n^1, \dotsc, x_n^n)\in\mathbb{R} (\mathbb{C}^n)$, $x=(x^1, \dotsc,x^n)\in\mathbb{R}^n (\mathbb{C}^n)$.
	
	$\lim\limits_{n\rightarrow\infty} x_n = x$ in $\mathbb{R}^n (\mathbb{C}^n)$ $\Leftrightarrow$ $\lim\limits_{n\rightarrow\infty} x_k^j = xj$ in $\mathbb{R}$ bzw. $\mathbb{C}\,\forall j=1,\dotsc,n$
\end{proposition}
\begin{proof}
	nur in $\real^n$ \\
	$(\Rightarrow)$: sei $x_k\to x$ in $\real^n$ bezüglich $\vert\cdot\vert_p\Rightarrow x_n\to x$ bezüglich $\vert\cdot\vert_\infty$. Wegen $\vert x_k^j-x^j\vert\le \vert x_k-x\vert_\infty\to 0$ hieraus folgt die Behauptung \\
	$(\Leftarrow)$: sei $x_k^j\to x^j\Rightarrow\vert x_k-x\vert_1=\vert x_k^1-x^1\vert +...+\vert x_k^n-x^n\vert\to 0\Rightarrow x_k\to x$ bezüglich $\vert\cdot\vert_1\beha$
\end{proof}

\begin{underlinedenvironment}[Hinweis]
	zukünftig bei Konvergenz in $\real^n$ oder $\comp^n$ in der Regel keine Angabe der konkreten Norm.
\end{underlinedenvironment}

\begin{remark}
	offenbar gilt: \\
	$z_n=x_n+iy_n\to z=x+iy\iff (x_n,y_n)\to (x,y)$ in $\real^2$ bezüglich $\vert\cdot\vert\iff \realz(z_n)\realz(z)$ und $\imagz(z_n)\to\imagz(z)$
\end{remark}

\begin{example}
	$\{x_k\}=\{(\sqrt{k+1}-\sqrt{k},\sqrt{k+\sqrt{k}}-\sqrt{k})\}$ Folgen in $\real^2$ \\
	es ist $0\le x_k^1=\sqrt{k+1}-\sqrt{k}=\frac{1}{\sqrt{k+1}+\sqrt{k}}<\frac{1}{\sqrt{k}}\to 0\Rightarrow x_k^1\to 0$ \\
	$x_k^2=\sqrt{k+\sqrt{k}}-\sqrt{k}=\frac{\sqrt{k}}{\sqrt{k+\sqrt{k}}+\sqrt{k}}=\frac{1}{\sqrt{1+\frac{1}{\sqrt{k}}}+1}\to \frac{1}{2}$ \\
	$\Rightarrow \lim\limits_{k\to\infty} x_k=\frac{1}{2}$
\end{example}

\begin{example}
	$z_k=\frac{1+ki}{1+k}\to i$, denn: \\
	$\realz(z_n)=\frac{1}{1+k}\to 0$ und $\imagz(z_n)=\frac{k}{k+1}\to 1\Rightarrow\to (0,1)=i$
\end{example}

\subsection{Konvergenz in $\mathbb{R}$}
\begin{proposition}
	Seien $\{x_n\},\{y_n\},\{z_n\}$ Folgen in $\mathbb{R}$. Dann
	\begin{enumerate}[label={\arabic*)}]
		\item $x_n \le y_n\,\forall n\ge n_0, x_n\rightarrow x, y_n\rightarrow y\;\Rightarrow x\le y$
		\item $x_n\le y_n\le z_n\,\forall n\ge n_0, x_n\rightarrow c,z_n\rightarrow c \;\Rightarrow y_n\rightarrow c$ (\begriff{Sandwich-Prinzip})
	\end{enumerate}
\end{proposition}
\begin{proof}
	\begin{enumerate}
		\item angenommen $x>y$, sei $\varepsilon :=\frac{1}{2}(x-y)>0$ \\
		$\Rightarrow\exists m: x_n\in B_{\varepsilon}(x), y_n\in B_{\varepsilon}(y)$ \\
		$\Rightarrow y_n< y+\varepsilon=x-\varepsilon<x_n\Rightarrow\lightning\beha$
		\item offenbar $0\le y_n-x_n\le z_n-x_n\to 0\Rightarrow y_n-x_n\to 0\Rightarrow\to c$
	\end{enumerate}
\end{proof}

\begin{definition}[monoton]
	Folge $\{x_n\}$ heißt \begriff[monoton!]{wachsend} / \begriff[monoton!]{fallend}, falls gilt:
	
	$x_n \le x_{n-1}\;(x_n\ge x_{n+1})\,\forall n\in\mathbb{N}$ (in beiden Fällen heißt Folge \begriff{monoton}).
	
	Falls stets "`$<$"' ("`$>$"') ist $\{x_n\}$ \begriff[monoton!]{strikt}
\end{definition}

\begin{proposition}
	Sei $\{x_n\}$ in $\mathbb{R}$ monoton und beschränkt.\[
	\{x_n\}\text{ konvergiert gegen }x:=
	\left\lbrace
		\begin{aligned}
			&\sup \{x_n \mid n\in\mathbb{N}\}, \\
			&\inf\{x_n \mid n\in\mathbb{N}\}, \\
		\end{aligned}
	\right.
	\text{ falls monoton }\;
	\begin{aligned}
		&\text{wachsend}\\
		&\text{fallend}
	\end{aligned}
	\]
\end{proposition}
\begin{proof}
	Sei $\{x_n\}$ monoton wachsend und beschränkt $\Rightarrow x=\sup\{x_n\}$ existiert $\Rightarrow \varepsilon>0\Rightarrow \exists m:x-\varepsilon\le x_m\le x_n\le x\beha$ \\
	Monoton fallend analog
\end{proof}

\begin{example}
	Sei $x_{n+1}=\frac{1}{2}(x_n+\frac{a}{x_n})$ \\
	vollständige Induktion: $x_n>0$, somit $\{x_n\}$ rekursiv eindeutig definiert \\
	$\Rightarrow x_{n+1}^2-a=\frac{1}{4}(x_n+\frac{a}{x_n})^2-a=\frac{1}{4}(x_n-\frac{a}{x_n})^2\ge 0$\\
	$\Rightarrow x_n-x_{n+1}=\frac{1}{2x_n}(x_n^2-a)\ge 0$ \\
	$\Rightarrow \{x_n\}$ ist mon. fallend, beschränkt $\Rightarrow x_n\to x\in\real$ \\
	da $x_{n+1}\cdot x_n=\frac{1}{2}(x_n^2+a)\Rightarrow x^2=\frac{1}{2}(x^2+a)\Rightarrow x^2=a\Rightarrow \lim\limits_{n\to\infty} x_n=\sqrt{a}$
	
	\textbf{Fehlerabschätzung:} $x_{n+1}-\sqrt{a}=\frac{1}{2x_n}(x_n-\sqrt{a})^2\le \frac{1}{2\sqrt{a}}(x_n-\sqrt{a})^2$, so genannte \begriff[Konvergenz!]{quadratische} Konvergenz (schnelle Konvergenz, vgl. Newton-Verfahren), d.h. die Anzahl der signifikanten Dezimalstellen verdoppelt sich mit jedem Schritt!
\end{example}

\begin{example}
	$\lim\limits_{n\to\infty} \frac{z^n}{n!}=0$ \\
	betrachte reelle Folge $a_n:=\frac{\vert z^n\vert}{n!}\Rightarrow a_{n+1}=\frac{\vert z\vert}{n+1}a_n$ \\
	$\Rightarrow\exists\tilde{n}:\{a_n\}$ fallend $\left( \frac{\vert z\vert}{\tilde{n}+1}<1\right)\Rightarrow a_n\to a$ \\
	$\Rightarrow a=0\cdot a=0\Rightarrow \vert \frac{z^n}{n!}-0\vert=\frac{vert z\vert^n}{n!}\to 0\beha$
\end{example}

\begin{theorem}[\person{Bolzano}-\person{Weierstraß}]\label{bolzano_weierstrass}
	$\{x_n\}$ beschränkte Folge in $\mathbb{R}$ $\Rightarrow$ $\{x_n\}$ hat konvergente \gls{tf}.
\end{theorem}
\begin{proof}
	es gibt $y_0,y'_0:y_0\le x_n\le y'_0$ \\
	rekursive Definition von $y_n,y'_n\in\real$ \\
	$z{n+1}:=\frac{y_n+y'_n}{2}\Rightarrow \left\lbrace 
	\begin{aligned}
	 &\text{unendlich viele }y_n\in[z_{n+1},y'_n] & y_{n+1}=z_{n+1} &\quad y'_{n+1}=y'_n \\
	 &\text{sonst} & y_{n+1}=y_n &\quad y'_{n+1}=z_{n+1} \\
	\end{aligned}\right.\Rightarrow$ Folge $Y_n=[y_n,y'_n]$ ist Intervallschachtelung in $\real\Rightarrow\exists y\in\bigcap Y_n\Rightarrow y$ ist HW in $\{x_n\}\beha$ 
\end{proof}

\begin{example}
	$\{z_n\}$ für $z\in\comp,\vert z\vert=1,z\neq 1$: ist divergent, aber $\{\realz(z_n)\}$ und $\{\imagz(z_n)\}$ sind beschränkte Folgen in $\real$ \\
	$\Rightarrow\exists$ TF $\{n'\}$ von $\{n\}$ mit $\realz(z^{n'})\to \alpha$ \\
	$\Rightarrow\exists$ TF $\{n''\}$ von $\{n\}$ mit $\imagz(z^{n''})\to \beta$ \\
	$\Rightarrow z^n\to \alpha+i\beta\Rightarrow \{z_n\}$ hat konvergente TF in $\comp$!
\end{example}

\subsection{Oberer und Unterer Limes}
\begin{definition}
	Seien $\{x_n\}$ beschränkte Folgen in $\mathbb{R}$.\\
	$H:=\{ \gamma\in\mathbb{R} \mid \gamma \text{ ist \gls{hw} von }\{x_n\}\}$ ($\neq \emptyset$ nach \ref{bolzano_weierstrass})
	
	\begin{tabularx}{\textwidth}{lX}
		\mathsymbol*{limsup}{$\limsup$} $\limsup\limits_{n\rightarrow\infty} x_n := \overline{\lim}_{n\rightarrow\infty} x_n =:\sup H$ & \begriff{Limes superior} von $\{x_n\}$ \\[0.5cm]
		\mathsymbol*{liminf}{$\liminf$} $\liminf\limits_{n\rightarrow\infty} x_n = \underline{\lim}_{n\rightarrow\infty} x_n :=\inf H$  & \begriff{Limes inferior} von $\{x_n\}$
	\end{tabularx}

\begin{underlinedenvironment}[beachte]
	$\limsup$ und $\liminf$ existieren stets für beschränkte Folgen!
\end{underlinedenvironment}
\end{definition}

\begin{proposition}
	Sei $\{x_n\}$ beschränkte Folge in $\mathbb{R}$. Dann
	\begin{enumerate}[label={\arabic*)}]
		\item Sei $\{x_{n'}\}$ \gls{tf} mit $x_{n'}\rightarrow\gamma \;\Rightarrow \;\liminf\limits_{n\rightarrow\infty} x_n \le \gamma \le \limsup\limits_{n\rightarrow\infty} x_n$
		\item $\gamma' :=\liminf\limits_{n\rightarrow\infty} x_n$ und $\gamma'' := \limsup\limits_{n\rightarrow\infty} x_n$ sind \gls{hw} von $\{x_n\}$
		
		\begin{tabular}{ll}
		(folglich)& $\inf H = \min H, \sup H = \max H$ und \\
		& $\exists$ \gls{tf} $\{x_{n'}\}, \{x_{n''}\}, x_{n'}\rightarrow \gamma', x_{n''}\rightarrow\gamma''$
		\end{tabular}
		\item $x_n\rightarrow \alpha \;\Leftrightarrow \;\alpha = \liminf\limits_{n\rightarrow\infty} x_n = \limsup\limits_{n\rightarrow\infty} x_n$
	\end{enumerate}
\end{proposition}
\begin{proof}
	\begin{enumerate}
		\item $x\in H\overset{\text{\propref{tfprinzip}}}{\Rightarrow}$ Behauptung
		\item $\varepsilon>0\Rightarrow\exists x\in H\cap B_{\varepsilon}(x')$ \\
		$B_{\varepsilon}(x')$ offen $\Rightarrow\exists\tilde{\varepsilon}>0:B_{\tilde{\varepsilon}}(x')\subset B_{\varepsilon}(x')\Rightarrow$ unendlich viele $x_n$ in $B_{\varepsilon}(x')\Rightarrow$ Behauptung für $\liminf$
		\item Übungsaufgabe, Selbststudium
	\end{enumerate}
\end{proof}

\begin{example}
	$\{q_n\}\in\real$ sei Folge alle rationalen Zahlen in $(0,1)$ \\
	$\Rightarrow$ Menge aller HW ist $H=[0,1]\Rightarrow \liminf q_n=0$ und $\limsup q_n=1$
\end{example}

\subsection{Uneigentliche Konvergenz}
\begin{definition}[Uneigentliche Konvergenz]
	Folge $\{x_n\}$ in $\mathbb{R}$ konvergiert \begriff[Konvergenz!]{uneigentlich} gegen $+\infty (-\infty)$, falls $\forall R>0\,\exists n_0\in\mathbb{N}: x_n \ge R (x_n \le -R)\,\forall n\ge n_0$
	
	(heißt auch \highlight{bestimmt divergent}) gegen $\infty$, "`uneigentlich"' wird meist weggelassen.
	
	\textbf{Notation}: $\lim\limits_{n\rightarrow\infty} x_n = \pm \infty$ bzw. $\xi_n\rightarrow \pm \infty$
\end{definition}

\begin{example}
	$\lim\limits_{n\to\infty} \frac{n^2+1}{n+1}=+\infty$, denn für $R>0$ gilt: $\frac{n^2+1}{n+1}=\frac{n+\frac{1}{n}}{1+\frac{1}{n}}\ge \frac{n}{2}\ge R$ für $n\ge 2R$
\end{example}

\begin{proposition}[Satz von \person{Stolz}]
	Sei $\{x_n\},\{y_n\}$ Folgen in $\mathbb{R}, \{y_n\}$ sei stren monoton wachsend, $\{y_n\}\rightarrow\infty$\\
	$\Rightarrow \lim\limits_{n\rightarrow\infty} \frac{x_n}{y_n} = \lim\limits_{n\rightarrow\infty} \frac{x_{n+1} - x_n}{y_{n+1} - y_n}$, falls rechter Grenzwert existiert (endlich oder unendlich)
\end{proposition}
\begin{proof}
	Grenzwert rechts sei $g\in\real$, oBdA $y_n>0$. \\
	Sei $\varepsilon>0\Rightarrow n_0:\vert \frac{x_{n+1}-x_n}{y_{n+1}-y_n}-g\vert<\varepsilon\Rightarrow (g-\varepsilon)\cdot(y_{n+1}-y_n)\le x_{n+1}-x_n\le (g+\varepsilon)\cdot(y_{n+1}-y_n)\overset{(*)}{\Rightarrow} (g-\varepsilon)(y_m-y_{n_0})\le x_m-x_{n_0}\le (g+\varepsilon)(y_m-y_{n_0})\Rightarrow (g-\varepsilon)(1-\frac{y_{n_0}}{y_m})\le \frac{x_m}{y_m}\le (g+\varepsilon)(1-\frac{y_{n_0}}{y_m})+\frac{x_{n_0}}{y_m}\Rightarrow g-\varepsilon \le \liminf\frac{x_m}{y_m}\limsup\frac{x_m}{y_m}\le g+\varepsilon\Rightarrow\lim\limits_{m\to\infty}\frac{y_m}{x_m}=g$ \\
	$(*)\; \sum_{n=n_0}^{m-1}$
\end{proof}

\begin{example}
	$\lim\limits_{n\to\infty} \frac{n^k}{z^n}=0$ für $z\in\comp,\vert z\vert>1,k\in\natur_{>0}$ \\
	$k=1$: $\frac{n+1-n}{\vert z\vert^{n+1}-\vert z\vert^{n}}=\frac{1}{\vert z\vert}\to 0\beha$ \\
	$k>1$: $\frac{n^k}{\vert z\vert^n}=\left( \frac{n}{\sqrt[k]{\vert z\vert}}^n\right)^k\to 0^k=0\beha$
\end{example}

\begin{proposition}
	Sei $\{x_n\}$ mit $x_n\rightarrow x$ im normierten Raum $X$.\\
	$\Rightarrow\frac{1}{n}\sum_{j=1}^n x_j \overset{n\rightarrow\infty}{\longrightarrow} x$
\end{proposition}
\begin{proof}
	Es ist $\Vert\frac{1}{n}\sum_{j=1}^{n} x_j-x\Vert=\frac{1}{n}\sum_{j=1}^n x_j-x\le \frac{1}{n}\sum_{j=1}^n \Vert x_j-x\Vert =:c_n$ \\
	$\frac{\sum_{j=1}^{n+1}\Vert x_j-x\Vert-\sum_{j=1}^{n} \Vert x_j-x\Vert}{n+1-n}=\frac{\Vert x_j-x\Vert}{1}\to 0\Rightarrow c_n\to 0\beha$
\end{proof}
\section{Vollständigkeit}
\begin{*definition}[\person{Cauchy}-Folge]
	Folge $\{x_n\}$ im metrischen Raum $(X,d)$ heißt \gls{cf} (Fundamentalfolge), falls
    \[
    \forall\epsilon > 0 \,\exists n_0\in\mathbb{N}: d(x_n, x_m) < \epsilon\quad\forall n,m\ge n_0.
    \]
\end{*definition}

\begin{proposition}
	\proplbl{satzcauchyfolge}
	Sei $\{x_n\}$ Folge im metrischen Raum $(X,d)$. Dann
	\begin{enumerate}[label={\arabic*)}]
		\item $x_n\rightarrow x \Rightarrow \{x_n\}$ ist \person{Cauchy}-Folge
		\item $\{x_n\}$ \gls{cf} $\Rightarrow \{x_n\}$ ist beschränkt und hat maximal einen \gls{hw}.
	\end{enumerate}
\end{proposition}
\begin{proof}
	\begin{enumerate}
		\item Sei $\varepsilon>0\Rightarrow n_0:d(x_{n_0},x)<\frac{\varepsilon}{2}\Rightarrow d(x_{n_0},x_m)\le d(x_{n_0},x)+d(x,x_m)<\varepsilon\beha$
		\item $\exists n_0:d(x_n,x_m)<1\Rightarrow$ fast alle $x_n\in B_1(x_{n_0})\Rightarrow$ Folge beschränkt \\
		Sei $g$ HW: $\varepsilon>0\Rightarrow$ unendlich viele $x_n\in B_{\varepsilon}(g)\Rightarrow$ fast alle $x_n\in B_{\varepsilon}(g)\Rightarrow$ nur 1 HW möglich $\beha$
	\end{enumerate}
\end{proof}

\begin{*definition}[Durchmesser]
	\begriff{Durchmesser} von $M\subset X$ beschränkt, $\neq 0$, $(X,d)$ metrischer Raum ist \mathsymbol{diam}{$\diam$}$M:=\sup\{d(x,y) | x,y\in M\}$
	
	Folge $\{A_n\}$ von abgeschlossenen Mengen heißt \begriff{Schachtelung} falls $A_n\neq\emptyset, A_{n+1}\subset A_n\,\forall n\in\mathbb{N}$ und $\diam A_n\overset{n\rightarrow\infty}{\longrightarrow}0$.
\end{*definition}

\begin{lemma}
	Sei $M\subset X$ beschränkt, $\neq 0\;\Rightarrow\;\diam M = \diam (\cl M)$.
\end{lemma}
\begin{proof}
	Übungsaufgabe, Selbststudium
\end{proof}

\begin{theorem}
	\proplbl{theorem_schachtelung}
	Sei $(X,d)$ metrischer Raum. Dann: für jede Schachtelung $A_n$ in $X$ gilt:\[ \bigcap_{n\in\mathbb{N}} A_n\neq \emptyset \;\Leftrightarrow \; \text{jede \gls{cf} in $\{x_n\}$ in $X$ ist konvergent} \]
\end{theorem}
\begin{proof}
	$(\Rightarrow)$ Sei $\{x_n\}$ CF in $X$, setze $A_n:=\cl\{x_k\mid k\ge n\}\Rightarrow\diam A_n\to 0$ und $\{A_n\}$ Schachtelung $\Rightarrow\exists x\in\bigcap A_n$ \\
	$\forall\varepsilon>0\quad\exists n_0:\diam A_{n_0}<\varepsilon\Rightarrow d(x_n,x)<\varepsilon\Rightarrow x_n\to x$ \\
	$(\Leftarrow)$ Sei $\{A_n\}$ Schachtelung, wähle $x_n\in A_n\Rightarrow x_k\in A_n\; (k\ge n)\Rightarrow \{x_n\}$ ist CF $\Rightarrow x_n\to x\Rightarrow x\in A_n\beha$
\end{proof}

\begin{lemma}
	In $\mathbb{R}$ gilt:
	\begin{center}
		\begin{tabular}{lcl}
			$\bigcap_{n\in\mathbb{N}} A_n\neq \emptyset$ & $\Leftrightarrow$ & $\bigcap_{n\in\mathbb{N}} X_n\neq \emptyset$ \\[5pt]
			$\forall$ Schachtelungen $\{A_n\}$ && $\forall$ Intervallschachtelungen $\{x_n\}$
		\end{tabular}
	\end{center}
\end{lemma}
\begin{proof}
	$(\Rightarrow)$ trivial \\
	$(\Leftarrow)$ Zeige: jede CF konvergiert in $\real$, dann folgt die Behauptung aus \propref{theorem_schachtelung} \\
	Sei $\{x_n\}$ CF in $\real, M_n:=\{x_k\mid k\ge n\}\Rightarrow X_n:=[\inf M_n,\sup M_n]$ Intervallschachtelung in $\real\Rightarrow\exists x\in\bigcap X_n\Rightarrow x_n\to x\beha$
\end{proof}

\begin{*definition}[Vollständigkeit]
	Metrischer Raum $(X,d)$ heißt \begriff{Vollständig}, falls jede \person{Cauchy}-Folge $\{x_n\}$ in $X$ konvergiert.
	
	Vollständiger, normierter Raum $(X,\Vert .\Vert)$ heißt \begriff{\person{Banach}-Raum}.
\end{*definition}

\begin{conclusion}
	Sei $\{x_n\}$ Folge im vollständigen metrischen Raum $(X,d)$. Dann:\[ \{x_n\}\text{ konvergent}\;\Leftrightarrow\; \{x_n\} \text{ \person{Cauchy}-Folge} \]
\end{conclusion}
\begin{proof}
	vergleiche Definition Vollständigkeit und \propref{satzcauchyfolge}
\end{proof}

\begin{theorem}
	$\mathbb{R}^n$ und $\mathbb{C}^n$ mit $|.|_p$ ($1\le p \le \infty$) sind vollständige, normierte Räume (d.h. \person{Banach}-Räume).
\end{theorem}
\begin{proof}
	für $\real^n$: $\{x_k\}$ mit $x_k=(x^1_k,...,x^n_k)$ CF in $\real^n$ bezüglich $\vert\cdot\vert_p$, offenbar $\{x_k\}$ auch CF bezüglich $\vert\cdot\vert_\infty$ \\
	$\Rightarrow \{x^j_k\}_k$ CF in $\real$ für jedes $j=1,...,n\Rightarrow \{x^j_k\}_k$ konvergiert in $\real\quad\forall j\Rightarrow \{x_k\}$ konvergiert in $\real^n\beha$ \\
	für $\comp$: Zurückführung auf $\real^2\to$ Realteile und Imaginärteile
\end{proof}
\section{Kompaktheit}
\begin{*definition}
Sei $(X,d)$ metrischer Raum, Mengensystem $\mathcal{U}\subset \{ U\subset X | U \text{ offen }\}$ heißt \begriff{offene Überdeckung} von $M\subset X$, falls $M\subset \bigcup_{U\in\mathcal{U}} U$.

Überdeckung $\mathcal{U}$ heißt endlich, falls $\mathcal{U}$ endlich (d.h. $\mathcal{U} = \{U_1,\dotsc,U_n\}$).

Menge $M\subset X$ heißt \highlight{(überdeckungs-)}\begriff[Menge!]{kompakt}, falls jede Überdeckung $\mathcal{U}$ eine endliche Überdeckung $\tilde{\mathcal{U}}\subset \mathcal{U}$ endhält (d.h. $\exists U_1,\dotsc, U_n\subset\mathcal{U}$ mit $M\subset\bigcup_{i=1}^n U_n$).

Menge $M\subset X$ heißt \begriff{folgenkompakt}, falls jede Folge $\{x_n\}$ aus $M$ (d.h. $x_n\in M\,\forall M$) eine konvergente Teilfolge $\{x_{n'}\}$ mit Grenzwert in $M$ besitzt (d.h. $\{x_n\}$ hat \gls{hw} in $M$ nach \ref{tfprinzip}).
\end{*definition}

\begin{boldenvironment}[Warnung]
	existiert endliche offene Überdeckung $\tilde{\mathcal{U}}$ von $M\Rightarrow M$ nicht unbedingt kompakt
\end{boldenvironment}

\begin{underlinedenvironment}[Hinweis]
	Eine Abbildung $A:I\to X$ nennt man auch \begriff{Familie} mit Indexmenge $I$ und schreibt $\{A_n\}_{i\in I}$Definition von "'kompakt"' in Literatur mittels Familien ist gleichwertig.
\end{underlinedenvironment}

\begin{theorem}
	\proplbl{theorem_kompakt_folgenkompakt}
	Sei $(X,d)$ metrischer Raum, $M\subset X$. Dann:\[M\text{ kompakt} \;\Leftrightarrow\; M\text{ folgenkompakt}\]
\end{theorem}
\begin{proof}
	\begin{itemize}
		\item $(\Rightarrow)$ Sei $\{x_n\}$ Folge in $M$, angenommen $\{x_n\}$ hat keinen HW in $M$ \\
		$\Rightarrow\exists\varepsilon_x>0:$ nur endlich viele $x_n\in B_{\varepsilon_x}(x)\Rightarrow M$ kompakt $\Rightarrow$ endlich viele $B_{\varepsilon_x}(x)$ überdecken $M\Rightarrow$ nur endlich viele Glieder $x_n$ in $M\Rightarrow$ aber Folge unendlich vieler Glieder $\Rightarrow\lightning\Rightarrow \{x_n\}$ hat HW in $M\beha$\\
		$(\Leftarrow)$ betrachte für $\varepsilon>0$ fest offene Überdeckung $U_{\varepsilon}:=\{B_{\varepsilon}(x)\mid x\in M\}$ von $M$. Angenommen, es gibt keine endliche Überdeckung $U'_{\varepsilon}\subset U_{\varepsilon}$ von $M$ \\
		$\Rightarrow\exists$ Folge $\{x_n\}$ in $M:x_1\in M$ und $x_{k+1}\in M\backslash\bigcup B_{\varepsilon}(x_i)\Rightarrow d(x_k,x_l)>\varepsilon\Rightarrow \{x_k\}$ hat keinen HW $\Rightarrow M$ folgenkompakt $\Rightarrow\lightning$
		\item Sei $U$ beliebige offene Überdeckung von $M$. Angenommen, es gibt keine endliche Überdeckung $U'\subset U$ von $M$ (1) \\
		nach 2.: $\varepsilon_k:=\frac{1}{k}$ gibt es offene Überdeckung $U_k$ von $M$ mit endlich vielen $\varepsilon_k$-Kugeln $\overset{(1)}{\Rightarrow}\forall k\;\exists x_k\in M:B_k:=B_{\varepsilon_k}(x_k)\in U_k$ und es gibt keine endliche Überdeckung $U'\subset U$ von $B_k\cap M$ (2) \\
		$\Rightarrow M$ folgenkompakt $\exists$ TF $x_{k'}\Rightarrow \tilde{x}\in M\Rightarrow\exists\tilde U\in U:\tilde{x}\in\tilde{U}\Rightarrow\tilde{U}$ offen $\Rightarrow\exists\tilde{\varepsilon}>0:B_{\tilde{\varepsilon}}(\tilde{x})\subset\tilde{U}\Rightarrow\exists k_0:d(x_{k_0},\tilde{x})<\frac{\tilde{\varepsilon}}{2}$ und $\frac{1}{k}=\varepsilon_{k_0}<\frac{\tilde{\varepsilon}}{2}\Rightarrow\forall x\in B_{k_0}: d(x,\tilde{x})\le d(x,x_{k_0})+d(x_{k_0},\tilde{x})<\tilde{\varepsilon}\Rightarrow B_{k_0}\subset B_{\tilde{\varepsilon}}(\tilde{x})\subset \tilde{U}\Rightarrow \{\tilde{U}\}\subset U$ ist endliche Überdeckung von $B_{k_0}\overset{(2)}{\Rightarrow}\lightning\Rightarrow 1$ falsch $\beha$
	\end{itemize}
\end{proof}

\begin{proposition}
	\proplbl{satzfolgenkompaktbeschraenktabgeschlossen}
	Sei $(X,d)$ metrischer Raum, $M\subset X$. Dann
	\begin{enumerate}[label={\arabic*)}]
		\item $M$ folgenkompakt $\Rightarrow$ $M$ beschränkt und abgeschlossen
		\item $M$ folgenkompakt, $A\subset M$ abgeschlossen $\Rightarrow$ $A$ folgenkompakt.
	\end{enumerate}
\end{proposition}
\begin{proof}
	\begin{enumerate}
		\item angenommen $M$ unbeschränkt $\Rightarrow\exists$ unbeschränkte Folge $\{x_n\}$ in $M$ ohne HW $\Rightarrow\exists$ keine konvergente TF $\Rightarrow\lightning\Rightarrow M$ beschränkt \\
		Sei $\{x_n\}$ Folge in $M$ mit $x_n\to x\Rightarrow M$ folgenkompakt $\Rightarrow x\in M\Rightarrow M$ abgeschlossen
		\item Sei $\{x_n\}$ Folge in $A\subset X\Rightarrow M$ folgenkompakt $\Rightarrow\exists$ TF $x_{n'}\to x\in M\Rightarrow A$ abgeschlossen $\Rightarrow x\in A\beha$
	\end{enumerate}
\end{proof}

\begin{theorem}[\person{Heine}-\person{Borell} kompakt, \person{Bolzano}-\person{Weierstraß} folgenkompakt]
	\proplbl{H_B_kompakt_B_W_folgenkompakt}
	Sei $X=\mathbb{R}^n$ (bzw. $\mathbb{C}^n$) mit beliebiger Norm, $M\subset X$. Dann \[ M \text{ kompakt} \;\Leftrightarrow\; M \text{ abgeschlossen und beschränkt} \] \\
	
	\begin{boldenvironment}[Warnung]
		Theorem gilt nicht in beliebigen metrischen Räumen! Betrachte $\real$ mit diskreter Metrik: $[0,1]$ nicht folgenkomakt, da $\left\lbrace \frac{1}{n}\right\rbrace$ keine HW hat.
	\end{boldenvironment}
\end{theorem}
\begin{proof}
	$(\Rightarrow)$ Folgt aus \propref{theorem_kompakt_folgenkompakt} und \propref{satzfolgenkompaktbeschraenktabgeschlossen} \\
	$(\Leftarrow)$ für $\real^n$: Norm in $\real^n$ ist äquivalent zu $\vert\cdot\vert_\infty$ \\
	Sei $\{x_k\}$ Folge in $M, x_k=(x^1_k,...,x^n_k)\in \real^n\Rightarrow M$ beschränkt $\Rightarrow \{\vert x_n\vert_\infty\}$ beschränkt in $\real\Rightarrow \{x^j_k\}$ beschränkt in $\real$ für $j=1,...,n\Rightarrow$ \person{Bolzano-Weierstraß} in $\real$ \\
	$\Rightarrow\exists$ TF $\{x_{k'}\}:x^1_{k'}\to x^1$ \\
	$\Rightarrow\exists$ TF $\{x_{k''}\}:x^2_{k''}\to x^2$, offenbar $x^1_{k''}\to x^1$ \\
	$\vdots$ \\
	$\Rightarrow\exists$ TF $\{x_{k*}\}:x^j_{k*}\to x^j\quad\forall j=1,...,n$ \\
	$\Rightarrow x_{k*}\to x=(x^1,...,x^n)$ in $\real^n\Rightarrow M$ abgeschlossen $\Rightarrow x\in M\Rightarrow M$ kompakt
\end{proof}

\begin{conclusion}
	Sei $\{x_n\}$ Folge in $X=\mathbb{R}^n$ (bzw. $\mathbb{C}^n$). Dann \[ \{x_n\}\text{ beschränkt} \;\Rightarrow \; \{x_n\} \text{ hat konvergente \gls{tf}}\]
\end{conclusion}
\begin{proof}
	folgt direkt aus dem Beweis von \propref{H_B_kompakt_B_W_folgenkompakt}
\end{proof}

\begin{proposition}
	\proplbl{aeqv_norm}
	Je 2 Normen aus $\mathbb{R}^n$ bzw. $\mathbb{C}^n$ sind äquivalent.
\end{proposition}
\begin{proof}
	zeige, dass beliebige Norm $\Vert.\Vert$ äquivalent zu $\vert\cdot\vert_\infty$ ist \\
	Sei $\{e_1,...,e_n\}$ Standardbasis, dann für $x=(x_1,...,x_n)\in \real^n, B=\sum_{j=1}^{n} \Vert e_j\Vert>0$ gilt: $\Vert x\Vert=\Vert\sum_{j=1}^n x_j\cdot e_j\Vert\le \sum_{j=1}^n \Vert x_j\Vert\cdot\Vert e_j\Vert\le B\cdot\vert x\vert_\infty$ (3) \\
	Sei $a:=\inf\{\Vert x\Vert\mid x\in S\}$ mit $S:=\{x\in \real^n\mid \vert x\vert_\infty=1\}$, angenommen, $a=0\Rightarrow^\exists \{x_k\}$ in $S:\Vert x_k\Vert\to 0$ \\
	$S$ beschränkt und abgeschlossen $\Rightarrow\exists$ TF $x_{k'}\to\tilde{x}\in S\Rightarrow \Vert\tilde{x}\Vert\le \Vert\tilde{x}-x_k\Vert+\Vert x_{k'}\Vert\le B\vert \tilde{x}-x_{k_0}\vert_\infty + \Vert x_k\Vert\to 0\Rightarrow \tilde{x}=0$, da $\vert 0\vert_\infty=0\Rightarrow\lightning\Rightarrow a>0\Rightarrow a\cdot\vert x\vert_\infty\le \Vert x\Vert\beha$
\end{proof}
\section{Reihen}
\begin{*definition}[Partialsumme]
	Sei $X$ normierter Raum. $\{x_n\}$ Folge im normierten Raum.\\
	$s_n :=\sum_{k=1}^n x_k = x_0 + \dotsc + x_n$ heißt \begriff{Partialsumme}.
	
	Folge $\{s_n\}$ der Partialsumme heißt \highlight{(unendliche)}\begriff{Reihe} mit Gliedern $x_k$.\\
	Notation: durch Symbol $\sum_{k=0}^\infty x_k = x_0 + \dotsc = \sum_k x_k = \{s_k\}_{k\in\mathbb{N}}$
	
	Existiert der Grenzwert $s = \lim\limits_{n\rightarrow\infty} s_n$, so heißt der \begriff[Reihe!]{Summe} der Reihe.\\
	Notation: $s = \sum_{k=0}^\infty x_n$.
\end{*definition}

\begin{proposition}[\person{Cauchy}-Kriterium]
	Sei $X$ normierter Raum, $\{x_k\}$ Folge in $X$. Dann
	\begin{enumerate}[label={\arabic*)}]
		\item $\sum_k x_k$ konvergiert $\Rightarrow\;\forall \epsilon > 0\,\exists n_0: \left|\left|\sum_{k=n}^m x_k\right|\right| < \epsilon\,\forall m\ge n\ge n_0$
		\item falls $x$ vollständiger, normierter Raum, gilt auch $\Leftarrow$ oben.
	\end{enumerate}
\end{proposition}
\begin{conclusion}
	Sei $X$ normierter Raum, $\{x_n\}$ Folge in $X$. Dann:\\
	$\sum_k x_k$ konvergiert $\Rightarrow$ $x_k\overset{k\rightarrow \infty}{\longrightarrow}0$
\end{conclusion}
\begin{example}
	\begriff{geometrische Reihe} $X=\mathbb{C}, a_k:= z^k, z\in\mathbb{C}$ fest.
	
	$\sum_{k=0}^\infty z^k = \frac{1}{1-z}\,\forall z\in\mathbb{C}$ mit $|z|<1$
	$\sum_{k=0}^\infty z^k$ divergent, falls $|z|>1$
\end{example}
\begin{example}
	\begriff{harmonische Reihe} $X=\mathbb{R}, x_k := \frac{1}{k}\;(k>1)$. Reihe divergiert.
\end{example}
\stepcounter{theorem}
\begin{example}
	$X=\mathbb{R}$:\[ \sum_{k=1}^\infty \frac{1}{k^s}\;\begin{cases}
	\text{konvergiert},& \text{für }s > 1\\ \text{divergiert},& \text{für }s \le 1
	\end{cases} \]
	Summe heißt \begriff{\person{Riemann}'sche Zetafunktion}\mathsymbol{zeta}{$\zeta(s)$} (für $s > 1$). Diese ist beschränkt und konvergent.
\end{example}
\begin{proposition}
	Sei $X$ normierter Raum, $\{x_n\}, \{y_n\}$ in $X, \lambda,\mu\in K$ ($\mathbb{R}$ oder $\mathbb{C}$). Dann:\\
	$\sum_k x_k, \sum_k y_k$ konvergernt $\Rightarrow\;\sum_{k=0}^\infty \lambda x_k + \mu x_k$ konvergent gegen $\lambda\sum_k x_k + \mu \sum_k y_k$.
\end{proposition}
\begin{*definition}
	Reihe $\sum_k x_k$ heißt \begriff[Reihe!]{absolut konvergent}, falls $\sum_k \Vert x_k\Vert$ konvergiert.
\end{*definition}
\begin{proposition}
	Sei $X$ vollständiger, normierter Raum. Dann:\\
	$\sum_k x_k$ absolut konvergent $\Rightarrow\;\sum_k x_k$ konvergent
\end{proposition}
\begin{proposition}[Konvergenzkriterien für Reihen]
	Sei $X$ normierter Raum, $\{x_k\}$ in $X, k_0\in\mathbb{N}$
	\begin{enumerate}[label={\alph*)}]
		\item Sei $\{x_k\}$ Folge in $\mathbb{R}$ \hfill\begriff{Majorantenkriterium}
		\begin{enumerate}[label={\alph*)}]
			\item $\Vert x_k\Vert \le \alpha_k\,\forall k\ge k_0,\sum_k \alpha_k$ konvergent $\Rightarrow\;\sum_k \Vert x_k\Vert$ konvergent
			\item $0 \le \alpha_k \le \Vert x_k\Vert\,\forall k\ge k_0,\sum_k \alpha_k$ divergent $\Rightarrow\sum_k\Vert x_k\Vert$ divergent.
		\end{enumerate}
		\item Sei $x_k\neq 0\,\forall k\ge k_0$\hfill\begriff{Quotientenkriterium}
		\begin{enumerate}[label={\alph*)}]
			\item $\frac{\Vert x_{k+1}\Vert}{\Vert x_k\Vert} \le q < 1\,\forall k\ge k_0 \;\Rightarrow\;\sum_k \Vert x_k\Vert$ konvergiert
			\item $\frac{\Vert x_{k+1}\Vert}{\Vert x_k\Vert}\,\forall k\ge k_0\;\Rightarrow \sum_k\Vert x_k\Vert$ divergiert.
		\end{enumerate}
		\item \hfill\begriff{Wurzelkriterium}
		\begin{enumerate}[label={\alph*)}]
			\item $\sqrt[k]{\Vert x_k\Vert}\le q < 1\,\forall k\ge k_0\;\Rightarrow\;\sum_k\Vert x_k\Vert$ konvergiert
			\item $\sqrt[k]{\Vert x_k\Vert} \ge 1\,\forall k\ge k_0\;\Rightarrow\;\sum_k \Vert x_k\Vert$ divergent.
		\end{enumerate}
	\end{enumerate}
\end{proposition}
\begin{example}
	\begriff{Exponentialreihe} $\exp z := \sum_{k=0}^\infty \frac{z^k}{k!}$ absolut konvergent $\forall z\in \mathbb{C}$.
	
	\mathsymbol{e}{$e$}$:=\exp(1)$ \begriff{\person{Euler}'sche Zahl}
\end{example}
\begin{example}
	\begriff{Potenzreihe}: $\sum_{k=0}^\infty a_k(z-z_0)^k$ für $z\in\mathbb{C}, a_k\in\mathbb{C}, z_0\in\mathbb{C}$.
	
	Sei \[L:=\begin{cases} \limsup\limits_{n\rightarrow\infty} \sqrt[k]{|a_k|},&\text{falls existiert}\\ \infty,&\text{sonst}\end{cases}\qquad R:=\frac{1}{L} \;(\text{mit }0 = \frac{1}{\infty}, \frac{1}{0} = \infty)\]
	
	$ |z - z_0| < R$: absolute Konvergenz,\\
	$|z-z_0| > R$: Divergenz,\\
	$|z-z_0| = R$: i.A. keine Aussage möglich.
	
	$B_R(z_0)$ heißt \begriff{Konvergenzkreis}, $R$ \begriff{Konvergenzradius}
\end{example}
\begin{example}
	\begriff{$p$-adische Brüche}. Sei $p\in\mathbb{N}_{\ge 2}$: betrachte $0,x_1x_2x_3\dotsc :=\sum_{k=1}^\infty x_k\cdot p^{-k}$ für $x_k\in\{0,1,\dotsc,p-1\}\,\forall k\in\mathbb{N}$.
\end{example}
\begin{proposition}[\person{Leibnitz}-Kriterium für alternierende Reihen in $\mathbb{R}$]
	Sei $\{x_n\}$ monoton fallende Nullfolge in $\mathbb{R}$. Dann:\\
	alternierende Reihe $\sum_{k=0}^\infty (-1)^k x_k = x_0 - x_1 + x_2 - \dotsc$ ist konvergent.
\end{proposition}
\stepcounter{theorem}
\begin{*definition}[Umordnung]
	Sei $\beta:\mathbb{N}\rightarrow\mathbb{N}$ bijektive Abbildung: $\sum_{k=0}^\infty x_{\beta(k)}$ heißt \begriff{Umordnung} der Reihe $\sum_k x_k$.
\end{*definition}
\begin{proposition}
	Sei $X$ normierter Raum. Dann:\\
	$\sum_{k=0}^\infty x_k = x$ absolut konvergent $\Rightarrow\;\sum_{k=0}\infty x_{\beta(k)}$ absolut konvergent für jede Umordnung.
\end{proposition}
\begin{proposition}
	Sei $\sum_{k=0}^\infty x_k$ konvergierende Reihe in $\mathbb{R}$, die nicht absolut konvergent ist. Dann:\\
	$\forall s\in\mathbb{R}\cup \{\pm\infty\}$ existiert $\beta:\mathbb{N}\rightarrow\mathbb{N}$ bijektiv mit $s=\sum_{k=0}^\infty x_{\beta_k}$
\end{proposition}
\begin{proposition}[\person{Cauchy}-Produkt]
	Sei $X$ normierter Raum über $\mathbb{K}$, $\sum_j x_j$ und $\sum_i \lambda_i$ absolut konvergent in $X$ bzw. $\mathbb{K}$. $\beta:\mathbb{N}\times \mathbb{N}\rightarrow \mathbb{N}$ bijektiv, $Y_{\beta(i,j)} = \lambda_i x_i\,\forall i,j\in\mathbb{N}$
	
	$\Rightarrow \sum_{l=0}^\infty Y_l = \sum_{i=0}^\infty \lambda_i \sum_{j=0}^\infty x_j$, wobei linke Reihe absolut konvergiert in $X$.
	
	\begin{tabular}{ll}
		\highlight{Spezialfall:} & $\beta(i,j) = \frac{(i+j)(i+j+1)}{2} + i$ liefert\\[5pt]
		& $\sum_{k=0}^\infty \sum_{l=0}^k \lambda_k x_{k-l} = \sum_{i=0}^\infty \lambda_i \sum_{j=0}^\infty x_j$
	\end{tabular}
\end{proposition}
\stepcounter{theorem}
\begin{proposition}[Doppelreihenproposition]
	Sei $\{x_{k,l}\}_{k,l\in\mathbb{N}}$ Doppelfolge im \person{Banach}-Raum $X$ und mögen $\sum_{l=0}^\infty \Vert x_{k,l}\Vert =:\alpha_k\,\forall k$ und $\sum_{k=0}^\infty x_k =: \alpha$ existieren.
	
	$\Rightarrow \sum_{k=0}^\infty \left(\sum_{l=0}^\infty x_{k,l}\right) = \sum_{l=0}^{\infty}\left( \sum_{k=0}^\infty x_{k,l}\right)$, wobei alle Reihen absolut konvergent sind.
\end{proposition}

\chapter{Funktionen und Stetigkeit}
\addtocounter{section}{12}
\section{Funktionen}
\begin{*definition}
	$f:\mathbb{R}\to \mathbb{R}$ \begriff{monoton}\begriff[monoton!]{falled}/\begriff[monoton!]{wachsend}, falls $x < y, x,y\in M \,\Rightarrow \,f(x) \le f(y)$ bzw. $f(x) \ge f(y)$
	
	Falls rechts stets $<$ bzw. $>$, sagt man auch \begriff[monoton!]{streng} monoton.
\end{*definition}

\begin{proposition}
	Sei $f:\mathbb{R}\rightarrow \mathbb{R}$ streng monoton fallend / wachsend.\\
	$\Rightarrow$ inverse Funktion $f^{-1}:\mathcal{R}\rightarrow M$ existiert und ist streng monoton fallend / wachsend.
\end{proposition}
\begin{example}
	\begriff{Allgemeine Potenzfunktion} in $\mathbb{R}$:\\
	$f:\mathbb{R}_{>0} \to \mathbb{R}$ mit $f(x) = x^r$ für $r\in\mathbb{R}$ fest.
	
	\begin{itemize}
		\item $r > 0:$ Satz \ref{proposition_potenz_r} $\Rightarrow$ $f$ streng monoton wachsend
		\item $r < 0$: $x^r = \frac{1}{x^{-r}}$ $\Rightarrow$ $f$ streng monoton fallend
	\end{itemize}
	$\overset{\text{Satz 1}}{\Rightarrow}$ $f^{-1}$ existiert für $r\neq 0$ auf $(0,\infty)$, wegen $ y = (y^{\frac{1}{r}})^r$ ist $f^{-1}(y) = y^{\frac{1}{r}}$
\end{example}
\begin{example}
	\begriff{Allgemeine Exponentialfunktion} in $\mathbb{R}$:\\
	$f:\mathbb{R}\rightarrow\mathbb{R}$ mit $f(x) = a^x$ für $a\in\mathbb{R}_{>0}$ fest.
	
	\ref{proposition_potenz_r} $\Rightarrow$ streng monoton wachsend für $a > 1$ bzw. fallend für $a < 1$ (benutze $\frac{1}{a} > 1$)\\
	$\overset{\text{Satz 1}}{\Rightarrow}$ $f^{-1}$ existiert auf $(0,\infty)$ für $a \neq 1$. Wegen $y = a^{\log_a y}$ (\ref{proposition_logarithmus_r}) ist $f^{-1} (y) = \log_a y$.
\end{example}
\begin{example}
	\begriff{Polynom} in $\mathbb{C}$:\\
	Abbidlung $f:\mathbb{C}\rightarrow\mathbb{C}$ heißt \highlight{Polynom}, falls $f(z) = a_n z^n + \dotsc + a_1 z + a_0$ für $a_0,\dotsc, a_n\in\mathbb{C}$ fest.
	\begin{itemize}
		\item \mathsymbol{grad}{$grad$}$f = n$ falls $a_n\neq 0$
		\item $f$ ist \begriff{Nullpolynom}, falls $f(z) = 0\,\forall z\in\mathbb{C}$
		
		Notation: $f=0$
		
		(Menge der Polynome in $\mathbb{C}$ ist ein Vektorraum über $\mathbb{C}$)
	\end{itemize}
\end{example}
\begin{proposition}\label{Polynomdiv}
	Seien $f,g$ Polynome mit $f(z) = \sum_{k=0}^n a_k z^k, g(z) = \sum_{k=0}^m a_k z^k$. Dann:
	\begin{enumerate}[label={\arabic*)}]
		\item $f,g\neq 0$, $\grad f\ge \grad g$\\
		$\Rightarrow$ existieren eindeutig bestimmte Polynome $q,r$ mit $f = q\cdot g + r$, wobei $r\neq 0$ oder $\grad r < \grad g$
		\item $z_0\in\mathbb{C}$ Nullstelle von $f\neq 0$ $\Leftrightarrow$ $f(z) = (z - z_0)q(z)$ für ein Plynom $q\neq 0$ mit $\grad q = \grad f -1$
		\item $f$ hat höchstens $\grad f$ Nullstellen falls $f\neq 0$
		\item $f(z_i) = g(z_j)$ für $n+1$ paarweise verschiedene Punkte $z_0, \dotsc, z_n\in\mathbb{C}, n = \grad f \ge \grad g$\\
		$\Rightarrow$ $f(z) = g(z) \,\forall z\in\mathbb{C}$ (d.hz. $a_k = b_k\,\forall k$)
	\end{enumerate}
\end{proposition}
\stepcounter{theorem}
\begin{*definition}
	Abbildung $f:X\rightarrow Y, Y$ metrischer Raum heißt \begriff{beschränkt}[!Funktion] auf $M\subset X$ , falls Menge $f(M)$ beschränkt in $Y$ ist, sonst unbeschränkt.
\end{*definition}
\begin{*definition}
	$f:X\to Y$ heißt \begriff{konstante Funktion}, falls $f(x) = a\,\forall x\in X$ und $a\in Y$ fest.
\end{*definition}
\begin{*definition}
	$M\subset X, X$ normierter Raum heißt \begriff{konvex}, falls $x,y\in M \,\Rightarrow \,tx+(1-t)y \in M\,\forall t\in(0,1)$
	
	$f:D\subset X\to \mathbb{R}$ heißt \begriff[konvex!]{strikt}\begriff{konvex}, falls $f(tx + (1-t)y) \underset{(<)}{\le} t f(x) + (1-t)f(y)\forall x,y\in D, t\in(0,1)$
	
	$f$ heißt \begriff{konkav} (bzw. \begriff[konkav!]{strikt}), falls $-f$ (strikt) konvex.
\end{*definition}
\stepcounter{theorem}

\subsection*{Lineare Funktionen} \proplbl{defLinearFunction}
\begin{*definition}
	Seien $X,Y$ normierte Räume über $K$.\\
	$f: X\rightarrow Y$ heißt \begriff[Abbildung!]{linear}, falls
	\begin{itemize}
		\item $f$ \begriff[Abbildung!linear!]{additiv}, d.h. $f(a+b) = f(a) + f(b) \,\forall a,b\in X$ und
		\item $f$ \begriff[Abbildung!linear!]{homogen}, d.h. $f(\lambda a) = \lambda f(a)\,\forall a\in X,\lambda\in K$
	\end{itemize}

	$f:X\to Y$ heißt \begriff[Abbildung!linear!]{affin}\highlight{linear}, falls $f+f_0$ linear für eine konstante Funktion $f_0$
	
	Offenbar $f$ linear $\Rightarrow\;f(0) = 0$
\end{*definition}
\stepcounter{theorem}
\begin{*definition}
	Lineare Abbildung $f:X\to Y$ heißt \begriff{beschränkt}[!lineare Funktion], falls $f$ beschränkt auf $\overline{B_1(0)}$, d.h. \begin{align}
		\tag{1}\exists\text{ konstante }c > 0: \Vert f(x)\Vert \le c\,\forall x: \Vert x\Vert \le 1
	\end{align}
	Wegen $\Vert f\left( \frac{x}{\Vert x \Vert}\right) = \frac{1}{\Vert x \Vert} \Vert f(x) \Vert$ ist (1) äquivalent zu
	\begin{align}
		\tag{1'} \Vert f(x) \Vert = \sup \{ \Vert f(x) \Vert | x \in \overline{B_1(0)}\}
	\end{align}
\end{*definition}
\begin{proposition}
	Seien $X,Y$ normierte Räume über $K$, dann:\\
	\mathsymbol{L}{$L$}$(X,Y):= \{ f:X\to Y \,|\, f \text{ linear und beschränkt} \}$ ist normierter Raum über $K$ mit $\Vert f \Vert = \sup \{ \Vert f(x) \Vert | x\in \overline{B_1(0)} \}$
\end{proposition}

\subsection*{Exponentialfunktion}
\begin{*definition}
	$\exp:\mathbb{C}\to \mathbb{C}$ mit $\exp(z) = \sum_{k=0}^\infty \frac{z^k}{k!}$
\end{*definition}

\begin{proposition}
	Sei $\{z_n\}$ Folge in $\mathbb{C}$ mit $z_n\to z$. Dann: $\lim\limits_{n\rightarrow\infty} \left( 1 + \frac{z_n}{n}\right)^n = \exp (z)$
\end{proposition}

\begin{lemma}
	\proplbl{lemma_13_10}
	Sei $z_n\to 0$ in $\mathbb{C}\;\Rightarrow\; \lim \frac{\exp(z_n) - 1}{z^n} = 1$
\end{lemma}

\begin{proposition}
	Sei $f:\mathbb{C}\rightarrow\mathbb{C}$ mit $f(z_1 + z_2) = f(z_1) \cdot f(z_2) \,\forall z_1, z_2\in\mathbb{C}$ und $\lim\limits_{n\rightarrow\infty} \dfrac{f\left( \frac{z}{n}\right) - 1}{\frac{z}{n}} = \gamma\in\mathbb{C}\,\forall z\in\mathbb{C}$ \\
	$\Rightarrow \;f(z) = \exp(\gamma z)\,\forall z\in\mathbb{C}$
\end{proposition}

\begin{conclusion}
	Funktion $\exp$ ist durch obiges Lemma und Satz eindeutig definiert.
\end{conclusion}
\begin{proposition}
	Es gilt: $e^x = \exp (x) \,\forall x\in \mathbb{R}$
	
	Definiert (!) in $\mathbb{C}:\; e^z := \exp(z) \,\forall z\in\mathbb{C}$ (als Potenz nicht erklärt)
\end{proposition}

\begin{*definition}
	\begriff{natürlicher Logarithmus}: $\ln x = \log_e x\,\forall x\in\mathbb{R}_{>0}$
	
	\begriff{Trigonometrische Funktion}:
	\begin{itemize}
		\item $\sin z := \frac{e^{iz} - e^{-iz}}{2i} = \sum_{k=0}^\infty (-1)^k \frac{z^{2k+1}}{(2k+1)!} = z - \frac{z^3}{3!} + \frac{z^5}{5!}+ \dotsc \,\forall z\in\mathbb{C}$
		\item $\cos z := \frac{e^{iz}+e^{-iz}}{2} = \sum_{k=0}^\infty (-1)^k \frac{z^{2k}}{(2k)!} = 1 - \frac{z^2}{4} + \frac{z^4}{24}+\dotsc \,\forall z\in\mathbb{C}$
	\end{itemize}
\end{*definition}

\begin{proposition}
	\proplbl{additionstheoreme}
	Es gilt:
	\begin{enumerate}[label={\arabic*)}]
		\item \begriff{\person{Euler}'sche Formel}: $e^{iz} = \cos z + i \sin z$
		\item $\sin^2 z + \cos^2 z = 1\,\forall z\in\mathbb{C}$ (beachte: $\cancel{\rightarrow}\;|\sin z|\le1, |\cos z| \le 1$, $\sin, \cos$ unbeschränkt auf $\mathbb{C}$)
		\item $\sin(-z) = -\sin z, \cos z = \cos(-z)$
		\item (\begriff{Additionstheoreme})
		\begin{itemize}
			\item $\sin(z+w) = \sin z \cos w + \sin w \cos z \,\forall z,w\in\mathbb{C}$
			\item $\cos (z+w) = \cos z \cos w - \sin z \sin w \,\forall z,w\in\mathbb{C}$
		\end{itemize}
		\item $\sin(2z) = 2\sin z \cos z, \cos(2z) = \cos^2 z - \sin^2 z\,\forall z\in\mathbb{C}$
		\item $\sin z - \sin w = 2\cos \frac{z+w}{2} - \sin \frac{z+w}{2}$\\
			  $\cos z - \cos w = -2\sin\frac{z+2}{2}\sin\frac{z-w}{2}$
	\end{enumerate}
\end{proposition}

\begin{proposition}
	Es gilt $\forall x\in \mathbb{R}:$\\
	$\,\left| e^{ix}\right| = 1, \sin x = \Im e^{ix}, \cos = \Re e^{ix}$ (insbesondere $\sin x,\cos x \in\mathbb{R}$), somit $e^{ix} = \cos x + i \sin x$
\end{proposition}

\begin{lemma}
	Es gilt in $\mathbb{R}$:
	\begin{enumerate}[label={\arabic*)}]
		\item $\cos$ streng fallend auf $[0,2]$
		\item $\cos 2 < 0$ und $\sin x > 0\,\forall x\in (0,2]$
		\item $\phi(x) = \phi(1) \,\forall x\in [0,2]$ und $45 < \phi(x) < 90$ (d.h. $\phi(x)$ proportional zu $x$)
		\item $\cos \frac{\pi}{2} = 0$ für $\pi := \frac{180°}{\phi(1)}$ ($=3,1415\dotsc$), $\frac{\pi}{2}$ einzige Nulsltelle in $[0,2]$
	\end{enumerate}
\end{lemma}
\stepcounter{theorem}
\begin{proposition}
	Für alle $z\in\mathbb{C}, k\in\mathbb{Z}$ gilt:
	\begin{enumerate}[label={\arabic*)}]
		\item $e^{z+2k\pi i} = e^z$, d.h. Periode $2\pi i$\\
		$\sin(z+2k\pi) = \sin z$ (d.h. Periode $2\pi$)\\
		$\cos(z+2k\pi) = \cos z$ (d.h. Periode $2\pi$)
		\item $e^{z+i\sfrac{\pi}{2}} = ie^z, e^{z+i\pi} = -e^z$
		\item $\sin(z+\pi) = -\sin z, \cos(z+\pi) = -\cos z$\\
		$\sin\left(z+\frac{\pi}{2}\right) = \cos z, \cos\left(z+\frac{\pi}{2}\right) = -\sin z$
	\end{enumerate}
\end{proposition}

\begin{proposition}
	Auf $\mathbb{C}$ gilt:
	\begin{itemize}
		\item $e^z = 1 \,\Leftrightarrow\,z=2k\pi i,\;k\in\mathbb{Z}$
		\item $\sin z = 0\,\Leftrightarrow\,z=k\pi,\;k\in\mathbb{Z}$
		\item $\cos z = 0\,\Leftrightarrow\,z =k\pi + \frac{\pi}{2},\;k\in\mathbb{Z}$
	\end{itemize}
\end{proposition}
\subsection*{$\sin$ / $\cos$ in $\mathbb{R}$}
\begin{centering}
	\begin{tabular}{c|ccccc}
		\toprule
		$x$ & 0 & $\frac{\pi}{6}$ & $\frac{\pi}{4}$ & $\frac{\pi}{3}$ & $\frac{\pi}{2}$ \\
		\midrule
		$\sin x$ & $0$ & $\frac{1}{2}$ & $\frac{\sqrt{2}}{2}$ & $\frac{\sqrt{3}}{2}$ & $1$ \\
		$\cos x$ & $1$ & $\frac{\sqrt{3}}{2}$ & $\frac{\sqrt{2}}{2}$ & $\frac{1}{2}$ & $0$ \\
		\bottomrule
	\end{tabular}
\end{centering}

\begin{*definition}
	$\sin\left[ -\frac{\pi}{2},\frac{\pi}{2}\right]\to [-1,1]$ streng monoton und surjektiv,\\
	$\cos[0,\pi]\to[-1,1]$ streng monoton und surjektiv\\
	$\Rightarrow$ Umkehrfunktion existiert: \begriff{Arcussinus}, \begriff{Arcuscosinus}:
	\begin{itemize}
		\item $\arcsin := \sin^{-1}: [-1,1]\to\left[-\frac{\pi}{2},\frac{\pi}{2}\right]$
		\item $\arccos := \cos^{-1}: [-1,1]\to [0,\pi]$
	\end{itemize}
\end{*definition}

\subsection*{Tangens und Cotangents}
\begin{*definition}
	$\tan z z := \frac{\sin z}{\cos z}\,\forall z\in\mathbb{C}\setminus\{ \left.\frac{\pi}{2} + k\pi \right| k\in\mathbb{Z}\}$\\
	$\cot z := \frac{\cos z}{\sin z}\,\forall z\in\mathbb{C}\setminus \{ k\pi | k\in\mathbb{Z}\}$
	
	$\left.\begin{aligned}
		\text{Offenbar }\tan (z+\pi) &= \frac{\sin (z+\pi)}{\cos(z+\pi)} = \frac{-\sin z}{-\cos z} = \tan z\\
		\cot(z+\pi) &= \cot (z)
	\end{aligned}\right\rbrace
	\begin{gathered}
		\forall z\in\mathbb{C}, \text{ d.h. Periode $\pi$}
	\end{gathered}
	$
\end{*definition}

\subsection*{Tangens auf $\mathbb{R}$}
\begin{*definition}
	$0 \le x_1 < x_2 < \sfrac{\pi}{2} \,\Rightarrow\,\tan x_1 = \frac{\sin x_1}{\cos x_1} < \frac{\sin x_2}{\cos x_2} = \tan x_2$ \\
	$\Rightarrow\,\tan (-x) = - \tan(x) $ $\Rightarrow$ streng wachsend auf $\left( \frac{\pi}{2},\frac{\pi}{2}\right)$ \\
	$\Rightarrow\,\arctan = \tan^{-1}: \mathbb{R}\to \left(-\frac{\pi}{2},\frac{\pi}{2}\right)$ existiert.
\end{*definition}
\begin{proposition}
	Es gilt:
	\begin{enumerate}[label={\arabic*)}]
		\item $\Re(exp) = \mathbb{C}\setminus\{0\}$
		\item (\begriff{Polarkoordinaten} auf $\mathbb{C}$)
		
		Für $z\in\mathbb{C}\setminus\{0\}$ existiert eindeutiges $\gamma\in[0,2\pi] mit z = |z|e^{i\gamma} = |z|\left( \cos \gamma + i\sin \gamma\right)$ (auch $[-\pi,\pi]$)
		\item (Wurzeln)
		
		Für $Z=|z|e^{i\gamma}\in\mathbb{C}\setminus\{0\}, n\ge 2$ gilt:\\
		$w^n = z \,\Leftrightarrow\, w\in\left\{ \left. \sqrt[n]{z} e^{i \frac{k}{n} + \frac{2k\pi}{n}} =: w_k \right| k=1,\dotsc,n\right\}$ (Lösungen bilden ein regelmäßiges $N$-Eck auf dem Kreis mit dem Radius $\sqrt[n]{|z|}$)
	\end{enumerate}
\end{proposition}

\subsection*{Logarithmen in $\mathbb{C}$} (sog. Hauptzweig)
\begin{*definition}
	$exp\left( \{ z\in\mathbb{C}\,|\, \Im z < \pi \}\right) \to \mathbb{C}\setminus (\infty, 0]$ ist bijektiv \\
	$\Rightarrow$ Umkehrabbildung $\ln:\mathbb{C}\setminus(-\infty,0]$ gilt: $e^{\ln |z| + i\gamma} = |z|e^{i\gamma} = z$\\
	$\Rightarrow\,\ln z = \ln |z| + i\gamma \,\forall z=|z|e^{i\gamma}\in\mathbb{C}\setminus(-\infty,0)$\\
	$\Rightarrow \,\ln z$ stimmt auf $\mathbb{R}_{>0}$ mit rellen $\ln$ überein.
\end{*definition}

\subsection*{Hyperbolische Funktionen}
\begin{*definition}
	\begin{itemize}
		\item $\sinh (z) = \frac{e^z - e^{-z}}{2} = \sum_{k=0}^\infty \frac{z^{2k+1}}{(2k+1)!}\,\forall z\in\mathbb{C}$ (\begriff{Sinus Hyperbolicus})
		\item $\cosh (z) = \frac{e^z+e^{-z}}{2} = \sum_{k=0}^\infty \frac{z^{2k}}{(2k+1)!}\,\forall z\in\mathbb{C}$ (\begriff{Cosinus Hyperbolicus})
		\item $\tanh (z) = \frac{\sinh (z)}{\cosh (z)}\,\forall z\in\mathbb{C}\setminus\left\lbrace \left.\frac{\pi}{2} + k\pi  \right| k\in\mathbb{Z} \right\rbrace$ (\begriff{Tangens Hyperbolicus})
		\item $\coth(z) = \frac{\cosh(z)}{\sinh(z)} \,\forall z\in\mathbb{C}\setminus \{ k\pi | k\in\mathbb{Z}\}$ (\begriff{Cotangens Hyperbolicus})
	\end{itemize}
\end{*definition}

\begin{proposition}
	Es gilt $\forall z,w\in\mathbb{C}$
	\begin{enumerate}[label={\arabic*)}]
		\item $\sin h = -i\sin(z), \cos (z) = \cosh(iz), \sinh(-z) = -\sinh(z), \cosh(-z) = \cosh(x)$ (gibt auch Nullstellen vom $\sinh / \cosh$)
		\item $\sinh, \cosh$ haben Periode $2\pi i$, $\tanh, \coth$ haben Periode $\pi i$
		\item $\cosh^2 z - \sin^2 z = 1$
		\item $\sinh(z+w) = \sinh z \cosh w + \sinh w \cosh z$\\
		$\cosh (z+w) = \cosh z \cosh w + \sinh z \sin w$
	\end{enumerate}
\end{proposition}
\rule{4cm}{0.4pt}
\begin{*definition}
	Sei $f_n X\to Y$, $Y$ metrischer Raum ($X$ beliebige Menge), $n\in\mathbb{N}$. $\{f_n\}_{n\in\mathbb{N}}$ heißt \begriff{Funktionenfolge}.
	
	Funktionenfolge $\{f_n\}$ konvergiert \begriff[Konvergenz!]{punktweise} gegen $f:X\to Y$ auf $M\subset X$, falls $f_n(x) \overset{n\rightarrow\infty}{\longrightarrow} f(x) \,\forall x\in M$
	
	Funktionenfolge $\{f_n\}$ konvergiert \begriff[Konvergenz!]{gleichmäßig} gegen $f:X\to Y$ auf $M\subset X$, falls \[ \forall \epsilon > 0 \,\exists n_0\in\mathbb{N}: d(f_n(x), f(x)) < \epsilon\quad \forall n\ge n_0\,\forall x\in M \]
	Notation: \mathsymbol*{->}{$\rightrightarrows$} $f_n(x) \overset{n\rightarrow\infty}{\rightrightarrows} f(x)$ bzw. $f_n\overset{n\rightarrow\infty}{\longrightarrow}f$ gleichmäßig auf $M$.
\end{*definition}

\begin{lemma}
	$f_n\to f$ gleichmäßig auf $M$ $\Rightarrow$ $f_n(x)\to f(x)\,\forall x\in M$ (d.h. punktweise auf $M$)
\end{lemma}

\begin{proposition}
	Seien $f_n, f\in B(X,Y)$. Dann ($X$ metrischer Raum):
	\begin{center}
		$f_n \to f$ gleichmäßig auf $X$ $\Leftrightarrow$ $f_n \to f$ in $(B(X,Y),\Vert.\Vert_1\infty)$
	\end{center}
\end{proposition}

\begin{*definition}
	Sei $f_n.:X\to Y$, $Y$ normierter Raum ($X$ beliebige Menge), $n\in\mathbb{N}$: $\sum_{n=0}^\infty f_n$ heißt \begriff{Funktionenreihe}
	
	Reihe $\sum_n f_n$ heißt \begriff[Konvergenz!]{punktweise}[!Funktionenreihe] (\begriff[Konvergenz!]{gleichmäßig}[!Funktionenreihe]) konvergent gegen $f:X\to Y$ auf $M\subset X$, falls dies für die zugehörige Folge (Partialsumme!) $\{s_n\}$ gilt.
\end{*definition}

\begin{proposition}
	Sei $\sum_{k=0}^\infty a_k(z-z_0)^k$ Potenzreihe in $\mathbb{C}$ mit Konvergenzradius $R\in(0,\infty]$ und sei $M\subset B_R(z_0)$ kompakt\\
	$\Rightarrow$ Potenzreihe konvergiert gleichmäßig auf $M$.
\end{proposition}

\section{Stetigkeit}
\begin{definition}
	Sei stets $f:D\subset X\rightarrow Y$, $X,Y$ metrischer Raum, $D=\mathcal{D}(f)\neq \emptyset, y_0\in Y$ heißt \begriff{Grenzwert}[!Funktion] der Funktion $f$ im Punkt $x_0\in \overline{D}$, falls gilt:
	\begin{center}
		$\{x_n\}$ Folge in $D$ mit $x_n\to x_0$ $\Rightarrow$ $f(x_n)\to y_0$
	\end{center}
	Notaton: $\lim\limits_{x\rightarrow x_0} = y_0, f(x)\overset{x\to x_0}{\longrightarrow } y_0$
\end{definition}
\stepcounter{theorem}
\begin{remark}
	Falls $x_0\in D$ \begriff{isolierter Punkt} von $D$, d.h. kein \gls{hp} von $D$, dann ist stets $\lim\limits_{x\rightarrow x_0} f(x) = f(x_0)$.
\end{remark}

\begin{proposition}[$\epsilon\delta$-Kriterium]
	Sei $f:D\subset X\to Y, x_0\in\overline{D}$. Dann
	\begin{center}
	$\lim\limits_{x\rightarrow x_0} f(x) = y_0 \;\Leftrightarrow \; \forall\epsilon > 0\,\exists \delta > 0: f(B_\delta(x_0)\cap D)\subset B_\epsilon(y_0)$
	\end{center}
\end{proposition}

\begin{proposition}[Rechenregeln] \label{proposition:rechenregel_stetigkeit}
	\begin{enumerate}[label={\arabic*)}]
		\item Sei $Y$ normierter Raum über $\mathbb{R}, f,g:D\subset X\to Y,\lambda: D\to K, x_0\in\overline{D}, f(x)\overset{x\to x_0}{\longrightarrow} y, g(x) \overset{x\to x_0}{\longrightarrow} \tilde{y}, \lambda(x)\overset{x\to x_0}{\longrightarrow} \alpha$. Dann:
		\begin{itemize}
			\item $(f+g)(x) \overset{x\to x_0}{\longrightarrow} y+\tilde{y}$
			\item $(\lambda \cdot f)(x) \overset{x\to x_0}{\longrightarrow} \alpha\cdot y$
			\item $\left(\frac{1}{\lambda}\right)(x) \overset{x\to x_0}{\longrightarrow} \frac{1}{\alpha}$ falls $\alpha\neq 0$
		\end{itemize}
		\item Sei $f: D\subset X\to Y, g:\tilde{D}\subset Y\to Z, \Re(f)\subset\tilde{D}, X,Y,Z$ metrische Räume, $x\in\overline{D}, f(x)\overset{x\to x_0}{\longrightarrow}y, g(y)\overset{y\to y_0}{\longrightarrow} z_0$. Dann:\\
		$g(f(x)) \overset{x\to x_0}{\longrightarrow} z_0$
	\end{enumerate}
\end{proposition}

\begin{definition}
	\proplbl{einseitige_grenzwerte}
	Für $f:D\subset X\to Y$ mit $X=\mathbb{R}$ definieren wir einen \begriff{einseitiger Grenzwert} $y_0\in Y$ heißt \begriff[einseitiger Grenzwert!]{linksseitig} bzw. \begriff[einseitiger Grenzwert!]{rechtsseitig} von $f$ im \gls{hp} $x_0$ von $D\cap(-\infty, x_0)$ bzw. $D\cap(x_0,\infty)$, falls gilt: $x_n\in D\cap(-\infty, x_0)$ bzw. $x_n\in D\cap (x_0,\infty)$ mit $x_n\to x_0\,\Rightarrow \,f(x_n)\to y_0$
	
	$\begin{aligned}
		\text{Notation: } \lim\limits_{x\uparrow x_0} f(x) &= y_0 =: f(x_0^-)& f(x)&\overset{x\uparrow x_0}{\longrightarrow} y_0 \\
		\lim\limits_{x\downarrow x_0}f(x) &= y_0 =:f(x_0^+) & f(x) &\overset{x\downarrow x_0}{\longrightarrow} y_0
	\end{aligned}$
\end{definition}

\begin{remark}
	Satz \ref{proposition:rechenregel_stetigkeit} gilt sinngemäß auch für einseitige Grenzwerte.
	
	Für $f:D\subset X\to Y$ mit $X=\mathbb{R}$ bzw. $Y=\mathbb{R}$ heißt der Grenzwert \begriff[Grenzwert!]{uneigentlich}\begriff*[Konvergenz!]{uneigentlich}[!Funktion]: \[\lim\limits_{x\to \pm \infty} f(x) = y_0, \lim\limits_{x\rightarrow x_0} f(x) = \pm \infty, \lim\limits_{x\to \pm \infty} f(x) = \pm \infty,\] indem wir einen Grenzwert definiert als $x_0=\pm \infty$ bzw. $y_0=\pm\infty$ wählen und bestimmte divergenzte Folgen $x_n\to \pm \infty$ mit $x_n\in D$) bzw. $f(x_n)\to \pm \infty$ betrachten.
\end{remark}
\stepcounter{theorem}
\subsection*{Landau-Symbole} (Vgl. von "`Konvergenzgeschwindigkeiten"')
\begin{definition}
	Sei $f:D\subset X\to Y, X$ metrischer Raum, $Y$ normierter Raum, $g:D\subset X\to \mathbb{R}$, $x_0\in \overline{D}$.
	\begin{itemize}
		\item $f(x)$ ist "`\begriff{klein o}"' von $g(x)$ für $x\to x_0$, falls \[ \lim\limits_{\stackrel{x\to x_0}{x\neq x_0}} \frac{\Vert f(x)\Vert}{g(x)} = 0 \]
		Notation: $f(x) = o(g(x))$\mathsymbol*{o}{$o$} (meist $x\neq x_0$ im "`$\lim$"' weggelassen)
		\item $f(x)$ ist "`\begriff{groß O}"' von $g(x)$ für $x\to x_0$, falls \[ \exists \delta > 0, c \ge 0: \frac{\Vert f(x)\Vert}{|g(x)|} \le c \quad \forall x\in (B_\delta(x_0) \setminus \{x_0\})\cap D \]
		Notation: $f(x) = \mathcal{O}(g(x))$\mathsymbol*{O}{$\mathcal{O}$} für $x\to x_0$
	\end{itemize}
\end{definition}
\stepcounter{theorem}
\stepcounter{theorem}
\subsection*{Relativtopologie}
\begin{definition}
	Sei $(X,d)$ metrischer Raum, für $D\subset X$ ist $(D,d)$ ein metrischer Raum mit der induzierten Metrik.
	\begin{itemize}
		\item $M\subset D$ heißt \begriff[Relativtopologie!]{offen} bzw. \begriff[Relativtopologie!]{abgeschlossen} \highlight{relativ zu $D$}, falls $M$ offen bzw. abgeschlossen im metrischen Raum $(D,d)$.
		\item $M\subset D$ heißt \begriff[Relativtopologie!]{Umgebung} von $x\in D$ relativ zu $D$, falls $M$ Umgebung von $x$ im metrischen Raum $(D,d)$.
	\end{itemize}
\end{definition}
\stepcounter{theorem}
\rule{4cm}{0.4pt}
\begin{definition}
	Sei $f:D\subset X\to Y$ metrischer Raum, $D=\mathcal{D}(f)$, Fkt. $f$ heißt \begriff{folgenstetig} im Punkt $x_0\in D$, falls \[ f(x_n)\to f(x_0) \forall \text{ Folgen $x_n\to x_0$ in $D$} \]
\end{definition}
\stepcounter{theorem}
\begin{definition}
	Funktion $f$ heißt \begriff{stetig} im Punkt $x_0\in D$, falls $\forall $ Umgebungen $V$ von $f(x_0)\,\exists $ Umgebung $U$ von $x_0$ in $D:\,f(U)\subset V$.
	
	\begin{tabularx}{\textwidth}{lX}
		\noindent\highlight{Interpretation:} & Input / Output Steuerung besteht Forderung, dass beliebig kleine Output-Toleranzen $\epsilon$ stets durch hinreichend kleine Input-Toleranzen $\delta$ erreicht werden können.
	\end{tabularx}
\end{definition}

\begin{proposition}
	Sei $f:D\subset X\to Y$, $X,Y$ metrischer Raum, $x_0\in D$. Dann:
	\begin{center}
		$f$ stetig in $x_0$ $\Leftrightarrow$ $f \,\epsilon\delta$-Stetig in $x_0$ $\Leftrightarrow$ $f$ folgenstetig in $x_0$
	\end{center}
\end{proposition}

\begin{definition}
	Funktion $f$ heißt stetig (folgen- / $\epsilon\delta$-stetig) auf $M\subset D$, falls $f$ stetig (folgen-/$\epsilon\delta$-stetig) in jedem Punkt $x_0\in M$.
\end{definition}
\stepcounter{theorem}
\begin{proposition}
	Sei $f:D\subset X\to Y, X,Y$ metrische Räume, dann sind folgende Aussagen äquivalent:
	\begin{enumerate}[label={\arabic*)}]
		\item $f$ stetig auf $D$
		\item $f^{-1}(V)$ offen in $D$ $\forall V\subset Y$ offen
		\item $f^{-1}(A)$ abgeschlossen in $D$ $\forall A\subset Y$ abgeschlossen
	\end{enumerate}
\end{proposition}
\begin{proposition}[Rechenregeln]
	\begin{enumerate}[label={\arabic*)}]
		\item Sei $Y$ normierter Raum über $K$, $f,g:D\subset X\to Y, \lambda: D\to U, f,g, ,y $ stetig in $x_0\in D$\\
		$\Rightarrow$ $f+g, \lambda\cdot f$ stetig in $x_0$, $\frac{1}{\lambda}$ stetig in $x_0$ falls $\lambda(x_0) \neq 0$
		\item Sei $f:D\subset X\to Y, y:\tilde{D}\subset Y\to Z, X, Y, Z$ metrischer Raum, $f$ stetig in $x_0$, $g$ stetig in $f(x_0)\in \tilde{D}$\\
		$\Rightarrow \,g\circ f$ stetig in $x_0$ 
	\end{enumerate}
\end{proposition}
\addtocounter{theorem}{3}
\begin{example}[\person{Dirichlet}-Funktion]
	$f:\mathbb{R}\to \mathbb{R}$ mit \[f(x) = \begin{cases}
	 1,&x\in\mathbb{Q}\\ 0,&\text{sonst}
	\end{cases} \] in keinem $x_0\in\mathbb{R}$ stetig.
\end{example}
\begin{proposition}
	\proplbl{chap_14_19}
	Sei $f_n, f:D\subset X\to X, f_n$ stetig in $x_0\in D$, $\forall n\in\mathbb{N}, f_n\to f$ gleichmäßig\\
	$\Rightarrow \, f$ stetig in $x_0$
\end{proposition}

\begin{conclusion}
	Falls alle $f_n$ stetig auf $M\subset D$ und $f_n\to f$ gleichmäßig auf $M$ \\
	$\Rightarrow\, f$ stetig auf $M$.
\end{conclusion}

\begin{proposition}
	Sei $f(z) := \sum_{k=0}^\infty a_k(z-z_0)^k\,\forall z\in B_r(z_0), R\in(0,\infty]$ Konvergrenzkreis, $a_k\in\mathbb{Z}\, \forall k\in \mathbb{N}$\\
	$\Rightarrow\, f:B_r(z_0) \to \mathbb{C}$ stetig auf $B_R(z_0)$
\end{proposition}
\addtocounter{theorem}{2}
\begin{definition}
	Bijektive Abbildung $f:D\subset X\to R\subset Y, X,Y$ metrische Räume, $D=\mathcal{D}(f), R=\mathcal{R}(f)$ heißt \begriff{Homöomorphismus}, falls $f$ und $f^{-1}$ stetig.
	
	Mengen $D$ und $R$ heißen \begriff[Menge!]{homöomorph} zueinander, falls es einen Homöomorphismus $f:D\to R$ mit $D=\mathcal{D}(f), R=\mathcal{R}(f)$ gibt.
	
	\highlight{beachte:} Homöomorphismus bildet offene (abgeschlossene) Mengen auf offene (abgeschlossene) Mengen ab.
\end{definition}
\stepcounter{theorem}
\begin{example}
	\begriff{stereographische Projektion}
	
	$X=\mathbb{R}^{n+1}, X_0 := \{(x_0, \dotsc, x_n{n+1}) \in\mathbb{R}^{n+1} \,|\, x_{n+1}=0\}, N = (0,\dotsc, 0,1)$ (Nordpol), $S_n = \{ x\in\mathbb{R}^{n+1} \,|\, |x|=1\}$ $n$-dimensionale Einheitsspäre.
	
	Betrachte $\sigma: \mathbb{R}^{n+1} \setminus\{ N\} \rightarrow \mathbb{R}^{n+1}$ mit $\sigma(x) = N \frac{2}{(x-N)^2}\langle x-N\rangle$ stetig. $\sigma$ ist Homöomorphismus mit $\sigma^{-1}(y) = N - \frac{2}{(y-N)^2}\langle Y-N\rangle$
\end{example}
\rule{4cm}{0.4pt}
\begin{proposition}
	Sei $f:D\subset \mathbb{R}\to \mathbb{R}$ streng monoton und stetig, $D$ Intervall \\
	$\Rightarrow f^{-1}$ existiert und ist stetig auf $\mathcal{R}(f)$.
\end{proposition}
\stepcounter{theorem}
\begin{proposition}
	Sei $f:X\to Y$ linear, $X,Y$ normierte Räume, $X=\mathcal{D}(f)$. Dann sind folgende Aussagen äquivalent:
	\begin{enumerate}[label={\arabic*)}]
		\item $f$ stetig in $x_0$
		\item $f$ ist stetig auf $X$
		\item $f$ ist beschränkt
	\end{enumerate}
\end{proposition}
\rule{4cm}{0.4pt}
\begin{definition}
	Funktion $f:D\subset X\to Y, X,Y$ metrische Räume, heißt \begriff{gleichmäßig stetig} auf $M\subset D$, falls \[ \forall \epsilon > 0 \,\exists \delta > 0: d(f(x), f(\tilde{x})) < \epsilon\quad \forall x,\tilde{x}\in M \text{ mit $d(x,\tilde{x}) < \delta$}, \]
	d.h. $f$ ist $\epsilon\delta$-stetig in jedem $\tilde{x}\in M$ \highlight{und} $\delta > 0$ kann unabhängig von $x\in M$ gewählt werden.
\end{definition}

\begin{proposition}
	Sei $f:D\subset X\to Y, X,Y$ metrischer Raum, $f$ stetig auf kompakten $M\subset D$ \\
	$\Rightarrow \,f$ gleichmäßig stetig auf $M$
\end{proposition}

\begin{definition}
	Funktion $f:D\subset X\to Y, X,Y$ metrischer Raum, heißt \begriff{\person{Lipschitz}-stetig} auf $M\subset D$, falls \begriff{\person{Lipschitz}-Konstante} $L>0$ existiert mit \begin{align}
		\tag{L} d(f(x), f(\tilde{x})) \le Ld(x,\tilde{x})
	\end{align}
	
	\highlight{Spezialfall:} $X,Y$ normierte Räume, dann hat $L$ die Form \begin{align}
		\tag{L'} \Vert f(x) - f(\tilde{x})\Vert \le L\Vert x - \tilde{x}\Vert \quad\forall x,\tilde{x}\in M
	\end{align}
	
	\highlight{Interpretation:} für $X=Y=\mathbb{R}$ fixiere $\tilde{x}$
	\begin{itemize}
		\item Graph von $f$ liegt im schraffierten Kegel
		\item muss $\forall \tilde{x}\in M$ gelten mit gleichem $L$
	\end{itemize}
\end{definition}

\begin{proposition}
	Sei $f:D\subset X\to Y$ \person{lipschitz}-stetig auf $M,X,Y$ metrische Räume\\
	$\Rightarrow$ $f$ gleichmäßig stetig auf $M$ (und damit auch stetig)
\end{proposition}

\addtocounter{theorem}{2}

\begin{definition}[Fortsetzung, Einschränkung]
    Funktion $\tilde{f}: D(\tilde{f}) \to Y$ heißt Fortsetzung (bzw. Einschränkung) von $f \mathcal{D}(f) \to Y$ auf $\mathcal{D}(f)$ falls $\mathcal{D} \subset \mathcal{D}(\tilde{f})$ (bzw. $\mathcal{D}(\tilde{f}) \subset \mathcal{D}(f)$) und $\tilde{f}(x) = f(x) \,\forall x \in \mathcal{D}$ (bzw. $\forall x \in \mathcal{D}(\tilde{f}$). Für eine eingeschränkte Funktion $f$ auf $\mathcal{D}(\tilde{f})$, schreibe $\tilde{f} = f_{\vert \mathcal{D}(\tilde{f})}$.
\end{definition}

\begin{proposition}
    Sei $f: D \subset X \to Y$ gleichmäßig stetig auf $D$, wobei $X,Y$  sind metrische Räume , $Y$ ist vollständig $\Rightarrow$ es existiert eindeutige stetige Fortsetzung $\tilde{f}$ von $f$ auf $\bar{D}$ und $\tilde{f}$ ist auf gleichmäßige stetige auf $\bar{D}$.
\end{proposition}

\begin{*remark}
    Falls $x_0$ kein Häufungspukt von $D$ ist, so kann man stets stetig auf $D\cup \{x_0\}$ fortsetzen (aber nicht eindeutig).
\end{*remark}

\addtocounter{theorem}{6}

\begin{conclusion}
    Sei $f: D \subset X \to Y$ linear, stetig, $Y$ vollständig $\Rightarrow$ es existiert eindeutig stetige Fortsetzung von $f$ auf $\bar{D}$.
\end{conclusion}
\section{Anwendung}\proplbl{chap_5}

Sei stets $f: D \subset X \to Y,X,Y$ metrische Räume, $D = \mathcal{D}(f)$.

\begin{proposition}
	\proplbl{satz_15_1}
    Sei $f: D \subset Y \to Y$ stetig, $M \subset D$ kompakt $\Rightarrow f(M)$ ist kompakt.
\end{proposition}

\begin{proposition}
    Sei $f; D \subset X \to Y$ stetig, injektiv, $D$ kompakt $\Rightarrow f^{-1}:f(D) \to D$ ist stetig.
\end{proposition}

\begin{theorem}[\index{Weierstraß}Weierstraß]\label{weierstrass}
	\proplbl{satz_von_weierstrass}
	\proplbl{chap_15_3}
    Sei $f: D \subset X \to Y$ stetig, $X$ metrischer Raum, $M \subset D$ kompakt, $M \neq \emptyset$
    \begin{align}
	    \Rightarrow \; \exists x_{min}, x_{max} : \left\{
	%\begin{cases}
       \begin{alignedat}{3}
	        f(x_{min}) &= \min&\{f(x)\mid x \in M\} &= \min_{x\in M} f(x),\\
		f(x_{max}) &= \max&\{f(x)\mid x \in M\} &= \max_{x\in M} f(x)
        %\end{cases}
	\end{alignedat}\right.
    \end{align}
    %TODO Fix x \in M under \max und \min
\end{theorem}

\begin{remark}
    Theorem \ref{weierstrass} ist wichtiger Satz für Existenz von Optimallösungen (stetige Funktion beseitzt auf kompakter Menge eine Minimum und Maximum). Folglich sind stetige Funktionen auf kompakten Mengen.
\end{remark}

\begin{proposition}
	\proplbl{chap_15_5}
    Sei $f: \mathbb{R}^n \to Y$ linear, $Y$ normierter Raum $\Rightarrow f$ ist stetig auf $\mathbb{R}^n$.\\
    Hinweis: Etwas allgemeiner hat man sogar $f: X \to Y$ linear, $X,Y$ normierte Räume, $\dim X < \infty \Rightarrow f $ ist stetig. (Ist i.a nicht richtig für $\dim X = \infty$.)
\end{proposition}

\begin{definition}[\index{Kurve}Kurve]
    Eine stetige Abbildung $f: I \subset X \to Y$, wobei $I$ Intervall und $Y$ metrischer Raum ist heißt Kurve in $Y$ (gelegentlich wird auch Mange $f(I)$ als Kurve und $f$ also zugehörige Parametrisierung bezeichnet).
\end{definition}

\begin{definition}[bogenzusammenhängende Menge]
    Menge $M \subset X$, wobei $X$ ist metrische Raum, heißt \begriff[Menge!]{bogenzusammenhängend} (bogenweise zusammenhängend) falls $\forall a,b \in M \,\exists$ Kurve $f: [a,b] \to M$ mit $f(\alpha) = a, f(\beta) = b$.\\
    Bemerkung: Eigentlich ist das die Definition für Wegzusammenhängend, leider ist das in der Literatur nicht eindeutig und manchmal wird zwischen Wegzusammenhängend und zusammenhängend noch das "`echt"' bogenzusammenhängend unterschieden. %TODO definition echt bogenzusammenhängend hinzufügen.
\end{definition}

\begin{definition}[\index{zusammenhängende Menge}zusammenhängende Menge]
    Menge $M \subset X$ heißt zusammenhängend, falls
    \begin{align}
        A, B \subset M \text{ sind offen in }M\text{, disjunkt, }\emptyset \Rightarrow M \neq A \cup B.
    \end{align}
\end{definition}

\begin{example}
    \begin{enumerate}[label={\arabic*)}]
    \item $x \in [0,2\pi] \to (x,\sin x) \in \mathbb{R}^2$ ist Kurve in $\mathbb{R}^2$
    \item $x \in [0,1] \to e^{î\pi x} \in \mathbb{C}$ oder $x \in [0,\pi]\to e^{i\pi} \in \mathbb{C}$ sind Kurven in $\mathbb{C}$
    \item Sei $Y$ normierter Raum, $a,b \in Y,f:[0,1] \to Y$ mit $f(t) = (1-t)\cdot a + t\cdot b$ ist Kurve (Strecke von $a$ nach $b$)
    \end{enumerate}
\end{example}

\begin{example}
    Sei $X=\mathbb{R}^2, M = \{(x,\sin x) \mid x \in (0,1]\} \cup \{(0,0)\}$. Dann ist $M$ zusammenhängend aber nicht bogenzusammenhängend.
\end{example}

\addtocounter{theorem}{1}
\begin{proposition}
    Sei $X$ metrischer Raum, $M \subset X$. Dann
    \begin{enumerate}[label={\arabic*)}]
    \item $X = \mathbb{R}: M$ ist zusammenhängend $\Leftrightarrow M$ ist Intervall (offen, abgeschlossen, halboffen, beschränkt, unbeschränkt).
    \item $M$ ist bogenzusammenhängend $\Rightarrow M$ ist zusammenhängend.
    \item Sei $X$ normierter Raum, dann: $M$ ist offen, zusammenhängend $\Rightarrow M$ ist bogenzusammenhängend.
    \end{enumerate}
\end{proposition}

\begin{definition}[\index{Gebiet}Gebiet]
    Sei $X$ metrischer Raum, $M \subset X$ heißt \begriff{Gebiet} falls $M$ offen und zusammenhängend ist.\\
    Beachte: Gebiet in einem normiertem Raum ist sogar bogenzusammenhängend.\\
    Offenbar: $M \subset X$ ist konvex $\Rightarrow M$ ist bogenzusammenhängend.
\end{definition}

\begin{proposition}
    Sei $f: D\subset X\to Y$ stetig, wobei $X,Y$ metrische Räume sind, dann gilt: $M \subset D$ ist zusammenhängend $\Rightarrow f(M)$ ist zusammenhängend.
\end{proposition}

\begin{theorem}[\index{Zwischenwertproposition}Zwischenwertproposition]
	\proplbl{zwischenwertsatz} \proplbl{satz_15_8}
    Sei $f: D \subset X \to \mathbb{R}, M \subset D$ zusammenhängend, $a,b \in M \Rightarrow f$ nimmt auf $M$ jeden Wert zwischen $f(a)$ und $f(b)$ an.
\end{theorem}

\addtocounter{theorem}{1}
%TODO add the example here or not?

\begin{example}
    $f:[a,b] \to \mathbb{R}$ sei stetig mit $f([a,b]) \subset [a,b] \Rightarrow$ besitzt \begriff{Fixpunkt}, d.h. $\exists x \in [a,b]\colon f(x)=x$.
\end{example}

\begin{theorem}[\index{Fundamentalproposition der Algebra}Fundamentalproposition der Algebra]\label{Fundam_algebra}
    Sei $f: \mathbb{C} \to \mathbb{C}$ Polynom vom Grad $n\geq 1$ (d.h $f(z) = a_n z^n + \dots + a_1 z + a_0,a_j \in \mathbb{C}, a_n \neq 0, n\geq 1$) $\Rightarrow f$ besitzt (mindestens eine) Nullstelle $z_0 \in \mathbb{C}$ (d.h. $f(z_0) = 0$).
\end{theorem}

\begin{conclusion}
    Jedes Polynom $f: \mathbb{C} \to \mathbb{C}$ von Grad $n, f\neq 0$ besitzt genau $n$ Nullstellen in $\mathbb{C}$ gezählt mit Vielfachen, d.h. $\exists z_1,\dots,z_l \in \mathbb{C}$, paarweise verschieden (=verschieden) $k_1,\dots, k_l \in \mathbb{N}_{\geq 0}$, $a_n \in \mathbb{C}\setminus\{0\}$ mit $k_1 + \dots + k_l = n$ und $f(z) = a_n \cdot (z-z_1)^{k_1}\cdot\dots\cdot(z-z_l)^{l}\,\forall z \in \mathbb{C}$. Hier heißt $k_j$ Vielfachheit der Nullstelle $z_j$.\\
    Hinweis: In dem Satz \ref{Polynomdiv} wurde gezeigt, das $f$ höchstens $n$ Nullstellen besitzt.
\end{conclusion}

\begin{definition}[\index{analytische Funktion}analytische Funktion]
    Abbildung $f:\mathbb{C} \to \mathbb{C}$ heißt analytisch auf $B_R(z_0)\subset \mathbb{C}$ falls $f$ auf $B_R(z_0)$ durch Potenzreihe in $z_0$ darstellbar ist, d.h.
    \[
    f(z)=\sum_{k=0}^{\infty} a_k(z-z_0)^k \quad \forall z \in B_R(z_0).
    \]
\end{definition}

\begin{proposition}
	\proplbl{chap_15_20}
    Sei $f:\mathbb{C}\to\mathbb{C}$ analytisch auf $B_R(z_0)$ und sei $B_r(z_1) \subset B_R(z_0)$ für $z_1 \in B_R(z_0),r>0 \Rightarrow f$ ist analytisch auf $B_r(z_1)$.
\end{proposition}

\begin{proposition}[\index{Identitätsproposition}Identitätsproposition]
    Seien $f,g:\mathbb{C} \to \mathbb{C}$ analytisch auf $B_R(z_0)$, sei $z_n \to \tilde{z},z_n\in B_R(z_0)\setminus\{\tilde{z}\}$ und $f(z_n) = g(z_n)\,\forall n \in \mathbb{N} \Rightarrow f(f) = g(z)\,\forall z \in B_R(z_0)$.
\end{proposition}

\begin{remark}
    Analytische Funktionen sind durch Werte auf "`sehr kleinen"' Mengen bereits festgelegt (z.B $\exp$, $\sin$, $\cos$ sind auf $\mathbb{C}$ eindeutig durch Werte auf $\mathbb{R}$ festgelegt).
\end{remark}

\begin{overview}
    Sei $X$ metrischer Raum, $Y$ normierter Raum.
    \begin{itemize}
    \item $B(X,Y):=\{f:X\to Y\mid \Vert f\Vert_{\infty} < \infty\}$ ist normierter Raum der beschränkten Funktionen mit $\Vert f\Vert_{\infty}=\sup\{\Vert f \Vert_{Y} \mid x \in X\}$.
    \item $C_b(X,Y):=\{f:X\to Y\mid \Vert f \Vert_{\infty} < \infty, f \text{ ist stetig}\}$ ist Menge der beschränkten stetigen Funktionen und offenbar eine linearer Unterraum von $B(X,Y)$ und damit auch Kern von $R \text{ mit } \Vert \cdot \Vert_{\infty}$.
    \item $C(X,Y):= \{f: X\to Y\mid f \text{ ist steig}\}$, Menge der stetigen Funktionen ist offenbar ein Vektorraum (enthält unbeschränkte Funktionen, z.B. $f(x)=\frac{1}{x} \text{ mit } x \in X = (0,1)$).
    \end{itemize}
\end{overview}

\begin{remark}
	Falls $X$ kompakt ist, dann kann man den Ausdruck $\Vert f \Vert_{\infty} < \infty$ in der Definition von $C_b(X,Y)$ weglassen (vgl. Theorem \ref{weierstrass}), d.h. $C_b(X,Y) = C(X,Y),f \text{ stetig }\Rightarrow X \to \Vert f(x)\Vert$ ist stetig $\overset{\text{Theorem 15.3}}{\Rightarrow} f$ ist beschränkt auf $X$. In diesem Fall ist auch $C(X,Y)$ mit $\Vert \cdot \Vert_{\infty}$ normierter Raum und $\Vert f\Vert_{\infty} = \max_{x\in M}\Vert f(x)\Vert_{Y}$.
\end{remark}

\begin{proposition}
    Sei $X$ metrischer Raum, $Y$ Banachraum $\Rightarrow B(X,Y)$ und $C_b(X,Y)$ und Banachräume (mit $\Vert \cdot \Vert_{\infty}$).
\end{proposition}

\begin{definition}[\index{Kontraktion}Kontraktion]
    Funktion $f: D \subset X \to X$, wobei $X$ metrischer Raum ist, heißt \begriff{Kontraktion} (bzw. kontraktiv) auf $M \subset D$ falls
    \[
    \exists L, 0 \leq L < 1\colon d(f(x),f(y)) \leq L\cdot d(x,y) \quad \forall x,y \in M.
    \]
    D.h. $f$ ist Lipschitz-stetig mit Lipschitzkonstante $L < 1$, folglich ist $f$ auch stetig.
\end{definition}

\begin{theorem}[\begriff*{Banacherscher Fixpunktproposition}Banacherscher Fixpunktproposition]\label{Banach_Fixpunkt}
    Sei $f : D\subset X \to Y$ Kontraktion auf $M \subset D, X$ vollständiger metrischer Raum (z.B. Banachraum), $M$ abgeschlossen und $f(M) \subset M$. Dann
    \begin{enumerate}
	    \item[(1)] $f$ besitzt genau einen Fixpunkt $\tilde{x}$ auf $M$ (d.h. $\exists$ genau ein $\tilde{x} \in M\colon f(\tilde{x}) = \tilde{x}$).
        \item[(2)] Für $\{x_n\}$ in $M$ mit $x_{n+1}=f(x_n),x_0 \in M$ (beliebig) gilt:
            \[
            x_n \to x \text{ und } d(x_n,\tilde{x}) \leq \frac{L^n}{1-L}\cdot d(x_0,x_1) \quad \forall n \in \mathbb{N}.
            \]
        \end{enumerate}
        Hinweis: Theorem \ref{Banach_Fixpunkt} ist eine wichtige Grundlage für Iterationsverfahren in der Numerik.
\end{theorem}

\subsection*{Partialbruchzerlegung}

\begin{definition}[\index{Pol der Ordnung $k$}Pol der Ordnung $k$]
    Sei $R: \mathbb{C} \to \mathbb{C}$ rationale Funktion, d.h. $R(z) = \frac{f(z)}{g(z)}$ für Polynome $f$, $g$ existieren mit
    \begin{align*}
	    R(z) &= \frac{\tilde{f}(z)}{(z-z_0)^k\cdot \tilde{g}}& &\text{und}& \tilde{f}(z_0) \neq 0,&\;\tilde{g}(z_0) \neq 0.
    \end{align*}
    Motivation: Gelgentlich ist gewisse additive Zerlegung von rationalen Funktionen wichtig (Integration) z.B.
    \[
    \frac{2x}{x^2 - 1} = \frac{2x}{(x-1)(x+1)} = \frac{1}{x+1}+\frac{1}{x-1}.
    \]
\end{definition}

\begin{lemma}
    Sei $R: \mathbb{C} \to \mathbb{C}$ rationale Funktion, $z_0 \in \mathbb{C}$ Pol der Ordnung $k\geq 1 \Rightarrow \,\exists ! a_1,\dots,a_k \in \mathbb{C},a_k\neq 0$ und $\exists !$ Polynom $\tilde{p}$ mit 
    \[
    R(z) = \sum_{i=1}^{k}
    \frac{a_i}{(z-z_0)^{î}} + \frac{\tilde{p}(z)}{\tilde{g}(z)} = H(z) +\frac{\tilde{p}(z)}{\tilde{g}(z)}
    \]
    $H(z)$ heißt Hauptteil von $R \text{ in } z_0$. Beachte das $\frac{\tilde{p}}{\tilde{g}}$ keine Pole in $z_0$ hat.
\end{lemma}

\begin{proposition}[\index{Partialbruchzerlegung}Partialbruchzerlegung]
    Sei $R: \mathbb{C} \to \mathbb{C}$ rationale Funktion, $R(z)=\frac{f(z)}{g(z)}$ für Polynome $f,g$. Sei $g(z) = \prod_{i=1}^{l}(z-z_i)^{k_i}$ gemäß Fundamentalproposition der Algebra(Theorem \ref{Fundam_algebra}). Seien $z_1,\dots,z_l$ keine Nullstellen von $f$ und seien $H_1,\dots,H_l$ Hauptteile von $R$ in $z_1,\dots,z_l \Rightarrow$
    \[
    \exists \text{ Polynom } p:R(z)=H_1(z)+\dots+H_l(z)+p(z) \quad\forall z \neq z_j \,\forall j = 1,\dots,l
    \]
    wobei $f(z) = p(z)\cdot g(z) + r(z)\,\forall z$ für Polynom $r$. $p=0$ falls $\grad(f) < \grad(g)$ (vgl Satz \ref{Polynomdiv} Polynomdivision)
\end{proposition}

\part{2. Semester}
\pagestyle{fancy}

\chapter{Differentiation}
\section{Wiederholung und Motivation}
Sei $K^n$ $n$-dim. VR über Körper mit $K=\mathbb{R}$ oder $K=\mathbb{C}, n\in\mathbb{N}_{\ge 0}$.
\begin{itemize}
	\item Elemente sind alle $x=(x_1, \dotsc, x_n)\in K^n$ mit $x_1, \dotsc, x_n\in K$.
	\item Standardbasis ist $\{e_1, \dotsc, e_n\}$
	\item alle Normen auf $K^n$ sind äquivalent $\Rightarrow$ Konvergenz unabhängig von der Norm, verwende in der Regel euklidische Norm
	\item Skalarprodukt
	\begin{itemize}
		\item $\langle x,y \rangle = \sum\limits_{j=1}^{n} x_j\cdot y_j$ in $\mathbb{R}^n$
		\item $\langle x,y \rangle = \sum\limits_{j=1}^{n} \overline{x}_j\cdot y_j$ in $\mathbb{C}^n$
	\end{itemize}
	\item \textsc{Cauchy}-\textsc{Schwarz}-Ungleichung ($\vert \langle x,y\rangle \vert \le \vert x \vert \cdot \vert y \vert\,\quad\forall x,y\in K^n$)
\end{itemize}

\subsection{Lineare Abbildungen}
Eine lineare Abbildung ist homogen und additiv
\begin{itemize}
	\item Lineare Abbildung $A: K^n \rightarrow K^m$ ist darstellbar durch $m\times n$-Matrizen bezüglich der Standardbasis 
	\begin{itemize}
		\item lineare Abbildung ist stetig auf endlich-dimensionalen Räumen (unabhängig von der Norm)
		\item transponierte Matrix: $A^T\in K^{n\times m}$
		\item $x^T\cdot y = \langle x,y\rangle$
		\item $x\cdot y^T = x\otimes y$, sogenanntes Tensorprodukt
	\end{itemize}	
	 \item $L(K^n,K^m) = \{ A: K^n \to K^m\mid \text{ $A$ linear}\}$ (Menge der linearen Abbildung, ist normierter Raum)
	\begin{itemize}
		 \item $\Vert A\Vert= \sup\{ \vert Ax\vert \mid \vert x \vert \le 1 \}$ (Operatornorm, $\Vert A \Vert$ hängt i.A. von Normen auf $K^n, K^m$ ab)
		 \item in der Regel wird euklidische Norm verwendet: $\vert A \vert = \sqrt{\sum_{k,l} \vert a_{kl} \vert^2}$
		 \item $L(K^n, K^m)$ ist isomorph zu $K^{m\times n}$ als VR \\
		 $\Rightarrow$ $L(K^n, K^m)$ ist $m\cdot n$-dim. VR
		 \item Es gilt: 
		 \begin{align}
		 	\vert Ax \vert \le \Vert A \Vert\cdot \vert x \vert \text{ und } \vert Ax\vert \le \vert A \vert \cdot\vert x \vert\notag
		 \end{align}
	\end{itemize}
	\item Abbildung $\tilde{f}: K^n \to K^m$ heißt affin linear, falls $\tilde{f}(x) = Ax + a$ für lineare Abbildung $A:K^n\to K^m, a\in K^m$
\end{itemize}

\subsection{\textsc{Landau}-Symbole}

\begin{*definition}[Landau-Symbole]
	Sei $f:D\subset K^n \to K^m$, $g:D\subset K^n \to K$, $x_0 \in \overline{D}$. Dann:
	\begin{itemize}
		\item $f(x) = o(g(x))$ für $x\to x_o$ gdw. $\lim\limits_{\substack{x\to x_0 \\ x\neq x_0}} \frac{\vert f(x) \vert}{g(x)} = 0$
		\item $f(x) = \mathcal{O}(g(x))$ für $x\to x_0$ gdw. $\exists \delta > 0, c \ge 0: \frac{\vert f(x) \vert}{\vert g(x) \vert} \le c \;\forall x\in \left( B_\delta(x_0)\setminus \{ x_0\}\right) \cap D$
	\end{itemize}
\end{*definition}

\begin{*definition}[Anschmiegen]
	$f(x) + \underbrace{f(x_0) + A(x-x_0)}_{\tilde{A}(x)} = o(\vert x-x_0\vert)$, \\
	d.h. die Abweichung wird schneller klein als $\vert x-x_0\vert$!
\end{*definition}

\begin{proposition}[Rechenregeln für \person{Landau}-Symbole]
	Für $r_k,\tilde{r}_l,R_l:D\subset K^n\to K^m,x_0\in D,k,l\in\natur$ mit
	\begin{align}
		r_k(x)=o(\vert x-x_0\vert^k) \notag \\
		\tilde{r}_l=o(\vert x-x_0\vert^l) \notag \\
		R_l(x)=\mathcal{O}(\vert x-x_0\vert ^l) \notag
	\end{align}
	für $x\to x_0$
	\begin{enumerate}
		\item $r_k(x)=o(\vert x-x_0\vert^j)=\mathcal{O}(\vert x-x_0\vert^j)\quad j\le k$ \\
		$R_l(x)=o(\vert x-x_0\vert^j)=\mathcal{O}(\vert x-x_0\vert^j)\quad j<l$
		\item $\frac{r_k(x)}{\vert x-x_0\vert^j}=o(\vert x-x_0\vert^{k-j})\quad j\le k$ \\
		$\frac{R_l(x)}{\vert x-x_0\vert^j}=\mathcal{O}(\vert x-x_0\vert^{l-j})=o(\vert x-x_0\vert^{l-j-1})\quad j\le l$
		\item $r_k(x)\pm \tilde{r}_l(x)=o(\vert x-x_0\vert ^k)\quad k\le l$
		\item $r_k(x)\cdot \tilde{r}_l(x)=o(\vert x-x_0\vert^{k+l}),r_k(x)\cdot R_l(x)=o(\vert x-x_0\vert^{k+l})$
	\end{enumerate}
\end{proposition}
\begin{proof}
	Sei $\frac{\vert R_l(x)\vert}{\vert x-x_0\vert^l}\le c$ nahe $x_0$, d.h. auf $(B_{\delta}(x_0)\backslash\{x_0\})\cap D$ für ein $\delta>0$
	\begin{enumerate}
		\item $\frac{r_k(x)}{\vert x-x_0\vert^j}=\frac{r_k(x)}{\vert x-x_0\vert^k}\vert x-x_0\vert^{k-j}\to 0$, folgl. $\frac{r_k(x)}{\vert x-x_0\vert^{\delta}}$ auch beschränkt nahe $x_0$ \\
		$\frac{R_l(x)}{\vert  x-x_0\vert^j}=\frac{R_l(x)}{\vert x-x_0\vert^l}\vert x-x_0\vert^{l-j}\to 0$, Rest wie oben
		\item $\frac{r_k(x)}{\vert x-x_0\vert^j \vert x-x_0\vert^{k-j}}=\frac{r_k(x)}{\vert x-x_0\vert^k}\to 0$ \\
		$\frac{R_l(x)}{\vert x-x_0\vert^j \vert x-x_0\vert^{l-j}}=\frac{R_l(x)}{\vert x-x_0\vert^l}\le c$ nahe $x_0$, Rest wie oben
		\item $\frac{r_k(x)}{\vert x-x_0\vert^k}\pm\frac{\tilde{r}_l(x)}{\vert x-x_0\vert^k}\overset{(2)}{=}o(1)\pm\underbrace{o(\vert x-x_0\vert^{l-k})}_{o(1)}\to 0$
		\item $\frac{r_k(x)\cdot \tilde{r}_l(x)}{\vert x-x_0\vert^{k+l}}=\frac{r_k(x)}{\vert x-x_0\vert^k}\cdot\frac{\tilde{r}_l(x)}{\vert x-x_0\vert^l}\to 0$ \\
		$\frac{\vert r_k(x)\cdot R_l(x)\vert}{\vert x-x_0\vert^{k+l}}=\frac{\vert r_k(x)\vert}{\vert x-x_0\vert^k}\cdot\frac{\vert R_l(x)\vert}{\vert x-x_0\vert^l}\to 0$
	\end{enumerate}
\end{proof}

\begin{example}
	\begin{itemize}
		\item offenbar in $K^n$: $\vert x-x_0\vert^k=\mathcal{O}(\vert x-x_0\vert^k)=o(\vert x-x_0\vert^{k-1})$, $x\to x_0$
		\item in $\real$ gilt für $x\to 0$:
		\begin{itemize}
			\item $x^5=o(\vert x\vert^4)$, $x^5=o(\vert x\vert)$, $x^5=\mathcal{O}(\vert x\vert^5)$, $x^5=\mathcal{O}(\vert x\vert^3)$
			\item $e^x=\mathcal{O}(1)=3+\mathcal{O}(1)$, $e^x=1+o(1)\neq 2+o(1)$
			\item $\sin(x)=\mathcal{O}(\vert x\vert)$, $\sin(x)=o(1)$, $x^3\cdot\sin(x)=o(\vert x\vert^3)$, $e^x\cdot \sin(x)=o(1)$
			\item $(1-\cos(x))x^2=\mathcal{O}(\vert x\vert^2)x^2=o(\vert x\vert^3)$
			\item $\frac{1}{o(1)+\cos(x)}=e^x+o(1)=1+o(1)$
		\end{itemize}
	\end{itemize}
\end{example}
\section{Ableitung} \setcounter{equation}{0}
\proplbl{section_ableitung}

\begin{*definition}[differenzierbar, Ableitung]
	Sei $f: D\subset K^n \to K^m$, $D$ offen, heißt \begriff{differenzierbar} in $x\in D$, falls es lineare Abbildung $A\in L(K^n, K^m)$ gibt mit \begin{align}
		\proplbl{definition_ableitung}
		\Aboxed{f(x) &= f(x_0) + A(x-x_0) + o(\vert x-x_0 \vert), x\to x_0}
	\end{align}
	
	Abbildung $A$ heißt dann \begriff{Ableitung} von $f$ in $x_0$ und wird mit $f'(x_0)$ bzw. $\mathrm{D}f(x_0)$ bezeichnet (statt dem Terminus Ableitung auch (totales) Differential, \person{Frechet}-Abbildung, \person{Jacobi}-Matrix, Funktionalmatrix).
	
	Andere Schreibweisen: $\frac{\partial f}{\partial x}(x_0), \left.\frac{\partial f(x)}{\partial x}\right|_{x=x_0}, \mathrm{d}f(x_0), \dotsc$
	
	Somit ist \propref{definition_ableitung} gleichwertig mit \begin{align}
		\proplbl{definition_ableitung_zwei_stern}\Aboxed{f(x) &= f(x_0) + f'(x_0) \cdot (x - x_0) + o(\vert x - x_0\vert), \text{ für } x\to x_0}\notag
	\end{align}
\end{*definition}

\begin{*anmerkung}
	Eine andere Erklärung der oben stehenden Definition wäre folgende: \\
	Eine Funktion $f$ ist genau dann differenzierbar an der Stelle $x_0$, wenn eine reelle Zahl $m$ (die von $x_0$ abhängen darf) und eine (ebenfalls von $x_0$ abhängige) Funktion $r$ (Fehler der Approximation) mit folgenden Eigenschaften existieren:
	\begin{itemize}
		\item $f(x_0+h)=f(x_0)+m\cdot h+r(h)$
		\item Für $h\to 0$ geht $r(h)$ schneller als linear gegen 0, d.h. $\frac{r(h)}{h}\to 0$ für $h\to 0$
	\end{itemize}
	Die Funktion $f$ lässt sich also in der Nähe von $x_0$ durch eine lineare Funktion $g$ mit $g(x_0+h)=f(x_0)+m\cdot h$ bis auf den Fehler $r(h)$ approximieren. Den Wert $m$ bezeichnet man als Ableitung von $f$ an der Stelle $x_0$.
	
	\begin{center}\includegraphics[width=0.4\textwidth]{pictures/diff-definition.png}\end{center} %70% der Textbreite%TODO: Bild selber machen
\end{*anmerkung}

\begin{*anmerkung}
	Neben der oben genannten Definition gibt es noch eine weitere Definition, die sich des Differentialquotienten bedient:
	\begin{align}
		f\text{ differentierbar in }x_0 \iff \lim\limits_{x\to x_0}\frac{f(x)-f(x_0)}{x-x_0}= \lim\limits_{h\to 0}\frac{f(x_0+h)-f(x_0)}{h}\text{ exisitiiert}\notag
	\end{align}
	Diese Definition lässt sich im Kontext komplexer oder mehrdimensionaler Funktionen nicht anwenden, zudem sind Beweise wegen des Quotienten schwerer zu führen.
\end{*anmerkung}

\begin{*remark}
	Affin lineare Abbildung $\tilde{f}(x) := f(x_0) + f'(x_0)\cdot(x-x_0)$ approximiert die Funktion $f$ in der Nähe von $x_0$ und heißt \begriff{Linearisierung} von $f$ in $x_0$ (man nennt \propref{definition_ableitung} auch Approximation 1. Ordnung von $f$ in der Nähe von $x_0$).
\end{*remark}

\begin{conclusion}
	\proplbl{definition_ableitung_proposition}
	Sei $f:D\subset K^n\to K^m$, $D$ offen. Dann: \\
	$f$ ist differenzierbar in $x_0\in D$ mit Ableitung $f'(x_0) \in L(K^n, K^m)$ \gls{gdw} eine der folgenden Bedingungen erfüllt ist:
	\begin{enumerate}[label={\alph*)},mode=unboxed]
		\item \label{satz_equivalenz_ableitungen_a} \itemEq{f(x) &= f(x_0) + f'(x_0) \cdot (x-x_0) + r(x) \quad \forall x\in D}
		für ein $r: D\to K^m$ mit $\lim\limits_{\substack{x\to x_0 \\ x\neq x_0}} \frac{r(x)}{\vert x - x_0 \vert} = 0$
		\item \proplbl{satz_equivalenz_ableitungen_b} $f(x) =f(x_0) + f'(x_0) \cdot (x-x_0) + R(x) (x-x_0)\quad\forall x\in D\proplbl{definition_ableitung_zwei}$\\
		für ein $R:D \to L(K^n, K^m)$ ($\cong K^{m\times n}$) mit $\lim\limits_{x\to x_0} R(x) = 0$ (d.h. Matrizen $R(x) \xrightarrow{x\to x_0}$ Nullmatrix in $K^{m\times n}$) 
		\item \proplbl{definition_ableitung_drei} \label{satz_equivalenz_ableitungen_c} $f(x) = f(x_0) + Q(x) (x - x_0) \quad \forall x\in D$\\ für ein $Q:D\to L(K^n, K^m)$ ($\cong K^{m\times n}$) mit $\lim\limits_{x\to x_0} Q(x) = f'(x_0)$ (d.h. Matrizen $Q(x) \xrightarrow{x\to x_0}$ Matrix $f'(x_0)$ in $K^{m\times n}$)
	\end{enumerate}
\end{conclusion}

\begin{*remark}
	Es gilt:
	\begin{align*}
	\text{\propref{definition_ableitung_eins}}\; \Leftrightarrow\; \lim\limits_{\substack{x\to x_0 \\ x\neq x_0}} \frac{f(x) - f(x_0) - f'(x_0) (x - x_0)}{\vert x - x_0 \vert} = 0
	\end{align*}
\end{*remark}

\begin{proof}
	\NoEndMark
	Aussage \ref{satz_equivalenz_ableitungen_a} ist leicht zu zeigen, anschließend erfolgt per Ringschluss die Äquivalenz der anderen Definitionen.
	\begin{enumerate}[label={zu \alph*)},leftmargin=\widthof{\ zu a)\ }]
		\item Offensichtlich ist $r(x) = o(\vert x - x_0 \vert ),$ $x\to x_0$ \\
		$\Rightarrow\;$ \ref{satz_equivalenz_ableitungen_a} $\Leftrightarrow$ $f$ ist differenzierbar in $x_0$ mit Ableitung $f'(x_0)$
	\end{enumerate}
	Ringschluss:
	\begin{itemize}[leftmargin=\widthof{\ a) $\rightarrow$ b):\ },topsep=-5pt]
		\item[a) $\Rightarrow$ b):] Sei $R: D\to K^{m\times n}$ gegeben durch \marginnote{$\otimes$: Tensorprodukt (siehe \cpageref{definition_tensorprodukt})}\begin{align*}
			R(x) &= \begin{cases}
				0, & x = x_0 \\
				\frac{r(x)}{\vert x - x_0\vert} \otimes (x - x_0)^T, & x\neq x_0
			\end{cases}\\
			\Rightarrow \;R(x) (x - x_0) &= \left( \frac{r(x)}{\vert x - x_0\vert^2} \otimes (x - x_0)^T \right) \cdot (x - x_0)\\
			 &= \frac{r(x)}{\vert x - x_0\vert ^2} \cdot \langle x - x_0 , x - x_0 \rangle = r(x) \quad \forall x\neq x_0
		\end{align*}
		Wegen $0 = r(x_0) = R(x_0)\cdot (x - x_0)$ folgt \begin{align*}
			\lim\limits_{x\to x_0} \vert R(x) \vert = \lim\limits_{\substack{x\to x_0 \\ x\neq x_0}} \frac{\vert r(x) \otimes (x - x_0)^T\vert }{\vert x - x_0 \vert^2} = \lim\limits_{\substack{x\to x_0 \\ x\neq x_0}} \frac{\vert r(x)\vert}{\vert x - x_0\vert} = 0
			\end{align*}
			
		\item[b) $\Rightarrow$ c):] Setzte $Q(x) := f'(x_0) + R(x) \; \forall x\in D$ $\Rightarrow$ \propref{definition_ableitung_drei}. Wegen $\lim\limits_{x\to x_0} Q(x) = f'(x_0)$ folgt \ref{satz_equivalenz_ableitungen_c}.
			
		\item[c) $\Rightarrow$ a):] Setzte $r(x) := (Q(x) - f'(x))\cdot (x - x_0) \;\forall x\in D$ $\Rightarrow$ \propref{definition_ableitung_eins}. Wegen $\vert r(x) \vert \le \vert Q(x) - f'(x_0) \vert \cdot \vert x - x_0 \vert $ folgt \zeroAmsmathAlignVSpaces \begin{align*}
			\lim\limits_{\substack{x\to x_0 \\ x\neq x_0}} \frac{\vert r(x) \vert}{\vert x - x_0 \vert} =  \lim\limits_{\substack{x\to x_0 \\ x\neq x_0}} \vert Q(x) - f'(x_0) \vert = 0
		\end{align*}
		\hfill$\square$
	\end{itemize}
\end{proof}

\begin{*definition}[stetig differenzierbar]
	Funktion $f$ heißt \begriff[stetig!]{stetig differenzierbar} bzw. \begriff{$C^1$-Funktion} auf $D$ falls $f'$ stetig auf $D$ \\
	$C^1(D,K^m):=\{f:D\to K^m\mid f\text{ stetig differenzierbar auf } D\}$, kurz $C^1(D)$
\end{*definition}

\begin{hint}
	Die obige Definition ist allgemein in normierten Räumen verwendbar. In der Literatur: statt differenzierbar $\to$ total differenzierbar
\end{hint}

\begin{proposition}
	\proplbl{diffbar_impl_stetig}
	Sei $f:D\subset K^n \to K^m$, $D$ offen, differenzierbar in $x_0\in D$. Dann:
	\begin{enumerate}[label={\arabic*)}]
		\item $f$ ist stetig in $x_0$
		\item Die Ableitung $f'(x_0)$ ist eindeutig bestimmt.
	\end{enumerate}
\end{proposition}

\begin{proof}
	\NoEndMark \hspace*{0pt}
	\begin{enumerate}
		\item \propref{definition_ableitung} liefert \begin{align*}
			& \lim\limits_{x\to x_0} f(x) = \lim\limits_{x\to x_0} \left( f(x_0) + f'(x_0) \cdot \underbrace{(x - x_0)}_{=0} + R(x) \underbrace{(x - x_0)}_{=0} \right) = f(x_0) \\
			\Rightarrow & \text{ Behauptung}
		\end{align*}
		\item Angenommen, $A_1, A_2\in L(K^n, K^m)$ sind Ableitungen von $f$ in $x_0$. Seien $R_1, R_2$ die zugehörigen Terme in \propref{definition_ableitung}. Dann gilt für $x=x_0 + ty \;\forall y\in K^n, t\in \mathbb{R}$:
		\vspace*{2mm} \ \\
		\renewcommand{\arraystretch}{1.2}
		\begin{tabularx}{\linewidth}{r@{\ \ }r@{\ }X}
			& $\vert (A_1 - A_2)(t\cdot y)\vert$ & $\le \vert R_1(x_0 + ty) \cdot (t y)\vert + \vert R_2(x_0 + ty) \cdot (ty) \vert$ \\
			& & $\le \vert R_1(x_0 + ty)\vert \cdot \vert ty \vert + \vert R_2(x_0 + ty)\vert \cdot \vert ty\vert$ \marginnote{$\left|  \cdot \frac{1}{\vert t \vert}\right.$}[-0.4em]\\
			$\xRightarrow{t \neq 0}$ &  $0 \le \vert (A_1 - A_2) \cdot y\vert$ & $\le \big( \vert R_1(x_0 + ty) \vert + \vert R_2(x_0 + t y)\vert \big) \cdot \vert y \vert \xrightarrow{t\to 0} 0$ \\
			$\Rightarrow$ & \multicolumn{2}{l}{$(A_1 - A_2) \cdot y = 0 \quad \forall y\in K^n$} \\
			$\Rightarrow$ & \multicolumn{2}{l}{$A_1 = A_2 \quad \Rightarrow \text{Behauptung}$ }
		\end{tabularx}
		
		\hfill\csname\InTheoType Symbol\endcsname
	\end{enumerate}
\end{proof}

\subsection{Spezialfälle für \texorpdfstring{$K=\mathbb{R}$}{K=R}}
\begin{enumerate}[label={\arabic*)},leftmargin=\widthof{1)\ },topsep=-5pt]
	\item \proplbl{spezialfall_ableitung_m1_item} \uline{$m=1\negthickspace:\, f\negthickspace:\mathbb{R}^n\to \mathbb{R}$}\\[0.6ex]
	$f'(x_0)\in \mathbb{R}^{1\times n}$ ist Zeilenvektor, $f'(x_0)$ betrachtet als Vektor im $\mathbb{R}^n$ auch \begriff{Gradient} genannt.
	
	Offenbar gilt $f'(x_0)\cdot y = \langle f'(x_0), y\rangle\;\forall y\in\mathbb{R}^n$ (Matrizenmultiplikation = Skalarprodukt) \\
	$\Rightarrow$ \propref{definition_ableitung} hat die Form \begin{align}
		\proplbl{spezialfall_ableitung_m1}
		f(x) = \underbrace{f(x_0) + \langle f'(x_0), x - x_0\rangle}_{\mathclap{\text{affin lineare Funktion: }\tilde{f}: \mathbb{R}\to \mathbb{R} \,(\text{in }x)}} + o\big( \vert x - x_0\vert \big)
	\end{align}
	Graph von $f$ ist Fläche im $\mathbb{R}^{n\times 1}$, genannt \begriff{Tangentialebene} vom Graphen von $f$ in $\big(x_0, f(x_0)\big)$.
	
	\item \proplbl{spezialfall_ableitung_n1} \uline{$n=1\negthickspace: f\negthickspace: D\subset \mathbb{R}\to \mathbb{R}^n$}\ \ (z.B. $D=(a,b)$)\\[0.6ex]
	$f$ (bzw.  Bild $f[D]$) ist Kurve im $\mathbb{R}^n$ ($\cong \mathbb{R}^{m\times 1}$). \propref{definition_ableitung} kann man schreiben als \begin{align}
		f(x_0 + t) = \underbrace{f(x_0) + t\cdot f'(x_0)}_{\mathclap{\text{Affin lineare Abb. }\tilde{f}:\mathbb{R}\to \mathbb{R}^m \text{ (in $t$)}}} + o(t), t\to 0, t\in\mathbb{R}
	\end{align}
	\zeroAmsmathAlignVSpaces
	\begin{alignat}{2}
		\notag &\Leftrightarrow\quad& \underbrace{\frac{f(x_0 + t) - f(x_0)}{t}}_{\mathclap{\text{\begriff{Differenzenquotient} von $f$ in $x_0$}}} &= f'(x_0) + o(1), t\to 0 \\
		\proplbl{differentialquotient} &\Leftrightarrow& \underbrace{\lim\limits_{t\to 0} \frac{f(x_0 + t) - f(x_0)}{t}}_{\mathclap{\text{Differentialquotient}}} &= f(x_0)
	\end{alignat}
	
	\emph{beachte:} \begin{itemize}
		\item $f$  \gls{diffbar} in $x_0$ $\Leftrightarrow$ Differentialquotient existiert in $x_0$
		\item \propref{differentialquotient} nicht erklärt im Fall von $n>1$
	\end{itemize}

	\begin{interpretation}[ für $m > 1$]
		$f'(x_0)$ heißt \begriff{Tangentenvektor} an die Kurve in $f(x_0)$. Falls $f$ nicht  \gls{diffbar} in $x_0$ bzw. $x_0$ Randpunkt in $D$ und ist $f(x_0)$ definiert, so betrachtet man in \propref{differentialquotient} auch einseitige Grenzwerte (vgl. \propref{einseitige_grenzwerte}).
		
		$\lim\limits_{t\downarrow 0} \frac{f(x_0 + t) - f(x_0)}{t} = f_r'(x_0)$ heißt \begriff[Ableitung!]{rechtsseitige} \uline{Ableitung} von $f$ in $x_0$ (falls existent), analog ist $\lim\limits_{t\uparrow 0}$ die \begriff[Ableitung!]{linksseitige} \uline{Ableitung} $f_l'(x_0)$.
	\end{interpretation}

	\item \uline{$n=m=1\negthickspace:\;f\negthickspace: D\subset \mathbb{R}\to \mathbb{R}$} (vgl. Schule)\\[0.6ex]
	$f'(x_0)\in \mathbb{R}$ ist Zahl und \propref{differentialquotient} gilt (da Spezialfall von \propref{spezialfall_ableitung_n1}).
	
	\emph{Beobachtung:} \propref{spezialfall_ableitung_n1} gilt allgemein für $n=1$, nicht für $n>1$!
\end{enumerate}
\vspace*{1.5
	em}

\begin{proposition}[Differentialquotient für $n=1$]
	Sei $f:D\subset K\to K^n$, $D$ offen. Dann:
	\begin{align}
		\notag& \text{$f$ ist differenzierbar in $x_0\in D$ mit Ableitung $f'(x_0)\in L(K, K^m)$} \\
		\Leftrightarrow\quad
		& \proplbl{differentialquotient_prop} \exists f'(x_0) \in L(K, K^m): \lim\limits_{y\to 0} \frac{f(x_0 + y) - f(x_0)}{y} = f'(x_0) \\
		\notag 
		& \text{alternativ: } \lim\limits_{x\to x_0} \frac{f(x) - f(x_0)}{x - x_0} = f'(x_0)
	\end{align}
\end{proposition}

\subsection{Einfache Beispiele für Ableitungen}
\begin{example}[affin lineare Funktionen]
	\proplbl{ableitung_linear}
	Sei $f:K^n\to K^m$ affin linear, d.h. \begin{align*}
		f(x) = A\cdot x + a\quad \forall x\in K^n, \text{ mit } A\in L(K^n, K^m), \, a\in K^m \text{ fest}
	\end{align*}
	Dann gilt für beliebiges $x_0\in K^n$:
	\zeroAmsmathAlignVSpaces**
	\begin{align*}
		f(x) &= A\cdot x_0 + a + A(x - x_0) \\
		&=f(x_0) + A(x - x_0)
	\end{align*}
	\zeroAmsmathAlignVSpaces
	\begin{align*}
		\xRightarrow{(\ref{definition_ableitung})}\;\; \text{$f$ ist  \gls{diffbar} in $x_0$ mit } f'(x_0) = A
	\end{align*}
	Insbesondere gilt für konstante Funktionen $f'(x_0) = 0$ \\
	Offenbar $x\to f(x)=A$ stetig auf $K^n$, d.h. $f\in C^1(K^n)$
	\begin{center}\begin{tikzpicture}
		\begin{axis}[
		xmin=-5, xmax=5, xlabel=$x$,
		ymin=-5, ymax=5, ylabel=$y$,
		samples=400,
		axis y line=middle,
		axis x line=middle,
		]
		\addplot+[mark=none] {2*x};
		\addlegendentry{$2x$}
		\addplot+[mark=none, dashed] {2};
		\addlegendentry{$2$}
		\end{axis}
		\end{tikzpicture}\end{center}
\end{example}
\begin{example}[quadratische Funktion]
	\proplbl{ableitung_beispiel_euklidische_norm}
	Sei $f:\mathbb{R}^n\to \mathbb{R}$, $f(x) = \vert x \vert ^2\;\forall x\in\mathbb{R}^n$
	
	Offenbar gilt:
	\begin{flalign*}
		 \vert x - x_0 \vert ^2 &= \langle x - x_0, x - x_0 \rangle \marginnote{Erweitert mit $\langle x_0, x_0\rangle$} &\\
		 &= \langle x \rangle^2 - 2 \langle x_0, x\rangle + 2 \langle x_0, x_0 \rangle - \langle x_0, x_0\rangle& \\
		 &= \vert x \vert ^2 - 2 \langle x_0, x - x_0 \rangle - \vert x_0 \vert ^2 &\\
		\qquad\Rightarrow \qquad f(x) &= f(x_0) + \langle 2x_0, x - x_0 \rangle + \underbrace{\vert x - x_0\vert^2}_{\mathclap{=o\big( \langle x - x_0 \rangle \big)}}&
 	\end{flalign*}
 	(vgl. auch \propref{spezialfall_ableitung_m1} im Spezialfall \ref{spezialfall_ableitung_m1_item})
 	
 	Wegen $2x_0\in L(\mathbb{R}^n, \mathbb{R})$ folgt $f = \vert \cdot \vert^2$ ist  \gls{diffbar} in $x_0$ mit $f'(x_0) = 2 x_0\;\forall x_0\in\mathbb{R}$
 	\begin{center}\begin{tikzpicture}
 		\begin{axis}[
 		xmin=-5, xmax=5, xlabel=$x$,
 		ymin=-5, ymax=5, ylabel=$y$,
 		samples=400,
 		axis y line=middle,
 		axis x line=middle,
 		]
 		\addplot+[mark=none] {abs(x)^2};
 		\addlegendentry{$\vert x\vert^2$}
 		\addplot+[mark=none, dashed] {2*x};
 		\addlegendentry{$2x$}
 		\end{axis}
 		\end{tikzpicture}\end{center}
\end{example}

\begin{example}[Funktionen mit höherem Exponent]
	Sei $f:K\to K$, $f(x) = x^k$, $k\in\mathbb{N}$.
	\begin{itemize}[leftmargin=\widthof{$\,k=0$:\ }]
		\item[$k=0$:] $f(x) = 1\;\forall x$ $\Rightarrow$ $f'(x_0) = 0\;\forall x_0\in\mathbb{C}$ (vgl. \propref{ableitung_linear})
		\item[$k\ge 1$:] Es gilt \\
		\renewcommand{\arraystretch}{1.2}
		\begin{tabularx}{\linewidth}{r@{\ \ }r@{$\,$}X}
			& $(x_0 + y)^k$ & $\displaystyle = \sum_{j=0}^{k}\binom{k}{j} x_0^{k-j}\cdot y^j = x_0^k + k\cdot x_0^{k-1}\cdot y + o(y),\;y\to 0$ \\
			$\Rightarrow$ & $f(x_0 + y)$ & $= f(x_0) + k\cdot x_0^{k-1}\cdot y + o(y), y\to 0$ \\
			$\xRightarrow{(\ref{definition_ableitung})}$ & $f'(x_0)$ & $= k\cdot x_0^{k-1}$
		\end{tabularx}
	\end{itemize}
	\emph{beachte:} gilt in $\mathbb{C}$ und $\mathbb{R}$.
	\begin{center}\begin{tikzpicture}
		\begin{axis}[
		xmin=-5, xmax=5, xlabel=$x$,
		ymin=-5, ymax=5, ylabel=$y$,
		samples=400,
		axis y line=middle,
		axis x line=middle,
		]
		\addplot+[mark=none] {x^3};
		\addlegendentry{$x^3$}
		\addplot+[mark=none, dashed] {3*x^2};
		\addlegendentry{$3x^2$}
		\end{axis}
		\end{tikzpicture}\end{center}
\end{example}

\begin{example}[Betragsfunktion]
	\proplbl{ableitung_beispiel_betrag}
	Sei $f:\mathbb{R}^n\to \mathbb{R}$, $f(x) = \vert x \vert$ $\forall x\in\mathbb{R}^n$.
	
	$f$ ist nicht \gls{diffbar} in $x_0=0$, denn angenommen die Ableitung $f'(0)\in\mathbb{R}^n$ ($\cong \mathbb{R}^{1 \times n}$) existiert, dann fixiere $x\in\mathbb{R}^n$ mit $\vert x \vert = 1$, und
	\begin{alignat*}{2}
		&& \vert t\cdot x\vert &= 0 + \langle f'(0), t\cdot x \rangle + o(t),\;t\to 0,\; t\in\mathbb{R}_{\neq 0} \marginnote{$\left| \cdot \frac{1}{t}\right.$} \\
		\xRightarrow{t\neq 0}\;&& \underbrace{\frac{\vert t \vert \cdot \vert x \vert}{t}}_{=\pm 1} &= \underbrace{\langle f'(0), x\rangle}_{\mathclap{\text{feste Zahl in $\mathbb{R}$}}} + \underbrace{\frac{o(t)}{t}}_{\mathclap{\xrightarrow{t\to 0}0}} \quad\Rightarrow \text{\Lightning}
	\end{alignat*}
	
	\emph{Anschaulich:} Es gibt keine Tangentialebene an den Graph von $f$ in $(0, \vert 0 \vert )\in\mathbb{R}^{n\times 1}$.\\
	\emph{folglich:} $f$ stetig in $x_0$ $\cancel{\Rightarrow}$ $f$ \gls{diffbar} in $x_0$, d.h. Umkehrung von \propref{diffbar_impl_stetig} gilt i.A. nicht.
	
	\begin{hint}
		Es gibt stetige Funktion $f:\mathbb{R}\to \mathbb{R}$, die in keinem Punkt $x$ \gls{diffbar} ist (siehe Hildebrand, Analysis 1 S. 192 oder Königsberger Analysis 1, Kap. 9.11)
	\end{hint}

	\begin{center}\begin{tikzpicture}
		\begin{axis}[
		xmin=-5, xmax=5, xlabel=$x$,
		ymin=-5, ymax=5, ylabel=$y$,
		samples=400,
		axis y line=middle,
		axis x line=middle,
		]
		\addplot+[mark=none] {abs(x)};
		\addlegendentry{$\vert x \vert$}
		\end{axis}
	\end{tikzpicture}\end{center}
\end{example}

\begin{example}[Exponentialfunktion]
	Sei $f:K\to K$ mit $f(x) = e^x\;\forall x\in K$.\\
	$\Rightarrow$ $f$ ist \gls{diffbar} mit $f'(x_0) = e^{x_0}\;\forall x_0\in K = \mathbb{R}\lor K=\mathbb{C}$.
	
	Denn: nach \propref{lemma_13_10} ist \begin{align}
		& \proplbl{exp_limit_1} \lim\limits_{y\to 0} \frac{e^y - 1}{y} = 1 \text{ in } \mathbb{C} \\
		\notag \Rightarrow\;& \lim\limits_{y\to 0} \frac{e^{x_0 + y} - e^{x_0}}{y} = \lim\limits_{y\to 0} e^{x_0} \cdot \frac{e^y - 1}{y} = e^{x_0} \;\xRightarrow{\eqref{differentialquotient_prop}}\; \text{Beh.}
	\end{align}
	
	\begin{center}\begin{tikzpicture}
		\begin{axis}[
		xmin=-5, xmax=5, xlabel=$x$,
		ymin=-5, ymax=5, ylabel=$y$,
		samples=400,
		axis y line=middle,
		axis x line=middle,
		]
		\addplot+[mark=none] {e^x};
		\addlegendentry{$e^x$}
		\addplot+[mark=none, dashed] {e^x};
		\addlegendentry{$e^x$}
		\end{axis}
		\end{tikzpicture}\end{center}
\end{example}

\begin{example}[Sinus und Cosinus]
	$\sin, \cos: K\to K$ ($\mathbb{R}$ bzw. $\mathbb{C}$) $\forall x_0\in K$.
	
	Denn:{\zeroAmsmathAlignVSpaces
	\begin{align*}
		 \frac{\sin y}{y} = \frac{e^{iy} - e^{-iy}}{2iy} = \frac{1}{2}\cdot \left( \frac{e^{iy} - 1}{iy} + \frac{e^{-iy} - 1}{-iy} \right) \xrightarrow[\text{vgl. \eqref{exp_limit_1}}]{y\to 0} 1,
	\end{align*}}
	folglich {\zeroAmsmathAlignVSpaces*
	\begin{align*}
		\lim\limits_{y\to 0} \frac{\sin(x_0 + y) - \sin(x_0)}{y} &\overset{\star}{=} \lim\limits_{y\to 0} \frac{2}{y} \cos\left( x_0 + \frac{y}{2}\right) \cdot \sin \left( \frac{y}{2}\right) \marginnote{$\star$: Additionstheoreme} \\
		&= \lim\limits_{y\to 0} \frac{2}{y}\cdot\sin\left(\frac{y}{2}\right)\cdot\cos\left(x_0 + \frac{y}{2}\right)\\
		&= \cos x_0 \quad\forall x_0\in K
		\end{align*}}
		
	Analog für den Kosinus.
	
	\begin{center}\begin{tikzpicture}
		\begin{axis}[
		xmin=-5, xmax=5, xlabel=$x$,
		ymin=-5, ymax=5, ylabel=$y$,
		samples=400,
		axis y line=middle,
		axis x line=middle,
		]
		\addplot+[mark=none] {sin(deg(x))};
		\addlegendentry{$\sin(x)$}
		\addplot+[mark=none, dashed] {cos(deg(x))};
		\addlegendentry{$\cos(x)$}
		\end{axis}
		\end{tikzpicture}
		\begin{tikzpicture}
		\begin{axis}[
		xmin=-5, xmax=5, xlabel=$x$,
		ymin=-5, ymax=5, ylabel=$y$,
		samples=400,
		axis y line=middle,
		axis x line=middle,
		]
		\addplot+[mark=none] {cos(deg(x))};
		\addlegendentry{$\cos(x)$}
		\addplot+[mark=none, dashed] {- sin(deg(x))};
		\addlegendentry{$-\sin(x)$}
		\end{axis}
		\end{tikzpicture}\end{center}
\end{example}

\subsection{Rechenregeln}
\begin{*definition}
	Sei $f:D\subset K^n \to K^m$, $D$ offen.
	
	Falls $f$  \gls{diffbar} in allen $x_0\in D$, dann heißt $f$ \begriff{differenzierbar} auf $D$ und Funktion $f':D\to L(K^n, K^m)$ heißt \begriff{Ableitung} von $f$.
	
	Ist zusätzlich Funktion $f': D\to L(K^n, K^m)$ stetig, dann heißt Funktion $f$ \begriff{stetig differenzierbar} (auf $D$) bzw. \mathsymbol{C1}{$C^1$}\emph{-Funktion} (auf $D$).
	
	$C^1(D, K^m):= \left\lbrace f: D\to K^m \mid f \text{ stetig  \gls{diffbar} auf } D \right\rbrace$
\end{*definition}

\begin{example}
	\begin{enumerate}[label={\alph*)}]
		\item $f(x) = x^k\;\forall x\in\mathbb{R},\, k\in\mathbb{N}_{\ge 0}$ \\
		$\Rightarrow$ $f'(x) = k\cdot x^{k-1}\;\forall x\in \mathbb{R}$ \\
		$\Rightarrow$ offenbar stetige Funktion \\ $\Rightarrow$ $f\in C^1(\mathbb{R}, \mathbb{R})$
		
		\item $f(x) = e^x\;\forall x\in\mathbb{C}$ \\
		$\Rightarrow f'(x) = e^x \;\forall x\in\mathbb{C}$ stetig \\
		$\Rightarrow$ $f\in C^1(\mathbb{C},\mathbb{C})$
		
		\item $f(x) = \vert x \vert^2\;\forall x\in\mathbb{R}^n$ \\
		$\Rightarrow$ $f(x) = 2x\;\forall x\in\mathbb{R}^n$, offenbar stetig \\
		$\Rightarrow$ $f\in C^1(\mathbb{R}^n, \mathbb{R})$
	\end{enumerate}
\end{example}

\begin{example}
	\proplbl{ableitung_beipsiel_unstetige_ableitung}
	Sei $f:\mathbb{R}\to \mathbb{R}$ mit $f(0) = 0$, $f(x)=x^2\cdot \sin\left(\frac{1}{x}\right)$ $\forall x\neq 0$.
	
	\begin{center}\begin{tikzpicture}
		\begin{axis}[
		xmin=-5, xmax=5, xlabel=$x$,
		ymin=-5, ymax=5, ylabel=$y$,
		samples=400,
		axis y line=middle,
		axis x line=middle,
		]
		\addplot+[mark=none] {x^2*sin(deg(1/x))};
		\addlegendentry{$x^2\cdot \sin(\frac{1}{x})$}
		\end{axis}
		\end{tikzpicture}\end{center}
	
	Wegen \begin{align*}
		\frac{\vert x^2 \cdot \sin \frac{1}{x}\vert}{\vert x \vert} \le \vert x \vert \xrightarrow{x\neq 0} 0
	\end{align*}
	folgt{ \zeroAmsmathAlignVSpaces \begin{align*}
		& f(x) = o(\vert x \vert), x\to 0 \\
		\Rightarrow\;& f(x) = f(0) + 0\cdot (x - 0) + o(\vert x - 0\vert), x\to 0 \\
		\Rightarrow\;& f \text{  \gls{diffbar} in $x=0$ mit $f'(0) = 0$}
	\end{align*}}
	
	Rechenregeln liefern $x\neq 0$: \begin{align*}
		f'(x) = 2x\cdot\sin\frac{1}{x}- \cos\frac{1}{x} \quad\forall x\neq 0
	\end{align*}
	
	Für $x_k := \frac{1}{k\pi}$ gilt: \begin{align*}
		& \lim\limits_{k\to\infty} 2 x_k \cdot \sin \frac{1}{x_k} = 0,\; \lim\limits_{k\to\infty} \cos \frac{1}{x_k} = \pm 1 \\
		\Rightarrow\;& \lim\limits_{x\to 0} f'(x) \text{ existiert nicht} \\
		\Rightarrow\;& f\notin C^1(\mathbb{R}, \mathbb{R}),
	\end{align*}
	d.h. Ableitung einer stetigen Funktion muss \emph{nicht} stetig sein.
\end{example}

\begin{boldenvironment}[Man beobachtet]
	\hspace{0pt}
\begin{itemize}[topsep=-2pt]
	\item \propref{definition_ableitung} bzw. \propref{definition_ableitung_zwei_stern} sind häufig ungeeignet zum Bestimmen von $f'(x_0)$
	\item \propref{differentialquotient_prop} ist durchaus nützlich für konkrete Fälle im Fall $n=1$
	
	\emph{$\rightarrow$ Strategie:} Zurückführung auf einfachere Fälle durch Rechenregeln und Reduktion
\end{itemize}
\end{boldenvironment}

\begin{proposition}[Rechenregeln]
	\proplbl{ableitung_rechenregeln}
	Sei $D\in K^n$ offen, $f,g: D\to K^m$, $\lambda: D\to K$  \gls{diffbar} in $x_0\in D$ \\
	$\Rightarrow$ $(f\pm g): D\to K^m, (\lambda\cdot f):D\to K^m, (f\cdot g):D\to K$ sind  \gls{diffbar} in $x_0\in D$ und $\frac{1}{\lambda}:D\to K$ ist  \gls{diffbar} in $x_0$, falls $\lambda(x_0)\neq 0$
	mit
	\begin{enumerate}[label={\alph*)}]
		\item $(f\pm g)'(x_0) = f'(x_0) \pm g'(x_0)\in K^{m\times 1}$
		\item $(\lambda\cdot f)'(x_0) = \lambda (x_0)\cdot f'(x_0) + f(x_0)\cdot \lambda'(x_0)\in K^{m\times n}$
		\item $(f\cdot g)' (x_0) = \transpose{f(x_0)}\cdot g'(x_0) + \transpose{g(x_0)}\cdot f'(x_0)\in K^{m\times n}$
		\item $\left( \frac{1}{\lambda}\right)'(x_0) = - \frac{1}{\lambda(x_0)^2}\cdot \lambda'(x_0)\in K^{1\times n}$
	\end{enumerate}
\end{proposition}

\begin{conclusion}
	\proplbl{ableitung_quotientenregel}
	Seien $\lambda$, $\mu:D\to K$  \gls{diffbar} in $x_0$, $D$ offen und $\lambda(x_0)\neq 0$ \\
	$\Rightarrow$ $\left( \frac{\mu}{\lambda} \right): D\to K$  \gls{diffbar} in $x_0$ mit \begin{align*}
		\left( \frac{\mu}{\lambda} \right)' (x_0) = \frac{\lambda(x_0)\cdot \mu'(x_0) - \mu(x_0) \cdot \lambda'(x_0)}{\lambda(x_0)^2}\in K^{1\times n}
	\end{align*}
\end{conclusion}

\begin{proof}[\propref{ableitung_quotientenregel}]
	Setzte in \propref{ableitung_rechenregeln} $f=\mu$ (d.h. $m=1$) und betr. Produkt $\frac{1}{\lambda}\cdot \mu$.
\end{proof}

\begin{proof}[\propref{ableitung_rechenregeln}]
	Nach \propref{definition_ableitung_proposition} c) existieren $P,Q: D\to L(K^n, K^m)$, $\Lambda:D\to L(K^n, K)$ mit
\begin{itemize}[topsep=\dimexpr -\baselineskip / 3\relax]
	\item $f(x) = f(x_0) + P(x) \cdot (x - x_0)$, $\lim\limits_{x\to x_0} P(x) = f'(x_0)$
	\item $g(x) = g(x_0) + Q(x) \cdot (x - x_0)$, $\lim\limits_{x\to x_0} Q(x) = g'(x_0)$
	\item $\lambda(x) = \lambda(x_0) + \Lambda(x_0) \cdot (x - x_0)$, $\lim\limits_{x\to x_0} \Lambda(x) = \lambda'(x_0)$
\end{itemize}
und mit \propref{definition_ableitung_proposition} c) ergibt sich die Behauptung wie folgt:
\begin{enumerate}[label={\alph*)}]
	\item $f(x) + g(x) = f(x_0) + g(x_0) + \underbrace{\big( P(x) + Q(x) \big)}_{\mathclap{x\to x_0:\; f'(x_0) + g'(x_0)\in L(K^n, K^m)}}\cdot (x - x_0)$ \\
	$\Rightarrow$ Behauptung
	\item $\lambda(x)\cdot f(x) = \lambda(x_0) \cdot f(x_0) + \underbrace{\left[ \lambda(x_0) \cdot P(x) + \underbrace{f(x_0)\cdot \Lambda(x)}_{\in K^{m\times n}} + \underbrace{\Lambda(x_) \cdot (x - x_0)}_{\in K} \cdot P(x) \right]}_{\xrightarrow{x\to x_0}\lambda(x_0)\cdot f'(x_0) + f(x_0)\cdot \lambda'(x_0)\in L(K^m, K^n)} \cdot (x - x_0)$ \\
	$\Rightarrow$ Behauptung
	\item analog zu b)
	\item $\frac{1}{\lambda(x)} = \frac{1}{\lambda(x_0)} - \frac{\lambda(x) - \lambda(x_0)}{\lambda(x_0)\cdot \lambda(x)} = \frac{1}{\lambda(x_0)} + \underbrace{\left( - \frac{1}{\lambda(x_0)\cdot\lambda(x)}\cdot \Lambda(x)\right)}_{\mathclap{\xrightarrow{x\to x_0}-\frac{1}{\lambda(x_0)^2}\cdot \lambda'(x_0)\in L(K^n, K)}} (x - x_0)$\\
	$\Rightarrow$ Behauptung
\end{enumerate}
\end{proof}

\begin{example}
	Sei $f:D\in K^n\to K^m$, $c\in K$, $f$  \gls{diffbar} in $x_0\in D$\\
	$\xRightarrow{\ref{ableitung_rechenregeln}\ b)} (c\cdot f) = c\cdot f'(x_0)$ (da $c$ konst. Funktion $D\to K$)
\end{example}

\begin{example}[Polynom]
	Sei $f:K\to K$, Polynom $f(x) = \sum\limits_{l=0}^{k}a_l x^l$ \\
	$\Rightarrow$ $f$  \gls{diffbar} $\forall x_0\in K$ mit $f'(x_0) = \sum\limits_{l=1}^k l a_l x_0^{l-1}$
\end{example}

\begin{example}
	Sei $f=\frac{f_1}{f_2}$ rationale Funktion auf $\mathbb{R}$ (d.h. $f_1, f_2:K\to K$ Polynom) \\
	$\Rightarrow$ $f$ ist  \gls{diffbar} auf $K\setminus \{ \text{Nullstellen von }f_2 \}$
\end{example}

\begin{example}[Tangens und Cotangens]
	\proplbl{ableitung_beispiel_tangens}
	$\tan: K\setminus \{ \frac{\pi}{2} + k\cdot \pi \mid k\in\mathbb{Z} \}\to K$, $\cot:K\setminus \{ k\cdot \pi \mid k\in\mathbb{Z} \} \to K$ \\[\dimexpr - \baselineskip / 2 \relax]
	\zeroAmsmathAlignVSpaces \begin{alignat*}{3}
	\xRightarrow{\text{Quotientenregel}}&\;\;& \tan'(x_0)&= \frac{\sin'(x_0)\cos (x_0) - \cos (x_0) \cdot \sin(x_0)}{\left( \cos(x_0)\right)^2} &&\\
	&& &= \frac{\cos^2(x_0) + \sin^2(x_0)}{\cos^2(x_0)} = \frac{1}{\cos^2(x_0)} && \forall x_0\in \text{ Definitionsbereich} \\
	&& \cot'(x_0) &= - \frac{1}{\sin^2(x_0)}&&\forall x_0\in\text{ Definitionsbereich}
	\end{alignat*}
	\begin{center}\begin{tikzpicture}
		\begin{axis}[
		xmin=-5, xmax=5, xlabel=$x$,
		ymin=-5, ymax=5, ylabel=$y$,
		samples=400,
		axis y line=middle,
		axis x line=middle,
		restrict y to domain=-5:5
		]
		\addplot+[mark=none] {tan(deg(x))};
		\addlegendentry{$\tan(x)$}
		\addplot+[mark=none, dashed] {1/(cos(deg(x)))^2};
		\addlegendentry{$\frac{1}{\cos^2(x)}$}
		\end{axis}
		\end{tikzpicture}
		\begin{tikzpicture}
		\begin{axis}[
		xmin=-5, xmax=5, xlabel=$x$,
		ymin=-5, ymax=5, ylabel=$y$,
		samples=400,
		axis y line=middle,
		axis x line=middle,
		restrict y to domain=-5:5
		]
		\addplot+[mark=none] {cot(deg(x))};
		\addlegendentry{$\cot(x)$}
		\addplot+[mark=none, dashed] {-1/(sin(deg(x))^2)};
		\addlegendentry{$-\frac{1}{\sin^2(x)}$}
		\end{axis}
		\end{tikzpicture}\end{center}
\end{example}

\begin{proposition}[Kettenregel]
	\proplbl{ableitung_kettenregel}
	Sei $f:D\subset K^n\to K^m$, $g:\tilde{D}\subset K^m\to K^l$, $D$,$\tilde{D}$ offen, $f$  \gls{diffbar} in $x_0\in D$, $g$  \gls{diffbar} in $f(x_0)\in\tilde{D}$ \\
	$\Rightarrow$ $g\circ f: D\to K^l$  \gls{diffbar} in $x_0$ mit $(g\circ f)' = g'(f(x))\cdot f'(x)$ ($\in K^{l\times n}$)
\end{proposition}

\begin{proof}
	Nach \propref{definition_ableitung_proposition} c) exisitert $P:D\to L(K^n, K^m)$, $Q:\tilde{D}\to K(K^m, K^l)$ mit
	\zeroAmsmathAlignVSpaces**
	\begin{alignat}{2}
	\proplbl{ableitung_kettenregel_beweis_f} && f(x) &= f(x_0) + P(x)(x - x_0), \lim\limits_{x\to x_0} P(x) = f'(x_0) \\
	\proplbl{ableitung_kettenregel_beweis_g} && g(y) &= g(f(x_0)) + Q(y)(y - f(x_0)), \lim\limits_{y\to f(x_0)} Q(y) = g'(f(x_0)) \\
	\notag \Rightarrow\quad && (g\circ f)(x) &= g(f(x)) \overset{\eqref{ableitung_kettenregel_beweis_g}}{=} g(f(x_0)) + Q(f(x)(f(x) - f(x_0)) \\
	\notag&& &= (g\circ f)(x_0) + \underbrace{\left[ Q(f(x)) \cdot P(x)\right]}_{\mathclap{\xrightarrow{x\to x_0}g'(f(x_0))\cdot f'(x_0)}} (x - x_0)
	\end{alignat}
	$\xRightarrow{\ref{definition_ableitung_proposition}\,c)}$ Behauptung
\end{proof}

\begin{example}[$x$ im Exponenten]
	\proplbl{ableitung_beispiel_exponentialfunktion}
	Sei $f:\mathbb{R}\to \mathbb{R}$, $f(x) = a^x$ ($a\in\mathbb{R}_{\ge 0}$, $a\neq 1$).
	
	Offenbar $a^x = \left(e^{\ln a}\right)^x = e^{x\cdot \ln a}$\\
	$\Rightarrow$ $f(x) = g(h(x))$ mit $g(y) = e^y$, $h(x) = x\cdot \ln a$
	
	Wegen $g'(y) = e^y$ $\forall y\in\mathbb{R}$, $h'(x) = \ln a$ $\forall x\in K$ \\[\dimexpr -\baselineskip / 2 \relax]
	\zeroAmsmathAlignVSpaces \begin{alignat*}{2}
	\xRightarrow{\text{\propref{ableitung_kettenregel}}}\quad&& f'(x_0) &= g'(x_0\cdot \ln a)\cdot f'(x_0) = e^{x_0\cdot \ln a}\cdot \ln a = a^x\cdot \ln a \quad\forall x_0\in\mathbb{R}
	\end{alignat*}
	\begin{center}\begin{tikzpicture}
		\begin{axis}[
		xmin=-5, xmax=5, xlabel=$x$,
		ymin=-5, ymax=5, ylabel=$y$,
		samples=400,
		axis y line=middle,
		axis x line=middle,
		]
		\addplot+[mark=none] {2^x};
		\addlegendentry{$2^x$}
		\addplot+[mark=none, dashed] {2^x*ln(2)};
		\addlegendentry{$2^x\cdot \ln(2)$}
		\end{axis}
		\end{tikzpicture}\end{center}
\end{example}

\begin{example}[Logarithmus]
	\proplbl{ableitung_beispiel_logarithmus}
	Sei $f:\mathbb{R}_{>0} \to \mathbb{R}$, $f(x) = \log_a x$ ($a\in\mathbb{R}_{>0}\setminus\{1\}$)
	
	Fixiere $x_0\in \mathbb{R}_{>0}$, sei $\{x_n\}$ beliebige Folge in $\mathbb{R}_{>0}$ mit $x_n\to x_0$
	\zeroAmsmathAlignVSpaces*[5pt] \begin{alignat*}{1}
		\xRightarrow{\text{$f$ stetig}}\;& y:= \log_a x_n \to \log_a x_0 =: y_0 \\
		\Rightarrow\;\;&\!\! \lim\limits_{n\to \infty} \frac{f(x_n) - f(x_0)}{x_n - x_0} = \lim\limits_{n\to \infty} \frac{\log_a(x_n) - \log_a(x_0)}{a^{\log_a(x_n)} - a^{\log_a(x_0)}} = \lim\limits_{n\to \infty} \dfrac{1}{\frac{a^{y_n} - a^{y_0}}{y_n - y_0}} \overset{\ref{ableitung_beispiel_exponentialfunktion}}{=}{\frac{1}{a^{y_0}\cdot \ln(a)}} \\
		\xRightarrow{\{x_n\}\text{ bel.}}\;& f'(x_0) = \frac{1}{x_0\cdot \ln a}\quad\forall x>0
	\end{alignat*}
	
	Spezialfall: $(\ln(x))' = \frac{1}{x}$ $\forall x>0$
	\begin{center}\begin{tikzpicture}
		\begin{axis}[
		xmin=-5, xmax=5, xlabel=$x$,
		ymin=-5, ymax=5, ylabel=$y$,
		samples=400,
		axis y line=middle,
		axis x line=middle,
		]
		\addplot+[mark=none] {log2(x)};
		\addlegendentry{$\log_2(x)$}
		\addplot+[mark=none, dashed] {1/(x*ln(2))};
		\addlegendentry{$\frac{1}{x\cdot\ln(2)}$}
		\end{axis}
		\end{tikzpicture}\end{center}
\end{example}

\begin{example}
	Sei $f:\mathbb{R}_{>0}\to \mathbb{R}$, $f(x) = x^r$ ($r\in\mathbb{R}$)
	
	Wegen $x^r = e^{r\cdot \ln x}$ liefert Kettenregeln (analog zu \propref{ableitung_beispiel_exponentialfunktion}) \begin{align*}
		f'(x_0) = \frac{r\cdot e^{r\cdot \ln x_0}}{x_0} = \frac{r\cdot x_0^r}{x_0} = r\cdot x_0^{r - 1} \quad\forall x_0>0
	\end{align*}
	
	Spezialfall: $f(x) = \frac{1}{x^k}$ $\Rightarrow$ $f'(x) = - \frac{k}{x^{k+1}}$
	
	Zu \propref{ableitung_beipsiel_unstetige_ableitung}:\begin{align*}
		f'(x) = 2x\cdot \sin\frac{1}{x} + x^2\cdot \cos\frac{1}{x} \cdot \left( - \frac{1}{x^2}\right) = 2x\cdot \sin\frac{1}{x} - \cos\frac{1}{x}
	\end{align*}
\end{example}

\begin{proposition}[Reduktion auf skalare Funktionen]
	\proplbl{ableitung_proposition_reduktion}
	Sei $f=(f_1, \dotsc, f_m): D\subset K^n\to K^m$, $D$ offen, $x_0\in D$. Dann gilt:\begin{center}
		$f$  \gls{diffbar} in $x_0$ $\Leftrightarrow$ alle $f_j$  \gls{diffbar} in $x_0$ $\forall j=1,\dotsc,m$
	\end{center}

	Im Fall der Differenzierbarkeit hat man: \begin{align}
		\proplbl{ableitung_jacobimatrix}
		f'(x_0) = \begin{pmatrix}
			f_1'(x_0) \\
			\vdots \\
			f_m'(x_0)
		\end{pmatrix} \in K^{m\times n}
	\end{align}
\end{proposition}
\smiley{} Wenn Sie das nächste mal aus der Disko kommen, zuviel getrunken haben und den Namen 
ihrer Freundin nicht mehr kennen, sollten sie sich daran aber noch erinnern: \smiley{} \\

\begin{remark}
	Mit \propref{ableitung_proposition_reduktion} kann man die Berechnungen der Ableitungen stets auf skalare Funktionen $f:D\subset K^n\to K$ zurückführen. Die Matrix in \propref{ableitung_jacobimatrix} besteht aus $m$ Zeilen $f_j'(x_0)\in K^{1\times m}$.
\end{remark}

\begin{example}
	Sei $f:\mathbb{R}\to \mathbb{R}^2$ mit \begin{align*}
		f(t) &= \begin{pmatrix}
			t\cdot \cos( 2\pi t) \\ t\cdot \sin(2\pi t)
		\end{pmatrix}, & f'(t) &= \begin{pmatrix}
			\cos(2\pi t) - t\cdot \sin(2\pi t)\cdot 2\pi \\ \sin(2\pi t)+ t\cdot\cos(2\pi t)\cdot 2\pi
		\end{pmatrix} \in \mathbb{R}^{2\times 1},
	\end{align*}
	und $f'(0) = \binom{1}{0}$, $f'(1) = \binom{1}{2\pi}$.
\end{example}

\begin{lemma}
	\proplbl{ableitung_spezialfall_reduktion_proposition}
	Sei $f=(f_1, f_2):D\subset K^n\to K^k\times K^l$, $D$ offen, $x_0\in D$.
	
	Funktion $f$ ist  \gls{diffbar} in $x_0$ genau dann, wenn $f_1:D\to K^k$ und $f_2 :D\to K^l$  \gls{diffbar} in $x_0$.
	
	Im Falle der Differenzierbarkeit gilt\begin{align}
		\proplbl{ableitung_spezialfall_reduktion}
		f'(x_0) = \begin{pmatrix}
			f_1'(x_0) \\ f_2'(x_0)
		\end{pmatrix} \in K^{(k+l)\times n}
	\end{align}
	
	\begin{hint}
		Da $K^k\times K^l$ mit $K^{k+l}$ identifiziert werden kann, kann man $f$ auch als Abbildung von $D$ nach $K^{k+l}$ ansehen. Dementsprechend kann die Matrix in \propref{ableitung_spezialfall_reduktion} der Form \begin{align*}
			\begin{pmatrix}
				(k\times n) \text{ Matrix} \\
				(l\times n) \text{ Matrix}
			\end{pmatrix}
		\end{align*}
		auch als $((k+l)\times n)$-Matrix aufgefasst werden.
	\end{hint}
\end{lemma}

\begin{proof}\hspace*{0pt}
	\NoEndMark
	\begin{itemize}[topsep=\dimexpr - \baselineskip / 3\relax]
		\item["`$\Rightarrow$"'] Man hat
		\zeroAmsmathAlignVSpaces[3pt][3pt]
		\begin{alignat}{2}
				\proplbl{ableitung_beweis_lemma_spezialfall_reduktion} && f(x) &= f(x_0) + f'(x_0)\cdot(x - x_0) + R(x)\cdot (x - x_0), \, \;R(x) \xrightarrow{x\to x_0}0
			\intertext{da $f'(x_0)$, $R(x)\in L(K^n, K^k\times K^l)$}
				\notag\Rightarrow&\;\;& f'(x_0) &= (A_1, A_2), \; R(x) = \big( R_1(x), R_2(x) \big))
			\intertext{mit $A_1, R_1(x)\in L(K^n, K^k)$, $A_2, R(x)\in L(K^n, K^l)$}
				\proplbl{ableitung_beweis_lemma_spezialfall_reduktion_einzelableitung} \xRightarrow{\eqref{ableitung_beweis_lemma_spezialfall_reduktion}}&& f_j(x)&= f_j(x_0) + A_j \cdot (x - x_0) + R_j(x) (x - x_0),\;R_j(x)\xrightarrow{x\to x_0}0 \\
				\notag\Rightarrow&& f_j & \text{ ist  \gls{diffbar} in $x_0$ mit $f_j'(x_0) = A_j$, $j=1,2$}
		\end{alignat}
		$\Rightarrow$ Behauptung
		\item["`$\Leftarrow$"'] (es gilt auch \eqref{ableitung_beweis_lemma_spezialfall_reduktion_einzelableitung} mit $A_j = f_j'(x_0)$)
		
		Setzte \begin{align*}
		 &A=\begin{pmatrix}
			f_1'(x) \\ f_2'(x)
		\end{pmatrix},\; R(x) = \begin{pmatrix}
			R_1(x) \\ R_2(x)
		\end{pmatrix} \\
		\xRightarrow{\eqref{ableitung_beweis_lemma_spezialfall_reduktion_einzelableitung}}\;& A, R(x)\in L(K^n, K^k\times K^l) \\
		\xRightarrow{\text{mit }A_j=f_j'(x_0)}\; & f(x)= f(x_0) + A(x - x_0) + R(x)(x - x_0), R(x)\xrightarrow{x\to x_0}0
		\end{align*}
		$\Rightarrow$ $f$  \gls{diffbar} in $x_0$ und \eqref{ableitung_spezialfall_reduktion} gilt.\hfill\csname\InTheoType Symbol\endcsname
	\end{itemize}
\end{proof}

\begin{proof}[\propref{ableitung_proposition_reduktion}]
	Mehrfache Anwendung von \propref{ableitung_spezialfall_reduktion_proposition} (z.B. mit $k=1, l = m - j$ für $j=1,\dotsc, m-1$)
\end{proof}
\section{Richtungsableitung und partielle Ableitung}\proplbl{richtungsableitung}  \setcounter{equation}{0}
Sei $f:D\subset K^n\to K^m$, $D$ offen, $x\in D$.

\begin{underlinedenvironment}[Ziel]
	Zurückführung der Berechnung der Ableitung $f(x)$ auf die Berechnung der Ableitung für Funktionen $\tilde{f}:\tilde{D}\subset K\to K$
	\begin{itemize}
		\item Reduktionssatz $\Rightarrow$ man kann sich bereits auf $m=1$ einschränken
		\item für Berechnung der Ableitung von $f$ ist neben den Rechen- und Kettenregeln auch der Differentialquotient verfügbar
	\end{itemize}
\end{underlinedenvironment}

\begin{underlinedenvironment}[Idee]
	Betrachte $f$ auf Geraden $t\to x + t\cdot z$ durch $x$ $\Rightarrow$ skalares Argument $t$, $t\in K$ $\Rightarrow$ Differentialquotient.
	
	Spezialfall: $z = e_j$ $\Rightarrow$ Partielle Ableitung
\end{underlinedenvironment}

\begin{*definition}
	Sei $f:D\subset K^n\to K^m$, $D$ offen, $x\in D$, $z\in K^n$.
	
	Falls $a\in L(K, K^m)$ ($\cong K^m$) existiert mit\begin{align}
		\proplbl{richtungsableitung_definition}
		f(x + t\cdot z) = f(x) + t\cdot a + o(t),\;t\to 0,\; t\in K,
	\end{align}
	dann heißt $f$ \gls{diffbar} in $x$ \begriff[differenzierbar!]{in Richtung $z$} und \mathsymbol{Dz}{$\mathrm{D}_z$}$f(x) := a$ heißt \begriff{Richtungsableitung} von $f$ in $x$ in Richtung $z$ (andere Bezeichnungen: $f(x; z)$, $\partial_z f(x)$, $\frac{\partial f}{\partial z}(x)$, $\partial f(x,z)$, $\dotsc$)
\end{*definition}
\begin{*remark}
	\begin{itemize}[topsep=\dimexpr -\baselineskip / 2\relax]
		\item Wegen $B_\epsilon(x)\subset D$ für ein $\epsilon > 0$ existiert $\tilde{\epsilon}$ mit $x + t\cdot z \in D$ $\forall t\in B_{\tilde{\epsilon}} (0) \subset K$
		\item $f'(x;0)$ existiert offenbar stehts für $z=0$ mit $f'(x;0) = 0$
	\end{itemize}
\end{*remark}

\begin{proposition}
	\proplbl{richtungsableitung_prop_equivalente_definition}
	Sei $f:D\subset K^n\to K^m$, $D$ offen, $x\in D$, $z\in K^n$. Dann:
	\begin{align}
		\notag &\text{$f$ \gls{diffbar} in $x$ in Richtung $z$ mit $\mathrm{D}_z f(x)\in L(K, K^m)$} \\
		\proplbl{richtungsableitung_definition_prop_eins}
		\Leftrightarrow\;\; & \text{für }\phi(t) = f(x + t\cdot z) \text{ existiert }\phi'(0) \text{ und } \mathrm{D}_z f(x) = \phi'(0) \\
		\proplbl{richtungsableitung_definitnion_prop_zwei}
		\Leftrightarrow\;\; & \lim\limits_{t\to 0} \frac{f(x + t\cdot z) - f(x)}{t} = a \;(\in L(K, K^m)) \text{ existiert und } \mathrm{D}_z f(x) = a
	\end{align}
\end{proposition}

\begin{example}
	Sei $f:\mathbb{R}^2\to\mathbb{R}$ mit $f(x) = x_1^2 + \vert x_2\vert$. Existiert eine Richtungsableitung in $x=(x_1, 0)$ in Richtung $z=(z_1, z_2)$?
	
	Sei $\phi(t) := f(x + t\cdot z) = (x_1 + t\cdot z_1)^2 + \vert t\cdot z_2\vert = \underbrace{x_1^2 + 2t\cdot x_1 z_1 + t^2 z_1^2}_{=\phi_1(t)} + \underbrace{\vert t \vert \cdot \vert z_2 \vert} _{=\phi_2(t)}$
	
	$\Rightarrow$ $\phi_1'(0) = 2\cdot x_1 z_1$ existiert $\forall x_1, z_1\in\mathbb{R}$ \\
	\phantom{$\Rightarrow$} $\phi_2'(0) = 0$ existiert \emph{nur} für $z_2 = 0$ (vgl. \propref{ableitung_beispiel_betrag}) \\
	$\Rightarrow$ $\phi_1'(0) = 2x_1z_1$ existiert \emph{nur} für $x_1$, $z_1\in\mathbb{R}$, $z_2 = 0$ \\
	$\xRightarrow{\eqref{richtungsableitung_definition_prop_eins}}$ Richtungsableitung von $f$ existiert für alle $ x = (x_1, 0)$ \emph{nur} in Richtung $z=(z_1, 0)$ mit $\mathrm{D}_z f(x) = 2x_1 z_1$
\end{example}

\begin{underlinedenvironment}[Frage]
	Existiert $\mathrm{D}_z f(x)$ $\forall z$, falls $f$ \gls{diffbar} in $x$?
\end{underlinedenvironment}

\begin{proposition}
	\proplbl{richtungsableitung_prop_existenz_prop}
	Sei $f:D\subset K^n\to K^m$, $D$ offen, $f$ \gls{diffbar} in $x\in D$.\\
	$\Rightarrow$ Richtungsableitung $\mathrm{D}_z f(x)$ existiert $\forall z\in K^n$ und \begin{align}
		\proplbl{richtungsableitung_prop_existenz}
		\mathrm{D}_z f(x) = f'(x) \cdot z \;(\in K^{m\times 1})
	\end{align}
	\emph{Hinweis:} Richtungsableitung ist linear in $z$!
\end{proposition}

\begin{proof}
	\NoEndMark
	$f$ \gls{diffbar} in $x$ \\
	\begin{tabularx}{\linewidth}{r@{\ \ }X}
		$\Rightarrow$ &$f(y) = f(x) + f'(x) (y - x) + o(\vert y - x\vert)$, $y\to x$ \\
		$\xRightarrow{y=x+t\cdot z}$& $f(x + tz) = f(x) + t\cdot f'(x)\cdot z + o(t)$, $t\to 0$ \\
		$\xRightarrow{\eqref{richtungsableitung_definition}}$&  Behauptung\hfill\csname\InTheoType Symbol\endcsname
	\end{tabularx}
\end{proof}

\begin{example}
	\proplbl{richtungsableitung_example_euklidische_norm}
	Betrachte $f:\mathbb{R}^n\to \mathbb{R}$ mit $f(x) = \vert x \vert ^2$ $\forall x$
	\begin{enumerate}[label={\alph*)}]
		\item Es gilt \zeroAmsmathAlignVSpaces \begin{alignat*}{2}
		 && \phi(t) &= \vert x + tz\vert ^2 = \sum_{i=1}^{n} (x_i + t z_i)^2 = \sum_{i=1}^n x_i^2 + 2t x_i z_i + t^2 z_i^2 \\
		 \Rightarrow && \phi'(t) &= \sum_{i=1}^n 2x_i z_i + 2t z_i^2 \\
		\xRightarrow{\eqref{richtungsableitung_definition_prop_eins}} &\;\;& \phi'(0) &= 2\sum_{i=1}^n x_i z_i = 2 \langle x,z\rangle = \mathrm{D}_z f(x)\quad\forall x, z\in\mathbb{R}^n
		\end{alignat*}
		\item \propref{ableitung_beispiel_euklidische_norm} liefert $f'(x) = 2x$ $\forall x\in\mathbb{R}^n$ \\
		$\xRightarrow{\eqref{richtungsableitung_prop_existenz}} $ $\mathrm{D}_z f(x) = 2x\cdot z = 2 \langle x,z\rangle$ $\forall x,z\in\mathbb{R}^n$
	\end{enumerate}
	folglich gilt: $\vert z \vert = 1$ und $x\in\mathbb{R}^n$ fest \begin{itemize}
		\item $\mathrm{D}_z f(x) = 0$ $\Leftrightarrow$ $x\perp z$
		\item $\mathrm{D}_z f(x) = \,$maximal ($x$ fest) $\Leftrightarrow$ $z = \frac{x}{\vert x \vert}$
	\end{itemize}
\end{example}

\subsection{Anwendung: Eigenschaften des Gradienten}
\begin{*definition}
	Sei $f:D\subset\mathbb{R}^n\to \mathbb{R}$, $D$ offen, $f$ \gls{diffbar} in $x\in D$.
	
	$N_C:= \{ y\in D \mid f(x) = f(y) \}$ heißt \begriff{Niveaumenge} von $f$ für $x\in \mathbb{R}$.

\end{*definition}	

\begin{*definition}
	Sei $\gamma: (-\delta, \delta)\to N_C$ ($\delta > 0$) Kurve mit $\gamma(0) = 0$, $\gamma$ \gls{diffbar} in $0$.
	
	Ein $z\in\mathbb{R}\setminus \{0\}$ mit $z = \gamma'(0)$ für eine derartige Kurve $\gamma$ heißt \begriff{Tangentialvektor} an $N_C$ in $x$.
	
	Offenbar gilt \zeroAmsmathAlignVSpaces
	\begin{align}
	 \notag & \phi(t) = f(\gamma(t)) = c \\
	 \notag \Rightarrow\;\;& \phi'(0) = f'(\gamma(0))\cdot \gamma'(0) = 0 \\
	 \proplbl{richtungsableitung_tangentialvektor_eigenschaft}
	 \Rightarrow\;\; &\mathrm{D}_{\gamma'(0)} f(x) \overset{\star}{=} \langle f'(x), \gamma'(0)\rangle = 0\marginnote{\star: vgl. \propref{richtungsableitung_prop_existenz_prop}}[\dimexpr -\baselineskip / 2 \relax]
	 \end{align}
\end{*definition}

\begin{proposition}[Eigenschaften des Gradienten]
	\proplbl{richtungsableitung_gradient_eigenschaften}
	Sei $f:D\subset\mathbb{R}^n\to\mathbb{R}$, $D$ offen, $f$ \gls{diffbar} in $x\in D$. Dann:
\begin{enumerate}[label={\arabic*)}]
	\item Gradient $f'(x)$ steht senkrecht auf der Niveaumenge $N_{f(x)}$, d.h. $\langle f'(x), z\rangle = 0$ $\forall$ Tangentialvektoren $z$ an $N_{f(x)}$ in $x$
	\item Richtungsableitung $\mathrm{D}_z f(x) = 0$ $\forall$ Tangentialvektoren $z$ an $N_{f(x)}$ in $x$
	\item Gradient $f(x)$ zeigt in Richtung des steilsten Anstieges von $f$ in $x$ und $\vert f'(x)\vert$ ist der steilste Anstieg, d.h. falls $f'(x)\neq 0$ gilt für Richtung $\tilde{z} := \frac{f'(x)}{\vert f'(x)\vert}$ \begin{align*}
		D_{\tilde{z}} f(x) = \max \left\lbrace \mathrm{D}_z f(x) \in\mathbb{R} \mid z\in\mathbb{R}^n \text{ mit } \vert z \vert = 1 \right\rbrace = \vert f(x)\vert
	\end{align*}
	
	(beachte: \person{euklid}ische Norm wichtig!)
\end{enumerate}
\end{proposition}

\begin{proof}\hspace*{0pt}
	\begin{enumerate}[label={\arabic*)},topsep=\dimexpr -\baselineskip / 2 \relax]
		\item folgt direkt aus \eqref{richtungsableitung_tangentialvektor_eigenschaft},\eqref{richtungsableitung_prop_existenz}
		\item analog oben
		\item für $\vert z \vert = 1$ gilt
		\zeroAmsmathAlignVSpaces \begin{align*}
			&\mathrm{D}_z f(x) = \langle f'(x), z \rangle = \vert f'(x) \vert \langle \tilde{z},z\rangle \\
			\overset{\star}{\le}\; &\vert f'(x) \vert  \vert \tilde{z}\vert \vert z \vert = \vert f'(x)\vert = \frac{\langle f'(x), f'(x)\rangle}{\vert f'(x) \vert} = \langle f'(x), \tilde{z} \rangle \overset{\eqref{richtungsableitung_prop_existenz}}{=} \mathrm{D}_{\tilde{z}}f(x)\marginnote{\star:\person{Cauchy} - \person{Schwarz}}
		\end{align*}
		$\Rightarrow$ Behauptung
	\end{enumerate}
\end{proof}

\begin{underlinedenvironment}[Feststellung]
	für $f:D\subset K^n\to K^m$: die lineare Abbildung $f'(x):K^n\to K^m$ ist durch Kenntnis für $n$ linear unabhängige Vektoren bestimmt\\
	$\xRightarrow{\eqref{richtungsableitung_prop_existenz}}$ $f'(x)$ eindeutig bestimmt durch Kenntnis von \begin{align*}
		\mathrm{D}_{e_j} f(x) = f'(x) \cdot e_j \;(\in K^{m\times 1}) \text{ für } j = 1,\dotsc,n
	\end{align*}
\end{underlinedenvironment}

\begin{*definition}
	Sei $f:D\subset K^n\to K^m$, $D$ offen, $x\in D$ (nicht notwendigerweise \gls{diffbar} in $x$).
	
	Falls Richtungsableitung $D_{e_j} f(x)$ existiert, heißt $f$ \begriff{partiell \gls{diffbar}} bezüglich $x_j$ im Punkt $x$ und $D_{e_j} f(x)$ heißt \begriff{partielle Ableitung} von $f$ bezüglich $x_j$ in $x$.
	
	\emph{Bezeichnung:} $\frac{\partial }{\partial z}f(x), \frac{\partial f}{\partial x_j}(x), \mathrm{D}_j f(x), f_{x_j}(x), \dotsc$
\end{*definition}

Wegen $f(x + t e_j) = f(x_1, \dotsc, x_{j-1}, x_j + t, x_{j+1}, \dotsc, x_n)$ liefert \propref{richtungsableitung_prop_equivalente_definition}:
\begin{conclusion}
	Sei $f:D\subset\mathbb{R}^n\to K^m$, $D$ offen. Dann:	\zeroAmsmathAlignVSpaces\begin{align}
		\notag & f \text{ ist partiell \gls{diffbar} bezüglich $x_j$ in $x$ mit Ableitung $\frac{\partial}{\partial x_j}f(x)$} \\
		\Leftrightarrow\;\; & \lim\limits_{t\to 0} \frac{f(x_1, \dotsc, x_{j-1}, x_j, x_{j+1}, \dotsc, x_n) - f(x_1, \dotsc, x_j, \dotsc, x_n)}{t} = a \text{ existiert}\\
		\notag & \text{ und } \frac{\partial }{\partial x_j}f(x) = a
	\end{align}
\end{conclusion}

\begin{remark}
	Zur Berechnung von $\frac{\partial}{\partial x_j} f(x)$ differenziert man skalare Funktionen \\ $x_j\to f(x_1, \dotsc, x_j, \dotsc, x_n)$ (d.h. alle $x_k$ mit $k\neq j$ werden als Parameter angesehen).
\end{remark}

\begin{example}
	Sei $f:\mathbb{R}^3 \to \mathbb{R}$ mit $f(x_1, x_2, x_3) = x_1^2 \sin x_2 + e^{x_3 - x_1}$, damit \begin{align*}
		\frac{\partial}{\partial x_1}f(x) &= 2x_1 \sin x_2 - e^{x_3 - x_1} & \frac{\partial}{\partial x_2} &= f(x) = x_1^2 \cos x_2 & \frac{\partial}{\partial x_3} f(x) &= e^{x_3 - x_1}
	\end{align*}
\end{example}

\begin{conclusion}
	\proplbl{richtungsableitung_prop_partielle_ableitung_ausrechnen}
	Sei $f:D\subset K^n\to K^m$, $D$ offen, $f$ \gls{diffbar} in $x\in D$ \zeroAmsmathAlignVSpaces  \begin{align}
	\proplbl{richtungsableitung_partielle_ableitung_ausrechnen}
	\Rightarrow \;\; D_z f(x) = \sum_{j=1}^n z_j \frac{\partial}{\partial x_j} f(x) \quad \forall z = (z_1, \dotsc, z_n)\in\mathbb{R}
	\end{align}
\end{conclusion}

\begin{proof}
	\NoEndMark
	\eqref{richtungsableitung_prop_existenz} liefert \zeroAmsmathAlignVSpaces\begin{align*}
		D_z f(x) = f'(x) \cdot z = f'(x) \cdot \sum_{j=1}^n z_j \cdot e_j = \sum_{j=1}^n z_j \left(f'(x)\cdot e_j\right) = \sum_{j=1}^n z_j \frac{\partial}{\partial x_j} f(x)\tag*{\csname\InTheoType Symbol\endcsname}
	\end{align*}
\end{proof}

\begin{example}
	Sei $f:\mathbb{R}^n\to \mathbb{R}$ mit $f(x) = \vert x \vert ^2 = \sum_{j=1}^n x_j^2$. $f$ ist \gls{diffbar} nach \propref{richtungsableitung_example_euklidische_norm} \\
	$\rightarrow$ $\frac{\partial}{\partial x_j} f(x) = 2 x_j$ und $j=1,\dotsc,n$ \\
	$\xRightarrow{\eqref{richtungsableitung_partielle_ableitung_ausrechnen}}$ $\mathrm{D}_z f(x) = \sum_{j=1}^n 2x_j\cdot z_j = 2\langle x,z\rangle$ (vgl. \propref{richtungsableitung_example_euklidische_norm})
\end{example}

\begin{theorem}[Vollständige Reduktion]
	\proplbl{richtungsableitung_vollstaendige_reduktion}
	Sei $f=(f_1, \dotsc, f_m): D\subset K^n\to K^m$, $D$ offen, $f$ \gls{diffbar} in $x\in D$. Dann:
	\begin{align}
		\proplbl{richtungsableitung_vollstaendige_reduktion_eq}
		f'(x) \overset{(a)}{=}\begin{pmatrix}
			f_1'(x) \\ \vdots \\ f_m'(x)
		\end{pmatrix} \overset{(b)}{=} \left( \frac{\partial}{\partial x_1} f(x)\;\dotsc\;\frac{\partial}{\partial x_n}f(x) \right) \overset{(c)}{=} \underbrace{\begin{pmatrix}
			\frac{\partial }{\partial x_1} f_1(x) & \dots & \frac{\partial}{\partial x_n} f_1(x) \\
			\vdots & & \vdots
			\\ \frac{\partial}{\partial x_n} f_m(x) & \dots & \frac{\partial}{\partial x_n} f_m(x)
		\end{pmatrix}}_{\mathcal{\text{\begriff{\person{Jacobi}-Matrix}}}}\in K^{m\times n}
	\end{align}
\end{theorem}

\begin{remark}
	Falls $f$ \gls{diffbar} in $x$, dann reduziert \propref{richtungsableitung_vollstaendige_reduktion} die Berechnung von $f'(x)$ auf Ableitung skalarer Funktionen $\tilde{f}:\tilde{D}\subset K\to K$.
\end{remark}

\begin{proof}[\propref{richtungsableitung_vollstaendige_reduktion}]\hspace*{0pt}
\begin{enumerate}[label={zu \alph*)},topsep=\dimexpr -\baselineskip / 2 \relax]
	\item \propref{ableitung_proposition_reduktion}
	\item Benutze $f'(x)\cdot z = \mathrm{D}_z f(x)$ und \propref{richtungsableitung_prop_partielle_ableitung_ausrechnen}
	\item Entweder $\frac{\partial}{\partial x_j} f(x) = \transpose{\left( \frac{\partial}{\partial x_j} f_1(x), \dotsc, \frac{\partial}{\partial x_j} f_n(x)\right)}$ oder $f_j'(x) = \left( \frac{\partial}{\partial x_1} f_j(x), \dotsc, \frac{\partial}{\partial x_n} f_j(x) \right)$, sonst analog zu b)
\end{enumerate}
\end{proof}

\begin{underlinedenvironment}[Frage]
	Gilt die Umkehrung von \propref{richtungsableitung_vollstaendige_reduktion} (\propref{richtungsableitung_prop_existenz_prop}), d.h. falls alle partiellen Ableitungen $\frac{\partial}{\partial x_j} f(x)$ bzw. alle Richtungsableitungen $\mathrm{D}_z f(x)$ existieren, ist dann $f$ \gls{diffbar} in $x$? Nein!
\end{underlinedenvironment}

\begin{example}
	Betrachte $f:\mathbb{R}^2\to\mathbb{R}$ mit \begin{align*}
		f(x_1, x_2) = \begin{cases}
			\frac{x_2^2}{x_1},& x_1\neq 0 \\
			0,& x_1 = 0
		\end{cases}
	\end{align*}
	
	Berechne Richtungsableitungen in $x=0$ mittels \eqref{richtungsableitung_definitnion_prop_zwei}.
	\begin{alignat*}{4}
		&\mathrm{D}_z f(0) = \lim\limits_{t\to 0} \frac{f(0 + tz)- f(0)}{t} = \lim\limits_{t\to 0} \frac{f(tz)}{t} \\
		\Rightarrow\;\; & \mathrm{D}_z f(0) = \lim\limits_{t\to 0} \frac{t^2 z_2^2}{t^2 z_1^2} = \frac{z_2^2}{z_1^2} \quad \forall z= (z_1, z_2)\in\mathbb{R}^2,\;z = 0
		\intertext{Betrachte möglicherweise problematische Richtung $z=(0,z_2)$}
		& D_{(0,z_2)} f(0) = \lim\limits_{t\to 0} \frac{0}{t} = 0 \\
		\Rightarrow\;\;& \mathrm{D}_z f(0) \text{ existiert } \forall z\in\mathbb{R}^2
	\end{alignat*}
	\emph{aber} ist $f$ überhaupt \gls{diffbar}? $\lim\limits_{n\to 0} f\left(\frac{1}{n^2},\frac{1}{n}\right) = \lim\limits_{n\to 0} \dfrac{\frac{1}{n^2}}{\frac{1}{n^2}} = 1 \; \neq \; 0 = f(0)$ \\
	$\Rightarrow$ $f$ nicht stetig in $x=0$ $\xRightarrow{\text{\propref{diffbar_impl_stetig}}}$ $f$ \emph{nicht \gls{diffbar}}.
\end{example}

\begin{underlinedenvironment}[Ausblick]
	Sind alle partiellen Ableitungen $\frac{\partial}{\partial x_j} f_j(x)$ stetige Funktionen in $x\in D$ \\
	$\Rightarrow$ $f$ \gls{diffbar} in $x$ und \propref{richtungsableitung_vollstaendige_reduktion_eq} gilt.
\end{underlinedenvironment}

\subsection{$\mathbf{\mathbb{R}}$-differenzierbar und $\mathbf{\mathbb{C}}$-differenzierbar}
Sei $f:D\subset K^n\to K^m$ ist \gls{diffbar} in $z_0 \in D$, $D$ offen

$\Leftrightarrow$ eine $k$-lineare Abbildung $A:K^n\to K^m$ existiert, die die Funktion $f$ in $z_0$ "`lokal approximiert"'.

$\rightarrow$ man müsste eigentlich genauer sagen: $f$ ist $k$-\gls{diffbar} in $z_0$ wegen $\mathbb{R}\subset\mathbb{C}$. Jeder \gls{vr} über $\mathbb{C}$ kann auch als \gls{vr} über $\mathbb{R}$ betrachtet werden (nicht umgekehrt!) und jede $\mathbb{C}$-lineare Abbildung zwischen $\mathbb{C}$-\gls{vr} kann auch als $\mathbb{R}$-linear betrachtet werden

$\Rightarrow$ jede $\mathbb{C}$-\gls{diffbar}e Funktion $f:D\subset \mathbb{C}^n\to \mathbb{C}^m$ ist auch $\mathbb{R}$-\gls{diffbar}.

Die Umkehrung gilt i.A. nicht!

\begin{example}
	Sei $f:\mathbb{C}\to\mathbb{C}$ mit $f(z) = \overline{z}$.
	\begin{enumerate}[label={\alph*)}]
		\item $f$ ist additiv und $f(tz) = t\cdot f(z)$ $\forall t\in \mathbb{R}$. \\
		$\Rightarrow$ $f$ ist $\mathbb{R}$-linear.
		
		Wegen $f(z) = \overline{z} = \overline{z_0} + \overline{z - z_0} = f(z_0) + f(z - z_0) + 0$ folgt: $\mathbb{R}$-\gls{diffbar} in $z_0$ $\forall Z-0\in\mathbb{C}$ mit $\mathbb{R}$-Ableitung $f'(z_0) = 1$
		
		\item Angenommen, $f$ ist $\mathbb{C}$-\gls{diffbar} in $z_0\in\mathbb{C}$.\\
		$\Rightarrow$ $f'(z_0) = \lim\limits_{z\to 0} \frac{\overline{z_0 + z} - \overline{z}}{z} = \lim\limits_{z\to 0} \frac{\overline{z}}{z} = \pm 1$ $\Rightarrow$ \Lightning\ (Grenzwert existiert nicht) \\
		$\Rightarrow$ $f$ nicht $\mathbb{C}$-\gls{diffbar}
	\end{enumerate}
\end{example}

\begin{*definition}
	$f:D\subset X\to Y$, $D$ offen, $(X,Y) = (\mathbb{R}^n, \mathbb{C}^m)$ bzw. $(\mathbb{C}^n,\mathbb{R}^m)$ oder $(\mathbb{C}^n, \mathbb{C}^m)$ heißt \begriff{$\mathbb{R}$-\gls{diffbar}} in $z_0\in D$, falls \eqref{definition_ableitung} im \propref{section_ableitung} gilt mit entsprechender $\mathbb{R}$-linearer Abbildung $A:X\to Y$ gibt.
	
	\uline{beachte:} falls $X$ oder $Y$ nur \gls{vr} über $\mathbb{R}$, dann $\mathbb{C}$-\gls{diffbar} nicht erklärt.
	\vspace*{1.5em}
\begin{underlinedenvironment}[Spezialfall]
	Sei $f:D\subset\mathbb{C}\to\mathbb{C}$, $D$ offen, $z_0\in D$. Vergleiche $\mathbb{R}$-\gls{diffbar} und $\mathbb{C}$-\gls{diffbar}:
	
	Sei $f$ $\mathbb{R}$-\gls{diffbar} in $z_0$, d.h. es existiert eine $\mathbb{R}$-lineare Abbildung $A:\mathbb{C}\to \mathbb{C}$ mit {\zeroAmsmathAlignVSpaces**\begin{align}
		\proplbl{richtungsableitung_differenzierbarkeit_r_diffbar}
		f(z_0 + z) = f(z_0) + A\cdot z + o(\vert z \vert z),\; z\to z_0
	\end{align}}
	\zeroAmsmathAlignVSpaces*
	\begin{alignat}{5}
	\notag &\text{für }& z=&x,\;&x\in\mathbb{R}:\;& A(1) &= \lim\limits_{\substack{x\to 0 \\ x\in\mathbb{R}}} \frac{f(z_0 + x) - f(z_0)}{x} &=: f_x(z_0) \\
	\proplbl{richtungsableitung_differenzierbar_partiell_y}
	&\text{für }& z=&iy,\;& y\in\mathbb{R}:\;& A(i) &= \lim\limits_{\substack{y\to 0 \\ y\in\mathbb{R}}} \frac{f(z_0 + iy) - f(z_0)}{y} &=: f_y (z_0)
	\end{alignat}
\end{underlinedenvironment}

	Nenne $f_x(z_0)$, $f_y(z_0)$ \begriff[Ableitung!]{partielle Ableitung}[!$\mathbb{C}$] von $f$ in $z_0$. Sei $f$ \begriff[Ableitung!]{$\mathbb{C}$-\gls{diffbar}} in $x_z$, d.h. \begin{align}
		\notag &f(z_0 + z) = f(z_0) + \underbrace{f'(z_0)}_{\in\mathbb{C}}\cdot z + o(\vert z \vert) \\
		\proplbl{richtungsableitung_differenzierbarkeit_vorform_cauchy_riemann}
		\xRightarrow{\eqref{richtungsableitung_differenzierbar_partiell_y}} \;& f'(z_0) = f_x(z_0) = -i f_y(x_0)
	\end{align}
\end{*definition}

\begin{proposition}
	Sei $f:D\subset\mathbb{C}\to\mathbb{C}$, $D$ offen, $z_0\in D$. Dann: \begin{align}
		\proplbl{richtungsableitung_differenzierbarkeit_equivalenz_c_r_diffbar}
		f\;\mathbb{C}\text{-\gls{diffbar} in }z_0 \; \; \Leftrightarrow \;\;f\;\mathbb{R}\text{-\gls{diffbar} in }z_0 \text{ mit }f_x(z) = -i f_y(z_0)
	\end{align}
\end{proposition}

\begin{proof}\hspace*{0pt}
	\NoEndMark
	\begin{itemize}[topsep=\dimexpr - \baselineskip / 2 \relax]
		\item["`$\Rightarrow$"'] vgl. oben \eqref{richtungsableitung_differenzierbarkeit_vorform_cauchy_riemann}
		\item["`$\Leftarrow$"'] mit $z=x + iy$ liefert \eqref{richtungsableitung_differenzierbarkeit_r_diffbar} 
			\begin{alignat*}{2}
			f(z_0 + z) &= f(z_0) + A(x + iy) + o(\vert z \vert) 
			&\;=\;& f(z_0) + x\cdot A(1) + yA(i) + o(\vert z \vert) \\
			&= f(z_0) - f_x(z_0)x + f_y(z_0) y + o(\vert z \vert)
			&\overset{\eqref{richtungsableitung_differenzierbarkeit_equivalenz_c_r_diffbar}}{=}& f(z_0) + f_x(z_0)(x + iy) + o(\vert z \vert) \\
			&= f(z_0) + \underbrace{f_x(z_0)}_{\mathclap{=:f'(z_0)\in\mathbb{C} \text{ als }\mathbb{C}\text{-Ableitung}}} \cdot z + o(\vert z \vert)&&
			\end{alignat*}
			\hfill\csname\InTheoType Symbol\endcsname
	\end{itemize}
\end{proof}

\subsection{\person{Cauchy}-\person{Riemann}-Differentialgleichungen}
Identifiziere $f:D\subset\mathbb{C}\to \mathbb{C}$ mit $\tilde{f}:\tilde{D}\subset\mathbb{R}^2\to\mathbb{R}^2$ gemäß $z = x + iy \equalhat \binom{x}{y}$, $f(z) = u(x,y) + iv(x,y) \equalhat \binom{u(x,y)}{v(x,y)} = \tilde{f}(x,y)$

Lineare Algebra: $A:\mathbb{C}\to\mathbb{C}$ linear $\Leftrightarrow$ $\exists w\in\mathbb{C}: Az = wz$ $\forall z\in\mathbb{C}$\marginnote{(Eigenwert)}\\
\phantom{Lineare Algebra:} $\tilde{A}:\mathbb{R}^2 \to \mathbb{R}^2$ $\mathbb{R}$-linear $\Leftrightarrow$ $\tilde{A} = \begin{pmatrix} a & b \\ c & d \end{pmatrix}\in\mathbb{R}^{2\times 2}$ bezüglich Standardbasis.

\begin{lemma}
	Sei $A:\mathbb{C}\to\mathbb{C}$ $\mathbb{R}$-linear. Dann: \begin{align*}
		&\text{$A$ ist auch $\mathbb{C}$-linear, d.h. $\exists w=\alpha + i\beta: Az = wz$ $\forall z\in\mathbb{C}$} \\ \Leftrightarrow\;\;& \text{$\exists \alpha,\beta\in\mathbb{R}: A(x + iy) \equalhat \begin{pmatrix} \alpha & -\beta \\ \beta & \alpha \end{pmatrix} \begin{pmatrix}
			x \\ y
		\end{pmatrix}$ $\forall x,y\in\mathbb{R}$}
	\end{align*}
\end{lemma}

\begin{proof}
	Selbststudium
\end{proof}

\begin{underlinedenvironment}[Somit]
	$\mathbb{C}$-lineare Abbildung $A:\mathbb{C}\to \mathbb{C}$ entspricht \emph{spezieller} $\mathbb{R}$-linearen Abbildung $\mathbb{R}^2\to\mathbb{R}^2$
\end{underlinedenvironment}

\begin{*definition}
Falls $\mathbb{R}$-\gls{diffbar} in $z_0$ liefert \eqref{richtungsableitung_differenzierbarkeit_vorform_cauchy_riemann} \begin{align*}
	f_x(z_0) &= u_x(x_0, y_0) + i v_x(x_0, y_0),& f_y(z_0) &= u_y(x_0, y_0) + iv_y(x_0, y_0)
\end{align*}
folglich \begin{align}
	\text{\propref{richtungsableitung_differenzierbarkeit_equivalenz_c_r_diffbar}} \;\Leftrightarrow\;\underbrace{\begin{alignedat}{2}
		u_x(x_0, y_0) &=& &v_y(x_0, y_0) \\
		u_y(x_0, y_0) &=&-&v_x(x_0, y_0)
	\end{alignedat}}_{\mathclap{\text{\begriff{\person{Cauchy}-\person{Riemann}-Differentialgleichungen}}}}
\end{align}
\end{*definition}

\begin{underlinedenvironment}[Somit]
	$\mathbb{C}$-lineare Abbilung $z \to f'(z_0)$ entspricht $\mathbb{R}$-linearer Abbildung \begin{align*}
	\begin{pmatrix}
		x \\ y
	\end{pmatrix}\to \begin{pmatrix}
		u_x & u_y \\ - u_y & u_x
	\end{pmatrix} \begin{pmatrix}
		x \\ y
	\end{pmatrix}
	\end{align*}
\end{underlinedenvironment}

\begin{hint}
	$\mathbb{C}$-\gls{diffbar}e Funktionen $f:D\subset \mathbb{C}\to\mathbb{C}$ werden in der Funktionentheorie untersucht.
	
	Es gilt z.B. $f$ $\mathbb{C}$-\gls{diffbar} auf $D$ $\Rightarrow$ Ableitung $f':D\to\mathbb{C}$ auch $\mathbb{C}$-\gls{diffbar} auf $D$ $\Rightarrow$ $f$ beliebig oft \gls{diffbar} auf $D$!
\end{hint}
\section{Mittelwertsatz und Anwendung}\setcounter{equation}{0}
\begin{*definition}[Maximum, Minimum]
	Wir sagen, $f:D\subset \mathbb{R}^n\to \mathbb{R}$ besitzt \begriff{Minimum} bzw. \begriff{Maximum} auf $D$, falls eine \begriff{Minimalstelle} bzw. \begriff{Maximalstelle} $x_0\in D$ existiert mit \begin{align}
		\proplbl{mittelwertsatz_extremalstellen}
		f(x_0) &\le f(x) & f(x) &\ge f(x) & \forall x&\in D
	\end{align}
	$f$ hat ein lokales Minimum bzw. lokales Maximum in $x_0\in D$ falls\begin{align}
		\proplbl{mittelwertsatz_lokale_extremstellen}
		\exists \epsilon > 0: f(x_0) &\le f(x) & f(x_0) &\ge f(x) & \forall x&\in B_{\epsilon}(x_0 \cap D)
	\end{align}
	Hat man in \eqref{mittelwertsatz_extremalstellen} bzw. \eqref{mittelwertsatz_lokale_extremstellen} für $x$ und $x_0$ "`$<$"' bzw. "`$>$"', so sagt man \begriff[Maximum!]{strenges} \begriff*[Minimum!]{streng} (lokales) Minimum bzw. Maximum.
\end{*definition}

\begin{theorem}[notwendige Optimalitätsbedingung]
	\proplbl{mittelwertsatz_optimalitaetsbedingung}
	Sei $f:D\subset \mathbb{R}^n \to \mathbb{R}$, $D$ offen, $f$ sei \gls{diffbar} in $x\in D$ und habe lokales Minimum bzw. Maximum in $x_0$. Dann:	\begin{align}
		\proplbl{mittelwertsatz_optimalitaetsbedingung_eq}
		f'(x_0) &= 0 \quad (\in\mathbb{R}^{1\times n})
	\end{align}
\end{theorem}

\begin{proof}
	\NoEndMark
	Für Minimum (Maximum analog) fixiere beliebiges $z\in\mathbb{R}^n$.
	
	$D$ offen\\
	\begin{tabularx}{\linewidth}{rX}
		\parbox{\widthof{\phantom{$\xRightarrow{t\to 0}$}}}{\hfill$\Rightarrow$} & $\exists \delta > 0: x_0 + t\cdot z\in D$ $\forall t\in (-\delta, \delta)$
	\end{tabularx}
	
	$f$ \gls{diffbar} in $x_0$, Minimum in $x_0$ \\
	(diff. $f$ im Pkt. $x_0$)
	\begin{tabularx}{\linewidth}{rX}
		$\Rightarrow$ & $ 0\le f(x_0 + t\cdot z) - f(x_0) = t\cdot f'(z_0) \cdot z + o(t)$, $t\to 0$ \marginnote{$\left| \cdot \frac{1}{t}\right.$} \\
		$\xRightarrow{t>0}$ & $0\le f'(x_0)\cdot z + o(1)$ \\
		$\xRightarrow{t\to 0}$ & $0 \le f'(x_0)\cdot z$ $\forall z\in\mathbb{R}^n$ \\
		$\xRightarrow{\pm z}$ & $f'(x_0) \cdot z = 0$ $\forall z\in\mathbb{R}^n$ \\
		$\Rightarrow$ & $f'(x_0) = 0$\hfill\csname\InTheoType Symbol\endcsname
	\end{tabularx}
\end{proof}

Einfache, aber wichtige Anwendung:
\begin{proposition}[Satz von Rolle]
	\proplbl{mittelwertsatz_rolle}
	Sei $f:[a,b]\subset\mathbb{R}\to\mathbb{R}$ stetig, $-\infty < a < b < \infty$, $f$ \gls{diffbar} auf $(a,b)$ und $f(a) = f(b)$.\\
	$\Rightarrow$ $\exists \xi\in(a,b): f(\xi) = 0$
\end{proposition}

\begin{proof}
	\NoEndMark
	$f$ stetig, $[a,b]$ kompakt \\
	$\xRightarrow{\text{Weierstrass}}$ $\exists x_1, x_2\in [a,b]: f(x_1) \le f(x) \le f(x_2)$ $\forall x$
	\begin{itemize}
		\item Angenommen, $f(x_1) = f(x_2) = f(a)$ $\Rightarrow$ $f$ konstante Funktion $\Rightarrow$ $f'(\xi) = 0$ $\forall \xi \in (a,b)$
		\item Andernfalls sei $f(x_1) < f(a)$ $\Rightarrow$ $\xi := x_1\in(a,b)$ $\xRightarrow{\text{\propref{mittelwertsatz_optimalitaetsbedingung}}}$ $f'(\xi) = 0$
		\item analog $f(x_2) > f(a)$\hfill\csname\InTheoType Symbol\endcsname
	\end{itemize}
\end{proof}

\begin{*definition}[abgeschlossenes, offenes Segment]
	Setze für $x,y\in K^n$
	\begin{itemize}
		\item $[x,y] := \{ x + t(y - x)\in\mathbb{R}^n \mid t\in [0,1] \}$ \begriff[Segment!]{abgeschlossenes} \begriff{Segment} (abgeschlossene Verbindungsstrecke)
		\item $(x,y) := \{ x + t(y - x)\in\mathbb{R}^n \mid t\in (0,1) \}$ \begriff[Segment!]{offenes} \begriff{Segment} (offene Verbindungsstrecke)
	\end{itemize}
\end{*definition}

\begin{theorem}[Mittelwertsatz]
	\proplbl{mittelwertsatz_mittelwertsatz}
	Sei $f:D\subset\mathbb{R}^n\to \mathbb{R}$, $D$ offen, $f$ \gls{diffbar} auf $D$ und seien $x,y\in D$ mit $[x,y]\subset D$. Dann \begin{align}
		\proplbl{mittelwertsatz_mittelwertsatz_eq}
		\exists \xi\in(x,y): f(y) - f(x) = f'(\xi) \cdot (y - x)
	\end{align}
\end{theorem}

\begin{remark}\vspace*{0pt}
	\begin{itemize}
		\item Für $n=1$ schreibt man \eqref{mittelwertsatz_mittelwertsatz_eq} auch als
			$f'(\xi) = \frac{f(y) - f(x)}{y - x} \quad\text{falls }x\neq y.$
		\item Der \gls{mws} gilt \emph{nicht} für $\mathbb{C}$ oder $m\neq 1$.
		\item \propref{mittelwertsatz_mittelwertsatz} gilt bereits für $D\subset\mathbb{R}^n$ beliebig, $f$ stetig auf $[x,y]\subset D$, $f$ \gls{diffbar} auf $(x,y) \subset \inn D$.
	\end{itemize}
\end{remark}

\begin{proof}
	Kontruiere eine Fkt. aus $f$, sd diese EGS von Satz von Rolle erfüllt, Ableitung von $f$ berechnen mit Kettenregel einsetzen in Satz von Rolle und fertig.
	\NoEndMark
	Setzte $\phi(t) = f(x + t(y - x)) - \big( f(y) - f(x) \big) t$ $\forall t\in[0,1]$ \\
	\begin{tabularx}{\linewidth}{rX}
	\parbox{\widthof{$\xRightarrow{\text{\propref{mittelwertsatz_rolle}}}$}}{\hfill$\xRightarrow{\text{$f$ \gls{diffbar}}}$}& $\phi: [0,1]\to \mathbb{R}$ stetig, $\phi(0) = \phi(1) = f(x)$
	\end{tabularx}

	$\phi$ \gls{diffbar} auf $(0,1)$ (verwende Kettenregel) mit \begin{align}
	\proplbl{mittelwertsatz_mittelwertsatz_beweis_eq}
	\phi'(t) = f'(x + t(y - x)) \cdot (y - x) - \big( f(y) - f(x) \big)
	\end{align}
	\begin{tabularx}{\linewidth}{rX@{}}

	$\xRightarrow{\text{Satz von Rolle}}$ & $f(y) - f(x) = f'(\underbrace{x + \tau (y - x)}_{=: \xi \in (x,y)}) \cdot (y - x)$ \\
	$\Rightarrow$ & Behauptung\hfill\csname\InTheoType Symbol\endcsname
	\end{tabularx}
\end{proof}

\begin{boldenvironment}[Frage]
	Der \gls{mws} gilt für $m=1$. Was ist bei $m > 1$?
\end{boldenvironment}
	
\begin{conclusion}
	Sei $f = (f_1, \dotsc, f_m): D\subset\mathbb{R}^n \to \mathbb{R}^m$, $D$ offen, \gls{diffbar} auf $D$, $[x,y]\subset D$. Dann
	\begin{align}
		\proplbl{mittelwertsatz_mittelwertsatz_m_gt_eins_eq}
		\exists \xi_1, \dotsc, \xi_m \in (x,y): f(y) - f(x) = \left( \begin{matrix}
			f_1'(\xi_1) \\ \vdots \\ f_m'(\xi_m) 
		\end{matrix} \right) \cdot (y - x)
	\end{align}
\end{conclusion}

\begin{proof}
	\propref{mittelwertsatz_mittelwertsatz_m_gt_eins_eq} ist äquivlanet zu $m$ skalaren Gleichungen \begin{align*}
		f_j(y) - f_j(x) = f_j'(\xi_j) \cdot (y - x), \quad j = 1,\dotsc,m
	\end{align*}
	und diese Folgen direkt aus \propref{mittelwertsatz_mittelwertsatz} für $f_j: D\to \mathbb{R}$.
\end{proof}

\begin{boldenvironment}[Frage]
	Ist in MWS auch $\xi_1 = \dotsc = \xi_m$ möglich? Im Allgemeinen nein.
\end{boldenvironment}

\begin{example}
	Sei $f:\mathbb{R}\to\mathbb{R}^2$ mit $f(x) = \binom{\cos x}{\sin x}$ $\forall x\in\mathbb{R}$.
	
	Angenommen, $\exists \xi\in (0,2\pi): f(2\pi) - f(0) = f'(\xi) \cdot (2\pi - 0) = 0$ \\
	\begin{tabularx}{\linewidth}{rX}
		$\Rightarrow$ & $0 = f'(\xi) = \binom{-\sin \xi}{\cos \xi}$, d.h. $\sin\xi = \cos\xi = 0$ \\
		$\Rightarrow$ & \Lightning \\
		$\Rightarrow$ & $\xi_1 = \xi_2$ in MWS ist nicht möglich.
	\end{tabularx}
\end{example}

\begin{theorem}[Schrankensatz]
	\proplbl{mittelwertsatz_schrankensatz}
	Sei $f:D\subset K^n\to K^m$, $D$ offen, $f$ \gls{diffbar} auf $D$. Seien $x,y\in D$, $[x,y]\subset D$. Dann\begin{align}
		\proplbl{mittelwertsatz_schrankensatz_eq}
		\exists \xi\in (x,y): \vert f(y) - f(x) \vert \le \vert f'(\xi) (y - x)\vert \le \Vert f'(\xi) \Vert \cdot \vert y - x\vert 
	\end{align}
	\emph{beachte:} \propref{mittelwertsatz_schrankensatz} gilt auch für $K=\mathbb{C}$.
\end{theorem}

\begin{proof}
	Setze Normalenvektor $v$ besteht aus der Differenz der Fktwerte. Konstruiere $\phi$ als Fkt des Realteils des Skalarprod. von Abl. von $f$ und $v$. Leite $\phi$ ab und nutze MWS und damit folgt die Beh.  \\
	Def.
	\NoEndMark
	Sei $f(x) \neq f(y)$ (sonst klar). Setzte $v:= \frac{f(y) - f(x)}{\vert f(y) - f(x)\vert} \in K^m$, offenbar $\vert v \vert = 1$.
	
	Betrachte $\phi: [0,1]  \to\mathbb{R}$ mit $\phi(t) := \Re \langle f(x + t (y - x)), v\rangle\marginnote{$\langle u,v\rangle = \sum_{i=1}^{n}\overline{u_i} v_i$}$
	Da $f$ \gls{diffbar}, gilt \begin{align*}
		\langle f(x + s(y - x)), v\rangle = \langle f(x + t(y - x)), v\rangle + \langle f'(x + t(y - x))\cdot (s  - t)(y - x), v \rangle + \underbrace{o(\vert s -  t\vert \cdot \vert y - x\vert)}_{=o(\vert s - t\vert)}, \; s\to t
	\end{align*} und damit ist auch $\phi$ \gls{diffbar} auf $(0,1)$ mit \begin{align*}
		\phi'(t) &= \Re \langle f'(x + t(y - x))\cdot (y - x), v \rangle \quad \forall t\in (0,1)
	\end{align*}
	MWS liefert: $\exists \tau \in (0,1): \underbrace{\phi(1) - \phi(0)}_{=\Re \langle f(y) - f(x), v\rangle} = \phi(\tau) \cdot (1 - 0)$ \\
	\begin{alignat*}{8}
		&\xRightarrow{\xi = x + \tau (y - x)}\;\;& \vert f(y) - f(x) \vert &&\,=\,& \Re \langle f(y) - f(x), v \rangle &&&\,=\,& \phi(1) - \phi(0) &&&\,=\,& \Re \langle f'(\xi) \cdot (y - x), v\rangle& \\
		&& &&\le& \vert \langle f'(\xi) \cdot (y - x), v \rangle \vert& &&\overset{\star}{\le}&\marginnote{$\star$: \person{Cauchy}-\person{Schwarz}} \vert f'(\xi) \cdot (y - x)\vert \cdot \underbrace{\vert v \vert}_{=1}&  \\
		&& &&\le& \Vert f'(\xi) \Vert \cdot \vert y - x\vert&
	\end{alignat*} \hfill\csname\InTheoType Symbol\endcsname
\end{proof}

\begin{boldenvironment}[bekanntlich]
	$f(x) = \mathrm{const}$ $\forall x$ $\Rightarrow$ $f'(x) = 0$
\end{boldenvironment}

\begin{proposition}
	\proplbl{mittelwertsatz_ableitung_null_konstante_funktion}
	Sei $f:D\subset K^n\to K^m$, $D$ offen, und zusammenhängend.
	
	\begin{tabularx}{\linewidth}{XcX}
		\hfill$f$ \gls{diffbar} auf $D$ mit $f'(x) = 0$ $\forall x\in D$ & $\Rightarrow$ & $f(x) = \mathrm{const}$ $\forall x\in D$.
	\end{tabularx}
\end{proposition}

\begin{proof}
	\NoEndMark \hspace*{0pt}
	\begin{enumerate}[label={\arabic*.},topsep=-\baselineskip]
		\item 
	\begin{itemize}
	\item $D$ offen, zusammenhängend, $K^n$ normierter Raum  $\xRightarrow{\text{Satz 15.8}}$ $D$ bogenzusammenhängend
	\item Wähle nun $x,y\in D$ $\Rightarrow$ $\exists \phi: [0,1] \to D$ stetig, $\phi(0) = x$, $\phi(1) = y$
	\item $D$ offen $\Rightarrow$ $\forall t\in [0,1]$ existiert $r(t) > 0: B_{r(t)}(\phi(t)) \subset D$ 
	\item Nach \propref{satz_15_1} ist $\phi([0,1])$ kompakt und $\{ B_{r(t)}(\phi(t)) \mid t \in [0,1] \}$ ist offene Überdeckung von $\phi([0,1])$ \\
	$\Rightarrow$ existiert endliche Überdeckung, d.h. $\exists t_1, \dotsc, t_n \in [0,1]$ mit $\phi([0,1]) \subset \bigcup\limits_{i = 1, \dotsc, n} B_{r(t_i)} (\phi(t_i))$.
	\end{itemize}
	
	\item Falls wir noch zeigen, dass $f$ konstant ist auf jeder Kugel $B_r(z)\subset D$ ist, dann wäre $f(x) = f(y)$ \\
	$\xRightarrow{x,y \text{ bel.}}$ Behauptung.
	
	\item 
	
	Sei $B_r(z)\subset D$, $x,y\in B_r(z)$
	
	\begin{tabularx}{\linewidth}{rX}
		$\xRightarrow{\text{\text{Schrankensatz}}}$ & $\vert f(y) - f(x) \vert \le \underbrace{\Vert f'(\xi) \Vert}_{= 0} \cdot \vert y - x\vert = 0$ \\
		$\Rightarrow$ & $f(x) = f(y)$ \\
		$\xRightarrow{x,y\text{ bel.}}$ & $f$ konst. auf $B_r(z)$\hfill\csname\InTheoType Symbol\endcsname
	\end{tabularx}
	\end{enumerate}
\end{proof}

\begin{example}
	Sei $f:D = (0,1)\cup (2,3) \to \mathbb{R}$ \gls{diffbar}, sei $f'(x) = 0$ auf $D$ \\
	\begin{tabularx}{\linewidth}{rX}
	$\xRightarrow{\text{\propref{mittelwertsatz_ableitung_null_konstante_funktion}}}$ & $f(x) = \mathrm{const}$ auf $(0,1)$ und $(2,3)$, aber auf jedem Intervall kann die Konstante anders sein.
	\end{tabularx}
\end{example}
\rule{0.4\linewidth}{0.1pt}

\begin{theorem}
	\proplbl{mittelwertsatz_existenz_partieller_ableitung}
	Sei $f:D\subset K^n\to K^m$, $D$ offen, $x\in D$.
	
	Falls partielle Ableitung $f_{x_j}(y)$, $j=1,\dotsc,n$ für alle $y\in B_r(x)\subset D$ für ein $r > 0$ existierten und falls $y\to f_{x_j}(y)$ stetig in $x$ für $j=1,\dotsc,n$ \\
	$\Rightarrow$ $f$ ist differentierbar in $x$ mit $f'(x) = \big( f_{x_1}(x), \dotsc, f_{x_n}(x) \big) \in K^{m\times n}$
\end{theorem}

\begin{proof}
	\NoEndMark
	Fixiere $y = (y_1, \dotsc, y_n)\in B_r(0)$.
	
	Betrachte die Eckpunkt eines Quaders in $D$: $a_0 = x, a_k := a_{k - 1} + y_k e_k$ für $k = 1,\dotsc,n$ \\
	$\Rightarrow$ $a_n = x + y$.
	
	Offenbar $\phi_k(t) = f(a_{k-1} + t e_k y_k) - f(a_{k - 1}) - tf_{x_k}(a_{k - 1}) y_k$ stetig auf $[0,1]$, \gls{diffbar} auf $(0,1)$ mit \begin{align*}\phi_k'(t) = f_{x_k}(a_{k - 1} + t e_k y_k) y_k - f_{x_k}(a_{k-1}) y_k
	\end{align*}
	$\xRightarrow{\text{\propref{mittelwertsatz_schrankensatz}}}$ $\vert \phi_k(1) - \phi_k(0)\vert = \vert f(a_k) - f(a_{k  - 1}) - f_{x_k} (a_{k  +1}) y_k \vert \le \sup\limits_{t\in (0,1)} \vert \phi_k'(\xi)\vert$, $k = 1,\dotsc,n$
	
	Es gilt mit $A := \big( f_1(x), \dotsc, x_{x_n}(x) \big)$:
	
	\begin{tabularx}{\linewidth}{r@{$\;$}c@{$\;$}c@{}l}
		\hfill $\vert f(x + y) - f(x) - Ay\vert$ & $=$ & & $\displaystyle\left\vert \sum_{k=1}^{n} f(a_k) - f(a_{k -1}) - f_{x_k}(x)y_k\right\vert$ \\
		& $\overset{\triangle\text{-Ungl}}{\le}$& & $\displaystyle\sum_{k=1}^n \big\vert f(a_k) - f(a_{k - 1}) - f_{x_k}(x) y_k \big\vert$ \\
		& $\underset{\text{Def. $\phi_k$}}{\overset{\triangle\text{-Ungl}}{\le}}$& & $\displaystyle\sum \vert \phi_k(1) - \phi_k(0)\vert + \vert f_{x_k} (a_{k - 1}) y_k - f_{x_k}(x) y_k \vert$ \\
		& $\le$ & $\vert y \vert$ & $\displaystyle\sum \sup\limits_{t\in(0,1)} \vert f_{x_k}( a_{ k - 1} + t \cdot e_k y_k) - f_{x_k}(a_{k - 1})\vert + \vert f_{x_k}(a_{k - 1}) - f_{x_k}(x) \vert$ \\
		& $\overset{\triangle\text{-Ungl}}{\le}$ & $\vert y \vert$ & $\displaystyle \underbrace{\sum_{k=1}^n \sup \vert f_{x_k} (a_{k-1} + t e_k y_k ) - f_{x_k}(x) \vert + 2 \vert f_{x_k}\ (a_{k - 1}) - f_{x_k}(x) \vert} _{=:\rho(y) \xrightarrow{y\to 0}0\text{, da part. Ableitung $f_{x_k}$ stetig in $x$}}$
	\end{tabularx}
	
	\begin{tabularx}{\linewidth}{rX@{}}
	$\Rightarrow$ & $f(x + y) = f(y) + Ay + R(y)$ mit $\frac{\vert R(y)\vert}{y} \le \rho(y) \xrightarrow{y\to 0} 0$ (d.h. $R(y) = o(\vert y)$) \\
	$\xLeftrightarrow{\propref{definition_ableitung_proposition}}$ & $f$ ist \gls{diffbar} in $x$ mit $f'(x) = A$\hfill\csname\InTheoType Symbol\endcsname
\end{tabularx}
\end{proof}

\subsection{Anwendung des Mittelwertsatzes in \texorpdfstring{$\mathbb{R}$}{R}}
\begin{proposition}[Monotonie]
	\proplbl{mittelwertsatz_anwendung_monotonie}
	Sei $f:(a,b)\subset\mathbb{R}\to \mathbb{R}$ \gls{diffbar}, dann gilt:
	\begin{enumerate}[label={\roman*)}]
		\item \proplbl{mittelwertsatz_anwendung_monotonie_aussage_eins}$f'(x) \ge 0$ ($\le 0$) $\forall x\in (a,b)$ $\Leftrightarrow$ $f$ monoton wachsend (monoton fallend)
		\item \proplbl{mittelwertsatz_anwendung_monotonie_aussage_zwei} 	$f'(x) > 0$ ($< 0$) $\forall x\in (a,b)$ $\Rightarrow$ $f$ streng monoton wachsend (fallend)
		\item \proplbl{mittelwertsatz_anwendung_monotonie_aussage_drei} $f'(x) = 0$ $\forall x\in (a,b)$ $\Leftrightarrow$ $f$ konst.
	\end{enumerate}
\end{proposition}

\begin{remark}
	In \ref{mittelwertsatz_anwendung_monotonie_aussage_zwei} gilt die Rückrichtung nicht! (Betr. $f(x) = x^3$ und $f'(0) = 0$)
\end{remark}

\begin{proof}[jeweils für wachsend, fallend analog]
	Sei $x,y\in (a,b)$ mit $x < y$.
	\begin{itemize}[topsep=\dimexpr -\baselineskip / 2 \relax]
		\item["`$\Rightarrow$"'] in \ref{mittelwertsatz_anwendung_monotonie_aussage_eins}, \ref{mittelwertsatz_anwendung_monotonie_aussage_zwei}, \ref{mittelwertsatz_anwendung_monotonie_aussage_drei}
		
		Nach \propref{mittelwertsatz_mittelwertsatz} $\exists \xi\in(a,b): f(y) - (x) = f'(\xi) (y - x) \stackrel{>}{=} 0$ $\xRightarrow{\text{$x,y$ bel.}}$ Behauptung
		
		\item["`$\Leftarrow$"'] in \ref{mittelwertsatz_anwendung_monotonie_aussage_eins}, \ref{mittelwertsatz_anwendung_monotonie_aussage_drei}
		
		$0 \stackrel{\le}{=} \frac{f(y) - f(x)}{y - x} \xrightarrow{y\to x} f'(x)$ $\Rightarrow$ Behauptung
	\end{itemize}
\end{proof}

\begin{proposition}[Zwischenwertsatz für Ableitungen]
	Sei $f:(a,b)\subset\mathbb{R}\to\mathbb{R}$ \gls{diffbar}, $a < x_1 < x_2 < b$. Dann
	
	\begin{center}
	\begin{tabular}{r@{$\;\;$}c@{\ \ }l}
		$f'(x_1) < \gamma < f'(x_2)$ & $\Rightarrow$ & $\exists \tilde{x}\in(x_1,x_2): f'(\tilde{x})=\gamma$
	\end{tabular}
	\end{center}
	(analog $f(x_2) < \gamma < f(x_1)$)
\end{proposition}

\begin{proof}
	Sei $g:(a,b)\to \mathbb{R}$ mit $g(x) = f(x) - \gamma x$ ist \gls{diffbar} auf $(a,b)$
	
	\begin{tabularx}{\linewidth}{r@{\ \ }X@{}}
		$\xRightarrow{\text{Weierstraß}}$ & $\exists \tilde{x}\in [x_1,x_2]$ mit $g(\tilde{x}) \le g(x)$ $\forall x\in[x_1,x_2]$ \\
		\multicolumn{2}{l}{Angenommen, $\tilde{x} = x_1$} \\
		$\Rightarrow$ & $0 \le \frac{g(x) - g(x_1)}{x - x_1} \xrightarrow{x\to x_1} g'(x_1) = f'(x_1) - \gamma < 0$ \\
		$\Rightarrow$ & \Lightning (für Minimum: $f'(x) \ge 0$) \\
		$\Rightarrow$ & $x_1 < \tilde{x}$, analog $\tilde{x} < x_2$
	\end{tabularx}
	$\xRightarrow{\text{\propref{mittelwertsatz_optimalitaetsbedingung}}}$ $0 = g'(\tilde{x}) = f'(\tilde{x}) - \gamma$ $\Rightarrow$ Behauptung 
\end{proof}

\rule{0.4\linewidth}{0.1pt}

Betrachte nun "`unbestimme Grenzwerte"' $\lim\limits_{y\to x} \frac{f(x)}{g(x)}$ der Form $\frac{0}{0}, \frac{\infty}{\infty}$, wie z.B. $\lim\limits_{x\to 0} \frac{x^2}{x} = \lim\limits_{x\to 0} x$, $\lim\limits_{x\to 0} \frac{\sin x}{x}$.

\begin{proposition}[Regeln von \person{de l'Hospital}]
	\proplbl{mittelwertsatz_krankenhaus}
	Seien $f,g:(a,b)\subset\mathbb{R}\to\mathbb{R}$ \gls{diffbar}, $g'(x) \neq 0$ $\forall x\in(a,b)$ und entwender
	\begin{enumerate}[label={\roman*)}]
		\item $\lim\limits_{x\downarrow a} f(x) = 0$, $\lim\limits_{x\downarrow 0} g(x) = 0$, oder
		\item $\lim\limits_{x\downarrow a} f(x) =\infty$, $\lim\limits_{x\downarrow a} g(x) = \infty$
	\end{enumerate}

	Dann gilt:
	\begin{align}
		\text{Falls $\lim\limits_{y\downarrow a} \frac{f'(y)}{g'(y)} \in\mathbb{R}\cup \{ \pm\infty \}$ ex.} \;\; \Rightarrow \;\; \lim\limits_{y\downarrow a} \frac{f(y)}{g(y)} \in\mathbb{R}\cup \{ \pm\infty \}\text{ ex. und }\lim\limits_{y\downarrow a} \frac{f(y)}{g(y)} = \lim\limits_{y\to a} \frac{f'(y)}{g'(y)}
	\end{align}
	
	(Analoge Aussagen für $x\uparrow b$, $x\to +\infty$, $x\to-\infty$)
\end{proposition}

\begin{remark}
	\vspace*{0pt}
	\begin{enumerate}[label={\arabic*)},topsep=\dimexpr-\baselineskip/2\relax]
		\item Analogie zu S. 9.34 Satz von Stolz
		\item Grenzwerte Form $0\cdot \infty$, $1^{\infty}$, $0^0$, $\infty^0$, $\infty - \infty$ angewendet werden, mit folg. Identitäten: \begin{align*}
			\alpha\cdot\beta &= \frac{\alpha}{\frac{1}{\beta}} & \alpha^\beta &= e^{\beta \cdot \ln \alpha} & \alpha - \beta &= \alpha \left( 1 - \frac{\beta}{\alpha} \right)
		\end{align*}
	\end{enumerate}
\end{remark}

\begin{proof}\hspace*{0pt}
	\NoEndMark
	\begin{enumerate}[topsep=\dimexpr-\baselineskip/2\relax,label={zu \roman*)},leftmargin=\widthof{\texttt{zu ii)}}]
		\item Mit $f(a) := 0$, $g(a) := 0$ sind $f,g$ stetig auf $[a,b)$ \\
		$\xRightarrow{\text{\propref{mittelwertsatz_mittelwertsatz_verallgemeinert}}}$ $\forall x\in(a,b)\;\exists\xi = \xi(x) \in (a,x): \frac{f(x)}{g(x)} = \frac{f'(\xi)}{g'(\xi)}$. Wegen $\xi(x)\to a$ für $x\to a$ folgt die Behauptung
		\item Sei $\lim\limits_{x\downarrow a} \frac{f'(x)}{g'(x)} =: \gamma\in\mathbb{R}$ ($\gamma = \pm \infty$ ähnlich)
		
		Sei \gls{obda} $f(x)\neq 0$, $g(x)\neq 0$ auf $(a,b)$. Sei $\epsilon> 0$ fest \\
		$\Rightarrow$ $\exists \delta > 0: \left\vert \frac{f'(\xi)}{g'(\xi)} - \gamma \right\vert < \epsilon$ $\forall \xi\in(a,a+\delta)$ und
		\begin{align*}
			\left\vert \frac{f(y) - f(x)}{g(y) - g(x)} - \gamma \right\vert \underset{\exists \xi\in(a,a+\delta)}{\overset{\propref{mittelwertsatz_mittelwertsatz_verallgemeinert}}{\le}} \underbrace{\left\vert \frac{f(y) - f(x)}{g(y) - g(x)} - \frac{f'(\xi)}{g'(\xi)}\right\vert}_{=0} + \left\vert \frac{f'(\xi)}{g'(\xi)} - \gamma\right\vert < \epsilon\quad\forall x,y\in(a,a+\delta),\;g(x)\neq g(y)
		\end{align*}
		Fixiere $y\in(a,a+\delta)$, dann $f(x)\neq f(y)$, $g(x) \neq g(y)$ $\forall x\in(a,a+\delta_1)$ für ein $0 < \delta_1 < \delta$ und \begin{align*}
			\frac{f(x)}{g(x)} = \frac{f(y) - f(x)}{g(y) - g(x)} \cdot \underbrace{\dfrac{1 - \frac{g(y)}{g(x)}}{1 - \frac{f(y)}{f(x)}}}_{\xrightarrow{x\downarrow a} 1}
		\end{align*}
		\begin{tabularx}{\linewidth}{r@{\ \ }X}
		$\Rightarrow$ & $\exists \delta_2 > 0: \delta_2 < \delta_1$ und $\left\vert \frac{f(x)}{g(x)} - \frac{f(y) - f(x)}{g(y) - g(x)} \right\vert < \epsilon \quad\forall x\in(a, a+\delta_2)$ \\
		$\Rightarrow$ & $\left\vert \frac{f(x)}{g(x)} - \gamma\right\vert \le \left\vert \frac{f(x)}{g(x)} - \frac{f(y) - f(x)}{g(y) - g(x)} \right\vert + \left\vert \frac{f(y) - f(x)}{g(y) - g(x)} - \gamma \right\vert < 2\epsilon \quad\forall x\in(a, a+ \delta_2)$
		\end{tabularx}
		\end{enumerate}
		$\xRightarrow{\text{$\epsilon > 0$ beliebig}}$ Behauptung
		
		andere Fälle:\begin{itemize}[topsep=\dimexpr -\baselineskip / 2\relax]
				\item $x\uparrow b$ analog
				\item $x\to +\infty$ mittels Transformation $x = \frac{1}{y}$ auf $y\downarrow 0$ zurückführen
				\item $x\to -\infty$ analog\hfill\csname\InTheoType Symbol\endcsname
			\end{itemize}
\end{proof}

\begin{example}
	\proplbl{mittelwertsatz_beispiel_sinx_x}
	$\lim\limits_{x\to 0} \frac{\sin x}{x} = 1$, denn $\lim\limits_{x\to 0} \frac{(\sin x)'}{x'} = \lim\limits_{x\to 0} \frac{\cos x}{1} = 1$
\end{example}

\begin{example}
	$\lim\limits_{x \to 0} x\cdot \ln x = \lim\limits_{x \to 0} \dfrac{\ln x}{\frac{1}{x}} = 0$, denn $\lim\limits_{x \to 0} \dfrac{(\ln x)'}{\left( \frac{1}{x}\right)'} = \lim\limits_{x \to 0}\dfrac{\frac{1}{x}}{-\frac{1}{x^2}} = 0$
\end{example}

\begin{example}
	$\lim\limits_{x \to 0} \frac{2 - 2\cos x}{x^2} = 1$, denn es ist $\lim\limits_{x \to 0} \frac{(2 - 2\cos x)'}{(x^2)'} = \lim\limits_{x \to 0} \frac{2\sin x}{2x} \overset{\propref{mittelwertsatz_beispiel_sinx_x}}{=} 1$.
	
	\emph{beachte:} \propref{mittelwertsatz_krankenhaus} wird in Wahrheit zweimal angewendet.\\
\end{example}

\begin{example}
	$\lim\limits_{x\to\infty} \left( 1 + \frac{y}{x}\right) ^x = e^y$ $\forall y\in\mathbb{R}$ mit \begin{align*}
		\left( 1 + \frac{y}{x}\right)^x &= e^{x\cdot \ln \left( 1 + \frac{y}{x}\right)} = e^{\frac{\ln \left( 1 + \sfrac{y}{x}\right)}{\sfrac{1}{x}}}, &
		\lim\limits_{x\to\infty} \dfrac{\left(\ln \left( 1 + \frac{y}{x}\right) \right)'}{\left(\frac{1}{x}\right)'} &= \lim\limits_{x\to\infty} \dfrac{yx^2}{\left(1 + \frac{y}{x}\right) x^2} = \lim\limits_{x\to\infty} \dfrac{y}{1 + \frac{y}{x}} = y
	\end{align*}
	
	(vgl. Satz 13.9)
\end{example}
\section{Stammfunktionen}
\setcounter{equation}{0}
Sei $f:D\subset K^n\to K^{m\times n}$

\begin{boldenvironment}[Frage]
	Existiert eine Funktion $F$ mit $F' = f$ auf $D$?
\end{boldenvironment}

\begin{*definition}[Stammfunktion, unbestimmtes Integral]
	$F: D\subset K^n\to K^m$ heißt Stammfunktion oder unbestimmtes Integral von $f$ auf $D$, falls $F$ diffbar und $F'(x) = f(x)$ $\forall x\in D$
\end{*definition}

Betrachte zunächst den Spezialfall $n=m=1$. Sei $f:D\subset K\to K$, $D$ offen. Die Beispiele zur Differentiation liefern folgende Stammfunktionen

\begin{minipage}{0.45\linewidth}
	\begin{flushleft}
		für $K=\mathbb{R}$ und $K = \mathbb{C}$:
	\end{flushleft}
	\vspace*{1mm}
	\renewcommand{\arraystretch}{1.2}
	\begin{tabularx}{\linewidth}{llX}
		\toprule
		$f(x)$ & \multicolumn{2}{l}{Stammfunktion $F(x)$} \\
		\midrule
		$\sin x$ & $-\cos x$ & \\
		$\cos x$ & $\sin x$ & \\
		$e^x$ & $e^x$ & \\
		$x^k$ & $\frac{1}{k+1} x^{k+1}$ & ($k\in\mathbb{Z}\setminus\{-1\})$ \\
		\bottomrule
	\end{tabularx}
\end{minipage}
\hfill%
\begin{minipage}{0.45\linewidth}
	\begin{flushleft}
		für $K=\mathbb{R}$:
	\end{flushleft}
	\vspace*{1mm}
	\renewcommand{\arraystretch}{1.2}
	\begin{tabularx}{\linewidth}{llX}
		\toprule
		$f(x)$ & \multicolumn{2}{l}{Stammfunktion $F(x)$} \\
		\midrule
		$a^x$ & $\frac{a^x}{\ln a}$ & \\
		$x^\alpha$ & $\frac{1}{\alpha + 1} x^{\alpha + 1}$ & ($x > 0$, $\alpha \in \mathbb{R}\setminus \{ - 1\})$ \\
		$\frac{1}{x}$ & $\ln\vert x\vert$ & ($x\in\mathbb{R}\setminus \{0\}$) \\
		$\frac{1}{1+x^2}$ & $\arctan x$ & \\
		\bottomrule
	\end{tabularx}
\end{minipage}
	
\begin{proposition}[partielle Integration]
	Seien $f,g:D\subset K\to K$, $D$ Gebiet mit zugehörigen Stammfunktion $F, G:D\to K$.
	
	Falls $f\cdot G:D\to K$ Stammfunktion, dann auch $(F\cdot g):D\to K$ mit 
	\begin{align}
		\int F\cdot g \D x = F(x) G(x) - \int f\cdot G\D x\notag
	\end{align}
\end{proposition}

\begin{proposition}[Integration durch Substitution]
	Sei $f:D\subset K\to K$, $D$ Gebiet, mit Stammfunktion $F:D\to K$ und sei $\phi:D\to D$ \gls{diffbar}. Dann hat $f(\phi(.))\cdot \phi'(.):D\to K$ eine Stammfunktion mit \begin{align}
		\proplbl{stammfunktion_substitution_eq}
		\int f(\phi(x))\cdot\phi'(x)\D x &= F(\phi(x))\notag
	\end{align}
\end{proposition}
\begin{proof}
	$F(\phi(.))$ ist nach der Kettenregel auf $D$ diffbar mit 
	\begin{align*}
		\frac{\D}{\D x} F(\phi(x)) &= F'(\phi(x)) \cdot \phi'(x) = f(\phi(x)) \cdot \phi'(x)
	\end{align*}
\end{proof}

\begin{proposition}
	Sei $f:I\subset \mathbb{R}\to \mathbb{R}$, $I$ offenes Intervall, $f(x)\neq 0$ auf $I$, dann gilt \begin{align}
		\int \frac{f'(x)}{f(x)} \D x = \ln \vert f(x) \vert\notag
	\end{align}
\end{proposition}

\chapter{Integration}
Integration kann betrachtet werden als
\begin{itemize}
	\item verallgemeinerte Summation, d.h. $\int_\mu f\D x$ ist Grenzwert von Summen
	\item lineare Abbildung $\int: \mathcal{F}\marginnote{$\mathcal{F}$: Menge der Funktionen}\to \mathbb{R}$ über $\int_a^b (\alpha f + \beta g)\D x = \alpha \int_a^b f \D x + \beta \int_a^b g \D x$ Funktionen, d.h. als Grundlage benötigt man ein "`Volumen"' (Maß) für allgemeine Mengen $M\subset\mathbb{R}$.
	
	Wir betrachten Funktionen $f:D\subset\mathbb{R}^n\to \mathbb{R}\cup \{ \pm \infty \}$, welche komponentenweise auf $f:D\subset\mathbb{R}\to K^k$ erweitert werden kann. Benutze $C^m \cong \mathbb{R}^{2m}$ für $K=\mathbb{C}$.
	
	Vgl. Buch:  Evans, Lawrence C.; Gariepy, Ronald F.: Measure theory and fine properties of functions
\end{itemize}
\section{Messbarkeit}\setcounter{equation}{0}
Wir führen zunächst das \lebesque-Maß ein und behandeln dann messbare Mengen und messbare Funktionen.

\subsection{\lebesque-Maß}
\begin{*definition}[Quader, Volumen]
	Wir definieren die Menge 
	\begin{align*}
		\mathcal{Q} &:= \left\{ I_1 \times \dotsc \times I_n \subset\mathbb{R}^n \mid I_j\subset\mathbb{R}\text{ beschränktes Intervall} \right\}
	\end{align*}
	$\emptyset$ ist auch als beschränktes Intervall zugelassen. $Q\in\mathcal{Q}$ heißt Quader.
	
	Sei $\vert I_j\vert :=$ Länge des Intervalls $I_j\subset\mathbb{R}$ (wobei $\vert\emptyset\vert = 0$), dann heißt 
	\begin{align}
		\proplbl{messbarkeit_definition_volumen_eq}
		v(Q) &:= \vert I_1\vert \cdot \dots \cdot \vert I_n\vert \quad \text{für}\; Q = I_1\times \dotsc\times I_n \in\mathbb{Q}\notag
	\end{align}
	Volumen von $Q$
\end{*definition}

\begin{*definition}[\person{Lebesgue}-Maß]
	Dafür betrachte eine (Mengen-) Funktion $\vert .\vert :\mathcal{P}(\mathbb{R}^n)\to [0,\infty]$ mit \begin{align}
		\proplbl{messbarkeit_definition_lebesque_mass}
		\vert \mu \vert &= \inf \left\{ \left. \sum_{j=1}^{\infty} v(Q_j) \;\right|\; M\subset\bigcup\limits_{j=1}^\infty Q_j, \; \text{$Q_j\in\mathcal{Q}$ Quader} \right\}\quad\forall M\subset\mathbb{R}^n,
	\end{align}
	die man \person{Lebegue}-Maß auf $\mathbb{R}^n$ nennt.
\end{*definition}

\begin{proposition}
	Es gilt: 
	\begin{align}
		M_1 \subset M_2 &\Rightarrow \vert M_1 \vert \le \vert M_2\vert\notag
	\end{align}
	und die Abbildung $\mu\mapsto \vert \mu\vert$ ist $\sigma$-subadditiv, d.h. 
	\begin{align}
		\left\vert \bigcup_{j=1}^\infty M_k\right\vert &\le \sum_{k=1}^\infty \vert M_k\vert, \quad\text{für } M_j\subset\mathbb{R}^n, \;j\in\mathbb{N}_{\ge 1}\notag
	\end{align}
\end{proposition}

\begin{proof}
	\begin{itemize}
		\item klar
		\item Finde Quader $Q_{k_j}$ mit $M_k\subset\bigcup Q_{k_j}$, $\sum v(Q_{k_j})\le\vert M_k\vert+\frac{\varepsilon}{2^k}$. Wegen $\bigcup_{k=1}^\infty M_k\subset \bigcup_{j,k=1}^\infty v(Q_{k_j}) \le \sum_{k=1}^\infty \vert M_k\vert + \epsilon$ folgt 
		\begin{align*}
		\left\vert\bigcup_{k=1}^\infty M_k\right\vert \le \sum_{j,k=1}^\infty v(Q_{k_j}) \le \sum_{k=1}^\infty \vert M_k\vert + \epsilon
		\end{align*}
	\end{itemize}
\end{proof}

\begin{*definition}[Nullmenge]
	$N\subset\mathbb{R}^n$ heißt Nullmenge, falls $\vert N \vert = 0$. Offenbar gilt:
\end{*definition}

\begin{conclusion}
	Es ist $v(Q) = \vert Q\vert$ $\forall Q\in\mathcal{Q}$
	
	Damit im folgenden Stets $\vert Q\vert$ statt $v(Q)$
\end{conclusion}
\begin{proof}
	$v(Q) = v(\cl Q)$ und $\vert Q\vert = \vert \cl Q\vert\Rightarrow Q$ abgeschlossen. Finde neue Quader $Q_j$ mit $Q\subset\bigcup Q_j$ und $\sum v(Q_j)\le\vert Q\vert+\varepsilon$. Da $Q$ kompakt $\Rightarrow$ Überdeckung durch endlich viele $Q_j$, geeignete Zerlegung von $Q_j\Rightarrow v(Q)\le\sum v(Q_j)\Rightarrow \vert Q\vert\le v(Q)\le\vert Q\vert+\varepsilon$
\end{proof}

\begin{*definition}
	Eine Eigenschaft gilt f.ü. auf $M\subset\mathbb{R}^n$, falls eine Nullmenge existiert, sodass die Eigenschaft $\forall x\in M\setminus N$ gilt. Man sagt auch, dass die Eigenschaft für fast alle $x\in M$ gilt.
\end{*definition}

\subsection{Messbare Mengen}
\begin{*definition}[messbar]
	Eine Menge $M\subset\mathbb{R}^n$ heißt messbar, falls 
	\begin{align}
		\vert \tilde{M}\vert = \vert \tilde{M}\cap M\vert + \vert \tilde{M}\setminus M\vert \quad\forall \tilde{M}\in\mathbb{R}\notag
	\end{align}
	Beim Nachweis der Messbarkeit muss man nur "`$\ge$"' prüfen.
\end{*definition}

\begin{proposition}
	\begin{enumerate}[label={(\alph*)}]
		\item $\emptyset$, $\mathbb{R}^n$ sind messbar
		\item $M\subset\mathbb{R}^n$ messbar $\Rightarrow$ $M^C = \mathbb{R}^n\setminus M$ messbar 
		\item $M_1, M_2, \dotsc\subset\mathbb{R}^n$ messbar $\Rightarrow$ $\bigcup_{j=1}^\infty M_j$, $\bigcap_{j=1}^\infty M_j$ messbar
	\end{enumerate}
\end{proposition}
\begin{proof}\hspace*{0pt}
	\begin{itemize}[topsep=\dimexpr-\baselineskip / 2\relax]
		\item wegen $\vert\emptyset\vert = 0$ und: $\vert \tilde{M}\vert \le \vert\tilde{M}\setminus\emptyset\vert = \vert\tilde{M}\vert$
		\item wegen $\tilde{M}\cap M = \tilde{M}\setminus M^C$, $\tilde{M}\setminus M = \tilde{M}\cap M^C$ $\Rightarrow$ Behauptung
		\item offenbar $M_1\cap...\cap M_k$ messbar und $M_1\cup...\cup M_k$ messbar, wähle $A=\bigcup M_j\Rightarrow A$ messbar
	\end{itemize}
\end{proof}

\begin{proposition}
	Es gilt: \begin{enumerate}[label={(\alph*)}]
		\item alle Quader sind Messbar ($Q\in\mathcal{Q}$)
		\item Offene und abgeschlossene $M\subset\mathbb{R}^n$ sind messbar
		\item alle Nullmengen sind messbar
		\item Sei $M\subset\mathbb{R}^n$ messbar, $M_0\subset\mathbb{R}^n$, beide Mengen unterscheiden sich voneinander nur um eine Nullmenge, d.h. $\vert (M\setminus M_0)\cup (M_0\setminus M)\vert = 0$ \\
		$\Rightarrow$ $M_0$ messbar.
	\end{enumerate}
\end{proposition}

\subsection{Messbare Funktionen}
\begin{*definition}[messbar]
	Eine Funktion $f:D\subset\mathbb{R}\to\overline{\mathbb{R}}$ heißt messbar, falls $D$ messbar ist und $f^{-1}(U)$ für jede offene Menge $U\subset\overline{\mathbb{R}}$ messbar ist.
\end{*definition}

\begin{*definition}[charakteristische Funktion]
	Für $M\subset\mathbb{R}^n$ heißt $\chi_\mu:\mathbb{R}^n\to\mathbb{R}$ mit \begin{align*}
		\chi_\mu = \begin{cases}
			1, &x\in M\\ 0, &x\in\mathbb{R}^n\setminus M
		\end{cases}
	\end{align*}
	charakteristische Funktion von $M$.
\end{*definition}

\begin{*definition}[Treppenfunktion]
	Eine Funktion $h:\mathbb{R}^n\to\mathbb{R}$ heißt Treppenfunktion, falls es $M_1, \dotsc, M_k\subset\mathbb{R}^n$  und $c_1,\dotsc,c_k\in\mathbb{R}$ gibt mit 
	\begin{align}
		h(x) = \sum_{j=1}^k a_j \chi_{\mu_j}(x)\notag
	\end{align}
\end{*definition}

\begin{*definition}[Nullfortsetzung]
	Für $f:D\subset\mathbb{R}^n\to\overline{\mathbb{R}}$ definieren wir die Nullfortsetzung $\overline{f}:\mathbb{R}^n\to\overline{\mathbb{R}}$ durch \begin{align}
		\overline{f}(x) &:= \begin{cases}
			f(x), &x\in D\\ 0,&x\in\mathbb{R}^n\setminus D
		\end{cases}\notag
	\end{align}
\end{*definition}

\rule{0.4\linewidth}{0.1pt}

\begin{example}
	Folgende Funktionen sind messbar
	\begin{itemize}
		\item Stetige Funktionen auf offenen und abgeschlossenen Mengen, insbesondere konstante Funktionen sind messbar
		\item Funktionen auf offenen und abgeschlossenen Mengen, die f.ü. mit einer stetigen Funktion übereinstimmen
		\item $\tan$, $\cot$ auf $\mathbb{R}$ (setzte z.b. $\tan\left(\frac{\pi}{2}+k\pi\right) = \cot(k\pi) = 0$ $\forall k$)
		\item $x\to \sin\frac{1}{x}$ auf $[-1,1]$ (setzte beliebigen Wert in $x=0$)
		\item $\chi_M:\mathbb{R}\to\mathbb{R}$ ist für $\vert\partial M\vert = 0$ messbar auf $\mathbb{R}$ (dann ist $\chi$ auf $\inn M$, $\ext M$ stetig)
	\end{itemize}
\end{example}

\subsection{Integral für Treppenfunktionen}
Sei $h:\mathbb{R}\to \mathbb{R}$ messbare Treppenfunktion mit \begin{align*}
	h &= \sum_{j=1}^{k} c_j \chi_{M_j}, \text{d.h. $c_j\in\mathbb{R}$, $M_j\subset\mathbb{R}$ messbar}
\end{align*}

\begin{*definition}[integrierbar, Integral, Integralabbildung]
	Sei $M\subset\mathbb{R}$ messbar.
	
	$h$ heißt \begriff{integrierbar} auf $M$, falls $\vert M_j\cap M\vert < \infty$ $\forall j: c_j\neq 0$ und \begin{align}
		\proplbl{integral_treppenfunktion_definition}
		\int_M h \D x := \int_M h(x) \D x := \sum_{j=1}^k c_m \vert M_j\cap M\vert
	\end{align}
	heißt (elementares) \begriff{Integral} von $h$ auf $M$.
	
	Menge der auf $M$ integrierbaren Treppenfunktionen ist \mathsymbol{T1}{$T^1(M)$}. $\int_M:T^1(M)\to\mathbb{R}$ mit $h\to \int_M h\D x$ ist die \begriff{Integral-Abbildung}.
\end{*definition}

Man verifiziert leicht
\begin{conclusion}
	\proplbl{integral_treppenfunktion_grundlegende_folgerung}
	Sei $M\subset\mathbb{R}^n$ messbar. Dann gilt:\begin{enumerate}[label={\alph*)}]
		\item (Linearität) Integralabbildung $\int_M:T^1(M)\to\mathbb{R}$ ist linear
		\item (Monotonie) Integral-Abbildung ist monoton auf $T^1(M)$ ,.d.h \begin{align*}
			h_1 \le h_2 \text{ auf $M$} \;\Rightarrow\;\int_M h_1 \D x \le \int_M h_2 \D x
		\end{align*}
		\item \proplbl{integral_treppenfunktion_grundlegende_folgerung_beschraenktheit}
		(Beschränktheit) Es ist $\vert \int_M h\D x \vert \le \int _M \vert h \vert \D x$ $\forall h\in T^{1}(M)$
		\item Für $h\in T^1(M)$ gilt: \\
		\begin{tabularx}{\linewidth}{X@{\ \ }c@{\ \ }X}
			\hfill $\displaystyle \int_M \vert h \vert \D x = 0$ & $\Leftrightarrow$ & $h=0$ \gls{fü} auf $M$
		\end{tabularx}
	\end{enumerate}

	\begin{underlinedenvironment}[Hinweis]
		$\int_M \vert h \vert \D x$ ist Halbnorm auf dem Vektorraum $T^1(M)$.
	\end{underlinedenvironment}
\end{conclusion}

\subsection{Erweiterung auf messbare Funktionen}
sinnvoll:
\begin{itemize}[topsep=-2\baselineskip]
	\item Linearität und Monotonie erhalten
	\item eine gewisse Stetigkeit der Integral-Abbildung
\end{itemize}
\vspace*{1em}
\begin{align}
	\proplbl{integral_messbare_funktion_forderung}
	h_k\to f\text{ in geeigneter Weise }\;\; &\Rightarrow\;\;\int_M h_k \D x \to \int_m f \D x
\end{align}
nach \propref{messbarkeit_funktion_approximation} sollte man in \eqref{integral_messbare_funktion_forderung} eine Folge von Treppenfunktionen $\{ h_k\}$ mit $h_k(x)\to f(x)$ \gls{fü} auf $M$ betrachten, \emph{aber} es gibt zu viele konvergente Folgen für einen konsistenten Integralbegriff.

\begin{example}
	\proplbl{integral_funktion_beispiel_striktere_konvergenz}
	Betrachte $f=0$ auf $\mathbb{R}$, wähle beliebige Folge $\{\alpha_k\}\subset\mathbb{R}$, dazu eine Treppenfunktion \begin{align*}
		h_k(x) = \begin{cases}
			k\cdot \alpha_k&\text{auf }(0,\frac{1}{k}) \\ 0&\text{sonst}
		\end{cases}
	\end{align*}
	Offenbar konvergiert $h_k$ gegen $0$ \gls{fü} auf $\mathbb{R}$ und man hat $h_k\to 0$ \gls{fü} auf $\mathbb{R}$ und $\int_{\mathbb{R}} h_k \D x = \alpha_k$
	\begin{tabularx}{\linewidth}{r@{\ \ }X}
		$\Rightarrow$ & je nach Wahl der Folge $\alpha_n$ liegt ganz unterschiedliches Konvergenzverhalten der Folge $\int_{\mathbb{R}} h_k \D x$ vor \\
		$\Rightarrow$ & kein eindeutiger Grenzwert in \eqref{integral_messbare_funktion_forderung} möglich \\
		$\Rightarrow$ & stärkerer Konvergenzbegriff in \eqref{integral_messbare_funktion_forderung} nötig
	\end{tabularx}
\end{example}

\textbf{Motivation}
	\hspace*{0pt}
	\begin{itemize}[topsep=\dimexpr -\baselineskip / 2\relax]
		\item Nur monotone Folgen von Treppenfunktionen, oder
		\item Beschränktheit aus \propref{integral_treppenfunktion_grundlegende_folgerung} erhalten
	\end{itemize}
	$\Rightarrow$ jeweils gleiches Ergebnis, jedoch ist die 1. Variante technisch etwas aufwendiger
	
	Beschränktheit aus \propref{integral_treppenfunktion_grundlegende_folgerung} \ref{integral_treppenfunktion_grundlegende_folgerung_beschraenktheit} bedeutet insbesondere \begin{align*}
		\left\vert \int_M h_k\D x - \int_M f \D x \right\vert = \left\vert \int_M h_k - f \D x \right\vert \le \int_M \vert h_k - f\vert \D x \quad\forall k
	\end{align*}
	
\textbf{man definiert:} $h_k\to f$ \gls{gdw} $\int_M \vert h_k - f\vert \D x\to 0$\\
	$\Rightarrow$ Integralabbildung stetig bezüglich dieser Konvergenz.
	
	Wegen $\int_M \vert h_k - h_l\vert \D x \le \int_m \vert h_k - f\vert \D x + \int_M \vert h_l -f \vert \D x$ müsste $\int_M \vert h_k - h_l\vert \D x$ klein sein $\forall h,l$ groß.

\subsection{\lebesque-Integral}
\begin{*definition}[$L^1$-\person{Chauchy}-Folge, \person{Lebesgue}-Integral]
	Sei $M\subset\mathbb{R}^n$ messbar, Folge $\{ h_k\}$ in $T^1(M)$ heißt \begriff{$L^1$-\person{Cauchy}-Folge} (kurz $L1$-CF), falls \begin{align*}
		\forall \epsilon > 0 \; \exists k_0\in\mathbb{N}:\;\int_M \vert h_k - h_l\vert \D x < \epsilon \quad\forall h,l > k_0
	\end{align*}
	
	\stepcounter{equation}
	Messbare Funktion $f:D\subset\mathbb{R}^n\to\overline{\mathbb{R}}$ heißt \begriff{integrierbar} auf $M\subset D$, falls Folge von Treppenfunktionen $\{ h_k\}$ in $T^1(M)$ existiert mit $\{ h_k\}$ ist $L1$-CF auf $M$ und $H_k\to f$ \gls{fü} auf $M$.\marginnote{\leqnos\begin{align}\proplbl{integral_funktion_definition}\,\end{align}Formel (3) unbekannt}[-1.5\baselineskip]
	
	Für integrierbare Funktion $f$ heißt eine solche Folge $\{h_k\}$ \begriff{zugehörige $L^1$-CF} auf $M$.
	
	Wegen\begin{align}
		\left\vert\int_M h_k\D x - \int_M h_l\D x\right\vert = \left\vert \int_M (h_k - h_l) \D x\right\vert \overset{\propref{integral_treppenfunktion_grundlegende_folgerung}}{\le} \int_M \vert h_k - h_l\vert \D x
	\end{align}
	ist $\{\int_M h_k\D x\}$ \person{Cauchy}-Folge in $\mathbb{R}$ und somit konvergent.
	
	Der Grenzwert \begin{align}
		\proplbl{integral_lebesque_funktion_definition}
		\int_m f \D x &:= \int_M f(x) \D x := \lim\limits_{k\to\infty} \int_M h_k\D x
	\end{align}
	
	heißt (\lebesque)-\begriff{Integral} von $f$ auf $M$.
\end{*definition}
\begin{underlinedenvironment}[Hinweis]
	Integrale unter dem Grenzwert in \eqref{integral_lebesque_funktion_definition} sind elementare Integrale gemäß \eqref{integral_treppenfunktion_definition}.
\end{underlinedenvironment}
\begin{underlinedenvironment}[Sprechweise]
	$f$ integrierbar auf $M$ bedeutet stets $f:D\subset\mathbb{R}^n\to\overline{\mathbb{R}}$ messbar und $M\subset D$ messbar
\end{underlinedenvironment}

\begin{*definition}[Menge der integrierbaren Funktionen]
Menge der auf $M$ integrierbaren Funktionen ist \mathsymbol*{L1}{$L^1$} \begin{align*}
	L^1(M) := \left\{ f:M\subset\mathbb{R}^n\to\overline{\mathbb{R}} \mid f \text{ integierbar auf $M$} \right\}
\end{align*}
\end{*definition}

\begin{remark}\vspace*{0pt}
	\begin{enumerate}[label={\alph*)},topsep=\dimexpr -\baselineskip / 2\relax]
		\item Integral in \eqref{integral_lebesque_funktion_definition} kann als vorzeichenbehaftetes Volumen des Zylinders im $\mathbb{R}^{n+1}$ unter (über) dem Graphen von $f$ interpretiert werden.
		\item Sei $0\le h_1 \le h_2 \le \dotsc$ monotone Folge von integrierbaren Treppenfunktionen mit $h_k\to f$ \gls{fü} auf $M$ und sei Folge $\{ \int_M h_k\D x\}$ in $\mathbb{R}$ beschränkt \\
		$\Rightarrow$ \eqref{integral_lebesque_funktion_definition} gilt und monotone Folge $\{ \int_m h_k \D x \}$ konvergiert in $\mathbb{R}$ (d.h. $\{ h_k \}$ ist $L^1$-CF zu $f$)
		\item $\{ h_k\}$ aus \propref{integral_funktion_beispiel_striktere_konvergenz} ist nur dann $L^1$-CF, falls $\alpha_k\to 0$.
	\end{enumerate}
\end{remark}

\textbf{Frage:} Ist die Definition des Integrals in \eqref{integral_lebesque_funktion_definition} unabhängig von der Wahl einer konkreten $L^1$-CF $\{ h_k\}$ zu $f$?

\begin{proposition}
	\proplbl{integral_funktion_unabhaengigkeit_leins_folge}
	Definition des Integrals in \eqref{integral_lebesque_funktion_definition} ist unabhängig von der speziellen Wahl einer $L^1$-CF $\{h_k\}$ zu $f$.
\end{proposition}

Vgl. Integral $\int_{M} h \D x$ einer Treppenfunktion gemäß \eqref{integral_treppenfunktion_definition} mit dem in \eqref{integral_lebesque_funktion_definition}:

Offenbar ist konstante Folge $\{ h_k\}$ mit $h_k = h$ $\forall k$ $L^1$-CF zu $h$ \\
$\xRightarrow[\eqref{integral_lebesque_funktion_definition}]{\propref{integral_funktion_unabhaengigkeit_leins_folge}}$ Integral $\int_M h \D x$ in \eqref{integral_lebesque_funktion_definition} stimmt mit elementarem Integral in \eqref{integral_treppenfunktion_definition} überein.

\begin{conclusion}
	Für eine Treppenfunktion stimmt das in \eqref{integral_treppenfunktion_definition} definierte elementare Integral mit dem in \eqref{integral_lebesque_funktion_definition} definierte Integral überein. Insbesondere ist der vor \eqref{integral_treppenfunktion_definition} eingeführte Begriff integrierbar mit dem in \eqref{integral_funktion_definition} identisch\\
	$\Rightarrow$ wichtige Identität \eqref{integral_treppenfunktion_definition} mit Treppenfunktion $\chi_M$ für $\vert M \vert < \infty$: \begin{align*}
		\vert M \vert &= \int_M 1\D x = \int_M \D x\quad\forall M\in\mathbb{R},\text{ $M$ messbar},
	\end{align*}
	d.h. das Integral liefert Maß für messbare Mengen.
\end{conclusion}

\begin{proof}[\propref{integral_funktion_unabhaengigkeit_leins_folge}]
	\NoEndMark
	beachte: alle Integrale im Beweis sind elementare Integrale gemäß \eqref{integral_treppenfunktion_definition}.
	
	\begin{itemize}
	
	\item Sei $f:M\subset\mathbb{R}\to\overline{\mathbb{R}}$ integrierbar und seien $\{ h_k\}$, $\{ \tilde{h}_k \}$ zugehörigen $L^1$-CF in $T^1(M)$.\\
	\begin{tabularx}{\linewidth}{r@{\ \ }X}
		$\Rightarrow$ & $\forall \epsilon > 0$ $\exists k_0$ mit \[ \int_M \vert (h_k + \tilde{h}_k) - (h_l + \tilde{h}_l)\vert \D x \le \int_M \vert h_k - h_l\vert + \vert \tilde{h}_k - \tilde{h}_l \vert \D x < \epsilon \quad\forall k,l\ge k_0  \] \\
		$\Rightarrow$ & $\{ h_k - \tilde{h}_k \}$ ist $L^1$-CF mit $(h_k - \tilde{h}_k)\to 0$ \gls{fü} auf $M$.
	\end{tabularx}

	Da $\{ \int_M h_k \D x\}$, $\{ \int_M \tilde{h}_k \D x \}$ in $\mathbb{R}$ konvergieren, bleibt zu zeigen: $\{ h_k\}$ ist $L^1$-CF 	in $T^1(M)$ mit $h_k\to 0$ \gls{fü} auf $M$
	\begin{flalign}
		\Rightarrow &\int_M h_k \D x \xrightarrow{k\to\infty} 0&
	\end{flalign}
	
	Da Konvergenz von $\{ \int_M h_k \D x \}$ bereits bekannt ist, reicht es, den Grenzwert für eine \gls{tf} zu zeigen.
	
	\item Wähle \gls{tf} derart, dass $\int_M \vert h_k - h_l \vert \D x \le \frac{1}{2^l}$ $\forall k\ge l$
	
	Fixiere $l\in\mathbb{N}$ und definiere $M_l := \{ x\in M\mid h_l(x) \neq 0 \}$, offenbar ist $M$ messbar mit $\vert M_l\vert < \infty$.
	
	Sei nun $\epsilon_l := \frac{1}{2^l \cdot \vert M_l\vert}$ falls $\vert M_l\vert > 0$ und $\epsilon_l = 1$ falls $\vert M_l\vert = 0$.
	
	Weiterhin sei $M_{l,k} := \{ x\in M_l \mid \vert h_k(x)\vert > \epsilon_l \}$, und für $k > l$ folgt
	\begin{align*}
		\left\vert\int_M h_k\D x \right\vert &\le \int_M \vert h_k\vert \D x = \int_{M_l} \vert h_k\vert \D x + \int_{M\setminus M_l} \vert h_k\vert \D x \\
		&\le \int_{ M\setminus M_{l,k}} \vert h_k\vert \D x + \int_{M_{l,k}} \vert h_k\vert \D x + \int_{M\setminus M_l} \vert h_k - h_l\vert \D x + \underbrace{\int_{M\setminus M_l} \vert h_l\vert \D x}_{=0} \\
		&\le \epsilon_l \vert M_l\vert + \int_{M_{l,k}} \vert h_k - h_l\vert \D x + \int_{M_{l,k}} \vert h_l\vert \D x + \frac{1}{2^l} \\
		&\le \frac{1}{2^l} + \frac{1}{2^l} + c_l \cdot \vert M_{l,k}\vert + \frac{1}{2^l}
	\end{align*}
	mit $c_l := \sup\limits_{x\in M} \vert h_l(x)\vert$, $\exists k_l > l$ mit \propref{messbarkeit_funktion_egorov} folgt $\vert \{ x\in M_l\mid \vert h_k(x)\vert > \epsilon_l \} \vert \le \frac{1}{2^l \cdot (c_l + 1)}$ $\forall k > k_l$
	
	\begin{tabularx}{\linewidth}{r@{\ \ }X}
	$\Rightarrow$ & $\displaystyle \left\vert \int_M h_k\D x \right\vert \le \frac{4}{2^l}$ $\forall k>k_l$ \\
	$\xRightarrow[\text{beliebig}]{l\in\mathbb{N}}$ & $\displaystyle \int_M h_k \D x\to 0$
	\end{tabularx}
	\end{itemize}
	\ \hfill\csname\InTheoType Symbol\endcsname
\end{proof}

\begin{proposition}[Rechenregeln]
	\proplbl{integral_funktion_rechenregeln}
	Seien $f$, $g$ integrierbar auf $M\subset\mathbb{R}^n$, $c\in\mathbb{R}$. Dann
	\begin{enumerate}[label={\alph*)}]
		\item \proplbl{integral_funktion_rechenregeln_a}
		(Linearität) $f\pm g$, $cf$ sind integrierbar auf $M$ mit \begin{align*}
			\int_M f \pm g \D x &= \int_M f\D x + \int_M g \D x \\
			\int_M c f \D x &= c \int_M f \D x
		\end{align*}
		\item \proplbl{integral_funktion_rechenregeln_b}
		Sei $\tilde{M}\subset\mathbb{M}$ messbar\\
		\begin{tabularx}{\linewidth}{r@{\ \ }X}
			$\Rightarrow$ & $f \chi_{\tilde{M}}$ ist integrierbar auf $M$ und $f$ ist integrierbar auf $\tilde{M}$ mit \[
				\int_M f\cdot \chi_{\tilde{M}} \D x = \int_{\tilde{M}} f \D x \]
		\end{tabularx}
		\item\proplbl{integral_funktion_rechenregeln_c}
		Sei $M = M_1\cup M_2$ für $M_1$, $M_2$ disjunkt und messbar \\
		\begin{tabularx}{\linewidth}{r@{\ \ }X}
			$\Rightarrow$ & $f$ ist integrierbar auf $M_1$ und $M_2$ mit 
		\end{tabularx}
		\begin{align*}
			\int_M f \D x &= \int_{M_1}  f \D x + \int_{M_2} f \D x
		\end{align*}
		\item\proplbl{integral_funktion_rechenregeln_d}
		Sei $f = \tilde{f}$ \gls{fü} auf $M$ \\\begin{tabularx}{\linewidth}{r@{\ \ }X}
			$\Rightarrow$ & $\tilde{f}$ ist integrierbar auf $M$ mit
		\end{tabularx}
		\begin{align*}
			\int_M f \D x = \int_M \tilde{f} \D x
		\end{align*}
		\item\proplbl{integral_funktion_rechenregeln_e}
		Die Nullfortsetung $\tilde{f}:\mathbb{R}^n\to\overline{\mathbb{R}}$ von $f$ (vgl. \propref{messbarkeit_funktion_nullfortsetzung}) ist auf jeder messbaren Menge $\tilde{M}\subset\mathbb{R}^n$ integrierbar mit \begin{align*}
			\int_{M\cap \tilde{M}} f \D x &= \int_{\tilde{M}} \overline{f}\D x
		\end{align*}
	\end{enumerate}
\end{proposition}

Aussage \ref{integral_funktion_rechenregeln_d} bedeutet, dass eine Änderung der Funktionswerte von $f$ auf einer Nullmenge das Integral nicht verändert.

\begin{proof}
	Seien $\{ h_k\}$ und $\{ \tilde{h}_k \}$ aus $T^1(\mathbb{R})^n$ $L^1$-CF zu $f$ und $g$.
	
	\begin{enumerate}[label={zu \alph*)},leftmargin=\widthof{\texttt{zu a) }},topsep=\dimexpr-\baselineskip/2\relax]
		\item Es ist $h_k + \tilde{h}_k\to f + g$ \gls{fü} auf $M$.
		
		Wegen \begin{align*}
			\int_M \vert (h_k + \tilde{h}_k) - (h_l + \tilde{h}_l)\vert \D x &\le \underbrace{\int_M \vert h_k - h_l\vert \D x}_{=\text{$L^1$-CF, $<\epsilon$}} + \underbrace{\int_M \vert \tilde{h}_k - \tilde{h}_l \vert \D x}_{=\text{$L^1$-CF, $<\epsilon$}}
		\end{align*}
		ist $\{ h_k + \tilde{h}_k\}$ $L^1$-CF zu $f+g$. \\
		$\Rightarrow$ $f+g$ ist integrierbar auf $M$ und Grenzübergang in \begin{align*}
			\int_M h_k + \tilde{h}_k \D x &= \int_M h_k \D x + \int_M \tilde{h}_k \D x
		\end{align*}
		liefert die Behauptung für $f+g$.
		
		Analog zu $cf$. Wegen $f - g$ = $f + (-g)$ folgt die letzte Behauptung.
		
		\item Offenbar ist $\{ \chi_{\tilde{m} h_k} \}$ $L^1$-CF zu $\chi_{\tilde{M}}f$ und $\{ h_k \}$ $L^1$-CF zu $f$ auf $\tilde{M}$.
		
		Mit \begin{align*}
			\int_M h_k \chi_{\tilde{M}} \D x &= \int_{\tilde{M}} h_k \D x\quad\forall k\in\mathbb{N}
		\end{align*}
		folgt die Behauptung durch Grenzübergang.
		\item Nach \ref{integral_funktion_rechenregeln_b} ist $f$ auf $M_1$ und $M_2$ integrierbar. Wegen $f = \chi_{M_1} f + \chi_{M_2} f$ folgt die Behauptung aus \ref{integral_funktion_rechenregeln_a} und \ref{integral_funktion_rechenregeln_b}.
		\item Da $\{ h_k\}$ auch $L^1$-CF zu $\tilde{f}$ ist, folgt die Integrierbarkeit mit dem gleichen Integral.
		\item Es ist $\{ \chi_{M\cap \tilde{M}} h_k \}$ $L^1$-CF zu $f$ auf $M\cap \tilde{M}$ und auch zu $\overline{f}$ auf $\tilde{M}$. Damit folgt die Behauptung.
	\end{enumerate}
\end{proof}

\begin{proposition}[Eigenschaften]
	\proplbl{integral_funktion_eigenschaften}
	Es gilt \begin{enumerate}[label={\alph*)}]
		\item \proplbl{integral_funktion_eigenschaften_integrierbarkeit}
		(Integierbarkeit) Für $f:M\subset\mathbb{R}^n\to\overline{\mathbb{R}}$ messbar gilt:\begin{center}
				$f$ integrierbar auf $M$ \ \ $\Leftrightarrow$ \ \  $\vert f \vert$ integrierbar auf $M$
		\end{center}
		\item\proplbl{integral_funktion_eigenschaften_beschraenktheit}
		(Beschränktheit)
		Sei $f$ integrierbar auf $M$, dann \begin{align*}
			\left\vert \int_M f \D x \right\vert &\le \int_M \vert f \vert \D x
		\end{align*}
		\item\proplbl{integral_funktion_eigenschaften_monotonie}
		(Monotonie)
		Seien $f$, $g$ integrierbar auf $M$. Dann \begin{center}
			$f\le g$ \gls{fü} auf $M$ \ \ $\Rightarrow$ \ \ $\displaystyle\int_M f\D x \le \int_M g \D x$
		\end{center}
		\item\proplbl{integral_funktion_eigenschaften_nullfunktion}
		 Sei $f$ integrierbar auf $M$, dann \begin{center}
				$\displaystyle \int_M \vert f \vert \D x = 0$\ \ $\Leftrightarrow$ \ \ $f = 0$ \gls{fü}
		\end{center}
	\end{enumerate}
	In Analogie zur Treppenfunktion ist $\Vert f\Vert _1 := \int_M \vert f \vert \D x$ auf $L^1(M)$ eine Halbnorm, aber keine Norm ($\Vert f \Vert = 0$ $\cancel{\Leftrightarrow}$ $f = 0$). $\Vert f\Vert_1$ heißt \begriff{$L^1$-Halbnorm} von $f$.
\end{proposition}

\begin{underlinedenvironment}[Hinweis]
	Eine lineare Abbildung $A:X\to Y$ ist beschränkt, wenn $\Vert Ax\Vert_Y \le c\Vert x \Vert _X$ \\
	$\Rightarrow$ Begriff der Beschränktheit in \ref{integral_funktion_eigenschaften_beschraenktheit}.
\end{underlinedenvironment}
\begin{proof}\hspace*{0pt}
	\NoEndMark
	\begin{enumerate}[label={zu \alph*)},topsep=\dimexpr -\baselineskip / 2\relax,leftmargin=\widthof{\texttt{zu b)\ }}]
		\item Sei $f$ integrierbar auf $M$ und sei $\{ h_k \}$ $L^1$-CF zu $f$ \\
		\ $\Rightarrow$ $\vert h_k \vert\to \vert f \vert$ \gls{fü} auf $M$.
		
		Wegen $\int_M \left\vert \vert h_k \vert - \vert h_l \vert\right\vert \D x$\marginnote{$\vert\vert \alpha\vert - \vert\beta\vert\vert \le \vert \alpha - \beta\vert$ $\forall \alpha,\beta\in\mathbb{R}$} $\overset{\cref{integral_treppenfunktion_grundlegende_folgerung}}{\le}$ $\int_M \vert h_k - h_l \vert \D x$ ist $\{ \vert h_k\vert \}$ $L^1$-CF zu $\vert f \vert$ \\
		\ $\Rightarrow$ $\vert f \vert$ ist integrierbar.
		
		\emph{beachte:} andere Richtung später
		
		\item Für eine $L^1$-CF $\{ h_k\}$ zu $f$ gilt nach \propref{integral_treppenfunktion_grundlegende_folgerung} c): \begin{align*}
			\left\vert \int_M h_k \D x \right\vert \le \int_M \vert h_k\vert \D x
		\end{align*}
		Da $\{ \vert h_k \vert \}$ $L^1$-CF zu $\vert f \vert$ ist, folgt die Behauptung durch Grenzübergang.
		
		\item Nach den Rechenregeln ist $g - f$ integrierbar, wegen $\vert g - f\vert = g - f$ \gls{fü} auf $M$ folgt \begin{align*}
			0 \le \left\vert \int_M g - f\D x\right\vert\overset{\ref{integral_funktion_eigenschaften_beschraenktheit}}{\le} \int_M \vert g - f\vert \D x \overset{ \cref{integral_funktion_rechenregeln}\;\ref{integral_funktion_rechenregeln_a}}{=} \int_M g \D x - \int_M f \D x
		\end{align*}
		$\Rightarrow$ Behauptung
		
		\item[zu a)] für "`$\Leftarrow$"' wähle $f^\pm$ ($ f = f^+ - f^-$) jeweils eine monotone Folge von \gls{tf} $\{ h_k^\pm \}$ gemäß \propref{messbarkeit_funktion_existenz_monotone_treppenfunktionen}. Folglich liefert $H_k = h_k^+ - h_k^-$ eine Folge von \gls{tf} mit $h_k\to f$ \gls{fü} auf $M$.
		
		Wegen $\vert h_k \vert \le \vert f \vert$ \gls{fü} auf $M$ ist $\int_M \vert h_k\vert \D x \le \int_M \vert f \vert \D x$.
		
		Folglich ist die monotone Folge $\int_M \vert h_k\vert \D x$ in $\mathbb{R}$ beschränkt \\
		$\Rightarrow$ konvergent.
		
		Da $h_k^\pm$ jeweils das Vorzeichen wie $f^\pm$ haben und die Folge monoton ist, gilt \begin{align*}
			\left\vert \vert h_l\vert - \vert h_k\vert \right\vert &= \vert h_l\vert - \vert h_k\vert = \vert h_l -  h_k \vert \quad\forall l>k
		\end{align*}
		und somit auch \begin{align*}
			\int_M \vert h_l - h_k \vert \D x &= \int_M \vert h_l\vert - \vert h_k\vert \D x = \left \vert \int_M \vert h_l\vert \D x - \int_M \vert h_k\vert \D x\right\vert \quad\forall l>k
		\end{align*}
		Als konvergente Folge ist $\{ \int_M \vert h_k \vert \D x \}$ \person{Cauchy}-Folge in $\mathbb{R}$ und folglich ist $\{ h_k \}$ $L^1$-CF und sogar $L^1$-CF zu $f$ \\
		$\Rightarrow$ $f$ integrierbar
		\item Für $f=0$ \gls{fü} auf $M$ ist offenbar $\int_M \vert f\vert \D x = 0$.
		
		Sei nun $\int_M \vert f \vert \D x = 0$, mit $M_k := \{ x\in M \mid \vert f \vert \ge \frac{1}{k} \}$ $\forall k\in\mathbb{N}$ ist \begin{align*}
			0 = \int_{M\setminus M_k} \vert f \vert \D x + \int_{M_k} \vert f \vert \D x \ge \int_{M\setminus M_k}0 \D x + \int_{M_k}\frac{1}{k}\D x \ge \frac{1}{k}\vert M_k\vert \ge 0
		\end{align*}
		\begin{tabularx}{\linewidth}{r@{\ \ }X}
			$\Rightarrow$ & $\vert M_k\vert = 0$ $\forall k$, wegen $\{ f \neq 0\} = \bigcup_{k\in\mathbb{N}} M_k$ \\
			$\Rightarrow$ & $\displaystyle \vert \{ f\neq 0 \} \vert \le \sum_{k=1}^\infty \vert M_k\vert= 0$ \\
			$\Rightarrow$ & Behauptung\hfill\csname\InTheoType Symbol\endcsname
		\end{tabularx}
	\end{enumerate}
\end{proof}

\begin{conclusion}
	\proplbl{integral_funktion_lemma_weitere_eigenschaften}
	Sei $f$ auf $M$ integrierbar\begin{enumerate}[label={\alph*)}]
		\item Für $\alpha_1$, $\alpha_2\in\mathbb{R}$ gilt:\begin{center}
			$\alpha_1\le f \le \alpha_2$ \gls{fü} auf $M$ \ \ $\Rightarrow$ \ \ $\displaystyle \alpha_1 \vert M \vert \le \int_M f \D x \le \alpha_2 \vert M \vert$
		\end{center}
		\item Es gilt $f\ge 0$ \gls{fü} auf $M$ \ \ $\Rightarrow$ \ \ $\int_M f \D x\ge 0$
		\item Es gilt: $\tilde{M}\subset M$ messbar, $f\ge 0$ \gls{fü} auf $M$ \\
		\ $\Rightarrow$ \ \ $\displaystyle \int_{\tilde{M}} f \D x \le \int_M f \D x$
		
		(linkes Integral nach \propref{integral_funktion_rechenregeln} \ref{integral_funktion_rechenregeln_b})
	\end{enumerate}
\end{conclusion}

\begin{proof}\hspace*{0pt}
	\NoEndMark
	\begin{enumerate}[label={zu \alph*)},topsep=\dimexpr-\baselineskip/2\relax,leftmargin=\widthof{\texttt{zu a)\ }}]
		\item 
		Wegen $\int_M \alpha_j \D x = \alpha_j \vert M \vert $ für $\vert M \vert$ endlich folgt a) direkt aus der Monotonie des Integrals.
		\item folgt mit $\alpha_1=0$ aus a)
		\item folgt, da $\chi_{\tilde{M}}\cdot f \le f$ \gls{fü} auf $M$ und aus der Monotonie\hfill\csname\InTheoType Symbol\endcsname
	\end{enumerate}
\end{proof}

In der Vorüberlegung zum Integral wurde eine gewisse Stetigkeit der Integralabbildung angestrebt. Das Integral ist bezüglich der $L^1$-Halbnorm stetig.
\begin{proposition}
	\proplbl{integral_funktionen_differenz_null_gleichheit}
	Seien $f$, $f_k:D\subset\mathbb{R}^n\to\overline{\mathbb{R}}$ integrierbar auf $M\subset\mathbb{R}^n$ und sei \begin{align*}
		& \lim\limits_{k\to\infty} \int_M \vert f_k - f\vert \D x = 0 \quad(\Vert f_k - f\Vert\to0)\\
		\Rightarrow\;\;&\lim\limits_{k\to\infty} \int_M f_k \D x = \int_M f\D x
	\end{align*}
	Weiterhin gibt es eine Teilfolge $\{ f_{k'}\}$ mit $f_{k'}\to f$ \gls{fü} auf $M$.
\end{proposition}

\begin{proof}
	\NoEndMark
	Aus der Beschränktheit nach \propref{integral_funktion_eigenschaften} folgt \begin{align*}
		\left\vert\int_M f_k\D x - \int_M f \D x \right\vert \le \int_M \vert f_k - f\vert \D x \xrightarrow{k\to 0} 0
	\end{align*}
	\ $\Rightarrow$\ \ 1. Konvergenzaussage
	
	Wähle nun eine \gls{tf} $\{ f_{k_l}\}_l$ mit $\int_M \vert f_{k_l} - f\vert \D x \le \frac{1}{2^{l+1}}$ $\forall l\in\mathbb{N}$.
	
	Für $\epsilon>0$ sei $M_\epsilon := \{ x\in M \mid \limsup\limits_{l\to\infty} \vert f_{k_l} - f\vert > \epsilon \}$ \\
	\begin{tabularx}{\linewidth}{r@{\ \ }X}
	$\Rightarrow$ & $\displaystyle M_\epsilon \subset\bigcup_{l=j}^\infty \{ \vert f_{k_l} - f \vert > \epsilon \}$ $\forall j\in\mathbb{N}$ \\
	$\Rightarrow$ & $\displaystyle M_\epsilon \le \sum_{l=j}^\infty \left\vert \left\{ f_{k_l} - f\vert > \epsilon \right\} \right\vert \le \frac{1}{\epsilon} \sum_{l=j}^\infty \int _M \vert f_{k_l} - f\vert \D x \le \frac{1}{\epsilon} \sum_{l=j}^\infty \frac{1}{2^{l+1}} = \frac{1}{2^j \epsilon}\quad\forall j\in\mathbb{N}$\\
	$\Rightarrow$ & $M_\epsilon = 0$ $\forall\epsilon > 0$ \\
	$\Rightarrow$ & $f_{k_l} \xrightarrow{l\to\infty} f$ \gls{fü} auf $M$ \hfill\csname\InTheoType Symbol\endcsname
	\end{tabularx}
\end{proof}

\begin{proposition}[Majorantenkriterium]
	\proplbl{integral_funktion_majorantenkriterium}
	Seien $f$, $g:D\subset\mathbb{R}^n\to\overline{\mathbb{R}}$ messbar, $M$ messbar, $\vert f \vert \le g$ \gls{fü} auf $M$, $g$ integrierbar auf $M$ \\
	$\;\Rightarrow$ $f$ integrierbar auf $M$
	
	Man nennt $g$ auch \begriff{integrierbare Majorante} von $f$.
\end{proposition}

\begin{lemma}
	\proplbl{integral_funktion_lemma_majorante}
	Sei $f:D\subset\mathbb{R}^n\to\overline{\mathbb{R}}$ messbar auf $M$, sei $f\ge 0$ auf $M$ und sei $\{ h_k\}$ Folge von Treppenfunktionen mit \begin{align}
		\proplbl{integral_funktion_lemma_majorante_eq}
		0 \le h_1 \le h_2 \le \dotsc \le f\quad \text{ und }\quad  \int_M h_k \D x\text{ beschränkt}
	\end{align}
	$\Rightarrow$ $\{ h_k\}$ ist $L^1$-CF zu $f$ und falls $\{ h_k\}\to f$ \gls{fü} auf $M$ ist $f$ integrierbar (vgl \propref{messbarkeit_funktion_existenz_monotone_treppenfunktionen})
\end{lemma}

\begin{proof}
	Offenbar sind alle $h_k$ integrierbar und wegen der Monotonie gilt \begin{align*}
		\left\vert \int_M h_k\D x - \int_M h_l\D x \right\vert &=\int_M \vert h_k - h_l\vert \D x\quad\forall k\ge l
	\end{align*}
	Da $\{ \int_M h_k \D x \}$ konvergent ist in $\mathbb{R}$ als monoton beschränkte Folge ist diese CF in $\mathbb{R}$ \\
	$\Rightarrow$ $\{ h_k\}$ ist $L^1$-CF
	
	Falls noch $h_k\to f$ \gls{fü} $\Rightarrow$ $\{h_k\}$ ist $L^1$-CF zu $f$ $\Rightarrow$ $f$ ist integrierbar
\end{proof}

\begin{proof}[\propref{integral_funktion_majorantenkriterium}]
	\NoEndMark
	(mit $f$ auch $\vert f \vert$ mesbbar nach \propref{messbarkeit_funktion_existenz_monotone_treppenfunktionen})
	
	Es existiert eine Folge $\{ h_k \}$ von Treppenfunktionen mit \begin{align*}
		0 \le h_1 \le h_2 \le \dotsc \le \vert f \vert \le g
	\end{align*}
	auf $M$ und $\{ h_k\}\to\vert f \vert$ \gls{fü} auf $M$.
	
	Da $\{ \int_M h_k \D x \}$ beschränkt ist in $\mathbb{R}$ da $g$ integrierbar ist \\
	{\renewcommand{\arraystretch}{1.3}\begin{tabularx}{\linewidth}{r@{\ \ }X}
		$\xRightarrow{\text{\propref{integral_funktion_lemma_majorante}}}$ & $\{ h_k\}$ ist $L^1$-Cf zu $\vert f \vert$ \\
		$\Rightarrow$ & $\vert f \vert$ integrierbar \\
		$\xRightarrow{\text{\propref{integral_funktion_eigenschaften}}}$ & $f$ integrierbar auf $M$\hfill\csname\InTheoType Symbol\endcsname
	\end{tabularx}}
\end{proof}

\begin{conclusion}
	Seien $f$, $g:M\subset\mathbb{R}^n\to\overline{\mathbb{R}}$ messbar, $\vert M \vert$ endlich. Dann\begin{enumerate}[label={\alph*)}]
		\item Falls $f$ beschränkt ist auf $M$, dann ist $f$ integrierbar auf $M$
		\item Sei $f$ beschränkt und $g$ integrierbar auf $M$\\
		$\Rightarrow$\ \ $f\cdot g$ ist integrierbar auf $M$
	\end{enumerate}
\end{conclusion}
\begin{underlinedenvironment}[Hinweis]
	Folglich sind stetige Funktionen auf kompaktem $M$ integrierbar (vgl. Theorem von Weierstraß)
\end{underlinedenvironment}

\begin{proof}
	Sei $\vert f \vert \le \alpha$ auf $M$ für $\alpha\in\mathbb{Q}$
	\begin{enumerate}[label={zu \alph*)},topsep=\dimexpr-\baselineskip/2\relax,leftmargin=\widthof{\texttt{zu a)\ }}]
	\item 
	$\Rightarrow$ \ \ konstante Funktion $f_1 = \alpha$ ist integrierbare Majorante von $\vert f \vert$
	\item Mit $f_2 = \alpha\cdot \vert g \vert$ ist $f_2$ integrierbare Majorante zu $\vert f\cdot g\vert$ \ \ 
	$\xRightarrow[\text{kriterium}]{\text{Majoranten-}}$ Behauptung
\end{enumerate}
\end{proof}

\subsection{Grenzwertsätze}
$\int_M f_k\D x \xrightarrow{?} \int_M f\D x$ Vertauschbarkeit von Integration und Grenzübergang ist zentrale Frage $\to$ grundlegende Grenzwertsätze $\int_M \vert f_k - f\vert \D x \to 0$
\begin{theorem}[Lemma von Fatou]
	\proplbl{integral_grenzwertsatz_fatou}
	Seien $f_k:D\subset\mathbb{R}^n\to [0,\infty]$ integrierbar auf $M\subset D$ $\forall k\in\mathbb{N}$ \\
	\ $\Rightarrow$ $f(x) := \liminf\limits_{k\to\infty} f_k(x)$ $\forall x\in M$ ist integrierbar auf $M$ und \begin{align*}
		\left( \int_M f\D x =\right) \int_M \liminf_{k\to\infty} f_k \D x &\le \liminf\limits_{k\to\infty} \int_M f_k \D x,
	\end{align*}
	falls der Grenzwert rechts existiert.
\end{theorem}

Keine Gleichheit hat man z.B. für $\{ h_k\}$ aus \propref{integral_funktion_beispiel_striktere_konvergenz} mit $\alpha_k = 1$ $\forall k$ \begin{align*}
	h_k &= \begin{cases}
		h\cdot \alpha_k & x\in \left[0,\frac{1}{k}\right] \\
		0 & \text{sonst}
	\end{cases}
	\intertext{Dann}
	\int_M \liminf\limits_{k\to\infty} h_k \D x &= \int_M 0 \D x = 0 < \liminf\limits_{k\to\infty} \int_{\mathbb{R}} h_k \D x = 1
\end{align*}

\begin{proof}
	Auf $M$ ist $0\le g_k := \inf\limits_{l\ge k} f_l \le f_j$ $\forall j\ge k$, $k\in\mathbb{N}$, $g_1 \le g_2 \le \dotsc$ und $\lim\limits_{k\to\infty} g_k = \liminf\limits_{k\to\infty} f_k = f$
	
	Alle $g_k$ sind messbar nach \propref{messbarkeit_funktionen_komposition}, \propref{integral_funktion_majorantenkriterium}
	
	Für jedes $k\in\mathbb{N}$ wählen wir gemäß \propref{messbarkeit_funktion_existenz_monotone_treppenfunktionen} eine Folge $\{ h_{k_l} \}_l$ von Treppenfunktionen mit $0\le h_{k_1} \le h_{k_2} \le \dotsc \le g_k$, $h_{k_l}\xrightarrow{l\to\infty} g_k$ \gls{fü} auf $M$.
	
	Nach \propref{integral_funktion_lemma_majorante} ist $\{ h_{k_l}\}_l$ $L^1$-CF zu $g_k$.
	
	Anwendung von \propref{messbarkeit_funktion_egorov} auf $g_k - f$ auf $B_k(0)\cap M$ \\
	$\Rightarrow$ $\exists A_k' \subset\mathbb{R}^n$ messbar mit $\vert A_k'\vert \le \frac{1}{2^{k+1}}$ und (ggf. \gls{tf}) $\vert g_k - f\vert < \frac{1}{k}$ auf $(B_k(0) \cap M)\setminus A_K'$.
	
	Analog für Folge $h_{k_l}\xrightarrow{l\to\infty} g_k: \exists A_K''\subset\mathbb{R}^k$ mit $\vert A_k''\vert < \frac{1}{2^{k+1}}$ und (evtl. \gls{tf}) $\vert h_{k_l} - g_k \vert < \frac{1}{k}$ auf $(B_k(0)\cap M)\setminus A_k''$
	
	Setzte $A_k = A_k'\cup A_k''$, offenbar $\vert A_k\vert < \frac{1}{2^k}$, $h_k := h_{k_l}$
	
	Definiere rekursiv $\tilde{h}_1 := h_1$, $\tilde{h}_k := \max( \tilde{h}_{k-1}, h_k)$ \\
	$\Rightarrow$ $h_k \le \tilde{h}_k \le g_k \le f_k$ und $\tilde{h}_{k-1} \le \tilde{h}_k$ $\forall k\in\mathbb{N}$ \\
	$\Rightarrow$ $\vert \tilde{h}_k - f\vert \overset{\triangle-\text{Ungl}}{\le}\vert \tilde{h}_k - g_k\vert + \vert g_k -f \vert \le \vert h_k - g_k\vert + \vert g_k -f \vert \le \frac{2}{k}$ auf $(B_k(0)\cap M)\setminus A_k$.
	
	Mit $\tilde{A}_l := \bigcup_{k=l}^\infty A_k$ folgt $\vert \tilde{A}_l\vert \le \frac{1}{2^{l-1}}$ und $\vert \tilde{h}_k - f \vert \le \frac{2}{k}$ auf $(B_k(0)\cap M)\setminus \tilde{A}_l$ $\forall k>l$.
	
	Folglich $\tilde{h}_l\to f$ \gls{fü} auf $M$ und wegen der Monotonie ist $\{ \tilde{h}_k\}$ $L^1$-CF zu $f$ \\
	$\Rightarrow$ $\int_M f \D x \overset{\text{Def}}{=} \lim\limits_{k\to\infty} \int_M \tilde{h}_k \D x \overset{\text{Monotonie}}{\le}\liminf\limits_{k\to\infty} \int_M f_k \D x$\\
	$\Rightarrow$ Behauptung
\end{proof}

\begin{theorem}[Monotone Konvergenz]
	\proplbl{integral_grenzwertsatz_monotone_konvergenz}
	Seien $f_k:D\subset\mathbb{R}^n\to\overline{\mathbb{R}}$ integrierbar auf $M\subset D$ $\forall k\in\mathbb{N}$ mit $f_1 \le f_2 \le \dotsc $ \gls{fü} auf $M$ \\
	\ $\Rightarrow$ $f$ ist integrierbar auf $M$ und \begin{align*}
		\left( \int_M f \D x = \right) \int_M \lim\limits_{k\to\infty} f_k(x) \D x &= \lim\limits_{k\to\infty} \int_M f_k \D x
	\end{align*}
	falls der rechte Grenzwert existiert.
\end{theorem}

\begin{remark}
	\propref{integral_grenzwertsatz_monotone_konvergenz} bleibt richtig, falls man $f_1 \ge f_2 \ge \dotsc$ \gls{fü} auf $M$ hat.
	
	Ferner ist wegen der Monotonie die Beschränktheit der Folge $\{ \int_M f_k \D x \}$ für die Existenz des Grenzwertes ausreichend.
\end{remark}

\begin{proof}[\propref{integral_grenzwertsatz_monotone_konvergenz}]
	Nach \propref{integral_grenzwertsatz_fatou} ist $f - f_1 = \lim\limits_{k\to\infty} f_k - f_1$ integrierbar auf $M$ und damit auch $f = (f - f_1) + f_1$
	\begin{align*}
	\Rightarrow\;\;\int_M f - f_1 \D x &\le \lim\limits_{k\to\infty} \int_M f_k - f_1 \D x\\
	&= \lim\limits_{k\to\infty} \int_M f_k \D x - \int_M f_1 \D x
	\overset{\text{Monotonie}}{\le} \int_M f \D x - \int_M f_1\D x\\
	&= \int_M f - f_1 \D x
	\end{align*}
\end{proof}

\begin{theorem}[Majorisierte Konvergenz]
	\proplbl{integral_grenzwertsatz_majorisierte_konvergenz}
	Seien $f_k$, $g:D\subset\mathbb{R}^n\to\overline{\mathbb{R}}$ messbar für $k\in\mathbb{N}$ und sei $g$ integrierbar auf $M\subset D$ mit $\vert f_k\vert \le g$ \gls{fü} auf $M$ $\forall k\in\mathbb{N}$ und $f_k\to:f$ \gls{fü} auf $M$
	\begin{align}
		\proplbl{integral_grenzwertsatz_majorisierte_konvergenz_eq}
		\Rightarrow\;\;\lim\limits_{k\to\infty} \int_M \vert f_k - f\vert \D x = 0
	\end{align}
	und \begin{align*}
		\left(\int_M f\D x = \right) \int_M \lim\limits_{k\to\infty} f_k \D x = \lim\limits_{k\to\infty} \int_M f_k \D x,
	\end{align*}
	wobei alle Integrale existieren.
\end{theorem}

\begin{proof}
	Nach dem Majorantenkriterium sind alle $f_k$ \gls{fü} integrierbar auf $M$.
	
	Nach \propref{integral_grenzwertsatz_fatou} gilt:\begin{align*}
		\int_M 2g \D x = \int_M \liminf\limits_{k\to\infty} \vert 2g - \vert f_k - f\vert \vert \D x \le \liminf\limits_{k\to\infty} \int_M 2g - \vert f_k - f \vert \D x
	\end{align*}
	$\Rightarrow$ $0 = \liminf\limits_{k\to\infty} -\int_M \vert f_k - f\vert \D x$ $\Rightarrow$ \eqref{integral_grenzwertsatz_majorisierte_konvergenz_eq} $\xRightarrow{\text{\propref{integral_funktionen_differenz_null_gleichheit}}}$ Behauptung
\end{proof}

\begin{conclusion}
	\proplbl{integral_grenzwertsatz_folgerung_fatou}
	Seien $f_k:D\subset\mathbb{R}^n\to\overline{\mathbb{R}}$ integrierbar auf $M$ $\forall k\in\mathbb{N}$. Sei $\vert M \vert < \infty$ und konvergieren die $f_k \to: f$ gleichmäßig auf $M$ \\
	\ $\Rightarrow$ $f$ ist integrierbar auf $M$ und $\int_M f \D x = \lim\limits_{k\to\infty} \int_M f_k \D x$
\end{conclusion}

\begin{proof}
	$\exists k_0\in \mathbb{N}$ mit $\vert f_k(x) \vert \le \vert f_{k_0}(x) + 1\vert$ $\forall x\in\mathbb{M}$, $k > k_0$.
	
	Da $f_{k_0}+1$ integrierbar auf $M$ folgt die Behauptung aus \propref{integral_grenzwertsatz_majorisierte_konvergenz}.
\end{proof}

\begin{theorem}[Mittelwertsatz der Integralrechnung]
	\proplbl{integral_grenzwertsatz_mittelwertsatz_integralrechnung}
	Sei $M\subset\mathbb{R}^n$ kompaket und zusammenhängend, und sei $f:M\to\mathbb{R}$ stetig
	\begin{align*}
	\Rightarrow\;\;\exists \xi\in M: \int_M f \D x = f(\xi) \cdot \vert M \vert
	\end{align*}
\end{theorem}

\begin{proof}
	Aussage klar für $\vert M \vert = 0$, deshalb wähle $\vert M \vert > 0$.
	
	Da $f$ stetig auf $M$ kompakt \\
	{\renewcommand{\arraystretch}{1.3}\begin{tabularx}{\linewidth}{r@{\ \ }X}
	$\xRightarrow[\propref{satz_von_weierstrass}]{\text{Weierstrass}}$ & $\exists$ Minimalstelle $x_1\in M$, Maximalstelle $x_2\in M$ und $\displaystyle\gamma := \int_M f \D x$ \\ $\xRightarrow{\text{\cref{integral_funktion_lemma_weitere_eigenschaften}}}$ & $f(x_1) \le \frac{\gamma}{\vert M \vert} \le f(x_2)$ \\
	$\xRightarrow[\propref{zwischenwertsatz}]{\text{Zwischenwertsatz}}$ & $\displaystyle\exists \xi\in M: f(\xi) = \frac{\gamma}{\vert M \vert}$ \\
	$\Rightarrow$ & Behauptung
	\end{tabularx}}
\end{proof}

\subsection{Parameterabhängige Integrale}
Sei $M\subset\mathbb{R}^n$ messbar, $P\subset\mathbb{R}^n$ eine Menge von Parametern und sei $f:M\times P\to\mathbb{R}$.

Betrachte parameterabhängige Funktion \begin{align}
\proplbl{integral_parameterabhaengig_grundgleichung_eq}
	F(p) &:= \int_M f(x,p) \D x
\end{align}

\begin{proposition}[Stetigkeit]
	Seien $M\subset\mathbb{R}^n$ messbar, $P\subset\mathbb{R}^n$ und $f:M\times P\to\mathbb{R}$ eine Funktion mit \begin{itemize}
		\item $f(\,\cdot\,,p)$ messbar $\forall p\in P$
		\item $f(x,\,\cdot\,)$ stetig für \gls{fa} $x\in M$
	\end{itemize}
Weiterhin gebe es integrierbare Funktion $g:M\to\mathbb{R}$ mit \begin{itemize}
		\item $\vert f(x,p)\vert \le g(x)$ für \gls{fa} $x\in M$
	\end{itemize}

$\Rightarrow$ Integrale in \eqref{integral_parameterabhaengig_grundgleichung_eq} existieren $\forall p\in P$ und $F$ ist stetig auf $P$.
\end{proposition}

\begin{proof}
	$f(\,\cdot\, ,p)$ ist integrierbar auf $M$ $\forall p\in P$ nach \cref{integral_funktion_majorantenkriterium}.
	
	Fixiere $p$ und $\{ p_k\}$ in $P$ mit $p_k\to p$.
	
	Setzte $f_k(x) := f(x, p_k)$
	
	Stetigkeit von $f(x,\,\cdot\,)$ liefert $f_k(x) = f(x, p_k)\xrightarrow{x\to\infty} f(x,p)$ für \gls{fa} $x\in M$.\\ \begin{tabularx}{\linewidth}{r@{\ \ }X}
	$\xRightarrow{\cref{integral_grenzwertsatz_majorisierte_konvergenz}}$ & $F(p_k) = \int_M f_k(x) \D x \to \int_M f(x,p)\D x = F(p)$ \\
	$\xRightarrow[\text{beliebig}]{p\in P}$ & Behauptung
	\end{tabularx}
\end{proof}

\begin{proposition}[Differenzierbarkeit]
	Seien $M\subset\mathbb{R}^n$ messbar, $P\subset\mathbb{R}^m$ offen und $f:M\times P\to\mathbb{R}$ mit $f(\,\cdot\, ,p)$ integrierbar auf $M$ $\forall p\in P$. und \begin{itemize}
		\item $f(x,\,\cdot\,)$ stetig \gls{diffbar} auf $P$ für \gls{fa} $x\in M$
	\end{itemize}
	Weiterhin gebe es eine integrierbare Funktion $g:M\to\mathbb{R}$ mit \begin{itemize}
		\item $\vert f_P(x,p)\vert \le g(x)$ für \gls{fa} $x\in M$ und $\forall p\in P$
	\end{itemize}

	$\Rightarrow$ $F$ aus \eqref{integral_parameterabhaengig_grundgleichung_eq} ist \gls{diffbar} auf $P$ mit \begin{align}
	\proplbl{integral_parameterabhaengig_differenzierbarkeit_eq}
		F'(p) &= \int_M f_p(x,p) \D x
	\end{align}
\end{proposition}

\begin{underlinedenvironment}[Hinweis]
	Das Integral in \eqref{integral_parameterabhaengig_differenzierbarkeit_eq} ist komponentenweise zu verstehen und liefert für jedes $p\in P$ einen Wert im $\mathbb{R}^m$.
	
	Betrachtet man für $p=(p_1, \dotsc, p_m)\in\mathbb{R}^n$ nur $p_j$ als Parameter und fixiert andere $p_i$, dann liefert \eqref{integral_parameterabhaengig_differenzierbarkeit_eq} die partielle ABleitung $F_{p_j} (p) = \int_m f_{p_j}(x,p) \D x$ für $j=1,\dotsc, m$.
\end{underlinedenvironment}

\begin{proof}
	Königsberger: Analysis 2 (Abschnitt 8.4)
\end{proof}

\subsection{\person{Riemann}-Integral}
Der klassische Integralbegriff hat konzeptionelle Bedeutung (Einführung etwas einfacher, keine messbaren Mengen und Funktionen) \\
$\Rightarrow$ weniger Leistungsfähig (Anwendung nur in speziellen Situationen)

\textbf{ebenfalls:} Approximation von der zu integrierenden Funktion $f$ durch geeignete Treppenfunktionen
	
	Sei $f:Q\subset\mathbb{R}^n\to\mathbb{R}$ mit $Q\in\mathcal{Q}$ eine beschränkte Funktion. Betrachte die Menge der Treppenfunktionen $T_{\mathcal{Q}}(Q)$, der Form \begin{alignat*}{3}
		h &= \sum_{j=1}^l c_j \chi_{Q_j} & &\quad\text{mit}\quad & \bigcup_{j=1}^l Q_j&= Q,
	\end{alignat*}
	$Q_j\in\mathcal{Q}$ paarweise disjunkt, $c_j\in \mathbb{R}$.
	
	Quader $\{ Q_j\}_{j=1,\dotsc,l}$ werden als Zerlegung zugehörig zu $h$ bezeichnet.

\begin{*definition}[Feinheit, \person{Riemann}-Summe, \person{Riemann}-Folge]
	Für Quader $Q' = F_1'\times \dotsc\times F_n'\in\mathcal{Q}$ mit Intervallen $F_j\subset\mathbb{R}$ heißt $\sigma_{Q'} := \max\limits_{j} \vert I_j'\vert$ ($\vert I_j'\vert$ - Intervalllänge) \begriff{Feinheit} von $Q'$ (setzte $\sigma_\emptyset = 0$).
	
	Für $h=\sum_{j=1}^l c_j \chi_{Q_j}$ heißt $\sigma_h := \max \sigma_{Q_j}$ Feinheit zur \begriff{Treppenfunktion} $h$.
	
	Treppenfunktion $h=\sum_{j=1}^l c_j \chi_{Q_j}\in T_{\mathcal{Q}}(Q)$ heißt \begriff{zulässig} (\person{Riemann}-zulässsig) für $f$ falls $\forall j$ $\exists x_j\in Q_j:$ $c_j = f(x_j)$, d.h. auf jedem Quader $Q_j$ stimmt $h$ mit $f$ in (mindestens) einem Punkt $x_j$ überein.
	
	Zu zulässigen $h$ nennen wir $S(h) := \sum_{j=1}^l c_j \vert Q_j\vert = \sum_{j=1}^l f(x_j) \cdot \vert Q_j\vert$ \begriff{\person{Riemann}-Summe} zu $h$.
	
	Folge $\{ h_k\}$ zulässiger Treppenfunktionen zu $f$, deren Feinheit gegen Null geht (d.h. $\sigma_{h_k}\to 0$) heißt \begriff{\person{Riemann}-Folge} zu $f$.
	
	$f$ heißt \person{Riemann}-integrierbar (kurz R-integrierbar) auf $Q$, falls $S\in \mathbb{R}$ existiert mit \begin{align}
	S = \lim\limits_{k\to\infty} S(h_k)\end{align} für \emph{alle} \person{Riemann}-Folgen $\{ h_k \}$ zu $f$.
	
	Grenzwert $\int_Q f(x) \D x := S$ heißt \begriff{\person{Riemann}-Integral} (kurz R-Integral) von $f$ auf $Q$.
\end{*definition}

\begin{proposition}
	\proplbl{integral_riemann_stetig_r_integrierbar}
	Sei $f:Q\subset\mathbb{R}^n\to\mathbb{R}$ stetig und $Q\in\mathcal{Q}$ abgeschlossen \\
	$\Rightarrow$ $f$ ist (\lebesque) integrierbar und \person{Riemann}-Integrierbar auf $Q$ mit R-$\int_Q f \D x = \int_Q f \D x$.
\end{proposition}

\begin{remark}
	Sei $f:Q\subset\mathbb{R}^n\to\mathbb{R}$ beschränkt und es sei $N:=\{ x\in Q \mid f$ nicht stetig in $x \}$.
	
	Dann kann man zeigen: $f$ ist \person{Riemann}-Integrierbar, wenn $n$ Nullmenge ist.
	
	\begin{center}
	\begin{tabular}{r@{\ \ }c@{\ \ }l}
		$f$ ist $R$-integrierbar & $\Leftrightarrow$ $N$ ist Nullmenge.
	\end{tabular}
	\end{center}

	Man sieht leicht: die \person{Dirichlet}-Funktion (\propref{messbarkeit_einfuehrung_dirichlet_funktion}) ist auf $[0,1]$ nicht R-integrierbar, da die Treppenfunktionen $h_0 = 0$ und $h_1 = 1$ auf $[0,1]$ mit belieb feiner Zerlegung $\{Q_j\}$ jeweils stets zulässig sind, sich jedoch in der \person{Riemann}-Summe $0$ bzw. $1$ unterscheiden. (Die \person{Dirichlet}-Funktion ist jedoch L-integrierbar)
\end{remark}

\begin{proof}[\propref{integral_riemann_stetig_r_integrierbar}]
	Als stetige Funktion ist $f$ auf $Q$ messbar und beschränkt und somit L-integrierbar.
	
	Fixiere $\epsilon > 0$ und sei $h=\sum_{j=1}^{l_k} f(x_{k_j}) \chi_{Q_j}$ \person{Riemann}-Folge von Treppenfunktionen zu $f$.
	
	Für $\vert Q \vert = 0$ folgt die Behauptung leicht, da $S(h_k) = 0$ $\forall k\in\mathbb{N}$
	
	Sei nun $\vert Q \vert > 0$. Da $f$ auf kompakter Menge $Q$ gleichmäßig stetig ist, existiert $\delta > 0$ mit $\vert f(x) - f(\tilde{x})\vert < \frac{\epsilon}{\vert Q \vert}$ falls $\vert x - \tilde{x}\vert < \delta$.
	
	Da $\sigma_{h_k}\to 0$ $\exists k_0\in\mathbb{N}:$ $\sigma_{h_k} < \frac{\delta}{\sqrt{n}}$ $\forall k\ge k_0$ \\
	\begin{tabularx}{\linewidth}{r@{\ \ }X}
	$\Rightarrow$ & $\vert x - \tilde{x}\vert < \delta$ $\forall x,\tilde{x}\in Q_{k_j}$ falls $k\ge k_0$ und $\vert f(x) - f(x_{j})\vert < \frac{\epsilon}{\vert Q \vert}$ $\forall x\in Q_{k_j}$ mit $k\ge k_0$\\
	$\Rightarrow$ & $\left\vert \int_Q f\D x - \int_Q h_k \D x \right\vert \le \int_Q \vert f - h_k\vert \D x \le \frac{\epsilon}{\vert Q \vert}\cdot \vert Q \vert = \epsilon$ $\forall k\ge k_0$
	\end{tabularx}
	
	Da $S(h_k) = \int_Q h_k \D x$ und $\epsilon > 0$ beliebig folgt $S(h_k)\to \int_Q f\D x$.
	
	Für jede \person{Riemann}-Folge $\{h_k\}$ zu $f$ ist $f$ R-integrierbar und Behauptung folgt.
\end{proof}
\section{Integration auf $\mathbb{R}$} \setcounter{equation}{0}

\subsection{Integrale konkret ausrechnen}
$\int_I f \D x$ auf Intervalle $I=(\alpha,\beta)\subset\overline{\mathbb{R}}$ (mit $\alpha\le\beta$) (da Randpunkte eines Intervalls $I\subset\mathbb{R}$ nur Nullmenge sind, könnte man statt offenem Intervall auch abgeschlossene bzw. halboffene Intervalle verwenden, ohne den Integralwert zu ändern)

Schreibweise:\begin{align*}
	\int_{\alpha}^{\beta} f \D x &:= \int_I f \D x & &\text{und}&  \int_{\beta}^{\alpha} f \D x &:= -\int_{\alpha}^{\beta} f \D x
\end{align*}
($\alpha = -\infty$ bzw. $\beta = +\infty$ zugelassen)

\begin{underlinedenvironment}[beachte]
	alle Intervalle sind messbare Mengen nach \propref{messbarkeit_satz_grundlegende_messbare_mengen}, \propref{messbarkeit_mengen_satz_acht}.
	
	$\int_{\alpha}^{\beta} f \D x$ heißt auch \begriff{bestimmtes Integral} von $f$ auf $I$.
\end{underlinedenvironment}

Nach \propref{messbarkeit_satz_grundlegende_messbare_mengen} \ref{messbarkeit_satz_sigma_algebra_zwei}:
\begin{proposition}
	\proplbl{integral_r_integrierbar_auf_teilintervalle}
	Sei $f:I\to\mathbb{R}$ integrierbar auf $I$. Dann ist $I$ auch auf allen Teilintervallen $\tilde{I}\subset I$ integrierbar.
\end{proposition}

\begin{theorem}[Hauptsatz der Differential- und Integralrechnung]
	\proplbl{integral_r_hauptsatz}
	Sei $f:I\to\mathbb{R}$ stetig und integrierbar auf Intervall $I\subset\mathbb{R}$ und sei $x_0\in I$. Dann
	\begin{enumerate}[label={\alph*)}]
		\item $\tilde{F}:I\to \mathbb{R}$ mit $\tilde{F}(x) := \int_{x_0}^x f(y) \D y$ $\forall x\in I$ ist Stammfunktion von $f$ auf $I$.
		\item Für jede Stammfunktion $F:I\to \mathbb{R}$ auf $F$ gilt: \begin{align}
			\proplbl{integral_r_hauptsatz_eq}
			F(b) - F(a) = \int_a^b f(x) \D x \quad\forall a,b\in I
		\end{align}
	\end{enumerate}
\end{theorem}

\begin{remark}\vspace*{0pt}
	\begin{itemize}[topsep=\dimexpr-\baselineskip/2\relax]
		\item damit besitzt jede stetige Funktion auf $I$ eine Stammfunktion
		\item \eqref{integral_r_hauptsatz_eq} ist zentrale Formel zur Berechnung von Integralen auf $f$ der reelen Achse; die linke Seite in \eqref{integral_r_hauptsatz_eq} schreibt man auch kurz \begin{align*}
			F(b) - F(a) &= \left.F(x)\right|_a^b = \left. F\right|_a^b = [ F(x) ]_a^b = [ F ]_a^b
		\end{align*}
	\end{itemize}
\end{remark}

\begin{proof}\hspace*{0pt}
	\NoEndMark
	\begin{enumerate}[label={zu \alph*},topsep=\dimexpr-\baselineskip/2\relax,leftmargin=\widthof{\texttt{zu a)\ }}]
		\item Fixiere $x\in I$. Dann gilt für $t\neq 0$ \begin{align*}
			\frac{\tilde{F}(x + t) - \tilde{F}(x)}{t} &= \frac{1}{t} \left( \int_{x_0}^{x + t} f \D y - \int_{x_0}^{x} f \D y \right) = \frac{1}{t} \int_x^{x + t} f \D y =: \phi(t),
		\end{align*}
		wobei nach \propref{integral_r_integrierbar_auf_teilintervalle} alle Integrale existieren. \\
		$\xRightarrow{\cref{integral_grenzwertsatz_mittelwertsatz_integralrechnung}}$ $\forall t\neq 0$ $\exists \xi_t\in [x, x+t]$ (bzw. $[x + t, x]$ für $t < 0$): $\phi(t) = \frac{1}{\vert t \vert} f(\xi) \vert t \vert = f(\xi_t)$ \\
		$\xRightarrow{\text{$f$ stetig}}$ $\tilde{F}'(x) = \lim\limits_{t\to 0} \phi(t) = f(x)$ \\
		$\Rightarrow$ Behauptung
		
		\item Für eine beliebige Stammfunktion $F$ von $f$ gilt: $F(x) = \tilde{F}(x) + C$ für ein $c\in \mathbb{R}$ (vgl \propref{stammfunktion_uneindeutigkeit_stammfunktion}) \\ \begin{tabularx}{\linewidth}{r@{\ \ }X}
		$\Rightarrow$ & $F(b) - F(a) = \tilde{F}(b) - \tilde{F}(a) = \int_{x_0}^{b} f \D x - \int_{x_0}^{a} f \D x = \int_a^b f \D x$ \\
		$\Rightarrow$ & Behauptung \hfill\csname\InTheoType Symbol\endcsname
		\end{tabularx}
	\end{enumerate}
\end{proof}

\begin{example}
	\begin{align*}
		\int_a^b \gamma x \D x &= \left.\frac{\gamma}{2} x^2\right|_a^b = \frac{\gamma}{2} (b^2 - a^2)
	\end{align*}
	\begin{tabularx}{\linewidth}{r@{\ }l@{\ }X}
	für $a = 0$: & Integral = $\frac{b( \gamma b)}{2}$ & (Flächenformel für's Dreieck) \\
	$a = -b <
	 0$: & Integral = 0 & (d.h. vorzeichenbehaftete Fläche)
	\end{tabularx}
\end{example}

\begin{example}
	\proplbl{integration_r_beispiel_5}
	\begin{align*} \int_0^\pi \sin x \D x = -\cos x | _0^\pi = 1 - (-1) = 2\end{align*}
\end{example}\begin{proposition}[Substitution für bestimmte Integrale]
	Sei $f:I\to\mathbb{R}$ stetig, $\phi:I\to\mathbb{R}$ stetig \gls{diffbar} und streng monoton, $a,b\in I$. Dann:
	\begin{align}
		\int_a^b f(x) \D x &= \int_{\phi(a)}^{\phi(b)}f (\phi(y)) \phi'(y) \D y
	\end{align}
	
	\begin{underlinedenvironment}[formal]
		ersetzte $\alpha = \phi(y)$ und $\D x = \frac{\D x}{\D y} \D y = \phi'(y) \D y$. 
		
		Ersetzung des Arguments von $f$ durch $x=\phi(y)$ bezeichnet man als \begriff{Substitution} bzw. Variablentransformation
	\end{underlinedenvironment}
\end{proposition}

\begin{proof}
	\NoEndMark
	Sei $F:I\to\mathbb{R}$ Stammfunktion von $f$ auf $I$ (existiert nach \propref{integral_r_hauptsatz}) \\
	\renewcommand{\arraystretch}{2}
	\begin{tabularx}{\linewidth}{r@{\ \ }X}
	$\xRightarrow{\propref{stammfunktion_substitution}}$ & $F(\phi(\,\cdot\,))$ ist Stammfunktion zu $f(\phi(\,\cdot\,))\phi'(\,\cdot\,)$ \\
	$\xRightarrow{\propref{integral_r_hauptsatz}}$ & $\displaystyle \int_{\phi^{-1}(a)}^{\phi^{-1}(b)} f(\phi(y))\phi'(y) \D y = F(\phi(y))|_{\phi^{-1}(a)}^{\phi^{-1}(b)} = F(b) - F(a) = \int_a^b f(x) \D x$\hfill\csname\InTheoType Symbol\endcsname
	\end{tabularx}
\end{proof}

\begin{example}
	\proplbl{integral_r_beispiel_7}
	\zeroAmsmathAlignVSpaces*
	\begin{align*}
		\int_0^1 \frac{1}{\sqrt{1 - x^2}} \D x \overset{x = \phi(x) = \sin y}{=} \int_0^{\sfrac{\phi}{2}} \frac{1}{\sqrt{1 - \sin^2 y}} \cdot \cos y \D y = \int_0^{\sfrac{\pi}{2}} 1 \D y = \frac{\pi}{2}
	\end{align*}
\end{example}

\begin{proposition}[partielle Integration für bestimmte Integrale]
	Seien $f$, $g:I\to\mathbb{R}$ stetig und $F$ bzw. $G$ die zugehörigen Stammfunktionen, $a$,$b\in I$. Dann \begin{align*}
		\int_a^b f G \D x = FG|^b_a - \int_a^b F g \D x
	\end{align*}
\end{proposition}

 \begin{proof}
 	Es gilt nach \propref{stammfunktion_partielle_integration}
 	\begin{align*}
	 	\int f G\D x &= F(x) G(x) - \int F g \D x
 	\end{align*}
 	und somit folgt aus \eqref{integral_r_hauptsatz_eq} \begin{align*}
	 	\int_a^b f G \D x = \left[ \int f G \D x \right]_a^b = [F \cdot G]_a^b - \left[ \int F g \D x \right] _a^b = F \cdot G |_a^b - \int_a^b F g \D x
 	\end{align*}
 \end{proof}

\begin{example}
	Fläche des Einheitskreises: betrachte $y = \sqrt{1 - x^2}$ und \begin{align*}
		\int_0^1 \sqrt{1-x^2}\D x &= \int_0^1 1 \cdot\sqrt{1 - x^2} \D x = \left[ x \cdot \sqrt{1 - x^2} \right]_0^1 - \int_0^1 x \cdot \frac{-2x}{2 \sqrt{1 - x^2}} \D x\\
		&= \int_0^1 \frac{1}{\sqrt{1 - x^2}} - \int \frac{1-x^2}{\sqrt{1-x^2}} \D x \overset{\text{\cref{integral_r_beispiel_7}}}{=} \frac{\pi}{2} - \int_0^1 \sqrt{1 - x^2} \D x
	\end{align*}
	$\Rightarrow$ Der Viertelkreis hat die Fläche $\int_0^1 \sqrt{1-x^2}\D x = \dfrac{\frac{\pi}{2}}{2} = \frac{\pi}{4}$ und folglich die Kreisfläche von $\pi$.
\end{example}

\begin{example}
	Berechne die Fläche zwischen den Graphen von $f(x) = x^2$, $g(x) = x+2$.
	
	Schnittpunkte: $x_1 = -1$, $x_2 = 2$
	\begin{align*}
		\int_{-1}^2 g - f \D x = \int_{-1}^2 x + 2 - x^2 \D x = \left[ \frac{1}{2}x^2 + 2x - \frac{1}{3} x^3 \right]_{-1}^2 = \frac{9}{2}
	\end{align*}
\end{example}

\begin{example}
	Berechne die Fläche zwischen den Graphen von $f(x) = x (x - 1)(x + 1) = x^3 - x$ und $g(x) = x_0$.
	
	Schnittpunkte: $x_{1,3} = \pm\sqrt{2}$, $x_2 = 0$
	
	Betrachte $g - f$ auf $[0,\sqrt{2}]$ \begin{align*}
		\int_0^{\sqrt{2}}g - f \D x &= \int_0^{\sqrt{2}} 2x - x^3 \D x = \left[ x^2 - \frac{x^4}{4} \right]_0^{\sqrt{2}} = 1,
	\end{align*}
	analog $\int_{-\sqrt{2}}^0 f - g \D x = 1$ \\
	$\Rightarrow$ Gesamtfläche = 2
\end{example}

\begin{proposition}[Differenz von Funktionswerten]
	Sei $f:D\subset\mathbb{R}^n\to\mathbb{R}^m$, $D$ offen, $f$ stetig \gls{diffbar}, $[x,y]\subset D$. Dann \begin{align*}
		f(y) - f(x) &= \int_0^1 f'(x + t(y - x)) \cdot (y - x) \D t = \int_0^1 f(x + t(y - x)) \D t (y - x)
	\end{align*}
	
	\begin{underlinedenvironment}[Hinweis]
		die linke Seite ist Element in $\mathbb{R}^n$ und die Integrale sind jeweils komponentenweise zu verstehen (Mitte = $\mathbb{R}^m$, rechts $\mathbb{R}^{n\times m}$). Man vergleiche den Mittelwertsatz (\propref{mittelwertsatz_mittelwertsatz}) und Schrankensatz (\propref{mittelwertsatz_schrankensatz}).
	\end{underlinedenvironment}
\end{proposition}

\begin{proof}
	\NoEndMark
	Sei $f = (f_1, \dotsc, f_n)$, $\phi_k: [0,1]\to\mathbb{R}$ mit $\phi_k(t) := f_K(x + t(y - x))$ \\\begin{tabularx}{\linewidth}{r@{\ \ }X}
	$\Rightarrow$ & $\phi_t$ ist \gls{diffbar} auf $[0,1]$ mit $\phi_k'(t) = f'(x + t(y - x)) \cdot (y - x)$ \\
	$\xRightarrow{\text{\propref{integral_r_hauptsatz}}}$ & $f_k(y) - f_k(x) = \phi_k(1) - \phi_k(0) = \int_0^1 \phi_k'(t) \D t$ \\
	$\Rightarrow$ & Behauptung \hfill\csname\InTheoType Symbol\endcsname
	\end{tabularx}
\end{proof}

\subsection{Uneigentliche Integrale}
\begin{underlinedenvironment}[Frage]
	$\int_I f \D x$ für $I$ unbeschränkt bzw. $f$ unbeschränkt?
\end{underlinedenvironment}
\begin{underlinedenvironment}[Strategie]
	Verwende den Hauptsatz mittels Grenzprozess
\end{underlinedenvironment}

\begin{proposition}
	\proplbl{integral_r_uneigentlich_satz}
	Sei $f:[a,b]\to\mathbb{R}$ stetig für $a$, $b\in\mathbb{R}$. Dann \begin{center}
			$f$ integrierbar auf $(a,b]$ \ \ $\Leftrightarrow$ \ \ $\displaystyle \lim\limits_{\substack{x\downarrow a \\ x\neq a}} \int_a^b \vert f \vert \D x$ existiert
	\end{center}
\begin{flalign}
\proplbl{integral_r_uneigentlich_satz_eq}
\Rightarrow \;\; \int_a^b f\D x &= \lim\limits_{k\to \infty} \int_{\alpha_k}^a f \D x \text{ für eine Folge $\alpha_k \downarrow a$}&
\end{flalign}
\end{proposition}

\begin{remark}\vspace*{0pt}
	\proplbl{integral_r_uneigentlich_bemerkung}
	\begin{enumerate}[label={\alph*)},topsep=\dimexpr-\baselineskip/2\relax]
		\item Eine analoge Aussage gilt für $f:[a,b)\to\mathbb{R}$
		\item Falls $f$ beschränkt auf $(a,b]$, dann stets integrierbar (vgl. \propref{integral_grenzwertsatz_folgerung_fatou})
		\item Nutzen: Integrale können mittels Hauptsatz berechnet werden
		\item Für uneigentliche Integrale $\int_a^b f \D x$ im Sinne von \person{Riemann}-Integralen muss nur $\lim\limits_{\alpha\downarrow a} \int_{\alpha}^b f \D x$ existieren (vgl. \propref{integral_r_uneigentlich_beispiel_19} unten)
	\end{enumerate}
\end{remark}

\begin{proof}
	Sei $\alpha_k\downarrow a$, $a < \alpha_k$ $\forall k$ und \begin{align*}
		f_k(x) &:= \begin{cases}
			f(x) & \text{auf $(\alpha_k, b]$} \\
			0 & \text{auf $(a, \alpha_k)$}
		\end{cases}
	\end{align*}
	Offenbar ist $\vert f_k\vert \le \vert f\vert$, $f_k\to f$, $\vert f_k\vert \to \vert f \vert$ \gls{fü} auf $(a,b)$.
	\begin{itemize}
		\item["`$\Rightarrow$"'] $f$ integrierbar auf $(a,b)$. Mit \propref{integral_grenzwertsatz_majorisierte_konvergenz} (Majorisierte Konvergenz) folgt \begin{align*}
		\lim\limits_{k\to\infty} \int_{\alpha_k}^b \vert f \vert \D x &= \lim\limits_{k\to\infty} \int_a^b \vert f_k\vert \D x = \int_a^b \vert f \vert \D x
		\end{align*}
		$\Rightarrow$ Behauptung $\xRightarrow[\text{Beträge}]{\text{ohne}}$ \eqref{integral_r_uneigentlich_satz_eq}
		
		\item["`$\Leftarrow$"'] Folge $\{ \vert f_k\vert \}$ monoton wachsend, \begin{align*}
			\lim\limits_{k\to\infty} \int_a^b \vert f_k\vert \D x &= \lim\limits_{k\to\infty} \int_{\alpha_k}^b \vert f \vert \D x\quad\text{existiert}
		\end{align*}
		$\xRightarrow[\text{Konvergenz}]{\text{majorisierte}}$ $f$ integrierbar
	\end{itemize}
\end{proof}

\begin{example}
	$\int_0^1 \frac{1}{x^\gamma} \D x$ existiert für $0 < \gamma < 1$ und \emph{nicht} für $\gamma \ge 1$
	
	Für $\gamma \neq 1$: $\displaystyle \int_{\alpha_k}^1 \frac{1}{x^\gamma} \D x = \left.\frac{1}{1 - \gamma}x^{1 - \gamma}\right|_{\alpha_k}^1 = \frac{1}{1-\gamma} (1 - \alpha_k)^{1 - \gamma} \xrightarrow{\alpha_k \downarrow 0} \frac{1}{1 - \gamma}$
	
	(keine Konvergenz für $1 - \gamma \le 0$, $\gamma=1$: analog mit Stammfunktion $\ln x$)
\end{example}

\begin{proposition}
	sei $f:[a,+\infty]\to\mathbb{R}$ stetig, dann \begin{center}
		$f$ integrierbar auf $[a,+\infty]$ \ \ $\Leftrightarrow$ \ $\displaystyle \lim\limits_{\beta \to \infty}  \int_a^\beta \vert f \vert \D x$ existiert
	\end{center}
	$\Rightarrow$ $\displaystyle \int_0^\infty f \D x = \lim\limits_{k\to\infty} \int_0^{\beta_k} f \D x$ für eine Folge $\beta_k\to\infty$
\end{proposition}

\begin{remark}
	Analoge Bemerkungen wie in \propref{integral_r_uneigentlich_bemerkung}
\end{remark}
\begin{proof}
	Analog zu \propref{integral_r_uneigentlich_satz}
\end{proof}

\begin{example}
	\proplbl{integral_r_unbestimmt_beispiel_18}
	{\zeroAmsmathAlignVSpaces*
	\begin{flalign*}
	\int_1^\infty &\frac{1}{x^\gamma} \D x \text{existiert für $\gamma > 1$ und nicht für $0 \le \gamma \le 1$}&
	\end{flalign*}}
	
	Für $\gamma \neq 1$:\begin{align*}
		\int_1^{\beta_k} \frac{1}{x^\gamma} \D x &= \left.\frac{1}{\gamma} x^{1 - \gamma}\right|_1^{\beta_k} = \frac{1}{\gamma - 1} (1 - \beta_k^{1 - \gamma}) \xrightarrow{\beta_k\to\infty} \frac{1}{\gamma - 1},
	\end{align*}
	falls $1 - \gamma < 0$ (keine Konvergenz für $1 - \gamma \ge 0$, $\gamma = 1$ analog mit Stammfunktion $\ln x$)
\end{example}

\begin{example}
	\proplbl{integral_r_uneigentlich_beispiel_19}
	{\zeroAmsmathAlignVSpaces\begin{flalign*}
	\int_0^\infty &\frac{\sin x}{x} \D x &
	\end{flalign*}}
	
	Offenbar ist $\int_{(k - 1)\pi}^{k\pi} \left\vert \frac{\sin x}{x} \right\vert \D x \ge \frac{1}{k\pi} \int_{(k - 1)\pi}^{k\pi} \vert \sin x \vert \D x = \frac{2}{k\pi}$ $\forall k\ge 1$ (vgl. \propref{integration_r_beispiel_5}) \\
	$\Rightarrow$ $\int_0^{k\pi} \left\vert \frac{\sin x}{x}\right\vert \D x \ge \frac{2}{\pi} \sum_{j=1}^k \frac{1}{j} \xrightarrow{k\to\infty} \infty$ \\
	$\Rightarrow$ $\frac{\sin x}{x}$ \emph{nicht} integrierbar auf $(0,\infty)$
	
	\emph{aber} $\int_1^\beta \frac{1}{x} \sin x \D x = \frac{\cos 1}{1} - \frac{\cos \beta}{\beta} - \int_1^\beta \frac{\cos x}{x^2} \D x$
	
	Wegen $\left\vert\frac{\cos x}{x^2}\right\vert\le\frac{1}{x^2}$ $\forall x\neq 0$, $\frac{1}{x^2}$ ist integrierbar nach \propref{integral_r_unbestimmt_beispiel_18} \\
	$\Rightarrow$ $\lim\limits_{\beta\to\infty} \int_1^\beta \frac{\cos x}{x^2} \D x$ existiert nach \propref{integral_funktion_majorantenkriterium} \\
	$\Rightarrow$ $\lim\limits_{\beta\to\infty} \int_1^\beta \frac{\sin x}{x}\D x$ existiert $\Rightarrow$ $\int_0^\infty \frac{\sin x}{x} \left( = \frac{\pi}{2}\right)$ existiert als uneigentliches Integral im Sinne des \person{Riemann}-Integral (vgl \propref{integral_r_uneigentlich_bemerkung}), aber nicht als \lebesque-Integral.
\end{example}
\section{Satz von \person{Fubini} und Mehrfachintegrale} \setcounter{equation}{0}

\begin{underlinedenvironment}[Ziel]
	Reduktion der Berechnung von Integralen auf $\mathbb{R}^n$ $\int_{\mathbb{R}^n} f \D x$ auf Integrale über $\mathbb{R}$.
\end{underlinedenvironment}

Betrachte Integrale auf $X\times Y$ mit $X=\mathbb{R}^p$, $Y=\mathbb{R}^q$, $(x,y)\in X\times Y$. $\vert M \vert_X$ Maß auf $X$, $\mathcal{Q}_X$ Quader in $X$ usw.

\begin{theorem}[\person{Fubini}]
	\proplbl{fubini_fubini}
	Sei $f:X\times Y\to\mathbb{R}$ integrierbar auf $X\times Y$. Dann \begin{enumerate}[label={\alph*)}]
		\item Für Nullmenge $N\subset Y$ ist $x\to f(x,y)$ integrierbar auf $X$ $\forall y\in Y\setminus N$
		\item Jedes $F:Y\to\mathbb{R}$ mit $F(y) := \int_X f(x,y) \D x$ $\forall y\in Y\setminus N$ ist integrierbar auf $Y$ und \begin{align}
			\proplbl{fubini_fubini_eq}
			\int_{X\times Y} f(x,y) \D(x,y) &= \int_Y F(y) \D y = \int_Y \left( \int_X f(x,y) \D x \right) \D y
		\end{align}
	\end{enumerate}
\end{theorem}

\begin{*definition}
	Rechte Seite in \eqref{fubini_fubini_eq} heißt \begriff{iteriertes Integral} bzw. \begriff{Mehrfachintegral}.
\end{*definition}

\begin{remark}
	Analoge Aussage gilt bei Vertauschungen von $X$ und $Y$ mit \begin{align}
		\proplbl{fubini_fubini_eq_2}
		\int_{X\times Y} f(x,y) \D (x,y) = \int_X \int_Y f(x,y) \D y \D x
	\end{align}
	
	\propref{fubini_fubini} mit $f=\chi_{N}$ für Nullmenge $N\subset X\times Y$ liefert Beschreibung von Nullmengen in $X\times Y$.
\end{remark}

\begin{conclusion}
	\proplbl{fubini_folgerung_nullmenge}
	Sei $N\subset X\times Y$ Nullmenge und $N_Y := \{ x\in X \mid (x,y) \in N \}$ \\
	$\Rightarrow$ $\exists$ Nullmenge $\tilde{N}\subset Y$ mit $\vert N_Y\vert_X = 0$ $\forall y\in Y\setminus \tilde{N}$
	
	\begin{underlinedenvironment}[Hinweis]
		$\tilde{N}\neq \emptyset$ tritt z.B. auch auf für $N=\mathbb{R}\times \mathbb{Q} \subset \mathbb{R}\times\mathbb{R}$ ($\tilde{N} = \mathbb{Q}$)
	\end{underlinedenvironment}
\end{conclusion}

\begin{proof}[\propref{fubini_fubini}, \propref{fubini_folgerung_nullmenge}]\hspace*{0pt}
	\begin{enumerate}[label={\alph*)},topsep=\dimexpr-\baselineskip/2\relax]
		\item \proplbl{fubini_fubini_beweis_teil_a}
		Zeige: \propref{fubini_fubini} gilt für $f=\chi_M$ mit $M\subset X\times Y$ messbar, $\vert M \vert _{X\times Y} < \infty$
		
		\begin{itemize}
		\item $\exists Q_{k_j}\in \mathcal{Q}_{X\times Y}$, paarweise disjunkt für festes $k$ mit $M\subset\bigcup_{j\in\mathbb{N}} Q_ {k_j} =: R_k$ \begin{align}
			\proplbl{fubini_fubini_beweis_3}
			\vert M \vert &\le \sum_{j=1}^\infty \vert Q_{k_j}\vert \le \vert M \vert + \frac{1}{k}, R_{k+1}\subset R_k
		\end{align}
		
		
		\item Wähle $Q_{k_j}' \in \mathcal{Q}_X$, $Q_{k_j}''\in \mathcal{Q}_Y$ mit $Q_{k_j} = Q_{k_j}'\times Q_{k_j}''$ $\forall k,j\in\mathbb{N}$
		
		\item Mit $M_Y := \{ x\in X \mid (x,y)\in M \}$ gilt: \begin{align}
			\proplbl{fubini_fubini_beweis_4}
			\vert M_Y\vert _X &\le \sum_{j=1}^\infty \vert Q_{k_j}' \vert_X \cdot \chi_{Q_{k_j}''}(y) =: \psi_k(y) \in [0,\infty]\quad\forall y\in Y
		\end{align}
		
		\item Für festes $k$ ist $y\to \psi_{k_l}(y) := \sum_{j=1}^l \vert Q_{k_j}'\vert_X \cdot  \chi_{Q_{k_j}}(y)$ monoton wachense Folge und Treppenfuntion in $T^1(Y)$ mit $\psi_k(y) = \lim\limits_{l\to\infty} \psi_{k_l} (y)$ \\
		\begin{tabularx}{\linewidth}{r@{\ \ }X}
		$\Rightarrow$ & $\displaystyle\int_Y \psi_{k_l}(y) \D y = \sum_{j=1}^l \vert Q_{k_j}'\vert_X \cdot \vert Q_{k_j}''\vert_Y = \sum_{j=1}^l \vert Q_{k_j}\vert_{X\times Y} \overset{\eqref{fubini_fubini_beweis_3}}{\le} \vert M \vert + \frac{1}{k}$
		\end{tabularx}
		
		\item Nach \propref{integral_funktion_lemma_majorante} ist $\{ \psi_{k_l}\}_l$ $L^1$-CF zu $\psi_k$ und $\psi_k$ ist integrierbar auf $Y$ mit \begin{align}
			\proplbl{fubini_fubini_beweis_5}
			\vert M \vert \overset{\eqref{fubini_fubini_beweis_3}}{\le}\int_Y \psi_k \D y &= \sum_{j=1}^\infty \vert Q_{k_j}\vert _{X\times Y} \overset{\eqref{fubini_fubini_beweis_3}}{\le} \vert M \vert + \frac{1}{k}
		\end{align}
		
		\item Da $\{ \psi_k \}$ monoton fallend (wegen $R_{k+1}\subset R_k$), existiert $\psi(y) = \lim\limits_{k\to\infty} \psi_k(y) \ge 0$ $\forall y\in Y$.
		
		\item Grenzwert \eqref{fubini_fubini_beweis_5} mittels majorisierter Konvergenz liefert \begin{align}
			\proplbl{fubini_fubini_beweis_6}
			\vert M \vert = \int_Y \psi \D y
		\end{align}
		
		\item Falls $\vert M \vert = 0$, folgt $\psi(y) = 0$ \gls{fü} auf $Y$ \\ \begin{tabularx}{\linewidth}{r@{\ \ }X}
		$\Rightarrow$ & \propref{fubini_folgerung_nullmenge} bewiesen.
		\end{tabularx}
		\end{itemize}
		\vspace*{\dimexpr-\baselineskip/2}
		\rule{0.5\linewidth}{0.1pt}
		
		\begin{itemize}
		\item $\{ \chi_{R_k}\}$ monoton fallend mit $\psi_{R_k}\to\chi_M$ \gls{fü} auf $X\times Y$ und $\chi_{R_k}$ integrierbar auf $X\times Y$ \\
		\begin{tabularx}{\linewidth}{r@{\ \ }X}
		$\Rightarrow$ & $\{ \chi_{R_k}\}$ ist $L^1$-CF zu $\chi_M$ und \[\int_{X\times Y} \psi_{R_k} \D (x,y) \to \int_{X\times Y} \chi_M \D (x,y).\]
		\end{tabularx}
		
		\item Nach \propref{fubini_folgerung_nullmenge} existiert Nullmenge $\tilde{N}\subset Y$ mit $\chi_{R_k}(\,\cdot\, , y)\to \chi_M(\,\cdot \, , y)$ \gls{fü} auf $X$ $\forall y\in Y\setminus\tilde{N}$ \\
		\begin{tabularx}{\linewidth}{r@{\ \ }X}
		$\xRightarrow{\eqref{fubini_fubini_beweis_3},\eqref{fubini_fubini_beweis_4}}$ & $\chi_{R_k} (\,\cdot\, , y)$ integrierbar auf $X$ $\forall k\in \mathbb{N}$, $y\in Y\setminus\tilde{N}$ \\
		$\xRightarrow[\text{Konvergenz}]{\text{majorisierte}}$ & $\chi_M(\,\cdot\, ,y)$ integrierbar auf $X$ $\forall y\in Y\setminus\tilde{N}$ mit
		\[\psi(y) = \int_X \chi_{R_k}(x,y)\D x \to \int_X \chi_M (x,y) \D y\] für \gls{fa} $y\in Y$ \\
		$\xRightarrow{\eqref{fubini_fubini_beweis_6}}$&  $\displaystyle \int_{X\times Y} \chi_M (x,y) \D (x,y) = \vert M \vert = \int_Y \left( \int_X \chi_m (x,y) \D x\right) \D y$
		\end{tabularx}
		
		\item D.h. Behauptung für $f=\chi_M$ \\ \begin{tabularx}{\linewidth}{r@{\ \ }X}
		$\xRightarrow[\text{des Integrals}]{\text{Linearität}}$ & Behauptung richtig für alle Treppenfunktionen
		\end{tabularx}
		\end{itemize}
		\item Sei $f\ge 0$ integrierbar auf $X\times Y$
		
		Wähle zu $f$ monotone Folge von Treppenfunktionen $\{ h_k\}$ gemäß \propref{messbarkeit_funktion_existenz_monotone_treppenfunktionen} \\ \begin{tabularx}{\linewidth}{r@{\ \ }X}
		$\Rightarrow$ & $\displaystyle \int_{X\times Y} h_k(x,y) \D (x,y) \overset{\text{a)}}{=} \int_Y \left( \int_X h_k \D x\right) \D y$
		\end{tabularx}
		
		Analog zu \ref{fubini_fubini_beweis_teil_a} folgt: $h_k(\,\cdot\, , y)\to f(\,\cdot\,,y)$ \gls{fü} auf $X$ für \gls{fa} $y\in Y$ \\ \begin{tabularx}{\linewidth}{r@{\ \ }X}
		$\xRightarrow[\text{Konvergenz}]{\text{Majorisierte}}$ & Behauptung für $f$.
		\end{tabularx}
		
		Allgemein: Zerlege $f = -f^- + f^+$ und argumentiere für $f^\pm$ separat.
	\end{enumerate}
\end{proof}

\begin{proposition}[Satz von \person{Tonelli}]
	\proplbl{fubini_tonelli}
	Sei $f:X\times Y\to\mathbb{R}$ messbar. Dann \begin{align}
		\proplbl{fubini_tonelli_eq}
		\text{$f$ integrierbar} \;\;\Leftrightarrow\;\; \int_Y \left( \int_X \vert f(x,y)\vert \D x\right) \D y \quad\text{oder}\quad\int_X \left(\int_Y \vert f(x,y)\vert \D y \right) \D x
	\end{align}
	existiert.
\end{proposition}

\begin{remark}\vspace*{0pt}
	\begin{enumerate}[label={\alph*)},topsep=\dimexpr -\baselineskip/2\relax]
		\item Falls eines der iterierten Integrale \eqref{fubini_tonelli_eq} mit $\vert f\vert$ existieren, dann gelte \eqref{fubini_fubini_eq}, \eqref{fubini_fubini_eq_2}
		\item Existiert z.B. $\int_Y \left( \int_X \vert f \vert \D x \right) \D y$ heißt dies: $\exists$ Nullmenge $\tilde{N}\subset Y$ mit \[F(y) := \int_X \vert f(x,y)\vert \D x\quad\forall y\in Y\setminus \tilde{N}\] und mit $F(y) := 0$ $\forall y\in \tilde{N}$ ist $F$ integrierbar auf $Y$
	\end{enumerate}
\end{remark}

\begin{proof}\hspace*{0pt}
	\NoEndMark
	\begin{itemize}
		\item["`$\Rightarrow$"'] Mit $f$ auch $\vert f \vert$ integrierbar und die Behauptung folgt aus \propref{fubini_fubini}
		
		\item["`$\Leftarrow$"'] Sei $W_k := (-k,k)^{p+q}\subset X\times Y$ Würfel, $f_k := \in \{ \vert f \vert, k\cdot \chi_{W_k} \}$ \\
		\begin{tabularx}{\linewidth}{r@{\ \ }X}
		$\Rightarrow$ & $f$ ist integrierbar auf $X\times Y$
		\end{tabularx}
		
		Offenbar sind die $\{ f_k \}$ wachsend, $f_k\to \vert f \vert$ \gls{fü} auf $X\times Y$. Falls oberes Integral in \eqref{fubini_tonelli_eq} existiert, gilt \begin{align*}
			\int_{X\times Y} f(x,y) \D(x,y) \overset{\text{Fubini}}{=} \int_Y \left( \int_X f_k \D x\right) \D y \le \int_Y \left( \int_X \vert f \vert \D x \right)\D y < \infty
		\end{align*}
		\begin{tabularx}{\linewidth}{r@{\ \ }X}
		$\Rightarrow$ & $\{\int_{X\times Y} f_k \D (x,y)\}$ beschränkte Folge \\
		$\xRightarrow[\text{Konvergenz}]{\text{Majorisierte}}$ & $\vert f \vert$ integrierbar $\xRightarrow{\text{\cref{integral_funktion_eigenschaften}}}$ $f$ integrierbar $\Rightarrow$ Behauptung \hfill\csname\InTheoType Symbol\endcsname
	\end{tabularx}
	\end{itemize}
\end{proof}

\begin{conclusion}
	\proplbl{fubini_tonelli_folgerung}
	Sei $f:\mathbb{R}^n\to\mathbb{R}$ integrierbar auf $\mathbb{R}^n$, $x = (x_1, \dotsc, x_n)\in\mathbb{R}^n$
	\begin{flalign}
		\proplbl{fubini_tonelli_folgerung_eq}
		\Rightarrow\;\;\int_{\mathbb{R}^n} f(x) \D x = \int_\mathbb{R} \dotsc \left( \int_\mathbb{R} f(x_1, \dotsc, x_n) \D x_1 \right) \dotsc \D x_n
	\end{flalign}
\end{conclusion}
\begin{proof}
	Mehrfachanwendung von \propref{fubini_fubini}
\end{proof}

\begin{remark}\vspace*{0pt}
	\begin{enumerate}[label={\arabic*)},topsep=\dimexpr -\baselineskip / 2\relax]
		\item Die Reihenfolge der Integration in \eqref{fubini_tonelli_folgerung_eq} ist beliebig
		\item Integrale reduzieren die Integration auf reelle Integrale über $\mathbb{R}$
		\item Für $\int_M f \D x$ ist $(\chi_M f)$ gemäß \eqref{fubini_tonelli_folgerung_eq} zu integrieren, wo ggf. $\int_{\mathbb{R}}\dotsc$ durch $\int_a^b\dotsc$ mit geeigneten Grenzen ersetzt wird.
	\end{enumerate}
\end{remark}

\begin{example}
	Sei $f:M\subset\mathbb{R}^2\to\mathbb{R}$ stetig, $M=[a,b]\times[c,d]$ \\ \begin{tabularx}{\linewidth}{r@{\ \ }X}
		$\Rightarrow$ & $f$ messbar, beschränkt auf $M$ \\
		$\Rightarrow$ & $f$ integrierbar auf $M$ \\
		$\Rightarrow$ & $\chi_M f$ ist integrierbar auf $\mathbb{R}^2$
	\end{tabularx}
	\zeroAmsmathAlignVSpaces*
	\begin{flalign*}
		\;\; & \begin{aligned} \Rightarrow\;\; \int_M f \D x &= \int_{\mathbb{R}^2} \chi_M f \D x = \int_{\mathbb{R}}\int_\mathbb{R} \chi_M (x_1, x_2) f(x_1, x_2) \D x_1 \D x_2 \\
		&= \int_\mathbb{R} \int_a^b \chi_{[c,d]} (x_2) f(x_1, x_2) \D x_1 \D x_2 = \int_c^d \int_a^b f(x_1, x_2) \D x_1 \D x_2\end{aligned} &
	\end{flalign*}
	
	Z.B. $f(x_1, x_2) = x_1\cdot \sin x_2$, $M=[0,1]\times [0,\pi]$
	\zeroAmsmathAlignVSpaces*
	\begin{flalign*}
		\;\;& \begin{aligned}\Rightarrow\;\; \int_M f \D x &= \int_0^\pi \int_0^1 x_1 \sin x_2 \D x_1 \D x_2 = \int_0^\pi \left[ \frac{1}{2} x_1^2 \sin x_2 \right]_0^1 \D x_2 \\
		&= \int_0^\pi \frac{1}{2}\sin x_2 \D x_2 = \left[ - \frac{1}{2} \cos x_2 \right]_0^\pi = 1
		\end{aligned} &
	\end{flalign*}
\end{example}

\begin{example}
	Sei $f:M\subset\mathbb{R}^2\to\mathbb{R}$ stetig, $M=\{ (x,y) \mid x^2 + y^2 = 1\}$ \\
	\begin{tabularx}{\linewidth}{r@{\ \ }X}
		$\Rightarrow$ & $\chi_M f$ integrierbar auf $\mathbb{R}^2$ \\
		$\Rightarrow$ & $\displaystyle \int_M f \D (x,y) = \int_{\mathbb{R}}\int_{\mathbb{R}} \chi_M f \D y \D x = \int_{-1}^1 \int_{\sqrt{1 - x^2}}^{\sqrt{1 - x^2}} f(x,y) \D y \D x$
	\end{tabularx}

	Z.B. $f(x,y) = \vert y \vert$
	\zeroAmsmathAlignVSpaces*
	\begin{flalign*}
		\;\;& \begin{aligned} \Rightarrow\;\; \int_M \vert y \vert \D (x,y) &= 2 \int_{-1}^1 \int_0^{\sqrt{1 - x^2}} y \D y \D x = 2 \int_{-1}^1 \left[ \frac{1}{2} y^2 \right]_0^{\sqrt{1 - x^2}} \D x \\
		&= 2 \int_{-1}^1 \frac{1}{2} (1 - x^2) \D x = \left[ x - \frac{1}{3}x^3 \right]_{-1}^1 = \frac{4}{3}\end{aligned} &
	\end{flalign*}
\end{example}

\begin{example}
	Sei $f:M\subset\mathbb{R}^3\to\mathbb{R}$ stetig, $M$ Tetraeder mit Ecken $0$, $e_1$, $e_2$, $e_3$
	\begin{align*}
		\int_M f \D (x,y,z) = \int_0^1 \int_0^{1-x} \int_0^{1-x-y} f(x,y,z) \D z \D y \D x
	\end{align*}
	
	Z.B: $f(x,y,z) = 1$: \begin{align*}
		\int_M 1 \D (x,y,z) &= \int_0^1 \int_0^{1-x} \int_0^{1-x-y} f(x,y,z) \D z \D y \D x = \int_0^1 \int_0^{1-x} [z]_0^{1-x-y} \D y \D x \\
		&= \int_0^1 \int_0^{1-x} 1 - x - y \D y \D z = \int_0^1 [y - xy - \frac{y^2}{2}]_{y=0}^{1-x} \D x = \int_0^1 \frac{1}{2} - x + \frac{x^2}{2} \D x\\
		& = \frac{1}{6},
	\end{align*}
	das Volumen eines Tetraeders.
\end{example}

\subsection{Integration durch Koordinatentransformation}
\begin{*definition}
Sei $f:U\subset K^n\to V\subset K^m$ bijektiv, wobei $U$, $V$ offen.

$f$ heißt \begriff{Diffeomorphismus}, falls $f$ und $f^{-1}$ stetig \gls{diffbar} auf $U$ bzw. $V$ sind.

$U$ und $V$ heißen dann \begriff{diffeomorph}.
\end{*definition}

\begin{theorem}[Transformationssatz]
	\proplbl{fubini_trafo_trafosatz}
	Seien $U$, $V\subset\mathbb{R}^n$ offen, $\phi: U\to V$ Diffeomorphismus. Dann 
	
	\begin{tabularx}{\linewidth}{X@{\ \ }c@{\ \ }X}
		\hfill$f:V\to\mathbb{R}$ integrierbar  & $\Leftrightarrow$ & $f(\phi(\,\cdot\,))\vert \det \phi'(y) \vert: U\to\mathbb{R}$ integrierbar
	\end{tabularx}
	und es gilt
	\begin{align}
		\proplbl{fubini_trafo_trafosatz_eq}
		\int_U f(\phi(y))\cdot\vert\phi'(y)\vert \D y = \int_V f(x) \D x
	\end{align}
\end{theorem}

\begin{proof}
	Vgl. Literatur (z.B. Königsberger Analysis 2, Kapitel 9)
\end{proof}

Sei $U=Q\in\mathcal{Q}$ Würfel, $V:= \phi(Q)$, $\tilde{y}\in \mathcal{Q}$, $x:= \phi(\tilde{y})$ \\
$\xRightarrow{\eqref{fubini_trafo_trafosatz_eq}}$ $\vert V \vert = \int_V 1 \D y = \int_Q \vert \det \phi'(y) \vert \D y \overset{\text{$Q$ klein}}{\approx} \vert \det \phi'(\tilde{y})\vert \cdot \vert Q \vert$, d.h. $\vert \det \phi'(y) \vert$ beschreibt (infinitesimale) relative Veränderung des Maßes unter Transformation $\phi$.

\begin{example}
	Sei $V=B_R(0) \subset\mathbb{R}^3$ Kugel mit Radius $R > 0$.
	
	Zeige: $\displaystyle \vert B_R(0) \vert = \int_V 1\D (x,y,z) = \frac{4}{3}\pi R^3$
	
	Benutze Kugelkoordinaten (Polarkoordinaten in $\mathbb{R}^2$) mit \begin{align*}
		\begin{pmatrix}
			x \\ y \\ z
		\end{pmatrix} &= \phi(r, \alpha, \beta) := \begin{pmatrix}
			r \cos \alpha \cos \beta \\ r\sin \alpha \cos \beta \\ r \sin \beta
		\end{pmatrix}
	\end{align*}
	Für $(r,\alpha,\beta)\in U: (0,R)\times(-\pi,\pi)\times\left(-\frac{\pi}{2},\frac{\pi}{2}\right)$.
	
	Mit $H:= \{ (x,0,z)\in\mathbb{R}\mid x\le 0 \}$ und $\tilde{V} := V\setminus H$ gilt: $\vert H\vert_{\mathbb{R}^3} = 0$
	
	$\phi: U\to\tilde{V}$ \gls{diffbar}, injektiv, und \begin{align*}
		\phi'(r,\alpha,\beta) &= \begin{pmatrix}
			\cos\alpha \cos \beta & -r\sin \alpha\cos\beta & -r\cos\alpha\sin\beta \\
			\sin\alpha\cos\beta & r \cos\alpha\cos\beta & -r\sin\alpha\sin\beta \\
			\sin\beta & 0 & r\cos\beta
		\end{pmatrix}
	\end{align*}
	$\Rightarrow$ Definiere $\phi'(r,\alpha,\beta) = r^2\cos\beta\neq 0$ auf $U$ \\
	% @TODO: Label setzen
	$\xRightarrow{Satz 27.8}$ $\phi:U\to\tilde{V}$ ist Diffeomorphismus
	\begin{flalign*}
	\;\;&\begin{aligned}\Rightarrow\;\; \vert B_R(0)\vert &= \int_V 1 \D (x,y,z) = \int_{\tilde{V}} 1 \D (x,y,z) + \int_H 1 \D (x,y,z) \\ & \overset{\eqref{fubini_trafo_trafosatz_eq}}{=} \int_U \vert \det \phi'(r,\alpha,\beta)\vert \D r \D \alpha \D \beta + \vert H \vert 
	\overset{\text{Fubini}}{=} \int_0^R \int_{-\pi}^\pi \int_{-\frac{\pi}{2}}^{\frac{\pi}{2}} r^2 \cos\beta \D \beta \D \alpha \D r \\
	&= \int_0^R \int_{-\pi}^\pi [r^2\sin \beta]_{-\frac{\pi}{2}}^{\frac{\pi}{2}} \D \alpha  \D r = \int_0^R \int_{-\pi}^\pi 2 r^2 \D \alpha \D r
	= \int_0^R 4 \pi r^2 \D r \\
	& = \left.\frac{4}{3}\pi r^3\right|_0^R  = \frac{4}{3}\pi R^3
	\end{aligned}\end{flalign*}
\end{example}

\begin{example}[Rotationskörper im $\mathbb{R}^3$]
	Sei $g:[a,b]\to[0,\infty]$ stetiger, rotierender Graphen von $g$ um die $z$-Achse. \\
	$\rightarrow$ Bestimme das Volumen des (offenen) Rotationskörpers $V\subset\mathbb{R}^3$.
	
	Benutze Zylinderkoordinaten:\begin{align*}
		\begin{pmatrix}
			x \\ y \\ z
		\end{pmatrix} = \phi(r,\alpha, z) := \begin{pmatrix}
			r\cos \alpha \\ r \sin\alpha \\ z
		\end{pmatrix}
	\end{align*}
	auf \[U= \{ (r,\alpha,z) \in\mathbb{R}^3 \mid r \in (0, g(z)), \alpha\in (-\pi,\pi),z\in(a,b) \},\] mit $H:= \{ (x,0,z) \in\mathbb{R}^3 \mid x \le 0 \}$, $\tilde{V} := V \setminus H $ gilt $\vert H \vert = 0$ und $\phi:U\to\tilde{V}$ \gls{diffbar}, injektiv, sowie \begin{align*}
		\phi'(r,\alpha,z) = \begin{pmatrix}
			\cos \alpha & - r\sin \alpha & 0 \\ \sin \alpha & r \cos \alpha & 0 \\ 0 & 0 & 1
		\end{pmatrix} = r > 0\text{ auf $U$}
	\end{align*}
	%@TODO: label setzten
	$\xRightarrow{\text{Satz 27.8}}$ $\phi:U\to\tilde{V}$ ist Diffeomorphismus
	
	$V$ messbar (da offen) $\Rightarrow$ $\tilde{V}$ messbar, und offenbar $f=1$ integrierbar auf $\tilde{V}$ \\
	\renewcommand{\arraystretch}{3}
	\begin{tabularx}{\linewidth}{r@{\ \ }r@{\ }c@{\ }l@{\ }c@{\ }X}
		$\Rightarrow$ & $\vert V \vert = \vert \tilde{V} \vert$ &=& $\displaystyle\int_{\tilde{V}} 1 \D (x,y,z)$ &$ \overset{\eqref{fubini_trafo_trafosatz_eq}}{=}$ &  $\displaystyle\int_U \vert \det \phi'(r,\alpha,z)\vert \D (x,y,z)$ \\
		& & $\overset{\text{Fubini}}{=}$ &  $\displaystyle \int_a^b \int_{-\pi}^\pi \int_0^{g(z)} r \D r \D \alpha \D z$ &=& $\displaystyle\int_a^b \int_{-\pi}^\pi \left[ \frac{r^2}{2} \right]_0^{g(z)} \D \alpha \D z$ \\
		& & =  & $\displaystyle\int_a^b \int_{-\pi}^\pi \frac{g(z)^2}{2} \D \alpha \D z$ &=& $\displaystyle\pi \int_a^b g(z)^2\D z$
	\end{tabularx}
	
	
	Z.B. $g(z) = R$ auf $[a,b]$: $\vert V \vert = \pi \int_a^b R^2 \D z = \pi R^2(b - a)$ (Volumen des Kreiszylinders)
\end{example}

\chapter{Differentiation II}
\addtocounter{section}{24}
\section{Höhere Ableitungen und \person{Taylor}-scher Satz} \proplbl{section_taylor} \setcounter{equation}{0}

\begin{boldenvironment}[Vorbetrachtung] Sei $X$ endlich dimensionaler, normierter Raum über $K$ (d.. Vektorraum über $K$ mit Norm $\Vert \,\cdot \Vert$, $\dim X =l\in\mathbb{N}$).
	
	Offebar sind $X$ und $K^l$ isomorph als Vektorraum, schreibe $X\cong K^l$, z.B. $X = L(K^n,K^m)\cong K^{m\cdot n}$.
	
	Für $g:D\subset K^n\to X$, $D$ offen, kann man die bisherigen Resultate bezüglich der Ableitung übertragen. $g'(x)\in L(K^n, X)$ heißt Ableitung von $g$ im Punkt $x\in D$, falls \begin{align*}
		g(x + y) = g(x) + g'x() y + o(\vert y\vert), \;y\to 0
	\end{align*}
\end{boldenvironment}
	
\begin{*definition}[zweite Ableitung]
Betrachte nun $f:D\subset K^n\to K^m$, $D$ offen, $f$ \gls{diffbar} auf $D$. Falls $g:= f':D\to L(K^n, K^m) =: y_1$ \gls{diffbar} in $x\in D$ ist, heißt \begin{align}
	f''(x) := g'(x)\in L(K^n, Y_1) = L\big( K^n, \underbrace{L(K^n, K^m)}_{\cong K^{m\times n}}\big)
\end{align}
\begriff{zweite Ableitung} von $f$ in $X$.
\end{*definition}

\begin{underlinedenvironment}
	Offenbar gilt dann: \begin{align}
	\notag f'(x + y) &= f'(x) + f''(x)y + o(\vert y \vert),\; y\to 0
	\intertext{bzw.}
	\proplbl{taylor_definition_hoehere_ableitung_zwei}
	f'(x+y)\cdot z &= f'(x)\cdot z + \underbrace{\big( \underbrace{f''(x)\cdot y}_{\in K^{m\times n}} \big) z}_{\in K^m} + o(\vert y \vert)\cdot z\quad\forall z\in K^n
	\end{align}
\end{underlinedenvironment}

\begin{boldenvironment}[Interpretation]
	Betrachte $f''(x)$ als kubische bzw. $3$-dimensionale "`Matrix"' (heißt auch \emph{Tensor} 3. Ordnung).
	
	\emph{beachte:} Ausdruck für $f''(x + y)\cdot z$ ist jeweils linear in $y$ und $z$.
\end{boldenvironment}

\begin{boldenvironment}[Frage] höhere Ableitungen, d.h. von $f'':D\to L(K^n, Y_1)$ usw.
	
	Offenbar: \begin{align*}
		g_2&:= L(K^n, Y_1) = L\big(K^n, L(K^n K^m)\big) \cong L(K^n, K^{m\times n}) \cong L(K^n, K^{m\times n}) \cong K^{m\cdot n^2}\\
		g_3&:= L(K^n, Y_2) \cong L(K^n, K^{m\cdot n^2}) \cong K^{k\cdot n^3}
	\end{align*}
	Endlich dimenionale, normierte Räume, man kann rekursiv $\forall k\in \mathbb{N}$ definieren: \begin{enumerate}[label={(\roman*)}]
		\item (Räume)
		
		\begin{tabularx}{\linewidth}{l@{\ }c@{\ }l@{\ }X}
			$Y_0$ & = & $K^n$ &  mit $\vert\,.\,\vert$ \\
			$Y_{k+1}$ &  :=  & $L(K^n, Y_k)$ & mit Standardnormen $\Vert A\Vert_{k+1} = \sup\limits_{\vert z \vert \le 1} \Vert Az\Vert_{Y_k}$ (vgl. Satz 13.8),
		\end{tabularx}
		analog zu oben ist $Y_k\cong K^{m\cdot n^k}$, $Y_k$ normierter Raum
		
		\item (Ableitungen)
		
		\begin{tabularx}{\linewidth}{l@{\ }c@{\ }X}
			$f^{(0)}$ & := & $f:D\subset K^n\to K^m$, $D$ offen.\\
		\end{tabularx}
		Falls $f^{(k)}:D\to Y_k$ \gls{diffbar} in $x\in D$ heißt \[ f^{(k+1)}(x) := \left( f^{(k)} \right) (x)\in L(K^n, Y_k)  \]
		\begriff{$(k+1)$-te Ableitung} von $f$ in $x$. (\emph{beachte:} $f^{(1)}(x) = f'(x)$)
		
		Somit gilt: \begin{align}
			f^{(k)} (x + y) = f^{(k)}(x) + f^{(k+1)}(x) \cdot y + o(\vert y \vert) \;(\in Y_k), \;y\to 0
		\end{align}
	\end{enumerate}
\end{boldenvironment}

\begin{*definition}[$k$-fach differenzierbar]
	$f$ heißt \begriff{$k$-fach differenzierbar} (auf $D$), falls $f^{(k)}$(x) existiert $\forall x\in D$.
	
	$f$ heißt $k$-fach stetig \gls{diffbar} (auf $D$) oder $C^k$-Funktion, falls $f$ $k$-fach \gls{diffbar} und $f^{(k)}:D\to Y_k$ stetig.
	
	$C^k(D, K^m) := \{ f:D\to K^m\mid $f$ \text{ $k$-fach stetig \gls{diffbar} auf $D$} \}$
\end{*definition}

\begin{underlinedenvironment}[Hinweis]
	Falls $f^(k)(x)$ existiert $\Rightarrow$ $f^{(k-1)}$ stetig in $X$  (vgl. \propref{diffbar_impl_stetig})
\end{underlinedenvironment}
\begin{boldenvironment}[Speziafall $n=1$]
	$f:D\subset K\to K^m$\\
	\begin{tabularx}{\linewidth}{r@{$\;$}l@{$\,$}c@{$\,$}X}
	$f'(x)$ & $\in Y_1 = L(K, K^n)$ & $\cong$ & $K^m$ \\
	$f''(x)$ & $\in Y_2 = L(K, Y_1)$ & $\cong$ &  $L(K, K^m)\cong K^m$
	\end{tabularx}
	Allgemein: $f^{(k)}(x) \in Y_k = L(K, Y_{k-1}) \cong L(K, K^m)\cong K^m$, d.h. für $n=1$ kann $f^{(k)}(x)$ stets als $m$-Vektor in $K^m$ betrachtet werden.
\end{boldenvironment}

\begin{example}
	Für $f:\mathbb{R}\to\mathbb{R}$ mit $f(x) = x\cdot \sin x$\\ \begin{tabularx}{\linewidth}{r@{\ \ }r@{\ }c@{\ }l}
		$\Rightarrow$ & $f'(x)$ & = & $\sin x + x\cdot \cos x $ \\
		$\Rightarrow$ & $f''(x)$ & = & $\cos x + \cos x - x \sin x = 2 \cos x - x \sin x$ \\
		$\Rightarrow$ & $f'''(x)$ & = & $-3\sin x - x\cos x$ usw.
	\end{tabularx}
	\begin{center}\begin{tikzpicture}
		\begin{axis}[
		xmin=-5, xmax=5, xlabel=$x$,
		ymin=-5, ymax=5, ylabel=$y$,
		samples=400,
		axis y line=middle,
		axis x line=middle,
		]
		\addplot+[mark=none] {x*sin(deg(x))};
		\addlegendentry{$x\cdot\sin(x)$}
		\addplot+[mark=none, dashed] {sin(deg(x))+x*cos(deg(x))};
		\addlegendentry{$\sin(x)+x\cdot\cos(x)$}
		\addplot+[mark=none, dotted] {2*cos(deg(x))-x*sin(deg(x))};
		\addlegendentry{$2\cos(x)-x\cdot\sin(x)$}
		\end{axis}
		\end{tikzpicture}\end{center}
\end{example}

\begin{example}
	sei $f:\mathbb{R}_{> 0}\to\mathbb{R}^2$ mit $f(x) = \binom{x^3}{\ln x}$. \begin{align*}
		\Rightarrow\;f'(x) &= \begin{pmatrix}
			3x^2 \\ \frac{1}{x}
		\end{pmatrix} & \Rightarrow\; f''(x) &= \begin{pmatrix}
			6x \\ -\frac{1}{x^2}
		\end{pmatrix} & \Rightarrow\; f'''(x) &= \begin{pmatrix}
			6 \\ \frac{2}{x^3}
		\end{pmatrix}
	\end{align*}
\end{example}

\begin{example}
	Sei $f:\mathbb{R}\to\mathbb{R}$ mit \begin{flalign*}
		f(x) &= \begin{cases}
			x^3 & x\ge 0 \\
			-x^3 & x < 0
		\end{cases} &
	\end{flalign*}
	Folglich
	\begin{align*}
		\Rightarrow\; f'(x) &= \begin{cases}
			3x^2 \\ -3x^2
		\end{cases} & \Rightarrow \; f''(x) &= \begin{cases}
			6x \\ -6x
		\end{cases}
	\end{align*}
	$\Rightarrow$ $f'''(0)$ existiert nicht, d.h. $f\in C^2(K, \mathbb{R})$ aber $f\notin C^3(\mathbb{R},\mathbb{R})$
\end{example}

\begin{example}
	\proplbl{tayler_hoehere_ableitungen_beispiel_4}
	Sei $f:\mathbb{R}\to\mathbb{R}$ mit \begin{align*}
		f(x) &= \begin{cases}
			e^{-\frac{1}{x}} & x > 0 \\
			0 & x\le 0
		\end{cases}
	\end{align*}
	$\Rightarrow$ $f^{(k)}(x)$ existiert $\forall x\in\mathbb{R}$, $k\in\mathbb{N}$ mit $f^{(k)}(0) = 0$ $\forall k$, d.h. $f\in C^k(\mathbb{R},\mathbb{R})$ $\forall k\in \mathbb{N}$.
	
	Man schreibt auch $f\in C^\infty(\mathbb{R},\mathbb{R})$
\end{example}

\begin{boldenvironment}[Räume $Y_k$] $=L(K^n, Y_{k-1}) \cong K^{m\times n^k}$.
\end{boldenvironment}

Für $A\in Y_k = L(K^n, Y_{k-1})$ und $y_1, \dotsc, y_k\in K^n$ gilt:

\begin{tabularx}{\linewidth}{l@{$\,$}l@{$\,$}X}
$A\cdot y_1$ & $\in Y_{k-1}$ & $= L(K^n, Y_{k-2})$,\\
$(A y_1)\cdot y_2$ & $\in Y_{k-2}$ & $ = L(K^n, Y_{k-3})$ \\
& $\vdots$ & \\
$(\dotsc (A y_1) y_2) \dotsc \cdot y_k)$ & $ \in Y_0$ & $ = K^m$
\end{tabularx}

Ausdrücke links sind offebar linear in jedem $y_j\in K^n$ separat, $j=1\dotsc,k$

\begin{*definition}[$k$-lineare Abbildung]
Betrachte \begin{align*}
X_k &:= L^k(K^n, K^m) \\
&:= \{ B: \underbrace{K^n \times \dotsc \times K^n}_{\text{$k$-fach}} \to K^m \mid y_j \to B(y_1, \dotsc, y_k) \text{ linear für jedes $j=1,\dotsc,k$ }\}
\end{align*}
$B\in X_k$ heißt \begriff{$k$-lineare Abbildung}. $X_k$ ist Vektorraum.
\end{*definition}

\begin{example}
	Für 3-lineare Abbildung $B\in L^3(\mathbb{R},\mathbb{R}^2)$ mit \begin{align*}
		B(x,y,z) &= \begin{pmatrix}
			xyz \\ (x+y) z
		\end{pmatrix}
	\end{align*}
	ist z.B. \emph{nicht} linear als Abbildung auf $\mathbb{R}^3$.
\end{example}

\begin{proposition}
	\proplbl{taylor_ismomorphismus_yk_xk}
	Für $k\in\mathbb{N}$ ist $I_k:Y_k\to X_k$ mit \begin{align}
		\proplbl{taylor_isomorphismus_yk_xk_eq}
		(I_k A)(y_1, \dotsc, y_k) &:= \left( \dotsc \big( (A y_1)  y_2\big) \dotsc y_k\right) \quad\forall A\in Y_k,\;y_j\in K^n,\;j=1,\dotsc,k
	\end{align}
	ein Isomorphismus bezüglich der Vektorraum-Struktur (also $X_k\cong Y_k$).
	
	\begin{underlinedenvironment}[Hinweis]
		Somit kann $f^{(k)}(x)$ auch als Element von $X_k$ betrachtet werden, d.h. $f^{(k)}(x)\in X_k = L^k(K^n, K^m)$
		
		Damit wird z.B. \eqref{taylor_definition_hoehere_ableitung_zwei} zu \begin{align}
			f'(x+y)\cdot z &= f'(x)\cdot z + f''(x)\cdot(y,z) + o(\vert y \vert)\cdot z\quad\forall z\in K^n
		\end{align}
		und für $n=1$ gilt \begin{align*}
			f^{(k)}(x) (y_1, \dotsc, y_k) &= \underbrace{f^{(k)}(x)}_{\in K^m} \cdot \underbrace{y_1 \cdot \dotsc y_k}_{\mathrlap{\text{Produkt von Zahlen}}} \quad\forall y_j \in K
		\end{align*}
	\end{underlinedenvironment}
\end{proposition}

\begin{proof}
	$I_k$ offenbar linear auf $Y_k$, $I_k$ injektiv, denn $I_k(A) = 0$ \gls{gdw} $A = 0$
	
	Zeige mittels Vollständiger Induktion: $I_,$ surjektiv.
	
	\begin{tabularx}{\linewidth}{@{}lX}
		\emph{IA:} & Offenbar ist $X_1 = Y_1$ und $I_1 A = A$ $\Rightarrow$ $I_1$ surjektiv \\
		\emph{IS:} & Sei $I_k$ surjektiv und wähle beliebiges $B\in X_{k+1}$.
		
		Setze $\tilde{B}_{y_1} := B(y_1, \,\cdot\, ,\dotsc,\,\cdot\,)\in X_k$ $\forall y_1\in K^n$, $\tilde{B}\in L(K^n, X_k)$
	\end{tabularx}
	\zeroAmsmathAlignVSpaces*
	\begin{flalign}
		\proplbl{taylor_ismorphismus_yk_xk_beweis_eq_6}
		\phantom{\emph{\texttt{IA:}}\ \ \ }\;&\Rightarrow \;\; A:=I_k^{-1} \tilde{B}\in L(K^n, Y_k) = Y_{k+1} & \\
		\notag&\; \begin{alignedat}{2}\Rightarrow\;\;(I_{k+1}A)(y_1,\dotsc,y_{k+1}) &\overset{\eqref{taylor_isomorphismus_yk_xk_eq}}{=} \left( \dotsc\big( (Ay_1)y_2 \big) \dotsc y_{k+1}\right) &\;=\;& \big(I_K(Ay_1)\big) (y_2, \dotsc, y_{k+1})\\
		& \overset{\eqref{taylor_ismorphismus_yk_xk_beweis_eq_6}}{=} (\tilde{B}y_1)(y_2, \dotsc, y_{k+1}) &\;=\;& B(y_1, \dotsc, y_{k+1})
		\end{alignedat}
	\end{flalign}
	\begin{tabularx}{\linewidth}{@{}ll@{\ \ }X}
		\phantom{\texttt{IS:}} & \ $\Rightarrow$ & $B = I_{k+1} \cdot A$ $\Rightarrow$ $I_{k+1}$ surjektiv
	\end{tabularx}
	$\Rightarrow$ $I_k$ Isomorphismus
\end{proof}

\begin{boldenvironment}[Norm] in $X_k$, $Y_k$: für $A\in Y_k$ folgt durch rekursive Definition \begin{align}
		\notag
		& \left(\dotsc\left(  \left( A \frac{y_1}{\vert y_1\vert}\right) \frac{y_2}{\vert y_2 \vert} \right) \dotsc \frac{y_k}{\vert y_k \vert} \right) \le \Vert A\Vert_{Y_k} \quad \forall y_j\in K^n,\; y_j\neq 0 \\
		\proplbl{taylor_hoehere_ableitungen_norm}
		\Rightarrow \;\;& \left( \dotsc \big( (Ay_1) y_2 \big) \dotsc y_k \right) \le \Vert A \Vert _{Y_k} \vert y_1\vert \vert y_2\vert \dotsc \vert y_k\vert \quad \forall y_1\,\dotsc,y_k\in K^n 
	\end{align}
	Norm für $A\in X_k = L^k(K^n, K^m)$: \begin{align*}
		\Vert A\Vert _{X_k} := \sup \{ \vert A(y_1, \dotsc, y_k)\vert \mid y_j \in K^n,\; \vert y_j\vert \le 1 \}
	\end{align*}
	Analog zu \eqref{taylor_hoehere_ableitungen_norm} folgt für $A\in X_k$:\begin{align}
		\proplbl{taylor_hoehere_ableitung_abschaetzung_norm}
		\vert A(y_1, \dotsc, y_k)\vert \le \Vert A\Vert_{X_k} \vert y_1 \vert \cdot \dotsc \cdot \vert y_k\vert \quad\forall y_j\in K^n
	\end{align}
\end{boldenvironment}

\begin{proposition}
	Mit Isomorphismus $I_k: Y_k\to X_k$ aus \propref{taylor_ismomorphismus_yk_xk} gilt: \begin{align*}
		\Vert I_(A)\Vert_{X_k} &= \Vert A \Vert_{Y_k} \quad\forall A\in Y_k
	\end{align*}
\end{proposition}
\begin{proof}
	Selbststudium / ÜA
\end{proof}

\begin{remark}
	\proplbl{taylor_partielle_ableitung_isomorphismus_bemerkung}
	$\Vert f^{(k)}(x)\Vert$ unabhängig davon, ob man $f^{(k)}(x)$ als Element von $X_k$ oder $Y_k$ betrachtet.
\end{remark}

\subsection{Partielle Ableitungen}
Sei $X=(x_1, \dotsc, x_k)\in K^n$; d.h. $x_j\in K$, $e_1, \dotsc, e_k$ die Standard-Einheitsvektoren

\begin{boldenvironment}[Wiederholung]
	Partielle Ableitung $f_{x_j} (x) = \frac{\partial}{\partial x_j}f(x) = D_{x_j} f(x)$ ist Richtungsableitung $f'(x, e_j) = D_{e_j} f(x) \in L(K, K^m)$.
\end{boldenvironment}

\begin{*definition}[partielle Ableitung]
	Nenne $f_{x_1}(x), \dotsc, f_{x_1}(x)$ \uline{partielle Ableitung }\begriff[partielle Ableitung!]{1. Ordnung} von $f$ in $X$
	
	Für $g:D\to X$ definieren wir die partielle Ableitung $\frac{\partial}{\partial x_j} g(x) = g_{x_j}(x)\in L(K, X)$ analog zu \propref{richtungsableitung}:\begin{align}
		g(x + t\cdot e_j) &= g(x) + g_{x_j}(x)t + o(t), \;t\to 0,\;t\in K
	\end{align}
	Für $g=f_x:D\to L(K, K^m)$ ist dann $g_{x_j}\in L\big( K, L(K, K^m) \big)$. Für $g = f_{x_j}: D\to L(K, K^m)$ ist dann $g_{x_j}\in L(K, L(K, K^m)) \cong L^2(K, K^m)\cong K^m$
	die  \begriff{partielle Ableitung} $f_{x_i x_j} (x)$ von $f$ in $x$ nach $x_i$ und $x_j$.
	
	Andere Notation: $\frac{\partial^2}{\partial x_j x_i} f(x), D_{x_i x_j} f(x), \dotsc$
	
	Die $f_{x_i x_j}(x)$ heißen \uline{partielle Ableitung} \begriff[partielle Ableitung!]{2. Ordnung} von $f$ in $x$.
	
	Mittels Rekursion \begin{align}
	\proplbl{taylor_partielle_ableitung_definition_10}
		f_{x_{j_1}\dots x_{j_k}}(x) := \frac{\partial}{\partial x_i} f_{x_{i_1} \dots x_{j_k}}
	\end{align}
	erhält man schrittweise die \uline{partielle Ableitung} \begriff[partielle Ableitung!]{der Ordnung $k\in\mathbb{N}$} von $f$ in $x$: \begin{align*}
		f_{x_{j_1}\dots x_{j_k}}(x) = D_{x_{j_1}\dots x_{j_k}} f(x) = \frac{\partial ^k}{\partial x_{j_k} \dots \partial _{x_{j_1}}} f(x) \in L^k(K, K^m)
	\end{align*}
	
	Berechnung durch schrittweises Ableiten von $x_{j_1}\to f(x_1, \dotsc, xn)$, $x_{j_2}\to f_{x_{j_1}}(x_1, \dotsc, x_n)$ usw.
\end{*definition}

\begin{example}
	\proplbl{taylor_partielle_ableitungen_beispiel_9}
	Sei $f:\mathbb{R}^2\to\mathbb{R}$ mit $f(x,y) = y\sin x$ $\forall x,y\in\mathbb{R}$ und \begin{align*}
		f_x(x,y) &= y\cos x & f_y(x,y) &= \sin x\\
		f_{xx}(x,y) &= -y\sin x & f_{yy}(x,y) &= 0 \\
		f_{xy}(x,y) &= \cos x & f_{yx}(x,y) &= \cos x
	\end{align*}
	
	\begin{boldenvironment}[Beobachtung]
		$f_{xy}(x,y) = f_{yx}(x,y)$
	\end{boldenvironment}
\end{example}

Abkürzende Schreibweise: \begin{align*}
	f_{x_j x_j x_j}(x) &= \frac{\partial^3}{\partial x_j \partial x_j \partial x_j} f(x) = \frac{\partial^3}{\partial x_j^3} f(x) \\
	f_{x_i x_j x_i x_l x_l}f(x) &= \frac{\partial}{\partial x_l^2 \partial x_j^2 \partial x_i} f(x)
\end{align*}
\begin{*definition}[\person{Hesse}-Matrix]
Für $m=1$ (d.h. $f:D\subset\mathbb{R}^n\to K$) ist 
\begin{align*}
	\begin{pmatrix}
		f_{x_1 x_1}(x) & \dotsc & f_{x_1 x_n}(x) \\
		\vdots & & \vdots \\
		f_{x_n x_1}(x) & \dotsc & f_{x_n x_n}(x)
	\end{pmatrix} &=: \mathrm{Hess}(f)
\end{align*}
die \begriff{\person{Hesse}-Matrix}, die alle partiellen Ableitungen 2. Ordnung enthält.
\end{*definition}

\begin{example}
	\proplbl{taylor_partielle_ableitung_beispiel_10}
	Sei $f=(f_1, f_2): \mathbb{R}^2\to\mathbb{R}^2$ mit \begin{align*}
		\begin{pmatrix}
			f_1(x_1, x_2) \\ f_2(x_1, x_2)
		\end{pmatrix} = \begin{pmatrix}
			x_1^2 x_2 \\ x_1 x_2 + x_2^2
		\end{pmatrix}
	\end{align*}
	Folglich \begin{align*}
		f_{x_1}(x_1, x_2) &= \begin{pmatrix}
			2 x_1 x_2 \\ x_2
		\end{pmatrix} & f_{x_2}(x_1, x_2) &= \begin{pmatrix}
			x_1^2 \\ x_1 + 2 x_2
		\end{pmatrix}
	\end{align*}
	und \begin{align*}
		\begin{pmatrix}
			2 x_1 x_2 & x_1^2 \\ x_2 & x_1 + 2x_2
		\end{pmatrix}
	\end{align*}
	ist die \person{Jacobi}-Matrix sowie
	\begin{align*}
		\textrm{Hess}(f_1) &= \begin{pmatrix}
			2 x_2 & 2 x_1 \\ 2x_1  & 0
		\end{pmatrix} & \textrm{Hess}(f_2) &= \begin{pmatrix}
			0 & 1 \\ 1 & 2
		\end{pmatrix}
	\end{align*}
	Anschaulich: alle partiellen Ableitungen 2. Ordnung bilden eine 3D Matrix.
\end{example}

\begin{boldenvironment}[Frage]
	Zusammenhang von $f^{(k)}(x)$ mit partiellen Ableitungen?
\end{boldenvironment}

\begin{theorem}
	\proplbl{taylor_partielle_ableitung_zusammenhang_hoehere_ableitung}
	Sei $f:D\subset K^n\to K^m$, $D$ offen, $x\in D$. Dann \begin{enumerate}[label={(\alph*)}]
		\item Falls $f^{(k)}(x)$ existiert, dann existieren alle partiellen Ableitungen der Ordnung $k$ in $x$ und \begin{align}
			\proplbl{taylor_partielle_ableitung_zusammenhang_hoehere_ableitung_eq}
			f_{x_{j_1}\dotsc x_{j_k}}(x) = f^{(k)}(x)(e_{j_k},\dotsc,e_{j_1})
		\end{align}
		\item \proplbl{taylor_partielle_ableitung_zusammenhang_hoehere_ableitung_b}
		Falls alle partiellen Ableitungen $f_{x_{j_1}\dots x_{j_k}}$ der Ordnung $k$ für alle $y\in B_r(x)\subset D$ existieren und falls diese stetig sind \\
		\ \ $\Rightarrow$ $f$ ist $k$-fach \gls{diffbar}, d.h. $f^{(k)}(x)$ existiert.
	\end{enumerate}
\end{theorem}

\begin{remark}
	\propref{taylor_partielle_ableitung_zusammenhang_hoehere_ableitung} \ref{taylor_partielle_ableitung_zusammenhang_hoehere_ableitung_b} ist ein wichtiges Kriterium zur Prüfung der \gls{diffbar}keit, $k$-te Ableitung kann dann mittels \eqref{taylor_partielle_ableitung_zusammenhang_hoehere_ableitung_eq} bestimmt werden.
\end{remark}

\begin{proof}
	Jeweils mittels vollständiger Induktion nach $K$ ausgeführt:\begin{enumerate}[label={\alph*)},topsep=\dimexpr-\baselineskip/2\relax]
		\item basiert auf \propref{richtungsableitung_vollstaendige_reduktion} 
		\item basiert auf \propref{mittelwertsatz_existenz_partieller_ableitung}
	\end{enumerate}
\end{proof}

\begin{example}[nochmal \propref{taylor_partielle_ableitung_beispiel_10}]
	$f^{(2)}(x) = f''(x) \in L^2(\mathbb{R}^2, \mathbb{R}^2)$ existiert $\forall x=(x_1, x_2)\in\mathbb{R}^2$ nach \propref{taylor_partielle_ableitung_zusammenhang_hoehere_ableitung} und kann als Vektor von der \person{Hesse}-Matrix dargestellt werden: \begin{align*}
		f^{(2)} (x) = \begin{pmatrix}
			\textrm{Hess} f_1 \\ \textrm{Hess} f_2
		\end{pmatrix} = \begin{pmatrix}
			\begin{pmatrix}
				2 x_2 & 2 x_1 \\ 2 x_1 & 0
			\end{pmatrix} \\ \begin{pmatrix}
				0 & 1 \\ 1 & 2
			\end{pmatrix}
		\end{pmatrix}
	\end{align*}
	Was ist nun $f''(x)(y_1, y_2)$ für (Vektoren) $y_1$, $y_2\in\mathbb{R}^2$?
	\begin{align*}
		f''(x)(y_1, y_2) &= f''(x) \begin{pmatrix}
			\binom{y_{11}}{y_{12}}, \binom{y_{21}}{y_{22}}
		\end{pmatrix} = f^{(2)}(x) (y_{11} e_1 + y_{12}e_2, y_{21}e_1 + y_{22}e_2) \\
		&= y_{11}f''(x)(e_1,y_2) + y_{12}f''(x)(e_2,y_2)\marginnote{Linearität!} \\
		&= y_{21}y_{11}f''(x)(e_1,e_1) + y_{12}y_{21}f''(x)(e_2,e_1) + y_{11}y_{22}f''(x)(e_1,e_2) + y_{12}y_{22}f''(x)(e_2, e_2) \\
		& \overset{\eqref{taylor_partielle_ableitung_zusammenhang_hoehere_ableitung_eq}}{=} y_{11}y_{21} f''_{x_1 x_1}(x) + y_{12}y_{21}f_{x_1 x_2}(x) + y_{21}y_{22}f_{x_2 x_1}(x) + y_{12} y_{22} f_{x_2 x_2}(x) \;(\in\mathbb{R}^2) \\
		&= \begin{pmatrix}
			\langle (\mathrm{Hess} f_1)(x) y_1, y_2\rangle \\
			\langle (\mathrm{Hess}f_2)(x) y_1, y_2 \rangle
		\end{pmatrix} \in\mathbb{R}^2\quad\forall y_1, y_2\in\mathbb{R}^2
	\end{align*}
\end{example}

Analoge Rechnung liefert allgemein
\begin{conclusion}
	Für $f=(x_1, \dotsc, f_m):D\subset K^n\to K^m$, $D$ offen, es existieren alle $f^{(2)}(x)$ für $x\in D$. Dann \begin{align}
		f^{(2)}(x) (y_1,y_2) = \begin{pmatrix}
			\langle (\mathrm{Hess} f_1)(x) y_1, y_2\rangle \\ 
			\vdots\\
			\langle (\mathrm{Hess} f_m)(x) y_1,y_2\rangle
		\end{pmatrix} \in K^m\; \forall y_1, y_2\in K^n
	\end{align}
\end{conclusion}

\begin{remark}
	Für höhere Ableitungen wird die Darstellung $f^{(k)}(x)(y_1, \dotsc, y_k)$ allgemein mittels partiellen Ableitungen immer komplexer, wird allerdings auch selten benötigt.
\end{remark}

\begin{boldenvironment}[Frage:]
	Kann man die Reihenfolge bei partiellen Ableitungen vertauschen? (vgl. \propref{taylor_partielle_ableitungen_beispiel_9})
\end{boldenvironment}

\begin{example}
	Sei $f:\mathbb{R}^2\to\mathbb{R}^2$ mit\begin{align*}
		f(x,y) = \begin{cases}
			\frac{x^3y - xy^3}{x^2+y^2} & (x,y)\neq(0,0)\\
			0 & (x,y)=(0,0)
		\end{cases}
	\end{align*}
	und folglich \begin{align*}
		f_x(x,y) &= \begin{cases}
			\frac{y(x^4 + 4x^2 y^2 - y^4)}{(x^2 + y^2)^2} & \text{für }(x,y) \neq (0,0) \\
			\lim\limits_{t\to 0}\frac{f(t,0) - f(,0,)}{t} = 0 & \text{sonst}
		\end{cases}
	\end{align*}
	\begin{tabularx}{\linewidth}{l@{\ }l@{\ }c@{\ }r@{\ }l@{\ }X}
		
	insbesondere & $f_x(0,y)$ &=& $-y$& $\forall y\in \mathbb{R}$,& also $f_{xy}(0,0) = -1$ \\
	analog & $f_y(x,0)$ &=& $x$ & $\forall x\in\mathbb{R}$,& also $f_{yx}(0,0) = +1$
	\end{tabularx}
\end{example}

\begin{proposition}[Satz von \person{Schwarz}]
	\proplbl{taylor_partielle_ableitung_schwarz}
	Für $f:D\subset\mathbb{R}^n\to\mathbb{R}^m$, $D$ offen. Mögen die partiellen Ableitungen $f_{x_i}$, $f_{x_j}$, $f_{x_i x_j}$ auf $D$ existieren. Falls $f_{x_i x_j}$ stetig in $x\in D$
	\stepcounter{equation}
	\begin{flalign}
		\proplbl{taylor_partielle_ableitung_schwarz_eq}
		\Rightarrow\;\;& f_{x_j x_i}(x)\text{ existiert und } f_{x_i x_j}(x) = f_{x_j x_i}(x) &\marginnote{(13) fehlt}
	\end{flalign}
\end{proposition}

\begin{conclusion}
	\proplbl{taylor_partielle_ableitung_schwarz_folgerung}
	Sei $f:D\subset\mathbb{R}^n\to\mathbb{R}^m$, $D$ offen, $f$ $k$-fach \gls{diffbar} (d.h. $f\in C^k(D,\mathbb{R}^m)$) \\
	\ $\Rightarrow$ alle partiellen Ableitung bis Ordnung $k$ existieren und die Reihenfolge kann vertauscht werden.
\end{conclusion}

\begin{proof}[\propref{taylor_partielle_ableitung_schwarz_folgerung}]
	Existenz der partiellen Ableitung und deren Stetigkeit folgen aus \propref{taylor_partielle_ableitung_zusammenhang_hoehere_ableitung}, beliebige Vertauschung der Reihenfolge kann durch schrittweises Vertauschen von zwei "`benachbarten Veränderlichen"' erreicht werden.\\
	\ $\xRightarrow{\text{\cref{taylor_partielle_ableitung_schwarz}}}$ Behauptung
	
	Zur Veranschaulichung: \begin{align*}
		f_{x_3 x_1 x_2}(x) & \overset{\eqref{taylor_partielle_ableitung_definition_10}}{=} D_{x_2} f_{x_3 x_1}(x) \overset{\text{\cref{taylor_partielle_ableitung_schwarz}}}{=} D_{x_2}f_{x_1 x_3}(x) \overset{\eqref{taylor_partielle_ableitung_definition_10}}{=} f_{x_1 x_3 x_2}(x) \\
		& \overset{\eqref{taylor_partielle_ableitung_definition_10}}{=} (f_{x_1})_{x_3 x_2}(x) \overset{\text{\cref{taylor_partielle_ableitung_schwarz}}}{=} (f_{x_1})_{x_2 x_3}(x) \overset{\eqref{taylor_partielle_ableitung_definition_10}}{=} f_{x_1 x_2 x_3}(x)
	\end{align*}
\end{proof}

\begin{proof}[\propref{taylor_partielle_ableitung_schwarz}]
	\gls{obda} $m=1$. Fixiere $\epsilon > 0$ $\Rightarrow$ $\exists \delta > 0$ mit \begin{align*}
		x + s\cdot e_i + t\cdot e_j\in D\quad\forall s,t\in (-\delta,\delta)
	\end{align*}
	und
	\begin{align}
		\proplbl{taylor_partielle_ableitung_schwarz_beweis_15}
		\vert f_{x_i x_j}(x + s\cdot e_i + t\cdot e_j) - f_{x_i x_j}(x)\vert < \epsilon \quad\forall s,t\in(-\delta,\delta)
	\end{align}
	
	Definiere $\phi(s) := f(x + s\cdot e_i + t\cdot e_j) - f(x + s\cdot e_i)$ ist \gls{diffbar} auf $(-\delta,\delta)$ $\forall t\in (-\delta,\delta)$ \\
	\begin{tabularx}{\linewidth}{r@{\ \ }X}
	$\xRightarrow{\text{MWS}}$ & $\exists \sigma \in (0,s): \phi(s) - \phi(0) = \phi'(\sigma)s = \left(f_{x_i}(x + \sigma e_i + t e_j) - f_{x_i}(x + \sigma e_i)\right)s$ \marginnote{MWS = Mittelwertsatz, \propref{mittelwertsatz_mittelwertsatz}}\\
	$\xRightarrow{\text{MWS}}$ & für $t\to f_{x_i}(x + \sigma e_i + t e_j)$: $\exists \tau \in (0,t): \phi(s) - \phi(0) = f_{x_i x_j}(\underbrace{x + \sigma e_i + \tau e_j}_{=: \tilde{x}}) s t$ ($\sigma$, $\tau$ abhängig von $s$, $t$)
	\end{tabularx}
	Daher gilt:
	{\zeroAmsmathAlignVSpaces*\begin{align}
	\proplbl{taylor_partielle_ableitung_schwarz_16}
	\notag \left\vert\frac{\phi(s) - \phi(0)}{st} - f_{x_i x_j}(x)\right\vert &\le \underbrace{\left\vert\frac{\phi(s) - \phi(0)}{st} - f_{x_i x_j}(\tilde{x})\right\vert}_{=0} + \left\vert f_{x_i x_j}(\tilde{x}) - f_{x_i x_j}(x) \right\vert& \\
	&\overset{\eqref{taylor_partielle_ableitung_schwarz_beweis_15}}{<} \epsilon \quad\forall s,t\in(-\delta,\delta),\; s,t\neq 0&
	\end{align}}
	Wegen \begin{align*}
		\lim\limits_{t\to 0} \frac{\phi(s) - \phi(0)}{t} = \lim\limits_{t\to 0}\frac{f(x + s\cdot e_i + t\cdot e_j) - f(x + s \cdot e_i)}{t} - \frac{f(x + t\cdot e_j) - f(x)}{t} = f_{x_j}(x + s\cdot e_i) - f_{x_j}(x)
	\end{align*}
	folgt aus \propref{taylor_partielle_ableitung_schwarz_16} \begin{align}
		\proplbl{taylor_partielle_ableitung_schwarz_beweis_17}
		\left\vert \frac{f_{x_j}(x + s\cdot e_i) - f_{x_j}(x)}{s} - f_{x_i x_j}(x) \right\vert < \epsilon\quad \forall s\in (-\delta, \delta);\; s\neq 0
	\end{align}
	\ $\xRightarrow{\epsilon > 0}$ $f_{x_j x_i}(x) = \lim\limits_{s\to 0}\frac{f_{x_j}(x + s\cdot e_i) - f_{x_j}(x)}{s}\overset{\eqref{taylor_partielle_ableitung_schwarz_beweis_17}}{=} f_{x_i x_j}(x)$
\end{proof}

\subsection{Anwendungen}
\begin{boldenvironment}[Frage]
	Wann besitzt $fD\subset\mathbb{R}^n\to\mathbb{R}^{m\times n}$ eine Stammfunktion? (Vgl. \propref{stammfunktion}, \gls{obda} $m=1$)
\end{boldenvironment}

\begin{proposition}[notwendige Integrabilitätsbedingung]
	\proplbl{taylor_anwendung_integrabilitaetsbedinung}
	Sei $f=(f_1, \dotsc, f_n): D\subset\mathbb{R}^n\to\mathbb{R}^n$, $D$ Gebiet\marginnote{Gebiet: offen, zusammenhängend}, $f$ stetig \gls{diffbar}.
	
	Damit $f$ eine Stammfunktion $F:D\to \mathbb{R}$ besitzt, muss folgende \begriff{Integrabilitätsbedingung} erfüllt sein: \begin{align}
		\proplbl{taylor_anwendung_integrabilitaetsbedinung_eq}
		\frac{\partial}{\partial x_i} f_j(x) = \frac{\partial}{\partial x_j} f_i(x)\quad\forall x\in D,\; i,j=1,\dotsc,n
	\end{align}
\end{proposition}

\begin{remark}
	\eqref{taylor_anwendung_integrabilitaetsbedinung_eq} ist hinreichend, falls z.B. $D$ konvex (siehe Analysis 3)
\end{remark}

\begin{proof}
	$f$ habe Stammfunktion $F$ $\Rightarrow$ $F\in C^2(D)$
	
	\begin{tabularx}{\linewidth}{r@{\ \ }l@{\ }c@{\ }l@{\ }l}
	$\Rightarrow$ &$F_{x_j}(x)$& = &$f_j(x)$ &$\forall x\in D,j,i$ \\
	$\Rightarrow$& $F_{x_j x_i}(x)$& = &$\frac{\partial}{\partial x_i} f_j(x)$ & $\forall x\in D,i,j$ \\
	$\xRightarrow{\text{Schwarz}}$ & $F_{x_j x_i}(x)$ &=& \multicolumn{2}{l}{$F_{x_i x_j}(x) = \frac{\partial }{\partial x_j} f_i(x)$}
	\end{tabularx}
\end{proof}

\begin{example}
	Nochmal \propref{stammfunktion_beispiel_11} mit Parameter $\alpha\in\mathbb{R}$: \begin{align*}
		f(x,y) &= \begin{pmatrix}
			\alpha xy \\ x^2 + y^2
		\end{pmatrix}
	\end{align*}
	Betrachte die Ableitungen \begin{align*}
		\frac{\partial}{\partial y} f_1(x,y) &= \alpha x, & \frac{\partial}{\partial x} f_2(x,y) &= 2x
	\end{align*}
	$\xRightarrow{\eqref{taylor_anwendung_integrabilitaetsbedinung_eq}}$ $\alpha = 2$
\end{example}

\begin{proposition}
	Sei $f:D\subset\mathbb{R}^n\to\mathbb{R}$, $D$ offen und konvex, $f$ stetig \gls{diffbar}. Dann:\begin{enumerate}[label={\alph*)}]
		\item $f$ konvex $\Leftrightarrow$ $\langle f'(x), y- x\rangle \le f(y)f(x)$ $\forall x,y\in D$
		\item falls sogar $f\in C^2(D)$, dann: \begin{center}
			%@TODO zu definit linken
				$f$ konvex $\Leftrightarrow$ $f''(x) = (\mathrm{Hess} f)(x)$ positiv definit $\forall x\in D$
		\end{center}
	\end{enumerate}
\end{proposition}
\begin{proof}
	Vgl. Literatur
\end{proof}

\subsection{\person{Taylor}-scher Satz}
\begin{boldenvironment}[Ziel]
	Bessere Approximation als durch Linearisierung
\end{boldenvironment}

Verwende \begriff{allgemeine Polynome} $\phi:K^n\to K$ der Ordnung $k$, d.h. \begin{align}
	\phi(x) = a_0 + \sum_{i=1}^n a_i x_i + \sum_{i,j=1}^n a_{ij} x_i x_j + \dotsc + \sum_{j_1,\dotsc,j_k}^n a_{j_1\dots j_k} x_{j_1}\cdot\dots\cdot x_{j_k}
\end{align}
mit $a_0$, $a_j$, $a_{ij}$ $\in K$ gegebene Koeffizienten

\begin{boldenvironment}[Notation]
	 $f^{(k)}(x)(y,\dotsc,y) = f^{(k)}(x) y^k$
\end{boldenvironment}

\begin{boldenvironment}[Wiederholung]
	$f\in C(D)$: $f(x+y) = f(x) + o(1)$, $y\to 0$ \\
	$f\in C^1(D)$: $f(x+y) = f(x) + f(x)y + o(\vert y \vert)$, $y\to 0$
\end{boldenvironment}

\begin{theorem}[\person{Taylor}-scher Satz]
	 \proplbl{taylor_taylor}
	Sei $f:D\subset K^n\to K^m$, $D$ offen, $k$-fach \gls{diffbar} auf $D$, $x\in D$. Dann \begin{align}
		\proplbl{taylor_taylor_eq}
		f(x+y) = f(x) + \sum_{j=1}^{k-1} \frac{1}{j!} f^{(j)}(x) y^j + R_k(y)\quad\text{falls $[x,x+y]\subset D$,}
	\end{align}
	wobei\begin{align}
		\proplbl{taylor_taylor_restglied_eq_eins}
		\vert R_k(y)\vert \le \frac{1}{k!} \left\vert f^{(k)} (x + \tau y) y^k\right\vert \le \frac{1}{k!}\left\Vert f^{(k)} (x + \tau y)\right\Vert \vert y\vert^k
	\end{align}
	für ein $\tau = \tau(y)\in(0,1)$
	
	Für $K=\mathbb{R}$, $m=1$ gilt auch \begin{align}
		\proplbl{taylor_taylor_restglied_eq_zwei}
		R_k(y) &= \frac{1}{k!} f^{(k)}(x + \tau y) y^k
	\end{align}
	(\person{Lagrange} Restglied)
	
	Falls $f\in C^k(D, K^m)$ gilt: \begin{align}
		\proplbl{taylor_taylor_restglied_eq_drei}
		R_k(y) = \frac{1}{k!} f^{(k)}(x) y^k + o(\vert y\vert^k),\,y\to 0
	\end{align}
\end{theorem}

\begin{remark}
	Entscheidente Aussage in \propref{taylor_taylor} ist nicht \eqref{taylor_taylor_eq}, sondern die Eigenschaften des Restglieds (dies wird klein).
\end{remark}

\begin{proof}
	Sei $[x,x+y]\subset D$, definiere \begin{align*}
		R_K(y) = f(x + y) - f(x) - \sum_{j=1}^{k-1} \frac{1}{j!} f^{(j)}(x) y^j \quad\Rightarrow\eqref{taylor_taylor_eq}
	\end{align*}
	und definiere \begin{align*}\phi(t):= f(x + y) - f(x + ty) - \sum_{j=1}^{k-1} \frac{(1 - t)^j}{j!} f^{(j)}(x+ty) y^j - (1 - t)^k R_k(y)
	\end{align*}
	Offenbar $\phi(1) = 0 = \phi(0)$.
	
	Da $f$ $k$-fach \gls{diffbar}\\
	\begin{tabularx}{\linewidth}{r@{\ \ }X}
	$\Rightarrow$ & $\phi:[0,1]\to K^m$ $\mathbb{R}$-\gls{diffbar} auf $(0,1)$ mit
	\end{tabularx} {\begin{align}
		\proplbl{taylor_taylor_beweis_24}
		\notag\phi'(t) &= -f'(x + ty) \cdot y + \sum_{j=1}^{k-1}\left( \frac{(1 - t)^{j-1}}{(j - 1)!} f^{(j)}(x + ty) y^j - \frac{(1 - t)^j}{j!} f^{(j+1)}(x + ty) y^{j+1}\right) + k (1 - t)^{k - 1} R_k(y) \\
		& = - \frac{(1 - t)^{k-1}}{(k - 1)!} f^{(k)}(x + ty) y^k + k (1 - t)^{k-1} R_k(y)
	\end{align}}
	
	\begin{enumerate}[label={(\alph*)}]
		\item \marginnote{MWS = Mittelwertsatz, \propref{mittelwertsatz_mittelwertsatz}} $K=\mathbb{R}$, $n=1$: nach MWS $\exists \tau\in (0,1)$ und \begin{align*}
			0 = \phi(1) - \phi(0) = \phi'(\tau) \quad \xRightarrow{\eqref{taylor_taylor_beweis_24}} \eqref{taylor_taylor_restglied_eq_zwei}
		\end{align*}
		
		\item zu \eqref{taylor_taylor_restglied_eq_eins} mit $K=\mathbb{R}$: Sei $\psi(t) := \langle \phi(t), v\rangle$ für $v\in\mathbb{R}^n$ \\
		\begin{tabularx}{\linewidth}{r@{\ \ }X}
		$\Rightarrow$ & $\psi:[0,1]\to\mathbb{R}$ \gls{diffbar} auf $(0,1)$ mit $\psi'(t) = \langle \phi'(t),r\rangle$ \\
		$\xRightarrow{\text{MWS}}$ & $\exists \tau\in(0,1)$: $0= \langle \phi'(\tau), v\rangle$
		\end{tabularx}
		 {\begin{flalign}
		 	\proplbl{taylor_taylor_beweis_25}
		\qquad\Rightarrow\;\;&\langle R_K(y),v\rangle = \frac{1}{k!}\langle f^{(k)}(x + \tau y)y^k, v\rangle&
		\end{flalign}}
		mit $v = \frac{R_k(y)}{\vert R_k(y)\vert}$  ($\vert R_k(y)\vert \neq 0$, sonst klar) und es folgt
		\begin{align*}
			\langle R_k(y), v\rangle = \vert R_k(y)\vert = \left\langle \frac{1}{k!} f^{(k)}(x + \tau y) y^k, v\right\rangle \overset{\vert v \vert = 1}{\le} \frac{1}{k!} \left\vert f^{(k)}(x + \tau y) y^k\right\vert \quad\xRightarrow{\eqref{taylor_hoehere_ableitung_abschaetzung_norm}} \eqref{taylor_taylor_restglied_eq_eins}
		\end{align*}
		
		\item $K=\mathbb{C}$: identifiziere $\mathbb{C}^m$ mit $\mathbb{R}^{2m}$ und setzte $\phi(t) = \langle \phi(t), r\rangle_{\mathbb{R}^{2m}}$.
		
		Beachte:\begin{itemize}
			\item $\phi:[0,1]\to\mathbb{R}$, $\frac{\D}{\D t} \Re \phi_j(t) = \Re \frac{\D}{\D t}\phi_j(t)$ $\forall j$
			\item $\langle R_k(y), R_k(y)\rangle_{\mathbb{R}^{2m}} = \vert R_k(y)\vert_{\mathbb{C}^m}^2$
		\end{itemize}
		 und argumentiere wie in b)
			
		\item zu \eqref{taylor_taylor_restglied_eq_drei}: Setzte $R_k(y) = \frac{1}{k!}f^{(k)}(x) y^k + r_k(y)$ in \eqref{taylor_taylor_beweis_25}, $r = \frac{r_k(y)}{\vert r_k(y)\vert}$ (falls $r_k(y) \neq 0$)\\
		\begin{tabularx}{\linewidth}{r@{\ \ }X}
		$\Rightarrow$ & $\displaystyle\frac{\vert r_k(y)\vert}{\vert y \vert^k} \le \frac{1}{k!\vert y\vert^k} \left| \left( f^{(k)}(x + \tau(y)y) - f^{(k)}(x)\right)y^k\right| \overset{\eqref{taylor_hoehere_ableitung_abschaetzung_norm}}{\le} \frac{1}{k!} \left\Vert f^{(k)}(x + \tau (y)y) - f^{(k)}(x)\right\Vert \xrightarrow{y\to 0}0$,
		\end{tabularx}
		d.h. $r_k(y) = o(\vert y \vert^k)$, $y\to 0$
	\end{enumerate}
\end{proof}

\begin{*definition}[Taylorpolynom, Taylorentwicklung]
	Rechte Seite in \eqref{taylor_taylor_eq} ohne Restglied heißt \uline{Taylorpolynom}\begriff*[Taylor-!]{polynom} von $f$ in $x$ vom Grad $k-1$.
	
	\eqref{taylor_taylor_eq} heißt \uline{Taylorentwicklung}\begriff*[Taylor-!]{entwicklung} von $f$ in $x$.
\end{*definition}

\begin{conclusion}[\person{Taylor}-Formel mit partiellen Ableitungen]
	Sei $f:D\subset K^n\to K^m$, $d$ offen, $f$ $k$-fach \gls{diffbar} auf $D$, $x\in D$, $[c,c+y]\subset D$: \begin{align}
		\proplbl{taylor_taylor_partielle_ableitungen_eq}
		f(x + y) = f(x) = \sum_{l=1}^{k-1} \frac{1}{l!} \sum_{j=1}^n f_{x_{j_l}\dots x_{j_1}}(x) y_{j_1}\dots y_{j_l} + R_k(y),
	\end{align}
	wobei $y = (y_1, \dotsc, y_n)\in K^n$ (d.h $y_j\in K$ Zahlen).
\end{conclusion}

\begin{proof}
	Benutze \eqref{taylor_partielle_ableitung_zusammenhang_hoehere_ableitung_eq}
\end{proof}

\begin{remark}
	Falls alle partiellen Ableitungen von $f$ bis Ordnung $k$ existieren und stetig sind auf $D$ \\
	$\Rightarrow$ $f\in C^k(D)$ und \eqref{taylor_taylor_partielle_ableitungen_eq} (vgl. \propref{taylor_partielle_ableitung_zusammenhang_hoehere_ableitung})
\end{remark}

\begin{example}
	Sei $f:\mathbb{R}\to\mathbb{R}$ mit $f(x) = \cos x$. Für $x=0$ gilt: \begin{align*}
		\cos y &= \cos 0 + \frac{1}{1!}\big( \cos'(0) \big)y + \frac{1}{2!}\big( \cos''(0) \big)y^2 + \dotsc + \frac{1}{k!} \big( \cos^{(k)} 0 \big)y^k + o(\vert y \vert^k) \\
		&\overset{k=8}{=} 1 - 0\cdot y - \frac{1}{2}y^2 + 0 y^3 + \frac{1}{24}y^4 - 0\cdot y  - \frac{1}{720} y^6 + 0\cdot y^7 + \frac{1}{40320}y^8 + o(\vert y\vert^8)
	\end{align*}
	(gilt auch für $K=\mathbb{C}$)
\end{example}

\begin{example}
	Sei $f:\mathbb{R}^2\to\mathbb{R}$ mit $f(x) = (x_1^2 + x_1 x_2 + \sin x_2)$ ($x = (x_1, x_2)$)
	
	Taylorentwicklung in $x_0 = (1,\pi)$, $y=(y_1, y_2)\in\mathbb{R}^2$.
	\begin{align*}
		f(x + y) = f(x_0) + f'(x_0) y + \frac{1}{2}f''(x_0) y^2 + \frac{1}{3}f'''(x_0) y^3 + o(\vert y\vert^3)
	\end{align*}
	Offenbar sind \begin{align*}
		f'(x) &= \begin{pmatrix}
			2x_1 + x_2 \\ x_1 + \cos x_2
		\end{pmatrix}& f''(x) &= (\mathrm{Hess} f)(x) = \begin{pmatrix}
			2 & 1 \\ 1 & -\sin x_2
		\end{pmatrix}
	\end{align*}
	und es ergibt sich \begin{align*}
		f(x_0 + y) &
		\! \begin{multlined}[t][0.7\linewidth]
		=f(x_0) + f_{x_1}(x_0) y_1 + f_{x_2}(x_0) y_2 \\
		+ \frac{1}{2!}f_{x_1 x_1}(x_0) y_1^2 + \frac{2}{2}f_{x_1 x_2}(x_0) y_1 y_2 + \frac{1}{2}f_{x_2 x_2}(x)y_2^2\marginnote{$f_{x_1 x_2} + f_{x_2 x_1} = 2 f_{x_1 x_2}$} \\
		 + \frac{1}{3}f_{x_2 x_2 x_2}(x_0) y_2^3 + o(\vert y \vert^3)\end{multlined} \\
		&= 1 + \pi + (2 + \pi) y_1 + 0\cdot y_2 + y_1^2 + y_1 y_2 + 0\cdot y_2^2 + \frac{1}{6}y_2^3 + o(\vert y \vert^3),\;y\to 0
	\end{align*}
\end{example}

\begin{boldenvironment}[Frage]
	Falls $f\in C^\infty(D)$ existiert, dann \begin{align}
		\proplbl{taylor_taylor_reihe_eq}
		f(x + y) = f(x) * \sum \frac{1}{k!}f^{(k)}(x) y^k + \cancel{o(\vert y \vert^k)}\quad \text{für $k=1,\dotsc,n$}
	\end{align}
\end{boldenvironment}

\begin{*definition}[Taylorreihe]
	Rechte Seite in \eqref{taylor_taylor_reihe_eq} heißt \uline{Taylorreihe}\begriff*[Taylor-!]{reihe} von $f$ in $x$.
\end{*definition}

\begin{example}
	Sei $f:\mathbb{C}\to\mathbb{C}$ mit $f(x) = \sin x$ für $x = 0$, dann \begin{align*}
		f^{(k)}(0) = \begin{cases}
			0& \text{$k$ gerade} \\
			(-1)^k & \text{für $k=2l + 1$}
		\end{cases}
	\end{align*}
	$\Rightarrow$ \eqref{taylor_taylor_reihe_eq} hat die folgende Form:\begin{align*}
		\sin y = y - \frac{y^3}{3!} + \frac{y^5}{5!} + \dotsc = \sum(-1)^l \frac{y^{2l+1}}{(2l+1)!} \text{ für }l = 0,\dotsc,\infty
	\end{align*}
	%@TODO: Label Referenz
	Diese gilt $\forall y\in\mathbb{C}$ (vgl. Definition Sinus in Kap. 13), analog Cosinus
\end{example}

\begin{example}
	Sei $f:\mathbb{R}\to\mathbb{R}$ mit \begin{align*}
		f(x) = \begin{cases}
			e^{-\frac{1}{x}} & x> 0 \\
			0 & x\le 0
		\end{cases}
	\end{align*}
	Nach \propref{tayler_hoehere_ableitungen_beispiel_4}: $f\in C^\infty(\mathbb{R})$, $f^{(k)}(0) = 0$ $\forall k\in \mathbb{N}$\\
	\begin{tabularx}{\linewidth}{r@{\ \ }X}
	$\xRightarrow{\eqref{taylor_taylor_reihe_eq}}$ & $f(y) = 0)$ $\forall y$ $\Rightarrow$ \Lightning \\
	$\Rightarrow$ & \eqref{taylor_taylor_reihe_eq} gilt \emph{nicht} für alle $f\in C^\infty(D)$
	\end{tabularx}
\end{example}

\begin{boldenvironment}[Wiedeholung]
	Eine Reihe ist konvergent, falls die Folge der Partialsummen konvergieren, und damit \eqref{taylor_taylor_reihe_eq} gilt, muss die Reihe \emph{auch} gegen $f(x+y)$ konvergieren!
\end{boldenvironment}

\begin{proposition}[Taylorreihe]
	Sei $f:D\subset K^n\to K^m$, $D$ offen, $f\in C^\infty(D, K^m)$, $x\in D$, $B_r(x)\subset D$. Falls \begin{align*}
		\lim\limits_{k\to\infty} R_k(y) = 0\quad\forall y\in B_r(x)
	\end{align*}
	$\Rightarrow$ Taylorformel \eqref{taylor_taylor_reihe_eq} gilt $\forall y\in B_r(x)$ und $f$ heißt \begriff{analytisch} in $x$.
\end{proposition}

\begin{proof}
	Folgt direkt aus \propref{taylor_taylor}
\end{proof}

\begin{example}
	$\sin$, $\cos$, $\exp:\mathbb{C}\to\mathbb{C}$ sind jeweils analytisch in allen $x\in\mathbb{C}$ und \eqref{taylor_taylor_reihe_eq} gilt jeweils $\forall y\in\mathbb{C}$ (klar für $x=0$) aus der Definition, für $x\neq 0$ erfolgt der Nachweis als ÜA / Selbststudium.
\end{example}
\section{Extremwerte} \setcounter{equation}{0}
\subsection{Lokale Extrema ohne Nebenbedingung}
Betrachte $f:D\subset\mathbb{R}^n\to\mathbb{R}$\marginnote{Zielraum $\mathbb{R}$ wegen Ordnung}, $D$ offen, $f$ \gls{diffbar}.

\begin{boldenvironment}[notwendige Bedingung] (\propref{mittelwertsatz_optimalitaetsbedingung}): $f$ hat lokales Minimum / Maximum in $x\in D$ $\Rightarrow$ $f'(x) = 0$
\end{boldenvironment}

\begin{boldenvironment}[Frage]
	Hinreichende Bedingung?
\end{boldenvironment}

\begin{*definition}[definit, semidefinit, indefinit]
	$f^{(k)}(x)$ für $k\ge $ heißt positiv \begriff{definit} (negativ definit), falls \begin{align}
		\proplbl{extremwerte_definit_definition_eq}
		f^{(k)}(x) y^k > 0 \;(< 0) \quad\forall y\in\mathbb{R}\setminus \{0\}
	\end{align}
	und positiv (negativ) \begriff{semidefinit} mit $\ge$ ($\le$).
	
	$f^{(k)}$ heißt \begriff{indefinit}, falls \begin{align}
		\proplbl{extremwerte_indefinit_definition_eq}
		\exists y_1, y_2\in\mathbb{R}^n\setminus \{0\}: f^{(k)}(x) y_1^k < 0 < f^{(k)} (x) y_2^k
	\end{align}
\end{*definition}

\begin{underlinedenvironment}[Hinweis]
	$k$ ungerade, $f^{(k)}(x)\neq 0$ $\Rightarrow$ $f^{(k)}(x)$ indefinit, denn $f^{(k)}(-y)^k = (-1)^k f^{(k)}(x) y^k$
\end{underlinedenvironment}

\begin{proposition}[Hinreichende Extremwertbedingung]
	\proplbl{extremwerte_hinreichende_bedingung}
	Sei $f:D\subset\mathbb{R}^n\to\mathbb{R}$, $D$ offen, $f\in C^k(D,\mathbb{R})$, $x\in D$, $k\ge 2$ und sei \begin{align}
		\proplbl{extremwerte_hinreiche_bedingung_eq}
		f'(x) = \dotsc = f^{(k-1)} = 0
	\end{align}
	Dann: \begin{enumerate}[label={\alph*)}]
		\item $f$ hat strenges lokales Minimum (Maximum), falls $f^{(k)}(x)$ positiv (negativ) definit
		\item \proplbl{extremwerte_hinreichende_bedinung_b}
		$f$ hat weder Minimum noch Maximum, falls $f^{(k)}(x)$ indefinit.
	\end{enumerate}
\end{proposition}

\begin{remark}\vspace*{0pt}
	\begin{enumerate}[label={\arabic*)}]
		\item Falls $f^{(k)}(x)$ positiv (negativ) semidefinit $\Rightarrow$ keine Aussage möglich.
		
		(betrachte $f:\mathbb{R}^2\to\mathbb{R}$ mit $f(x_1, x_2) = x_1^2 + x_2^4$, hat Minimum in $x=0$, aber $f(x_1, x_2) = x_1^2 + x_2^3$ hat weder Minimum noch Maximum in $x=0$)
		
		\item \ref{extremwerte_hinreichende_bedinung_b} liefert: $f^{(k)}(x) \neq 0$ positiv (negativ) semidefinit ist notwendige Bedingung für ein lokales Minimum bzw. Maximum, falls \eqref{extremwerte_hinreiche_bedingung_eq} gilt
	\end{enumerate}
\end{remark}

\begin{proof}\hspace*{0pt}
	\begin{enumerate}[label={zu \alph*)},topsep=\dimexpr-\baselineskip / 2\relax,leftmargin=\widthof{\texttt{zu a)\ }}]
		\item \proplbl{extremwerte_hinreichende_bedingung_beweis_a}
		Für Minimum (Maximum analog):
		
		Sei $f^{(k)}(x)$ positiv definite Abbildung, $y\to f^{(k)}(x) y^k$ stetige Abbildung (folgt aus \propref{taylor_partielle_ableitung_isomorphismus_bemerkung}).
		
		Sei $S=\{ y\in\mathbb{R}^n \mid \vert y \vert = 1 \}$ ist kompakt \\
		{\renewcommand{\arraystretch}{1.5}
		\begin{tabularx}{\linewidth}{r@{\ \ }r@{\ }c@{\ }X}
		$\xRightarrow{\text{\propref{chap_15_3}}}$ & \multicolumn{3}{l}{$\exists\tilde{y}\in S: f^{(k)}(x) y^k \ge f^{(k)}(x)\tilde{y}^k =: \gamma > 0\;\forall y\in S$} \\
		$\xRightarrow{\text{\propref{taylor_taylor}}}$ & $f(x+y)$& = &$f(x) + \frac{1}{k!}f^{(k)}(x) y^k + o(\vert y\vert^k)$\\
		& & = &$f(x) + \frac{1}{k!}\vert y \vert^k \left(\underbrace{f^{(k)}(x) \left(\frac{y}{\vert y \vert}\right)^k}_{\ge \gamma} + \underbrace{o(1)}_{\mathclap{\ge -\frac{\gamma}{2}}}\right),\;\vert y \vert\to 0$ \\
		&& $\ge$ & $f(x) + \frac{\gamma}{2k!}\cdot \vert y \vert^k$ $\forall y\in B_r(0)$ falls $y\in B_r(0)$, $r>0$ klein \\
		$\Rightarrow$ &\multicolumn{3}{l}{$x$ ist strenges, lokales Minimum $\Rightarrow$ Behauptung}
		\end{tabularx}}
		
		\item Wähle $y_1, y_2$ gemäß \eqref{extremwerte_indefinit_definition_eq}, \gls{obda} $\vert y_1 \vert = \vert y_2 \vert = 1$ \\
		\begin{tabularx}{\linewidth}{r@{\ \ }X}
		$\xRightarrow[\text{$\vert t \vert$  klein}]{\text{analog \ref{extremwerte_hinreichende_bedingung_beweis_a}}}$ & $f(x + ty_1) = f(x) + \frac{t^k}{k!}\left(f^{(k)}(x) y_1^k + o(1) \right) < f(x)$, \\
		&$f(x + ty_2) = f(x) + \frac{t^k}{k!} \left( f^{(k)}(x) y_2^k + o(1) \right) > f(x)$
		\end{tabularx}
		$\Rightarrow$ Behauptung
	\end{enumerate}
\end{proof}

\begin{boldenvironment}[Test Definitheit in Anwendungen]
	$k=2$ wichtig (vgl. lineare Algebra).
	
	$f''(x) \in L^2(\mathbb{R}^2,\mathbb{R})\cong \mathbb{R}^{n\times n}$ (\person{Hesse}-Matrix)
	
	$f''(x) y^2 = f''(x) (y,y) = \langle (\mathrm{Hess} f)(x)y, y\rangle$, vgl. \propref{taylor_partielle_ableitung_beispiel_10}
	
	Matrix $A\in\mathbb{R}^{n\times}$ ist \begin{itemize}
		\item positiv (negativ) definit $\Leftrightarrow$ alle Eigenwerte sind positiv (negativ) 
		\item indefinit $\Rightarrow$ $\exists$ positive und negative Eigenwerte
	\end{itemize}
\end{boldenvironment}

\subsection{Sylvester'sches Definitheitskriterium}
Eine symmetrische Matrix $A=(a_{ij})\in\mathbb{R}^{n\times n}$ ist positiv definit \gls{gdw} alle führenden Hauptminoren positiv sind, d.h. \begin{align*}
	\alpha_k := \det\begin{pmatrix}
		\alpha_{11} & \dots & \alpha_{1k} \\ \vdots && \vdots \\ \alpha_{k1} & \dots & \alpha_{kk}
	\end{pmatrix} > 0\quad\forall k\in\{1,\dotsc,n\}
\end{align*}

\begin{underlinedenvironment}[beachte]
	$A$ negativ definit $\Leftrightarrow$ $-A$ positiv definit
\end{underlinedenvironment}
\begin{boldenvironment}[Spezialfall $n=2$]\vspace*{0pt}
	\begin{itemize}[topsep=\dimexpr-\baselineskip/2\relax]
		\item $\det A <0 $ $\Leftrightarrow$ indefinit
		\item $\alpha_1 < 0$ und $\det A > 0$ $\Leftrightarrow$ negativ definit
	\end{itemize}
\end{boldenvironment}

\begin{example}
	\proplbl{extremwerte_sylvester_beispiel_3}
	Sei $f:\mathbb{R}^2\to\mathbb{R}$ mit $f(x_1, x_2) = x_1^2 + \cos x_2$ \\
	\begin{tabularx}{\linewidth}{r@{\ \ }X}
	$\Rightarrow$ & $f'(x_1, x_2) = (2x_1) - \sin x_2 = 0$ \\
	$\Rightarrow$  &$x_1 = 0$, $x_2 = k\cdot\pi$, d.h. $\tilde{x} = (0, k\cdot\pi)$ für $k\in\mathbb{Z}$ sind Kandidaten für Extrema.
	\end{tabularx} \begin{align*}
		f''(x_1, x_2) &= \begin{pmatrix}
			2 & 0 \\ 0 & -\cos x_2
		\end{pmatrix} & \Rightarrow\;\; f(\tilde{x}) &= \begin{pmatrix}
			2 & 0 \\ 0 & (-1)^{k+1}
		\end{pmatrix}
	\end{align*}
	entsprechend ergeben sich folgende Fälle:\\
	\begin{tabularx}{\linewidth}{r@{\ \ }Xr@{\ \ }X}
	$\Rightarrow$ & $f''(\tilde{x})$ ist positiv definit für $k$ ungerade & $\Rightarrow$ & $f''(\tilde{x})$ ist indefinit für $k$ gerade \\
	$\Rightarrow$ & lokales Minimum,&
	$\Rightarrow$ & kein Extremum
	\end{tabularx}
\end{example}

\subsection{Lokale Extrema mit Gleichungsnebenbedingung}
Betrachte $f:D\subset\mathbb{R}^n\to\mathbb{R}$ \gls{diffbar}, $D$ offen, $g:D\subset\mathbb{R}^n\to\mathbb{R}$ \gls{diffbar}

\begin{boldenvironment}[Frage:]
	Bestimmen von Extrema von $f$ auf der Menge $G:= \{ x\in\mathbb{R}\mid g(x) = 0 \}$, d.h. suche notwendige Bedingung (für hinreichende Bedingung sieh Vorlesung Optimierung)
\end{boldenvironment}

\begin{boldenvironment}[Motivation]
	Für $m\ge 1$: notwendige Bedingung: $f'(\mathrm{max})$ steht senkrecht auf der Niveaumenge $G$ \marginnote{(vgl. \propref{richtungsableitung_gradient_eigenschaften})} \\
	$\Rightarrow$ $\exists\lambda\in\mathbb{R}: f'(x_{\max}) + \lambda g'(x_{\max}) = 0$
\end{boldenvironment}

\begin{proposition}[Lagrange-Multiplikatorregel, notwendige Bedingung]
	Seien $f:D\subset\mathbb{R}^n\to\mathbb{R}$, $g:D\to\mathbb{R}^m$ stetig, \gls{diffbar}, $D$ offen und sei $x\in D$ lokales Extremum von $f$ bezüglich $G$, d.h. \begin{align*}\exists r > 0: f(x)\; \substack{\le \\ \ge}\; f(y)\quad\forall y\in B_r(x)\end{align*} mit $g(y) = 0$.
	
	Falls $g'(x)$ regulär, d.h. \begin{align}
		\proplbl{extremwerte_lokale_extrema_mit_gleichungsnebenbedingung_lagrange_multiplikator_eq_4}
	\mathrm{rang}\; g'(x) = m, \end{align}dann
	\begin{align}
	\proplbl{extremwerte_lokale_extrema_mit_gleichungsnebenbedingung_lagrange_multiplikator_eq_5}
	\exists \lambda\in\mathbb{R}^m: f'(x) + \transpose{\lambda} g'(x) = 0\end{align}
\end{proposition}

\begin{*definition}[Lagrangescher Multiplikator]
	$\lambda$ oben heißt \begriff{Lagrangescher Multiplikator}
\end{*definition}

\begin{remark}\vspace*{0pt}
\begin{itemize}
	\item Offenbar nur für $m\le n$
	\item $x$ mit \eqref{extremwerte_lokale_extrema_mit_gleichungsnebenbedingung_lagrange_multiplikator_eq_4} heißt \begriff{reguläres Extrema}.
	\item Kandidaten für Extrema bestimmen: \eqref{extremwerte_lokale_extrema_mit_gleichungsnebenbedingung_lagrange_multiplikator_eq_5} liefert $n$ Gleichungen für $n+m$ Unbekannte $(x,\lambda)$, \emph{aber} \eqref{extremwerte_lokale_extrema_mit_gleichungsnebenbedingung_lagrange_multiplikator_eq_5} mit $g(x) = 0$ liefert $n+m$ Gleichungen für $(x,\lambda)$
\end{itemize}
\end{remark}

\begin{proof}
	Vgl. Literatur.
\end{proof}

\begin{example}
	Bestimme reguläre Extrema von $f$ auf $G=\{ g=0\}$ mit
	\begin{alignat*}{2}
		f&:\mathbb{R}^3\to\mathbb{R},&\, (x,y,z) \mapsto\;&x^2 + y^2 + z^2 \\
		g&:\mathbb{R}^3\to\mathbb{R}^2,& (x,y,z) \mapsto& \begin{pmatrix}
	x^2 + 4y^2 - 1 \\ z
	\end{pmatrix}
	\end{alignat*}
	Betrachte $\transpose{\lambda} = (\lambda_1,\lambda_2)$: \begin{align}
		0 &= f'(x,y,z) + \transpose{\lambda}g'(x,y,z) = (2x, 2y, 2z) + \transpose{\lambda}\cdot\begin{pmatrix}
			2x & 8y & 0 \\ 0 & 0 & 1
		\end{pmatrix} \\
		\notag 0 &= g(x,y,z)
	\end{align}
	Das heißt \begin{align*}
		2x + 2\lambda_1 x &= 0 & x^2 + 4y^2 &= 1 \\
		2y+8\lambda_1 y &= 0 & z &= 0 \\
		2z + \lambda_2 &= 0 & &
	\end{align*}
	$\Rightarrow$ $z=0$, $\lambda_2 = 0$, und \begin{align*}
		x(1 + \lambda_1) &= 0 & y(1 + 4\lambda_1) &= 0 & x^2 + 4y^2 &= 1
	\end{align*}
	\begin{tabularx}{\linewidth}{@{}r@{\ }l@{\ }l@{\ }c@{\ }r@{\ }l@{\ }c@{\ }r@{\ }l@{\ \ }c@{\ \ }r@{$\,$}r@{$\,$}r@{$\,$}r@{$\,$}lX}
	falls:& $\bullet$ $x\neq 0$:& $\lambda_1$ & = & $-1$, &$y$ & = & $0$, &$x$ = $\pm 1$ &$\Rightarrow$& $($ & $\pm1$, & 0, &0 & $)$ & \multirow{2}{*}{$\left.\phantom{\dfrac{1}{1}}\right\} \text{Kandidaten für reguläre Extrema}$} \\
    &	$\bullet$ $x = 0$:  & $y$ & = &$\pm\frac{1}{2}$, &$\lambda_1$ & = & $-\frac{1}{1}$& & $\Rightarrow$& $($ & 0,&$\pm\frac{1}{2}$, &0 &$)$	\end{tabularx}

	Offenbar ist $\mathrm{rang}\;g'(x,y,z)=2$ für alle Kandidaten.
	
	Da $G$ Ellipse in der $x$-$y$-Ebene ist, und $f$ die Norm in's Quadrat, prüft man leicht: Minimum in $(0,\pm\frac{1}{2},0)$ und Maximum in $(\pm 1,0,0)$.
\end{example}

\subsection{Globale Extrema mit Abstrakter Nebenbedinung}
Betrachte $f:\overline{D}\subset\mathbb{R}\to\mathbb{R}$, $D$ offen, $f$ stetig auf $\overline{D}$, \gls{diffbar} auf $D$.

\begin{boldenvironment}[Existenz]
	nach \propref{chap_15_3}:
	
	$D$ beschränkt $\xRightarrow{\text{$\overline{D}$ kompakt}}$ $f$ besitzt auf $\overline{D}$ ein Minimum und ein Maximum
\end{boldenvironment}

\begin{boldenvironment}[Frage]
	Bestimme sogenannte globale Extremalstelle $x_{\min}$, $x_{\max}$.
\end{boldenvironment}

\begin{boldenvironment}[Strategie:]\vspace*{0pt}
	\begin{enumerate}[label={\alph*)},topsep=\dimexpr-\baselineskip/2\relax]
		\item Bestimmte lokale Extrema in $D$
		\item Bestimme globale Extrema auf $\partial D$
		\item Vergleiche Extrema aus a) und b)
	\end{enumerate}
\end{boldenvironment}

\begin{example}
	Sei $f(x_1, x_2) = x_1^2 + \cos x_2$ mit $D=(-1,1)\times(0,4)$ (vgl. \propref{extremwerte_sylvester_beispiel_3}).
	
	Lokale Extrema in $D$: $f(0,\pi) = -1$ Minimum.
	
	Globale Extrema auf $\partial D$:\begin{itemize}
		\item $x_1 = \pm 1:$ Betrachte $x_2 \to f(\pm 1, x_2) = 1 + \cos x_2$ auf $[0,4]$.
		
		Offenbar $0 = f(\pm 1, \pi) \le f(\pm 1, x_2) \le f(\pm 1, 0) = 2$
		
		\item $x_2=0$: $x_1\to f(x_1, 0) = x_1^2 + 1$ auf $[-1,1]$
		
		Offenbar $1=f(0,0) \le f(x_1,0)\le f(\pm 1, 0) = 2$
		
		\item $x_2 = 4$: Betrachte $x_1\to x_1^2+\cos 4$ mit $[-1,1]$
		
		$\cos 4 \le f(0,4) \le f(x_1, 4) \le f(\pm 1, 4) = 1 + \cos 4$
	\end{itemize}
	Vergleich liefert:$ x_{\min}=(0,\pi)$, $x_{\max} = (\pm 1,0)$
	
	\begin{underlinedenvironment}[Hinweis]
		Bentze für Extrema evtl. partielle Ableitungen \begin{align*}
			f_{x_2}(\pm 1, x_2) &= -\sin x_2 = 0 \\
			\text{bzw.}\quad f_{x_1}(x_1, 0) &= 2x_1 = 0 \qquad\qquad\text{usw.}
		\end{align*}
	\end{underlinedenvironment}
\end{example}
\section{Inverse und implizite Funktionen}\setcounter{equation}{0}

\begin{*definition}[(lokale) Lösung]
	Funktion $\tilde{y}:\tilde{D}\subset K^n\to K^m$ heißt (lokale) Lösung von in $x$ auf $\tilde{D}$ falls \begin{align}
		f(x,\tilde{y}(x)) = 0 \quad\forall x\in\tilde{D}\notag
	\end{align}
\end{*definition}

\begin{theorem}[Satz über implizite Funktionen]
	Sei $f:D\subset \mathbb{R}^m \times K^m\to K^m$, $D$ offen, $f$ stetig und \begin{enumerate}[label={\alph*)}]
		\item $f(x_0, y_0) = 0$ für ein $(x_0, y_0)\in D$
		\item die partielle Ableitung $f_y:D\to L(K^m, K^n)$ existiert, ist stetig in $(x_0, y_0)$ und $f_y(x_0, y_0)$ ist regulär
	\end{enumerate}
	Dann:\begin{enumerate}[label={\arabic*)}]
		\item $\exists r,\rho > 0$: $\forall x\in B_r(x_0)\;\exists! y=\tilde{y}\in B_\rho(y_0)$ mit $f(x,\tilde{y}(x)) = 0$ und $\tilde{y}:B_r(x_0)\to B_\rho(y_0)$ stetig
		\item
		falls zusätzlich $f:D\to K^m$ stetig diffbar\\
		$\Rightarrow$ auch $\tilde{y}$ stetig diffbar auf $B_r(x_0)$ mit \begin{align*}
			\tilde{y}'(x) &= -\underbrace{f_y(x,\tilde{y}(x))^{-1}}_{m\times n} \cdot \underbrace{f_x(x,\tilde{y}(x))}_{m\times n}\quad\in K^{m\times n}
		\end{align*}
	\end{enumerate}
\end{theorem}

\begin{example}
	Sei $f:\mathbb{R}\times\mathbb{R}\to\mathbb{R}$ mit $f(x,y) = x^2(1 - x^2) - y^2$ $\forall x,y\in\mathbb{R}$.
	
	Offenbar ist \begin{align*}
		f_x(x,y) &= 2x(1 - x^2) - 2x^3 = 2x - 4x^3 \\
		f_y(x,y) &= -2y
	\end{align*}
	
	Suche Lösungen von $f(x,y) = 0$ \\
	\renewcommand{\arraystretch}{1.5}
	\begin{tabularx}{\linewidth}{c@{\ }l@{$\;\,$}X}
		$\bullet$ & $y_0=0$:& $f_y(x_0, 0) = 0$ nicht regulär $\Rightarrow$ Theorem nicht anwendbar \\
		$\bullet$ & $y_0\neq 0$: & $f_y(x_0, y_0)\neq 0$, also regulär.
		
		Sei $f(x_0, y_0)$ = 0 $\Rightarrow$ z.B. $(x_0, y_0) = (\frac{1}{3}, \frac{2\cdot\sqrt{2}}{9})$ ist Nullstelle von $f$  \\
		&&$\Rightarrow$ $\exists r,\rho > 0$, Funktion $\tilde{y}:f(x,\tilde{y}(x)) = 0$ $\forall x\in B_r(\frac{1}{3})$
		
		$\tilde{y}(\frac{1}{3}) = \frac{2\cdot \sqrt{2}}{9}$ und $\tilde{y}(x)$ ist einzige Lösung um $B_\rho(\frac{2\sqrt{2}}{9})$\\
		
		&& $\begin{aligned}\tilde{y}'\left(\frac{1}{3}\right) &= -f_y\left(\frac{1}{3}, \frac{2\sqrt{2}}{9}\right)^{-1}\cdot f_x\left(\frac{1}{3}, \frac{\sqrt{2\sqrt{2}}}{9}\right) \\
		&= -\left(-\frac{4\sqrt{2}}{9}\right)^{-1}\cdot\left(\frac{2}{3} - \frac{4}{27}\right) = \frac{7}{6\sqrt{2}} \approx 0,8\end{aligned}$ \\
		
		$\bullet$ & $y_0 = 0$, $x_0 = 1$: & hier ist $f_x(1,0) = -2$, also regulär \\
		&& $\Rightarrow$ $\exists$ lokale Lösung $\tilde{x}(y)$: $f(\tilde{x}(y), y) = 0$ $\forall y\in B_{\tilde{r}}(0)$ und $\tilde{x}'(0) = 0$ \\
		
		$\bullet$ & $y_0 = 0$, $x_0 = 0$: & $f_x(0,0) = f_y(0,0) = 0$ nicht regulär\\
		&& $\Rightarrow$ in keiner Variante Anwendbar.
	\end{tabularx}
\end{example}

\begin{theorem}[Satz über inverse Funktionen]
	Sei $f:U\subset K^n\to K^n$, $U$ offen, $f$ stetig diffbar, $f'(x)$ regulär für ein $x_0\in U$ \\
	$\Rightarrow$ Es existiert eine offene Umgebung $U_0\subset U$ von $x_0$, sodass $V_0:= f(U_0)$ offene Umgebung von $y_0 := f(x_0)$ ist, und die auf $U_0$ eingeschränkte Abbildung $f:U_0\to V_0$ ist Diffeomorphismus.
\end{theorem}
\begin{proof}
	benutze $\tilde{f}:D\times K^n\to K^n$ und $\tilde{f}(x)=f(x)-y\Rightarrow$, $\tilde{f}$ stetig und stetig diffbar $\Rightarrow$ Satz über implizite Funktionen $\Rightarrow f'$ stetig diffbar $\Rightarrow f$ ist Diffeomorphismus
\end{proof}

\begin{proposition}[Ableitung der inversen Funktion]
	Sei $f:D\subset K^n\to K^n$, $D$ offen, $f$ injektiv und \gls{diffbar}, $f^{-1}$ \gls{diffbar} in $y\in \mathrm{int}\, f(D)$ \begin{align}
		\Rightarrow \quad\left(f^{-1}\right)'(y) &= f'\left( f^{-1}(y)\right)^{-1}\notag
	\end{align}
	(bzw. $(f^{-1})'(y) = f'(x)^{-1}$ falls $y=f(x)$)
	
	Spezialfall$ n = m = 1$: $(f^{-1})'(y) = \frac{1}{f'(f^{-1}(y))}$
\end{proposition}

\begin{proof}
	benutze $f(f^{-1}(y))=y$ und $f^{-1}(f(x))=x$, Kettenregel, $f'(f^{-1}(y))\cdot (f^{-1})'(y)=\id$, andere Gleichung analog, gleichsetzen, Behauptung
\end{proof}

\begin{proposition}
	Sei $f:D\subset K^n\to K^n$, $D$ offen, $f$ stetig diffbar, $f'(x)$ regulär $\forall x\in D$
	\begin{enumerate}[label={(\alph*)}]
		\item (Satz über offene Abbildungen) $f(D)$ ist offen
		\item (Diffeomorphiesatz) $f$ injektiv $\Rightarrow$ $f:D\to f(D)$ ist Diffeomorphismus
	\end{enumerate}
\end{proposition}

\begin{proof}\hspace*{0pt}
	\begin{enumerate}[label={(\alph*)}]
		\item $M\subseteq D$ offen, $y_0\in f(M)\Rightarrow\exists x_0\in M:y_0=f(x_0)\Rightarrow$ Satz über inverse Funktionen $\Rightarrow\exists V_0\subseteq f(M)$ von $y_0\beha$
		\item offenbar $\exists f^{-1}:f(D)\to D\Rightarrow$ Satz über inverse Funktionen $\Rightarrow f^{-1}$ stetig diffbar $\beha$
	\end{enumerate}
\end{proof}
\section{Funktionsfolgen}\setcounter{equation}{0}

Betrachte $f_k:D\subset K^n\to K^m$, $D$ offen, $f_k$ \gls{diffbar} für $k\in\mathbb{N}$

\begin{boldenvironment}[Frage:]
	Wann konvergiert $\{ f_k\}_{k\in\mathbb{N}}$ gegen \gls{diffbar}e Funktion $f$ mit $f_k'\to f'$
\end{boldenvironment}

\begin{boldenvironment}[Wiederholung]
	alle $f_k$ stetig, $f_k\to f$ gleichmäig auf $D$ $\xRightarrow{\text{\propref{chap_14_19}}}$ $f$ stetig
\end{boldenvironment}

\begin{*example}
	Sei $f_k:\mathbb{R}\to\mathbb{R}$ mit $f_k(x) = \frac{\sinh^2 x}{k}$.
	
	Wegen $\vert f_k(x)\vert \le \frac{1}{k}$ $\forall k$ $\Rightarrow$ $f_k\to f$ gleichmäßig auf $\mathbb{R}$ für $f=0$

	Aber $f_k'(x) = k\cdot\cosh^2 x \cancel{\rightarrow} f'(x) = 0$
\end{*example}

\begin{example}
	Sei $f_k:\mathbb{R}\to\mathbb{R}$ mit $f_k(x) = \sqrt{x^2 + \frac{1}{k}}$, wobei $f(x) = \vert x \vert$\\
	$\Rightarrow$ alle $f_k$ \gls{diffbar}, $f_k \to f$ gleichmäßig auf $[-1,1]$ 
	und $(\vert f_k(x)  - f(x)\vert \le f_k(0)\frac{1}{\sqrt{k}}$ \emph{aber} $f$ nicht \gls{diffbar} %TODO abbildung
\end{example}

\begin{example}
	Sei $f_k:\mathbb{R}\to\mathbb{R}$ mit $f_k(x) = \frac{\sin kx}{x}$, $\Rightarrow f_k \to f(x) = 0$ gleichmäßig auf $\real$ (da $\vert f_k(x) \vert \le \frac{1}{k} \forall x \in \real$) \\
	\emph{aber} $f^{'}_{k}(x) = \cos kx \not\to f^{'}(x) = 0$
\end{example}

\begin{proposition}[Differentiation bei Funktionsfolgen]
	\proplbl{funktionsfolgen_differentiation}
	Sei $f_K:D\subset K^n\to K^m$, $D$ offen, beschränkt, $f_k$ \gls{diffbar} $\forall k$ und\begin{enumerate}[label={(\alph*)}]
		\item $f_k'\to: g$ gleichmäßig auf $B_r(x)\subset D$
		\item \proplbl{funktionsfolgen_differentiation_b} $\{ f_k(x_0)\}_{k}$ konvergiert für ein $x_0\in B_r(x)$
	\end{enumerate}
	$\Rightarrow$ $f_k\to: f$ gleichmäßig auf $B_r(x)$ und $f$ ist \gls{diffbar} auf $B_r(x)$ mit 
	\begin{align*}
		f_k'(y) \rightarrow f'(y) \quad\forall y\in B_r(x)
	\end{align*}
\end{proposition}

\begin{underlinedenvironment}[Hinweis]
	Betrachte $f_k(x) := \frac{\sin x}{k} + k$ auf $\real$ um zu sehen ($g = 0$), dass Voraussetzung \ref{funktionsfolgen_differentiation_b} wichtig ist.
\end{underlinedenvironment}

\begin{proof}
	Für $\epsilon > 0$ $\exists k_0\in \mathbb{N}$ mit 
	\begin{align}
		\proplbl{funktionsfolgen_differentiation_beweis_1}
		\vert f_k(x_0) - f_l(x_0)\vert < \epsilon \quad \forall k,l \ge k_0 \text{ und }\\
		\Vert g(y) - f^{y}_k\Vert < \epsilon, \Vert f^{'}_k(y) - f^{'}_k(y)\Vert < \epsilon \forall k,l \ge k_0, y \in B_r(x)
	\end{align}
	Weiter gilt (eventuell für größeres $k_0$) $\Vert g(z) - f_k'(z)\Vert < \epsilon$ und  \begin{align}
		\proplbl{funktionsfolgen_differentiation_beweis_2}
		\Vert f_k'(y) - f_l'(y)\Vert < \epsilon \quad\forall k,l\ge k_0,\;z,y\in B_r(x)
	\end{align}
	Schrankensatz: $\forall z,y\in B_r(x)$, $k,l\ge k_0$ $\exists \xi\in [y,z]$ mit {\zeroAmsmathAlignVSpaces**\begin{align}
		\proplbl{funktionsfolgen_differentiation_beweis_3}
		\left\Vert \left(f_k(y) - f_l(y)\right) - \left( f_k(z) - f_l(z)\right)\right\Vert \le \Vert f_k'(\xi) - f_l'(\xi)\Vert \cdot \vert y - z\vert \le \epsilon \vert y -z\vert < 2r\cdot\epsilon
	\end{align}}
	{\zeroAmsmathAlignVSpaces*\begin{flalign}
		\notag\Rightarrow\;\;\vert f_k(y) - f_l(y)\vert &\le \vert \big(f_k(y) - f_l(y)\big) - \big(f_k(x_0) - f_l(x_0)\big)\vert + \vert f_k(x_0) - f_l(x_0)\vert& \\
		\proplbl{funktionsfolgen_differentiation_beweis_4}
		&\le 2r \epsilon + \epsilon = \epsilon (2r + 1)\quad y\in B_r(x),\; k,l\ge k_0&
	\end{flalign}}
	$\Rightarrow$ $\{ f_k(y)\}_{k\in\mathbb{N}}$ ist \gls{cf} in $K^m$ $\forall y$ \\
	$\Rightarrow$ $f_k(y) \xrightarrow{k\to\infty}: f(y)$ $\forall y\in B_r(x)$
	
	Mit $l\to\infty$ in \eqref{funktionsfolgen_differentiation_beweis_4}: $f_k\to f$ gleichmäßig auf $B_r(x)$
	
	Fixiere $\tilde{x}\in B_r(x)$, $k=k_0$. Dann liefert $l\to\infty$ in \eqref{funktionsfolgen_differentiation_beweis_3} \begin{align*}
		\vert f(y) - f(\tilde{x}) - \big( f_k(y) - f_k(\tilde{x} \big) \vert \le \epsilon \vert y - \tilde{x}\vert \quad\forall y\in B_r(x)
	\end{align*}
	Da $f_k$ \gls{diffbar} $\exists \rho = \rho (\epsilon) > 0$ mit {\zeroAmsmathAlignVSpaces**\begin{align*}
		\vert f_k(y) - f_k(\tilde{x}) - f_k'(\tilde{x})\cdot(y - \tilde{x})\vert \le \epsilon \vert y - \tilde{x}\vert\quad\forall y\in B_\rho(\tilde{x})\subset B_r(x)
	\end{align*}}
	{\zeroAmsmathAlignVSpaces*\begin{flalign}
		\notag \Rightarrow\;\; \vert f(y) - f(\tilde{x}) - g(\tilde{x}) \cdot(y-\tilde{x})\vert &\le 
		\begin{multlined}[t][\dimexpr\linewidth/2]\vert f(y) - f(\tilde{x})\vert  + \vert f_k(y) - f_k(\tilde{x}))\vert \\
		+ \vert f_k(y) - f_k(\tilde{x}) - f_k'(\tilde{x}) \cdot (y - \tilde{x})\vert \\
		  + \vert f_k'(\tilde{x})\cdot (y - \tilde{x}) - g(\tilde{x})(y - \tilde{x})\vert\end{multlined}& \\
		\proplbl{funktionsreihe_differentiation_beweis_5}
		&\le \epsilon\vert y - \tilde{x}\vert + \epsilon \vert y - \tilde{x}\vert + \epsilon \vert y - \tilde{x}\vert = 3\epsilon \vert y - \tilde{x}\vert \quad\forall y\in B_\rho(\tilde{x})&
	\end{flalign}}
	
	Beachte: $\forall \epsilon > 0$ $\exists \rho > 0$ und mit \eqref{funktionsreihe_differentiation_beweis_5}
	
	\begin{tabularx}{\linewidth}{r@{\ \ }X}
		$\Rightarrow$ & $f(y) - f(\tilde{x}) - g(\tilde{x})\cdot(y - \tilde{x}) = o(\vert y -\tilde{x}\vert)$, $y\to \tilde{x}$ \\
		$\Rightarrow$ & $f(\tilde{x}) = g(\tilde{x})$ $\xRightarrow{\text{$\tilde{x}$ beliebig}}$ Behauptung
	\end{tabularx}
\end{proof}

\subsection{Anwendung auf Potenzreihen}
	Sei $f:B_R(x_0)\subset K\to K$ gegeben durch eine Potenzreihe 
	\begin{align}
		\proplbl{funktionsfolgen_potenzreihe}
		f(x) &= \sum_{k=0}^\infty a_k(x  - x_0)^k\quad\forall x\in B_{\underbrace{\text{\scriptsize$R$}}_{\mathclap{\text{Konvergenzradius}}}}(x_0)
	\end{align}

\begin{boldenvironment}[Wiederholung]
	$R=\frac{1}{\limsup_{k\to \infty} \sqrt[k]{\vert a_k\vert}}$
\end{boldenvironment}

\begin{boldenvironment}[Frage]
	Ist $f$ \gls{diffbar} und kann man gliedweise differenzieren?
\end{boldenvironment}

\begin{proposition}
	\proplbl{funktionsfolgen_satz_3}
	Sei $f:B_r(x_0)\subset K\to K$ Potenzreihe gemäß \eqref{funktionsfolgen_potenzreihe} \\
	\hspace*{1.5ex}$\Rightarrow$ $f$ ist \gls{diffbar} auf $B_r(x_0)$ mit 
	\begin{align}
		\proplbl{funktionsfolgen_satz_3_eq}
		f'(x) &= \sum_{k=1}^\infty k a_k (x - x_0)^{k-1}\quad\forall x\in B_r(x_0)
	\end{align}
\end{proposition}

\begin{conclusion}
	Sei $f:B_R(x_0)\subset K\to K$ Potenzreihe gemäß \eqref{funktionsfolgen_potenzreihe} \\
	\hspace*{1.5ex}$\Rightarrow$ $f\in C^\infty (B_R(x_0), K)$ und 
	\begin{align}
		\proplbl{funktionsfolgen_eq_8}
		a_k = \frac{1}{k!}\cdot f^{(k)})(x_0)
	\end{align}
	(d.h die Potenzreihe stimmt mit der Taylorreihe von $f$ in $x_0$ überein)
\end{conclusion}

\begin{proof}
	$k$-fache Anwendung von \cref{funktionsfolgen_satz_3} liefert $f\in C^k(B_r(x_0), K)$ $\forall k\in \mathbb{N}$\\
	$\xRightarrow{\eqref{funktionsfolgen_satz_3_eq}}$ $f'(x) = a_1$, $f''(x_0) = 2a_k$, $\dotsc$ rekursiv folgt \eqref{funktionsfolgen_eq_8}.
\end{proof}

\begin{proof}[\propref{funktionsfolgen_satz_3}]
	Betrache die Partialsummen
	\begin{align*}
		f_k(x) := \sum_{j=0}^k a_j(x- x_0)^j\quad\forall x\in B_R(x_0)
	\end{align*}
	$\Rightarrow$ $f_k(x_0)\xrightarrow{k\to\infty} f(x_0)$ und $f_k$ \gls{diffbar} mit 
	\begin{align*}
		f_k'(x) = \sum_{j=1}^k j a_j(x - x_0)^{j-1}\quad\forall x\in B_R(x_0)
	\end{align*}
	Wegen 
	\begin{align*}
		\limsup\limits_{k\to\infty} \sqrt[k]{(k+1)\vert a_{k+1}\vert} = \limsup \sqrt[k]{k\left(1 + \frac{1}{k}\right)} \cdot \left( \sqrt[k+1]{\vert a_{k+1}\vert}\right)^{\frac{k+1}{k}} = \limsup \sqrt[k]{\vert a_k\vert} = \frac{1}{R}
	\end{align*}
	hat die Potenzreihe 
	\begin{align*}
		g(x) := \sum_{k=1}^\infty k a_k(x-x_0)^{k-1}
	\end{align*}
	den Konvergenzradius $R$ \\
	\begin{tabularx}{\linewidth}{r@{\ \ }X}
	\ $\Rightarrow$ & Reihe $g$ konvergiert gleichmäßig auf $B_r(x_0)$ $\forall r\in (0,R)$ (vgl. 13.1), d.h. $f_k'\to g$ gleichmäßig auf $B_r(x_0)$ \\
	$\xRightarrow{\text{\cref{funktionsfolgen_differentiation}}}$&  $f$ ist \gls{diffbar} auf $B_r(x_0)$ mit \eqref{funktionsfolgen_satz_3_eq} auf $B_r(x_0)$.
	\end{tabularx}
	
	Da $r\in(0,R)$ beliebig, folgt die Behauptung.
\end{proof}

\begin{example}
	Es gilt \begin{align}
		\proplbl{funktionsfolgen_eq_9}
		\ln(1+x) = f(x) = \sum_{k=0}^\infty \frac{(-1)^k}{k+1}x^{k+1}\quad\forall x\in (-1,1)\subset \mathbb{R}
	\end{align}
\end{example}
	
\begin{proof}
	$f(x)$ sei Potenzreihe in \eqref{funktionsfolgen_eq_9}, hat Konvergenzradius $R=1$, $x_0=0$ \\
	$\xRightarrow{\text{\cref{funktionsfolgen_satz_3}}}$ $f$ \gls{diffbar} auf $(-1,1)$ und 
	\begin{align*}
		f'(x) = \sum_{k=0}^\infty (-x)^k = \frac{1}{1-(-x)} = \frac{1}{1+x}\qquad\text{geometrische Reihe}
	\end{align*}
	und
	\begin{align*}
		\frac{\D}{\D x}\ln (1+x) &= \frac{1}{1+x} = f'(x) \\
		f(x) &= \ln(1+x) + \mathrm{const}
	\end{align*}
	Wegen $f(0) = 0=\ln 1$ $\Rightarrow$ $f(x) = \ln(1+x)$ $\forall x\in(-1,1)$, d.h. \eqref{funktionsfolgen_eq_9} gilt.
\end{proof}


\phantomsection\addcontentsline{toc}{part}{Anhang}
\part*{Anhang}
\appendix
\patchcmd{\chapter}{\thispagestyle{plainChapter}}{\thispagestyle{fancy}}{}{}
%\titleformat{command}[shape]{format}{label}{sep}{before-code}[after-code]
%\titlespacing{command}{left}{before-sep}{after-sep}
\renewcommand{\chaptername}{Anhang}
\renewcommand{\thechapter}{\Alph{chapter}}
\titleformat{\chapter}[hang]{\bfseries}{\LARGE\chaptername\ \thechapter:}{0.5em}{\LARGE\bfseries}
\titlespacing{\chapter}{0pt}{-0.75cm}{0pt}
\renewcommand{\thesection}{\Alph{chapter}.\arabic{section}}

%from ntheorem.sty
%\def\thm@@thmline@name#1#2#3#4#5{%
%	\ifx\\#5\\%
%		\@dottedtocline{-2}{0em}{2.3em}%
%		{#1 \protect\numberline{#2}#3}%
%		{#4}
%	\else
%		\ifHy@linktocpage\relax\relax
%			\@dottedtocline{-2}{0em}{2.3em}%
%			{#1 \protect\numberline{#2}#3}%
%			{\hyper@linkstart{link}{#5}{#4}\hyper@linkend}%
%		\else
%			\@dottedtocline{-2}{0em}{2.3em}%
%			{\hyper@linkstart{link}{#5}%
%			{#1 \protect\numberline{#2}#3}\hyper@linkend}%
%			{#4}%
%		\fi
%	\fi
%}

\makeatletter
%update ntheorem macro to provide space between theorem numbers and any optional comment
\renewcommand{\thm@@thmline@name}[5]{%
	\def\thm@@thmline@name@tmp{%
		\if\relax\detokenize{#3}\relax%
			{:}%
		\else%
			{:\hspace*{1em}#3}%
		\fi%
	}%
	\ifx\\#5\\%
		\@dottedtocline{-2}{0em}{2.3em}%
		{#1 \protect\numberline{#2}{\thm@@thmline@name@tmp}}%
		{#4}
	\else
		\ifHy@linktocpage\relax\relax
			\@dottedtocline{-2}{0em}{2.3em}%
			{#1 \protect\numberline{#2}{\thm@@thmline@name@tmp}}%
			{\hyper@linkstart{link}{#5}{#4}\hyper@linkend}%
		\else
			\@dottedtocline{-2}{0em}{2.3em}%
			{%
			{#1 \protect\numberline{#2}{\thm@@thmline@name@tmp}}}%
			{\hyper@linkstart{link}{#5}{#4}\hyper@linkend}%
		\fi
	\fi
}
\makeatother

\chapter{Listen}
\section{Liste der Theoreme}
\theoremlisttype{allname}
\listtheorems{theorem}

\pagebreak
\section{Liste der benannten Sätze}
\theoremlisttype{optname}
\listtheorems{proposition}

\printglossary[type=\acronymtype]
\addcontentsline{toc}{chapter}{Akronyme}

\printindex
\printindex[symbols]

\end{document}