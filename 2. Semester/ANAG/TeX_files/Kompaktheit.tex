\section{Kompaktheit}
\begin{definition}
Sei $(X,d)$ metrischer Raum, Mengensystem $\mathcal{U}\subset \{ U\subset X | U \text{ offen }\}$ heißt \begriff{offene Überdeckung} von $M\subset X$, falls $M\subset \bigcup_{U\in\mathcal{U}} U$.

Überdeckung $\mathcal{U}$ heißt endlich, falls $\mathcal{U}$ endlich (d.h. $\mathcal{U} = \{U_1,\dotsc,U_n\}$).

Menge $M\subset X$ heißt \highlight{(überdeckungs-)}\begriff[Menge!]{kompakt}, falls jede Überdeckung $\mathcal{U}$ eine endliche Überdeckung $\tilde{\mathcal{U}}\subset \mathcal{U}$ endhält (d.h. $\exists U_1,\dotsc, U_n\subset\mathcal{U}$ mit $M\subset\bigcup_{i=1}^n U_n$).

Menge $M\subset X$ heißt \begriff{folgenkompakt}, falls jede Folge $\{x_n\}$ aus $M$ (d.h. $x_n\in M\,\forall M$) eine konvergente Teilfolge $\{x_{n'}\}$ mit Grenzwert in $M$ bessitzt (d.h. $\{x_n\}$ hat \gls{hw} in $M$ nach \ref{tfprinzip}).
\end{definition}

\begin{theorem}
	Sei $(X,d)$ metrischer Raum, $M\subset X$. Dann:\[M\text{ kompakt} \;\Leftrightarrow\; M\text{ folgenkompakt}\]
\end{theorem}

\begin{proposition}
	Sei $(X,d)$ metrischer Raum, $M\subset X$. Dann
	\begin{enumerate}[label={\arabic*)}]
		\item $M$ folgenkompakt $\Rightarrow$ $M$ beschränkt und abgeschlossen
		\item $M$ folgenkompakt, $A\subset M$ abgeschlossen $\Rightarrow$ $A$ folgenkompakt.
	\end{enumerate}
\end{proposition}
\begin{theorem}[\person{Heine}-\person{Borell} kompakt, \person{Bolzano}-\person{Weierstraß} folgenkompakt]
	Sei $X=\mathbb{R}^n$ (bzw. $\mathbb{C}^n$) mit beliebiger Norm, $M\subset X$. Dann \[ M \text{ kompakt} \;\Leftrightarrow\; M \text{ abgeschlossen und beschränkt} \]
\end{theorem}
\begin{conclusion}
	Sei $\{x_n\}$ Folge in $X=\mathbb{R}^n$ (bzw. $\mathbb{C}^n$). Dann \[ \{x_n\}\text{ beschränkt} \;\Rightarrow \; \{x_n\} \text{ hat konvergente \gls{tf}}\]
\end{conclusion}
\begin{proposition}
	\proplbl{aeqv_norm}
	Je 2 Normen aus $\mathbb{R}^n$ bzw. $\mathbb{C}^n$ sind äquivalent.
\end{proposition}