\section{Wiederholung und Motivation}
Sei $K^n$ $n$-dim. \gls{vr} über Körper mit $K=\mathbb{R}$ oder $K=\mathbb{C}, n\in\mathbb{N}_{\ge 0}$.
\begin{itemize}
	\item Elemente sind alle $x=(x_1, \dotsc, x_n)\in K^n$ mit $x_1, \dotsc, x_n\in K$.
	\item \begriff{Standardbasis} ist $\{e_1, \dotsc, e_n\}$ mit $e_j=(0,\dotsc,0,\underbrace{1}_{\text{$j$-te Stelle}},0,\dotsc,0)$
	\item alle Normen auf $K^n$ sind äquivalent (\propref{aeqv_norm}) \\
	$\Rightarrow$ Kovergenz unabhängig von der Norm
	
	Verwende in der Regel euklidische Norm $\Vert x \Vert_2 = \vert x \vert = \sqrt{\sum\limits_{i}\vert x_i \vert^2}$
	\item \begriff{Skalarprodukt}
	\begin{itemize}
		\item $\langle x,y \rangle = \sum\limits_{j=1}^{n} x_j\cdot y_j$ in $\mathbb{R}^n$
		\item $\langle x,y \rangle = \sum\limits_{j=1}^{n} \overline{x}_j\cdot y_j$ in $\mathbb{C}^n$
	\end{itemize}
	\item \textsc{Cauchy}-\textsc{Schwarz}-Ungleichung ($\vert \langle x,y\rangle \vert \le \vert x \vert \cdot \vert y \vert\,\quad\forall x,y\in K^n$)
\end{itemize}

\subsection{Lineare Abbildungen}
\proplbl{definition_tensorprodukt}
Eine \begriff{lineare Abbildung} ist homogen und additiv (siehe \propref{defLinearFunction}).
\begin{itemize}
	\item Lineare Abbildung $A: K^n \rightarrow K^m$ ist darstellbar durch $m\times n$-Matrizen bezüglich der Standardbasis 
	(\emph{beachte:} $A$ sowohl Abbildung als auch Matrix)
	\begin{itemize}
		\item lineare Abbildung ist stetig auf endlich-dimensionalen Räumen (unabhängig von der Norm, siehe \propref{chap_15_5})
		\item transponierte Matrix: $A^T\in K^{n\times m}$
		
		\begin{hint}
		$x=(x_1,\dotsc, x_n)\in K^n$ idR platzsparender als Zeilenvektor geschrieben, \emph{aber} bei Matrix-Multiplikation $x$ Spalten-Vektor, $x^T$ Zeilenvektor, d.h.	\begin{align*}
		 x^T \cdot y &= \langle x,y\rangle, &&\text{falls $m=n$} \\
		 x \cdot y^T &= x \otimes y\in K^{m\times n}, && \text{sog. \begriff{Tensorprodukt}}
		 \end{align*}
		\end{hint}
	\end{itemize}	
	 \item \mathsymbol{L}{$L(K^n, K^m)$}$ = \{ A: K^n \to K^m, \text{ $A$ linear}\}$ (Menge der linearen Abbildung, ist normierter Raum)
	\begin{itemize}
		 \item \mathsymbol{|A|}{$\Vert A \Vert$}$= \sup\{ \vert Ax\vert \mid \vert x \vert \le 1 \}$ (\begriff{Operatornorm}, $\Vert A \Vert$ hängt i.A. von Normen auf $K^n, K^m$ ab)
		 \item $L(K^n, K^m)$ ist isomorph zu $K^{m\times n}$ als \gls{vr} \\
		 $\Rightarrow$ $L(K^n, K^m)$ ist $m\cdot n$-dim. \gls{vr} ($\Rightarrow$ alle Normen äquivalent, $\Rightarrow$ Konvergenz von $\{A_n\}$ von linearer Abbildungen in $L(K^n, K^m)$ ist normunabhängig)
		 
		 Nehmen in der Regel statt $\Vert A \Vert$ euklidische Norm $\vert A \vert = \sqrt{\sum\limits_{k,l} \vert a_{kl} \vert ^2}$.\\
		 Es gilt: \[ \vert Ax \vert \le \Vert A \Vert\cdot \vert x \vert \text{ und } \vert Ax\vert \le \vert A \vert \cdot\vert x \vert \]
	\end{itemize}
	\item Abbildung $\tilde{f}: K^n \to K^m$ heißt \begriff[linear!]{affin} \highlight{linear}, falls $\tilde{f}(x) = Ax + a$ für lineare Abbildung $A:K^n\to K^m, a\in K^m$
\end{itemize}

\subsection{\textsc{Landau}-Symbole}

\begin{*anmerkung}
	Eine Approximation besitzt zwangsläufig immer einen Fehler. Eine gute Approximation zeichnet sich dadurch aus, dass der Fehler bzw. Rest möglichst klein wird. Dieser Fehler wird mit \person{Landau}-Symbolen beschrieben. Dabei bedeutet anschaulich:
	\begin{itemize}
		\item $f=o(g)$: $f$ wächst langsamer als $g$
		\item $f=\mathcal{O}(g)$: $f$ wächst nicht wesentlich schneller als $g$
	\end{itemize}
\end{*anmerkung}

Sei $f:D\subset K^n \to K^m$, $g:D\subset K^n \to K$, $x_0 \in \overline{D}$. Dann:
\begin{itemize}
	\item $f(x) = o(g(x))$ für $x\to x_o$ \gls{gdw} $\lim\limits_{\substack{x\to x_0 \\ x\neq x_0}} \frac{\vert f(x) \vert}{g(x)} = 0$
	\item $f(x) = \mathcal{O}(g(x))$ für $x\to x_0$ \gls{gdw} $\exists \delta > 0, c \ge 0: \frac{\vert f(x) \vert}{\vert g(x) \vert} \le c \;\forall x\in \left( B_\delta(x_0)\setminus \{ x_0\}\right) \cap D$
	
	\emph{wichtiger Spezialfall:} $g(x) = \vert x - x_0\vert ^k, k\in\mathbb{N}$
\end{itemize}

\begin{example}[gute Approximation durch konstante Funktion nahe $x=x_0$]
	Sei $f:D\subset K^n\to K^m$, $x_0\in D$ \gls{hp} von $D$. Dann:
	\begin{align}
		\notag f\text{ stetig in } x_0 &\Leftrightarrow \lim\limits_{\substack{x\to x_0 \\ x\neq x_0}} f(x) = f(x_0) \\
		\notag &\Leftrightarrow \lim\limits_{\substack{x\to x_0 \\ x\neq x_0}} \frac{f(x) - f(x_0)}{1} = 0 \\
		&\Leftrightarrow \boxed{f(x) = f(x_0) + o(1)} \text{ für }x\to x_0\proplbl{chap15specialCase}
	\end{align}
	
	\begin{center}\begin{tikzpicture}
		\begin{axis}[
		xmin=0, xmax=5, xlabel=$x$,
		ymin=0, ymax=5, ylabel=$y$,
		samples=400,
		axis y line=middle,
		axis x line=middle,
		]
		\addplot+[mark=none] {x^2};
		\addlegendentry{$f$}
		\addplot+[mark=none] {sqrt(2)};
		\addlegendentry{Apprx}
		\addplot+[mark=none, dashed] {x^2-sqrt(2)};
		\addlegendentry{$r$}
		\end{axis}
		\end{tikzpicture}\end{center}
	
	\begin{boldenvironment}[Interpretation von \eqref{chap15specialCase}]
	
	Setze $r(x) := f(x) - f(x_0)$
	\zeroAmsmathAlignVSpaces
	\begin{flalign}
		&\notag \overset{\text{(\ref{chap15specialCase})}}{\Rightarrow} r(x) = o(1) \text{ für } x\to x_0& \\
		&\label{chap15interpretationSpecialCase} \Rightarrow r(x) \overset{x\to x_0}{\longrightarrow} 0,&
	\end{flalign}
	d.h. $o(1)$ ersetzt eine "`Rest-Funktion"' $r(x)$ mit Eigenschaft (\ref{chap15interpretationSpecialCase}).
	\end{boldenvironment}

	\begin{*anmerkung}
		Man kann als Approximation auch $x=3$ wählen, allerdings stimmt dann die Aussage $r\to 0$ für $x\to x_0$ nicht mehr.
	\end{*anmerkung}

	Wegen $o(1) = o(\vert x-x_0\vert^0)$ (d.h. $k=0$) sagt man auch, \propref{chap15specialCase} ist die Approximation 0. Ordnung der Funktion $f$ in der Nähe von $x_0$.
\end{example}

\begin{example}[gute Approximation durch (affin) lineare Funktion nahe $x=x_0$]
	Sei $f:D\subset \mathbb{R}^n\to \mathbb{R}$, $x_0\in D$, $D$ offen. Was bedeutet \begin{align}
		\proplbl{chap15meaningSpecialCase} f(x) = \underbrace{f(x_0)+A(x-x_0)}_{\tilde{f}\text{ affin lineare Funktion}} + o(\vert x-x_0\vert),\;x\to x_0?		
	\end{align}
	
		\begin{boldenvironment}[Zentrale Frage]
			Wie sollte ein guter Rest sein?
		\end{boldenvironment}
		$\graph\tilde{f}$ ist die $n$-dimensionale Ebene in $K^{n+m}$ (affin-lin. UR) \\
		$\graph f$ sollte sich an diese Ebene anschmiegen ($\graph\tilde{f}=$Tangentialebene) \\
		$\Rightarrow$ Rest sollte sich an den Grafen der Nullfunktion anschmiegen \\
		\\
		Sei
		\begin{align}
			\proplbl{eqvanschmiegen}g(t)=\sup\limits_{\vert x-x_0\vert \le t}\vert r(x)\vert\Rightarrow \vert r(x)\vert\le g(\vert x-x_0\vert)\quad\forall x
		\end{align}
		anschmiegen: $g(t)=o(1),t\to 0$ nicht ausreichend \\
		angenommen $g(t)=o(t),t\to 0$: dann ist für ein festes $v\in K^n$ mit $\Vert v\Vert=1$
		\begin{align}
			\vert r(x_0+tv)\vert\le g(t) &\Rightarrow\frac{\vert r(x_0+tv)-r(x_0)\vert}{t}\le \frac{g(t)}{t}\to 0 \notag\\
			&\Rightarrow\text{anschmiegen}\notag
		\end{align}
		Wegen \propref{eqvanschmiegen} folgt: $\frac{\vert r(x)\vert}{\vert x-x_0\vert}\le \frac{g(\vert x-x_0\vert)}{\vert x-x_=\vert}\to 0$ \\
		$\Rightarrow r(x)=o(\vert x-x_0\vert)$ für $x\to x_0=o(1)\vert x-x_0\vert$ \\
		$\Rightarrow$ betrachte $\tilde f$ als gute lineare Approximation von $f$ nahe $x=x_0$ falls Fehler $=f(x)-(f(x_0)-A(x-x_0))=o(\vert x-x_0\vert)$ für $x\to x_0$ \\
		man sagt: Fehler wird schneller kleiner als $\vert x-x_0\vert$! $\tilde f$ heißt Approximation 1. Ordnung von $f$ in $x_0$
\end{example}

\begin{*definition}[Anschmiegen]
	$f(x) + \underbrace{f(x_0) + A(x-x_0)}_{\tilde{A}(x)} = o(\vert x-x_0\vert)$, \\
	d.h. die Abweichung wird schneller klein als $\vert x-x_0\vert$!
\end{*definition}

\smiley{} Vielleicht hatten Sie eine andere Vorstellung von "'anschmiegen"', aber wir machen hier 
Mathematik \smiley{}

\begin{proposition}[Rechenregeln für \person{Landau}-Symbole]
	Für $r_k,\tilde{r_l},R_l:D\subset K^n\to K^m,x_0\in D,k,l\in\natur$ mit \\
	$r_k(x)=o(\vert x-x_0\vert^k),\tilde{r_k}=o(\vert x-x_0\vert^l),R_l(x)=O(\vert x-x_0\vert ^l),x\to x_0$
	\begin{enumerate}
		\item $r_k(x)=o(\vert x-x_0\vert^j)=\mathcal{O}(\vert x-x_0\vert^j)\quad j\le k$ \\
		$R_l(x)=o(\vert x-x_0\vert^j)=\mathcal{O}(\vert x-x_0\vert^j)\quad j<l$
		\item $\frac{r_k(x)}{\vert x-x_0\vert^j}=o(\vert x-x_0\vert^{k-j})\quad j\le k$ \\
		$\frac{R_l(x)}{\vert x-x_0\vert^j}=\mathcal{O}(\vert x-x_0\vert^{l-j})=o(\vert x-x_0\vert^{l-j-1})\quad j\le l$
		\item $r_k(x)\pm \tilde{r_l}(x)=o(\vert x-x_0\vert ^k)\quad k\le l$
		\item $r_k(x)\cdot \tilde{r_l}(x)=o(\vert x-x_0\vert^{k+l}),r_k(x)\cdot R_l(x)=o(\vert x-x_0\vert^{k+l})$
	\end{enumerate}
\end{proposition}
\begin{proof}
	Sei $\frac{\vert R_l(x)\vert}{\vert x-x_0\vert^l}\le c$ nahe $x_0$, d.h. auf $(B_{\delta}(x_0)\backslash\{x_0\})\cap D$ für ein $\delta>0$
	\begin{enumerate}
		\item $\frac{r_k(x)}{\vert x-x_0\vert^j}=\frac{r_k(x)}{\vert x-x_0\vert^k}\vert x-x_0\vert^{k-j}\to 0$, folgl. $\frac{r_k(x)}{\vert x-x_0\vert^{\delta}}$ auch beschränkt nahe $x_0$ \\
		$\frac{R_l(x)}{\vert  x-x_0\vert^j}=\frac{R_l(x)}{\vert x-x_0\vert^l}\vert x-x_0\vert^{l-j}\to 0$, Rest wie oben
		\item $\frac{r_k(x)}{\vert x-x_0\vert^j \vert x-x_0\vert^{k-j}}=\frac{r_k(x)}{\vert x-x_0\vert^k}\to 0$ \\
		$\frac{R_l(x)}{\vert x-x_0\vert^j \vert x-x_0\vert^{l-j}}=\frac{R_l(x)}{\vert x-x_0\vert^l}\le c$ nahe $x_0$, Rest wie oben
		\item $\frac{r_k(x)}{\vert x-x_0\vert^k}\pm\frac{\tilde{r_l}(x)}{\vert x-x_0\vert^k}\overset{(2)}{=}o(1)\pm\underbrace{o(\vert x-x_0\vert^{l-k})}_{o(1)}\to 0$
		\item $\frac{r_k(x)\cdot \tilde{r_l}(x)}{\vert x-x_0\vert^{k+l}}=\frac{r_k(x)}{\vert x-x_0\vert^k}\cdot\frac{\tilde{r_l}(x)}{\vert x-x_0\vert^l}\to 0$ \\
		$\frac{\vert r_k(x)\cdot R_l(x)\vert}{\vert x-x_0\vert^{k+l}}=\frac{\vert r_k(x)\vert}{\vert x-x_0\vert^k}\cdot\frac{\vert R_l(x)\vert}{\vert x-x_0\vert^l}\to 0$
	\end{enumerate}
\end{proof}