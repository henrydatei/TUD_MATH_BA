\setcounter{dummy}{16}
\addtocounter{section}{15}
\addtocounter{chapter}{4}

\chapter{Differentiation}
	Differentiation ist lokale Linearisierung.

\section{Wiederholung und Motivation}
	\begin{ueberblick}
	$K^n$ ist ein $n$-dimensionaler VR über dem vollständigen Körper $K=\real$ 
	oder $K=\comp$. Die Elemente sind $x=(x_1,...,x_n)\in K^n$ mit 
	$x_1,...,x_n\in K$. Basis ist die Standardbasis $(e_1,...,e_n)$. \\
	
	
	Alle Normen sind auf $K^n$ äquivalent $\Rightarrow$ Konvergenz ist 
	unabhängig von der Norm. Trotzdem verwenden wir die euklidische Norm: 
	$|x|=\sqrt{\sum\limits_{j=1}^n |x_j|^2}$. \\
	
	
	\underline{Skalarprodukt:} $\langle x,y \rangle=\sum\limits_{j=1}^n x_j\cdot y_j$ in 
	$\real$ bzw. $\langle x,y \rangle=\sum\limits_{j=1}^n \overline{x_j}
	\cdot y_j$ in $\comp$. \\
	
	
	\underline{\person{Cauchy}-\person{Schwarz}-Ungleichung:} $|\langle x,y \rangle| \le
	|x|\cdot |y|$. \\

	
	lineare Abbildung: $A:K^n \to K^m$, Darstellung mittels $m\times n$-Matrix 
	bezüglich Standardbasen in $K^n$ und $K^m$. Beachte: $A$ steht für die 
	lineare Abbildung und die Matrix, die die lineare Abbildung beschreibt. 
	Lineare Abbildungen sind stets stetig (unabhängig von der Norm). Hinweis: 
	$x=(x_1,...,x_n)$ in der Regel als Zeilenvektor geschrieben, aber bei 
	Matrixmultiplikation ist $x$ Spaltenvektor und $x^t$ Zeilenvektor, d.h. \\
	$x^t\cdot y=\langle x,y \rangle$, falls $m=n$ \\
	$x\cdot y^t=x\otimes y$, sogenanntes Tensor-Produkt \\
	
	
	$L(K^n,K^m)=\{A: K^n \to K^m \mid A \text{ linear}\}$ Menge aller 
	linearen Abbildungen mit $||A||=\sup\{|Ax| \mid |x|\le 1\} \to$ Norm 
	hängt im Allgemeinen von Normen auf $K^n$ und $K^m$ ab. \\
	$L(K^n,K^m)$ ist isomorph zu $Mat_{m\times n}(K)$ ist isomorph zu $K^{mn}$ 
	jeweils als VR $\Rightarrow$ $L(K^n,K^m)$ ist ein $m\cot n$-dimensionaler 
	VR $\Rightarrow$ alle Normen sind äquivalent $\Rightarrow$ Konvergenz von 
	$\{A_n\}$ in $L(K^n,K^m)$ unabhängig von Norm, nehme in der Regel statt 
	$||A||$ die euklidische Norm $|A|=\sqrt{\sum\limits_{k=1}^n \sum\limits_
	{l=1}^n |a_{kl}|^2} \Rightarrow$ es gilt: $|Ax|\le ||A||\cdot |x|$ und 
	$|Ax|\le |A|\cdot |x|$. \\
	
	
	Abbildung $f:K^n\to K^m$ heißt affin linear falls $f(x)=Ax+a$ für eine 
	lineare Abbildung $A:K^n\to K^m$. \\
	
	\underline{\person{Landau}-Symbole}: Sei $f:D\subset K^n \to K^m$, $g:D\subset K^n\to K$, 
	$x_0\in \overline{D}$
	\begin{compactitem}
		\item $f(x)=o(g(x))$ für $x\to x_0$ genau dann, wenn $\lim\limits_{\substack{x\to x_0 \\ x\neq x_0}} \frac{|f(x)|}{g(x)}=0$
		\item $f(x)=O(g(x))$ für $x\to x_0$ genau dann, wenn $\exists\delta>0$, $0\le c<\infty$ mit 
		$\frac{|f(x)|}{|g(x)|}\le c \quad \forall x\in (B_{\delta}(x_0)\backslash \{x_0\})\cap D$
	\end{compactitem}
	\end{ueberblick}

	%TODO was soll der wichtige Spezialfall im Skript? Ist der wichtig?

	\begin{beispiel}[$f: D\subset K^n \to K^m$]
		$x_0\in D$, $x_0$ Häufungspunkt von $D$. Dann: $f$ stetig in $x_0 \iff \lim\limits_{x\to x_0} f(x)
		=f(x_0)\iff \lim\limits_{x\to x_0} \frac{|f(x)-f(x_0)|}{1}=0 \iff f(x)=f(x_0)+o(1)$ für $x\to x_0$ \\
		Interpretation: Setze $r(x)=f(x)-f(x_0)\Rightarrow r(x)=o(1)$ für $x\to x_0\Rightarrow r(x)\to 0$, 
		d.h. $o(1)$ ersetzt die Restfunktion $f(x)-f(x_0)$. Wegen $o(1)=o(|x-x_0|^0)$ sagt man auch, 
		dass $f(x)=f(x_0)+o(1)$ die Approximation 0-ter Ordnung der Funktion $f$ in der Nähe von $x_0$.
	\end{beispiel}

	\begin{beispiel}[$f:D\subset \real^n \to \real$]
		$x_0\in D$, $D$ offen, das bedeutet: $f(x)=f(x_0)+o(|x-x_0|)$, $x\to x_0 \quad (*)$
		\begin{compactitem}
			\item betrachte $f$ auf Strahl $x=x_0+ty$, $y\in \real^n$ fest, $|y|=1$, $t\in \real$ \\
			$(*)\Rightarrow 0=\lim\limits_{\substack{x\to x_0 \\ x\neq x_0}} \frac{|f(x)-f(x_0)|}{|x-x_0|}=
			\lim\limits_{\substack{t\to 0 \\ t\neq 0}} \frac{|f(x_0+ty)-f(x_0)|}{|t|} \Rightarrow \frac
			{|\Delta f|}{|t|}=|\text{Anstieg der Sekante}|\to 0$
			\item $(*)\Rightarrow f(x)=f(x_0)+\underbrace{\frac{o(|x-x_0|)}{|x-x_0|}}_{o(1)}|x-x_0|
			\Rightarrow f(x)=f(x_0)+o(1)|x-x_0| \Rightarrow f(x)=f(x_0)+r(x)|x-x_0|\Rightarrow 
			|f(x)-f(x_0)| \le \varrho(t)|x-x_0|$ falls $|x-x_0|\le t$ mit $\varrho(t)=\sup\limits_{|x-x_0|\le t} 
			|r(x)|\to 0$ \\
			Graph von $f$ liegt nahe $x_0$ in "'immer flacheren kegelförmigen Mengen"' $\Rightarrow$ 
			Graph "'schmiegt sich"' an horizontale Ebene durch Punkt $(x_0,f(x_0))$
			\item $(*)$ erfüllt offenbar nicht die Beobachtung: horizontale Ebene ist Graph einer affin 
			linearen Funktion $\tilde A: \real^n \to \real$
		\end{compactitem}
	\end{beispiel}

	\textbf{zentrale Frage:} Gibt es zur Funktion $f:D\subset K^n \to K^m$, $x_0\in D$, eine affin 
	lineare Funktion $\tilde A:K^n\to K^m$, so dass sich in der Nähe von $x_0$ der Graph von $f$ 
	an den Graph von $\tilde A$ "'anschmiegt"'? \\
	\textbf{Antwort:} Ja, wegen $f(x_0)=\tilde A|x_0|$ folgt $\tilde Ax=A(x-x_0)+f(x_0)$
	
	\begin{definition}[Anschmiegen]
		$f(x_0)-(f(x_0)+A(x-x_0))=o(|x-x_0|)$ \\
		d.h. die Abweichung wird schneller kleiner als $|x-x_0|$! \\
		
		Vielleicht hatten Sie bisher eine andere Vorstellung von "'anschmiegen"', aber wir machen hier 
		Mathematik!
	\end{definition}
