\addtocounter{section}{1} % section ``Fundament der Mathematik'' fehlt
\section{Natürliche Zahlen}
\begin{definition}[Peano Axiome]
$\mathbb{N}$ sei Menge, die die \begriff{\person{Peano}-Axiome} erfüllen, d.h.
\begin{enumerate}[label={P\arabic*)}]
	\item $\mathbb{N}$ sei indutkiv, d.h. es ex.
	\begin{itemize}
		\item Nullelement $0\in \mathbb{N}$ und
		\item injektive (Nachfolger-) Abb. \mathsymbol{nu}{$\nu$}$:\mathbb{N}\rightarrow\mathbb{N}$ mit $\nu(n)\neq 0\,\forall n\in \mathbb{N}$
	\end{itemize}
	\item (Induktionsaxiom)
	
	Falls $N\subset\mathbb{N}$ induktiv in $\mathbb{N}$ (d.h. $0,\nu(n)\in\mathbb{N}$ falls $n\in\mathbb{N}$)\\
	$\Rightarrow N=\mathbb{N}$ ($N$ ist die kleinste indutkive Menge)
\end{enumerate}

Nach Mengenlehre ZF existiert eine Solche Menge der \begriff{natürliche Zahlen} mit üblichen Symbolen.
\end{definition}

\begin{theorem}
	\proplbl{naturliche_zahlen_isomorph}
	Falls $\mathbb{N}$ und $\mathbb{N}*$ \person{Peano}-Axiome erfüllen, dann sind sie isomorph bezüglich Nachfolger-Abbildung und Nullelement (Anfangselement).
\end{theorem}

\begin{proposition}[Prinzip der vollständigen Induktion] \proplbl{prin_voll_induktion}
	\begriff*{vollständigen Induktion}
	Sei $\{A_n | n\in\mathbb{N}\}$ Aussagenmenge mit d. Eigenschaften
	\begin{itemize}
		\item[(IA)] $A_0$ ist wahr (\begriff{Induktionsanfang})
		\item[(IS)] $\forall n\in\mathbb{N}$ gilt: $A_n$ (wahr) $\Rightarrow A_{n+1}$
	\end{itemize}
	$\Rightarrow A_n$ ist wahr $\forall n\in\mathbb{N}$
\end{proposition}

\begin{proof}
	Sei $N := \{n \in \mathbb{N} \mid A_n \text{ ist wahr}\} \subset \mathbb{N}$, offenbar $0 \in \mathbb{N}$ und $\nu(n) \in \mathbb{N}$, falls $n \in \mathbb{N} \Rightarrow \mathbb{N}$ induktiv in $\mathbb{N} \overset{P2)}{\Rightarrow} N = \mathbb{N}$
\end{proof}

\begin{lemma}
	\proplbl{lemma_nachfolgerabb}
	Es gilt:
	\begin{enumerate}[label={\alph*)}]
		\item $\nu(\mathbb{N})\cup \{0\}=\mathbb{N}$
		\item $\nu(n)\neq n\,\forall n\in\mathbb{N}$
	\end{enumerate}
\end{lemma}

\begin{proof}
	\begin{itemize}
		\item[a)] $N := \{n \in \mathbb{N} \mid n = \nu(m) \text{ für } n\in \mathbb{N} \} \cup \{ 0 \}$ ist induktiv in $\mathbb{N} \overset{P2)}{\Rightarrow} N = \mathbb{N}$
		\item[b)] Beweis mittels vollständiger Induktion
		\begin{itemize}
			\item[(IA)] $\nu(0) \neq 0$ nach P1)
			\item[(IS)] Zeige: $(\nu(n) \overset{\text{Ind.vorraus.}}{\neq} n \Rightarrow \nu(\nu(n)) \neq \nu(n) \forall n \in \mathbb{N}$
		indirekter Beweis: \\
		Angenommen $\nu(\nu(n)) = \nu(n) \overset{\nu \text{ inj.}}{\Rightarrow} \nu(n) = n \overset{IV}{\Rightarrow} \lightning \Rightarrow$ (1) $\Rightarrow$ b) nach Prinzip der vollst. Induktion (vgl.\propref{prin_voll_induktion})
		\end{itemize}
	\end{itemize}
\end{proof}

\begin{proposition}[Rekursive Definition / Rekursion]
	\proplbl{rekursive_def_abb}
	\begriff*{Rekursion}
	Sei b$B$ Menge, $b\in B$ u. $F:B\times\mathbb{N}\rightarrow B$ Abbildung. Dann liefert die Vorschrift \begin{align}\label{rekur_definition}
		f(0) &:= b,\\f(n+1)&:=F(f(n),n)\quad\forall n\in \mathbb{N}
	\end{align}
	genau eine Abbildung für $f:\mathbb{N}\rightarrow B$ (d.h. solche Abbildung ist eindeutig)
\end{proposition}

\begin{proof}
	mittels vollständiger Induktion:
	\begin{itemize}
		\item[IA] $f(0) = b$ eindeutig definiert 
		\item[IS] angenommen $f(n)$ eindeutig definiert $\overset{\text{1)}}{\Rightarrow}$ 
		$f(n+1) \overset{\text{\propref{prin_voll_induktion}}}{\Rightarrow}$ Behauptung gilt nach Prinzip der vollständigen Induktion
	\end{itemize}
\end{proof}

\begin{proof}[\propref{naturliche_zahlen_isomorph}]
	$\mathbb{N}$ und $\mathbb{N}^{*}$ mögen \person{Peano}-Axiome erfüllen mit $(\nu, 0)$ bzw. $(\nu^{*}, 0^{*})$. Betrachte rekursive eindeutige definierte Abbildung: $I: \mathbb{N} \to \mathbb{N}^{*}$ (\propref{rekursive_def_abb} $B=\mathbb{N}^{*}$, $F(n^{*}, n) = \nu^{*}(n^{*})$)
	$I(0) = 0^{*}, I(\nu(n)) = \nu^{*}(I(n)) \forall n \in \mathbb{N}$ $I$ enthält Nullelement und Nachfolgerabbildung. Falls $I$ bijektiv, dann ist $I$ ein Isomorphismus und Behauptung folgt.\\
		Zeige $I$ surjektiv: offenbar $0^{*} \in I(\mathbb{N})$, falls $n^{*} \in I(\mathbb{N}) \Rightarrow \exists n \in \mathbb{N}:n^{*} = I(n) \Rightarrow \nu^{*}(n^{*}) = \nu^{*}(I(n)) = I(\nu(n)) \in I(\mathbb{N})$ (Bild). Folglich ist $I(\mathbb{N}) \subset \mathbb{N}^{*}$ induktiv in $\mathbb{N}^{*} \overset{\text{P2)}}{\Rightarrow} I(\mathbb{N}) = \mathbb{N}^{*}$.\\
		Zeige $I$ injektiv: $I(n) \neq I(m)\forall n\neq m$ (*) vollständige Induktion nach $m$ (jeweils $\forall n \neq m$)
		\begin{itemize}
			\item[IA)] $m=0: \forall n \neq 0 \exists n \in \mathbb{N} \colon n = \nu(n^{'})$ (vgl. \propref{lemma_nachfolgerabb}) $\Rightarrow I(n) = I(\nu(n^{'})) = \nu^{*}(I(n^{'})) \overset{\text{P1)}}{\neq} 0^{*} = I(0) \forall n \neq 0$ (ist gerade (*)))
			\item[IS)] IV: Sei $I(n) \neq I(m) \forall n \neq m$, dann 
			für $n = 0$, $n = \nu(m) \text{ mit } I(0) = 0^{*} \neq \nu^{*}(I(m)) = I(\nu(m))$.\\
			für $n \neq 0$, $n \overset{\propref{lemma_nachfolgerabb}}{=} \nu(n^{'}) \neq \nu(m) \overset{\nu \text{ inj.}}{\Rightarrow} n^{'} \overset{\text{IV)}}{\neq} m$ und $I(n) = I(\nu(n{'})) = \nu^{*}(I(n^{'})) \neq \nu^{*}(I(m)) = I(\nu(m))$\\
			$\Rightarrow$ in der Behauptung $I(n) \neq I(\nu(m)) \forall n \neq \nu(m) \Rightarrow$(*) mittels vollständiger Induktion, \\ d.h. $I$ ist injektiv
		\end{itemize}
\end{proof}

\subsection*{Rechenoperationen}
\begin{definition}[Rechenoperation auf $\boldsymbol{\mathbb{N}}$]
	Definiere \begriff{Addition}[!natürliche Zahlen] $+:\mathbb{N}\times\mathbb{N}\rightarrow \mathbb{N}$ auf $\mathbb{N}$ durch $n+0:=n, n+\nu(m) :=\nu(n+m)\,\forall n,m\in\mathbb{N}$
	
	Definiere \begriff{Multiplikation}[!natürliche Zahlen] $\cdot:\mathbb{N}\times\mathbb{N} \rightarrow\mathbb{N}$ auf $\mathbb{N}$ durch $n\cdot 0 = 0, n\cdot\nu(m) = n\cdot m+n\,\forall m,n\in\mathbb{N}$
\end{definition}

\begin{proposition}
	Addition und Multiplikation haben folgende Eigenschaften, d.h. $\forall k,m,n\in\mathbb{N}$ gilt:
	
	\begin{tabular}{clll}
		\toprule
		&& Addition & Multiplikation\\
		\midrule
		a)& $\exists$ neutrales Element & $n+0=n$ & $n\cdot 1 =  n$\\
		b)& kommutativ & $m+n=n+m$ & $m\cdot n = n\cdot m$ \\
		c)& assoziativ & $(k+m)+n = k+(m+n)$ & $(k\cdot m)\cdot n = k\cdot (m\cdot n)$ \\
		d)&distributiv & \multicolumn{2}{c}{$k(m+n) = k\cdot m + k\cdot n$} \\
		\bottomrule
	\end{tabular}
\end{proposition}

\begin{conclusion}
	Es gilt $\forall k,m,n\in\mathbb{N}$:
	\begin{enumerate}[label={\alph*)}]
		\item $m\neg 0 \Rightarrow m+n \neg 0$
		\item $m\cdot n = 0 \Leftrightarrow m = 0 \lor n = 0$
		\item $m + k = n + k \Leftrightarrow m = n$ (Kürzungsregel Addition)
		\item $k\neg 0: m\cdot k = n\cdot k \Leftrightarrow m = n$ (Kürzungsregel Multiplikation)
	\end{enumerate}
\end{conclusion}

\subsection*{Ordnung auf $\boldsymbol{\mathbb{N}}$}
\begin{definition}[Ordnung auf $\boldsymbol{\mathbb{N}}$]
	Betr. Relation $R:=\{(m,n) \in\mathbb{N}\times\mathbb{N}|m \le n\}$
\end{definition}
\begin{proposition}
	Es gilt auf $\mathbb{N}$:
	\begin{enumerate}[label={\arabic*)}]
		\item $m\le n \;\Rightarrow \;\exists!k\in\mathbb{N}: n = m + k$, nenne $n - m=:k$ \begriff{Differenz}
		\item Relation $R$ (bzw. "`$\le$"') ist Totalordnung auf $\mathbb{N}$
		\item Ordnung "`$\leq$"' ist verträglich mit Addition und Multiplikation
	\end{enumerate}
\end{proposition}