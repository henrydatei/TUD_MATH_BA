\section{Reelle Zahlen}
\stepcounter{theorem}%Example 1 is missing (not important)
\subsection*{Struktur von archimedisch angeordneten Körpern}
\begin{proposition}
	\proplbl{arch_koerper}
	Sei $K$ Körper. Dann gilt $\forall a,b\in K$:
	\begin{enumerate}[label={\arabic*)}]
		\item $0,1,(-a),b^{-1} (b\neq 0)$ sind eindeutig bestimmt
		\item $(-0) = 0, 1^{-1} = 1$
		\item $-(-a) = a, (b^{-1})^{-1} = b (b\neq 0)$
		\item $-(a+b) = (-a) + (-b), (ab)^{-1} = a^{-1}b^{-1} (a,b\neq 0)$
		\item $-a = (-1) a, (-a)(-b) = ab,\;a\cdot 0 = 0$
		\item $ab = 0 \Leftrightarrow a=0\lor b = 0$
		\item $a+x = b$ hat eindeutige Lösung $x = b+(-a) =: b-a$ \begriff{Differenz}
		$ax=b (a\neq 0)$ hat eindeutige Lösung $x=a^{-1}b =:\frac{b}{a}$ \begriff{Quotient}
	\end{enumerate}
\end{proposition}

\begin{*definition}
	\begin{itemize}
	\item \begriff{Vielfache}: $na := \sum_{k=1}^{n}a$
	
	Damit:
	\begin{itemize}
		\item $(-n)a := n(-a), 0_\mathbb{N} a := a_K$ für $n\in\mathbb{N}_{\ge 1}$
		\item $ma + na = (m+n)a, na + nb = n(a+b)$
		\item $(ma)\cdot(na) = (mn)a^2, (-n)a = -(na)$
	\end{itemize}
	\item \begriff{Potenz}: $a^n$ von $a\in K, n\in\mathbb{Z}:=\prod_{k=1}^{n} a$
	
	Damit
	\begin{itemize}
		\item $a^{-n} :=(a^{-1})^n, a^{0_K}:=1_K$ für $n\in\mathbb{N}_{\ge 1}, a\neq 0$
		\item $a^m a^n = a^{m+n}, (a^m)^n = a^{mn}, a^nb^n = (ab)^n, a^{-n} = (a^n)^{-1}$
	\end{itemize}
	\item \begriff{Fakkultät} für $n\in\mathbb{N}:$\mathsymbol*{n}{$n"!$} $n!:=\prod_{k=1}^n k, 0!=1$
	\item \begriff{Binomialkoeffizient} \mathsymbol{noverm}{$\binom{n}{k}$}$:=\frac{n!}{k!(n-k)!}\in\mathbb{N}$ $\forall k,n\in\mathbb{N}, 0\le k\le n$
	\begin{itemize}
		\item $\binom{k+1}{n+1} = \binom{n}{k} + \binom{n}{k+1}$
		\item Rechenregel führt auf \begriff{\person{Pascal}'sches Dreieck}
	\end{itemize}
	\end{itemize}
\end{*definition}

\begin{proposition}[Binomischer Satz]
	In Körper $K$ gilt: $(a+b)^n = \sum_{k=0}^n\binom{n}{k}a^n b^{n-k}, ,b\in K, n\in\mathbb{N}$
\end{proposition}

\begin{proof}
	ÜA %TODO schreibe Beweis aus Übung ab!
\end{proof}
\begin{proposition}
	Sei $K$ angeordneter Körper. Dann gilt $\forall a,b,c,d\in K$:
	\begin{enumerate}[label={\alph*)}]
		\item $a < b \Leftrightarrow 0 < b-a$
		\item $a < b, c < d \Leftrightarrow a+c < b+d$
		
		$0 \le a < b, 0 \le c < d \Leftrightarrow a\cdot c < b\cdot d$
		\item $a < b \Leftrightarrow -b < -a$ (insbes. $a > 0 \Leftrightarrow -a < 0$)
		
		$a < b, c < 0 \Leftrightarrow a\cdot c > b \cdot c$
		\item $a\neq 0 \Leftrightarrow a^2 > 0$ (insbes. 1 > 0)
		\item $a > 0 \Leftrightarrow a^{-1} > 0$
		\item $0 < a < b \Leftrightarrow b^{-1} < a^{-1}$
	\end{enumerate}
\end{proposition}

\begin{proof}
	Betrachte Ordnung veträglich mit Addition und Multiplikation.
	\begin{itemize}
		\item[a)] $a < b \Leftrightarrow a+(-a) < b (-a) \Leftrightarrow 0 \leq b -a$
		\item[b)] $a < b, c < d \Rightarrow a + c < b + d \overset{\text{transitiv}}{\Rightarrow} a + c < b + d$ Multi. analog
		\item[c)] $a < b \Leftrightarrow a-a-b < b-a-b \Leftrightarrow  - b < -a$\\
		$a < b, - c > 0 \overset{\text{Ord. vertr. Multi}}{\Leftrightarrow} a \cdot(-1) < b \cdot (-1) \Rightarrow (-1)ac < -1(bc) \Rightarrow -(ac) < - (bc) \overset{\text{c)}}{\Rightarrow} ac > bc$
		\item[d)] Sei $a > 0 \overset{\text{2)}}{\Rightarrow} a^{2} > 0$. Sei $a > 0 \Rightarrow (-a)>0 \overset{\text{b)}}{\Rightarrow} 0 < (-a)^2 \overset{\text{\propref{arch_koerper}}}{=} a$
		\item[e)] ``$\Rightarrow$'': $(a^{-1})^2 > 0$ nach d) $\overset{\text{vertr. mit Multi}}{\Rightarrow} a \cdot(a^{-1})^2 = a^{-1} > 0$\\
		``$\Leftarrow$'': Analog zu ``$\Rightarrow$'' ersetze $a^{-1} \text{ durch } a$
		\item[6)] $ab > 0$ nach b) $\overset{\text{5)}}{\Rightarrow} 0 < (ab)^{-1} \overset{\text{\propref{arch_koerper}}}{=} a^{-1}b^{-1}$ wegen $a < b \Rightarrow b^{-1} = a^{-1}b^{-1}a \leq a^{-1}b^{-1}b = a^{-1}$
	\end{itemize}
\end{proof}

\begin{*definition}
	\begriff{Absolutbetrag} $\vert\cdot\vert:K\rightarrow K$ (auf angeordneten Körper $K$) \[\vert a \vert:=\begin{cases}
	a&\text{für }a \ge 0 \\ -a& \text{für }a < 0\end{cases}\]
\end{*definition}

\begin{proposition}
	\proplbl{k_angeordneter_koerper}
	Sei $K$ angeordneter Körper. Dann gilt $\forall a,b\in K$:
	\begin{enumerate}[label={\arabic*)}]
		\item $\vert a\vert\ge 0, \vert a\vert\ge a$
		\item $\vert a\vert = 0$ \gls{gdw} $a=0$
		\item $\vert a\vert = \vert -a\vert$
		\item $\vert a\vert\cdot\vert b\vert = \vert a\cdot b\vert$
		\item $\left\vert \frac{a}{b}\right\vert = \frac{\vert a\vert}{\vert b\vert} (b\neq 0)$
		\item \begriff{Dreiecksungleichung}
		
		$\vert a+b\vert \le \vert a\vert + \vert b\vert$ ($\vert a-b\vert = \vert a+(-b)\vert \le \vert a\vert + \vert b\vert$)
		\item $\left\vert a\vert - \vert b\right\vert \le \vert a+b\vert$
		\item \begriff{\person{Bernoulli}-Ungleichung}
		
		$(1+a)^n \ge 1 + n\cdot a \,\forall a\ge -1, n\in\mathbb{N} (a\neq -1 \text{ bei }n = 0)$
		
		(Gleichheit \gls{gdw} $n=0,1$ oder $a=0$)
	\end{enumerate}
\end{proposition}
\begin{*definition}
	Betr. $f:\mathbb{Q}\rightarrow K$ mit $f\left(\frac{m}{n}\right):= \frac{m\cdot 1_K}{n\cdot 1_K}=(m 1_k)(n 1_K)^{-1}\,\forall m\in\mathbb{Z},k\in\mathbb{Z}_{\neq 0}$
\end{*definition}

\begin{proof}
	\begin{itemize}
		\item[1)] klar
		\item[2)] klar
		\item[3)] Fallunterscheidung SeSt
		\item[4)] Fallunterscheidung SeSt
		\item[5)] $a = \frac{a}{b}\cdot a \overset{\text{4)}}{\Rightarrow} \vert a \vert = \vert \vert \frac{a}{b} \vert  \cdot \vert b \vert \vert \overset{\cdot \vert b \vert^{-1}}{\Rightarrow} \frac{\vert a \vert}{\vert b \vert} = \vert \frac{a}{b} \vert$
		\item[6)] nach 1) $a \leq \vert a \vert, b \leq \vert b \vert \xRightarrow{\text{\propref{k_angeordneter_koerper}}} a+b \leq \vert a \vert + \vert b \vert$ analog $-a-b \leq \vert a + \vert b\vert \Rightarrow$ Behauptung
		\item[7)] $\vert a \vert = \vert a+b-b \vert \overset{\text{6)}}{\leq} \vert a+ b \vert + \vert b \vert \Rightarrow \vert a \vert - \vert b \vert \leq \vert a + b \vert $ analog $\vert b \vert - \vert a \vert \leq \vert a + b \vert \Rightarrow$ Behauptung
		\item[8)] für $n = 0,1$, $a = 0$ klar\\
		Zeige: $(1+a)^n > 1 + na \forall n \leq 2, a \neq 0$ durch voll. Induktion ÜA
	\end{itemize}
\end{proof}

	Betrachte: $f:\mathbb{Q}\rightarrow K$ mit $f\left(\frac{m}{n}\right):= \frac{m\cdot 1_K}{n\cdot 1_K}=(m 1_k)(n 1_K)^{-1}\,\forall m\in\mathbb{Z},k\in\mathbb{Z}\setminus\{0\} =:\mathbb{Z}_{\neq 0}$
	
\begin{proposition}
	\proplbl{inj_abb_ange_koerper}
	Sei $K$ angeordneter Körper\\
	$\Rightarrow$ $f:\mathbb{Q}\rightarrow K$ ist injektiv und $f$ erhält die Körperstruktur und Ordnung, d.h. $\forall p,q\in\mathbb{Q}$:
	\begin{itemize}
		\item[a)] $f(p+q) = f(p) + f(q), f(0) = 0_K, f(-p) = -f(p)$
		\item[b)] $f(p\cdot q) = f(p)\cdot f(q), f(1) = 1_K, f(p^{-1}) = f(p)^{-1} (p\neq 0)$
		\item[c)] $p \le_\mathbb{Q} q \Leftrightarrow f(p) \le_K f(q)$
	\end{itemize}
\end{proposition}

\begin{proof}
	\begin{itemize}
		\item[a)] $0_K \overset{\text{\propref{k_angeordneter_koerper}}}{<} 1 \overset{\text{voll. Ind.}}{\Rightarrow} 0_k < n 1_k \forall n \in \mathbb{N}
		\xRightarrow[\text{Vielfache}]{\text{\propref{k_angeordneter_koerper}}}(-n)1_K = -(n1_k) < 0_K \Rightarrow n 1_K \neq 0_K \forall n \in \mathbb{Z}_{\neq 0} \Rightarrow f$ auf $\mathbb{Q}$ definiert
		\item[b)] Sei $f(\frac{m}{m^{'}}) = f(\frac{n}{n^{'}}) \Rightarrow \frac{m1_K}{m^{'}1_K} = \frac{n 1_K}{n^{'}1_K} \Rightarrow (m1_K)(n^{'}1_K) = (n 1_K)(m^{'}1_K)$\\
		$\Rightarrow (mn^{'})1_K = (nm^{'})1_K \Rightarrow (mn^{'}-m^{'}n)1_K = 0_K \xRightarrow{\text{a)}} mn^{'} = m^{'}n =_{\mathbb{Z}} 0 \Rightarrow \frac{m}{m^{'}} =_{\mathbb{Q}} \frac{n}{n^{'}} \Rightarrow f$ injektiv
		\item[c)] $f(\frac{m}{m^{'}}+\frac{n}{n^{'}}) = f(\frac{mn^{'} + m^{'}n}{m{'}n^{'}}) = \frac{mn^{'} + m^{'}n}{m{'}n^{'}}\frac{1_K}{1_K} \overset{\text{b)}}{=} \frac{m1_K}{m^{'}1_K} + \frac{n1_K}{n^{'}1_K} \overset{f\text{ inj}}{=} f(\frac{m}{m^{'}}) + f(\frac{n}{n^{'}})$\\
		Multi., spezielle Elemente SeSt, Ordnung ÜA
	\end{itemize}
\end{proof}

\begin{conclusion}
	Es gilt im angeordneten Körper:
	\begin{enumerate}[label={\arabic*)}]
		\item $\mathbb{Q}_K = f(\mathbb{Q})$ ist mit Addition, Multiplikation und Ordnung von $K$ selbst angeordneter Körper
		\item $\mathbb{Q}_K$ ist isomorph zu $\mathbb{Q}$ bzgl. Körperstruktur und Ordnung.
	\end{enumerate}
\end{conclusion}

\begin{proof}
	\begin{itemize}
		\item[1)] $\mathbb{Q}_K \subset K$ und Addition und Multi. führen nicht aus $\mathbb{Q}_K$ (vgl. \propref{inj_abb_ange_koerper}) $\Rightarrow \mathbb{Q}_K$ selbst Körper mit Ordnung von $K \Rightarrow$ Behauptung
		\item[2)] nach \propref{inj_abb_ange_koerper} is $f$ entsprechender Isomorphismus
	\end{itemize}
\end{proof}

\begin{*anmerkung}
	$\mathbb{Q}_K \subset K$ und $\mathbb{Q}$ sind strukturell gleich $\Rightarrow$ können identifiziert werden.\\
	Analog $\mathbb{N}_K \subset\mathbb{Z}_K \subset K$, identifiziere: $n_K := n\cdot 1_K$ mit $n \in \mathbb{N}$ bzw. $n \in \mathbb{Z} \rightarrow$ Schreibe kurz (im angeordneten Körper $K$) $\mathbb{N} \subset \mathbb{Z} \subset \mathbb{Q} \subset K$\\
	$\rightarrow$ Vielfachheit $ma = (1_Ka + \dots + I_K a) = (1_K + \dots + 1_K)a = (m 1_K)\cdot a = m_K \cdot a$\\
	angeordneter Körper $K$ heißt archimedisch falls: \[\forall a \in K \exists n \in \mathbb{N} \subset K\quad a < n\]
\end{*anmerkung}

\begin{*definition}
	Angeordneter Körper heißt \begriff[Körper!]{archimedisch}, falls $\forall a\in K\,\exists n\in\mathbb{N}\subset K: a < n$.
\end{*definition}
\begin{proposition}
	\proplbl{k_archimedisch_angeordneter_koerper}
	Sei $K$ archimedisch angeordneter Körper. Dann\begin{enumerate}[label={\arabic*)}]
		\item $\forall a,b\in K$ mit $a,b>0\,\exists n\in\mathbb{N}: n\cdot a > b$
		\item $\forall a\in K\,\exists!\,[a]\in\mathbb{Z}: [a]\le a \le [a] +1$, \mathsymbol{a}{$[a]$} heißt \begriff{ganzer Anteil} von $a$
		\item $\forall \epsilon \in K$ mit $\epsilon > 0\,\exists n\in\mathbb{N}_{\neq 0}: \frac{1}{n}< \epsilon$ (beachte: $0 < \frac{1}{n}$)
		\item $\forall a,b\in K$ mit $a>1\,\exists n\in\mathbb{N}: a^n > b$
		\item $\forall a,\epsilon > 0\,\exists p,q\in\mathbb{Q}: p \le a  q$ und $q - p < \epsilon$
		
		(d.h. $a\in K$ kann auch rationale Zahlen beliebig genau approximiert werden, $\mathbb{Q}$ "`dicht"' in $K$)
		\item $\forall a,b\in K, a < b\,\exists q\in\mathbb{Q}:a < q < b$.
	\end{enumerate}
\end{proposition}

\begin{proof}
	\begin{itemize}
		\item[1)] $a > 0 \Rightarrow \frac{b}{a} \in K \Rightarrow \exists n \in \mathbb{N} \colon n > \frac{b}{a} \xRightarrow{\cdot a}$ Behauptung
		\item[2)] es ist $N:=\{n \in \mathbb{Z} \mid 0 < n\}\neq \emptyset$:\\
		$N$ hat \highlight{kleinstes Element} $\tilde{n} \in N$ (d.h. $\tilde{n} \leq n \forall n \in \mathbb{N}$) vgl. ÜA\\
		Setze $[a]:=\tilde{n} -1 \xRightarrow{\text{Def }\tilde{n}} [a] = \tilde{n} -1 \leq a < \tilde{n} = [a] +1$ falls $alpha$ ganzer Anteil mit $\alpha < [a] \Rightarrow [a] \leq a < \alpha +1 \xRightarrow{-\alpha} 0 < \underbrace{[a] - \alpha}_{\in \mathbb{N}} < \alpha \Rightarrow \lightning \xRightarrow{\text{\gls{obda}}} [a]$ eindeutig
		\item[3)] Wähle $n > \frac{1}{\epsilon} \Rightarrow$ Behauptung
		\item[4)] $\exists n \in \mathbb{N} b \overset{\text{1)}}{<} n(a-1) < 1 + n(a-1) \overset{\text{Bernoulli-Ungl.}}{\leq} (1+(a-1))^n = a^n$
		\item[5)] Verwende 4) mit $\tilde{a}:=\frac{1}{a},\tilde{b}:=\frac{1}{\epsilon}$
		\item[6)] nach 3) $\exists n \in \mathbb{N}_{\neq 0}$ mit $\frac{1}{n} < \epsilon, p:= \frac{[na]}{n},q:= \frac{}{den}$ % to be finished!
	\end{itemize}  
\end{proof}

\begin{*definition}[Intervall]
	\begriff{Intervall} für angeordneten Körper $K$: Sei $a,b\in K$:
	\begin{itemize}
		\item \begriff{beschränktes Intervall}
		\begin{itemize}
			\item $[a,b]:=\{ x\in K | a \le x \le b \}$ \begriff[Intervall!]{abgeschlossen}
			\item $(a,b):=\{a < x < b\}$ \begriff[Intervall!]{offen}
			\item $[a,b) := \{a \le x < b\}, (a,b]:=\{a < x \le b\}$ \begriff[Intervall!]{halboffen}
		\end{itemize}
		\item \begriff{unbeschränktes Intervall}
		\begin{itemize}
			\item $[a,\infty]:=\{x\in K\mid a \le x\}$
			\item $(a,\infty):=\{x\in K\mid a > x\}$
            \item $(-\infty, b]:= \{x \in K \mid x< a\}$
            \item $(-\infty, b) := \{x\in K\mid x \leq b\}$
		\end{itemize}
	\end{itemize}
\end{*definition}

\begin{*definition}[Folge]
    Eine \begriff{Folge} in Menge $M$ ist eine Abbildung $\alpha:\mathbb{N}\rightarrow M$ (evtl. $\alpha:\mathbb{N}_{\ge n}\rightarrow M$), $\alpha_n := \alpha(n)$ heißen \begriff{Folgenglieder}, und \begriff{Folgenindex}.
    
	Notation: $\{a_n\}_{n\in\mathbb{N}}, \{\alpha_n\}_{k=1}^\infty$ bzw. $\alpha_0, \alpha_1, \dotsc$\\
	kurz: $\{\alpha_n\}_n, \{\alpha_n \}$
		
	Hinweis: $\{x\}_n$ ist \begriff{konstante Folge}, d.h. $\alpha_n = \alpha\,\forall n$
\end{*definition}

Aussage gilt für \gls{fa} $n\in\mathbb{N}$, wenn höchstens für endlich viele $n$ falsch.

\begin{*definition}[Intervallschachtelung]
	Folge $\{x_n\}_{n\in\mathbb{N}} =:\mathcal{X}$ von abgeschlossenen Intervallen $X_n=[x_n, x_n']\subset K$ $(x_n, x_n'\in K)$ heißt \begriff{Intervallschachtelung} (im angeordneten Körper K), falls
	\begin{enumerate}[label={\alph*)}]
		\item $X_n\neq \emptyset$ und $X_{n+1}\subset X_n\,\forall n\in\mathbb{N}$
		\item $\forall\epsilon > 0$ in $K$ existiert $n\in\mathbb{N}: l(X_n):= x_n' - x_n < \epsilon$, mit $l$ \begriff{Intervalllänge}
	\end{enumerate}
\end{*definition}

\begin{lemma}
	Sei $\mathcal{X} = \{X_n\}_{n\in\mathbb{N}}$ Intervallschachtelung im angeordneten Körper $K$\\
	$\Rightarrow \bigcap_{n\in\mathbb{N}} X_n$ enthält höchstens ein Element.
\end{lemma}


\begin{*definition}
	Archimedisch angeordneter Körper heißt \begriff[Körper]{vollständig}, falls $\bigcap_{n\in\mathbb{N}} X_n\neq \emptyset$ für jede Intervallschachtelung $\mathcal{X} = \{x_n\}$ in $K$.
\end{*definition}

\begin{*definition}
	$Q:=\{ (\{x_n\}, \{y_n\})\in I_\mathbb{Q}\times I_\mathbb{Q} \}$ ist Relation auf $I_\mathbb{Q}$, $I_\mathbb{Q}:=$ Menge aller Intervallschachtelungen $\mathcal{X}=\{x_n\} \in \mathbb{Q}$.
\end{*definition}

\begin{proposition}
	$Q$ ist Äquivalenzrelation auf $I_\mathbb{Q}$.
\end{proposition}

\begin{*definition}
	setze $\mathbb{R} := \{ [\mathcal{X}] \mid \mathcal{X}\in I_\mathbb{Q} \}$ Menge der \begriff{reellen Zahlen}.
	
	\begin{itemize}
		\item $\bigcap_{n\in\mathbb{N}} X_n \neq 0 \rightarrow [\mathcal{X}]$ ist "`neue"' sog. \begriff{irrationale Zahl}
	\end{itemize}
\end{*definition}

\subsection{Rechenoperationen}
\begin{*definition}
	Für Intervalle $X=[x,x'], Y=[y,y']$ in $\mathbb{Q}$ defineren wir Intervall in $\mathbb{Q}$:
	\begin{itemize}
		\item $X + Y := \{\xi + \eta \mid \xi \in X, \eta\in Y\} = [x + y, x' + y']$
		\item $X\cdot Y :=\{\xi \cdot \eta \mid \xi \in X, \eta\in Y\} = [\tilde{x}\tilde{y}, \tilde{x}'\tilde{y}']$, wobei $\tilde{x},\tilde{x}'\in\{x,x'\},\tilde{y},\tilde{y}'\in\{y,y'\}$
		\item $-X := [-x,-x']$, $X^{-1}:=[\frac{1}{x'}, \frac{1}{x}]$ falls $0\in X$
	\end{itemize}

	Für relle Zahl $[\mathcal{X}] = [\{x_n\}], [\mathcal{Y}]=[\{y_n\}]$ sei
	\begin{itemize}
		\item $[\mathcal{X}]+\mathcal{Y} :=[\{x_n + y_n\}]$
		\item $[\mathcal{X}]\cdot[\mathcal{Y}] :=[\{x_n\cdot y_n\}]$
		\item $-[\mathcal{X}]:=[\{-x_n\}]$
			
			$[\mathcal{X}]^{-1} := [\{x_n^{-1}\}]$ falls $[\mathcal{X}]\neq 0_\mathbb{R}$
	\end{itemize}
\end{*definition}

\begin{proposition}
	\begin{enumerate}[label={\arabic*)}]
		\item Addition, Multiplikation und Inverse sind in $\mathbb{R}$ eindeutig definiert
		\item $\mathbb{R}$ ist damit und neutralen Elementen ein Körper.
	\end{enumerate}
\end{proposition}

\subsection{Ordnung auf \texorpdfstring{$\mathbb{R}$}{R}}
\begin{*definition}
	Betr. Relation "`$\le$"': $R:=\{ ([\{x_n\}],[\{y_n\}])\in\mathbb{R}\times\mathbb{R} | x_n \le y_n\,\forall n\in\mathbb{N}\}$
\end{*definition}
\begin{proposition}
	$\mathbb{R}$ ist mit "`$\le$"' angeordneter Körper.
\end{proposition}
\begin{proposition}
	$\mathbb{R}$ ist archimedisch angeordneter Körper.
\end{proposition}
\begin{theorem}
	$\mathbb{R}$ ist vollständiger, archimedisch angeordneter Körper.
\end{theorem}
\begin{theorem}
	Sei $K$ vollständiger, archimedisch angeordneter Körper\\
	$\Rightarrow K$ ist isomorph zu $\mathbb{R}$ bzgl. Körperstruktur und Ordnung.
\end{theorem}

\begin{*definition}
	Sei $M\subset K$, $K$ angeordneter Körper.
	\begin{itemize}
		\item $s\in K$ ist \begriff[Schranke!]{obere} / \begriff[Schranke!]{untere} \begriff{Schranke} von $M$, falls $x \le s (x \ge s)\,\forall x\in M$
		
		$M$ ist nach \begriff[beschränkt!]{oben} / \begriff[beschränkt!]{unten} \highlight{beschränkt}, falls obere ( untere ) Schranke existiert.
		\item $M$ \begriff{beschränkt}[!Menge im Körper], falls $M$ nach oben und unten beschränkt.
		\item kleinste obere (größte untere) Schranke $\tilde{s}$ von $M$ ist \begriff{Supremum} (\begriff{Infimum}) von $M$, d.h. \\
		\mathsymbol{sup}{$\sup$}$ M:= \tilde{s} \le s ($\mathsymbol{inf}{$\inf$}$ M = s \ge \tilde{s}) \;$ obere (untere) Schranken $s\in M$.
		\item Falls $\sup M \in M (\inf M\in M)$ nennt man dies auch \begriff{Maximum} (\begriff{Minimum}) von $M$.
		
		kurz: \mathsymbol{max}{$\max$}$M = \sup M ($\mathsymbol{min}{$\min$}$M = \inf M)$
		\item falls $M$ nach oben (unten) \begriff[Menge!]{unbeschränkt}, d.h. nicht beschränkt, schreibt man auch $\sup M = \infty (\inf M = -\infty)$
	\end{itemize}

	Man hat
	\begin{align*}
	\sup M &= \min\{s \mid s \text{ obere Schranke von } M\}\\
	\inf M &= \max\{s \mid s \text{ untere Schranke von } M\}
	\end{align*}
\end{*definition}
\stepcounter{theorem}
\begin{proposition}
	Sei $K$ angeordneter Körper, $M\subset K$. Falls $\sup M\;(\inf M)$ existiert, dann
	\begin{enumerate}[label={\arabic*)}]
		\item $\sup M\;(\inf M)$ eindeutig
		\item $\forall \epsilon > 0\,\exists y\in M: \sup M < y + \epsilon\;(\inf M > y - \epsilon)$
	\end{enumerate}
\end{proposition}

\begin{theorem}
	Sei $K$ archimedisch angeordneter Körper. Dann
	\[ K \text{ vollständig } \Leftrightarrow \sup M \slash \inf M \text{ ex. }\forall M\in K, M\neq \emptyset \text{ nach oben \slash unten beschränkt} \]
\end{theorem}

\subsection{Anwendung: Wurzeln, Potenzen, Logarithmen in \texorpdfstring{$\mathbb{R}$}{R}}

\begin{proposition}[Wurzeln]
	Sei $a\in\mathbb{R}_{>0}, k\in\mathbb{N}_{>0} \;\Rightarrow \; \exists ! x\in \mathbb{R}_{>0}: x^k = a, \sqrt[k]{a}:=a^{\frac{1}{k}} = x$ heißt \highlight{k-te} \begriff{Wurzel} von $a$.
\end{proposition}

\begin{*definition}[Potenz]
	$n$-te \begriff{Potenz} von $a\in\mathbb{R}_{>0}, r\in\mathbb{R}$:
	
	Zunächst $r=\frac{m}{n}\in\mathbb{Q}$ (\gls{obda}) $n\in\mathbb{N}_{>0}$): $ a^{\frac{m}{n}}:= (a^m)^{\frac{1}{n}}$
	Allgemein für $a\ge 0, a > : a^r := \sup \{ a^q \mid 0 \le q \le r,q\in\mathbb{Q} \}$
	offenbar eindeutig definiert und allgemeine Definition konsistent mit Definition für $\frac{m}{n}\in\mathbb{Q}$.
	Damit: \begriff{Exponentialfunktion}
\end{*definition}

\begin{proposition}\label{proposition_potenz_r}
	\proplbl{satz_potenz_r}
	Seien $a,b\in\mathbb{R}_{>0}, r,s\in\mathbb{R}$. Dann
	\begin{enumerate}[label={\arabic*)}]
		\item $a^r b^r = (ab)^r, (a^r)^s = a^{rs}, a^ra^s = a^{r+s}$
		\item f. $r > 0: a < b \Leftrightarrow a^r < b^r$
		\item für $a > 1: r < s \Leftrightarrow a^r < a^s$
	\end{enumerate}
\end{proposition}

\begin{*definition}[Logarithmus]
	Sei $a,b\in\mathbb{R}_{<0}, a\neq 1$: \begriff{Logarithmus}\highlight{von $b$ zur Basis $a$} ist \begin{align*}
	 \log_a b :=\begin{cases}
	 \sup \{ r \in \mathbb{R} \mid a^r \le b\}& a > 1\\
 	\sup \{r\in\mathbb{R}\mid a^r \ge b\}& 0 < a < 1
	 \end{cases}
	\end{align*}
\end{*definition}

\begin{proposition}\label{proposition_logarithmus_r}
	Se $a,b,c\in\mathbb{R}_{>0}, a\neq 1$. Dann
	\begin{enumerate}[label={\arabic*)}]
		\item $log_a b$ ist eindeutige Lösung von $a^x = b$, d.h. $a^{log_a b} = b$
		\item $\log_a a = 1, log_a 1 = 0$
		\item $\log_a b^\gamma = \gamma \log_a b \,\forall \gamma\in\mathbb{R}$
		\item $\log_a(bc) = \log_a b + \log_a c, \log_a \frac{b}{c} = \log_a b - \log_a c$
		\item $\log_a b = \frac{\log_\alpha b}{\log_\alpha a}\,\forall \alpha\in\mathbb{R}_{>0},\alpha\neq 1$
	\end{enumerate}
\end{proposition}

\subsection{Mächtigkeit von Mengen}

\begin{*definition}
	$M$ \begriff[Mächtigkeit!]{endlich}, falls $M$ endlich viele Elemente hat, sonst \begriff[Mächtigkeit!]{unendlich}.
	
	Unendliches $M$ ist \begriff[Mächtigkeit!]{abzählbar}, falls bijektive Abbildung $f:\mathbb{N}\to M$ existiert, sonst ist $M$ \begriff[Mächtigkeit!]{überabzählbar}.
\end{*definition}

\begin{proposition}
	Es gilt:
	\begin{enumerate}[label={\arabic*)}]
		\item $\mathbb{Z},\mathbb{Q}$ abzählbar
		\item $M$ abzählbar, $n\in\mathbb{N}_{>0} \Rightarrow M^n$ abzählbar ($\Rightarrow \mathbb{Z}^n, \mathbb{Q}^n$ abzählbar)
		\item Ein offenes Intervall $I\in\mathbb{R}\neq \emptyset $ ist überabzählbar
		\item $\mathcal{P}(\mathbb{N})$ ist überabzählbar.
	\end{enumerate}
\end{proposition}