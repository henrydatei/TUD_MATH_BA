Integration kann betrachtet werden als
\begin{itemize}
	\item verallgemeinerte Summation, d.h. $\int_\mu f\D x$ ist Grenzwert von Summen
	\item lineare Abbildung $\int: \mathcal{F}\marginnote{$\mathcal{F}$: Menge der Funktionen}\to \mathbb{R}$ über $\int_a^b (\alpha f + \beta g)\D x = \alpha \int_a^b f \D x + \beta \int_a^b g \D x$ Funktionen, d.h. als Grundlage benötigt man ein "`Volumen"' (Maß) für allgemeine Mengen $M\subset\mathbb{R}$.
	
	Wir betrachten Funktionen $f:D\subset\mathbb{R}^n\to \mathbb{R}\cup \{ \pm \infty \}$, welche komponentenweise auf $f:D\subset\mathbb{R}\to K^k$ erweitert werden kann. Benutze $C^m \cong \mathbb{R}^{2m}$ für $K=\mathbb{C}$.
	
	Vgl. Buch:  Evans, Lawrence C.; Gariepy, Ronald F.: Measure theory and fine properties of functions
\end{itemize}