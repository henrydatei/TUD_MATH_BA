\section{Counting Sort}

Der Counting Sort-Algorithmus ist dann besonders gut, wenn alle möglichen Schlüsselwerte aus der Menge $S=\{0,1,2,...,k-1\}$ kommen. Wenn $k=\mathcal{O}(n)$, dann können wir mit Counting Sort in der Zeit $T(n)=\mathcal{O}(n)$ sortieren. Benötigt werden 3 eindimensionale Felder:
\begin{itemize}
	\item \texttt{A(1:n)} Originaldaten/-schlüsselwerte
	\item \texttt{B(1:n)} für sortierte Werte
	\item \texttt{C(0:k-1)} Feld von Zählern
\end{itemize}
Der Algorithmus zählt, wie oft jeder Wert in der Eingabe vorkommt. Diese Anzahlen speichert er in einem zusätzlichen Array mit $k$ Feldern ab. Mit Hilfe dieses Arrays wird anschließend für jeden Schlüsselwert die Zielposition in der Ausgabe berechnet.

\begin{*anmerkung}
	Der Counting Sort ist also dann besonders gut, wenn man alle Einwohner Deutschlands ($\sim$ 81 Millionen) nach ihrem Alter ($S=\{0,...,150\}$) sortieren möchte. \\
	Der Counting Sort eignet sich nicht dafür die Studenten der PROG-Vorlesung ($\sim$ 30 \smiley{}) nach ihrem Nachnamen (40 Zeichen $\Rightarrow S=\{0,...,26^{40}\}$) zu sortieren.
\end{*anmerkung}

\begin{lstlisting}
subroutine counting_sort_a(array)
 integer, dimension(:), intent(inout) :: array

 call counting_sort_mm(array, minval(array), maxval(array))
end subroutine counting_sort_a

subroutine counting_sort_mm(array, tmin, tmax)
 integer, dimension(:), intent(inout) :: array
 integer, intent(in) :: tmin, tmax
 integer, dimension(tmin:tmax) :: cnt
 integer :: i, z

 cnt = 0 ! Initialize to zero to prevent false counts
 
 forall (i=1:z)  
  ! Not sure that this gives any benefit over a DO loop.
  cnt(array(i)) = cnt(array(i))+1
 end forall
 
 ! ok - cnt contains the frequency of every value
 ! let's unwind them into the original array
 
 z = 1
 do i = tmin, tmax
  do while ( cnt(i) > 0 )
   array(z) = i
   z = z + 1
   cnt(i) = cnt(i) - 1
  end do
 end do
end subroutine counting_sort_mm
\end{lstlisting}