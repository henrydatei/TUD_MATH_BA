\section{Begriffe und Definitionen}

Beim \begriff{linearen Suchen} sucht man in einem unsortierten Feld mit $n$ Elementen. Der Aufwand liegt zwischen 1 und $n$, ist also linear abhängig von der Anzahl der Elemente. $T(n)=\mathcal{O}(n)$

Beim \begriff{binären Suchen} muss das Feld schon sortiert sein. Man fragt dabei den Schlüsselwert des mittleren Elements ab und kann so den zu durchsuchenden Bereich in jedem Schritt halbieren. $T(n)=\mathcal{O}(\log_2 n)$

\begin{*anmerkung}
	Dieses Verfahren wurde im letzten Semester schon in der Aufgabe zum Zahlenraten verwendet.
\end{*anmerkung}

Man kann Sortierverfahren nach ihrem Speicherplatzbedarf unterteilen: \begriff{in-situ Sortierverfahren} vs. \begriff{externes Sortierverfahren}
%\begin{center}
%	\begin{tabularx}{\textwidth}{l|l} %TODO: Tabelle
%		\textbf{in-situ} & \textbf{extern} \\
%		\hline
%		\begin{itemize}
%			\item alle $n$ Elemente sind schon zu Beginn (in beliebiger Anordnung) im Feld gespeichert
%			\item die Elemente werden (bis auf eventuelle notwendige kurzzeitige Auslagerung eines Elements)
%			immer in diesem Feld gehalten
%			\item insbesondere wird nur eine sehr kleine Zahl zusätzlicher skalarer Varianten benötigt (und keine
%			zusätzliche Datenstruktur mit $c \cdot n$ Elementen ($0 < c < 1$))
%		\end{itemize} & 
%		\begin{itemize}
%			\item Sortierung erfolgt nicht nur im Originalfeld
%			\item benötigt typischerweise $\mathcal{O}(n)$ zusätzlichen Speicherplatz
%		\end{itemize} \\
%	\end{tabularx}
%\end{center}

Ein Sortierverfahren ist \begriff{stabil}, wenn es die relative Ordnung von Elementen mit dem selben Schlüsselwert nicht ändert.

Ein \begriff{Mikroschritt} bzw. eine \begriff{Elementaroperation} besteht in der Regel aus einem Vergleich von 2 Schlüsselwerten und einer Kopier- oder Tauschoperation. Ein \begriff{Makroschritt} bzw. \begriff{Durchlauf} besteht aus $\mathcal{O}(n)$ Mikroschritten, zum Beispiel der Durchlauf durch alle noch zu sortierenden Elemente.

Die Zeitkomplexität von Algorithmen $T(n)$ wird mit den \person{Landau}-Operatoren angegeben:
\begin{itemize}
	\item $\mathcal{O}(g(n))=\{f(n):\exists c>0, n_0\in\natur_0\mid 0\le f(n)\le c\cdot g(n)\quad\forall n\ge n_0\}$ (Obergrenze)
	\item $\Omega(g(n)) = \{f(n):\exists c>0, n_0\in\natur_0\mid 0\le c\cdot g(n)\le f(n)\quad\forall n\ge n_0\}$ (Untergrenze)
	\item $\Theta(g(n)) = \{f(n):\exists c_1,c_2>0,n_0\in\natur_0\mid 0\le c_1\cdot g(n)\le f(n)\le c_2\cdot g(n)\quad\forall n\ge n_0\}$ (Sandwich)
\end{itemize}

Also gilt: $T(n) = \mathcal{O}(n^2) = \mathcal{O}(n^2\cdot\log n) = \mathcal{O}(n^2\cdot\sqrt{n}) = \mathcal{O}(n^3) = \dots = \mathcal{O}(2^n) = \mathcal{O}(n^n)$.

\begin{*anmerkung}
	Die Schreibweise kann ziemlich verwirren; es hilft sich $\mathcal{O}(n^2)$ als Menge vorzustellen, die alle Funktionen enthält, die maximal so schnell wie $n^2$ wachsen. Die Schreibweise $\mathcal{O}(n^2)=\mathcal{O}(n^2\cdot\log n)$ bedeutet dann nicht, dass diese Mengen gleich sind, sondern dass die eine Menge in der anderen enthalten ist: Es gilt also $\mathcal{O}(n^2)\in\mathcal{O}(n^2\cdot\log n)$, denn $x^2 \le x^2\cdot\log x$ für alle $x$.
\end{*anmerkung}

Sortieralgorithmen bekommen in der Regel 3 Komplexitätsangaben:
\begin{itemize}
	\item \begriff{worst case}: $\mathcal{O}(\dots)$ oder $\Theta(\dots)$
	\item \begriff{average case}: $\mathcal{O}(\dots)$, $\Omega(\dots)$ oder $\Theta(\dots)$
	\item \begriff{best case}: $\Omega(\dots)$
\end{itemize}

Im folgenden wird die \textbf{allgemeine Annahme} gelten: Sortiert wird immer in einem eindimensionalen Feld $A$ mit Indexmenge $I$ mit der Relation $\le$ bezüglich des Schlüssels in aufsteigender Reihenfolge.

Es gibt mindestens 2 Möglichkeiten, Datenelemente im Feld $A$ zu sortieren:
\begin{enumerate}
	\item \begriff{direktes Sortieren}: Bewegen der Datenelemente inklusive key
	\item \begriff{indirektes Sortieren}: Erzeugen einer Sortierpermutation $\sigma$ der Indizes, wobei nur die Indizes in einem eigenen Feld und nicht die Datenelemente bewegt werden
\end{enumerate}
Im sortierten Zustand gilt für alle $i,j\in I$:
\begin{itemize}
	\item direktes Sortieren: $i < j \Rightarrow A(i) \le A(j)$
	\item indirektes Sortieren: $i < j \Rightarrow A(\sigma(i)) \le A(\sigma(j))$
\end{itemize}

Eine \begriff{Sortierpermutation} $\sigma$ einer Liste $A$ auf einer Indexmenge $I=\{1,\dots,n\}$ ist eine Permutation von $I$, das heißt $\{\sigma(1),\dots,\sigma(n)\}$ mit $\sigma(i)\neq\sigma(j)$ für $i\neq j$.