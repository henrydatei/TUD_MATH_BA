\section{Gruppen}

\begin{definition}[(Halb-)Gruppe]
	Sei $G$ eine Menge. Eine (innere, zweistellige) Verknüpfung
	auf $G$ ist eine Abbildung $*: G \times G \to G, (x,y) \mapsto x*y$. Das Paar $(G,*)$ ist eine
	\begriff[Gruppe!]{Halbgruppe}, wenn das folgende Axiom erfüllt ist:
	\begin{itemize}
		\item (G1) Für $x,y,z \in G$ ist $(x*y)*z=x*(y*z)$.
	\end{itemize}
		Eine Halbgruppe $(G,*)$ ist ein \begriff{Monoid}, wenn zusätzlich das folgende Axiom gilt:
	\begin{itemize}
		\item (G2) Es gibt ein Element $e \in G$, welches für alle $x \in G$ die Gleichung $x*e=e*x=x$
		erfüllt. Dieses Element heißt dann \begriff{neutrales Element} der Verknüpfung $*$.  
	\end{itemize}
\end{definition}

\begin{example}
	\begin{itemize}
		\item Für jede Menge $X$ ist $(\Abb(X,Y), \circ)$ eine Halbgruppe (\propref{1_2_10}) mit dem neutralen Element
		$\id_x$, also ein Monoid.
		\item $\mathbb N$ bildet mit der Addition eine Halbgruppe $(\mathbb N,+)$, aber kein Monoid,
		da die 0 nicht in Fehm's Definition der natürlichen Zahlen gehörte
		\item $\mathbb N_0$ bildet mit der Addition ein Monoid $(\mathbb N_0,+)$
		\item $\mathbb N$ bildet mit der Multiplikation ein Monoid $(\mathbb N, \cdot)$
		\item $\mathbb Z$ bildet mit der Multiplikation ein Monoid $(\mathbb Z, \cdot)$
	\end{itemize}
\end{example}

\begin{proposition}[Eindeutigkeit des neutralen Elements]
	Ein Monoid $(G,*)$ hat genau ein neutrales Element. 
\end{proposition}
\begin{proof}
	Nach Definition besitzt $(G,*)$ mindestens ein neutrales Element. Seien $e_1,e_2\in G$ neutrale Elemente. Dann 
	ist $e_1=e_1 * e_2=e_2$. Damit besitzt $(G,*)$ höchstens ein neutrales Element, also genau ein neutrales Element.
\end{proof}

\begin{definition}[(abelsche) Gruppe]
	Eine \begriff{Gruppe} ist ein Monoid $(G,*)$ mit dem neutralen Element
	$e$, in dem zusätzlich das folgende Axiom gilt:
	\begin{itemize}
		\item (G3) Für jedes $x \in G$ gibt es ein $x' \in G$ mit $x'*x=x*x'=e$.
	\end{itemize}
	Gilt weiterhin
	\begin{itemize}
		\item (G4) Für alle $x,y \in G$ gilt $x*y=y*x$, so heißt diese Gruppe \begriff[Gruppe!]{abelsch}.
	\end{itemize}
\end{definition}

Ein $x'$ heißt \begriff{inverses Element} zu $x$. \\

\begin{example}
	\begin{itemize}
		\item $\mathbb N_0$ bildet mit der Addition keine Gruppe $(\mathbb N_0,+)$
		\item $\mathbb Z$ bildet mit der Addition eine abelsche Gruppe $(\mathbb Z,+)$
		\item Auch $(\mathbb Q,+)$ und $(\mathbb R,+)$ sind abelsche Gruppen
		\item $(\mathbb Q,\cdot)$ ist keine Gruppe, aber $(\mathbb Q\backslash\{0\},\cdot)$ schon
	\end{itemize}
\end{example}

\begin{proposition}[Eindeutigkeit des Inversen]
	Ist $(G,*)$ eine Gruppe, so hat jedes $x \in G$ genau ein inverses Element.
\end{proposition}
\begin{proof}
	Nach Definition hat jedes $x\in G$ mindestens ein Inverses. Seien $x',x''\in G$ inverse Elemente zu $x$. Dann ist 
	$x'=x'*e=x'*(x*x'')=(x'*x)*x''=e*x''=x''$. Es gibt also genau ein Inverses zu $x$.
\end{proof}

\begin{example}
	\proplbl{1_3_7}
	\begin{itemize}
		\item Eine triviale Gruppe besteht nur aus ihrem neutralen Element. Tatsächlich ist $G=\{e\}$ mit
		$e*e=e$ eine Gruppe.
		\item Sei $X$ eine Menge. Die Menge $\Sym(X) := \{f \in \Abb(X,X) \mid f$ ist bijektiv$\}$ der
		Permutationen von $X$ bildet mit der Komposition eine Gruppe $(\Sym(X),\circ)$, die 
		\begriff[Gruppe!]{symmetrische Gruppe} auf $X$. Für $n \in \mathbb N$ schreibt man $S_n := \Sym(\{1,2,...,n\})$. 
		Für $n \ge 3$ ist $S_n$ nicht abelsch.
	\end{itemize}
\end{example}

\begin{remark}
	Häufig benutzte Notationen für die Gruppenverknüpfung $\cdot$:
	\begin{itemize}
		\item In der multiplikativen Notation schreibt man $\cdot$ statt $*$ (oft auch $xy$ statt 
		$x \cdot y$), bezeichnet das neutrale Element mit $1$ oder $1_G$ und das Inverse zu $x$ mit
		$x^{-1}$.
		\item In der additiven Notation schreibt man $+$ für $*$, bezeichnet das neutrale Element
		mit $0$ oder $0_G$ und das Inverse zu $x$ mit $-x$. Die additive Notation wird nur verwendet,
		wenn die Gruppe abelsch ist.
	\end{itemize}
\end{remark}

In abelschen Gruppen notiert man Ausdrücke auch mit dem Summen- und Produktzeichen. \\

\begin{proposition}
	Sei $(G,\cdot)$ eine Gruppe. Für $x,y \in G$ gelten
	\begin{align}
		(x^{-1})^{-1} &= x \notag \\
		(xy)^{-1} &= x^{-1} \cdot y^{-1} \notag
	\end{align}
\end{proposition}
\begin{proof}
	Nach Definition erfüllt $z=x$ die Identitäten $x^{-1}z=zx^{-1}=1$ und somit ist $(x^{-1})^{-1}=z=x$. Ebenso ist 
	$(y^{-1}x^{-1})\cdot (xy)=y^{-1}(x^{-1}x)y=1$ und $(xy)\cdot (x^{-1}y^{-1})=x(yy^{-1})x^{-1}=1$, also $y^{-1}
	x^{-1}=(xy)^{-1}$.
\end{proof}

\begin{proposition}
	\proplbl{1_3_10}
	Sei $(G,\cdot)$ eine Gruppe. Für $a,b \in G$ haben die Gleichungen $ax=b$ und
	$ya=b$ eindeutige Lösungen in $G$, nämlich $x=a^{-1} \cdot b$ und $y=b \cdot a^{-1}$. 
	Insbesondere gelten die folgenden Kürzungsregeln: $ax=ay \Rightarrow x=y$ und $xa=ya 
	\Rightarrow x=y$.
\end{proposition}
\begin{proof}
	Es ist $a \cdot a^{-1} \cdot b = 1b=b$, also ist $x=a^{-1} \cdot b$ eine Lösung. Ist umgekehrt
	$ax=b$ mit $x \in G$, so ist $a^{-1} \cdot b = a^{-1} \cdot ax = 1x = x$ die Lösung und somit
	eindeutig. Für die zweite Gleichung argumentiert man analog. Den "'Insbesondere"'-Fall erhält
	man durch Einsetzen von $b=ay$ bzw. $b=xa$.
\end{proof}

\begin{remark}
	Wenn aus dem Kontext klar ist, welche Verknüpfung gemeint ist, schreibt man auch einfach
	$G$ anstatt $(G, \cdot)$ bzw. $(G,+)$. Eine Gruppe $G$ heißt endlich, wenn die Menge $G$ endlich
	ist. Die Mächtigkeit $|G|$ von $G$ nennt man dann die Ordnung von $G$. Eine endliche Gruppe kann 
	durch ihre Verknüpfungstafel vollständig beschrieben werden.
\end{remark}

\begin{example}
	\begin{itemize}
		\item die triviale Gruppe $G=\{e\}$
		\begin{center}
			\begin{tabular}{|c|c|}
				\hline
				$\cdot$ & $e$\\
				\hline
				$e$ & $e$ \\
				\hline
			\end{tabular}
		\end{center}
		\item die Gruppe $\mu_2 = \{1,-1\}$ der Ordnung 2
		\begin{center}
			\begin{tabular}{|c|c|c|}
				\hline
				$\cdot$ & $1$ & $-1$\\
				\hline
				$1$ & $1$ & $-1$ \\
				\hline
				$-1$ & $-1$ & $1$ \\
				\hline
			\end{tabular}
		\end{center}
		\item die Gruppe $S_2= \Sym(\{1,2\}) = \{\id_{\{1,2\}},f\}$, wobei $f(1)=2$ und $f(2)=1$
		\begin{center}
			\begin{tabular}{|c|c|c|}
				\hline
				$\circ$ & $\id_{\{1,2\}}$ & $f$\\
				\hline
				$\id_{\{1,2\}}$ & $\id_{\{1,2\}}$ & $f$ \\
				\hline
				$f$ & $f$ & $\id_{\{1,2\}}$ \\
				\hline
			\end{tabular}
		\end{center}
	\end{itemize}
\end{example}

\begin{definition}[Untergruppe]
	Eine \begriff{Untergruppe} einer Gruppe $(G,\cdot)$ ist eine 
	nichtleere Teilmenge $H \subset G$, für die gilt:
	\begin{itemize}
		\item (UG1) Für alle $x,y \in H$ ist $x \cdot y \in H$ (Abgeschlossenheit unter Multiplikation).
		\item (UG2) Für alle $x \in H$ ist $x^{-1} \in H$ (Abgeschlossenheit unter Inversen).
	\end{itemize}
\end{definition}

\begin{proposition}
	\proplbl{1_3_14}
	Sei $(G,\cdot)$ eine Gruppe und $\emptyset \neq H \subset G$. Genau dann ist
	$H$ eine Untergruppe von $G$, wenn sich die Verknüpfung $\cdot: G \times G \to G$ zu einer
	Abbildung $\cdot_H: H \times H \to H$ einschränken lässt (d.h. $\cdot\vert_{H \times H}=
	\iota_H \circ \cdot_H$, wobei $\iota_H \cdot \cdot_H \to G$ die Inklusionsabbildung ist) und
	$(H,\cdot_H)$ eine Gruppe ist.
\end{proposition}
\begin{proof}
	$\Rightarrow$: Sei $H$ eine Untergruppe von $G$. Nach (UG1) ist $\Image(\cdot|_{H \times H}) \subset H$
	und somit lässt sich $\cdot$ zu einer Abbildung $\cdot_H: H \times H \ to H$ einschränken. Wir 
	betrachten jetzt $H$ mit dieser Verknüpfung. Da $G$ (G1) erfüllt, erfüllt auch H (G1). Da
	$H \neq \emptyset$ existiert ein $x \in H$. Nach (UG1) und (UG2) ist $x \cdot x^{-1}=e \in H$. Da 
	$e_G \cdot y=y \cdot e_G=y$ für alle $y \in G$, insbesondere auch für alle $y \in H$ (G2). Wegen
	(UG2) erfüllt $H$ auch das Axiom (G3). $H$ ist somit eine Gruppe. \\
	$\Leftarrow$: Sei nun umgekehrt $(H,\cdot_H)$ eine Gruppe. Für $x,y \in H$ ist dann $xy=x \cdot_H
	y \in H$, also erfüllt $H$ (UG1). Aus $e_H \cdot e_H=e_H=e_H \cdot e_G$ folgt $e_H=e_G$. Ist also
	$x'$ das Inverse zu $x$ aus der Gruppe $H$, so ist $x'x=xx'=e_G=e_H$, also $x^{-1}=x' \in H$ und
	somit erfüllt $H$ auch (UG2). Wir haben gezeigt, dass $H$ eine Untergruppe von $G$ ist.
\end{proof}

\begin{remark}
	Wir nennen nicht nur die Menge $H$ eine Untergruppe von $G$, sondern auch die Gruppe $(H,\cdot_H)$.
	Wir schreiben $H \subseteq G$.
\end{remark}

\begin{example}
	\proplbl{1_3_16}
	\begin{itemize}
		\item Jede Gruppe $G$ hat die triviale Untergruppe $H=\{e_G\}$ und $H=G$
		\item Ist $H \subseteq G$ und $K \subseteq H$, so ist $K \subseteq G$ (Transitivität)
		\item Unter Addition ist $\mathbb{Z} \subseteq \mathbb{Q} \subseteq \mathbb{R}$ eine Kette von Untergruppen
		\item Unter Multiplikation ist $\mu_2 \subseteq \mathbb{Q}^+ \subseteq \mathbb{R}^+$ eine Kette von 
		Untergruppen
		\item Für $n \in \mathbb{N}_0$ ist $n\mathbb{Z} := \{nx \mid x \in \mathbb{Z}\} \subseteq \mathbb{Z}$ 
	\end{itemize}
\end{example}

\begin{lemma}
	\proplbl{1_3_17}
	Ist $G$ eine Gruppe und $(H_i)_{i \in I}$ eine Familie von Untergruppen von $G$,
	so ist auch $H := \bigcap H_i$ eine Untergruppe von $G$.
\end{lemma}
\begin{proof}
	Wir haben 3 Dinge zu zeigen
	\begin{itemize}
		\item $H \neq \emptyset:$ Für jedes $i \in I$ ist $e_G \in H$, also auch $e_G \in \bigcap
		H_i =H$
		\item (UG1): Seien $x,y \in H$. Für jedes $i \in I$ ist $x,y \in H_i$, somit $xy \in H_i$,
		da $H_i \subseteq G$. Folglich ist $xy \in \bigcap H_i=H$.
		\item (UG2): Sei $x \in H$. Für jedes $i \in I$ ist $x \in H_i$, somit $x^{-1} \in H_i$,
		da $H_i \subseteq G$. Folglich ist $x^{-1} \in \bigcap H_i=H$.
	\end{itemize}
\end{proof}

\begin{proposition}
	Ist $G$ eine Gruppe und $X \subset G$. so gibt es eine eindeutig bestimmte
	kleinste Untergruppe $H$ von $G$, die $X$ enthält, d.h. $H$ enthält $X$ und ist $H'$
	eine weitere Untergruppe von $G$, die $X$ enthält, so ist $H \subset H'$.
\end{proposition}
\begin{proof}
	Sei $\mathcal{H}$ die Menge aller Untergruppen von $G$, die $X$ enthalten. Nach \propref{1_3_17}
	ist $H:=
	\bigcap \mathcal{H} := \bigcap H$ eine Untergruppe von $G$. Da $X \subset H'$ für jedes $H' \in 
	\mathcal H$ ist auch $X \subset H$. Nach Definition ist $H$ in jedem $H' \subseteq G$ mit $X \subset H'$
	enhalten.
\end{proof}

\begin{definition}[erzeugte Untergruppe]
	Ist $G$ eine Gruppe und $X \subseteq G$, so nennt man diese
	kleinste Untergruppe von $G$, die $X$ enthält, die von $X$ \begriff[Untergruppe!]{erzeugte Untergruppe} von $G$ und
	bezeichnet diese mit $\langle X\rangle$, falls $X = \{x_1,x_2,...,x_n\}$ enthält auch mit $\langle x_1,x_2,
	...,x_n\rangle$. Gibt es eine endliche Menge $X \subset G$ mit $G=\langle X\rangle$, so nennt man $G$ endlich
	erzeugt.
\end{definition}

\begin{example}
	\begin{itemize}
		\item Die leere Menge $X=\emptyset \subseteq G$ erzeugt stets die triviale Untergruppe $\langle \emptyset\rangle
		=\{e\} \subseteq G$
		\item Jede endliche Gruppe $G$ ist endlich erzeugt $G=\langle G\rangle$
		\item Für $n \in \mathbb{N}_0$ ist $n\mathbb{Z}=\langle n\rangle \subseteq \mathbb{Z}$. Nach \propref{1_3_16} ist $n\in n\whole\subseteq\whole$. Ist $H \subseteq \mathbb{Z}$
		mit $n \in H$, so ist auch $kn=nk=n+n+...+n \in H$ und somit auch $n\mathbb{Z} \subseteq H$.
	\end{itemize}
\end{example}
