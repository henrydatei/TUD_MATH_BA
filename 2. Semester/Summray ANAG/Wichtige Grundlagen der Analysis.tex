\documentclass[ngerman,a4paper]{article}

\usepackage{amsmath}
\usepackage{amssymb}
\usepackage{enumitem}
\usepackage[left=2.1cm,right=3.1cm,bottom=3cm]{geometry}
\usepackage[ngerman]{babel}

\title{\textbf{Wichtige Methoden der Analysis}}
\author{H. Haustein, P. Lehmann}

\begin{document}
\maketitle

\section{Grenzwerte berechnen}
\begin{enumerate}[label=\textbf{\arabic*.}]
	\item Kann man die Grenze in die Funktion einsetzen und ausrechnen, ohne dass es zu Problemen kommt? 
	\item Geschicktes Ausklammern im Nenner, dann kürzen im Zähler.
	\item Regel von \textsc{l'Hospital} (mehrfach) verwenden, klappt aber nur, wenn Zähler und Nenner differenzierbar sind:
	\begin{align}
	\lim\limits_{x\to x_0}\frac{f(x)}{g(x)}=\lim\limits_{x\to x_0}\frac{f'(x)}{g'(x)}\notag
	\end{align}
\end{enumerate}

\section{Reihen}
\begin{enumerate}
	\item Cauchykrit. undersuche Differenz von aufeinanderfolgenden Partialsummen, müssen kleiner als $\epsilon$ sein (Konvergenz für Folgen eigentlich)
	\item Sei $X$ normierter Raum, $\{x_k\}$ in $X, k_0\in\mathbb{N}$
	\begin{enumerate}[label={\alph*)}]
		\item \emph{Majorantenkriterium} Sei $\{x_k\}$ Folge in $\mathbb{R}$
		\begin{enumerate}[label={\alph*)}]
			\item $\Vert x_k\Vert \le \alpha_k\,\forall k\ge k_0,\sum_k \alpha_k$ konvergent $\Rightarrow\;\sum_k \Vert x_k\Vert$ konvergent
			\item $0 \le \alpha_k \le \Vert x_k\Vert\,\forall k\ge k_0,\sum_k \alpha_k$ divergent $\Rightarrow\sum_k\Vert x_k\Vert$ divergent.
		\end{enumerate}
		\item \emph{Quotientenkriterium} Sei $x_k\neq 0\,\forall k\ge k_0$ 
		\begin{enumerate}[label={\alph*)}]
			\item $\frac{\Vert x_{k+1}\Vert}{\Vert x_k\Vert} \le q < 1\,\forall k\ge k_0 \;\Rightarrow\;\sum_k \Vert x_k\Vert$ konvergiert
			\item $\frac{\Vert x_{k+1}\Vert}{\Vert x_k\Vert}\,\forall k\ge k_0\;\Rightarrow \sum_k\Vert x_k\Vert$ divergiert.
		\end{enumerate}
		\item \emph{Wurzelkriterium}
		\begin{enumerate}[label={\alph*)}]
			\item $\sqrt[k]{\Vert x_k\Vert}\le q < 1\,\forall k\ge k_0\;\Rightarrow\;\sum_k\Vert x_k\Vert$ konvergiert
			\item $\sqrt[k]{\Vert x_k\Vert} \ge 1\,\forall k\ge k_0\;\Rightarrow\;\sum_k \Vert x_k\Vert$ divergent.
		\end{enumerate}
	\end{enumerate}
	\item \emph{Konvergenzradius} Habe Potenzreihe $\sum_{k=0}^\infty a_k(z-z_0)^k$ dann
	\begin{align}
		L:=\begin{cases} \limsup\limits_{n\rightarrow\infty} \sqrt[k]{|a_k|},&\text{falls existiert} \notag\\ \infty,&\text{sonst}\end{cases}\qquad R:=\frac{1}{L} \;(\text{mit }0 = \frac{1}{\infty}, \frac{1}{0} = \infty) \notag
	\end{align}
	$$
\end{enumerate}

\section{Stetigkeit}
\begin{enumerate}
	\item wenn funktioniert, Rechenregeln und Beispiele aus Vorlesung (elementare Funktionen sind stetig)
	\item Summen, Produkte, Komposition, Skalarmultiplikation von/mit stetigen Funktionen sind wieder stetig
	\item wenn Rechenregel nicht funktionieren, dann über folgenstetigkeit argumentieren
	\begin{align}
		f(x_n) \to f(x_0) \forall \text{ Folgen } x_n \to x_0 \text{ in } D\notag
	\end{align}
\end{enumerate}

\section{Partialbruchzerlegung}
\begin{enumerate}[label=\textbf{\arabic*.}]
	\item Bestimmung der Nullstellen des Nenner-Polynoms
	\item Umschreiben des Polynoms (mit 3 Nullstellen $n_1,n_2,n_3$):
	\begin{align}
		\frac{f}{(x-n_1)(x-n_2)(x-n_3)}=\frac{A}{x-n_1}+\frac{B}{x-n_2}+\frac{C}{x-n_3}\notag
	\end{align}
	\item kommt eine Nullstelle doppelt vor, so ergibt sich
	\begin{align}
		\frac{f}{(x-n_1)^2}=\frac{A}{x-n_1}+\frac{B}{(x-n_1)^2}\notag
	\end{align}
	\item bei komplexen Nullstellen:
	\begin{align}
		\frac{A}{a-ib-z}+\frac{B}{a+ib-z} \text{ in die Form } \frac{C+Dz}{(a-z)^2+b^2}\notag
	\end{align}
	\item Multiplikation beider Seiten mit $x-n_1$, Kürzen auf der linken Seite nicht vergessen!
	\item Einsetzen: $x=n_1$, Brüche mit $B$ und $C$ werden zu 0, linke Seite $= A$
	\item diesem Schritt mit $n_2$ und $n_3$ wiederholen
\end{enumerate}

\section{Ableitung}
\subsection{(normale) Ableitung}
\begin{enumerate}[label=\textbf{\arabic*.}]
	\item Rechenregeln verwenden:
	\begin{align}
		(f\pm g)' &= f'\pm g'\notag \\
		(cf)' &= c\cdot f'\notag \\
		(x^n)' &= nx^{n-1}\notag \\
		(fg)' &= f'\cdot g + f\cdot g' \notag \\
		\left(\frac{f}{g}\right)' &= \frac{f'\cdot g-f\cdot g'}{g^2}\notag \\
		f(g(x))' &= f'(g(x))\cdot g'(x)\notag \\
		(\ln f)' &= \frac{f'}{f}\notag
	\end{align}
	\item bei mehrdimensionalen Funktionen: komponentenweise ableiten
	\item affin lineare Funktionen sind diffbar $Ax+b$ (folgt aus Definition diffbar Kap. 17)
\end{enumerate}

\subsection{Richtungsableitung und partielle Ableitung}
\begin{enumerate}[label=\textbf{\arabic*.}]
	\item Berechnung der Richtungsableitung von $f$ in $x$ in Richtung $v$:
	\begin{align}
		\mathrm{D}_vf(x)=\lim\limits_{t\to 0}\frac{f(x+tv)-f(x)}{t}\notag
	\end{align}
	\item bei partieller Ableitung: Behandeln aller Variablen, die nicht abzuleiten sind, als Konstanten
\end{enumerate}

\section{Integration}

\subsection{partielle Integration}

\subsection{Integration durch Substitution}

\section{Extremwerte}

\end{document}