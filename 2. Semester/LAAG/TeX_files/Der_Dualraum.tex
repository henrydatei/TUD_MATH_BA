\section{Der Dualraum}

Sei $V$ ein $K$-Vektorraum.

\begin{definition}[Dualraum]
	Der \begriff{Dualraum} zu $V$ ist der $K$-Vektorraum
	\begin{align}
		V^*=\Hom_K(V,K)=\{\phi:V\to K\text{ linear}\}\notag
	\end{align}
	Die Elemente von $V^*$ heißen \begriff{Linearformen} auf $V$.
\end{definition}

\begin{example}
	Ist $V=K^n=\Mat_{n\times 1}(K)$, so wird $V^*=\Hom_K(V,K)$ durch $\Mat_{1\times n}(K)\cong K^n$. Wir können also die Elemente von $V$ als Spaltenvektoren und die Linearformen auf $V$ als Zeilenvektoren auffassen.
\end{example}

\begin{lemma}
	Ist $B(x_1)_{i\in I}$ eine Basis von $V$, so gibt es zu jedem $i\in I$ genau $x_i^*\in V^*$ mit $x_i^*(x_j)=\delta_{ij}\quad\forall j\in I$.
\end{lemma}
\begin{proof}
	Siehe LAAG1 III.5.1, angewandt auf die Familie $(y_j)_{j\in I}$, $y_j\delta_{i.j}$ in $W=K$. %TODO: Verlinkung
\end{proof}

\begin{proposition}
	Ist $B=(x_1)_{i\in I}$ eine Basis von $V$, so ist $B^*=(x_i^*)_{i\in I}$ linear unabhängig. Ist $I$ endlich, so ist $B^*$ eine Basis von $V^*$.
\end{proposition}
\begin{proof}
	Ist $\phi=\sum_{i\in I} \lambda_ix_i^*$, $\lambda_i\in K$, fast alle gleich 0, so ist $\phi(x_j)=\sum_{i\in I} \lambda_j x_i^*(x_j)=\lambda_j$ für jedes $j\in I$. Ist also $\phi=0$, so ist $\lambda_j=\phi(x_j)=0\quad\forall j\in I$, $B^*$ ist somit linear unabhängig. \\
	Ist zudem $I$ endlich und $\psi\in V^*$, so ist $\psi=\psi'=\sum_{i\in I} \psi(x_i)x_i^*$, denn $\psi'(x_j)=\sum_{i\in I} \psi(x_i)x_i^*(x_j)=\psi(x_i)\quad\forall j\in I$, und somit ist $B^*$ ein Erzeugendensystem von $V^*$.
\end{proof}

\begin{definition}[duale Basis]
	Ist $B=(x_i)_{i\in I}$ eine endliche Basis von $V$, so nennt man $B^*=(x_i^*)_{i\in I}$ die zu $B$ \begriff{duale Basis}.
\end{definition}

\begin{conclusion}
	Zu jeder Basis $B$ von $V$ gibt es einen eindeutig bestimmtem Monomorphismus
	\begin{align}
		f_V\to V^*\text{ mit } f(B)=B^*\notag
	\end{align}
	Ist $\dim_K(V)<\infty$, so ist dieser ein Isomorphismus.
\end{conclusion}

\begin{conclusion}
	Zu jedem $=0\neq x\in V$ gibt es eine Linearform $\phi\in V$ mit $\phi(x)=1$.
\end{conclusion}
\begin{proof}
	Ergänze $x_1=x$ zu einer Basis $(x_i)_{i\in I}$ von $V$ (\propref{3_1_11}) und $\phi=x_1^*$.
\end{proof}

\begin{example}
	Ist $V=K^n$ mit Standardbasis $\mathcal{E}=(e_1,...,e_n)$, so können wir $V^*$ mit dem Vektorraum der Zeilenvektoren identifizieren, und dann ist
	\begin{align}
		e_i^* = e_i^t\notag
	\end{align}
\end{example}

\begin{definition}[Bidualraum]
	Der \begriff{Bidualraum} zu $V$ ist der $K$-Vektorraum
	\begin{align}
		V^{**}=(V^*)^*=\Hom_K(V^*,K)\notag
	\end{align}
\end{definition}

\begin{proposition}
	Die kanonische Abbildung
	\begin{align}
		\iota:\begin{cases}
		V\to V^{**} \\ x\to \iota_x
		\end{cases}\text{ wobei } \iota_x(\phi)=\phi(x)\notag
	\end{align}
	ist ein Monomorphismus. Ist $\dim_K(V)<\infty$, so ist $\iota$ ein Isomorphismus.
\end{proposition}