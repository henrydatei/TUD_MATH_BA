\section{Das Lemma von Zorn}

Sei $K$ ein Körper und $U,V,W$ seien $K$-Vektorräume. Zudem sei $X$ eine Menge.

\begin{definition}[Relation]
	Eine \begriff{Relation} ist eine Teilmenge $R\subseteq X\times X$. Man schreibt $(x,x')\in R$ als $xRx'$. $R$ heißt
	\begin{itemize}
		\item \begriff[Relation!]{reflexiv}, wenn $\forall  x\in X$: $xRx$
		\item \begriff[Relation!]{transitiv}, wenn $\forall x,y,z\in X$: $xRy$ und $yRz\Rightarrow xRz$
		\item \begriff[Relation!]{symmetrisch}, wenn $\forall x,y\in X$: $xRy\Rightarrow yRx$
		\item \begriff[Relation!]{antisymmetrisch}, wenn $\forall x,y\in X$: $xRy$ und $yRx\Rightarrow y=x$
		\item \begriff[Relation!]{total}, wenn $\forall x,y\in X$: $(x,y)\notin R\Rightarrow (y,x)\in R$
	\end{itemize}
\end{definition}

\begin{definition}[Äquivalenzrelation]
	Eine \begriff{Äquivalenzrelation} ist eine reflexive, transitive und symmetrische Relation.
\end{definition}

\begin{definition}[Halbordnung]
	Eine \begriff{Halbordnung} ist eine reflexiv, transitive und antisymmetrische Relation. Eine totale Halbordnung heißt \begriff{Totalordnung} oder \begriff{lineare Ordnung}
\end{definition}

\begin{example}
	\begin{itemize}
		\item Die natürliche Ordnung auf $\real$, $\ratio$, $\whole$ und $\natur$.
		\item Teilbarkeit ist eine Halbordnung auf $\natur$, aber Teilbarkeit ist keine Halbordnung auf $\whole$, da $1\vert -1$ und $-1\vert 1$, aber $1\neq -1$!
		\item $\mathcal{P}(X)$ ist die Potenzmenge. "'$\subseteq$"' ist eine Halbordnung auf $\mathcal{P}$, aber für $\vert X\vert>1$ ist "'$\subseteq$"' keine Totalordnung.
		\item Sei $(X,\le)$ eine Halbordnung, sei $Y\subseteq X$, so ist $(Y,\subseteq\vert_Y)$ eine Halbordnung.
	\end{itemize}
\end{example}

\begin{definition}[Kette]
	Sei $(X,\le)$ eine Halbordnung, $Y\subseteq X$. $Y$ heißt \begriff{Kette}, wenn $(Y,\le\vert_Y)$ total ist.
	
	$x\in Y$ heißt ein \begriff[Kette!]{minimales Element} von $Y$, wenn $\forall x'\in Y$: $x<x'$.
	
	$x\in Y$ heißt \begriff[Kette!]{untere Schranke} von $Y$, wenn $\forall y\in Y$: $y\ge x$.
	
	$x\in Y$ heißt \begriff[Kette!]{kleinstes Element} von $Y$, wenn $x$ untere Schranke von $Y$ ist.
	
	Analog: \begriff[Kette!]{maximales Element}, \begriff[Kette!]{obere Schranke}, \begriff[Kette!]{größtes Element}.
\end{definition}

\begin{center}
	\begin{tikzpicture}
		\node[place] (A) {1};
		\node[place] (B) [below=of A] {3};
		\node[place] (C) [left=of B] {2};
		\node[place] (D) [right=of B] {5};
		\node[place] (E) [right=of D] {7};
		\node[place] (dots) [right=of E] {...};
		
		\node[place] (F) [below=of C] {4};
		\node[place] (G) [below=of B] {6};
		\node[place] (H) [below=of E] {15};
		\node[place] (I) [right=of G] {10};
		\node[place] (dots2) [right=of H] {...};
		
		\node[place] (J) [below=of F] {8};
		\node[place] (K) [below=of J] {16};
		\node[place] (L) [below=of K] {32};
		
		\draw[->,thick] (A.west) -- (C.north);
		\draw[->,thick] (A.south) -- (B.north);
		\draw[->,thick] (A.east) -- (D.north);
		\draw[->,thick] (A.east) -- (E.north);
		\draw[->,thick] (A.east) -- (dots.north);
		
		\draw[->,thick] (C.south) -- (F.north);
		\draw[->,thick] (C.south) -- (G.north);
		\draw[->,thick] (B.south) -- (G.north);
		\draw[->,thick] (C.south) -- (I.north);
		\draw[->,thick] (D.south) -- (I.north);
		\draw[->,thick] (B.south) -- (H.north);
		\draw[->,thick] (D.south) -- (H.north);
		\draw[->,thick] (E.south) -- (dots2.north);
		\draw[->,thick] (D.south) -- (dots2.north);
		
		\draw[->,thick] (F.south) -- (J.north);
		\draw[->,thick] (J.south) -- (K.north);
		\draw[->,thick] (K.south) -- (L.north);
	\end{tikzpicture}
	$Y=\{2^n\mid n\in\natur\}$ ist eine Kette
\end{center}

\begin{remark}
	\begin{itemize}
		\item Hat $Y$ ein kleinstes Element, so ist dies eindeutig bestimmt. Ein kleinstes Element ist minimal.
		\item Jede endliche Halbordnung hat minimale Elemente. Jede endliche Totalordnung hat ein kleinstes Element. Analog für maximale Elemente und größtes Element.
	\end{itemize}
\end{remark}

\begin{example}
	$(\natur,\le)$ hat als kleinstes Element die 1, aber kein größtes Element oder maximale Elemente.
\end{example}

\begin{example}
	$V=\real^3$, $\mathcal{X}$ die Menge der Untervektorräume des $\real^3$. $(\mathcal{X},\le)$ ist eine Halbordnung auf $Y\subseteq X$ mit $Y=\{U\in\mathcal{X}\mid \dim_\real(U)\le 2\}$. 
	\begin{itemize}
		\item $Y$ hat ein kleinstes Element: $\{0\}$.
		\item Es gibt unendlich viele maximale Elemente in $Y$, nämlich die Untervektorräume von $V$, die die Dimension 2 haben. Es gibt also kein größtes Element.
		\item $V$ ist die obere Schranke von $Y$.
	\end{itemize}
\end{example}

\begin{theorem}[Das Lemma von Zorn]
	Sei $(X,\le)$ eine Halbordnung, die nicht leer ist. Wenn jede Kette eine obere Schranke hat, dann hat $X$ ein maximales Element.
\end{theorem}
\begin{proof}
	Dieses Theorem ist äquivalent zum Auswahlaxiom. \frownie{}
\end{proof}

\begin{conclusion}
	Zu jeder Familie $(x_i)$, nicht leer, gibt es eine \begriff{Auswahlfunktion}, das heißt eine Abbildung:
	\begin{align}
		f: I\to \bigcup X_i\text{ mit } f(i)\in X_i\quad\forall i\notag
	\end{align}
\end{conclusion}
