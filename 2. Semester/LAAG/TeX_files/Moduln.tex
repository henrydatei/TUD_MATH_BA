In diesem ganzen Kapitel sei $R$ ein kommutativer Ring mit Einselement.

\section{Moduln}

\begin{definition}
	Ein $R$-\begriff{Modul} ist ein Tripel $(M,+,\cdot)$ bestehend aus einer Menge $M$, einer Verknüpfung $+:M\times M\to M$ und der Abbildung $\cdot:R\times M\to M$ (Skalarmultiplikation) für die gelten:
	\begin{itemize}
		\item (M1): $(M,+)$ ist eine abelsche Gruppe
		\item (M2): Addition und Skalarmultiplikation sind verträglich. Für alle $x,y\in M$ und $a,b\in R$ gelten
		\begin{itemize}
			\item $a(x+y)=ax+ay$
			\item $(a+b)x=ax+bx$
			\item $a\cdot bx=ab\cdot x$
			\item $1\cdot x=x$
		\end{itemize}
	\end{itemize}
\end{definition}

\begin{example}
	\begin{itemize}
		\item Ist $R=K$ ein Körper, so sind die $R$-Moduln genau die $K$-Vektorräume.
		\item Ist $R=\whole$, so sind die $R$-Moduln genau die abelschen Gruppen mit der einzig möglichen Skalarmultiplikation 
		\begin{align}
			\whole\times A\to A,(k,a)\mapsto ka=\underbrace{1+...+1}_{k\text{-mal}}a=\underbrace{a+...+a}_{k\text{-mal}}\notag
		\end{align}
		vergleiche Laag 1 III.2.3 %TODO: Verlinkung
		\item Jedes Ideal $M\subseteq R$ ist ein $R$-Modul mit Einschränkung der Multiplikation als Skalarmultiplikation.
		\item Ist $K$ ein Körper, $V$ ein $K$-Vektorraum und $f\in\End_K(V)$, so wird $V$ durch $P(t)\cdot x:=P(f)(x)$ zu einem Modul über dem Ring $R=K[t]$, siehe auch V.5.2 %TODO: Verlinkung
	\end{itemize}
\end{example}

\begin{remark}
	Sei $M$ ein $R$-Modul. Wie für Vektorräume (LAAG 1 II.1.5) überzeugt man sich leicht, dass $0x=0$, $a0=0$, $(-a)x=a(-x)=-ax$ für alle $a\in R$, $x\in M$. 
	
	Im Gegensatz zu Vektorräumen folgt aber aus $ax=0$ nicht, dass $a=0$ oder $x=0$, siehe zum Beispiel das $\whole$-Modul $M=\whole/n\whole$. Es ist
	\begin{align}
		n\cdot\overline{1}=\overline{n}=\overline{0}\in\whole/n\whole\notag
	\end{align}
	aber $0\neq n\in\whole$.
\end{remark}

\begin{definition}[Homomorphismus von $R$-Moduln]
	Seien $M,M'$ $R$-Moduln. Eine Abbildung $f:M\to M'$ ein \begriff[Modul!]{Homomorphismus} von $R$-Moduln (oder $R$-Homomorphismus oder $R$-linear), wenn
	\begin{align}
		f(x+y)&=f(x)+f(y) \notag \\
		f(ax) &= a\cdot f(x)\notag
	\end{align}
	Wir bezeichnen die Menge der $R$-Homomorphismen $f:M\to M'$ mit $\Hom_R(M,M')$. Wie üblich definiert man den \begriff[Modul!]{Kern} eines $R$-Homomorphismus, sowie die Begriffe \begriff[Modul!]{Monomorphismus}, \begriff[Modul!]{Epimorphismus}, \begriff[Modul!]{Isomorphismus}, \begriff[Modul!]{Endomorphismus} und \begriff[Modul!]{Automorphismus} von $R$-Moduln.
\end{definition}

\begin{example}
	\begin{itemize}
		\item Ist $R=K$, so sind die $R$-Homomorphismen genau die lineare Abbildungen.
		\item Ist $R=\whole$, so sind die $R$-Homomorphismen genau die Gruppenhomomorphismen.
	\end{itemize}
\end{example}

\begin{example}
	Für jedes $a\in R$ ist die Abbildung
	\begin{align}
		\begin{cases}
		M\to M \\ x\mapsto ax
		\end{cases}\notag
	\end{align}
	einen Endomorphismus von $M$.
\end{example}

\begin{definition}[Untermodul, Erzeugendensystem]
	Ein \begriff{Untermodul} ist eine nichtleere Teilmenge $N\subseteq M$, für die gilt:
	\begin{itemize}
		\item Sind $x,y\in N$, so ist auch $x+y\in N$.
		\item Ist $a\in R$ und $x\in N$, so ist auch $ax\in N$.
	\end{itemize}

	Für eine Familie $(x_i)_{i\in I}$ ist
	\begin{align}
		\sum_{i\in I} Rx_i=\{\sum_{i\in I} ax_i\mid a\in R\text{, fast alle gleich 0}\}\notag
	\end{align}
	der von $(x_i)_{i\in I}$ \begriff[Untermodul!]{erzeugte Untermodul} von $M$. Ist $\sum_{i\in I} Rx_i=M$, so ist $(x_i)_{i\in I}$ ein \begriff[Modul!]{Erzeugendensystem} von $M$. Der $R$-Modul $M$ ist \begriff[Modul!]{endlich erzeugt}, wenn er ein endliches Erzeugendensystem besitzt.
\end{definition}

\begin{remark}
	Wieder ist der Kern eines $R$-Homomorphismus $f:M\to M'$ ein Untermodul von $M$. Leicht sieht man auch hier, dass $\sum_{i\in I} Rx_i$ ein Untermodul von $M$ ist, und zwar der kleinste, der alle $x_i$ enthält.
\end{remark}

\begin{example}
	\begin{itemize}
		\item Ist $R=K$ ein Körper, so sind die Untermoduln von $M$ genau die Untervektorräume.
		\item Ist $R=\whole$, so sind die Untermoduln von $M$ genau die Untergruppen und der von einer Familie erzeugte Untermodul ist genau gleich der davon erzeugten Untergruppe. \\
		Ist zum Beispiel $M=\whole$, so sind alle $n\whole$ Untermoduln von $M$.
	\end{itemize}
\end{example}

\begin{definition}[freie Familie, Basis]
	Eine Familie $(x_i)_{i\in I}$ in $M$ ist \begriff[Familie!]{frei} oder ($R$-linear unabhängig), wenn es keine Familie $(\lambda_i)_{i\in I}$ von Elementen von $R$, fast alle gleich 0, aber nicht alle gleich 0, mit $\sum_{i\in I} \lambda_ix_i=0$ gibt.
	
	Ein freies Erzeugendensystem heißt \begriff[Modul!]{Basis}. Besitzt $M$ eine Basis, so nennt man $M$ \begriff[Modul!]{frei}.
\end{definition}