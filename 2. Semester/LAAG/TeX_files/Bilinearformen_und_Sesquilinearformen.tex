\section{Bilinearformen und Sesquilinearformen}

Sei $K=\real$ oder $K=\comp$.

\begin{definition}[Bilinearform, Sesquilinearform]
	Eine \begriff{Bilinearform} ($K=\real$) bzw. \begriff{Sesquilinearform} ($K=\comp$) ist eine Abbildung $s:V\times V\to K$ für die gilt:
	\begin{itemize}
		\item Für $x,x',y\in V$ ist $s(x+x',y)=s(x,y)+s(x',y)$
		\item Für $x,y,y'\in V$ ist $s(x,y+y')=s(x,y)+s(x,y')$
		\item Für $x,y\in V$, $\lambda\in K$ ist $s(\lambda x,y)=\lambda s(x,y)$
		\item Für $x,y\in V$, $\lambda\in K$ ist $s(x,\lambda y)=\kringel{white}{\overline{\lambda}} s(x,y)$
	\end{itemize}
\end{definition}

\begin{remark}
	Im Fall $K=\real$ ist $\lambda=\overline{\lambda}$. Wir werden der Einfachheit halber auch in diesem Fall von Sesquilinearformen sprechen , vgl. \propref{2_1_12}
\end{remark}

\begin{example}
	Für $A=(a_{ij})_{i,j}\in\Mat_n(K)$ ist $s_A:K^n\times K^n\to K^n$ gegeben durch
	\begin{align}
		s_A(x,y)=x^tA\overline{y}=x^t\left( \sum_{j=1}^n a_{ij}\overline{y}_j\right)_i=\sum_{i,j=1}^n a_{ij}x_i\overline{y}_j\notag
	\end{align}
	eine Sesquilinearform auf $V=K^n$.
\end{example}

\begin{definition}
	Sei $s$ eine Sesquilinearform auf $V$ und $B=(v_1,...,v_n)$ eine Basis von $V$. Die \begriff[Sesquilinearform!]{darstellende Matrix} von $s$ bzgl. $B$ ist
	\begin{align}
		M_B(s)=(s(v_i,v_j))_{i,j}\in\Mat_n(K)\notag
	\end{align}
\end{definition}

\begin{example}
	Die darstellende Matrix des Standardskalarprodukts $s=s_{\mathbbm{1}_n}$ auf den Standardraum $V=K^n$ bzgl. der Standardbasis $\mathcal{E}$ ist
	\begin{align}
		M_{\mathcal{E}}(s)=\mathbbm{1}_n\notag
	\end{align}
\end{example}