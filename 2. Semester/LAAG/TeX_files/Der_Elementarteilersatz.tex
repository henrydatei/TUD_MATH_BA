\section{Der Elementarteilersatz}

Sei $R$ Hauptidealring.

\begin{definition}
	Seien $a,b,x,y\in R$. Für $i,j\in\{1,...,n\}$ ist
	\begin{align}
		E_{ij} = (\delta_{\sigma,i},...,\delta_{\mu,j})_{\sigma,\mu}\in\Mat_n(\real)\notag
	\end{align}
	Sei
	\begin{align}
		E_{ij}(a,b,x,y) = \mathbbm{1}_n-E_{ii}-E_{jj}+aE_{ii}+bE_{ij}+xE_{jj}+yE_{ji}\notag
	\end{align}
	%TODO: Matrix ergänzen von Pascal
\end{definition}

\begin{lemma}
	Ist $ax-by\in R^\times$, so ist
	\begin{align}
		E_{ij}(a,b,x,y)\in \GL_n(\real)\notag
	\end{align}
\end{lemma}
\begin{proof}
	Folgt aus LAAG1 IV.3.4, da
	%TODO: Verlinkung
	\begin{align}
		\det(E_{ij}(a,b,x,y))=ax-by\in R^\times\notag
	\end{align}
	Oder direkt: Das Inverse ist $E_{ij}(xc^{-1},bc^{-1}, ac^{-1},-yc^{-1})$, zum Beispiel
	\begin{align}
		\begin{pmatrix} a & b \\ y & x\end{pmatrix}\begin{pmatrix}xc^{-1} & -bc^{-1} \\ -yc^{-1} & ac^{-1}\end{pmatrix}=\begin{pmatrix}(ax-by)c^{-1} & 0 \\ 0 & (ax-by)c^{-1}\end{pmatrix}\notag
	\end{align}
\end{proof}

\begin{remark}
	Multiplikation mit $E_{ij}(a,b,x,y)$ von links an $(a_1,...,a_n)^t\in\Mat_n(\real)$
	%TODO: konnte den Rest nicht erkennen, noch ergänzen
\end{remark}

\begin{theorem}[Elementarteilersatz für Matrizen]
	Sei $A\in\Mat_{m\times n}(\real)$. Es gibt
	\begin{align}
		0\le r \le\min\{n,m\}\notag \\
		S\in\GL_m(\real)\notag \\
		T\in\GL_n(\real)\notag
	\end{align}
	mit 
	\begin{align}
		SAT &= \diag(d_1,...,d_r,Q) \notag \\
		0&=Q\in\Mat_{m-r\times n-r}\notag
	\end{align}
	wobei $d_i\in R\backslash\{0\}$ mit $d_i\mid d_{i+1}$ für $i=1,...,n-1$
\end{theorem}