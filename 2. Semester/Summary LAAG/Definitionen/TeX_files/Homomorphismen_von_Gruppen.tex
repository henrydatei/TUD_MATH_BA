\section{Homomorphismen von Gruppen}

Seien $G,H$ zwei multiplikativ geschriebene Gruppen.

\begin{definition}[Gruppenhomomorphismus]
	Eine Abbildung $f: G \to H$ ist ein \begriff{Gruppenhomomorphismus}, wenn gilt:
	\begin{itemize}
		\item (GH): $f(xy)=f(x)\cdot f(y)$
	\end{itemize}
	Die Menge der Homomorphismen $f:G\to H$ bezeichnet man mit $\Hom(G,H)$.
\end{definition}

\begin{definition}[Arten von Homomorphismen]
	Ein Homomorphismus ist
	\begin{itemize}
		\item ein \begriff{Monomorphismus}, wenn $f$ injektiv ist
		\item ein \begriff{Epimorphismus}, wenn $f$ surjektiv ist
		\item ein \begriff{Isomorphismus}, wenn $f$ bijektiv ist.
	\end{itemize}
	Die Gruppen $G$ und $H$ heißen \begriff{isomorph}, in Zeichen $G\cong H$, wenn 
	es einen Isomorphismus $G\to H$ gibt.
\end{definition}

\begin{definition}[Kern]
	Der \begriff{Kern} eines Gruppenhomomorphismus $f:G\to H$ ist $\Ker(f):= f^{-1}(\{1\})=\{x\in G \mid
	f(x)=1_H\}$.
\end{definition}

\begin{definition}[Normalteiler]
	Ist $N\le G$ mit $x^{-1}y\in N$ für alle $x\in G$ und $y\in N$, so nennt man $N$ 
	einen \begriff{Normalteiler} von $G$ und schreibt $N\vartriangleleft G$.
\end{definition}