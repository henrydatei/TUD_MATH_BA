\section{Diagonalisierbarkeit}

\begin{definition}[diagonalisierbar]
	Man nennt $f$ \begriff{diagonalisierbar}, wenn $V$ eine Basis $B$ besitzt, für die $M_B(f)$ eine Diagonalmatrix ist.
\end{definition}

\begin{definition}[$a$ teilt $b$]
	Sei $R$ ein kommutativer Ring mit seien $a,b\in R$. Man sagt, $a$ \begriff{teilt} $b$ (in Zeichen $a\mid b$), wenn es $x\in R$ mit $b=ax$ gibt.
\end{definition}

\begin{definition}[Vielfachheit]
	Für $0\neq P\in K[t]$ und $\lambda\in K$ nennt man $\mu(P,\lambda)=\max\{r\in \natur_{>0}\mid (t-r)^r\mid P\}$ die \begriff{Vielfachheit} der Nullstelle $\lambda$ von $P$.
\end{definition}

\begin{definition}[algebraische und geometrische Vielfachheit]
	Man nennt $\mu_a(f,\lambda)=\mu(\chi_f,\lambda)$ die \begriff[Vielfachheit!]{algebraische Vielfachheit} und $\mu_g(f,\lambda)=\dim_K(\Eig(f,\lambda))$ die  \begriff[Vielfachheit!]{geometrische Vielfachheit} des Eigenwertes $\lambda$ von $f$.
\end{definition}