\section{Das Minimalpolynom}

\begin{definition}
	Für ein Polynom $P(t)=\sum_{i=0}^n c_it^i\in K[t]$ definieren wir $P(f)=\sum_{i=0}^m c_if^i\in\End_K(V)$, wobei $f^0=\id_V$, $f^1=f$, $f^2=f\circ f$, ...
	
	Analog definiert man $P(A)$ für $A\in\Mat_n(K)$.
\end{definition}

\begin{definition}[Minimalpolynom]
	Das eindeutig bestimmte normierte Polynom $0\neq P\in K[t]$ kleinsten Grades mit $P(f)=0$ nennt man das \begriff{Minimalpolynom} $P_f$ von $f$.
	
	Analog definiert man das Minimalpolynom $P_A\in K[t]$ einer Matrix $A\in\Mat_n(K)$.
\end{definition}

\begin{definition}[$f$-zyklisch]
	Ein $f$-invarianter UVR $W\le V$ heißt $f$-\begriff{zyklisch}, wenn es ein $x\in W$ mit $W=\Span_K(x,f(x),f^2(x),...)$ gibt.
\end{definition}