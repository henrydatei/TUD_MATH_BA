\section{Definition und Beispiele}

In diesem Kapitel sei $K$ ein Körper.

\begin{definition}[Vektorraum]
	Ein $K$-\begriff{Vektorraum} (auch Vektorraum über $K$) ist ein Tripel $(V,+,\cdot)$ 
	bestehend aus einer Menge $V$, einer Verknüpfung $+: V \times V \to V$, genannt Addition, und einer Abbildung 
	$\cdot: K \times V \to V$, genannt Skalarmultiplikation, für die gelten:
	\begin{itemize}
		\item (V1): $(V,+)$ ist eine abelsche Gruppe
		\item (V2): Addition und Skalarmultiplikation sind verträglich:
		\begin{itemize}
			\item $\lambda(x+y)=(\lambda\cdot x)+(\lambda\cdot y)$
			\item $(\lambda+\mu)\cdot x = (\lambda\cdot x)+(\mu\cdot x)$
			\item $\lambda(\mu\cdot x)=(\lambda\cdot\mu)\cdot x$
			\item $1\cdot x = x$
		\end{itemize}
	\end{itemize}
\end{definition}

\begin{definition}[Untervektorraum]
	Sei $V$ ein $K$-Vektorraum. Ein \begriff{Untervektorraum} (Untervektorraum) von $V$ ist eine nichtleere
	Teilmenge $W \subseteq V$ mit:
	\begin{itemize}
		\item (UV1): Für $x,y \in W$ ist $x+y\in W$.
		\item (UV2): Für $x \in W$ und $\lambda \in K$ ist $\lambda\cdot x\in W$.
	\end{itemize}
	
\end{definition}

\begin{definition}[Erzeugendensystem]
	Ist $V$ ein $K$-Vektorraum und $X\subseteq V$, so nennt man den kleinsten Untervektorraum von 
	$V$, der $X$ enthält den von $X$ erzeugten Untervektorraum von $V$ und bezeichnet diesen mit $\langle X\rangle$. Eine Mengen $X\subseteq V$ 
	mit $\langle X\rangle=V$ heißt \begriff{Erzeugendensystem} von $V$. Der Vektorraum $V$ heißt endlich erzeugt, wenn er ein endliches Erzeugendensystem 
	besitzt.
\end{definition}