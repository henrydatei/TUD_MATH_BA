\section{Orthogonalität}

Sei $V$ ein euklidischer bzw. unitärer Vektorraum.

\begin{definition}[orthogonal, orthogonales Komplement]
	Zwei Vektoren $x,y\in V$ heißen \begriff{orthogonal}, in Zeichen $x\perp y$, wenn $\skalar{x}{y}=0$. Zwei Mengen $X,Y\subseteq V$ sind \emph{orthogonal}, in Zeichen $X\perp Y$, wenn $x\perp y$ für alle $x\in X$ und $y\in Y$.
	
	Für $U\subseteq V$ bezeichnet 
	\begin{align}
		U^{\perp}=\{x\in V\mid x\perp u\text{ für alle } u\in U\}\notag
	\end{align}
	das \begriff{orthogonale Komplement} zu $U$.
\end{definition}

\begin{definition}[orthonormal]
	Eine Familie $(x_i)_{i\in I}$ von Elementen von $V$ ist \emph{orthogonal}, wenn $x_i\perp x_j$ für alle $i\neq j$, und \begriff{orthonormal}, wenn zusätzlich $\Vert x_i\Vert=1$ für alle $i$. Eine orthogonale Basis nennt man eine \emph{Orthogonalbasis}, eine orthonormale Basis nennt man eine \emph{Orthonormalbasis}.
\end{definition}