\section{Summen von Vektorräumen}

Sei $V$ ein $K$-Vektorraum und $(W_i)$ eine Familie von Untervektorräumen von $V$.

\begin{definition}[Summe von Vektorräumen]
	Die \begriff[Vektorraum!]{Summe} der $W_i$ ist der Untervektorraum
	\begin{align}
		\sum_{i\in I} W_i := \Span_K\left(\bigcup W_i\right)\notag
	\end{align} 
	Im Fall $I=\{1,...,n\}$ schreibt man auch $W_1+...+W_n$ für $\sum_{i=1}^n W_i$. 
\end{definition}

\begin{definition}[direkte Summe]
	\proplbl{2_4_5}
	Ist jedes $x\in \sum W_i$ eindeutig als Summe von $x_i$ mit $x_i\in W_i$ 
	darstellbar, so sagt man, dass $\sum W_i$ die \begriff{direkte Summe} der Untervektorräume $W_i$ ist und schreibt $\oplus W_i$ für 
	$\sum W_i$. Im Fall $I=\{1,...,n\}$ schreibt man auch $W_1\oplus W_2 \oplus ... \oplus W_n$ für $\oplus W_i$.
\end{definition}

\begin{theorem}[Dimensionsformel]
	\proplbl{2_4_12}
	Sei $\dim_K(V)<\infty$. Für Untervektorräume $W_1,W_2$ von $V$ gilt:
	\begin{align}
		\dim_K(W_1+W_2) + \dim_K(W_1 \cap W_2) = \dim_K(W_1) + \dim_K(W_2)\notag
	\end{align}
\end{theorem}
\begin{proof}
	Da $\dim_K(V)<\infty$ haben alle Untervektorräume von $V$ Basen. Sei also $B_0=(X_1,...,x_n)$ eine Basis von $W_1\cap W_2$. Nach 
	dem Basisergänzungssatz (\propref{2_3_12}) können wir $B_0$ zu den Basen $B_1=(x_1,...,x_n,y_1,...,y_p)$ von $W_1$ und $B_2=(x_1,...,
	x_n,z_1,...,z_q)$ von $W_2$ ergänzen. Wir behaupten, dass $B=(x_1,...,x_n,y_1,...,y_p,z_1,...,z_q)$ eine Basis von 
	$W_1+W_2$ ist. Offenbar ist $B$ ein Erzeugendensystem von $W_1+W_2$. Seien nun $\lambda_1,...,\lambda_n,\mu_1,...,
	\mu_p,\eta_1,...,\eta_q \in K$ mit $\sum_{i=1}^n \lambda_ix_i + \sum_{j=1}^p \mu_jy_j + \sum
	_{k=1}^q \eta_kz_k=0$. Dann ist $\sum_{i=1}^n \lambda_ix_i + \sum_{j=1}^p \mu_jy_j = -\sum
	_{k=1}^q \eta_kz_k \in W_1 \cap W_2$. Da $\Span_K(B_0)=W_1\cap W_2$ und $B_1$ linear unabhängig ist, ist 
	$\mu_j=0$. Analog zeigt man auch, dass $\eta_k=0$. Aus $B_0$ linear unabhängig folgt dann auch, dass $\lambda_i=0$. 
	Somit ist $B$ linear unabhängig. Wir haben gezeigt, dass $B$ eine Basis von $W_1+W_2$ ist. \\
	$\Rightarrow \dim_K(W_1)+\dim_K(W_2)=|B_1|+|B_2|=(n+p)+(n-q)=(n+p+q)+n=|B|+|B_0|=\dim_K(W_1+W_2)+\dim_K(W_1\cap W_2)$.
\end{proof}

\begin{definition}[externes Produkt]
	Das \begriff{externe Produkt} einer Familie $(V_i)$ von $K$-Vektorräumen ist der $K$-Vektorraum 
	$\prod V_i$ bestehend aus dem kartesischen Produkt der $V_i$ mit komponentenweiser Addition und 
	Skalarmultiplikation, $(x_i)+(x'_i) := (x_i+x'_i)$ und $\lambda(x_i) := (\lambda x_i).$
\end{definition}

\begin{definition}[externe Summe]
	Die \begriff{externe Summe} einer Familie $(V_i)$ von $K$-Vektorräumen ist der Untervektorraum 
	$\oplus V_i := \{(x_i) \in \prod V_i \mid x_i=0 \text{; für fast alle }i\}$ des $K$-Vektorraum $\prod V_i$.
\end{definition}