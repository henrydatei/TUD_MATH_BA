\section{Mannigfaltigkeiten}
\begin{definition}
	Eine \begriff*{topologische Mannigfaltigkeit}\index{Mannigfaltigkeit!topologisch} von Dimension $n\in\N$ ist ein zweitabzählbarer Haudroff-Raum $M$ mit der Eigenschaft, dass jedes $p\in M$ eine offene Umgebung $p\in U\subset M$ hat, die homöomorph zu $\R^n$ ist (d.h. $\exists x\colon U\to\R^n$, stetig, bijektiv, $x^{-1}\colon \R^n\to U$ auch stetig).
\end{definition}

\begin{remark}
	Da $\R^n\cong B(0,1)\subset\R^n$, könnte man auch $B(0,1)$ oder beliebige offene Teilmengen vom $\R^n$ statt $\R^n$ verwenden. Das führt auf äquivalente Definitionen.
\end{remark}

\begin{definition}
	Sei $M$ topologische Mannigfaltigkeit von Dimension $n$. Ein \begriff*{differenzierbarer Atlas}\index{Atlas} $\mathcal A$ auf $M$ ist eine Familie $\mathcal A = \{(U,x)\mid U\subset M\;\text{offen},\,x\colon U\xrightarrow{\cong}\R^n\;\text{Homöomorphismus}\}$ mit folgenden Eigenschaften: \begin{enumerate}[label={(\arabic*)}]
		\item die $U$ überdecken $M$: $\bigcup_{(U,x)\in\mathcal A} U = M$,
		\item Wenn $(U,x)$, $(V,y)\in\mathcal A$, $U\cap V\neq \emptyset$, dann gilt: \begin{align*}
			y\circ x^{-1}\colon x(U\cap V)\subset\R^n\to y(U\cap V)\subset\R^n
		\end{align*}
		ist glatt.
	\end{enumerate}
	Elemente eines Atlas' heißen \begriff*{Karten}\index{Karte}.
\end{definition}

\begin{definition}
	Zwei Atlanten $\mathcal A$, $\mathcal A'$ heißen \begriff*{äquivalent}\index{Atlas!äquivalent}, wenn $\mathcal A\cup\mathcal A'$ ein Atlas ist.
	
	Äquivalent: $\forall (U,x)\in \mathcal A$, $(V,y)\in\mathcal A'$ ist $y\circ x^{-1}\colon x(U\cap V)\to y(U\cap V)$ glatt.
\end{definition}

\begin{definition}
	Eine \begriff*{glatte Mannigfaltigkeit}\index{Mannigfaltigkeit!glatt} $M$ ist eine topologische Mannigfaltigkeit zusammen mit einer Äquivalenzklasse von Atlanten (diese wird auch \begriff*{glatte Struktur} genannt).
\end{definition}

\begin{example}
	$\R$ ist eine 1-dimensionale topologische Mannigfaltigkeit: \begin{itemize}
		\item $\mathcal A := \{ (R,\id\colon \R\to \R) \}$
		\item $\mathcal A' = \{ (\R,\sqrt[3]{\,\cdot\,}\colon\R\to\R \}$.
	\end{itemize}
	Diese Atlanten sind nicht äquivalent.
\end{example}

\begin{example}
	\begin{enumerate}[label={(\arabic*)}]
		\item $\R^n$: $\mathcal A = \{ (\R^n,\id) \}$ ist eine $n$-dimensionale Mannigfaltigkeit.
		\item $M$ Mannigfaltigkeit, $U\subset M$ offen. Dann ist auch $U$ eine Mannigfaltigkeit (schneide $U$ mit allen Kartenumgebungen).
		\item Ist $V$ ein $\R$-Vektorraum der Dimension $n$, dann ist $V$ eine $n$-dimensionale Mannigfaltigkeit. Die Wahl einer Basis definiert eine Karte $V\to\R^n$; und zwei solche Karten sind kompatibel, weil die Vergleichsabbildung durch Multiplikation mit der Basiswechselmatrix gegeben ist:
		
		$S^n = \{x\in\R^{n+1}\mid \Vert x\Vert_2 = 1\}\subset\R^{n+1}$. Sei $V_i^{\pm} = \{x\in S^n\mid \pm x_i > 0\}\subset S^n$ ist offen, $\bigcup_{i=1}^{n+1} (V_i\cup V_i^-) = S^n$. Dann ist \begin{align*}
			\pi_i\colon V_i^{\pm}\to B(0,1)\subset \R^n,\;(x_1,\dots,x_{n+1})\mapsto (x_1,\dots,\hat x_i,\dots,x_{n+1})
		\end{align*}
		ein Homöomorphismus mit der Umkehrabbildung \begin{align*}
			(x_1,\dots, x_i,\dots,x_{n+1})\mapsto (x_1,\dots,x_{i-1},\pm\sqrt{1-\sum_{j=1}^n x_j^2},x_{i+1},\dots,x_n).
		\end{align*}
		Was ist $\pi_i\circ \pi_j^{-1}$ wo es definiert ist?: \begin{align*}
			(x_1,\dots,\hat x_j,\dots,x_{n+1})\mapsto (x_1,\dots,\hat x_i,\dots,\pm\sqrt{1-\sum_{k\neq j} x_k^2},\dots,x_{n+1}).
		\end{align*}
		\item $\GL(n,\R) = \{ A\subset \Mat_n(\R)\mid \det A\neq 0\}\subset \R^{n^2}$ offen, ist Mannigfaltigkeit von Dimension $n^2$.
		\item $(M,\mathcal A_M)$, $(N,\mathcal A_n)$ Mannigfaltigkeiten. Dann ist auhc $(M\times N, \mathcal A_M\times \mathcal A_N)$ Mannigfaltigkeit.
	\end{enumerate}
\end{example}