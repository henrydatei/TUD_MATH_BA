\section{Sigma-Algebren}

\textbf{Ziel:} Charakterisierung der Definitionsgebiete von Maßen.

\begin{definition}[$\sigma$-Algebra, messbar]
	Eine \begriff{$\sigma$-Algebra} über einer beliebigen Grundmenge $X \neq \emptyset$ ist eine Familie von Mengen in $\mathcal{P}(X), \mathcal{A} \subset \mathcal{P}(X)$:
	\begin{itemize}
		\item (S1): $X \in \mathcal{A}$
		\item (S2): $A \in \mathcal{A} \to A^C = X \setminus A \in \mathcal{A}$
		\item (S3): $(A_n)_{n\in \natur} \subset \mathcal{A} \Rightarrow \bigcup_{n\in \natur}A_n \in \mathcal{A}$
	\end{itemize}
	Eine Menge $A\in\mathcal{A}$ heißt \begriff{messbar}.
\end{definition}

\begin{proposition}[Eigenschaften einer $\sigma$-Algebra]
	Sei $\mathcal{A}$ eine $\sigma$-Algebra über $X$.
	\begin{enumerate}[label=(\alph*)]
		\item $\emptyset\in\mathcal{A}$
		\item $A,B\in\mathcal{A}\Rightarrow A\cup B\in\mathcal{A}$
		\item $(A_n)_{i\in\natur}\subset\mathcal{A}\Rightarrow\bigcap_{n\in \natur}A_n\in\mathcal{A}$
		\item $A,B\in\mathcal{A}\Rightarrow A\cap B\in\mathcal{A}$
		\item $A,B\in\mathcal{A}\Rightarrow A\setminus B\in\mathcal{A}$
	\end{enumerate}
\end{proposition}
\begin{proof}
	\begin{enumerate}[label=(\alph*)]
		\item $\emptyset=X^C\in\mathcal{A}$
		\item $A_1=A$, $A_2=B$m $A_3=A_4=...=\emptyset\Rightarrow A\cup B=\bigcup_{n\in \natur} A_n\in\mathcal{A}$
		\item $A_n\in\mathcal{A}\xRightarrow{\text{S2}}A_n^C\in\mathcal{A}\xRightarrow{\text{S3}}\bigcup_{n\in \natur} A_n^C\in\mathcal{A}\Rightarrow\bigcap_{n\in \natur}A_n=\left(\bigcap_{n\in \natur} A_n^C\right)^C\in\mathcal{A}$
		\item wie (b)
		\item $A\setminus B=A\cap B^C\in\mathcal{A}$
	\end{enumerate}
\end{proof}

\textbf{Fazit:} Auf einer $\sigma$-Algebra kann man alle üblichen Mengenoperationen abzählbar oft durchführen ohne $\mathcal{A}$ zu verlassen!

\begin{example}
	$X\neq\emptyset$ Menge, $A,B\subset X$
	\begin{enumerate}[label=(\alph*)]
		\item $\mathcal{P}(X)$ ist eine $\sigma$-Algebra (größtmögliche)
		\item $\{\emptyset,X\}$ ist eine $\sigma$-Algebra (kleinstmögliche)
		\item $\{\emptyset,A,A^C,X\}$ ist eine $\sigma$-Algebra
		\item $\{\emptyset,B,X\}$ ist eine $\sigma$-Algebra, wenn $B=\emptyset$ oder $B=X$
		\item $\mathcal{A}=\{A\subset X\mid \#A\le \#\natur\text{ oder } \#A^C\le \#\natur\}$ ist eine $\sigma$-Algebra
	\end{enumerate}
\end{example}