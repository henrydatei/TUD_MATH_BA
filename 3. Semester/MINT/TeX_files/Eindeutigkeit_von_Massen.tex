\section{Eindeutigkeit von Maßen}
Algebra = $\sigma$-Algebra und $G$ ist Grundmenge.\\
\underline{Ziel:} $\lambda^d$ (oder algemeines Maß) auf Erzeuger $\mathscr{G}$ definieren und dann auf $\sigma(\mathscr{G})$ fortsetzen.
\underline{Brauche:} Wohldefiniertheit $\longleftrightarrow \exists !$ Fortsetzung\\
\underline{Problem:} $\sigma(\mathscr{G})$ im Allgeimeinen nicht ``konstruierbar''\\

\begin{definition}[\person{Dynkin}-System}]
	$\mathscr{D} \subset \mathscr{P}(E)$ heißt \begriff{\person{Dynkin}-System}}, wenn
	\begin{itemize}
		\item $(D_1)$ $E \in \mathscr{D}$
		\item $(D_2)$ $D \in \mathscr{D} \Rightarrow D^C \in \mathscr{D}$
		\item $(D_3)$ $(D_n)_{n\in \natur} \subset \mathscr{D}$ und paarweise Disjunktheit $\Rightarrow \biguplus_{n \in \natur} D_n \in \mathscr{D}$
	\end{itemize}
\end{definition}

\begin{remark}
	\begin{enumerate}[label=(\alph*)]
		\item Jede $\sigma$-Algebra ist insbesondere eine Dynkin-System, da $(D_3)$ schwächer als $S_3$.
		\item Wie in \propref{2_2}a)b) sieht man $\emptyset \in \mathscr{D}, A,B \in \mathscr{D}$ und $A \cap B = \emptyset \Rightarrow A \sqcup B \in \mathscr{D}$
	\end{enumerate}
\end{remark}

\begin{proposition}
	\proplbl{4_3}
	\begin{enumerate}[label=(\alph*)]
		\item Für $\mathscr{G} \subset \mathscr{P}(E)$ beliebig existiert ein kleinstes (minimales) Dynkin-System $\mathscr{D}$ mit $\mathscr{G} \subset \mathscr{D}$ mit Notation $\delta(\mathscr{G})$. $\delta/\mathscr{G}$ ist von $\mathscr{G}$ erzeugtes Dynkin-System.
		\item $\mathscr{G} \subset \delta(\mathscr{G}) \subset \sigma(\mathscr{G})$
	\end{enumerate}
\end{proposition}

\begin{proof}
	\begin{enumerate}[label=(\alph*)]
		\item wie in \propref{2.4} wörtlich %TODO set ref
		\item todo
	\end{enumerate}
\end{proof}

\underline{Ziel:} Zusammenhang Dynkin-System $\leftrightarrow \sigma$-Algebra 

\begin{lemma}
	\proplbl{4_4}
	Ein Dynkin-System $\mathscr{D}$ ist eine Algebra $\Leftrightarrow \mathscr{D}$ ist $\cap$-stabil $(\forall D,F \in \mathscr{D}\colon D \cap F \in \mathscr{D})$  
\end{lemma}

\begin{proof}
	\begin{itemize}
		\item $(\Rightarrow)$ Wenn $\mathscr{D}$ Algebra dann insbesondere
			\begin{enumerate}[label=(\alph*)]
				\item $\mathscr{D}$ Dynkin-System, da $S_1=D_1, S_2=D_2, S_3 \to D_3$
				\item $\mathscr{D}$ $\cap$-stabil nach \propref{2_2}c)
			\end{enumerate}
		\item $(\Leftarrow)$ Sei $\mathscr{D}$ ein $\cap$-stabiles Dynkin-System. Zeige $S_3$, d.h. $(D_n)_{n\in \natur} \subset \mathscr{D} \Rightarrow \bigcap_{n\in \natur} D_n \in \mathscr{D}$\\
		Idee: disjunkt machen, also $F_{n+1} := ((D_{n+1}\setminus D_n)\setminus D_{n-1})\cdots \setminus D_1$, wobei $F_1 := D_1$\\
		Bemerke: $F_{n+1} = D_{n+1} \cap \left(\bigcap_{i=1}^n D_i^C\right) \in \mathscr{D}$, da $\cap$-stabil und $F_n$ disjunkt\\
		$\Rightarrow D:= \biguplus_{n\in \natur} D_n = \biguplus_{n \in \natur} F_n \in \mathscr{D}$ (wegen $D_3$)
		\end{itemize}
\end{proof}

\begin{proposition}
	\proplbl{4_5}
	$\mathscr{G} \subset \mathscr{P}(E)$ $\cap$-stabil $\Rightarrow \delta(\mathscr{G}) = \sigma(\mathscr{G})$.
\end{proposition}

\begin{proof}
	\begin{enumerate}[label=(\arabic*)]
		\item $\delta(\mathscr{G}) \subset \sigma(\mathscr{G})$ Klar!
		\item Wäre $\sigma(\mathscr{G})$ eine Algebra, dann $\sigma(\mathscr{G}) \subset \delta(\mathscr{G})$\\
		Grund: $\mathscr{G} \subset \delta(\mathscr{G})$ wegen Defintion und $\delta(\mathscr{G})$ wäre Algebra, damit folgt $\sigma(\mathscr{G}) \subset \delta(\mathscr{G})$ ($\sigma(\mathscr{G})$ minimale Algebra mit $\mathscr{G} \subset \sigma(\mathscr{G})$), dann folgt mit 1) und \propref{4_3}a) $\delta(\mathscr{G}) \subset \sigma(\mathscr{G})$
		\item Zeige $\delta(\mathscr{G})$ $\cap$-stabil. Dann \propref{4_4} $\Rightarrow \delta(\mathscr{G})$ Algebra, fertig
		\item $D \in \delta(\mathscr{G})$ fest und behaupte $\mathscr{D}_{D} = \{Q \subset E \mid Q \cap D \in \delta(\mathscr{D})\}$ ist ein Dynkin-System:
		\begin{itemize}
			\item $(D_1)$ $\emptyset \in \mathscr{D}_D$, da $Q=\emptyset$ setzen kann
			\item $(D_2)$ Sei $Q \in \mathscr{D}_D$. zu zeigen: $Q^C \in \mathscr{D}_D$
			\begin{align}
				Q^C \cap D = (Q^C \cup D^C)\cap D &\overset{\ast}{=} (Q \cap D)^C \cap D \quad& \ast: \text{ de Morgan} \notag\\
				&= ((Q \cap D)) \uplus D^C)^C \in \delta(\mathscr{G}) \label{4_4_eq} \tag{\#}
			\end{align}
			Damit folgt aus der Definition von $\mathscr{D}_D$, dass $Q^C \in \mathscr{D}_D$. In \eqref{4_4_eq} wurde benutzt, das $Q \cap D \subset D$ und $D^C \not \subset D$
			\item $(D_3)$ $(Q_n)_{n\in \natur} \subset \mathscr{D}_D$ disjunkt $\Rightarrow (Q_N \cap D)_{n \in \natur} \subset \delta(\mathscr{D})$ disjunkt (gilt wegen Def von $\mathscr{D}_D$)\\
			$\delta(\mathscr{G}) \overset{D_3}{\ni} \sqcup_{n \in \natur} (Q \cap D) = (\biguplus_{n \in \natur} Q_n)\cap D \in \delta(\mathscr{G}) \overset{\text{Def. }\mathscr{D}_D}{\Rightarrow} \uplus_{n \in \natur} Q_n \in \mathscr{D}_D$
		\end{itemize}
	\item Zeige $\delta(\mathscr{G}) \subset \mathscr{D}_D$ für $D \in \delta(\mathscr{G})$ fest aber beliebig\\
	$\forall D \in \delta(\mathscr{G}) \colon \delta(\mathscr{G}) \cap D \in \delta(\mathscr{G}) \Rightarrow \delta(\mathscr{G}) \subset \delta(\mathscr{G})$ wäre $\cap$-stabil\\
	Klar $\mathscr{G} \subset \delta(\mathscr{G})$ und $\mathscr{G}$ sei $\cap$-stabil (Vorraussetzung) $\mathscr{G} \subset \mathscr{D}_G\quad \forall G \in \mathscr{D}_G$
	\begin{align}
		&\Rightarrow \delta(\mathscr{G}) \subset \mathscr{D}_G\quad \forall G \in \mathscr{G} \mathscr{D}_G\text{ Dynkin-System}\notag \\
		&\overset{\text{Def. } \mathscr{D}_G}{\Rightarrow} G \cap D \in \delta(\mathscr{G}) \forall D \in \mathscr{G} \forall D \in \delta(\mathscr{G})\notag \\
		&\overset{\text{Def. }\mathscr{D}_D}{\Rightarrow} G \in \mathscr{D}_D \forall G \in \mathscr{G} \forall D \in \delta(\mathscr{G}) \quad \text{ Tausche } D \leftrightarrow G \notag \\
		&\Rightarrow \mathscr{G} \subset \mathscr{D}_D \forall D \in \delta(\mathscr{G}) \notag\\
		&\xRightarrow[\text{Dyn. Sys}]{\mathscr{D}_D} \delta(\mathscr{G}) \subset \mathscr{D}_D \forall D \in \delta(\mathscr{G}) \notag\\
		&\Rightarrow \forall Q \in \delta(\mathscr{G}) \colon Q \cap D \in \delta(\mathscr{G})\notag \\
		&\Rightarrow \delta(\mathscr{G}) \cap\text{-stabil}\notag
	\end{align} %TODO fix arrows.
	\end{enumerate}
\end{proof}

Wir brauchen \propref{4_5} an 2 Stellen: hier und bei Produktmaßen

\begin{proposition}[Eindeutigkeitssatz]
	\proplbl{4_6}
	$(E, \mathscr{A})$ beliebiger Messraum, $\mu, \nu$ zwei Maße und $\mathscr{A} = \sigma(\mathscr{G})$ und 
	\begin{enumerate}[label=(\alph*)]
		\item $\mathscr{G}$ ist $\cap$-stabil
		\item $\exists (G_n)_{n \in \natur} \subset \mathscr{G}, G_n \uparrow E, \mu(G_n), \nu(G_n) < \infty$
	\end{enumerate}
	$\Rightarrow \forall G \in \mathscr{G}\colon \mu(G) = \nu(G) &\Rightarrow \forall A \in \sigma(G)\colon \mu(A) = \nu(A)$
	Kurznotation: $\mu_{\vert_G} = \nu\vert_G &\Rightarrow &\mu=\nu$
\end{proposition}

\begin{proof}
	$\forall n \colon \mathscr{D}_n := \{A \in \mathscr{A} \mid \mu(G_n \cap A) = \nu(G_n \cap A)\}$
	\begin{enumerate}[label=(\alph*)]
		\item $\mathscr{D}_n$ ist Dynkin-System $\forall n, n$ fest
	\end{enumerate}
\end{proof}

\begin{remark}[Sonderfall]
	$\mu, \nu$ $W$-Maße (oder $\mu(E) = \nu(E) < \infty$), dann kann man b) weglassen\\
	\underline{Grund:} $\mathscr{G} \rightsquigarrow \mathscr{G} \cup \{E\} = \{B\colon B \in \mathscr{G} \text{ oder }B = E\}$ und $G_n := E \uparrow E \Rightarrow$ b)
\end{remark}

%TODO finish proofs.