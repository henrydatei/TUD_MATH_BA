Für die Vorlesung \textit{Maß und Integral} von Prof. \person{Schilling} im WS 2018/19 gibt es zwar schon ein Buch von Prof. \person{Schilling}, was sich jeder Kapitel für Kapitel über die SLUB herunterladen kann. Trotzdem haben wir es uns nicht nehmen lassen auch für diese Vorlesung ein Skript zu schreiben.\footnote{Also zumindest haben wir das vor; zu dem
 Zeitpunkt, an dem ich dieses Vorwort schreibe, ich das Skript noch lange nicht fertig.}

Dem Fakt geschuldet, dass Prof. \person{Schilling} seine Vorlesung sehr lebhaft\footnote{Seine Vorlesung lässt sich mit folgenden Wort eigentlich ganz gut beschreiben: \textit{fabulös}} hält und mit mindestens 3 Farben und jeder Menge Pfeilen arbeitet, war es relativ schwierig daraus ein vernünftiges Skript zu schreiben. Deswegen sind die nachfolgenden Seiten eher eine zusammengefasste und verbesserte Abschrift seines Buches.

Auch wenn wir uns Mühe geben dieses Skript frei von Fehlern zu halten - perfekt sind auch wir nicht. Falls du deswegen einen Fehler beim Lesen findest sind wir froh über jeden Issue, den du auf \url{https://github.com/henrydatei/TUD_MATH_BA} erstellst. So hilfst du deinen jetzigen und zukünftigen Kommilitonen!

Genieße auf jeden Fall die Show von Prof. \person{Schilling} \smiley{}! Ich habe bis jetzt keine Vorlesung erlebt, die mit so viel Begeisterung gehalten wurde.