\section{Integration auf Kartengebieten}

\begin{underlinedenvironment}[Frage]
	Oberflächeninhalt bzw. $d$-dimensionaler Inhalt auf Mannigfaltigkeit $M$?
\end{underlinedenvironment}

\begin{underlinedenvironment}[Idee]
	Approximiere durch stückweise "`ebene"' Mannigfaltigkeit.
	
	\begin{enumerate}[label={\alph*)}]
		\item (d=2) Verbinde Punkte auf $M$ zu Dreiecken (einbeschriebene Approximation).
		
		Fläche $M = \sup \sum_{\triangle}$ Dreiecksflächen
		
		$\rightarrow$ funktioniert nur für Kurven und nicht für $d > 1$. Z.B. Zylinderoberfläche in $M\subset\mathbb{R}^3$ $\Rightarrow$ Fläche $M = \infty$, siehe dazu auch Hildebrandt: Analysis 2, Kapitel 6.1 (Schwarz'scher Stiefel)
		
		\item (d=2) Nehme tangentiale Parallelogramme (äußere Approximation).
		
		Fläche $M = \lim\limits_\text{Feinheit $\rightarrow\infty$}$ $\sum_j$ Fläche($\phi'(x_j)(Q_j)$).
	\end{enumerate}
\end{underlinedenvironment}

\begin{underlinedenvironment}[Hinweis]
	Eine allgemeine Theorie für den $d$-dimensionalen Inhalt liefert das Hausdorff-Maß $\mathcal{H}^d$.
\end{underlinedenvironment}

\begin{*definition}
	Seien $a_1$, $\dotsc$, $a_d\in\mathbb{R}^n$ $(d\le n)$. Dann heißt die Menge \begin{align*}
		P(a_1,\dotsc,a_d) := \left\lbrace \left. \sum_{j=1}^{n}t_j a_j \;\right|\; t_j\in [0,1],\;j = 1,\dotsc, d\right\rbrace
	\end{align*}
	das von $a_1$, $\dotsc$, $a_d$ aufgespannte \begriff{Parallelotop} (auch $d$-Spat).
\end{*definition}

\begin{plainenvironment}[Wiederhole]
	Lebesgue-Maß $\mathcal{L}^n$ in $\mathbb{R}^n$.
\end{plainenvironment}

\begin{proposition}
	Seien $a_1$, $\dotsc$, $a_n\in\mathbb{R}^n$ und das Volumen $v(a_1,\dotsc,a_n) := \mathcal{L}^n(p(a_1,\dotsc,a_n))$.
	
	\hspace*{0.5em}$\Rightarrow$\begin{minipage}[t]{0.8\linewidth}
		\vspace{-0.5\baselineskip}
		\begin{enumerate}[label={\roman*)}]
			\item $v(a_1,\dotsc,\lambda a_n,\dotsc,a_n) = \vert \lambda \vert v(a_1,\dotsc, a_n)$ $\forall \lambda\in\mathbb{R}$
			\item $v(a_1,\dotsc, a_k + a_j,\dotsc, a_n) = v(a_1,\dotsc, a_n)$ falls $k\neq j$ (Prinzip des Cavalieri)
			\item $v(a_1,\dotsc, a_n) = 1$ falls $\{a_1,\dotsc, a_n\}$ ein Orthonormalensystem im $\mathbb{R}^n$ bilden
			\item $v(a_1,\dotsc, a_n) = \vert \det A\vert$ wenn $A = (a_1\mid \dotsc\mid a_n)\in\mathbb{R}^{n\times n}$, d.h. die Determinante liefert das Volumen
		\end{enumerate}
	\end{minipage}
\end{proposition}

\begin{underlinedenvironment}[beachte]
	Eigenschaften i) -- iii) implizieren bereits iv) (argumentiere wie bei $\det$)
\end{underlinedenvironment}

\begin{proof}\hspace*{0pt}
	\proplbl{integration_mf_satz_1}
	\NoEndMark
	\vspace*{-0.8\baselineskip}
	\begin{enumerate}[label={\alph*)}]
		\item $a_1$, $\dotsc$, $a_n$ linear abhängig:\\
		\hspace*{0.5em}$\Rightarrow$ $P(a_1,\dotsc, a_n)$ ist flach $\Rightarrow$ $v(a_1,\dotsc, a_n) = 0$ \\
		\hspace*{0.5em}$\Rightarrow$ iv) $\Rightarrow$ i), ii) richtig
		\item $a_1$, $\dotsc$, $a_n$ linear unabhängig: Sei $\{e_1,\dotsc, e_n\}$ das Standard-Orthonormalensystem, damit ist iii) wahr nach der Defintion des Lebegue-Maßes.
		
		Sei $U := P(e_1,\dotsc, e_n)$, $V:= P(a_1,\dotsc, a_n)$ \\\begin{tabularx}{\linewidth}{r@{$\;\;$}X}
		$\Rightarrow$& $A\colon \inn U\to \inn V$ ist Diffeomorphismus. Offenbar ist $A'(y) = A$ $\forall y$. \\
		$\xRightarrow[\text{satz}]{\text{Trafo-}}$ & $\displaystyle \mathcal{L}(V) = \int_V \;\mathrm{d}x = \int_U \vert \det A \vert \;\mathrm{d}y = \vert \det A\vert \underbrace{\mathcal{L}(U)}_{=1} = \vert \det A\vert$ \\
		$\Rightarrow$ & iv) $\Rightarrow$ i), ii), iii) nach Eigenschaften der Determinante \hfill \csname\InTheoType Symbol\endcsname
		\end{tabularx}
	\end{enumerate}
\end{proof}

\begin{underlinedenvironment}[Ziel]
	$d$-dimensionaler Inhalt $v_d(P(a_1,\dotsc,a_n))$
\end{underlinedenvironment}
\begin{underlinedenvironment}[Idee]
	Betrachte $P(a_1,\dotsc,a_d)$ als Teilmenge eines $d$-dimensionalen Vektorraumes $X$ und nehme $d$-dimensionales Lebesgue-Maß in $X$.
	
	Somit sollte $v_d\colon\underbrace{\mathbb{R}^n\times \dotsc\times\mathbb{R}^n}_{\text{$d$-fach}}\to\mathbb{R}$ folgende Eigenschaften innehaben:\begin{enumerate}[label={(V\arabic*)}]
		\item $v_d(a_1,\dotsc,\lambda a_k,\dotsc,a_d) = \vert\lambda\vert v_d(a_1,\dotsc,a_d)$
		\item $v_d(a_1,\dotsc, a_k+a_j,\dotsc,a_d) = v_d(a_1,\dotsc, a_d)$ $\forall k\neq j$
		\item $v_d(a_1,\dotsc,a_d) = 1$ falls $a_1,\dotsc,a_n$ orthonormal
	\end{enumerate}
\end{underlinedenvironment}

\begin{proposition}
	$v_d$ ist eindeutig bestimmt und es gilt \begin{align}
		\proplbl{eq:integration_mf_gleichung_1}
		v_d(a_1,\dotsc,a_d) = \sqrt{\det(\transpose{A}A)}\text{ mit } A = (a_1\mid \dotsc \mid a_d)
	\end{align}
\end{proposition}

\begin{*remark}\hspace{0pt}
	\vspace*{-1.5\baselineskip}
	\begin{enumerate}[label={\arabic*)}]
		\item Für $d=n$ liefert \eqref{eq:integration_mf_gleichung_1} iv) in \propref{integration_mf_satz_1}
		\item $\transpose{A}A$ ist symmetrisch und positiv definit und somit auch $\det(\transpose{A}A) \ge 0$
		\item $v_d(a_1,\dotsc,a_d) = 0$ $\Leftrightarrow$ $a_1$, $\dotsc$, $a_d$ linear abhängig
	\end{enumerate}
\end{*remark}

\begin{proof}
	Sei $\alpha_1{ij} = \langle \alpha_i,\alpha_j\rangle$, dann ist \begin{align*}
		\transpose{A}A = \begin{pmatrix}
			\alpha_{11} & \dots & \alpha_{1d} \\ \vdots & & \vdots \\ \alpha_{d1} & \dots & \alpha_{dd}
		\end{pmatrix}.
	\end{align*}
	Die Eigenschaften der Determinante implizieren, dass die rechte Seite in \eqref{eq:integration_mf_gleichung_1} (V1) bis (V3) erfüllt. Wie bei der Determinante zeigt man auch, dass (V1) bis (V3) $v_d$ eindeutig bestimmen (Zurückführen von $v_d$ auf eine Orthonormalbasis mittels i), ii) liefert eindeutigen Wert).
\end{proof}

\begin{example}
	Sei $d=n-1$. Seien $a_1$, $\dotsc$, $a_{n-1}\in\mathbb{R}^n$, $a:= a_1\land\dotsc\land a_{n-1}$
	\zeroAmsmathAlignVSpaces \begin{flalign}
		\proplbl{eq:integration_mf_2}
		\;\;\Rightarrow\;\;& v_{n-1}(a_1,\dotsc,a_d) = \vert a \vert_2 &
	\end{flalign}
	(d.h. euklidische Norm des äußeren Produktes liefert das Volumen)
	
	Denn wegen $\langle a,a_j\rangle = 0$ und $A$ wie in \eqref{eq:integration_mf_gleichung_1} folgt \begin{flalign*}
		& \left( \frac{\transpose{a}}{\transpose{A}}\right) \cdot \left( a \mid A\right) = \begin{pmatrix}
			\langle a,a\rangle & 0 \\ 0 & \transpose{A}A
		\end{pmatrix} & \\
		\Rightarrow\;\;& \vert a \vert^2 \cdot \det (\transpose{A}A)  = [ \det (a\mid A) ]^2 \overset{\propref{mf_satz_4}}{=} \vert a \vert^4 \\
		\Rightarrow\;\;  & \det (\transpose{A}A) = \vert a \vert^2
	\end{flalign*}
\end{example}

\begin{underlinedenvironment}[Frage]
	Für Mannigfaltigkeit $M$: Ist für die Transformation $v_d(\text{Quader} Q)$ $\xrightarrow{\phi'(A)}$ $v_d(\text{Paralleltotop} P)$ für Quader $Q = P(b_1,\dotsc,b_d)\subset\mathbb{R}^d$ das $P(a_1,\dotsc, a_d) \subset T_u(M)\subset\mathbb{R}^n$ das zugehörige Parallelotop falls $a_j = \phi'(x) b_j$ $j=1,\dotsc,d$?
\end{underlinedenvironment}

\begin{proposition}
	Sei $M\subset\mathbb{R}^n$ $d$-dimensionale Mannigfaltigkeit, $\phi$ Parametrisierung mit $\phi(x) = u$ $\forall u\in M$ und ist $Q = P(a_1,\dotsc,a_d)\subset\mathbb{R}^d$ Quader und $a_j := \phi'(x) \cdot b_j$
	
	\begin{flalign}
		\proplbl{eq:integration_mf_3}
		\;\;\Rightarrow\;\; & v_d(a_1,\dotsc,a_d) = \sqrt{\det(\transpose{\phi'(x)}\phi'(x)}\cdot v_d(b_1,\dotsc,b_d)&
	\end{flalign}
	
	$\transpose{\phi'(x)}\phi'(x)$ heißt \begriff{Maßtensor} von $\phi$ in $x$ und $g^\phi(x)= \det(\transpose{\phi'(x)}\phi'(x))$ heißt \begriff{Gram'sche Determinante} von $\phi$ in $x$.
\end{proposition}

\begin{proof}
	Sei $B = (b_1 \mid \dotsc\mid b_d)$, $A = (a_1\mid\dotsc\mid a_d)$\\
	\hspace*{0.5em}$\xRightarrow{\eqref{eq:integration_mf_gleichung_1}}$ $v_d(a_1,\dotsc,a_d) = \sqrt{\det(\transpose{A}A)} = \sqrt{\det\big(\transpose{(\phi'(x)B)}(\phi'(x)B\big)} = \sqrt{\det(\transpose{\phi'(x)}\phi'(x)}\cdot\sqrt{\det(\transpose{B}B)}$
\end{proof}

\begin{*definition}
	Sei $M\subset\mathbb{R}^n$ $d$-dimensionale Mannigfaltigkeit, $\phi\colon V\to U$ lokale Parametrisierung, $f\colon U\to\mathbb{R}$ eine Funktion auf dem Kartengebiet $U$. Motiviert durch das Riemann-Integral \begin{align*}
		\sum f(U_i)\cdot v_d(P_i) = \sum f(\phi(x_i))\cdot \sqrt{g^\phi(x)}\cdot v_d(Q_i)
	\end{align*}
	setzt man \begin{align}
		\proplbl{eq:integration_mf_4}
		\int_U f\;\mathrm{d}a := \int_V f(\phi(x))\cdot\sqrt{g^\phi(x)} \;\mathrm{d}x
	\end{align}
	als Integral von $f$ über dem Kartengebiet $U$ falls dieses existiert. $f$ heißt dann \begriff{integrierbar} auf $U$.
\end{*definition}

\begin{*remark}\hspace*{0pt}
	\vspace*{-1.5\baselineskip}
	\begin{itemize}
		\item Die rechte Seite in \eqref{eq:integration_mf_4} ist Lebesgue-Integral im $\mathbb{R}^d$.
		\item Damit definiert \eqref{eq:integration_mf_4} sinnvoll ist, sollte die rechte Seite unabhängig von $\phi$ sein.
		\item Mittels des Hausdorff-Maßes $\mathcal{H}^d$ kann $\int_U f \mathrm{d}a$ vollkommen analog zum Lebesgue-Integral definiert werden.
		\item Für $n$-dimensionale Mannigfaltigkeit $M\subset\mathbb{R}^n$: $\int_u f\mathrm{d}a$ = Lebesgue-Integral $\int_U f\mathrm{d}x$.
	\end{itemize}
\end{*remark}

\begin{proposition}
	Sei $M\subset\mathbb{R}^n$ $d$-dimensionale Mannigfaltigkeit, $U\subset M$ ein Kartengebiet und $f\colon U\to\mathbb{R}$ sowie $\phi\colon V_i\to U$ ($i=1,2)$ seien zugehörige Parametrisierungen\begin{flalign*}
		\Rightarrow\;\;&\int_{V_1} f(\phi_1(x))\sqrt{g^{\phi_1}(x)}\;\mathrm{d}x = \int_{V_2} f(\phi_2(x))\sqrt{g^{\phi_2}(x)}\;\mathrm{d}x
	\end{flalign*}
	$\Rightarrow$ Somit: \eqref{eq:mf_gleichung_4} unabhängig von $\phi$:
	\begin{align}
		\proplbl{eq:integration_mf_5}
		f(\,\cdot\,)\text{ integrierbar auf }U & \Leftrightarrow \begin{aligned}[t]
			&f(\phi(\,\cdot\,))\sqrt{g^\phi(x)} \; \text{integrierbar auf V} \\
			&\text{für eine Parametrisierung }\phi\colon U\to V
		\end{aligned}
	\end{align}
\end{proposition}

\begin{proof}
	$\psi\colon \phi_1^{-1}\circ\phi_2\colon V_2\to V_1$ ist Diffeomorphismus nach \propref{mf_lemma_5}
	\zeroAmsmathAlignVSpaces*
	\begin{flalign*}
		\;\;\xRightarrow[\text{satz}]{\text{Trafo-}} \;\;&\int_{V_1} f(\phi_1(x))\sqrt{g^{\phi_1}(x)}\;\mathrm{d}x = \int_{V_2} f(\phi_1(\psi(y)))\cdot \underbrace{\sqrt{\det\left( \phi_1'(\psi(y))\cdot \phi_1'(\psi(y))\right)}\cdot\det(\psi'(y))}_{=\sqrt{\det(\transpose{\psi'}\cdot\transpose{\phi_1'}\phi_1'\cdot \psi'))} = \sqrt{\det(\transpose{(\phi_1'\psi')}(\phi_1\psi)}}\mathrm{d}y &
	\end{flalign*}
	Wegen $\phi_2(y) = \phi_1(\psi(y))$ $\xRightarrow[\text{regel}]{\text{Ketten}}$ $\phi_2'(y) = \phi_1'(\psi(y))\cdot \psi'(y)$
\end{proof}

\begin{*definition}
	Falls $f=1$ integrierbar über einem Kartengebiet $U\subset M$ ist, dann heißt \begin{align}
		v_d(U) = \int_U 1 \mathrm{d}a
	\end{align}
	der \begriff{$d$-dimensionale Inhalt von $U$}. $\sqrt{g^\phi(x)}$ heißt \begriff{Flächenelement} von $U$ bezüglich $U$.
\end{*definition}

\begin{*remark}\hspace*{0pt}
	\vspace*{-1.5\baselineskip}
	\begin{enumerate}[label={\arabic*)}]
		\item $v_d(U) = \mathcal{H}^d(U)$, d.h. der $d$-dimensionale Inhalt stimmt für Kartengebiete mit dem Hausdorff-Maß überein.
		\item Nach \eqref{eq:integration_mf_4}: $v_d(U) = 0$ $\Leftrightarrow$ $\mathcal{L}^d \phi^{-1}(U)) = 0$
	\end{enumerate}
\end{*remark}

\begin{example}
	Sei $M:= \{ u = (u_1,u_2,u_3)\in\mathbb{R}^3\mid\vert u \vert = r, u_1> 0 \}$ (Halbsphäre mit Radius $r$).	Berechne $\int_M f\mathrm{d}a$.
	
	Parametrisierung von $M$ (Kugelkoordinaten): \begin{align*}
		\phi(x_1,x_2) = r \cdot \begin{pmatrix}
			\cos x_2 \cdot \cos x_1 \\ \cos x_2 \cdot \sin x_1 \\ \sin x_2
		\end{pmatrix}
	\end{align*}
	für $(x_1,x_2)\in \left( -\frac{\pi}{2},\frac{\pi}{2}\right)\times \left( -\frac{\pi}{2},\frac{\pi}{2}\right) = V$.
	
	Offenbar ist $\phi\colon V\to M\in C^1$, regulär und Homöomorphismus.\\
	\hspace*{0.5em}$\Rightarrow$ $\phi$ ist Parametrisierung von $M$, d.h. $M$ ist Mannigfaltigkeit und $M$ auch Kartengebiet.
	
	{\zeroAmsmathAlignVSpaces*
	\begin{align*}
		\phi'(x) &= r \cdot \begin{pmatrix}
			-\cos x_2\cdot\sin x_1 & -\sin x_2\cdot \cos x_1 \\ \cos x_2\cdot \cos x_1 & -\sin x_2\cdot \sin x_1 \\ 0 & \cos x_2
		\end{pmatrix} \\
		\transpose{\phi'(x)}\cdot \phi(x) &= r^2\begin{pmatrix}
			\cos^2 x_2 & 0 \\ 0 & 1
		\end{pmatrix}, \quad \sqrt{g^\phi(x)} = \cos^2 x_2
	\end{align*}}
	Damit lässt sich dann obiges Integral berechnen: \begin{align*}
		\int_M f\mathrm{d}a &= r^2 \int_V f(\phi(x))\cdot\cos^2 x_2\mathrm{d}x = r^2 \int_{-\frac{\pi}{2}}^{\frac{\pi}{2}} \cos x_2\cdot \int_{-\frac{\pi}{2}}^{\frac{\pi}{2}} f(\phi(x_1)) \mathrm{d}x_1 \mathrm{d}x_2
	\intertext{z.B. mit $f(u) = u_1^2 + u_2^2$:}
		\int_M u_1^2 + u_2^2\mathrm{d}a &= r^4\int_{-\frac{\pi}{2}}^{\frac{\pi}{2}} \cos^3 x_2\int_{-\frac{\pi}{2}}^{\frac{\pi}{2}} \mathrm{d}x_1 \mathrm{d}x_2 = \pi r^4 \int_{-\frac{\pi}{2}}^{\frac{\pi}{2}} \cos^3 x_2 \mathrm{d}x_2 = \left[ \sin x_2 - \frac{1}{3}\sin^3 x_2\right]_{-\frac{\pi}{2}}^{\frac{\pi}{2}}\\
		&= \pi r^4 \left(1 - \frac{1}{3}\right)\cdot 2 = \frac{4}{3}\pi r^4,
	\intertext{z.B. für $f=1$:}
		v_d(U) = \int_M \mathrm{d}a &= \pi r^2 \int_{-\frac{\pi}{2}}^{\frac{\pi}{2}}\cos x_2 = \pi r^2 [ \sin x_2]_{-\frac{\pi}{2}}^{\frac{\pi}{2}} = 2\pi r^2 \quad\Rightarrow\quad\text{Kugeloberfläche im }\mathbb{R}^3\colon \;4\pi r^2
	\end{align*}
\end{example}

\begin{proposition}[Integration über $n-1$-dimensionale Graphen]
	Sei $g\colon V\subset\mathbb{R}^{n-1} \to \mathbb{R}$ stetig differenzierbar, $V$ offen, $\Gamma = \{ (x,g(x))\in\mathbb{R}^n \mid x\in V \}$. \begin{flalign}
	\;\;\Rightarrow\;\; & \text{für $f\colon\Gamma\to\mathbb{R}$ gilt:} \int_{\Gamma} f\, \mathrm{d}a = \int_V f(x,g(x))\sqrt{1 + (g'(x))^2}\;\mathrm{d}x, \text{ falls die rechte Seite ex.}&
	\end{flalign}
\end{proposition}

\begin{proof}
	$\Gamma$ ist $(n-1)$-dimensionale Mannigfaltigkeit und auch Kartengebiet bezüglich der Parametrisierung $\phi\colon V\to\Gamma$ mit $\phi(x) = (x,g(x))$.
	
	Offenbar ist $\gamma=\sqrt{\det(\transpose{\phi'(x)}\cdot\phi'(x)}  \overset{\eqref{eq:integration_mf_gleichung_1}}{=} v_{n-1}(\phi_{x_1}(x),\dotsc, \phi_{x_{n-1}}(x))  \overset{\eqref{eq:integration_mf_2}}{=}\vert\phi_{x_1}\land\dotsc\land\phi_{x_{n-1}}(x))\vert$.
	
	Wegen $\phi_{x_1}(x)\land\dotsc\land\phi_{x_{n-1}}(x) = (-1)^n\binom{g'(x)}{-1}\in\mathbb{R}^n$\\
	\hspace*{0.5em}$\Rightarrow$ $\gamma = \sqrt{1+ \vert g'(x)\vert^2}$ (euklidische Norm) $\xRightarrow{\eqref{eq:integration_mf_4}}$ $\displaystyle \int_{\Gamma} f \mathrm{d}a = \int_V f(\phi(x)) \cdot\sqrt{1 + \vert g'(x)\vert^2}\;\mathrm{d}x$
\end{proof}

\begin{underlinedenvironment}[Flächeninhalt]
	von $\Gamma$ ist somit \begin{align}
		\proplbl{eq:integration_mf_8}
		v_{n-1}(\Gamma) = \int_V \sqrt{1 + \vert g'(x)\vert^2}\;\mathrm{d}x
	\end{align}
\end{underlinedenvironment}

\begin{example}
	\proplbl{integration_mf_beispiel_3}
	Halbspähre $S_{+}^{n-1} = \{ x\in\mathbb{R}^n\mid \vert x \vert = 1, x_4 > 0 \}$.
	
	Offenbar ist $S_+^{n-1}$ Graph von $g(x) = \sqrt{1 - \vert x \vert^2}$ $\forall x\in B_1(0)$
	{\zeroAmsmathAlignVSpaces \begin{flalign*}
		\Rightarrow\;\; & v_{n-1}(S_+^{n-1}) = \int_{B_1(0)} \sqrt{1 + \frac{\vert x \vert^2}{1 - \vert x \vert^2}} \;\mathrm{d}x = \int_{B_1(0)} \frac{1}{\sqrt{1 - x^2}}\;\mathrm{d}x&\\
	\end{flalign*}}
	$f(x) = \sqrt{\frac{1}{1 - \vert x \vert^2}}$ ist rotationssymmetrisch auf $B_1(0)$, d.h. $f(x)= \tilde{f}(x)$ für $\tilde{f}\colon [0,\infty]\to\mathbb{R}$.
	
	Königsberger 2: \begin{align}
		\proplbl{eq:integration_mf_9}
		\int_{B_r(0)} f(x) \mathrm{d}x = n \cdot \kappa_n \int_{0}^r \tilde{f}(\gamma) \gamma^{n-1}\mathrm{d}\gamma
	\end{align}
	für $B_r(0)\subset\mathbb{R}^n$, $\kappa_n = \mathcal{L}^n(B_1(0))$ \zeroAmsmathAlignVSpaces \begin{flalign*}
		\;\;\xRightarrow[\text{statt $n$}]{n-1} \;\;&\begin{aligned}[t]
			v_{n-1}(S_+^{n-1}) &= (n-1)\kappa_{n-1}\int_0^1 \frac{r^{n-2}}{\sqrt{1 - r^2}}\;\mathrm{d}r = (n-1)\kappa_{n-1}\int_0^1 r^n \frac{1}{r^2 \sqrt{1 - r^2}}\;\mathrm{d}r \\
			& \overset{\mathclap{\text{part.}}}{\underset{\mathclap{\text{Int.}}}{=}} n\cdot(n-1) \kappa_{n-1}\int_0^1 r^{n-1} \frac{\sqrt{1 - r^2}}{r}\;\mathrm{d}r \overset{\eqref{eq:integration_mf_9}}{=}n \cdot \underbrace{\int_{B_1(0)} \sqrt{1-\vert x \vert^2}\;\mathrm{d}\gamma}_{\text{Volumen unter Halbsphäre}}\\
			&= \sum^n \kappa_n
		\end{aligned}&
	\end{flalign*}
	Sei $\omega_n  = v_{n-1}(S_{n-1}) = 2v_{n-1}(S_+^{n-1})$ Oberfläche, dann gilt \begin{align}\omega_n = n\cdot\kappa_n,
	\end{align}
	z.B. \begin{enumerate}[label={\uline{$n$=\arabic*:}},start=2,leftmargin=4em]
		\item $2\pi = 2\cdot \pi$
		\item $4\pi = 3\cdot \frac{4}{3}\pi$
	\end{enumerate}
\end{example}

\begin{underlinedenvironment}[Hinweis]
	$v_n(B_r(0)) = \mathcal{L}^n(B_r(0)) =r^n \kappa_n$ (verwende Trafosatz),\\
	$v_{n-1}(\partial B_r(0)) = r^{n-1}\omega_n = r^{n-1}n\kappa_n$ (\propref{integration_mf_beispiel_3} mit $B_r(0)$ statt $B_1(0)$)
\end{underlinedenvironment}

\begin{example}[Kurvenintegral]
	Betrachte Kurve $\phi\colon I\subset\mathbb{R}\to\mathbb{R}^n$, $I$ offenes Intervall, sodass $C:= \phi(I)$ $1$-dimensionale Mannigfaltigkeit ist (beachte: $\phi$ regulär für $\phi'(x)\neq 0$).
	
	Offenbar ist $\det (\transpose{\phi'(t)}\phi'(t)) = \vert \phi'(t)\vert^2$. Für $f\colon C\to\mathbb{R}^n$ ist (falls es existiert) \begin{align}
		\proplbl{eq:integration_mf_11}
		\int_C f \mathrm{d}a = \int_a^b f(\phi(t))\vert\phi'(t)\vert \,\mathrm{d}t.
	\end{align}
	Das Integral heißt \begriff{Kurvenintegral} von $f$ über $C$. Der $1$-dimensionale Inhalt \begin{align}
		\proplbl{eq:integration_mf_12}
		v_1(C) = \int_a^b \vert \phi'(t)\vert\,\mathrm{d}t
	\end{align}
	heißt \begriff{Bogenlänge} der Kurve $C$.
	
	Falls $\vert\phi'(t)\vert = 1$ $\forall t\in I$ heißt $\phi$ \begriff{Bogenlänge-Parametrisierung} von $C$ (denn: $v_1(\phi(t_2-t_1)) = t_2 - t_1$, d.h. die Parameter liefern die Bogenlänge). 
	
	Mit \begin{align}
	\proplbl{eq:integration_mf_star}
	\tag{\star}\sigma(s) := \int_a^b \vert \phi'(t)\vert\mathrm{d}t
	\end{align}
	ist $\psi\colon (0,v_1(C))\to\mathbb{R}^n$ mit $\psi(I) = \phi(\sigma^{-1}(I))$ stets die Bogenlängenparametrisierung von $C$. Denn: Offenbar ist $\sigma\in C^1$ und streng monoton wachsend $\Rightarrow$ $\sigma^{-1}\in C^1$ existiert.
	\begin{flalign*}
	\Rightarrow\;\; & \vert\psi'(\tau)\vert = \vert\psi'(\sigma^{-1}(\tau))\cdot\left(\sigma^{-1}\right)'(\tau)\vert = \vert \phi'(\sigma^{-1}(\tau))\vert\cdot \frac{1}{\vert \sigma'(\sigma^{-1}(\tau))} \overset{\eqref{eq:integration_mf_star}}{=} 1,
	\end{flalign*}
	d.h. ohne Beschränkung der Allgemeinheit kann man die Kurve stets als Bogenlängenparametrisierung angeben.
\end{example}

\begin{*definition}
	Eine beliebige stetige Kurve $\phi\colon [a,b]\to\mathbb{R}^n$, $C= \phi([a,b])$, heißt \begriff{rektifizierbar}, falls \begin{align*}
		l(C) := \sup\limits_{Z} \left\lbrace \left. \sum_{j=1}^k \vert \phi(t_j) - \phi(t_{j-1})\vert\;\right|\; \{ t_0,\dotsc, t_k\} \in Z\right\rbrace < \infty,
	\end{align*}
	wobei $Z$ die Menge der Zerlegungen $a = t_0 < t_1 < \dotsc < t_k = t_1$, $k\in\mathbb{N}$ ist.
\end{*definition}

\begin{proposition}[Rektifizierbare Kurven]
	Sei $\phi\colon [a,b]\to \mathbb{R}^n$ stetig differenzierbar. Dann: \begin{enumerate}[label={\arabic*)}]
		\item $\phi$ ist rektifizierbar
		\item $C:= \phi([a,b])$ sei $1$-dimensionale Mannigfaltigkeit mit Parametrisierung $\phi$ \\
		\hspace*{0.5em} $\Rightarrow$ $l(C) = v_d(C)$
	\end{enumerate}
\end{proposition}

\begin{proof}\hspace*{0pt}
	\vspace*{-0.8\baselineskip}
	\begin{enumerate}[label={zu \arabic*)},leftmargin=4.5em]
		\item $\phi$ ist Lipschitz-stetig auf $[a,b]$ mit Lipschitz-Konstante $L = \max_{t\in [a,b]} \vert\phi'(t)\vert$ {\zeroAmsmathAlignVSpaces*\begin{flalign*}
			\;\;\Rightarrow\;\;&\sum_{j=1}^k \vert\phi(t_j) - \phi(t_{j-1})\vert \le L\sum_{j=1}^k \vert t_j - t_{j-1}\vert = L\vert b - a\vert \text{ für jede Zerlegung $\{t_0,\dotsc,t_k\}\in Z$} \\
			\Rightarrow\;\;& l(\phi([a,b])) < L(b-a)\\
			\Rightarrow\;\;&\phi\text{ rektifizierbar}&
		\end{flalign*}}
		\item Für beliebige Zerlegung $\{t_0,\dotsc,t_k\}$ gilt \begin{flalign}
			\notag&\sum_{j=1}^k \vert\phi(t_j)-\phi(t_{j-1})\vert = \sum_{j=1}^k \left\vert \int_{t_{j-1}}^{t_j} \phi'(t)\mathrm{d}t\right\vert \le \sum_{j=1}^k \int_{t_{j-1}}^{t_j} \vert\phi'(t)\vert\,\mathrm{d}t = \int_a^b \vert\phi'(t)\vert\,\mathrm{d}t& \\
			\proplbl{eq:integration_mf_star_star}
			\tag{\star\star}\;\;\Rightarrow\;\;&´l(C) \le \int_a^b \vert\phi'(t)\vert \,\mathrm{d}t = v_1(C)
		\end{flalign}
		Sei $l(t) := l(\phi([a,b]))$ $\forall t\in[a,b]$ und sei $h\in\mathbb{R}$, $t+h\in [a,b]$ {\allowdisplaybreaks\zeroAmsmathAlignVSpaces\begin{flalign*}
			\;\;\xRightarrow{h>0} \;\; & \left.\left\vert \int_t^{t+h}\phi'(\tau)\mathrm{d}\tau\right\vert = \vert\phi(t+h)-\phi(t)\vert \le \underbrace{l(t+h) - l(t)}_{\mathclap{l(\phi([t,t+h]))}} \overset{\eqref{eq:integration_mf_star_star}}{\le} \int_t^{t+h}\vert\phi'(\tau)\vert\,\mathrm{d}\tau\qquad\right| \cdot\frac{1}{h} & \\
			\Rightarrow\;\;& l \text{ ist differenzierbar mit }l'(t) = \vert\phi'(t)\vert \\
			\Rightarrow\;\;& l(b) = \int_a^b l'(t)\mathrm{d}t = \int_a^b \vert\phi'(t)\vert\,\mathrm{d}t = v_1(C)
		\end{flalign*}
		}
	\end{enumerate}
\end{proof}

\begin{example}[Umfang des Einheitskreises]
	Betrachte $\phi\colon (-\pi,\pi)\to\mathbb{R}^2$ mit $\phi(t) = \binom{\cos t}{\sin t}$. Dann ist $C:= \phi((-\pi,\pi))$ eine $1$-dimensionale Mannigfaltigkeit (der Einheitskreis ohne den Punkt $(-1\mid 0)$).
	\begin{align*}
		v_1(C) = \int_{-\pi}^\pi \vert\phi'(t)\vert\,\mathrm{d}t = \int_{-\pi}^\pi \left\vert\begin{pmatrix}
			-\sin t\\\cos t
		\end{pmatrix}\right\vert\,\mathrm{d}t = \int_{-\pi}^\pi \mathrm{d}t = 2\pi
	\end{align*}
	(beachte: $\phi$ ist Bogenlängenparametrisierung)
\end{example}

\begin{proposition}[Eigenschaften des Integrals]
	\proplbl{integration_mf_7}
	Seien $f$, $g$, $f_k\colon U\to \mathbb{R}$, $U$ Kartengebiet der Mannigfaltigkeit $M\subset\mathbb{R}^n$. Dann: \begin{enumerate}[label={\arabic*)}]
		\item $f$ integrierbar auf $U$ $\Leftrightarrow$ $\vert f \vert$ integrierbar auf $M$ $\Leftrightarrow$ $f^+$ und $f^-$ integrierbar auf $U$
		\item $f$, $g$ integrierbar, $c\in\mathbb{R}$ $\Leftrightarrow$ $\int_U cf \pm g\mathrm{d}a = c\int_Uf\mathrm{d}a \pm\int_U g\mathrm{d}a$
		\item $f$, $g$ integrierbar auf $U$, $g$ beschränkt auf $U$ $\Rightarrow$ $\cdot g$ integrierbar auf $U$
		\item $f$, $g$ integrierbar, $f\le g$ auf $U$ $\Rightarrow$ $\int_Uf\,\mathrm{d}a\le\int_Ug\,\mathrm{d}a$
		\item (Monotone Konvergenz)
		
		Seien $f_k$ integrierbar auf $U$, $f_1\le f_2\le \dotsc$, Folge $\int_U f_k\,\mathrm{d}a$ beschränkt und $f(u) = \lim_{k\to\infty} f_k(u)$ $\forall u\in U$ \\
		\hspace*{0.5em}$\Rightarrow$ $f$ integrierbar auf $U$ mit $\int_U f\,\mathrm{d}a = \lim_{k\to\infty} \int f_k\,\mathrm{d}a$
		\item (Majorisierte Konvergenz)
		
		Seien $f_k$, $g$ integrierbar auf $U$, $\vert f_k\vert\le g$ $\forall k$, $f(u) = \lim_{k\to\infty} f_k(u)$ $\forall u\in U$ \\
		\hspace*{0.5em}$\Rightarrow$ $f$ ist integrierbar auf $U$ mit $\int_U f\,\mathrm{d}a = \lim_{k\to\infty} \int_U f_k\,\mathrm{d}a$
	\end{enumerate}
\end{proposition}

\begin{proof}
	Sei $\phi\colon V\to U$ Parametrisierung des Kartengebiets $U$. Somit:\begin{itemize}
		\item $f$ integrierbar auf $U$ $\Leftrightarrow$ $f(\phi(\,\cdot\,))\sqrt{g^\phi(\,\cdot\,)}$ integrierbar auf $V$, und \item $f\le g$ auf $U$ $\Leftrightarrow$ $f(\phi(\,\cdot\,))\sqrt{g^\phi(\,\cdot\,)} \le g(\phi(\,\cdot\,))\sqrt{g^\phi(\,\cdot\,)}$ auf $V$,
		\item $f(u) = \lim_{k\to\infty} f_k(u)\in U$ $\Leftrightarrow$ $f(\phi(x)) = \lim_{k\to\infty}  f_k(x)$ $\forall x\in V$,
	\end{itemize}
	somit folgen die Behauptungen direkt aus den Eigenschaften des Lebesgue-Integrals (Kapitel 22).
\end{proof}