\documentclass[ngerman,a4paper,order=firstname,sectionreset]{../../texmf/tex/latex/mathscript/mathscript}
\usepackage{../../texmf/tex/latex/mathoperators/mathoperators}

\title{\textbf{Einführung in die Numerik WS2018/19}}
\author{Dozent: Prof. Dr. Andreas Fischer}

\begin{document}
\pagenumbering{roman}
\pagestyle{plain}

\maketitle

\hypertarget{tocpage}{}
\tableofcontents
\bookmark[dest=tocpage,level=1]{Inhaltsverzeichnis}

\pagebreak
\pagenumbering{arabic}
\pagestyle{fancy}

\chapter*{Vorwort}
Wir freuen uns, dass du unser Skript für die Vorlesung \textit{Geometrie} bei Prof. Dr. Arno Fehm im WS2018/19 gefunden hast. Da du ja offensichtlich seit einem Jahr Mathematik studierst, kannst du dich glücklich schätzen zu dem einen Drittel zu gehören, dass nicht bis zum zweiten Semester abgebrochen hat.

Wenn du schon das Vorwort zu \textit{Lineare Algebra und analytische Geometrie 1+2} gelesen hast, weißt du sicherlich, dass Prof. Fehm ein Freud der Algebra ist.\footnote{In Zukunft wird sich Prof. Fehm richtig freuen dürfen, denn im Zuge einer neuen Studienordnung, die am 1.4.2019 in Kraft tritt, kommt so gut wie keine Geometrie im \textit{Bachelor Mathematik} vor.} Auf die Frage eines Kommilitonen, wo in seinem Inhaltsverzeichnis (Gruppen, Ringe, Körper) die Geometrie vorkomme, antwortete er:
\begin{quote}
	\textit{Die Frage ist nicht, wieso wir in dieser Vorlesung Algebra statt Geometrie machen, sondern warum hier seit 20 Jahren Geometrie unterrichtet wird.}
\end{quote}

Wie auch im letzten Vorwort können wir dir nur empfehlen die Vorlesung immer zu besuchen, denn dieses Skript ist kein Ersatz dafür. Es soll aber ein Ersatz für deine unleserlichen und (hoffentlich nicht) unvollständigen Mitschriften sein und damit die Prüfungsvorbereitung einfacher machen. Im Gegensatz zu letztem Semester veröffentlicht Prof. Fehm auf seiner Homepage (\url{http://www.math.tu-dresden.de/~afehm/lehre.html}) kein vollständiges Skript mehr, sondern nur noch eine Zusammenfassung.

Der Quelltext dieses Skriptes ist bei Github (\url{https://github.com/henrydatei/TUD_MATH_BA}) gehostet; du kannst ihn dir herunterladen, anschauen, verändern, neu kompilieren, ... Auch wenn wir das Skript immer wieder durchlesen und Fehler beheben, können wir leider keine Garantie auf Richtigkeit geben. Wenn du Fehler finden solltest, wären wir froh, wenn du ein neues Issue auf Github erstellst und dort beschreibst, was falsch ist. Damit wird vielen (und besonders nachfolgenden) Studenten geholfen.

Und jetzt viel Spaß bei \textit{Geometrie}!

\begin{flushright}
	Henry, Pascal und Daniel
\end{flushright}

\chapter{Interpolation}
\section{Grundlagen}

\textbf{Aufgabe:} \\
Gegeben sind $n+1$ Datenpaare $(x_0,f_0),\dots, (x_n,f_n)$, alles reelle Zahlen und paarweise verschieden. \\
Gesucht ist eine Funktion $F:\real\to\real$, die die \begriff{Interpolationsbedingungen}
\begin{align}
	\label{interpolationsbedingung}
	F(x_0) = f_0, \, \dots, \, F(x_n)=f_n
\end{align}
genügt.

\begin{*definition}[Stützstellen, Stützwerte]
	Die $x_0$ bis $x_n$ werden \begriff{Stützstellen} genannt.
	
	Die $f_0$ bis $f_n$ werden \begriff{Stützwerte} genannt.
\end{*definition}

Die oben gestellte Aufgabe wird zum Beispiel durch 
\begin{align}
	F(x) = \begin{cases}
		0 & x\notin \{x_0,\dots,x_n\} \\
		f_i & x=x_i
	\end{cases}\notag
\end{align}
gelöst. Weitere Möglichkeiten sind: Polygonzug, Treppenfunktion, Polynom, \dots
\begin{itemize}
	\item In welcher Menge von Funktionen soll $F$ liegen?
	\item Gibt es im gewählten \begriff{Funktionenraum} für beliebige Datenpaare eine Funktion $F$, die den Interpolationsbedingungen genügt (eine solche Funktion heißt \begriff{Interpolierende})?
	\item Ist die Interpolierende in diesem Raum eindeutig bestimmt?
	\item Welche weiteren Eigenschaften besitzt die Interpolierende, zum Beispiel hinsichtlich ihrer Krümmung oder der Approximation einer Funktion $f:\real\to\real$ mit $f_k=f(x_k)$ für $k=0, \dots, n$
	\item Wie sollte man die Stützstellen wählen, falls nicht vorgegeben?
	\item Wie lässt sich die Interpolierende effizient bestimmen, gegebenenfalls auch unter der Berücksichtigung, dass neue Datenpaare hinzukommen oder dass sich nur die Stützwerte ändern? 
\end{itemize}

\begin{example}
	\hspace*{1.5em}
	\begin{center}
		\begin{tabular}{c|cccccc}
			$k$ & 0 & 1 & 2 & 3 & 4 & 5 \\
			\hline
			$x_k$ in s & 0 & 1 & 2 & 3 & 4 & 5 \\
			\hline
			$f_k$ in °C & 80 & 85,8 & 86,4 & 93,6 & 98,3 & 99,1
		\end{tabular}
	\end{center}
Interpolation im
\begin{itemize}
	\item Raum der stetigen stückweise affinen Funktionen
	\item Raum der Polynome höchstens 5. Grades
	\item Raum der Polynome höchstens 4. Grades (Interpolation im Allgemeinen nicht lösbar, Regression nötig)
\end{itemize}
\end{example}

\section{Interpolation durch Polynome}

$\Pi_n$ bezeichne den Vektorraum der Polynome von Höchstgrad $n$ mit der üblichen Addition und Skalarmultiplikation. Für jedes $p\in\Pi_n$ gibt es $a_0,\dots,a_n\in\real$, sodass
\begin{align}
	p(x) = a_nx^n+a_{n-1}x^{n-1}+\dots+a_1x+a_0
\end{align}
und umgekehrt.

\subsection{Existenz und Eindeutigkeit}

\begin{proposition}
	Zu $n+1$ Datenpaaren $(x_0,f_0),\dots,(x_n,f_n)$ mit paarweise verschiedenen Stützstellen existiert genau ein Polynom $p\in\Pi_n$, dass die Interpolationsbedingung  \cref{interpolationsbedingung} erfüllt.
\end{proposition}
\begin{proof}
	\begin{itemize}
		\item Existenz: Sei $j\in\{0,\dots,n\}$ und $L_j:\real\to\real$ mit
		\begin{align}
			L_j(x) := \prod_{\substack{i=0\\ i\neq j}}^{n} \frac{x-x_i}{x_j-x_i}= \frac{(x-x_0)\cdot\dots\cdot(x-x_{j-1})(x-x_{j+1})\cdot\dots\cdot(x-x_n)} {(x_j-x_0)\cdot\dots\cdot(x_j-x_{j-1})(x_j-x_{j+1})\cdot\dots\cdot(x_j-x_n)}\notag
		\end{align}
		das \begriff{\person{Lagrange}-Basispolynom} vom Grad $n$. Offenbar gilt $L_j\in\Pi_n$ und 
		\begin{align}
			\label{1.3}
			L_j(x_k)=\begin{cases}
				1 & k=j \\ 0 & k\neq j
			\end{cases} = \delta_{jk}
		\end{align}
		Definiert man $p:\real\to\real$ durch
		\begin{align}
			\label{1.4}
			p(x) := \sum_{j=0}^{n} f_j\cdot L_j(x)
		\end{align}
		so ist $p\in\Pi_n$ und außerdem erfüllt $p$ wegen \cref{1.3} die Interpolationsbedingung \cref{interpolationsbedingung}
		\item Eindeutigkeit: Angenommen es gibt Interpolierende $p,\tilde{p}\in\Pi_n$ mit $p\neq\tilde{p}$. Dann folgt $p-\tilde{p}\in\Pi_n$ und $(p-\tilde{p})(x_k)=p(x_k)-\tilde{p}(x_k)=0$ für $k=0,\dots,n$. Also hat $(p-\tilde{p})$ mindestens $n+1$ Nullstellen, hat aber Grad $n$. Das heißt, dass $(p-\tilde{p})$ das Nullpolynom sein muss.
	\end{itemize}
\end{proof}

\begin{*definition}[Interpolationspolynom]
	Das Polynom, dass die Interpolationsbedingung erfüllt, heißt \begriff{Interpolationspolynom} zu $(x_0,f_0),\dots,(x_n,f_n)$.
\end{*definition}

\begin{remark}
	\begin{itemize}
		\item Die Darstellung \cref{1.4} heißt \begriff{\person{Lagrange}-Form} des Interpolationspolynoms.
		\item Um mittels \cref{1.4} einen Funktionswert $p(x)$ zu berechnen, werden $\mathcal{O}(n^2)$ Operationen genötigt; bei gleichabständigen Stützstellen kann man diesen Aufwand auf $\mathcal{O}(n)$ verringern. Ändern sich die Stützwerte, kann man durch Wiederverwendung von den $L_j(x)$ das $p(x)$ in $\mathcal{O}(n)$ Operationen berechnen.
		\item Man kann zeigen, dass $L_0$ bis $L_n$ eine Basis von $Pi_n$ bilden.
	\end{itemize}
\end{remark}

\section{Interpolation durch Polynomsplines}

\subsection{Polynomsplines}

Zur Abkürzung bezeichne $\Delta$ eine Zerlegung des Intervall $[a,b]$ durch die Stützstellen $a=:x_0<...<x_n:=b$.

\begin{definition}[Polynomspline]
	Ein \begriff{Polynomspline} vom Grad $m\in\natur$ und Glattheit $l\in\natur$ zur Zerlegung $\Delta$ ist eine Funktion $s\in C^l[a,b]$ mit
	\begin{align}
		s_k := s\vert_{[x_k,x_{k+1}]}\in\Pi_n\quad\text{für } k=0,...,n-1\notag
	\end{align}
	Dabei bezeichnet $s\vert_{[x_k,x_{k+1}]}$ die Einschränkung von $s$ auf das Intervall $[x_k,x_{k+1}]$. Die Menge aller Splines wird mit $\mathcal{S}^l_m(\Delta)$ bezeichnet.
	
	Folglich ist ein Polynomspline $s\in\mathcal{S}^l_m(\Delta)$ auf jedem der Teilintervall $[x_k,x_{k+1}]$ ein Polynom vom Höchstgrad $m$. Außerdem ist $s\in\mathcal{S}^l_m(\Delta)$ in allen Punkten $x\in[a,b]$ (also auch in den Stützstellen) $l$-mal stetig differenzierbar. $\mathcal{S}^l_m(\Delta)$ ist mit der üblichen Addition und Multiplikation ein Vektorraum. Speziell ist $\mathcal{S}^0_1(\Delta)$ die Menge aller stetigen stückweise affin linearen Funktionen.
\end{definition}

\subsection{Interpolation durch kubische Polynomsplines}

Gegeben sei eine Zerlegung $\Delta$ und die Stützwerte $f_0,...,f_n$. Gesucht ist eine Funktion $s\in\mathcal{S}^l_3(\Delta)$ mit $l=1,2$ derart, dass
\begin{align}
	\label{1.6}
	s(x_k)=f_k\quad\text{für } k=0,...,n
\end{align}
Jede derartige Funktion heißt \begriff{kubischer Interpolationspline}.

\textbf{Konstruktion eines solchen Splines:}
\begin{align}
	h_k &:= x_{k-1}-x_k\quad\text{für } k=0,...,n-1 \notag \\
	m_k &:= s'(x_k) \quad\text{für } k=0,...,n-1\notag
\end{align}
Wegen $l\in\{1,2\}$ ist $s$ zunächst stetig differenzierbar. Wegen $s_k=s\vert_{[x_k,x_{k+1}]}$ für $k=0,...,n-1$ und $m=3$ kann man folgenden Ansatz für $s_k$ benutzen:
\begin{align}
	\label{1.7}
	s_k(x)=a_k(x-x_k)^3+b_k(x-x_k)^2+c_k(x-x_k)+d_k
\end{align}
Aus den Interpolationsbedingungen \cref{1.6} und der stetigen Differenzierbarkeit aller Funktionen in $s\in\mathcal{S}^l_m(\Delta)$ für $l\ge 1$ ergeben sich folgende Forderungen an $s_k$, $k=0,...,n-1$:
\begin{equation}
	\label{1.8}
	\begin{split}
		s_k(x_k) &= f_k \quad\text{und }\quad s_k(x_{k+1}) = f_{k+1} \\
		s'_k(x_k) &= m_k \quad\text{und }\quad s'_k(x_{k+1}) = m_{k+1}
	\end{split}
\end{equation}
Diese liefern:
\begin{equation}
	\label{1.9}
	\begin{split}
		d_k &= s_k(x_k)=f_k \\
		c_k &= s'_k(x_k)=m_k
	\end{split}
\end{equation}
und damit:
\begin{align}
	s_k(x_{k+1}) &= a_kh_k^3 + b_kh_k^2+m_kh_k + f_k = f_{k+1} \notag \\
	s'_k(x_{k+1}) &= 3a_kh_k^2 + 2b_kh_k + m_k = m_{k+1} \notag
\end{align}
Damit ergeben sich $a_k$ und $b_k$ als eindeutige Lösung für das lineare Gleichungssystem
\begin{align}
	\label{1.10}
	\begin{pmatrix}
		h_k^3 & h_k^2 \\ 3h_k^2 & 2h_k
	\end{pmatrix}
	\begin{pmatrix}
		a_k \\ b_k
	\end{pmatrix}=
	\begin{pmatrix}
		f_{k+1}-f_k-m_kf_k \\
		m_{k+1}-m_k
	\end{pmatrix}
\end{align}
Die Determinante ist $-h_k^4\neq 0$.

\chapter{numerische Quadratur und Integration}
\section{Integration von Interpolationspolynomen}

Für eine Funktion $f\in C[a,b]$ ist eine Näherung für den Wert des bestimmten Integrals
\begin{align}
	J(f) := \int_a^b f(x)\diff x\notag
\end{align}
gesucht. Seien $a\le x_0<...<x_n\le b$ Stützstellen und $f_k=f(x_k)$ für $k=0,...,n$. Weiter bezeichne $p_n\in\Pi_n$ das zugehörige Interpolationspolynom. Dann kann man
\begin{align}
	Q_n(f) := J(p_n) = \int_a^b p_n(x)\diff x\notag
\end{align}
als Näherung für $J(f)$ verwenden. Mit der \person{Lagrange}-Form des Interpolationspolynoms sieht man, dass
\begin{align}
	Q_n(f)=\int_a^b \sum_{k=0}^n f_k\cdot L_k(x)\diff x=\sum_{k=0}^n f_k\cdot\int_a^b L_k(x)\diff x\notag
\end{align}
das heißt die Quadraturformel $Q_n(f)$ ist die gewichtete Summe von Funktionswerten der Funktion $f$ mit den Gewichten $\int_a^b L_k(x)\diff x$.
\section{\person{Newton-Cotes}-Formeln}

Falls die Stützstellen gleichabständig sind mit $x_0=a$ und $x_n=b$, das heißt
\begin{align}
	\label{2.1}
	x_{k+1}=x_k+h\quad\text{für } k=0,...,n-1
\end{align}
mit der Schrittweite $h=\frac{b-a}{n}$ gilt, so nennt man $Q_n(f)$ \begriff[Newton-Cotes-Formel!]{geschlossene \person{Newton-Cotes}-Formel}. Ist $a<x_0$ und $x_n<b$ und gilt \cref{2.1} mit $h=\frac{b-a}{n+2}$, so bezeichnet man $Q_n(f)$ als \begriff[Newton-Cotes-Formel!]{offene \person{Newton-Cotes}-Formel}. Im Folgenden wollen wir uns auf den Fall geschlossener \person{Newton-Cotes}-Formeln beschränken.
\section{spezielle \person{Newton-Cotes}-Formeln}
\section{Zusammengesetzte \person{Newton-Cotes}-Formeln}

Um den Quadraturfehler weiter zu reduzieren, bietet es sich unter Berücksichtigung der Abschätzung des Quadraturfehlers in \propref{2_3_1} an, das Intervall $[a,b]$ in $r$ Teilintervalle zu zerlegen und auf jedem der Teilintervalle dieselbe Quadraturformel (niedriger Ordnung) anzuwenden. Dazu wird das Intervall $[a,b]$ in $l=rn$ Elementarintervalle gleicher Länge zerlegt, wobei $n$ die Ordnung des auf jedem Teilintervall zu verwendenden Interpolationspolynoms ist. Mit $f_0,...,f_l$ werden die Funktionswerte an den Stellen $x_k=a+kh$ für $k=0,...,l$ bezeichnet, wobei $h=\frac{b-a}{l}$ die Länge des Elementarintervalls ist. Die \begriff{zusammengesetzte Trapezformel} ist dann gegeben durch:
\begin{align}
	T_h(f) = \frac{h}{2}(f_0+2f_1+2f_2+...+2f_{l-1}+f_l)\notag
\end{align}
die \begriff{zusammengesetzte \person{Simpson}-Formel} durch
\begin{align}
	S_h(f) = \frac{h}{3}(f_0+4f_1+2f_2+4f_3+...+2f_{l-2}+4f_{l-1}+f_l)\notag
\end{align}

\begin{proposition}
	\begin{enumerate}[label=(\alph*)]
		\item Für $f\in C^2[a,b]$ gilt
		\begin{align}
			\vert T_h(f) - J(f)\vert \le \frac{b-a}{12}h^2\Vert f''\Vert_\infty\notag
		\end{align}
		\item Für $f\in C^4[a,b]$ gilt
		\begin{align}
			\vert S_h(f) - J(f)\vert \le \frac{b-a}{180}h^4\Vert f^{(4)}\Vert_\infty\notag
		\end{align}
	\end{enumerate}
\end{proposition}
\begin{proof}
	Wendet man \propref{2_3_1} auf die \person{Simpson}-Formel (unter Beachtung von \cref{2.5}) für $[x_k,x_{k+2}]$ anstelle von $[a,b]$ an, so folgt
	\begin{align}
		\vert S_h(f)-J(f)\vert = \left| S_h(f)-\sum_{k=0}^{r-1} \int_{x_{2k}}^{x_{2k+2}} f(x)\diff x\right| \le \frac{1}{90} rh^5 \Vert f^{(4)}\Vert_\infty \le \frac{b-a}{2\cdot 90} h^4\Vert f^{(4)}\Vert_\infty\notag
	\end{align}
	und damit Behauptung b). Behauptung a) zeigt man auf ähnliche Weise.
\end{proof}
\section{\person{Gauss}'sche Quadraturformeln}

Wir gehen zunächst vom Interpolationsfehler (vgl. \propref{1_2_9})
\begin{align}
	f(x)-p_n(x) = \frac{f^{(n+1)}(\xi(x))}{(n+1)!}w_n(x)\quad\text{für } x\in [a,b]\notag
\end{align}
mit $w_n(x)=(x-x_0)...(x-x_n)$ aus. Bezogen auf das ganze Intervall $[a,b]$, kann man etwa $\Vert f-p_n\Vert_\infty$ oder $\Vert f-p_n\Vert_2$ als Maß für diesen Fehler verwenden. Da man über $f^{(n+1)}$ nicht verfügt, wird anstelle dessen $\Vert w_n\Vert_\infty$ oder $\Vert w_n\Vert_2$ untersucht. Dieses Fehlermaß ist offenbar nur von der Lage der Stützstellen $x_0,...,x_n\in [a,b]$ abhängig. Zur Vereinfachung beschränkt man sich zunächst auf das Intervall $[-1,1]$. Die Aufgabe, die Funktion
\begin{align}
	F_\infty :\begin{cases}
		\real^{n+1}&\to \real \\
		F_\infty(x_0,...,x_n) &\mapsto \max\limits_{x\in [-1,1]} \vert (x-x_0)...(x-x_n)\vert
	\end{cases}\notag
\end{align}
unter der Bedingung $(x_0,...,x_n)\in [-1,1]^{n+1}$ zu minimieren, hat als Lösung die Nullstellen des sogenannten \begriff{\person{Tschebyschow}-Polynoms} $T_{n+1}$ der Ordnung $n+1$. Die Funktion
\begin{align}
	F_2:\begin{cases}
		\real^{n+1} &\to \real \\
		F_2(x_0,...,x_n) &\mapsto \int_{-1}^{1} (x-x_0)^2...(x-x_n)^2\diff x
	\end{cases}\notag
\end{align}
wird unter der Bedingung $(x_0,...,x_n)\in [-1,1]^{n+1}$ durch die Nullstellen des sogenannten \begriff{\person{Legendre}-Polynoms} $P_{n+1}$ minimiert. Die \person{Legendre}-Polynome können rekursiv wie folgt definiert werden:
\begin{align}
	P_0(t) &= 1 \notag \\
	P_1(t) &= t \notag \\
	\vdots \notag \\
	(k+1)P_{k+1}(t) &= (2k+1)tP_k(t)-kP_{k-1}(t)\notag
\end{align}
für alle $t\in\real$ und $k=1,2,...$. Für $n=1$ erhält man zum Beispiel $P_2(t)=t^2-\frac{1}{3}$ mit den Nullstellen $x_{1/2}=\pm \frac{\sqrt{3}}{3}$. Das Interpolationspolynom zu diesem Stützstellen und den Stützwerten $f_0=f(x_0)$ sowie $f_1=f(x_1)$ lautet dann (in der \person{Lagrange}-Form)
\begin{align}
	q_1(x) = f_0\frac{x-x_1}{x_0-x_1}+f_1\frac{x-x_0}{x_1-x_0}\notag
\end{align}
Wegen
\begin{align}
	\int_{-1}^1 \frac{x-x_1}{x_0-x_1}\diff x = -\frac{\sqrt{3}}{2}\int_{-1}^1 \left(x-\frac{\sqrt{3}}{3}\right)\diff x&=1 \notag \\
	\int_{-1}^1 \frac{x-x_0}{x_1-x_0}\diff x &= 1\notag
\end{align}
hat man die \person{Gauss-Legendre}-Quadraturformel für das Integral $\int_{-1}^1 f(x)\diff x$ für $n=1$:
\begin{align}
	f\left(-\frac{\sqrt{3}}{3}\right) + f\left(\frac{\sqrt{3}}{3}\right)\notag
\end{align}
Man kann zeigen, dass die \person{Gauss-Legendre}-Quadraturformel  mit $n+1$ Stützstellen (also den Nullstellen von $P_{n+1}$) Polynome bis zum Grad $2n+1$ exakt integriert. Bei den geschlossenen \person{Newton-Cotes}-Formeln ist dies nur bis Grad $n$ (falls $n$ ungerade) bzw. bis zum Grad $n+1$ (falls $n$ gerade) möglich (vgl. Übungsaufgabe). Spezielle Modifikationen der \person{Gauss-Legendre}-Quadratur beziehen einen Randpunkt (\person{Gauss-Radau}) oder beide Randpunkte (\person{Gauss-Lobatto}) des Integrals mit ein.

Falls über $[a,b]$ zu integrieren ist, führt eine Variablentransformation auf ein Integral über $[-1,1]$ zum Ziel. Mit
\begin{align}
	y=\frac{2}{b-a}\left(x-\frac{a+b}{2}\right)\notag
\end{align}
ergibt sich $x=\frac{b-a}{2}y+\frac{a+b}{2}$, $\diff x=\frac{b-a}{2}\diff y$ und
\begin{align}
	\int_a^b f(x)\diff x = \frac{b-a}{2}\int_{-1}^1 f\left(\frac{b-a}{2}y+\frac{a+b}{2}\right)\diff y\notag
\end{align}

\chapter{direkte Verfahren für lineare Gleichungssysteme}
\section{\person{Gauss}'scher Algorithmus für quadratische Systeme}

\subsection{Grundform des \person{Gauss}'schen Algorithmus}

\begin{example}
	\begin{align}
		\sysdelim{.}{.}\systeme{2x_1-2x_2+4x_3=10@E_{*},x_1+3x_2+6x_3=25,-x_1+2x_2+x_3=6}\notag
	\end{align}
	$E_1$ behalten $\to E_1'$, $E_2-\frac{1}{2}E_1\to E_2'$, $E_3+\frac{1}{2}E_1\to E_3'$
	\begin{align}
		\sysdelim{.}{.}\systeme{2x_1-2x_2+4x_3=10@E'_{*},4x_2+4x_3=20,x_2+3x_3=11}\notag
	\end{align}
	$E_1'$ behalten $\to E_1''$, $E_2'$ behalten $\to E_2''$, $E_3'-\frac{1}{4}E_2'\to E_3''$
	\begin{align}
		\sysdelim{.}{.}\systeme{2x_1-2x_2+4x_3=10@E''_{*},4x_2+4x_3=20,2x_3=6}\notag
	\end{align}
	$\Rightarrow x_3=3$, $x_2=2$, $x_1=1$
\end{example}

Alle drei Systeme sind äquivalent, das heißt ihre Lösungsmengen sind gleich. Das letzte System wird \begriff{Dreieckssystem} oder System in \begriff{Zeilenstufenform} oder \begriff{gestaffeltes System} genannt.

Gegeben seien $A=(a_{ij})\in\real^{n\times n}$ und $b=(b_i)\in\real^n$. Gesucht ist, falls vorhanden, eine Lösung des linearen Gleichungssystems
\begin{align}
	\begin{array}{ccccccc}
		a_{11}x_1 & + & ... & + & a_{1n}x_n & = & b_1 \\
		\vdots & && & \vdots & \vdots & \vdots \\
		a_{n1}x_1 & + & ... & + & a_{nn}x_n & = & b_n
	\end{array}\notag
\end{align}
bzw. in Matrix-Schreibweise: $Ax=b$.

\subsubsection*{Prinzipielles Vorgehen}

\begin{enumerate}[label=\textbf{\arabic*.}]
	\item Vorwärtselimination (unter Voraussetzung der Durchführbarkeit): Schrittweise Transformation der erweiterten Koeffizientenmatrix
	\begin{align}
		(A,b) = (A^{(1)},b^{(1)})\to (A^{(2)},b^{(2)})\to ...\to (A^{(n)},b^{(n)})=(U,z)\notag
	\end{align}
	wobei $U$ eine obere Dreiecksmatrix ist. Der Eliminationsschritt $(A^{(k)},b^{(k)})\to(A^{(k+1)},b^{(k+1)})$ für $k=1,...,n-1$ verwendet die Eliminationsfaktoren
	\begin{align}
		l_{ik} = \frac{a_{ik}^{(k)}}{a_{kk}^{(k)}}\notag
	\end{align}
	um die $i$-te Zeile der neuen Matrix aus der alten Matrix zu bestimmen
	\begin{align}
		\text{neue Zeile }i &= \text{alte Zeile }i &\text{für } i=1,...,k \notag \\
		\text{neue Zeile }i &= \text{alte Zeile }i - l_{ik}\cdot\text{neue Zeile }k &\text{für } i=k+1,...,n\notag
	\end{align}
	\item Rücksubstitution (unter Voraussetzung der Durchführbarkeit): Lösung des Gleichungssystems $Ux=z$ nach $x$ für gegebenes $U,z$
\end{enumerate}

\begin{algorithm}[Vorwärtselimination]
	\proplbl{3_1_2}
	Input: $n$, $A$, $b$
\begin{lstlisting}
do k= 1, n-1
 do i = k+1, n
  %$l_{ik}$% = %$a_{ik}$% / %$a_{kk}$%
  %$b_i$% = %$b_i$% - %$l_{ik}b_k$%
  do j = k+1, n 
   %$a_{ij}$% = %$a_{ij}$% - %$l_{ik}a_{kj}$%
  end do
 end do
end do
\end{lstlisting}
	Output: $(U,z)$ und $l_{ik}$ für $i>k$. $U$ steht in der oberen Hälfte von $A$ mit Hauptdiagonale, $b$ enthält $z$, die Zahlen $l_{ik}$ lassen sich in der unteren Hälfte von $A$ abspeichern.
\end{algorithm}

\begin{algorithm}[Rücksubstitution]
	\proplbl{3_1_3}
	Input: $n$, $U$, $z$
\begin{lstlisting}
do i = n, 1, -1
 s = 0
 do j = i+1, n
  s = s + %$u_{ij}x_j$%
 end do
end do
\end{lstlisting}
	Output: $x$
\end{algorithm}

Der Aufwand bei uneingeschränkter Durchführbarkeit von \propref{3_1_2} ist $\sim \frac{2}{3}n^3$ und \propref{3_1_3} ist $\sim n^2$

\subsubsection*{Durchführbarkeit}

Der \propref{3_1_2} ist genau dann durchführbar, wenn $a_{kk}^{(k)}\neq 0$ für alle $k=1,...,n-1$ gilt. Gilt auch $a_{nn}^{(n)}\neq 0$, so folgt $u_{ii}\neq 0$ für $i=1,...,n$ und damit die Durchführbarkeit von \propref{3_1_3}.

\begin{definition}[streng diagonaldominant]
	Eine Matrix $A=(a_{ij})\in\real^{n\times n}$ heißt \begriff{streng diagonaldominant}, wenn
	\begin{align}
		\vert a_{ii}\vert > \sum_{\substack{j=1\\ j\neq i}}^{n}\vert a_{ij}\vert\quad\text{für alle } i=0,...,n\notag
	\end{align}
\end{definition}

\begin{lemma}
	Ist die Matrix $A\in\real^{n\times n}$ streng diagonaldominant, so sind \propref{3_1_2} und \propref{3_1_3} durchführbar
\end{lemma}
\begin{proof}[nicht in der Vorlesung]
	Die Matrix $A^{(1)}$ sei streng diagonaldominant. Weiter seien die Matrizen $A^{(k)}$ für ein $k\in\{1,...,n-1\}$ durch Vorwärtselimination erzeugt und streng diagonaldominant. Dies zieht $\vert a_{kk}^{(k)}\vert >0$ nach sich, so dass die Erzeugung von $A^{(k+1)}$ durch Vorwärtselimination wohldefiniert ist. Es wird nun gezeigt, dass $A^{(k+1)}$ wieder streng diagonaldominant ist. Da $A^{(1)}=A$ als streng diagonaldominant vorausgesetzt wurde, folgt dann die Durchführbarkeit der gesamten Vorwärtselimination durch vollständige Induktion. Sei $i>k$ eine Zeile der Matrix $A^{(k+1)}$. Dann hat man
	\begin{align*}
		\sum_{\substack{j=1\\ j\neq i}}^n \vert a_{ij}^{(k+1)}\vert \sum_{\substack{j=k+1\\ j\neq i}}\vert a_{ij}^{(k+1)}\vert &= \sum_{\substack{j=k+1\\ j\neq i}}^n \left|a_{ij}^{(k)} - \frac{a_{kj}^{(k)} a_{ik}^{(k)}}{a_{kk}^{(k)}}\right| \\
		&\le \sum_{\substack{j=k+1\\ j\neq i}}^n \vert a_{ij}^{(k)}\vert + \left|\frac{a_{ik}^{(k)}}{a_{kk}^{(k)}}\right| \sum_{\substack{j=k+1\\ j\neq i}}^n \vert a_{kj}^{(k)}\vert \\
		&< \vert a_{ii}^{(k)}\vert - \vert a_{ik}^{(k)}\vert + \left|\frac{a_{ik}^{(k)}}{a_{kk}^{(k)}}\right| \left( \vert a_{kk}^{(k)}\vert - \vert a_{ki}^{(k)}\vert  \right) \\
		&= \vert a_{ii}^{(k)}\vert  - \left|\frac{a_{ik}^{(k)} a_{ki}^{(k)}}{a_{kk}^{(k)}}\right| \\
		&\le \left| a_{ii}^{(k)} - \frac{a_{ik}^{(k)} a_{ki}^{(k)}}{a_{kk}^{(k)}} \right| \\
		&= \vert a_{ii}^{(k+1)}\vert
	\end{align*}
	Falls $i\le k$, so ändert sich $A^{(k+1)}$ gegenüber $A^{(k)}$ bezüglich der Zeile $i$ nicht. Also ist $A^{(k+1)}$ streng diagonaldominant und man schließt auf die Durchführbarkeit der Vorwärtselimination für $k=1,...,n-1$ und insbesondere auf $\vert a_{ii}^{(n)}\vert >0$ für $i=1,...,n$. DIe Matrix $A^{(n)}$ enthält die Matrix $U$ im oberen Dreieck, deren Diagonalelemente sind gerade $a_{11}^{(n)},...,a_{nn}^{(n)}$, also ist auch die Rücksubstitution wohldefiniert.
\end{proof}

\subsection{Pivotisierung}

Die Regularität der Matrix $A\in\real^{n\times n}$ ist zwar äquivalent zur Lösbarkeit des linearen Gleichungssystems $Ax=b$, für jeden beliebigen Vektor $b\in\real^n$, jedoch sichert die Regularität nicht die Durchführbarkeit der Grundform des \person{Gauss}'schen Algorithmus. Um die Durchführbarkeit bei regulärem $A$ zu erzwingen, kann man eine \begriff[Pivotisierung!]{Spaltenpivotisierung} der Matrix durchführen. Dabei werden in jedem Durchlauf der Vorwärtselimination auf bestimmte Weise Zeilen der Matrix $(A,b)$ vertauscht:
\begin{itemize}
	\item Bestimme $p=p(k)\in\{k,...,n\}$, sodass $\vert a_{pk}^{(k)}=\max\limits_{k\le i\le n}\vert a_{ik}^{(k)}\vert$. \\
	$k$-te Spalte von $A^{(k)}$ heißt \begriff{Pivotspalte}, $a_{pk}^{(k)}$ heißt \begriff{Pivotelement}, die Regularität von $A$ sichert dann $a_{pk}^{(k)}\neq 0$
	\item Vertausche die Zeilen $p$ und $k$ in der Matrix $(A^{(k)},b^{(k)})$. \\
	\emph{praktisch:} Zeilentausch nicht ausführen, sondern einen Permutationsvektor mitführen. \\
	\emph{formal:} Beschreibung der Zeilen- und Spaltenvertauschungen durch Permutationsmatrizen. Dazu sei $\pi:\{1,...,n\}\to\{1,...,n\}$ eine Permutation und $e_i$ bezeichne den $i$-ten kanonischen Einheitsvektor. Dann heißt $P_\pi=(e_{\pi(1)},...,e_{\pi(n)})$ \begriff{Permutationsmatrix}.
\end{itemize}

\begin{proposition}
	Ist die Matrix $A$ regulär, so ist der \person{Gauss}'sche Algorithmus mit Spaltenpivotisierung (bei exakter Arithmetik) durchführbar.
\end{proposition}

Weitere Pivotisierungstechniken sind insbesondere die \begriff[Pivotisierung!]{Zeilenpivotisierung} (in Analogie zur Spaltenpivotisierung) und die \begriff[Pivotisierung!]{vollständige Pivotisierung}.

\subsection{LU-Faktorisierung}

Der $k$-te Schritt von \propref{3_1_2} (ohne Pivotisierung) lässt sich schreiben als 
\begin{align}
	\begin{split}
	A^{(k+1)} &= L_kA^{(k)} \\
	b^{k+1} &= L_kb^{(k)}
	\end{split}
\end{align}
mit der \person{Gauss}'schen Eliminationsmatrix
\begin{align}
	L_k = \begin{blockarray}{cccccc}
	a & b & c & d & e \\
	\begin{block}{(ccccc)c}
	1 & 1 & 1 & 1 & 1 & f \\
	0 & 1 & 0 & 0 & 1 & g \\
	0 & 0 & 1 & 0 & 1 & h \\
	0 & 0 & 0 & 1 & 1 & i \\
	0 & 0 & 0 & 0 & 1 & j \\
	\end{block}
	\end{blockarray}
\end{align}

\subsection{\person{Gauss}'scher Algorithmus für trigonale Systeme}
\include{./TeX_files/Cholesky-Faktorisierung_fuer_symmetrsiche_positiv_definite_Matrizen}
\section{Lineare Quadratmittelprobleme}

Das lineare Gleichungssystem $Ax=b$ mit $A\in\real^{m\times n}$ und $b\in\real^m$ besitzt genau dann eine Lösung, wenn $\rang(A)=\rang((A,b))$. Falls $m>n$, so heißt das Gleichungssystem \begriff[lineares Gleichungssystem!]{überbestimmt}. Im Allgemeinen gilt dann $\rang(A)\le n<\rang((A,b))$, so dass das Gleichungssystem keine Lösung besitzt. Der Fall der Nichtlösbarkeit kann auch für $m\le n$ eintreten, falls $\rang(A)<m$. Anstelle des Systems $Ax=b$ betrachtet man folgende \begriff{Ersatzaufgabe}:
\begin{align}
	\label{3.8}
	\Vert Ax-b\Vert_2\to\min
\end{align}
die als \begriff{lineares Quadraturmittelproblem} bezeichnet wird. Kurz schreibt man dafür auch $Ax\cong b$.

\begin{proposition}
	\proplbl{3_3_1}
	Seien $A\in\real^{m\times n}$ und $b\in\real^m$ gegeben. Da ist das lineare Quadraturmittelproblem \cref{3.8} lösbar.
\end{proposition}
\begin{proof}
	Die restringierte Optimierungsaufgabe 
	\begin{align}
		\label{3.9}
		f(y) = \Vert y-b\Vert_2\to\min\quad\text{mit } y\in L=\{Ax\mid x\in\real^n\}
	\end{align}
	ist offenbar genau dann lösbar, wenn \cref{3.8} eine Lösung besitzt. Wegen $f(0)=\Vert b\Vert_2$ und $0\in L$ hat
	\begin{align}
		\label{3.10}
		f(y)\to\min\quad\text{mit } y\in L, \, \Vert y-b\Vert_2\le \Vert b\Vert_2
	\end{align}
	dieselbe Lösungsmenge wie \cref{3.9}. Der zulässige Bereich $\{x\in L\mid \Vert y-b\Vert_2\le \Vert b\Vert_2\}$ dieser Optimierungsaufgabe ist nicht-leer, abgeschlossen und beschränkt. Da zudem $f:\real^m\to\real$ stetig ist, besitzt \cref{3.10} nach dem Satz von \person{Weierstrass} eine Lösung. Also sind auch \cref{3.9} und \cref{3.8} lösbar.
\end{proof}

\begin{example}
	Seien
	\begin{align}
		A = \begin{henrysmatrix}
		1 \\1
		\end{henrysmatrix}\quad\text{und}\quad b=\begin{henrysmatrix}
		2 \\0
		\end{henrysmatrix}\notag
	\end{align}
	Dann ist der lineare Teilraum $L$ gegeben durch $L=\left\lbrace \begin{henrysmatrix}1\\1\end{henrysmatrix}x\Bigg| x\in\real\right\rbrace$. Anschaulich ergibt sich, dass eine Lösung $y^*\in L$ von \cref{3.9} der Bedingung $(y^*-b)\perp L$ genügen muss. Daraus folgt
	\begin{align}
		\left(\begin{henrysmatrix}
		 y_1^* \\ y_2^*
		\end{henrysmatrix} - \begin{henrysmatrix}
		2 \\ 0
		\end{henrysmatrix}\right)^T\begin{henrysmatrix}
		1\\1
		\end{henrysmatrix} x=0\quad\text{und}\quad \begin{henrysmatrix}
		y_1^* \\ y_2^*
		\end{henrysmatrix} = \begin{henrysmatrix}
		1\\1
		\end{henrysmatrix} x\notag
	\end{align}
	für alle $x\in\real$. Einzige Lösung von \cref{3.9} ist damit $y^*=(1,1)^T$. Somit ist $x^*=1$ die einzige Lösung von \cref{3.8}.
	\begin{center}
		\begin{tikzpicture}
		\draw[->] (-3,0) -- (3,0);
		\draw[->] (0,-3) -- (0,3);
		
		\draw (-3,-3) -- (3,3);
		\node at (3.5,3.5) (L) {$L$};
		
		\draw[fill=black] (2,0) circle (0.1);
		\node at (2,-0.5) (b) {$b$};
		\draw[dashed] (1,1) -- (2,0);
		\draw[fill=black] (1,1) circle (0.05);
		\node at (1,1.5) (y) {$y^*$};
		\node at (3.2,0.5) (k) {kleinster Abstand};
		\end{tikzpicture}
	\end{center}
\end{example}

\subsection{Die \person{Gauss}'schen Normalgleichungen}

\begin{proposition}
	\proplbl{3_3_3}
	Seien $A\in\real^{m\times n}$ und $b\in\real^m$ gegeben. Dann gilt:
	\begin{enumerate}[label=(\alph*)]
		\item Jede Lösung des linearen Quadraturmittelproblems \cref{3.8} löst die \\ \begriff{\person{Gauss}'schen Normalgleichungen}
		\begin{align}
			\label{3.11}
			A^TAx = A^Tb
		\end{align}
		und umgekehrt.
		\item Falls $\rang(A)=n$ (dies impliziert $m\ge n$), so ist $A^TA$ positiv definit und \cref{3.8} besitzt genau eine Lösung, nämlich $x^*=(A^TA)^{-1}A^Tb$.
		\item Falls $\rang(A)<n$, so ist $A^TA$ positiv semidefinit und singulär und \cref{3.8} besitzt unendlich viele Lösungen.
	\end{enumerate}
\end{proposition}
\begin{proof}
	\begin{enumerate}[label=(\alph*)]
		\item Die Zielfunktion $\phi:\real^n\to\real$ der zu \cref{3.8} äquivalenten Aufgabe
		\begin{align}
			\label{3.12}
			\phi(x) = \frac{1}{2}\Vert Ax-b\Vert_2^2\to\min
		\end{align}
		lässt sich schreiben als
		\begin{align}
			\phi(x) &= \frac{1}{2}(Ax-b)^T(Ax-b) \notag \\
			&= \frac{1}{2} (x^TA^TAx - 2b^TAx+b^Tb)\notag
		\end{align}
		Die notwendige Optimalitätsbedingung für \cref{3.12} lautet $\nabla\phi(x)=0$, das heißt
		\begin{align}
			A^TAx = A^Tb\notag
		\end{align}
		Also ist jede Lösung von \cref{3.8} auch eine Lösung der \person{Gauss}'schen Normalgleichungen \cref{3.11}. Da $\phi$ eine konvexe Funktion ist (wegen $\nabla^2\phi(x)=A^TA$ positiv semidefinit), ist \cref{3.11} zugleich eine hinreichende Optimalitätsbedingung, das heißt jede Lösung von \cref{3.11} löst \cref{3.8}.
		\item Sei $\rang(A)=n$. Dann hat $A$ vollen Spaltenrang und $Ax\neq 0$ für alle $x\neq 0$. Folglich gilt $x^TA^TAx=(x^TA^T)(Ax)=\Vert Ax\Vert_2^2>0$ für alle $x\neq 0$. Also ist $A$ positiv definit und damit regulär. Somit sind die \person{Gauss}'schen Normalgleichungen \cref{3.11} eindeutig lösbar, ihre Lösung ist $x^*=(A^TA)^{-1}A^Tb$. Wegen Teil (a) ist dies auch die einzige Lösung von \cref{3.8}.
		\item Sei $\rang(A)>n$. Dann gibt es $\hat{x}\neq 0$ mit $A\hat{x}=0$. Folglich ist einerseits $A$ positiv semidefinit (denn $x^TA^TAx=\Vert Ax\Vert_2^2\ge 0$) aber andererseits $A^TA\hat{x}=0$ und $A^TA$ daher singulär. Da nach \propref{3_3_1} das lineare Quadraturmittelproblem \cref{3.8} eine Lösung besitzt, muss nach Teil (a) auch \cref{3.11} lösbar sein. Aufgrund der Singularität von $A^TA$ hat \cref{3.11} unendlich viele Lösungen.
	\end{enumerate}
\end{proof}

Sei $x^*$ eine Lösung von \cref{3.8}. Dann gilt wegen \propref{3_3_3} 
\begin{align}
	0 = A^TAx^* - A^Tb = A^T(Ax^*-b)\notag
\end{align}
Dies ist äquivalent zu folgenden Aussagen
\begin{itemize}
	\item $0=x^TA^T(Ax^*-b)$
	\item $(Ax^*-b)\perp Ax$
	\item $(Ax^*-b)\perp L$
\end{itemize}

\begin{algorithm}[Prinzip des Normalgleichungsverfahrens]
	Input: $A\in\real^{m\times n}$ mit $\rang(A)=n$, $b\in\real^m$
	\begin{lstlisting}
G = transpose(A) * A
c = transpose(A) * b
compute L ! als Cholesky-Faktor von G
solve Lz=c
solve transpose(L)x = z
	\end{lstlisting}
	Output: $x$, $L$
\end{algorithm}

\begin{remark}
	Der Aufwand beträgt etwa $mn^2$ Operationen zur Berechnung der unteren Hälfte von $G$, $\frac{n^3}{3}$ für die \person{Cholesky}-Faktorisierung sowie je $n^2$ für die Lösung der Dreieckssysteme. Offenbar ist der Aufwand für kleine $n$ günstig. Nachteilig bezüglich numerischer Fehler kann sich beim Normalgleichungsverfahren die schlechte Kondition (siehe später) der Matrix $A^TA$ auswirken. Abhilfe schaffen geeignete Nachiterationen oder andere Verfahren (\person{Householder}, SVD) zur Lösung des linearen Quadraturmittelproblems.
\end{remark}

\subsection{Orthonormalisierungsverfahren nach \person{Householder}}

Ziel ist zunächst die Beschreibung eines Verfahrens zur sogenannten $QR$-Faktorisierung einer Matrix $A\in\real^{m\times n}$, das heißt es sollen Matrizen $Q\in\real^{m\times n}$ und $R\in\real^{m\times n}$ bestimmt werden, so dass
\begin{align}
	A = QR\notag
\end{align}
gilt, wobei $Q$ eine orthogonale Matrix ($Q^{-1}=Q^T$) und $R$ eine verallgemeinerte obere Dreiecksmatrix der Form
\begin{align}
	R &= \begin{henrysmatrix}
	R_1 \\ 0
	\end{henrysmatrix} \quad\text{(falls $m\ge n$)} \notag \\
	R &= (R_1,R_2) \quad\text{(falls $m < n$)} \notag
\end{align}
mit einer oberen Dreiecksmatrix $R_1\in\real^{n\times n}$ bzw. einer oberen Dreiecksmatrix $R_1\in\real^{m\times m}$ und einer Matrix $R_2\in\real^{m\times (n-m)}$ ist. Später wird die $QR$-Zerlegung zur Lösung von Quadraturmittelproblemen (für den Fall $\rang(A)=n$) eingesetzt.

\begin{proposition}
	\proplbl{1_3_6}
	Sei $w\in\real^m$ gegeben mit $w^Tw=1$. Dann ist die \begriff{\person{Householder}-Matrix}
	\begin{align}
		H = \mathbbm{1} - 2ww^T\notag
	\end{align}
	symmetrisch und orthogonal, das heißt es gilt $H=H^T=H^{-1}$.
\end{proposition}
\begin{proof}
	Offenbar gilt $H^T=\mathbbm{1}-2ww^T=H$. Weiter erhält man
	\begin{align}
		H^TH=(\mathbbm{1}-2ww^T)(\mathbbm{1}-2ww^T) = \mathbbm{1} - 4ww^T + 4w(w^Tw)w^T = 1\notag
	\end{align}
\end{proof}

Die Wirkung einer \person{Householder}-Matrix $H$ auf einen Vektor $a\in\real^m$ (bei Multiplikation mit diesem Vektor) lässt sich wie folgt veranschaulichen. Zunächst hat man
\begin{align}
	Ha=(\mathbbm{1}-2ww^T)a = a-2ww^Ta = a-(2w^Ta)w\notag
\end{align}
Wegen $(a-w^Taw)^Tw=a^Tw-w^Ta=0$ (beachte $w^Tw=1$) liegt $a-w^Taw$ auf der Ebene $\mathcal{E}=\{y\in\real^m\mid y^Tw=0\}$ und $Ha$ liegt bezüglich dieser Ebene (als Spiegelebene) spiegelbildlich zu $a$.

\begin{proposition}
	\proplbl{1_3_7}
	Es seien $a\in\real^m$ mit $a\notin\Span(e_1)$ und 
	\begin{align}
		w = \frac{a+pe_1}{\Vert a+pe_1\Vert_2}\quad\text{mit } p\in\{\Vert a\Vert_2,\, -\Vert a\Vert_2\}\notag
	\end{align}
	gegeben. Dann gilt
	\begin{align}
		Ha = -pe\notag
	\end{align}
\end{proposition}
\begin{proof}
	Wegen $a\notin\Span(e_1)$ folgt $a+pe_1\neq 0$. Also ist $w$ wohldefiniert mit $w^Tw=1$ und man erhält
	\begin{align}
		\label{3.13}
		Ha = (\mathbbm{1}-2ww^T)a = a-(2w^Ta)w = a-2\frac{a^T(a+pe_1)}{\Vert a+pe_1\Vert}\frac{a+pe_1}{\Vert a+pe_1\Vert_2}
	\end{align}
	Da $p\in\{\Vert a\Vert_2,\, -\Vert a\Vert_2\}$, gilt
	\begin{align}
		\Vert a+pe_1\Vert_2^2 = a^Ta+2pa^Te_1+p^2 = 2a^T(a+pe_1)\notag
	\end{align}
	Deshalb liefert \cref{3.13} $Ha = a-(a+pe_1) = -pe_1$.
\end{proof}

\propref{1_3_7} wird für die schrittweise \person{Householder}-Transformation einer Matrix $A\in\real^{n\times m}$ in eine verallgemeinerte obere Dreiecksmatrix ausgenutzt. Dazu sei $A^{(1)}=A$ eine Matrix von Rang $n$. Ohne Beschränkung der Allgemeinheit gelte $m>n$. Weiter sei $A^{(k)}$ für ein $k\in\{1,...,n+1\}$ in der Form
\begin{align}
	A^{(k)} = \begin{pmatrix}
	a_{11}^{k} & \dots & a_{1k}^{(k)} & \dots & a_{1n}^{(k)} \\
	& \ddots & \vdots & & \vdots \\
	&& a_{kk}^{(k)} &  \dots & a_{kn}^{(k)} \\
	&& \vdots & & \vdots \\
	&& a_{mk}^{(k)} & \dots & a_{mn}^{(k)} 
	\end{pmatrix}\notag
\end{align}
gegeben. Der Vektor $a^k=(0,...,0a_{kk}^{(k)},...,a_{mk}^{(k)})\in\real^m$ übernimmt die Rolle von $a$. Er soll durch Multiplikation mit $H_k\in\real^{m\times m}$ auf $p_ke_k$ transformiert werden, wobei
\begin{align}
	H_k &= \mathbbm{1} - 2w_kw_k^T \notag \\
	w_k &= \frac{a^k+p_ke_k}{\Vert a^k+p_ke_k\Vert_2} \notag \\
	p_k &\in \{\Vert a^k\Vert_2,-\Vert a^k\Vert_2\} \notag
\end{align}
Zur Vermeidung von Stellenauslöschung in $a^k+p_ke_k$ wird man
\begin{align}
	p_k = \begin{cases}
	\Vert a_k\Vert_2 & \text{falls } a_{kk}^{(k)}\ge 0 \\
	-\Vert a_k\Vert_2 & \text{falls } a_{kk}^{(k)}<0
	\end{cases}\notag
\end{align}
wählen. Die Operation $A^{(k)}\mapsto H_kA^{(k)}$ lässt die ersten $k-1$ Zeilen und Spalten der Matrix $A^{(k)}$ unverändert und es gilt:
\begin{align}
	H_kA^{(k)} = A^{(k+1)}  = \begin{pmatrix}
	a_{11}^{(k)} & \dots & a_{1,k-1}^{(k)} & a_{1k}^{(k+1)} & a_{1,k+1}^{(k+1)} & \dots & a_{1n}^{(k+1)} \\
	& \ddots & \vdots  & \vdots & \vdots & & \vdots \\
	&& a_{k-1,k-1}^{(k)} & a_{k-1,k}^{(k+1)} & a_{k-1,k+1}^{(k+1)} & \dots & a_{k-1,n}^{(k+1)} \\
	&&& a_{kk}^{(k+1)} & a_{k,k+1}^{(k+1)} & \dots & a_{kn}^{(k+1)} \\
	&&&& a_{k+1,k+1}^{(k+1)} & \dots & a_{k+1,n}^{(k+1)} \\
	&&&& a_{m,k+1}^{(k+1)} & \dots & a_{mn}^{(k+1)}
	\end{pmatrix} \notag
\end{align}
speziell mit $a_{kk}^{(k+1)}=-p_k$. Dabei garantiert die Bedingung $\rang\left(A^{(k)}\right)=n$ dasselbe für den Rang von $A^{(k+1)}$. Die Hintereinanderausführung von \person{Householder-Transformationen} liefert
\begin{align}
	\label{3.14}
	R = A^{(n+1)} = H_nH_{n-1}\dots H_1A
\end{align}
Dabei ist $R$ eine verallgemeinerte obere Dreiecksmatrix. Wegen \propref{1_3_6} existiert
\begin{align}
	\label{3.15}
	Q = (H_n\dots H_1)^{-1}
\end{align}
und es gilt
\begin{align}
	Q &= H_1^{-1}\dots H_n^{-1} = H_1\dots H_n \notag \\
	Q^TQ &= (H_n^T\dots H_1^T)(H_1\dots H_n) = \mathbbm{1}\notag
\end{align}
das heißt $Q$ ist orthogonal. Wegen \cref{3.14} und \cref{3.15} folgt schließlich noch $A=QR$.

\begin{proposition}
	Sei $A\in\real^{m\times n}$ mit $m\ge n=\rang(A)$ gegeben. Dann gibt es eine orthogonale Matrix $Q\in\real^{m\times m}$ und eine verallgemeinerte Dreiecksmatrix
	\begin{align}
		R = \begin{henrysmatrix}
			R_1 \\ \mathbb{0}
		\end{henrysmatrix}\in\real^{m\times n}\notag
	\end{align}
	mit einer regulären oberen Dreiecksmatrix $R_1\in\real^{n\times n}$, so dass $A=QR$.
\end{proposition}

\begin{proposition}
	Seien $A\in\real^{m\times n}$ mit $m\ge n=\rang(A)$ und $b\in\real^m$ gegeben. Weiter seien Matrizen $Q$ und $R$ bzw. $R_1$ aus der $QR$-Zerlegung von $A$ bekannt und Vektoren $y_1\in\real^n$ und $y_2\in\real^{m-n}$ so gegeben , dass
	\begin{align}
		\begin{henrysmatrix}
			y_1 \\ y_2
		\end{henrysmatrix} = Q ^Tb\notag
	\end{align}
	gilt. Dann ist das lineare Quadraturmittelproblem \cref{3.8} äquivalent zum linearen Gleichungssystem
	\begin{align}
		R_1x=y_1\notag
	\end{align}
\end{proposition}
\begin{proof}
	Wegen $A=QR$ und $QQ^T=\mathbbm{1}$ gilt
	\begin{align}
		\Vert Ax-b\Vert_2^2 = \Vert QRx-QQ^Tb\Vert_2^2 = \Vert Q(Rx-Q^Tb)\Vert_2^2\notag
	\end{align}
	Da $\Vert Qz\Vert_2^2 = z^TQ^TQz=z^Tz=\Vert z\Vert_2^2$ für beliebige $z\in\real^m$ ist, folgt
	\begin{align}
		\Vert Ax-b\Vert_2^2 = \left\Vert \begin{henrysmatrix}
			R_1x \\ \mathbb{0}
		\end{henrysmatrix} - \begin{henrysmatrix}
			y_1 \\ y_2
		\end{henrysmatrix}\right\Vert_2^2 = \Vert R_1x-y_1\Vert_2^2 + \Vert y_2\Vert_2^2 \notag
	\end{align}
	Also nimmt $\Vert Ax-b\Vert_2^2$ sein Minimum genau dann an, wenn $x$ das lineare Gleichungssystem $R_1x=y_1$ löst.
\end{proof}

\subsection{Anwendung in der Ausgleichsrechnung}

Gegeben seien Messpunkte $(t_i,y_i)\in\real\times\real$ für $i=1,...,m$ mit $t_i\neq t_j$ für $i\neq j$. Weiter seien sogenannte \begriff{Basisfunktionen} $\phi_j:\real\to\real$ und die Funktion $f:\real\times\real^n\to\real$ durch
\begin{align}
	f(t,x) = \sum_{j=1}^{n} x_j\phi_j(t)\quad\text{für } (t,x)\in\real\times\real^n\notag
\end{align}
gegeben. Gesucht ist ein  Parametervektor $x^\ast=(x_1,...,x_n)^T$, so dass
\begin{align}
	f(t_i,x^\ast) \approx y_i\quad\text{für } i=1,...,m\notag
\end{align}
Eine Möglichkeit ein solches $x^\ast$ zu bestimmen ist die Lösung des Optimierungsproblems (Ausgleichsproblems)
\begin{align}
	\label{3.16}
	\sum_{i=1}^{m} (y_i - f(t_i,x))^2\to\min
\end{align}
Mit 
\begin{align}
	A = \begin{pmatrix}
		\phi_1(t_1) & \dots & \phi_n(t_1) \\
		\vdots && \vdots \\
		\phi_1(t_m) & \dots & \phi_n(t_m)
	\end{pmatrix} \quad\text{und}\quad b= \begin{henrysmatrix}
		y_1 \\ \vdots  \\  y_m
	\end{henrysmatrix}\notag
\end{align}
gilt $r(x)=\Vert ax-b\Vert_2^2$, man beachte $y_i-f(t_i,x)=y_i-\sum_{j=1}^n x_j\phi_j(t_i)=y_i-A_ix$, wobei $A_i$ die $i$-te Zeile von $A$ bezeichnet. Also ist \cref{3.16} ein lineares Quadraturmittelproblem.

\begin{example}[Ausgleichsgerade]
	Seien $m=3$ und $n=2$. Es seien $(t_1,y_1)=(0,1)$, $(t_2,y_2)=(3,8)$, $(t_3,y_3)=(4,10)$ und $\phi_1(t)=1$, $\phi_2(t)=t$ für $t\in\real$. Dann ist $f(t,x)=x_1+tx_2$, 
	\begin{align}
		A = \begin{pmatrix}
			1 & 0 \\ 1 & 3 \\ 1 & 4
		\end{pmatrix}\quad\text{und}\quad b = \begin{henrysmatrix}
			1 \\ 8 \\ 10
		\end{henrysmatrix} \notag
	\end{align}
	und das Ausgleichsproblem \cref{3.16} hat die Lösung $x^\ast=(1.0385..., 2.2692...)^T$.
\end{example}

\section{Kondition linearer Gleichungssysteme}

Seien $A\in\real^{n\times n}$ und $b\in\real^n$ mit $b\neq 0$ gegeben. Es stellt sich die Frage, wie sich die Fehler in $A$ bzw. $b$ auf die Lösung $x=A^{-1}b$ des linearen Gleichungssystems $Ax=b$ auswirken. Dazu seien $\Delta A\in\real^{n\times n}$ bzw. $\Delta b\in\real^n$ Störungen kleiner Norm, insbesondere soll $A+\Delta A$ noch regulär sein. Weiter sei
\begin{align}
	\Delta x = (A+\Delta A)^{-1}(b+\Delta b) - A^{-1}b\notag
\end{align}
der absolute Fehler zwischen den Lösungen des gestörten und des ungestörten Gleichungssystems in Abhängigkeit von den Fehlern $\Delta A$ und $\Delta b$ der Eingangsdaten $A$ und $b$. Es wird nun eine obere Schranke für de relativen Fehler 
\begin{align}
	\frac{\Vert \Delta x\Vert}{\Vert x\Vert}\notag
\end{align}
in Abhängigkeit von den relativen Fehlern der Eingangsdaten $\frac{\Vert \Delta A\Vert}{\Vert A\Vert}$ und $\frac{\Vert \Delta b\Vert}{\Vert b\Vert}$ gesucht.

\subsection{Normen}

\begin{proposition}
	Sei $\Vert\cdot\Vert:\real^n\to [0,\infty)$ eine Vektornorm. Dann ist durch
	\begin{align}
		\Vert A\Vert_\ast = \sup_{\substack{x\in\real^n \\ x\neq 0}} \frac{\Vert Ax\Vert}{\Vert x\Vert} \quad\forall A\in\real^{n\times n}\notag
	\end{align}
	eine \begriff{Matrixnorm} $\Vert \cdot\Vert_\ast:\real^{n\times n}\to[0,\infty)$ definiert. Diese der Vektornorm \begriff[Matrixnorm!]{zugeordnete Matrixnorm} ist mit der Vektornorm \begriff[Matrixnorm!]{verträglich}, das heißt
	\begin{align}
		\Vert Ax\Vert \le \Vert A\Vert_\ast\Vert x\Vert\quad\forall A\in\real^{n\times n}\text{ und } b\in\real^n\notag
	\end{align}
	\begriff[Matrixnorm!]{submultiplikativ}, das heißt
	\begin{align}
		\Vert A\cdot B\Vert_\ast \le \Vert A\Vert_\ast\cdot \Vert B \Vert_\ast\quad\forall A,B\in\real^{n\times n}\notag
	\end{align}
	und es gilt $\Vert \mathbbm{1}\Vert_\ast = 1$.
\end{proposition}

Beispiele für eine Vektornorm und eine zugeordnete Matrixnorm sind:
\begin{itemize}
	\item Der \begriff{Maximum-Norm} $\Vert x\Vert_\infty = \max_{1\le i\le n} \vert x_i\vert$ ist die \begriff{Zeilensummen-Norm} $\Vert A\Vert_\infty = \max_{1\le i\le n}\sum_{k=1}^{n}\vert a_{ik}\vert$ zugeordnet.
	\item Der \begriff{Summen-Norm} $\Vert x\Vert_1=\sum_{i=1}^n \vert x_i\vert$ ist die \begriff{Spaltensummen-Norm} $\Vert A\Vert_1 = \max_{1\le k\le n}\sum_{i=1}^n \vert a_{ik}\vert$ zugeordnet.
	\item Der \begriff{euklidischen Norm} $\Vert x\Vert_2 = \sqrt{\sum_{i=1}^n x_i^2}$ ist die \begriff{Spektralnorm} $\Vert A\Vert_2 = \sqrt{\rho(A^TA)}$ zugeordnet, wobei $\rho(B)=\max\{\vert \lambda\vert\mid \lambda \text{ ist Eigenwert von } B\}$. Also ist $\Vert A\Vert_2^2$ gleich dem betragsgrößten Eigenwert von $A^TA$.
\end{itemize}

\subsection{Störungslemma}

\begin{lemma}[\person{von Neumann}'sches Störungslemma]
	Seien $\Vert\cdot\Vert$ eine Vektornorm im $\real^n$ bzw. die zugeordnete Matrixnorm und $B\in\real^{n\times n}$ mit $\Vert B\Vert<1$. Dann ist $\mathbbm{1}+B$ regulär und es gilt
	\begin{align}
		\Vert (\mathbbm{1}+B)^{-1}\Vert \le \frac{1}{1-\Vert B\Vert}\notag
	\end{align}
\end{lemma}
\begin{proof}
	Mit der Dreiecksungleichung folgt für jedes $x\in\real^n$
	\begin{align}
		\label{3.17}
		\begin{split}
			\Vert (\mathbbm{1}+B)x\Vert &= \Vert x+Bx\Vert \\
			&\ge \Vert x\Vert - \Vert Bx\Vert \\
			&\ge \Vert x\Vert - \Vert B\Vert\Vert x\Vert \\
			&= \Vert x\Vert (1-\Vert B\Vert) \\
			&> 0
		\end{split}
	\end{align}
	Also gilt $(\mathbbm{1}+B)x=0$ genau dann, wenn $x=0$. Somit ist $\mathbbm{1}+B$ regulär. Aus \cref{3.17} hat man
	\begin{align}
		\Vert y\Vert = \Vert (\mathbbm{1}+B)(\mathbbm{1}-B)^{-1}y\Vert \ge (1-\Vert B\Vert)\Vert (\mathbbm{1}+B)^{-1}y\Vert \notag
	\end{align}
	und damit
	\begin{align}
		\frac{\Vert (\mathbbm{1}+B)^{-1}y\Vert}{\Vert y\Vert} \le \frac{1}{1-\Vert B\Vert}\notag
	\end{align}
	für alle $y\in\real^n\backslash\{0\}$. Dies zieht unter Beachtung der Definition der zugeordneten Matrixnorm die zweite Behauptung des Lemmas nach sich.
\end{proof}

\subsection{Kondition}

\chapter{Kondition von Aufgaben und Stabilität von Algorithmen}
\section{Maschinenzahlen und Rundungsfehler}

Ein Computer kann nur endlich viele Maschinenzahlen in normalisierter Gleitpunktdarstellung 
\begin{align}
	z = \sigma\cdot d_0d_1...d_{t-1}\cdot b^e\notag
\end{align}
exakt speichern, wobei
\begin{itemize}
	\item $b\in\natur$ mit $b\ge 2$ die Basis
	\item $d_0d_1...d_{t-1}$ mit $d_0,d_1,...,d_{t-1}\in\{0,...,b-1\}$ die Mantisse mit den Ziffern $d_i$
	\item $t\in\natur$ mit $t\ge 1$ die Mantissenlänge
	\item $e\in\natur$ mit $-m\le e\le M$ der Exponent
	\item $\sigma\in\{+1,-1\}$ das Vorzeichen
\end{itemize}
bedeuten. Zusätzlich ist 0 eine Maschinenzahl. Mit $\mathbb{M}=\mathbb{M}(b,t,m,M)$ wird die Menge aller Maschinenzahlen bezeichnet. Jede andere Zahl $x\in\real$, die im Computer gespeichert werden soll (auch Zwischenergebnisse), wird vorher auf eine Zahl $\rd(x)\in\mathbb{M}$ so gerundet, dass der durch die Rundung entstehende relative Fehler durch
\begin{align}
	\frac{\vert\rd(x)-x\vert}{\vert x\vert} = \min\limits_{z\in\mathbb{M}}\frac{\vert z-x\vert}{\vert x\vert}\quad\text{für } x\in\real\setminus\mathbb{M}\notag
\end{align}
gegeben ist. 

\begin{lemma}
	Für jedes $x\in\real\backslash\{0\}$ mit $b^{-m}\le \vert x\vert\le b^M$ gilt
	\begin{align}
		\frac{\vert\rd(x)-x\vert}{\vert x\vert} \le \text{\eps} = \frac{1}{2}b^{1-t}\notag
	\end{align}
\end{lemma}
\begin{proof}
	Ohne Beschränkung der Allgemeinheit sei $x>0$. Dann gibt es Zahlen $e\in\whole$ mit $m\le e\le M$ und eine (gegebenenfalls unendliche) Ziffernfolge $(x_k)\subset \{0,...,b-1\}$, so dass 
	\begin{align}
		x = (x_0x_1...x_{t-1}x_tx_{t+1}...)\cdot b^e\notag
	\end{align}
	Damit folgt
	\begin{align}
		\vert\rd(x)-x\vert &\le \frac{b}{2}b^{e-t} \notag \\
		\frac{\vert\rd(x)-x\vert}{\vert x\vert} &\le \frac{b^{e-t+1}}{2b^e}\le \frac{1}{2}b^{1-t} \notag
	\end{align}
\end{proof}

Die Zahl \eps wird als \begriff{Maschinengenauigkeit} bezeichnet und gibt den maximalen relativen Rundungsfehler für $x\in [-b^m,b^M]$ an.
\section{Fehleranalyse}

\subsection{Die Kondition einer Aufgabe}

Unter Aufgabe wird hier die Auswertung einer zumindest stetig differenzierbaren Abbildung
\begin{align}
	\Phi: D\to\real^n\notag
\end{align}
verstanden. Die Lösung der Aufgabe für ein Argument $a\in D\subset\real^n$ besteht also darin, $\Phi(a)$ zu ermitteln. Wir interessieren uns nun für die Frage, welchen Einfluss ein Fehler in $a$ (also die Verwendung der Maschinenzahl $\rd(a)$ statt $a$) auf das Ergebnis $\Phi(a)$ bei ansonsten exakter Rechnung hat. Dazu bezeichne $\tilde{a}\in D$ das fehlerbehaftete Argument. Für $\tilde{a}$ nahe bei $a$ erhält man aus der \person{Taylor}-Formel
\begin{align}
	\Phi(\tilde{a}) - \Phi(a) \approx\nabla\Phi(a)^T(\tilde{a}-a)= \frac{\partial \Phi_i(a)}{\partial a_j}_{\substack{i=1,...,m \\ j=1,...,n}} (\tilde{a}-a)\notag
\end{align}
und damit (unter der Bedingung $a_j\neq 0$ und $\Phi(a)_i\neq 0$)
\begin{align}
	\label{4.1}
	\frac{\Phi_i(\tilde{a}) - \Phi_i(a)}{\Phi_i(a)} \approx \frac{1}{\Phi_i(a)}\sum_{j=1}^{n}\frac{\partial \Phi_i(a)}{\partial a_j} (\tilde{a_j}-a_j) = \sum_{j=1}^n \frac{a_j}{\Phi_i(a)}\frac{\partial\Phi_i(a)}{\partial a_j}\frac{\tilde{a_j}-a_j}{a_j}
\end{align}

\subsection{Stabilität von Algorithmen}

\chapter{\person{Newton}-Verfahren zur Lösung nichtlinearer Gleichungssysteme}
\section{Das \person{Newton}-Verfahren}
\section{Gedämpftes \person{Newton}-Verfahren}

Es wird eine Möglichkeit zur Erreichung globaler Konvergenzeigenschaften des \person{Newton}-Verfahrens vorgestellt. Dazu verwenden wird die Funktion
\begin{align}
	\upphi(x): \begin{cases}
		\real^n&\to \real \\
		x&\mapsto \frac{1}{2}\Vert F(x)\Vert_2^2
	\end{cases}\notag
\end{align}
zur Beurteilung der Güte der Näherung $x$. Falls $F'(x)$ regulär und $d(x)\in\real^n$ die \person{Newton}-Richtung im Punkt $x$ ist, das heißt die Gleichung $F(x)+F'(x)d=0$ löst, dann folgt
\begin{align}
	\label{5.6}
	\upphi'(x)d(x) = F(x)^TF'(x)d(x) = F(x)^TF'(x)\left(-F'(x)^{-1}F(x)\right) = -2\upphi(x)< 0
\end{align}
und
\begin{align}
	\label{5.7}
	\begin{split}
		\upphi(x+td(x)) &= \upphi(x) + t\upphi'(x)d(x)+o(t) \\
		&= \upphi(x) - 2t\upphi(x) + o(t) \\
		&= (1-2t)\upphi(x) + o(t)
	\end{split}
\end{align}
das heißt $\upphi(x+td(x))<\upphi(x)$ für alle $t>0$ hinreichend klein. Die \person{Newton}-Richtung $d(x)$ ist also eine Abstiegsrichtung von $\upphi$ im Punkt $x$. Die Idee besteht nun darin, eine Iteration der Form
\begin{align}
	x^{k+1} = x^k + t_kd^k\notag
\end{align}
durchzuführen, wobei $d^k=d(x^k)$ die \person{Newton}-Richtung im Punkt $x^k$ und $t_k>0$ eine Schrittweite ist, die so gewählt wird, dass zumindest $\upphi(x^{k+1})<\upphi(x^k)$ gilt. Zur Bestimmung einer geeigneten Schrittweite wird im folgenden Algorithmus die \person{Armijo}-Schrittweitenstrategie verwendet. Dazu sei
\begin{align}
	S = \left\lbrace 2^{-i}\mid i\in\natur\right\rbrace = \left\lbrace 1, \frac{1}{2}, \frac{1}{4}, \frac{1}{8}, \dots\right\rbrace \notag
\end{align}

\begin{algorithm}
	\proplbl{5_2_1}
	Input: $x^0\in\real^n$, $\epsilon\ge 0$, $q\in (0,1)$ und $F$ in geeigneter Form
	\begin{lstlisting}
k = 0
do while %$\Vert F(x^k)\Vert>\epsilon$%
 compute %$d^k$%
 ! als Loesung von %$F'(x^k)d^k+F(x^k)=0$%
 compute %$t_k$% = %$\max\{t\in S\mid 
 \upphi(x^k+td^k)\le (1-qt)\upphi(x^k)\}$%
 %$x^{k+1}$% = %$x^k$% + %$t_kd^k$%
 k = k + 1
end do
	\end{lstlisting}
	Output: $x^k$
\end{algorithm}

Zur Formulierung des folgenden Satzes wird die Niveaumenge
\begin{align}
	W(x^0) = \{x\in\real^n\mid \upphi(x) \le \upphi(x^0)\}\notag
\end{align}
benötigt.

\begin{proposition}
	Es sei $F:\real^n\to\real^n$ differenzierbar. Weiter sei $F':\real^n\to\real^{n\times n}$ lokal \person{Lipschitz}-stetig und $F'$ regulär für alle $x\in W(x^0)$. Dann ist der \propref{5_2_1} wohldefiniert. Falls die vom Algorithmus erzeugte Folge $\{x^k\}$ eine gegen $x^\ast$ konvergente Teilfolge besitzt, so gilt $F(x^\ast)=0$ und es gibt $k_0\in\natur$, so dass $t_k=1$ für alle $k\ge k_0$ (Übergang ins ungedämpfte \person{Newton}-Verfahren). Außerdem konvergiert $\{x^k\}$ dann Q-quadratisch gegen $x^\ast$.
\end{proposition}
\begin{proof}
	Sei $x^k\in W(x^0)$. Dann folgt nach Voraussetzung die Regularität von $F'(x^k)$. Somit ist die \person{Newton}-Richtung $d^k=d(x^k)$ und (unter Beachtung von \cref{5.7}) die Schrittweite $t_k\in S$ wohldefiniert. Also gilt $\upphi(x^{k+1})<\upphi(x^k)$ und somit $x^{k+1}\in W(x^0)$. Die Wohldefiniertheit des Algorithmus folgt damit induktiv, da $x^0\in W(x^0)$. \\
	Wir zeigen nun, dass $F(x^\ast)=0$. Sei $\delta>0$ zunächst beliebig gewählt. Mit \propref{5_0_1} und \cref{5.6} folgt
	\begin{align}
		\label{5.8}
		\begin{split}
			\upphi(x+td(x)) &= \upphi(x) + t\upphi'(x)d(x) + \int_0^1 \Big(\upphi'(x+std(x)) - \upphi'(x)\Big)td(x)\diff s \\
			&= (1-2t)\upphi(x) + \int_0^1 \Big(\upphi'(x+std(x)) - \upphi'(x)\Big)td(x)\diff s
		\end{split}
	\end{align}
	für alle $x\in B(x^\ast,\delta)$. Nach Voraussetzung sind $F$ und $F'$ stetig in $\real^n$. Wegen $x^\ast\in W(x^0)$ ist nach Voraussetzung $F'(x^\ast)$ regulär. Wir können damit $\delta>0$ klein genug voraussetzen, so dass die Abbildung $x\mapsto d(x)=-F'(x)^{-1}F(x)$ stetig in $B(x^\ast,\delta)$ ist. Daher gibt es $c>0$ mit 
	\begin{align}
		\label{5.9}
		\Vert d(x)\Vert \le c\quad\forall x\in B(x^\ast,\delta)
	\end{align}
	Die lokale \person{Lipschitz}-Stetigkeit von $F'$ zieht die lokale \person{Lipschitz}-Stetigkeit von $\upphi'=F^TF'$ nach sich, das heißt es gibt ein $L>0$, so dass
	\begin{align}
		\label{5.10}
		\Vert \upphi'(x) - \upphi'(y)\Vert \le L\Vert x-y\Vert\quad\forall x,y\in B(x^\ast,\delta)
	\end{align}
	Da zumindest eine Teilfolge $\{x^k\}_N$ von $\{x^k\}$ gegen $x^\ast$ konvergiert, gibt es wegen \cref{5.9} ein $\overline{t}>0$ und ein $k_0\in\natur$, so dass
	\begin{align}
		\label{5.11}
		x^k, x^k+td(x^k)\in B(x^\ast,\delta)\quad\forall k\in N\text{ mit } k\ge k_0\text{ und alle } t\in [0,\overline{t}] 
	\end{align}
	Somit liefert \cref{5.8} unter Beachtung von \cref{5.9} und \cref{5.10}
	\begin{align}
		\upphi(x^k+td(x^k)) &= (1-2t) + \max\limits_{s\in [0,1]}\left\lbrace \Vert \upphi'(x^k+std(x^k)) - \upphi'(x^k)\Vert \right\rbrace t\Vert d(x^k)\Vert \notag\\
		&= (1-2t)\upphi(x^k) + t^2Lc^2 \notag
	\end{align}
	für alle $t\in [0,\overline{t}]$ und alle $k\in N$ mit $k\ge k_0$. Um die Schrittweitenbedingung vom \propref{5_2_1} zu realisieren, muss $t_k$ als größtes Element aus $S$ bestimmt werden, so dass
	\begin{align}
		\upphi(x^k + t_kd(x^k)) \le (1-qt_k)\upphi(x^k)\notag
	\end{align}
	gilt. Daraus folgt dann für $k\in N$ und $k\ge k_0$
	\begin{align}
		t_k\ge \min\left\lbrace \overline{t},\frac{(2-q)\upphi'(x^k)}{2Lc^2}\right\rbrace \notag
	\end{align}
	Angenommen $\upphi(x^\ast)>0$. Dann zieht dies $\upphi(x^k)\ge \frac{1}{2}\upphi(x^\ast)$ für unendlich viele $k\in N$ nach sich. Also gilt
	\begin{align}
		t_k\ge \hat{t} = \min\left\lbrace \overline{t},\frac{(2-q)\upphi'(x^\ast)}{4Lc^2}\right\rbrace >0\notag
	\end{align}
	für unendlich viele $k\in N$. Da $\{\upphi(x^k)\}$ monoton fällt und unendlich oft
	\begin{align}
		\upphi(x^{k+1}) = \upphi(x^k+t_kd(x^k)) \le (1-qt_k)\upphi(x^k) \le (1-q\hat{t})\upphi(x^k)\notag
	\end{align}
	erfüllt ist, muss $\{\upphi(x^k)\}$ gegen 0 konvergieren. Die Stetigkeit von $\upphi$ zieht dann $\upphi(x^\ast)=0$ (und damit einen Widerspruch zur Annahme) nach sich. Also ist $x^\ast$ eine Nullstelle von $F$. \\
	Nun wird der Übergang ins ungedämpfte ($t_k=1$) \person{Newton}-Verfahren gezeigt. Mit \propref{5_0_1} erhält man
	\begin{align}
		\label{5.12}
		F(x^k+d^k) = F(x^k) + F'(x^k)d^k + \int_0^1 \Big(F'(x^k+sd^k) - F'(x^k)\Big)d^k\diff s
	\end{align}
	Mit den getroffenen Glattheits- und Regularitätsvoraussetzungen ist $F'(x)$ regulär für alle $x\in B(x^\ast,\delta)$ (für $\delta$ hinreichend klein) und es gibt Zahlen $L_0,M>0$, so dass
	\begin{align}
		\Vert F'(x) - F'(y)\Vert &\le L_0\Vert x-y\Vert \notag \\
		\Vert F'(x)^{-1}\Vert &\le M\notag
	\end{align}
	für alle $x,y\in B(x^\ast,\delta)$ gilt. Für $k\in N$ hinreichend groß folgt (unter Beachtung von $F(x^k) + F'(x^k)d^k=0$)
	\begin{align}
		\label{5.13}
		\Vert d^k\Vert \le \Vert F'(x^k)^{-1}\Vert\Vert F(x^k)\Vert \le M\Vert F(x^k)\Vert
	\end{align}
	und (unter Beachtung von $F(x^\ast)=0$ und $\lim_{k\in N}x^k=x^\ast$)
	\begin{align}
		\Vert F'(x^k+sd^k) - F'(x^k)\Vert \le L_0\Vert d^k\Vert\le L_0M\Vert F(x^k)\Vert\quad\forall s\in [0,1]\notag
	\end{align}
	Aus \cref{5.12} erhält man damit für $k\in N$ hinreichend groß
	\begin{align}
		\Vert F(x^k+d^k)\Vert \le\max\limits_{s\in [0,1]} \left\lbrace \Vert F'(x^k + sd^k) - F'(x^k)\Vert\right\rbrace \Vert d^k\Vert\le L_0M^2\Vert F(x^k)\Vert^2\notag
	\end{align}
	Wählt man $\delta>0$ auch so klein, dass $\Vert F(x)\Vert\le L_0^{-1}M^{-2}\sqrt{1-q}$ für alle $x\in B(x^\ast,\delta)$, dann folgt weiter
	\begin{align}
		\Vert F(x^k+d^k)\Vert \le \sqrt{1-q}\Vert F(x^k)\Vert\notag 
	\end{align}
	und damit
	\begin{align}
		\label{5.14}
		\upphi(x^k+d^k) \le (1-q)\upphi(x^k)
	\end{align}
	Für $\delta>0$ hinreichend klein zieht also $x^k\in B(x^\ast,\delta)$ nach sich, dass $t_k=1$ und wegen \cref{5.13} auch dass $x^{k+1}=x^k+d^k\in B(x^\ast,\delta)$. \\
	Da unendlich viele Iterierte der Folge $\{x^k\}$ in der Kugel $B(x^\ast,\delta)$ liegen, gilt \cref{5.14} für unendlich viele $k\in \natur$. Wegen $\upphi(x^{k+1})<\upphi(x^k)$ für alle $k\in\natur$, folgt $\lim_{k\to\infty}\upphi(x^k)=0$. Da $\upphi$ stetig ist, muss $x^\ast$ eine Nullstelle von $\upphi$ und damit von $F$ sein. \propref{5_1_3} liefert damit schließlich die Q-quadratische Konvergenz der Folge $\{x^k\}$ gegen $x^\ast$.
\end{proof}


\chapter{lineare Optimierung}
\section{Ecken und ihre Charakterisierung}

\begin{definition}[konvex, konvexes Polyeder]
	Eine Menge $G\subseteq\real^n$ heißt \begriff{konvex}, wenn
	\begin{align}
		\lambda x + (1-\lambda)y\in G
	\end{align}
	für beliebige $x,y,\lambda\in G\times G\times (0,1)$ gilt. Gilt mit gewissen $d\in\real^p$ und $C\in\real^{p\times n}$, dass
	\begin{align}
		G=\{x\in\real^n\mid Cx\le d\}\notag
	\end{align}
	so heißt $G$ \begriff{konvexes Polyeder}.
\end{definition}

\begin{definition}[Ecke]
	Seien $G\subseteq\real^n$ und $z\in G$ gegeben. Dann heißt $z$ \begriff{Ecke} von $G$, wenn 
	\begin{align}
		\frac{1}{2}x + \frac{1}{2}y = z\Rightarrow x=y=z\notag
	\end{align}
	für alle $x,y\in G$ gilt.
\end{definition}

\begin{definition}[Basislösung]
	\proplbl{6_1_3}
	Sei $\rang(A)=m$. Ein Vektor $\overline{x}$ mit $A\overline{x}=b$ heißt \begriff{Basislösung} von $G_P$, wenn es Indexmengen $B,N\subseteq I=\{1,...,n\}$ gibt so dass
	\begin{itemize}
		\item $B\cup N=I$, $B\cap N=\emptyset$ und $\vert B\vert =m$
		\item $A_B$ regulär und
		\item $\overline{x}_N=0$
	\end{itemize}
	Eine Basislösung $\overline{x}$ heißt \begriff[Basislösung!]{zulässig}, wenn weiterhin $\overline{x}\ge 0$ gilt.
	
	Eine zulässige Basislösung $\overline{x}$ wird \begriff[Basislösung!]{nicht entartet} genannt, wenn $\overline{x}_B>0$ (das heißt $\overline{x}_i>0$ für $i\in B$). Andernfalls, wenn ein $j\in B$ existiert mit $\overline{x}_j=0$, so heißt $\overline{x}$ \begriff[Basislösung!]{entartet}. Die zu $i\in B$ bzw. $i\in N$ gehörenden Variablen $x_i$ werden \begriff{Basisvariable} bzw. \begriff{Nichtbasisvariable} zur Basislösung $\overline{x}$ genannt.
\end{definition}

\begin{*anmkerung}
$A_B$ ist die Matrix, die genau die Spalten von $A$ enthält, deren Nummer in $B$ ist. Beispiel $A_{\{1,2,4\}}$ enthält die erste, zweite und vierte Spalte von $A$.
\end{*anmerkung}

\begin{proposition}
	\proplbl{6_1_4}
	Sei $\rang(A)=m$. Dann ist jede zulässige Basislösung von $G_P$ eine Ecke von $G_P$ und umgekehrt.
\end{proposition}
\begin{proof}
	Sei $\overline{x}$ eine zulässige Basislösung von $G_P$. Falls $\overline{x}=\frac{1}{2}x+\frac{1}{2}y$ für $x,y\in G_P$ gilt, so folgt wegen $x,y\ge 0$ und $\overline{x}_N=0$, dass
	\begin{align}
		\label{6.3}
		x_N=y_N=\overline{x}_N=0
	\end{align}
	Damit und wegen $\overline{x}\in G_P$ sowie der Regularität von $A_B$ erhält man weiter
	\begin{align}
		\label{6.4}
		\begin{split}
			b &= A\overline{x} = A_B\overline{x}_B + A_N\overline{x}_N = A_B\overline{x}_N \\
			\overline{x}_B &= A^{-1}_Bb
		\end{split}
	\end{align}
	Analog folgt $x_B=y_B=A^{-1}_Bb$. Also ist $\overline{x}_B=x_B=y_B$. Daher und wegen \cref{6.3} muss $\overline{x}$ Ecke von $G_P$ sein. \\
	Sei nun $\overline{x}$ Ecke von $G_P$ und $\mathcal{P}=\{i\in I\mid \overline{x}_i>0\}$. Angenommen die Spalten der Matrix $A_\mathcal{P}$ sind linear abhängig. Dann gibt es einen Vektor $w\in\real^n\backslash\{0\}$ mit $w_{I\setminus\mathcal{P}}=0$ und $Aw=0$. Für
	\begin{align}
		x(t) = \overline{x} + tw\notag
	\end{align}
	folgt $Ax(t)=A\overline{x} + tAw=b$. Wegen $\overline{x}_\mathcal{P}>0$ ergibt sich $x_\mathcal{P}(t)\ge 0$ und damit $x(t)\ge 0$ für alle $t$ mit $\vert t\vert$ hinreichend klein. Also gibt es ein $\overline{t}>0$, so dass $x(-\overline{t}),x(\overline{t})\in G_P$. Da $x(-\overline{t})\neq x(\overline{t})$ und $\frac{1}{2}x(-\overline{t}) + \frac{1}{2}x(\overline{t})=\overline{x}$ widerspricht dies der Voraussetzung, dass $\overline{x}$ Ecke von $G_P$ ist. Folglich ist die Annahme falsch, das heißt die Spalten von $A_\mathcal{P}$ sind linear unabhängig. Falls $\vert \mathcal{P}\vert=m$, setzen wir $B=\mathcal{P}$. Andernfalls kann wegen $\rang(A)=m$ die Menge $\mathcal{P}$ durch Hinzunahme von Indizes aus $I\setminus \mathcal{P}$ so zu $B$ ergänzt werden, dass die Spalten der Matrix $A_B$ linear unabhängig sind. Mit $N=I\setminus B$ sieht man, dass $\overline{x}$ alle Eigenschaften einer zulässigen Basislösung besitzt.
\end{proof}

\begin{proposition}
	\proplbl{6_1_5}
	Sei $G_P\neq 0$. Dann besitzt $G_P$
	\begin{enumerate}[label=(\alph*)]
		\item mindestens eine Ecke
		\item  höchstens endlich viele Ecken
	\end{enumerate}
\end{proposition}
\begin{proof}
	O.B.d.A. kann man zunächst $\rang(A)=m$ annehmen.
	\begin{enumerate}[label=(\alph*)]
		\item Zum Selbststudium empfohlen.
		\item $G_P$ besitzt höchstens endlich viele zulässige Basislösungen (und damit nach \propref{6_1_4} höchstens endlich viele Ecken), da es maximal $\binom{n}{m}$ Teilmengen der Menge $I$ mit Kardinalität $m$ gibt, die als Indexmenge $B$ in Frage kommen (und für jede dieser Indexmengen $B$ gibt es höchstens eine Basislösung).
	\end{enumerate}
\end{proof}

\begin{proposition}
	Falls die Optimierungsaufgabe \cref{6.2} lösbar ist, dann gibt es eine Ecke $\overline{x}$ von $G_P$, die \cref{6.2} löst.
\end{proposition}
\begin{proof}
	O.B.d.A. kann $\rang(A)=m$ angenommen werden. Sei $\mathcal{P}(x) = \{i\in I\mid x_i>0\}$. Unter allen Lösungen der Optimierungsaufgabe \cref{6.2} wähle man eine Lösung $\overline{x}$ so, dass $\mathcal{P}=\mathcal{P}(\overline{x})$ die minimal mögliche Anzahl von Elementen aufweist. 
	\begin{itemize}
		\item Falls die Matrix $A_\mathcal{P}$ Vollrang hat, so kann $\mathcal{P}$ erforderlichenfalls so durch geeignete Indizes aus $I\setminus\mathcal{P}$ zu $B$ ergänzt werden, dass $A_B$ dann regulär und $\overline{x}$ damit zulässige Basislösung (das heißt Ecke von $G_P$) ist.
		\item Angenommen $A_\mathcal{P}$ besitzt keinen Vollrang. Dann gibt es ein $w\in\real^n\backslash\{0\}$ mit $w_{I\setminus\mathcal{P}}=0$ und $Aw=0$. Ähnlich wie im Beweis von \propref{6_1_4} erhält man ein $\overline{t}\in\real$, so dass $x(-\overline{t}),x(\overline{t})\in G_P$, $x_{I\setminus\mathcal{P}}=0$ und $x_i(\overline{t})=0$ für ein $i\in\mathcal{P}$. Auf Grund der Minimaleigenschaft von $\overline{x}$ gilt $c^T\overline{x}\le c^T(\overline{x}+tw)$ für $t\in\{-\overline{t},\overline{t}\}$ und es folgt $c^Tw=0$. Dies liefert $c^Tx(\overline{t})=c^T(\overline{x}+w\overline{t})=c^T\overline{x}$. Folglich ist $\overline{x}$ eine Lösung der Optimierungsaufgabe \cref{6.2} mit weniger als $\vert\mathcal{P}\vert$ Nullelementen. Dies widerspricht der Auswahl von $\overline{x}$.
	\end{itemize}
\end{proof}

\section{Simplex-Verfahren}

O.B.d.A setzen wir hier $\rang(A)=m$ voraus. Zur Bestimmung einer Lösung der Optimierungsaufgabe \cref{6.2} würde es also genügen, alle Ecken des Polyeders $G_P$ hinsichtlich ihres Zielfunktionswertes zu untersuchen. Um diesem im Allgemeinen nicht leistbaren Aufwand zu reduzieren, konstruiert man im Simplex-Verfahren eine Folge von Ecken von $G_P$, deren Elemente nicht steigende Zielfunktionswerte aufweisen. Sei $\overline{x}$ eine beliebige Ecke (zulässige Basislösung) von $G_P$ mit den Indexmengen $B$ und $N$. Dann gilt:
\begin{align}
	A_Bx_B + A_Nx_N = b\quad\text{und}\quad x_B=A_B^{-1}(b-A_Nx_N)\notag
\end{align}
Die Aufgabe \cref{6.2} kann damit äquivalent geschrieben werden als
\begin{align}
	c_B^TA_B^{-1}(b-A_Nx_N) + c_N^Tx_N+c_0\to\min \notag
\end{align}
bei $A^{-1}_B(b-A_Nx_N)\ge 0$, $x_N\ge 0$ und $x_B=A^{-1}_B(b-A_Nx_N)$. Mit den Bezeichnungen
\begin{align}
	P &= (p_{ik}) = -A_B^{-1}A_N \notag \\
	p &= (p_i) = A_B^{-1}b \notag \\
	q &= (q_k) = (c_N^T-c_B^TA_B^{-1}A_N)^T \notag \\
	q_0 &= c_B^TA_B^{-1}b+c_0 \notag
\end{align}
ergibt sich die zu \cref{6.2} äquivalente Formulierung
\begin{align}
	\label{6.5}
	q^Tx_N+q_0\to\min\quad\text{bei } Px_N+p\ge 0,\, x_N\ge 0,\, x_B=Px_N+p
\end{align}

\begin{proposition}
	\proplbl{6_2_1}
	Sei $\overline{x}$ eine Ecke von $G_P$ mit zugehörigen Indexmengen $B$ und $N$. Dann tritt genau einer der folgenden Fälle ein:
	\begin{enumerate}[label=(\alph*)]
		\item Wenn $q\ge 0$, dann löst 
		\begin{align}
			\overline{x} = \begin{henrysmatrix}
				\overline{x}_B \\ \overline{x}_N
			\end{henrysmatrix} = \begin{henrysmatrix}
				p \\ 0
			\end{henrysmatrix} = \begin{henrysmatrix}
				A_B^{-1}b \\ 0
			\end{henrysmatrix}\notag
		\end{align}
		sowohl \cref{6.5} als auch \cref{6.2}.
		\item Es gibt ein $k\in N$ mit $q_k<0$ und $Pe_k\ge 0$ ($k$-te Spalte von $P$ ist nicht-negativ). Dann sind die Zielfunktionen des Problems \cref{6.5} und des Problems \cref{6.2} auf dem zulässigen Bereich nicht nach unten beschränkt und weder \cref{6.5} noch \cref{6.2} besitzen eine Lösung.
		\item Es gibt ein $k\in N$ mit $q_k<0$ und für jedes derartige $k$ ist $\{i\in B\mid p_{ik}<0\}$ nicht leer. Dann ist mit $t_k\ge 0$ und $l\in B$ gegeben durch
		\begin{align}
			t_k = \frac{p_l}{-p_{lk}} = \min\left\lbrace\frac{p_i}{-p_{ik}}\,\Bigg|\, i\in B\colon p_{ik}<0 \right\rbrace \notag
		\end{align}
		der Vektor $\overline{x}$ mit
		\begin{align}
			\hat{x}_i = \overline{x}_i + t_k\begin{cases}
				p_{ik} & i\in B \\ \delta_{ik} & i\in N
			\end{cases}\notag
		\end{align}
		eine zulässige Basislösung von $G_P$ mit den Indexmengen
		\begin{align}
			\hat{B} &= (B\setminus \{l\})\cup \{k\} \notag \\
			\hat{N} &= (N\setminus \{k\})\cup \{l\} \notag
		\end{align}
		für die Basis- bzw. Nichtbasisvariable. Weiter gilt $c^T\hat{x}+c_0\le c^T\overline{x}+c_0$, wobei $<$ immer angenommen wird, wenn $\overline{x}$ eine nicht entartete zulässige Basislösung ist.
	\end{enumerate}
	Die Fälle (a) und (b) werden \begriff{entscheidbar} genannt.
\end{proposition}
\begin{proof}
	\begin{enumerate}[label=(\alph*)]
		\item  Wegen $p=A^{-1}_Bb\ge 0$ löst $(x_B,x_N)=(p,0)$ das Problem \cref{6.5} und damit auch \cref{6.2}. Nach \propref{6_1_3}, \propref{6_1_4} und \cref{6.4} gilt $(\overline{x}_B,\overline{x}_N)=(p,0)$.
		\item Wegen $p=\overline{x}_B\ge 0$ erhält man für 
		\begin{align}
			x(\lambda)=\begin{henrysmatrix}
				x_B(\lambda) \\ x_N(\lambda)
			\end{henrysmatrix} \notag
		\end{align}
		mit $x_i(\lambda)=\lambda\delta_{ik}$ für $i\in N$ und $x_B(\lambda)=Px_N(\lambda)+p$, dass
		\begin{align}
			x_B(\lambda) = \lambda P[:,k]+p\ge 0,\, x_N(\lambda)\ge 0\quad\forall\lambda\ge 0\notag
		\end{align}
		wobei $P[:,k]$ die zum Index $k\in N$ gehörende Spalte der Matrix $P$ bezeichnet. Somit ist $x(\lambda)$ für jedes $\lambda\ge 0$ ein zulässiger Punkt der Optimierungsaufgabe \cref{6.5}. Für den Zielfunktionswert von \cref{6.5} an der Stelle $x(\lambda)$ ergibt sich
		\begin{align}
			q^Tx_N(\lambda) + q_0 = \lambda q_k+q_0\to -\infty\quad\text{für } \lambda\to\infty\notag
		\end{align}
		das heißt die Aufgabe \cref{6.5} und damit auch \cref{6.2} ist nicht lösbar.
		\item Es wird zunächst gezeigt, dass $\hat{x}\in G_P$. Mit $\overline{x}_B=p\ge 0$ erhält man für $i\in B$
		\begin{align}
			\hat{x}_i = \overline{x}_i + t_kp_{ik} = \overline{x}_i + \frac{p_l}{-p_{lk}}p_{ik} \ge \begin{cases}
				\overline{x}_i + \frac{p_i}{-p_{ik}}p_{ik} = p_i-p_i = 0 & p_{ik}<0 \\
				\overline{x}_i \ge 0 & p_{ik}\ge 0 
			\end{cases}\notag
		\end{align}
		Speziell gilt
		\begin{align}
			\label{6.6}
			\hat{x}_l = 0
		\end{align}
		Für $i\in N$ folgt
		\begin{align}
			\label{6.7}
			\hat{x}_i = \overline{x}_i + t_k\delta_{ik}\ge 0
		\end{align}
		Außerdem gilt mit der Definition von $P$
		\begin{align}
			A\hat{x} &= A_B(\overline{x}_B + t_kP[:,k]) + t_kA_N[:,k] \notag \\
			&= b + t_kA_BP[:,k] + t_kA_N[:,k] \notag \\
			&= b-t_kA_BA_B^{-1}A_N[:,k] + t_kA_N[:,k] \notag \\
			&= b \notag
		\end{align}
		Somit ist die Zulässigkeit von $\hat{x}$ gezeigt. Es wird nun nachgewiesen, dass $\hat{x}$ eine Basislösung zu den Indexmengen $\hat{B}$ und $\hat{N}$ ist. Dazu bezeichne $a_l$ bzw. $a_k$ die $l$-te bzw. $k$-te Spalte von $A$. Weiter sei $(l)$ die Spaltennummer der Spalte $a_l$ innerhalb der Matrix $A_B$. Dann gilt
		\begin{align}
			A_{\hat{B}} &= A_B + (a_k-a_l)e_{(l)}^T \notag \\
			1 + e^T_{(l)}A_B^{-1}(a_k-a_l) &= 1+ e^T_{(l)}(A_B^{-1}a_k - A_B^{-1}a_l) \notag \\
			&= 1 + e^T_{(l)}(-P[:,k] - e_{(l)}) \notag \\
			&= 1 - p_{lk} - 1 \notag \\
			&= -p_{lk} > 0 \notag
		\end{align}
		und mit $v=a_k-a_l$
		\begin{align}
			\left( \mathbbm{1} - \frac{A_B^{-1}ve^T_{(l)}}{1 + e^T_{(l)}A_B^{-1}v}\right)A_B^{-1}A_{\hat{B}} &= \left( \mathbbm{1} - \frac{A_B^{-1}ve^T_{(l)}}{1 + e^T_{(l)}A_B^{-1}v}\right) (\mathbbm{1} + A_B^{-1}ve^T_{(l)}) \notag \\
			&= \mathbbm{1} + A_B^{-1}ve^T_{(l)} \left( 1 + \frac{-1 - e^T_{(l)}A_B^{-1}v}{1 + e^T_{(l)}A_B^{-1}v}\right) \notag \\
			&= \mathbbm{1} \notag
		\end{align}
		Also ist $A_{\hat{B}}$ regulär mit
		\begin{align}
			A_{\hat{B}}^{-1} = \left( \mathbbm{1} - \frac{A_B^{-1}ve^T_{(l)}}{1 + e^T_{(l)}A_B^{-1}v}\right)A_B^{-1} \notag
		\end{align}
		Dies kann auch direkt mit Hilfe der \person{Shermann-Morrison}-Formel geschlossen werden. Wegen \cref{6.6} und \cref{6.7} gilt $\hat{x}_{\hat{N}}=0$. Somit ist $\hat{x}$ Basislösung von $G_P$. Schließlich ergibt sich
		\begin{align}
			c^T\hat{x}+c_0 &= q^T\hat{x}_N + q_0 \notag \\
			&= q^T\overline{x}_N + t_kq_k + q_0 \notag \\
			&\le q^T\overline{x}_N + q_0 \notag \\
			&= c^T\overline{x} + c_0 \notag
		\end{align}
		Bei einer nicht entarteten zulässigen Basislösung $\overline{x}$ ist die Schrittweite $t_k$ für jedes $k$ mit $q_k<0$ positiv, da $p_B$ in allen Komponenten positiv ist. Also folgt dann
		\begin{align}
			c^T\hat{x}+c_0 &= q^T\overline{x}_N + q_0 + t_kq_k \notag \\
			&< q^T\overline{x}_N + q_0 \notag \\
			&= c^T\overline{x} + c_0 \notag
		\end{align}
	\end{enumerate}
\end{proof}

\begin{algorithm}[Simplex-Verfahren]
	\proplbl{6_2_2}
	Input: $A\in\real^{m\times n}$ mit $\rang(A)=m$, $b\in\real^m$, $c\in\real^n$, $c_0\in\real$
	\begin{lstlisting}
r = 0
do while "in %$x^r$% liegt kein entscheidbarer Fall vor"
 choose %$k\in N$% mit %$q_k<0$%
 compute %$l\in B$% und %$t_k\ge 0$% so dass &
 & %$t_k = \frac{p_l}{-p_{lk}} = \min\left\lbrace\frac{p_i}{-p_{ik}}\,\Bigg|\, i\in B\colon p_{ik}<0 \right\rbrace$%
 switch i
  case %$i\in B$%:
   %$x_i^{r+1}=x_i^r + t_kp_{ik}$%
  break
  case i=k:
   %$x_i^{r+1}=x_i^r + t_k$%
  break
  case %$i\in N\setminus \{k\}$%:
   %$x_i^{r+1}=x_i^r$%
  break
 end switch
 %$\hat{B} = (B\setminus \{l\})\cup \{k\}$%
 %$\hat{N} = (N\setminus \{k\})\cup \{l\}$%
 r = r + 1
end do
	\end{lstlisting}
	Output: $x^r$, gegebenenfalls liegt entscheidbarer Fall (b) - Nichtlösbarkeit - vor
\end{algorithm}

\begin{proposition}
	Seien $\rang(A=m$ und $G_P\neq\emptyset$. Dann ist der \propref{6_2_2} wohldefiniert. Falls alle zulässigen Basislösungen von $G_P$ nicht entartet sind, dann bricht der Algorithmus nach endlich vielen Schritten mit einem entscheidbaren Fall ab.
\end{proposition}
\begin{proof}
	Wegen Teil (a) von \propref{6_1_5} und \propref{6_1_4} gibt es mindestens eine zulässige Basislösung $x^0$. Die Durchführbarkeit der übrigen Schritte von \propref{6_2_2} ist unter Beachtung von \propref{6_2_1} offensichtlich. Nach Teil (c) von \propref{6_2_1} gilt für zwei aufeinander folgende von \propref{6_2_2} erzeugte Iterierte 
	\begin{align}
		c^Tx^{r+1} \le c^T x^r \notag
	\end{align}
	denn (nach Voraussetzung) sind alle zulässigen Basislösungen nicht entartet. Da es nach Teil (b) von \propref{6_1_5} nur endlich viele Basislösungen gibt und keine Basislösung mehr als als einmal unter den erzeugten Iterierten auftreten kann, bricht der \propref{6_2_2} nach endlich vielen Schritten mit einem entscheidbaren Fall ab.
\end{proof}

Es gibt Techniken, mit denen man einen Zyklus zwischen verschiedenen Darstellungen ein und derselben Ecke verhindert, so dass man die Forderung nach Nichtentartung in \propref{6_2_1} fallen lassen kann.

\begin{proposition}
	Die Aufgabe \cref{6.2} besitzt genau dann eine Lösung, wenn $G_P\neq\emptyset$ und eine Schranke $U\in\real$ existiert mit $c^Tx\ge U$ für alle $x\in G_P$. 
\end{proposition}
\begin{proof}[nur Beweisskizze]
	Die Notwendigkeit der beiden Bedingungen für die Lösbarkeit von \cref{6.2} ist offensichtlich. Um zu sehen, dass sie auch hinreichend sind, können wir o.B.d.A. $\rang(A)=m$ voraussetzen. Wegen $G_P\neq\emptyset$ garantiert das Simplex-Verfahren (mit einer Technik, die auch bei entarteten zulässigen Basislösungen funktioniert) dann die Erreichung eines entscheidbaren Falles nach endlich vielen Schritten. Dabei kann der entscheidbare Fall (b) wegen $c^Tx\ge U$ für alle $x\in G_P$ nicht eintreten sondern nur der Abbruch mit einer Lösung. 
\end{proof}
\section{Die Tableauform des Simplex-Verfahrens}

Der Übergang von der Ecke $\overline{x}$ zur Ecke $\hat{x}$ zieht den Wechsel von $(B,N)$ zu $\hat{B} = (B\setminus \{l\})\cup \{k\}$ und $ \hat{N} = (N\setminus \{k\})\cup \{l\}$ nach sich. Das zu $\overline{x}$ gehörende Tableau
\begin{center}
	\begin{tabular}{c|c|c}
		& $x_N^T$ & 1 \\
		\hline
		$x_B=$ & $P$ & $p$ \\
		\hline
		$z = $ & $q^T$ & $q_0$ \\
	\end{tabular}
\end{center}
ist zeilenweise zu lesen, das heißt
\begin{align}
	x_i &= \sum_{j\in N} p_{ij}x_j + p_i \notag \\
	z &= \sum_{j\in N} q_jx_j + q_0 \notag 
\end{align}
Das Element $p_{lk}$ heißt \begriff{Pivot}. Die Zeile zu $l\in B$ heißt \begriff{Pivotzeile}, die Spalte zu $k\in N$ \begriff{Pivotspalte}. Zur Vereinfachung bei Handrechnung kann man das Tableau auch durch eine sogenannte Kellerzeile ergänzen:
\begin{align}
	\text{Kellerzeile} = -\frac{\text{Pivotzeile (ohne Pivot)}}{\text{Pivot}}\notag
\end{align}
Durch Austausch von $x_l$ gegen $x_k$ erhält man das zu $\hat{x}$ gehörende Tableau
\begin{center}
	\begin{tabular}{c|c|c}
		& $x_{\hat{N}}^T$ & 1 \\
		\hline
		$x_{\hat{B}}=$ & $\hat{P}$ & $\hat{p}$ \\
		\hline
		$z = $ & $\hat{q}^T$ & $\hat{q}_0$ \\
	\end{tabular}
\end{center}
mit 
\begin{center}
	\begin{tabular}{p{6cm}|p{8cm}}
		\textbf{Stellung im Tableau} & \textbf{Wert im neuen Tableau} \\
		\hline
		Element (außer Pivotzeile und -spalte) & $\text{Element} + \text{Pivotspaltenelement}\cdot \text{Kellerzeilenelement}$ \\
		Pivotzeile (ohne Pivot) & Kellerzeile (ohne Pivot) \\
		Pivotspalte (ohne Pivot) & $\frac{\text{Pivotspalte (ohne Pivot)}}{\text{Pivot}}$ \\
		Pivot & $\frac{1}{\text{Pivot}}$ \\
	\end{tabular}
\end{center}

\begin{example}
	\begin{align}
		-13x_1 + 37x_2 + 12x_3 + 48\to\min \notag \\
		\text{bei } x_1-2x_2-x_3 &\le 2 \notag \\
		2x_1-5x_2-x_3 &\le 4 \notag \\
		x_1-3x_2-x_3&\le 1 \notag \\
		x_1,x_2,x_3 &\ge 0 \notag
	\end{align}
	Durch Einführen der Schlupfvariablen $x_4,x_5,x_6\ge 0$ ergibt sich ein Problem in Standardform
	\begin{align}
	-13x_1 + 37x_2 + 12x_3 + 48\to\min \notag \\
	\text{bei } x_1-2x_2-x_3+x_4 &= 2 \notag \\
	2x_1-5x_2-x_3 + x_5 &= 4 \notag \\
	x_1-3x_2-x_3 + x_6&= 1 \notag \\
	x_1,x_2,x_3,x_4,x_5,x_6 &\ge 0 \notag
	\end{align}
	Offenbar kann man einer erste zulässige Basislösung $x^0$ ablesen. Das zugehörige Tableau (mit Pivot \textcolor{red}{-1} und Kellerzeile) lautet dann \\
	\begin{minipage}[c]{0.6\textwidth}
		\begin{center}
			\begin{tabular}{c|ccc|c}
				$T_0$ & $x_1$ & $x_2$ & $x_3$ & 1 \\
				\hline
				$x_4=$ & -1 & 2 & 1 & 2 \\
				\hline 
				$x_5=$ & -2 & 5 & 1 & 4 \\
				\hline 
				$x_6=$ & \textcolor{red}{-1} & 3 & 1 & 1 \\
				\hline
				$z=$ & -13 & 37 & 12 & 48 \\
				\hline
				& $\times$ & 3 & 1 & 1
			\end{tabular}
		\end{center}
	\end{minipage}
	\begin{minipage}[c]{0.3\textwidth}
		\begin{align}
			x^0 &= (0,0,0,2,4,1)^T\notag \\
			B &= \{4,5,6\},\, N=\{1,2,3\}\notag \\
			z(x^0) &= c^Tx^0 + c_0 = 48 \notag \\
			&\text{nicht entscheidbar} \notag 
		\end{align}
	\end{minipage}
	Dabei ist die Wahl der Pivotspalte $k=1\in N$ eindeutig, da nur ein negativer Zielfunktionskoeffizient vorhanden ist. Die Wahl  der Pivotzeile $l=6\in B$ ist auch eindeutig, da unter den Quotienten $\frac{p_i}{-p_{i,1}}$ mit $i\in \{B\mid p_{i,1}<0\}$ der Quotient $\frac{p_6}{-p_{6,1}}=\frac{1}{-(-1)}=1$ am kleinsten ist. Der Fortgang des Simplex-Verfahrens in Tableauform ist nun wie folgt: \\
	\begin{minipage}[c]{0.6\textwidth}
		\begin{center}
			\begin{tabular}{c|ccc|c}
				$T_1$ & $x_6$ & $x_2$ & $x_3$ & 1 \\
				\hline
				$x_4=$ & 1 & \textcolor{red}{-1} & 0 & 1 \\
				\hline 
				$x_5=$ & 2 & -1 & -1 & 2 \\
				\hline 
				$x_1=$ & -1 & 3 & 1 & 1 \\
				\hline
				$z=$ & 13 & -2 & -1 & 35 \\
				\hline
				& 1 & $\times$ & 0 & 1
			\end{tabular}
		\end{center}
	\end{minipage}
	\begin{minipage}[c]{0.3\textwidth}
		\begin{align}
		x^1 &= (1,0,0,1,2,0)^T\notag \\
		B &= \{1,4,5\},\, N=\{2,3,6\}\notag \\
		z(x^1) &= c^Tx^1 + c_0 = 35 \notag \\
		&\text{nicht entscheidbar} \notag 
		\end{align}
	\end{minipage} \\
	\begin{minipage}[c]{0.6\textwidth}
		\begin{center}
			\begin{tabular}{c|ccc|c}
				$T_2$ & $x_6$ & $x_4$ & $x_3$ & 1 \\
				\hline
				$x_2=$ & 1 & -1 & 0 & 1 \\
				\hline 
				$x_5=$ & 1 & 1 & \textcolor{red}{-1} & 1 \\
				\hline 
				$x_1=$ & 2 & -3 & 1 & 4 \\
				\hline
				$z=$ & 11 & 2 & -1 & 33 \\
				\hline
				& 1 & 1 & $\times$ & 1
			\end{tabular}
		\end{center}
	\end{minipage}
	\begin{minipage}[c]{0.3\textwidth}
		\begin{align}
		x^2 &= (4,1,0,0,1,0)^T\notag \\
		B &= \{1,2,5\},\, N=\{3,4,6\}\notag \\
		z(x^2) &= c^Tx^2 + c_0 = 33 \notag \\
		&\text{nicht entscheidbar} \notag 
		\end{align}
	\end{minipage} \\
	\begin{minipage}[c]{0.6\textwidth}
		\begin{center}
			\begin{tabular}{c|ccc|c}
				$T_3$ & $x_6$ & $x_4$ & $x_5$ & 1 \\
				\hline
				$x_2=$ & 1 & -1 & 0 & 1 \\
				\hline 
				$x_3=$ & 1 & 1 & -1 & 1 \\
				\hline 
				$x_1=$ & 3 & -2 & -1 & 5 \\
				\hline
				$z=$ & 10 & 1 & 1 & 32 \\
				\hline
				&  &  &  & 
			\end{tabular}
		\end{center}
	\end{minipage}
	\begin{minipage}[c]{0.3\textwidth}
		\begin{align}
		x^3 &= (5,1,1,0,0,0)^T\notag \\
		B &= \{1,2,3\},\, N=\{4,5,6\}\notag \\
		z(x^3) &= c^Tx^3 + c_0 = 32 \notag \\
		&\text{entscheidbar, } x^3 \text{ ist Lösung} \notag 
		\end{align}
	\end{minipage}
\end{example}
\section{Revidiertes Simplex-Verfahren}

Hier wird auf die explizite Berechnung der Matrix $P$ verzichtet. Vielmehr wird beim Übergang von einer Ecke $x^r$ mit dem Indexmengen $B_r$ und $N_r$ zur nächsten Ecke $x^{r+1}$ mit den Indexmengen $B_{r+1}$ und $N_{r+1}$ ausgenutzt, dass sich die Indexmengen jeweils nur in einem Index ändern und der Unterschied der Matrix $A_{B_r}$ zur Matrix $A_{B_{r+1}}$ in der Änderung einer einzigen Spalte besteht.

Die an der Ecke $x^{r+1}$ zur Bestimmung des neuen Pivots erforderlichen Vektoren, zum Beispiel $p^{r+1}$ und $q^{r+1}$, werden dann mit Hilfe der aus $A_{B_r}^{-1}$ mit $\mathcal{O}(m^2)$ Operationen zu gewinnenden Matrix $A_{B_{r+1}}^{-1}$ bestimmt, vergleiche die Aufdatierungsformel im Beweis von \propref{6_2_1}, zum Beispiel erhält man
\begin{align}
	p^{r+1} = A_{B_{r+1}}^{-1}b\notag
\end{align}
mit
\begin{align}
	A_{B_{r+1}}^{-1} &= \left(\mathbbm{1}-\frac{A_{B_r}^{-1}ve^T_{(l)}}{1+e^T_{(l)}A_{B_r}^{-1}v} \right) A_{B_r}^{-1}\notag \\
	&= A_{B_r}^{-1} - \frac{1}{1+e^T_{(l)}(A_{B_r}^{-1}v)}(A_{B_r}^{-1}v)(e^T_{(l)}A_{B_r}^{-1})\notag
\end{align}
Die Matrizen $A_{B_r}^{-1}$ müssen nicht explizit gespeichert werden. Da mit geeigneten Vektoren $u,w,\tilde{w}\in\real^m$ offenbar
\begin{align}
	A_{B_{r+1}}^{-1} = (\mathbbm{1}+uw^T)A_{B_r}^{-1} = A_{B_r}^{-1} + u\tilde{w}^T\notag
\end{align}
genügt es, etwa $u,\tilde{w}$ für jeden Basiswechsel sowie $A_{B_0}$ zu speichern. Ein Restart ist hier nach einer größeren Anzahl von Basiswechseln empfehlenswert, um den Einfluss von Rundungsfehlern zu begrenzen.
\section{Bestimmung einer ersten zulässigen Basislösung}

O.B.d.A kann man in der Standardform $c^Tx+c_0\to\min$ bei $Ax=b$, $x\ge 0$ einer linearen Optimierungsaufgabe voraussetzen, dass
\begin{align}
	\label{6.8}
	b\ge 0
\end{align}
gilt. Andernfalls multipliziere man die Gleichungen $Ax=b$ mit $b_i<0$ einfach mit -1. Anstelle des zulässigen Bereiches $G_P$ betrachten wir nun den erweiterten Bereich
\begin{align}
	\tilde{G}_P = \{(x,y)\in\real^n\times\real^m\mid y=-Ax+b,\, x,y\ge 0\}\notag
\end{align}

\begin{lemma}
	Es gilt $G_P\times \{0\}\subset\tilde{G}_P$. Ist $(\overline{x},0)$ eine Ecke von $\tilde{G}_P$, so ist $\overline{x}$ Ecke von $G_P$ und umgekehrt.
\end{lemma}
\begin{proof}
	Die erste Aussage folgt sofort aus der Definition der Mengen $G_P$ und $\tilde{G}_P$. Sei nun $(\overline{x},0)$ Ecke von $\tilde{G}_P$. Weiter seien $x^1,x^2\in G_P$ beliebig gegeben, so dass
	\begin{align}
		\frac{1}{2}(x^1+x^2) = \overline{x}\notag
	\end{align}
	Wegen $-Ax^1+b=0$ und $-Ax^2+b=0$ folgt zunächst $(x^1,0),(x^2,0)\in\tilde{G}_P$. Da $(\overline{x},0)$ Ecke von $\tilde{G}_P$ ist, folgt weiter $x^1=x^2=\overline{x}$. Also ist $\overline{x}$ Ecke von $G_P$. Die Umkehrung folgt leicht mit ähnlichen Überlegungen.
\end{proof}

Um also eine Ecke $\overline{x}$ von $G_P$ zu ermitteln, genügt es eine Ecke $\overline{x},\overline{y}$ von $\tilde{G}_P$ zu bestimmen, für die $\overline{y}=0$ gilt. Dies versucht man mit der Lösung folgender Hilfsaufgabe zu erreichen:
\begin{align}
	\label{6.9}
	h(y) = \sum_{i=1}^{m}y_i\to\min\quad\text{bei } y = -Ax+b,\, x,y\ge 0
\end{align}
Die Zielfunktion $h$ von \cref{6.9} wird als \begriff{Hilfszielfunktion}, die Variablen $y_1,...,y_m$ werden als \begriff{künstliche Variablen} und das Vorgehen selbst wird als \begriff{Hilfszielfunktionsmethode} bezeichnet.

Offenbar ist $(\overline{x},\overline{y})=(0,b)$ mit $x$ als Nichtvariable und $y$ als Basisvariable eine erste zulässige Basislösung von $\tilde{G}_P$. Also ist $\tilde{G}_P$ insbesondere nicht leer. Außerdem ist die Zielfunktion von \cref{6.9} wegen $y\ge 0$ auf $\tilde{G}_P$ durch 0 nach unten beschränkt. Mit \propref{6_2_4} folgt daher die Lösbarkeit der linearen Optimierungsaufgabe \cref{6.9}. Die Anwendung des Simplex-Verfahrens liefert daher (vergleiche \propref{6_2_3} und die Anmerkung danach) nach endlich vielen Schritten eine Lösung $(x^\ast,y^\ast)$ von \cref{6.9}. Die Darstellung der Hilfszielfunktion $h$ durch die Nichtbasisvariablen $x$ an der Basislösung $(0,b)$ ist dabei gegeben durch 
\begin{align}
	y_i = -(Ax)_i + b_i\quad\text{und}\quad \sum_{i=1}^m y_i = e^Ty = -e^T(Ax)+e^Tb\notag
\end{align}
Damit ergibt sich folgendes Anfangstableau für das Simplex-Verfahren zur Lösung von \cref{6.9}
\begin{center}
	\begin{tabular}{c|c|c}
		& $x^T$ & 1 \\
		\hline
		$y=$ & $-A$ & $b$ \\
		\hline
		$z=$ & $c^T$ & $c_0$ \\
		\hline
		$h=$ & $-e^TA$ & $e^Tb$ \\
	\end{tabular}
\end{center}
Sinnvollerweise wird in der Tableauform die Darstellung der eigentlichen Zielfunktion der Optimierungsaufgabe \cref{6.2} mit Hilfe der jeweiligen Nichtbasisvariablen in einer zusätzlichen Zeile simultan mit angepasst. Sobald eine künstliche Variable $y_i$ Nichtbasisvariable wird, kann die entsprechende Zeile gestrichen werden.

Falls in der Lösung von \cref{6.9} $y^\ast\neq 0$ (das heißt $h(y^\ast)>0$) gilt, ist der zulässige Bereich $G_P$ der Ausgangsaufgabe \cref{6.2} leer. Andernfalls hat man mit $x^\ast$ eine erste Ecke von $G_P$ gefunden. Falls künstliche Variablen noch als Basisvariable auftreten, sind diese durch weitere Austauschschritte zu Nichtbasisvariablen zu machen; geht das nicht, sind Nullzeilen zu streichen.

\begin{example}
	\begin{align}
		z(x) = x_1 - x_2 + 2x_3\to\min \notag \\
		\text{bei } 2x_1 - 3x_2 + x_3 &\ge 3 \notag \\
		x_1 - 2x_2 - x_3 &\ge 1 \notag \\
		3x_1 - 5x_2 &\ge 4 \notag \\
		x_1,x_2,x_3&\ge 0\notag
	\end{align}
	Einführen von Schlupfvariablen führt auf die Standardform
	\begin{align}
		z(x) = x_1 - x_2 + 2x_3\to\min \notag \\
		\text{bei } 2x_1 - 3x_2 + x_3 - x_4 &= 3 \notag \\
		x_1 - 2x_2 - x_3 - x_5 &= 1 \notag \\
		3x_1 - 5x_2  - x_6 &= 4 \notag \\
		x_1,x_2,x_3,x_4,x_5,x_6&\ge 0\notag
	\end{align}
	Das Hilfsproblem ergibt sich nun zu
	\begin{align}
		h(y) = y_1+y_2+y_3 \to\min \notag \\
		\text{bei } y_1 &= -2x_1 + 3x_2 - x_3 + x_4 + 3 \notag \\
		y_2 &= -x_1 + 2x_2 + x_3 + x_5 + 1 \notag \\
		y_3 &= -3x_1 + 5x_2  + x_6 + 4 \notag \\
		x_1,...,x_6,y_1,y_2,y_3&\ge 0\notag
	\end{align}
	Das Simplex-Verfahren (in Tableauform) zur Lösung der Hilfsaufgabe läuft nun wie folgt ab:
	\begin{minipage}[c]{0.5\textwidth}
		\begin{center}
			\begin{tabular}{c|cccccc|c}
				$H_0$ & $x_1$ & $x_2$ & $x_3$ & $x_4$ & $x_5$ & $x_6$ & 1 \\
				\hline
				$y_1 = $ & -2 & 3 & -1 & 1 & 0 & 0 & 3 \\
				$y_2 = $ & \textcolor{red}{-1} & 2 & 1 & 0 & 1 & 0 & 1 \\
				$y_3 = $ & -3 & 5 & 0 & 0 & 0 & 1 & 4 \\
				\hline
				$z = $ & 1 & -1 & 2 & 0 & 0 & 0 & 0 \\
				\hline
				$h= $ & -6 & 10 & 0 & 1 & 1 & 1 & 8 \\
				\hline 
				& $\times$ & 2 & 1 & 0 & 1 & 0 & 1 \\
			\end{tabular}
		\end{center}
	\end{minipage}
	\begin{minipage}[c]{0.5\textwidth}
		\begin{center}
			\begin{tabular}{c|cccccc|c}
				$H_1$ & $y_2$ & $x_2$ & $x_3$ & $x_4$ & $x_5$ & $x_6$ & 1 \\
				\hline
				$y_1 = $ &  & \textcolor{red}{-1} & -3 & 1 & -2 & 0 & 1 \\
				$x_1 = $ &  & 2 & 1 & 0 & 1 & 0 & 1 \\
				$y_3 = $ &  & -1 & -3 & 0 & -3 & 1 & 1 \\
				\hline
				$z = $ &  & 1 & 3 & 0 & 1 & 0 & 1 \\
				\hline
				$h= $ &  & -2 & -6 & 1 & -5 & 1 & 2 \\
				\hline 
				&  & $\times$ & -3 & 1 & -2 & 0 & 1 \\
			\end{tabular}
		\end{center}
	\end{minipage} \\
	\begin{minipage}[c]{0.5\textwidth}
		\begin{center}
			\begin{tabular}{c|cccccc|c}
				$H_2$ & $y_2$ & $y_1$ & $x_3$ & $x_4$ & $x_5$ & $x_6$ & 1 \\
				\hline
				$x_2 = $ & &  & -3 & 1 & -2 & 0 & 1 \\
				$x_1 = $ &  &  & -5 & 2 & -3 & 0 & 3 \\
				$y_3 = $ &  &  & 0 & \textcolor{red}{-1} & -1 & 1 & 0 \\
				\hline
				$z = $ &  &  & 0 & 1 & -1 & 0 & 2 \\
				\hline
				$h= $ &  &  & 0 & -1 & -1 & 1 & 0 \\
				\hline 
				&  &  & $\times$ &  &  &  &  \\
			\end{tabular}
		\end{center}
	\end{minipage}
	\begin{minipage}[c]{0.5\textwidth}
		\begin{center}
			\begin{tabular}{c|cccccc|c}
				$H_3$ & $y_2$ & $y_1$ & $x_3$ & $y_3$ & $x_5$ & $x_6$ & 1 \\
				\hline
				$x_2 = $ & &  & -3 &  & -3 & 1 & 1 \\
				$x_1 = $ &  &  & -5 &  & -5 & 2 & 3 \\
				$x_4 = $ &  &  & 0 &  & -1 & 1 & 0 \\
				\hline
				$z = $ &  &  & 0 &  & -2 & 1 & 2 \\
				\hline
				$h= $ &  &  & 0 &  & 0 & 0 & 0 \\
			\end{tabular}
		\end{center}
	\end{minipage}
	Somit ist $(\overline{x})=(3,1,0,0,0,0)$ eine zulässige Basislösung der Aufgabe in Standardform mit $B=\{1,2,4\}$ und $N=\{3,5,6\}$. Mit Tableau $H_3$ kann nun das Simplex-Verfahren zur Lösung der Standardaufgabe angeschlossen werden. Dies liefert nach zwei Schritten die Lösung der Aufgabe in Standardform zu $x^\ast=(\frac{3}{2},0,0,0,\frac{1}{2},\frac{1}{2})^T$ und $z(x^\ast)=\frac{3}{2}$.
\end{example}

\part*{Anhang}
\addcontentsline{toc}{part}{Anhang}
\appendix

\chapter{Listen}
\section{Liste der Theoreme}
\theoremlisttype{allname}
\listtheorems{theorem}

\pagebreak
\section{Liste der benannten Sätze, Lemmata und Folgerungen}
\theoremlisttype{optname}
\listtheorems{proposition,lemma,conclusion}

%\printglossary[type=\acronymtype]

\printindex

\end{document}
