\section{Lineare Quadratmittelprobleme}

Das lineare Gleichungssystem $Ax=b$ mit $A\in\real^{m\times n}$ und $b\in\real^m$ besitzt genau dann eine Lösung, wenn $\rang(A)=\rang((A,b))$. Falls $m>n$, so heißt das Gleichungssystem \begriff[lineares Gleichungssystem!]{überbestimmt}. Im Allgemeinen gilt dann $\rang(A)\le n<\rang((A,b))$, so dass das Gleichungssystem keine Lösung besitzt. Der Fall der Nichtlösbarkeit kann auch für $m\le n$ eintreten, falls $\rang(A)<m$. Anstelle des Systems $Ax=b$ betrachtet man folgende \begriff{Ersatzaufgabe}:
\begin{align}
	\label{3.8}
	\Vert Ax-b\Vert_2\to\min
\end{align}
die als \begriff{lineares Quadraturmittelproblem} bezeichnet wird. Kurz schreibt man dafür auch $Ax\cong b$.

\begin{proposition}
	\proplbl{3_3_1}
	Seien $A\in\real^{m\times n}$ und $b\in\real^m$ gegeben. Da ist das lineare Quadraturmittelproblem \cref{3.8} lösbar.
\end{proposition}
\begin{proof}
	Die restringierte Optimierungsaufgabe 
	\begin{align}
		\label{3.9}
		f(y) = \Vert y-b\Vert_2\to\min\quad\text{mit } y\in L=\{Ax\mid x\in\real^n\}
	\end{align}
	ist offenbar genau dann lösbar, wenn \cref{3.8} eine Lösung besitzt. Wegen $f(0)=\Vert b\Vert_2$ und $0\in L$ hat
	\begin{align}
		\label{3.10}
		f(y)\to\min\quad\text{mit } y\in L, \, \Vert y-b\Vert_2\le \Vert b\Vert_2
	\end{align}
	dieselbe Lösungsmenge wie \cref{3.9}. Der zulässige Bereich $\{x\in L\mid \Vert y-b\Vert_2\le \Vert b\Vert_2\}$ dieser Optimierungsaufgabe ist nicht-leer, abgeschlossen und beschränkt. Da zudem $f:\real^m\to\real$ stetig ist, besitzt \cref{3.10} nach dem Satz von \person{Weierstrass} eine Lösung. Also sind auch \cref{3.9} und \cref{3.8} lösbar.
\end{proof}

\begin{example}
	Seien
	\begin{align}
		A = \begin{henrysmatrix}
		1 \\1
		\end{henrysmatrix}\quad\text{und}\quad b=\begin{henrysmatrix}
		2 \\0
		\end{henrysmatrix}\notag
	\end{align}
	Dann ist der lineare Teilraum $L$ gegeben durch $L=\left\lbrace \begin{henrysmatrix}1\\1\end{henrysmatrix}x\Bigg| x\in\real\right\rbrace$. Anschaulich ergibt sich, dass eine Lösung $y^*\in L$ von \cref{3.9} der Bedingung $(y^*-b)\perp L$ genügen muss. Daraus folgt
	\begin{align}
		\left(\begin{henrysmatrix}
		 y_1^* \\ y_2^*
		\end{henrysmatrix} - \begin{henrysmatrix}
		2 \\ 0
		\end{henrysmatrix}\right)^T\begin{henrysmatrix}
		1\\1
		\end{henrysmatrix} x=0\quad\text{und}\quad \begin{henrysmatrix}
		y_1^* \\ y_2^*
		\end{henrysmatrix} = \begin{henrysmatrix}
		1\\1
		\end{henrysmatrix} x\notag
	\end{align}
	für alle $x\in\real$. Einzige Lösung von \cref{3.9} ist damit $y^*=(1,1)^T$. Somit ist $x^*=1$ die einzige Lösung von \cref{3.8}.
	\begin{center}
		\begin{tikzpicture}
		\draw[->] (-3,0) -- (3,0);
		\draw[->] (0,-3) -- (0,3);
		
		\draw (-3,-3) -- (3,3);
		\node at (3.5,3.5) (L) {$L$};
		
		\draw[fill=black] (2,0) circle (0.1);
		\node at (2,-0.5) (b) {$b$};
		\draw[dashed] (1,1) -- (2,0);
		\draw[fill=black] (1,1) circle (0.05);
		\node at (1,1.5) (y) {$y^*$};
		\node at (3.2,0.5) (k) {kleinster Abstand};
		\end{tikzpicture}
	\end{center}
\end{example}

\subsection{Die \person{Gauss}'schen Normalgleichungen}

\begin{proposition}
	\proplbl{3_3_3}
	Seien $A\in\real^{m\times n}$ und $b\in\real^m$ gegeben. Dann gilt:
	\begin{enumerate}[label=(\alph*)]
		\item Jede Lösung des linearen Quadraturmittelproblems \cref{3.8} löst die \\ \begriff{\person{Gauss}'schen Normalgleichungen}
		\begin{align}
			\label{3.11}
			A^TAx = A^Tb
		\end{align}
		und umgekehrt.
		\item Falls $\rang(A)=n$ (dies impliziert $m\ge n$), so ist $A^TA$ positiv definit und \cref{3.8} besitzt genau eine Lösung, nämlich $x^*=(A^TA)^{-1}A^Tb$.
		\item Falls $\rang(A)<n$, so ist $A^TA$ positiv semidefinit und singulär und \cref{3.8} besitzt unendlich viele Lösungen.
	\end{enumerate}
\end{proposition}
\begin{proof}
	\begin{enumerate}[label=(\alph*)]
		\item Die Zielfunktion $\phi:\real^n\to\real$ der zu \cref{3.8} äquivalenten Aufgabe
		\begin{align}
			\label{3.12}
			\phi(x) = \frac{1}{2}\Vert Ax-b\Vert_2^2\to\min
		\end{align}
		lässt sich schreiben als
		\begin{align}
			\phi(x) &= \frac{1}{2}(Ax-b)^T(Ax-b) \notag \\
			&= \frac{1}{2} (x^TA^TAx - 2b^TAx+b^Tb)\notag
		\end{align}
		Die notwendige Optimalitätsbedingung für \cref{3.12} lautet $\nabla\phi(x)=0$, das heißt
		\begin{align}
			A^TAx = A^Tb\notag
		\end{align}
		Also ist jede Lösung von \cref{3.8} auch eine Lösung der \person{Gauss}'schen Normalgleichungen \cref{3.11}. Da $\phi$ eine konvexe Funktion ist (wegen $\nabla^2\phi(x)=A^TA$ positiv semidefinit), ist \cref{3.11} zugleich eine hinreichende Optimalitätsbedingung, das heißt jede Lösung von \cref{3.11} löst \cref{3.8}.
		\item Sei $\rang(A)=n$. Dann hat $A$ vollen Spaltenrang und $Ax\neq 0$ für alle $x\neq 0$. Folglich gilt $x^TA^TAx=(x^TA^T)(Ax)=\Vert Ax\Vert_2^2>0$ für alle $x\neq 0$. Also ist $A$ positiv definit und damit regulär. Somit sind die \person{Gauss}'schen Normalgleichungen \cref{3.11} eindeutig lösbar, ihre Lösung ist $x^*=(A^TA)^{-1}A^Tb$. Wegen Teil (a) ist dies auch die einzige Lösung von \cref{3.8}.
		\item Sei $\rang(A)>n$. Dann gibt es $\hat{x}\neq 0$ mit $A\hat{x}=0$. Folglich ist einerseits $A$ positiv semidefinit (denn $x^TA^TAx=\Vert Ax\Vert_2^2\ge 0$) aber andererseits $A^TA\hat{x}=0$ und $A^TA$ daher singulär. Da nach \propref{3_3_1} das lineare Quadraturmittelproblem \cref{3.8} eine Lösung besitzt, muss nach Teil (a) auch \cref{3.11} lösbar sein. Aufgrund der Singularität von $A^TA$ hat \cref{3.11} unendlich viele Lösungen.
	\end{enumerate}
\end{proof}

Sei $x^*$ eine Lösung von \cref{3.8}. Dann gilt wegen \propref{3_3_3} 
\begin{align}
	0 = A^TAx^* - A^Tb = A^T(Ax^*-b)\notag
\end{align}
Dies ist äquivalent zu folgenden Aussagen
\begin{itemize}
	\item $0=x^TA^T(Ax^*-b)$
	\item $(Ax^*-b)\perp Ax$
	\item $(Ax^*-b)\perp L$
\end{itemize}

\begin{algorithm}[Prinzip des Normalgleichungsverfahrens]
	Input: $A\in\real^{m\times n}$ mit $\rang(A)=n$, $b\in\real^m$
	\begin{lstlisting}
G = transpose(A) * A
c = transpose(A) * b
compute L ! als Cholesky-Faktor von G
solve Lz=c
solve transpose(L)x = z
	\end{lstlisting}
	Output: $x$, $L$
\end{algorithm}

\begin{remark}
	Der Aufwand beträgt etwa $mn^2$ Operationen zur Berechnung der unteren Hälfte von $G$, $\frac{n^3}{3}$ für die \person{Cholesky}-Faktorisierung sowie je $n^2$ für die Lösung der Dreieckssysteme. Offenbar ist der Aufwand für kleine $n$ günstig. Nachteilig bezüglich numerischer Fehler kann sich beim Normalgleichungsverfahren die schlechte Kondition (siehe später) der Matrix $A^TA$ auswirken. Abhilfe schaffen geeignete Nachiterationen oder andere Verfahren (\person{Householder}, SVD) zur Lösung des linearen Quadraturmittelproblems.
\end{remark}

\subsection{Orthonormalisierungsverfahren nach \person{Householder}}

\subsection{Anwendung der Ausgleichsrechnung}