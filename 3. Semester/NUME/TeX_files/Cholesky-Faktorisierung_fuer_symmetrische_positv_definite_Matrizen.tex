\section{\person{Cholesky}-Faktorisierung für symmetrische positiv definite Matrizen}

\begin{definition}[positiv definit]
	Eine Matrix $A\in\real^{n\times n}$ heißt \begriff{positiv definit}, wenn
	\begin{align}
		x^TAx>0\quad\forall x\in\real^n\backslash\{0\}\notag
	\end{align}
\end{definition}

Bekannt sind folgende Zusammenhänge:
\begin{itemize}
	\item Falls $A$ symmetrisch ist, besitzt $A$ nur reelle Eigenwerte.
	\item Sei $A$ symmetrisch. Dann ist $A$ genau dann positiv definit, wenn alle Eigenwerte von $A$ positiv sind.
	\item Eine Matrix $A$ ist genau dann positiv definit, wenn ihr symmetrischer Anteil $\frac{1}{2}(A+A^T)$ positiv definit ist.
\end{itemize}

\subsection{Existenz der \person{Cholesky}-Faktorisierung}

\begin{proposition}
	Sei $A\in\real^{n\times n}$ symmetrisch und positiv definit. Dann existiert genau eine untere Dreiecksmatrix $L=(l_{ik})$ mit $l_{kk}>0$, sodass
	\begin{align}
		A=LL^T\notag
	\end{align}
\end{proposition}

\subsection{Berechnung des \person{Cholesky}-Faktors}