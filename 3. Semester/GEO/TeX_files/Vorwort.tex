Wir freuen uns, dass du unser Skript für die Vorlesung \textit{Geometrie} bei Prof. Dr. Arno Fehm im WS2018/19 gefunden hast. Da du ja offensichtlich seit einem Jahr Mathematik studierst, kannst du dich glücklich schätzen zu dem einen Drittel zu gehören, dass nicht bis zum zweiten Semester abgebrochen hat.

Wenn du schon das Vorwort zu \textit{Lineare Algebra und analytische Geometrie 1+2} gelesen hast, weißt du sicherlich, dass Prof. Fehm ein Freud der Algebra ist.\footnote{In Zukunft wird sich Prof. Fehm richtig freuen dürfen, denn im Zuge einer neuen Studienordnung, die am 1.4.2019 in Kraft tritt, kommt so gut wie keine Geometrie im \textit{Bachelor Mathematik} vor.} Auf die Frage eines Kommilitonen, wo in seinem Inhaltsverzeichnis (Gruppen, Ringe, Körper) die Geometrie vorkomme, antwortete er:
\begin{quote}
	\textit{Die Frage ist nicht, wieso wir in dieser Vorlesung Algebra statt Geometrie machen, sondern warum hier seit 20 Jahren Geometrie unterrichtet wird.}
\end{quote}

Wie auch im letzten Vorwort können wir dir nur empfehlen die Vorlesung immer zu besuchen, denn dieses Skript ist kein Ersatz dafür. Es soll aber ein Ersatz für deine unleserlichen und (hoffentlich nicht) unvollständigen Mitschriften sein und damit die Prüfungsvorbereitung einfacher machen. Im Gegensatz zu letztem Semester veröffentlicht Prof. Fehm auf seiner Homepage (\url{http://www.math.tu-dresden.de/~afehm/lehre.html}) kein vollständiges Skript mehr, sondern nur noch eine Zusammenfassung.

Der Quelltext dieses Skriptes ist bei Github (\url{https://github.com/henrydatei/TUD_MATH_BA}) gehostet; du kannst ihn dir herunterladen, anschauen, verändern, neu kompilieren, ... Auch wenn wir das Skript immer wieder durchlesen und Fehler beheben, können wir leider keine Garantie auf Richtigkeit geben. Wenn du Fehler finden solltest, wären wir froh, wenn du ein neues Issue auf Github erstellst und dort beschreibst, was falsch ist. Damit wird vielen (und besonders nachfolgenden) Studenten geholfen.

Und jetzt viel Spaß bei \textit{Geometrie}!

\begin{flushright}
	Henry, Pascal und Daniel
\end{flushright}