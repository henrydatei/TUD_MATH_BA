\section{Abelsche Gruppen}

Sei $G$ ein Gruppe.

\begin{definition}[zyklische Gruppe]
	Eine Gruppe $G$ ist \begriff[Gruppe!]{zyklisch} $\Leftrightarrow G=\langle g\rangle$ für ein $g\in G$.
\end{definition}

\begin{example}
	\begin{enumerate}[label=(\alph*)]
		\item $\whole=\langle 1\rangle$
		\item $\whole/n\whole=\langle\overline{1}\rangle$
		\item $C_n=\langle (1\, 2\, ...\, n)\rangle\le S_n$
		\item Ist $\#G=p$ eine Primzahl, so ist $G$ zyklisch (Übung 6)
	\end{enumerate}
\end{example}

\begin{lemma}
	\proplbl{1_4_3}
	Die Untergruppen von $(\whole,+)$ sind genau die $\langle k\rangle=\whole k$ mit $k\in\natur_0$ und für $k_1,...,k_r\in\whole$ ist $\langle k_1,...,k_r\rangle=\langle k\rangle$ mit
	\begin{align}
		k=\ggT(k_1,...,k_r)\notag
	\end{align}
\end{lemma}
\begin{proof}
	Zwei Beweise sind möglich:
	\begin{enumerate}
		\item Jede Untergruppe von $\whole$ ist ein Ideal von $(\whole,+,\cdot)$ und $\whole$ ist ein Hauptidealring.
		\item Sei $H\le\whole$. Setze $k=\min\{H\cap N\}$, ohne Einschränkung $H\neq\{0\}$.
		\begin{itemize}
			\item $H=\langle k\rangle$: $n\in H\Rightarrow n=qk+r$ mit $q,r\in\whole$, $0\le r<k\Rightarrow r=n-\underbrace{qk}_{k+...+k}\in H\xRightarrow[\text{mal}]{k\text{ mini-}}r=0\Rightarrow n\in\langle k\rangle$
			\item $\langle k_1,...,k_r\rangle=\langle k\rangle\Rightarrow k=\ggT(k_1,...,k_r)$: \\
			$k_i\in\langle k\rangle\Rightarrow k\vert k_i\quad\forall i$ \\
			$k\in\langle k_1,...,k_r\rangle\Rightarrow k=n_1k_1+...+n_rk_r$ mit $n_i\in\whole$ $\exists d\vert k_i\Rightarrow d\vert k\Rightarrow k=\ggT(k_1,...,k_r)$
		\end{itemize}
	\end{enumerate}
\end{proof}

\begin{proposition}[Klassifikation von zyklischen Gruppen]
	Sei $G=\langle g\rangle$ zyklisch. Dann ist $G$ abelsch und
	\begin{enumerate}[label=(\alph*)]
		\item $G\cong (\whole,+)$ \emph{oder}
		\item $G\cong (\whole/n\whole,+)$ mit $n=\#G<\infty$
	\end{enumerate}
\end{proposition}
\begin{proof}
	Betrachte 
	\begin{align}
		\phi: \begin{cases}
		\whole\to G\\ k\mapsto g^k
		\end{cases}\notag
	\end{align}
	$\phi$ ist ein Homomorphismus und surjektiv, da $G=\langle g\rangle$. Nach \propref{1_3_9} ist $G=\Image(\phi)\cong \lnkset{\whole}{\Ker(\phi)}$. Nach \propref{1_4_3} ist $\Ker(\phi)=\langle n\rangle$ für ein $n\in\natur_0$.
	\begin{itemize}
		\item \emph{$n=0$}, so ist $\Ker(\phi)=\langle 0\rangle$, also $\phi$ injektiv und $G\cong\whole$.
		\item \emph{$n>0$}, so ist $G\cong\whole/n\whole$ und $n=\#\whole/n\whole=\#G$.
	\end{itemize}
\end{proof}

\begin{proposition}
	\proplbl{1_4_5}
	Sei $G=(G,+)=\langle g \rangle$ zyklisch der Ordnung $n \in \natur$.
	\begin{enumerate}[label=(\alph*)]
		\item Zu jedem $d \in \natur$ mit $d\mid n$ hat $G$ genau eine Untergruppe der Ordnung $d$, nämlich $U_d=\langle \frac{n}{d}g\rangle$
		\item Für $d \mid n$ und $d'\mid n$ ist $U_d \le U_{d'} \Leftrightarrow d \mid d'$
		\item Für $k_1, \dots , k_k \in \whole$ ist $\langle k_1 g, \dots, k_r g \rangle = \langle eg\rangle = U_{\sfrac{n}{e}}$ mit $e = \ggT(k_1,\dots,k_r,n)$
		\item Für $k \in \whole$ ist $\ord(kg)= \frac{n}{\ggT(k,n)}$
	\end{enumerate}
\end{proposition}
\begin{proof}
	Betrachte wieder 
	\begin{align}
		\phi: \begin{cases}\overline{k} \to G \\k \mapsto kg\end{cases}\notag
	\end{align}
	\begin{enumerate}[label=(\alph*)]
		\item Nach \propref{1_3_7} und \propref{1_4_3} liefert $\phi$ Bijektion 
		\begin{align}
			\{ e \in \natur \mid n\whole \le e\whole \} \xrightarrow{\propref{1_1_1}} \{ H \le G \}\notag
		\end{align}
		 und $n\whole \le e\whole \Leftrightarrow e \mid n$. Ist $H = \phi(e\whole) = \langle eg \rangle$, so ist $H \cong \lnkset{e\whole}{n\whole}$, also $n = (\whole :  n \whole) = (\whole : e\whole)\cdot(e\whole : n\whole) = e \cdot \#H$
		\item $U_d \le U_{d'} \Leftrightarrow \langle \frac{n}{d}g \rangle \le \langle \frac{n}{d'}g \rangle \Leftrightarrow \frac{n}{d}\whole \le \frac{n}{d'}\whole \Leftrightarrow \frac{n}{d'} \mid \frac{n}{d} \Leftrightarrow d \mid d'$
		\item Mit $H = \langle k_1, \dots, k_r,n \rangle \le \whole$ ist $n\whole \le H$, $\phi(H) = \langle k_1 g, \dots, k_r g\rangle$. Nach \propref{1_4_3} ist $H = \langle e \rangle$ mit $e = \ggT(k_1, \dots, k_r, n)$, somit $\langle k_1 g, \dots, k_r g \rangle = \phi(e\whole) = U_{\sfrac{n}{e}}$
		\item $\ord(kg) = \#\langle kg \rangle \overset{c)}{=} \#U_{\sfrac{n}{e}}$ mit $e = \ggT(k,n)$
	\end{enumerate}
\end{proof}

\begin{lemma}
	\proplbl{1_4_6}
	Seien $a,b \in G$. Kommutieren $a$ und $b$ und sind $\ord(a)$ und $\ord(b)$ teilerfremd, so ist
	\begin{align}
		\ord(a,b) = \ord(a)\cdot \ord(b) \notag
	\end{align}
\end{lemma}
\begin{proof}
	Nach \propref{1_2_12} ist $\langle a \rangle \cap \langle b \rangle = 1$. Ist $(ab)^k = 1 = a^k b^k$, so ist $a^k = b^{-k} \in \langle a \rangle \cap \langle b \rangle = 1$, also $a^k = b^k = 1$. Somit ist $(ab)^k = 1 \Leftrightarrow a^k = 1$ und $b^k =1$ und damit $\ord(ab) = \kgV(\ord(a), \ord(b)) = \ord(a) \cdot \ord(b)$
\end{proof}

\begin{conclusion}
	\proplbl{1_4_7}
	Ist $G$ abelsch und sind $a,b \in G$ mit $\ord(a) = m < \infty$, $\ord(b) = n < \infty$, so existiert $c \in G$ mit
	\begin{align}
		\ord(c) = \kgV(\ord(a), \ord(b)) \notag
	\end{align}
\end{conclusion}
\begin{proof}
	Schreibe $m = m_0 m'$ und $n = n_0 n'$ mit $m_0 n_0 = \kgV(m,n)$ und $\ggT(m_0, n_0) = 1 \Rightarrow \ord(a^{m'}) = m_0$, $\ord(b^{n'}) = n_0 \Rightarrow \ord(b^{n'} \cdot a^{m'}) \overset{\propref{1_4_6}}{=} m_0 \cdot n_0 = \kgV(m,n)$.
\end{proof}

\begin{theorem}[Struktursatz für endlich erzeugte abelsche Gruppen]
	\proplbl{1_4_8}
	Jede endliche erzeugte abelsche Gruppe $G$ ist eine direkte Summe zyklischer Gruppen
	\begin{align}
		G \cong \whole^{r} \oplus\bigoplus_{i=1}^k \lnkset{\whole}{d_i \whole} \notag
	\end{align}
	mit eindeutig bestimmten $d_1, \dots, d_k > 1$ die $d_i \mid d_{i+1}$ für alle $i$ erfüllen.
\end{theorem}
\begin{proof}
	\begin{itemize}
		\item Existenz: LAAG 2: VIII. 6.14
		\item Eindeutigkeit: Für $d \in \natur$ ist 
		\begin{align}
			\# \lnkset{G}{dG} &= \#\left( \lnkset{\whole}{d \whole}\right)^r \oplus \bigoplus_{i=1}^k \lnkset{\left(\lnkset{\whole}{d_i\whole}\right)}{d\cdot\left(\lnkset{\whole}{d_i\whole}\right)} \notag \\
			&\overset{\propref{1_4_5}}{=} d^r \cdot \prod_{i=1}^{n} \frac{d_i}{\ggT(d,d_i)}\notag
		\end{align} 
	\end{itemize}
und daraus kann man $r, k, d_1, \dots , d_k$ erhalten.
\end{proof}

\begin{lemma}
	Sei $G=(G,+) = \langle g\rangle$ zyklisch der Ordnung $n \in \natur$. Die Endomorphismen von $G$ sind genau die 
	\begin{align}
		\phi_{\overline{k}}: \begin{cases}
		G \to G \\
		x \mapsto kx
		\end{cases} \text{ für } \overline{k} = k + n\whole \in \whole/n\whole\notag
	\end{align}
	Dabei ist $\phi_{\overline{l}}\circ\phi_{\overline{k}} = \phi_{\overline{kl}}$.
\end{lemma}
\begin{proof}
	\begin{itemize}
		\item $\phi_{\overline{k}}$ wohldefiniert $\overline{k_1} = \overline{k_2} \Rightarrow k_2 = k_1 +an$ mit $a \in \whole\Rightarrow k_2 x = k_1 x + a n \cdot x$, aber $nx=0$.
		\item $\phi_{\overline{k}} \in \Hom(G,G)$: klar, da $G$ abelsch
		\item $\overline{k}=\overline{l}\Leftrightarrow \phi_{\overline{k}} = \phi_{\overline{l}}$: $\phi_{\overline{k}}(g) = \phi_{\overline{l}}(g) \Rightarrow (k-l)g = 0 \xRightarrow[=n]{\ord(g)} n \mid (k-l) \Rightarrow \overline{k} = \overline{l}$
		\item $\phi \in \Hom(G,G) \Rightarrow \phi = \phi_{\overline{k}}$ für ein $k \in \whole$: $\phi(g) = kg$ für ein $k \Rightarrow \phi = \phi_{\overline{k}}$
		\item $\phi_{\overline{k}} \circ \phi_{\overline{l}} = \phi_{\overline{kl}}$: $l(kx) = (lk)x$
	\end{itemize}
\end{proof}

\begin{proposition}
	Ist $G$ zyklisch von Ordnung $n \in \natur$, so ist
	\begin{align}
		\Aut(G) \cong (\whole/n\whole)^{\times}\notag
	\end{align}
\end{proposition}
\begin{proof}
	$\Aut(G) \subseteq \Hom(G,G) = \{ \phi_{\overline{k}} \mid \overline{k} \in \whole/n\whole\}$, $\phi_{\overline{k}}\in\Aut(G)\Leftrightarrow$ es existiert ein $\overline{l}\in\whole/n\whole$ mit $\phi_{\overline{l}}\circ\phi_{\overline{k}}=\phi_{\overline{1}}$ also existiert ein $\overline{l}\in\whole/n\whole$ mit $\overline{kl}=1\Leftrightarrow\overline{k}\in (\whole/n\whole)^\times$ und 
	\begin{align}
		\begin{cases}
		(\whole/n\whole)^\times&\to \Aut(G) \\
		\overline{k}&\mapsto \phi_{\overline{k}}
		\end{cases}\notag
	\end{align}
	ist ein Isomorphismus.
\end{proof}

\begin{definition}[\person{Euler}'sche Phi-Funktion]
	\begin{align}
		\Phi(n) = \#(\whole/n\whole)^\times\notag
	\end{align}
	ist die \begriff{\person{Euler}'sche Phi-Funktion}.
\end{definition}

\begin{example}
	$p$ prim $\Rightarrow$ $\phi(p) = p-1$, da $\whole / p\whole$ Körper ist.
\end{example}

\begin{proposition}
	Ist $K$ ein Körper und $H\leq K^{\times}$ abelsch, so ist $H$ zyklisch.
\end{proposition}
\begin{proof}
	Setze $m=\max\{\ord(h) \colon h \in H\}$. Nach \propref{1_4_7} gilt $\ord(h) \mid m \quad\forall h \in H$. $\Rightarrow$ Jedes $h \in H$ ist Nullstelle von $f = x^m -1 \in K[x]$. $\Rightarrow$ $\# H \leq \deg f = m \leq \# H \Rightarrow \# H = m$. Ist $h \in H$ mit $m = \ord(h)$, so ist dann $H = \langle h \rangle$.
\end{proof}

\begin{conclusion}
	\proplbl{1_4_14}
	Für $p \in \natur$ prim ist
	\begin{align}
		\Aut(C_p) \cong (\whole / p \whole)^{\times} \cong C_{p-1}\notag
	\end{align}
\end{conclusion}