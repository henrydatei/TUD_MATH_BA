\section{Einfache Gruppen}

Sei $G$ eine Gruppe und $n \in \natur$.

\begin{definition}[Einfache Gruppe]
	Eine Gruppe $G$ heißt \begriff{einfach}, wenn $G \neq 1$ und es kein $1 \neq N \not\unlhd G$ gibt. %TODO negation of \unlhd? please FIXME!!!
\end{definition}

\begin{remark}
	Die einfachen Gruppen sind die grundlegenden "'Bausteine'" der Gruppen, siehe Kapitel 1.10.
\end{remark}

\begin{example}
	\begin{enumerate}[label=(\alph*)]
		\item $C_n$ ist einfach $\Leftrightarrow n$ ist prim, da dann $\#C_n$ keine weiteren Teiler hat, also auch keine Untergruppen
		\item Sei $G$ endlich abelsch. Dann: $G$ ist einfach $\Leftrightarrow G \cong C_p$, $p$ prim
		\item Sei $G$ eine $p$-Gruppe. Dann: $G$ ist einfach $\xLeftrightarrow{\propref{1_7_4}} G \cong C_p$,$ p$ prim
		\item $A_2 = \{\id\}$ und damit einfach
		\item $A_3$ ist einfach, da $A_3 \cong C_3$
		\item $A_4$ ist nicht einfach (da $V_4 \unlhd A_4$ gilt)
	\end{enumerate}
\end{example}

\begin{definition}[Typ]
	Sei $\sigma \in S_n$. Ist $\sigma = \sigma_1\cdots \sigma_k$ eine Zerlegung in paarweise disjunkte Zykel $\sigma_i$ mit $\ord(\sigma_i)=r_i$, wobei $r_1 \geq r_2 \geq \dots \geq r_k \geq 2$, so heißt
	\begin{align}
	\Typ(\sigma) := (r_1,\dots,r_k,\underbrace{1,\dots,1}_{=\#\Fix(\sigma)})\notag
	\end{align}
	der \begriff{Typ} von $\sigma$.
\end{definition}

\begin{example}
	Sei $\sigma = (12)(25) \in S_5$. Die Zykelzerlegung ist $\sigma=(1\, 5\, 2)$. Also ist $\Typ(\sigma)=(3,1,1)$. Die beiden Fixpunkte sind 3 und 4.
\end{example}

\begin{definition}[Partition]
	Eine \begriff{Partition} von $n$ ist eine endliche Folge $(r_1,\dots,r_k)$ mit $r_1,\dots,r_k \in \natur$, $r_1\geq\dots\geq r_k$ und $n = \sum_{i=1}^{k} r_i$.
\end{definition}

\begin{lemma}
	$\{\Typ(\sigma) \mid \sigma \in S_n\}$ ist genau die Menge der Partitionen von $n$.
\end{lemma}
\begin{proof}
	\begin{itemize}
		\item Hinrichtung: klar
		\item Rückrichtung: Ist $(r_1,...,r_k)$ eine Partition von $n$, so ist $(r_1,...,r_k)=\Typ(\sigma)$ für
		\begin{align}
			\sigma = (1\, ...\, r_1)(r_1+1\, ...\, r_1+r_2)...(1+\sum_{i=1}^{k-1}r_i \, ...\, n)\notag
		\end{align}
	\end{itemize}
\end{proof}

\begin{proposition}
	\proplbl{1_9_8}
	Für $\sigma, \sigma' \in S_n$ sind äquivalent:
	\begin{enumerate}[label=(\alph*)]
		\item $\sigma, \sigma'$ sind konjugiert in $S_n$.
		\item $\Typ(\sigma) = \Typ(\sigma')$
	\end{enumerate}
\end{proposition}
\begin{proof}
	Schreibe $\sigma=\sigma_1...\sigma_k$ paarweise disjunkte Zykel, $r_\nu = \ord(\sigma_\nu)$, $r_1\ge ...\ge r_k$, $\sigma_\nu=(i_{\nu,1}\, ...\, i_{\nu,r_\nu})$, $\{1,...,n\}=\{i_{\nu,\mu} \mid \nu,\mu\}$
	\begin{itemize}
		\item (a) $\Rightarrow$ (b): Ist $\sigma'=\sigma^\tau$ mit $\tau\in S_n$, so ist
		\begin{align}
			\sigma^\tau = \sigma_1^\tau...\sigma_k^\tau \quad\text{und}\quad \sigma_\nu^\tau = (i_{\nu,1}^\tau\, ...\, i_{\nu,r_\nu}^\tau)\notag
		\end{align}
		Insbesondere $\Typ(\sigma)=\Typ(\sigma')$.
		\item (b) $\Rightarrow$ (a): Ist $\Typ(\sigma)=\Typ(\sigma')$, so ist $\sigma'=\sigma_1'...\sigma_k'$ mit paarweise disjunkten Zykeln mit $\ord(\sigma_\nu')=r_\nu$ und $\sigma_\nu'=(i_{\nu,1}'\, ...\, i_{\nu,r_\nu}')$ und deshalb wieder $\{1,...,n\}=\{i_{\nu,\mu}' \mid \nu=1,...,k,\mu=1,...,r_\nu\}$. Mit $\tau\in S_n$ definiert durch 
		\begin{align}
			i_{\nu,\mu}^\tau = i_{\nu,\mu}'\notag
		\end{align}
		ist dann $\sigma^\tau=\sigma_1^\tau...\sigma_k^\tau$ mit $\sigma_\nu^\tau = (i_{\nu,1}^\tau\, ...\, i_{\nu,r_\nu}^\tau)=(i_{\nu,1}'\, ...\, i_{\nu,r_\nu}')=\sigma_\nu'$.
	\end{itemize}
\end{proof}

\begin{lemma}
	Ist $n \geq 5$, so sind je zwei $3$-Zykel konjugiert in $A_n$.
\end{lemma}
\begin{proof}
	Seien $\sigma=(i_1\, i_2\, i_3)$ und $\sigma'=(i_1'\, i_2'\, i_3')$. Nach \propref{1_9_8} existiert $\tau\in S_n$ mit $\sigma'=\sigma^\tau$. Ist $\tau\in A_n$, so sind wir fertig. Andernfalls gibt es wegen $n\ge 5$ $j_1\neq j_2\in\{1,...,n\}\setminus \{i_1',i_2',i_3'\}$. Dann ist $\tau(j_1,j_2)\in A_n$ und
	\begin{align}
		\sigma^{\tau(j_1,j_2)} = (\sigma^\tau)^{(j_1,j_2)} = (\sigma')^{(j_1,j_2)} = \sigma'\notag
	\end{align}
\end{proof}
