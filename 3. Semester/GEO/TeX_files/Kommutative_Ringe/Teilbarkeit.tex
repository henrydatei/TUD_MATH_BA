\section{Teilbarkeit}

Sei $R$ ein \textbf{nullteilfreier} Ring.

\begin{definition}[teilt, assoziert, irreduzibel, prim, ggT, kgV]
	\proplbl{2_4_1}
	Seinen $x,y,z,p \in R$.
	\begin{enumerate}[label=(\alph*)]
		\item $x$ \begriff{teilt} $y$ (in Zeichen $x \mid y$) $\Leftrightarrow \exists z \in R: xz = y$
		\item $x$ ist \begriff{assoziert} zu $y$ (in Zeichen $x \sim y$) $\Leftrightarrow \exists z \in R^{\times}, xz = y$
		\item $p$ ist \begriff{irreduzibel} $\Leftrightarrow p \not \in R^{\times} \cup \{  0\}$ und $\forall x,y \in R$:
		\begin{align}
			p = xy &\Rightarrow x \in R^{\times} \text{ oder } y \in R^{\times} \notag
		\end{align}
		\item $p$ ist \begriff{prim} $\Leftrightarrow  p \not \in R^{\times} \cup \{ 0 \}$ und $\forall x,y \in R$:
		\begin{align}
			p \mid xy &\Rightarrow p \mid x \text{ oder } p \mid y \notag
		\end{align}
		\item $z$ ist \begriff{größter gemeinsamer Teiler} von $x,y$ (in Zeichen $z = \ggT(x,y)$) $\Leftrightarrow  z \mid x$ und $z \mid y$ und $\forall a \in R: a\mid x$ und $a \mid y\Rightarrow a \mid z$
		\item $z$ ist \begriff{kleinstes gemeinsames Vielfaches} von $x,y$ (in Zeichen $z = \kgV(x,y)$) $\Leftrightarrow x \mid z$ und $y \mid z$ und $\forall a \in R: x \mid a$ und $y \mid a \Rightarrow z \mid a$ 
	\end{enumerate}
\end{definition}

\begin{remark}
	\begin{enumerate}[label=(\alph*)]
		\item $x \mid y \Leftrightarrow y \in (x) \Leftrightarrow (y) \subseteq (x)$
		\item $x \sim y \Leftrightarrow x \mid y \text{ und } y \mid x \Leftrightarrow (x) = (y)$
		\item $p$ prim $\Leftrightarrow (p)$ prim und $p \neq 0$
		\item $p$ prim $\Rightarrow p$ irreduzibel
		\item Analog zu e) und f) in \propref{2_4_1} kann man ein $\ggT$ bzw. $\kgV$ von endlich vielen Elementen von $R$ definieren.
	\end{enumerate}
\end{remark}
\begin{proof}
	\begin{itemize}
		\item[d)] $p = xy \Rightarrow p \mid xy \xRightarrow{p \text{ prim}} p \mid x \text{ oder } p\mid y$, ohne Einschränkung $p \mid x$, das heißt $x = px'$ mit $x' \in R \Rightarrow p(1-x'y) =0 \Rightarrow 1 - x'y = 0 \Rightarrow 1 = x'y \Rightarrow y \in R^{\times}$
	\end{itemize}
\end{proof}

\begin{definition}[euklidisch, Hauptidealring, faktoriell]
	\begin{enumerate}[label=(\alph*)]
		\item $R$ ist \begriff{euklidisch} $\Leftrightarrow$ es gibt eine euklidische Gradfunktion:
		\begin{align}
			\delta: R\setminus \{0\} \to \natur_0 \notag
		\end{align}
		mit $\forall x,y \in R\setminus \{0\} \exists q,r \in R$ mit $x = qy + r$ und $r=0$ oder $\delta(r) < \delta(y)$
		\item $R$ ist \begriff{Hauptidelring} $\Leftrightarrow$ Jedes Ideal $I \unlhd R$ ist ein Hauptideal.
		\item $R$ ist \begriff{faktoriell} $\Leftrightarrow$ Jedes $0 \neq x \in R\setminus R^{\times}$ ist ein Produkt von Primelementen.
	\end{enumerate}
\end{definition}

\begin{proposition}
	$R$ euklidisch $\Rightarrow R$ Hauptidealring $\Rightarrow R$ faktoriell.
\end{proposition}

\begin{proof}
	LAAG VIII.3.6 und VIII.4.4.
\end{proof}

\begin{example}
	\begin{enumerate}[label=(\alph*)]
		\item $\whole$, $K[x]$: euklidisch
		\item $K$ Körper: euklidisch
		\item $\whole[i] = \{ a+b\sqrt{-1} \mid a,b \in \whole \} \subseteq \comp$: euklidisch mit $\delta(z) = \vert z \vert^2$
		\item $K[x,y], \whole[x]$: keine Hauptidealringe
		\item $\whole[\sqrt{-5}] = \{ a+b\sqrt{-5} \mid a,b \in \whole \} \subseteq \comp$: nicht faktoriell, da $6 = 2\cdot 3 = (1 + \sqrt{-5})(1-\sqrt{-5})$
	\end{enumerate}
\end{example}

\begin{remark}
	Ist $R$ faktoriell, so gilt:
	\begin{enumerate}[label=(\alph*)]
		\item $p$ irreduzibel $\Leftrightarrow p$ prim
		\item Eine Darstellung von $0 \neq x \in R \setminus R^{\times}$ als Produkt von Primelementen ist eindeutig bis auf Reihenfolge und Assoziiertheit.
	\end{enumerate}
\end{remark}