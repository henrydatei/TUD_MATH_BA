\part{Metrische Räume und Konvergenz}
\begin{description}
	\item[Konvergenz:] grundlegender Begriff in Analysis %(benötigt Abstandsbegriff (Metrik))
\end{description}
\chapter{Grundlegen Ungleichungen}

\begin{satz}[Geometrisches und arithmetisches Mittel]
	Seien $x_1, \dots, x_n \in \real_{>0}$\\
$\Rightarrow$
	\begin{tabular}{ccc}
		$ \sqrt[n]{x_1, \dots, x_n}$ & $=$ & $\frac{x_1, \dots, x_n}{n}$ \\
		geoemtrisches Mittel &  & arithmetisches Mittel \\
	\end{tabular}\\
Gleichheit gdw $x_1 = \dots = x_n$.
\end{satz}

\begin{proof}
	Zeige zunächst mit vollständiger Induktion\\
	\begin{equation}
	\prod_{i=1}^{n}x_i=1 \Rightarrow \sum_{i=1}^{n} x_i \geq n
	\end{equation}
	\item 
	\QEDA
\end{proof}
%continue