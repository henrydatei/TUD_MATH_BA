\part{Grundlagen der Mathematik}
\chapter{Grundbegriffe aus Mengelehre und Logik}
\textbf{Mengenlehre:} Universalität von Aussagen \\
\textbf{Logik:} Regeln des Folgerns, wahre/falsche Aussagen


\begin{mydef}[Definition Aussage]
	Sachverhalt, dem man entweder den Wahrheitswert "wahr" oder "falsch" zuordnen kann, aber nichts anders.
\end{mydef}

Beispiele:
\begin{exmp}
	\item 5 ist eine Quadratzahl $\to$ falsch (Aussage)
	\item Die Elbe flie{\ss}t durch Dresden $\to$ wahr (Aussage)
	\item Mathematik ist rot $\to$ ??? (keine Aussage)
\end{exmp}

\begin{mydef}[Menge]
	Zusammenfassung von bestimmten wohlunterscheidbaren Objekten der Anschauung oder des Denkens, welche die Elemente der Menge genannt werden, zu einem Ganzen.\\ (\textsc{Cantor}, 1877)
\end{mydef}

Beispiele:
\begin{compactitem}
	\item $M_1 :=$ Menge aller Städte in Deutschland
	\item $M_2 := \{1;2;3\}$ 
\end{compactitem}

$\newline$ 
Für ein Objekt $m$ und eine Menge $M$ gilt stets $m \in M$ oder $m \notin M$ \\
Für die Mengen $M$ und $N$ gilt $M=N$, falls dieselben Elemente enthalten sind 
$\{1;2;3\} = \{3;2;1\} = \{1;2;2;3\}$ \\
- $N \subseteq M$, falls $n \in M$ für jedes $n \in N$ \\
- $N \subset M$, falls zusätzlich $M \neq N$ \\

\begin{mydef}[Aussageform]
	Sachverhalt mit Variablen, der durch geeignete Ersetzung der Variablen zur Aussage wird.
\end{mydef}

\begin{exmpn}
	\begin{itemize}
		\item $A(X) := $ Die Elbe fließt durch X
		\item $B(X;Y;Z) := X + Y = Z$
		\item aber $A(Dresden) ,B(2;3;4)$ sind Aussagen, $A(Mathematik)$ ist keine Aussage
		\item $A(X)$ ist eine Aussage f\"u jedes $X \in M_1$ $\to$ Generalisierung von Aussagen durch Mengen
	\end{itemize}
\end{exmpn}

\section*{Bildung und Verknüpfung von Aussagen}

\begin{tabular}{|c|c|c|c|c|c|c|}
	\hline
	$A$ & $B$ & $\lnot A$ & $A \land B$ & $A \lor B$ & $A \Rightarrow B$ & $A \iff B$\\
	\hline
	w & w & f & w & w & w & w\\
	\hline
	w & f & f & f & w & f & f\\
	\hline
	f & w & w & f & w & w & f\\
	\hline
	f & f & w & f & f & w & w\\
	\hline
\end{tabular}

\begin{exmpn}
	\begin{itemize}
		\item $\lnot$(3 ist gerade) $\to$ w
		\item (4 ist gerade) $\land$ (4 ist Primzahl) $\to$ f
		\item (3 ist gerade) $\lor$ (3 ist Primzahl) $\to$ w
		\item (3 ist gerade) $\Rightarrow$ (Mond ist Würfel) $\to$ w
		\item (Die Sonne ist heiß) $\Rightarrow$ (es gibt Primzahlen) $\to$ w
	\end{itemize}
\end{exmpn}

\noindent Auschließendes oder: (entweder $A$ oder $B$) wird realisiert durch $\lnot(A \iff B)$.\\
Aussageform $A(X)$ sei f\"ur jedes $X \in M$ Aussage: neue Aussage mittels Quantoren

\begin{compactitem}
	\item $\forall$: "für alle"
	\item $\exists$: "es existiert"
\end{compactitem}

\begin{exmpn}
	$\forall n \in \mathbb{N}: n$ ist gerade $\to$ f\\
	$\exists n \in \mathbb{N}: n$ ist gerade $\to$ w 
\end{exmpn}

\begin{mydef}[Tautologie bzw. Kontraduktion/Widerspruch]
	Zusammengesetzte Aussage, die unabhängig vom Wahrheitsgehalt der Teilaussagen stest wahr bzw. falsch ist.
\end{mydef}

\newpage
\begin{exmpn}
	\begin{itemize}
		\item Tautologie (immer wahr): 
		$(A) \lor (\lnot A), \lnot (A  \land (\lnot A)), (A \land B) \Rightarrow A$
		\item Widerspruch (immer falsch): $A \land (\lnot A), A \iff \lnot A$  
		\item besondere Tautologie: $(A \Rightarrow B) \iff (\lnot B \Rightarrow \lnot A)$
	\end{itemize}
\end{exmpn}

\begin{satz}[Morgansche Regeln]
	Folgende Aussagen sind Tautologien:
	\begin{itemize}{ }
		\item $\lnot(A \land B) \iff \lnot A \lor \lnot B$
		\item $\lnot(A \lor B) \iff \lnot A \land \lnot B$
	\end{itemize}
\end{satz}

\section*{Bildung von Mengen}
Seien $M$ und $N$ Mengen
\begin{compactitem}
	\item Aufzählung der Elemente: $\{1;2;3\}$
	\item mittels Eigenschaften: $\{X \in M \mid A(X)\}$
	\item $\emptyset:=$ Menge, die keine Elemente enthält
	\begin{compactitem}
		\item leere Menge ist immer Teilmenge jeder Menge $M$
		\item \textbf{Warnung:} $\{\emptyset\} \neq \emptyset$
	\end{compactitem}
	\item Verknüpfung von Mengen wie bei Aussagen
\end{compactitem}

\begin{mydef}[Mengensystem]
	Ein Mengensystem $\mathcal M$ ist eine Menge, bestehend aus anderen Mengen.
	\begin{compactitem}
		\item $\bigcup M := \{X \mid \exists M \in \mathcal M: X \in M\}$ (Vereinigung aller Mengen in 
		$\mathcal M$)
		\item $\bigcap M := \{X \mid \forall M \in \mathcal M: X \in M\}$ (Durchschnitt aller Mengen in 
		$\mathcal M$)
	\end{compactitem}
\end{mydef}

\begin{mydef}[Potenzmenge]
	Die Potenzmenge $\mathcal P$ enth\"alt alle Teilmengen einer Menge $M$. \\
	$\mathcal P(X) := \{\tilde M \mid \tilde M \subset M\}$ 
\end{mydef}

Beispiel:
\begin{compactitem}
	\item $M_3 := \{1;3;5\}$ \\
	$\to \mathcal P(M_3) = \{\emptyset, \{1\}, \{3\}, \{5\}, \{1;3\}, \{1;5\}, \{3;5\}, \{1;3;5\}\}$
\end{compactitem}

\begin{framed}
	\textbf{Satz (de Morgansche Regeln f\"ur Mengen):}
	\begin{compactitem}
		\item $(\mathop{\bigcup}_{N \in \mathcal N} N)^C = \mathop{\bigcap}_{N \in \mathcal N} N^C$ 
		\item $(\mathop{\bigcap}_{N \in \mathcal N} N)^C = \mathop{\bigcup}_{N \in \mathcal N} N^C$ 
	\end{compactitem}
\end{framed}

\begin{mydef}[Kartesisches Produkt]
	$M \times N := \{m,n \mid m \in M \land n \in N\}$ \\
	$(m,n)$ hei{\ss}t geordnetes Paar (Reihenfolge wichtig!) \\
	allgemeiner: $M_1 \times ... \times M_k := \{(m_1,...,m_k) \mid m_j \in M_j, j=1, .., k\}$ \\
	$M^k := M \times ... \times M := \{(m_1,...,m_k) \mid m_j \in M_j, j=1, .., k\}$ 
\end{mydef}

\begin{satz}[Auswahlaxiom]
	Sei $\mathcal M$ ein Mengensystem nichtleerer paarweise disjunkter Mengen $M$.
	\begin{compactitem}
		\item Es existiert eine Auswahlmenge $\tilde M$, die mit jedem $M \in \mathcal M$ genau 1 Element 				gemeinsam hat.
		\item beachte: Die Auswahl ist nicht konstruktiv!
	\end{compactitem}
\end{satz}