\chapter{Ganze und rationale Zahlen}
\textbf{Frage:} Existiert eine natürliche Zahl $x$ mit  $n=n'+x$ für ein gegebenes $n$ und $n'$? \\
\textbf{Antwort:} Das geht nur falls $n \le n'$, dann ist $x=n-n'$ \\
\textbf{Ziel:} Zahlenbereichserweiterung, sodass die Gleichung immer l\"osbar ist. Ordne jedem Paar
$(n,n') \in \mathbb N \times \mathbb N$ eine neue Zahl als L\"osung zu. Gewisse Paare liefern die
gleiche L\"osung, z.B. $(6,4),(5,3),(7,5)$. Diese m\"ussen mittels Relation identifiziert werden. \\
$\newline$

$\mathbb Q := \{(n_1,n_1'),(n_2,n_2') \in (\mathbb N \times \mathbb N) \times (\mathbb N \times 
\mathbb N) \mid n_1+n_2'=n_1'+n_2\}$ \\
$\newline$

\begin{framed}
	\textbf{Satz:} $\mathbb Q$ ist die \"Aquivalenzrelation auf $\mathbb N \times \mathbb N$
\end{framed}
$\newline$

\textbf{Beispiele:} \\
\begin{compactitem}
	\item $(5,3) \sim (6,4) \sim (7,5)$ bzw. $(5-3) \sim (6-4) \sim (7-5)$
	\item $(3,6) \sim (5,8)$ bzw. $(3-6) \sim (5-8)$
\end{compactitem}
$\newline$

\textit{Beweis: \\
	\begin{compactitem}
		\item offenbar $((n,n'),(n,n')) \in \mathbb Q \Rightarrow$ reflexiv
		\item falls $((n_1,n_1'),(n_2,n_2')) \in \mathbb Q \Rightarrow (n_2,n_2'),(n_1,n_1')) \in
		\mathbb Q \Rightarrow$ symmetrisch
		\item sei $((n_1,n_1'),(n_2,n_2')) \in \mathbb Q$ und $((n_2,n_2'),(n_3,n_3')) \in \mathbb Q
		\Rightarrow n_1+n_2'=n_1'+n_2, n_2+n_3'=n_2'+n_3 \Rightarrow n_1+n_3'=n_1'+n_3 \Rightarrow
		((n_1,n_1'),(n_3,n_3')) \in \mathbb Q \Rightarrow$ transitiv
	\end{compactitem}
} 
$\newline$

setze $\overline{\mathbb{Z} := \{[(n,n')] \mid n,n' \in \mathbb{N}\}}$ Menge der ganzen Zahlen, 
[ganze Zahl] \\
Kurzschreibweise: $\overline m := [(m,m')]$ oder $\overline n := [(n,n')]$ \\
$\newline$

\begin{framed}
	\textbf{Satz:} Sei $[(n,n')] \in \overline{\mathbb{Z}}$. Dann existiert eindeutig $n* \in 
	\mathbb N$ mit 
	$(n*,0) \in [(n,n')]$, falls $n \ge n'$ bzw. $(0,n*) \in [(n,n')]$ falls $n < n'$.
\end{framed}

\textit{Beweis: \\
	\begin{compactitem}
		\item $n \ge n' \Rightarrow \exists ! n* \in \mathbb N: n=n'+n* \Rightarrow (n*,0) \sim (n,n')$
		\item $n < n' \Rightarrow \exists ! n* \in \mathbb N: n+n*=n' \Rightarrow (0,n*) \sim (n,n')$
\end{compactitem}}
$\newline$

\textbf{Frage:} Was hat $\overline{\mathbb{Z}}$ mit $\whole$ zu tun?\\
\textbf{Antwort:} identifiziere $(n,0)$ bzw. $(n-0)$ mit $n \in \natur$ und identifiziere $(0,n)$ 
bzw. $(0-n)$ mit Symbol $-n$ \\
$\Rightarrow$ ganze Zahlen kann man eindeutig den Elementen folgender Mengen zuordnen: $\mathbb Z :=
\natur \cup \{(-n) \mid n \in \natur\}$ \\
$\newline$

\textbf{Rechenoperationen auf $\overline{\whole}$:} \\
\begin{compactitem}
	\item Addition: $\overline m + \overline n = [(m,m')]+[(n,n')]=[(m+n,m'+n')]$
	\item Multiplikation: $\overline m \cdot \overline n = [(m,m')] \cdot [(n,n')]=[(mn+m'n',mn'+m'n)]$
\end{compactitem}
$\newline$

\begin{framed}
	\textbf{Satz:} Addition und Multiplikation sind eindeutig definiert, d.h. unabh\"angig von 
	Repr\"asentant bez\"uglich $\mathbb Q$
\end{framed}
\textit{Beweis: \\
	Sei $(m_1,m_1') \sim (m_2,m_2'), (n_1,n_1') \sim (n_2,n_2') \Rightarrow m_1+m_2'=m_1'+m_2, n_1
	+n_2'=n_1'+n_2 \Rightarrow m_1+n_1+m_2'+n_2'=m_1'+n_1'+m_2+n_2 \Rightarrow (m_1,m_1')+(n_1,n_1')
	\sim (m_2,m_2')+(n_2,n_2')$} \\
$\newline$

\begin{framed}
	\textbf{Satz:} F\"ur Addition und Multiplikation auf $\mathbb Z$ gilt $\forall \overline m, 
	\overline{n} \in \overline{\whole}$:
	\begin{compactitem}
		\item es existiert eine neutrales Element: $0:=[(0,0)]$, $1:=[(1,0)]$
		\item jeweils kommutativ, assoziativ und gemeinsam distributiv
		\item $- \overline{n} := [(n',n)] \in \whole$ ist invers bez\"uglich der Addition zu 
		$[(n,n')] = \overline n$
		\item $(-1) \cdot \overline n = - \overline n$
		\item $\overline m \cdot \overline n = 0 \iff \overline m =0 \lor \overline n=0$
	\end{compactitem}
\end{framed}

\textit{Beweis: \\
	\begin{compactitem}
		\item offenbar $\overline n +0=0+\overline n=\overline n$ und $\overline n \cdot 1 = 1 \cdot 
		\overline n = \overline n$
		\item Flei{\ss}arbeit
		\item offenbar $\overline n+(- \overline n) = (- \overline n)+\overline n=[(n+n',m+m')]=0$
		\item $(-1)\cdot \overline n = [(0,1)]\cdot [n,n']=[n',n]=-\overline n$
		\item \"Ubungsaufgabe
\end{compactitem}}

\begin{framed}
	\textbf{Satz:} F\"ur $\overline m, \overline n \in \mathbb Z$ hat die Gleichung $\overline m=
	\overline n + \overline x$ die L\"osung $\overline x=\overline m+(-\overline n)$
\end{framed}
$\newline$

Ordnung auf $\overline{\whole}:$ betrachte Relation $R := \{(\overline{m},\overline{n}) \in 
\overline{\whole} \times \overline{\whole} \mid \overline{m} \le \overline{n}\}$

\begin{framed}
	\textbf{Satz:} $R$ ist Totalordnung auf $\mathbb Z$ und verträglich mit Addition und 
	Multiplikation
\end{framed}
$\newline$

Ordnung verträglich mit Addition: $\overline n < 0 \iff 0=\overline n+(-\overline n) < -\overline n
= (-1) \cdot \overline n$ \\
$\newline$

beachte: $\mathbb Z := \mathbb N \cup \{(-n) \mid n \in \mathbb N_{>0}\}$ \\
$\newline$

\begin{framed}
	\textbf{Satz:} $\whole$ und $\overline{\whole}$ sind isomorph bezüglich Addition, 
	Multiplikation und Ordnung.
\end{framed}

\textit{Beweis: \\
	betrachte Abbildung $I: \mathbb Z \to \overline{\whole} $ mit $I(k):=[(k,0)]$ und $I(-k):=[(0,k)]
	\quad \forall k \in \natur \Rightarrow$ Übungsaufgabe}
$\newline$

Notation: verwende stets $\mathbb Z$, schreibe $m,n,...$ statt $\overline m, \overline n,...$ f\"ur
ganze Zahlen in $\mathbb Z$ \\
$\newline$

\textbf{Frage:} Existiert eine ganze Zahl mit $n=n' \cdot x$ f\"ur $n,n' \in \mathbb Z, n' \neq 0$ \\
\textbf{Antwort:} im Allgemeinen nicht
\textbf{Ziel:} Zahlbereichserweiterung analog zu $\mathbb N \to \mathbb Z$ \\
ordne jedem Paar $(n,n') \in \mathbb Z \times \mathbb Z$ neue Zahl $x$ zu \\
schreibe $(n,n')$ auch als $\frac{n}{n'}$ oder $n:n'$ \\
identifiziere Paare wie z.B. $\frac 4 2, \frac 6 3, \frac 8 4$ durch Relation \\
$\mathbb Q := {(\frac{n_1}{n'_2}, \frac{n_2}{n'_2}) \in (\mathbb Z \times \mathbb Z_{\neq 0}) 
	\times (\mathbb Z \times \mathbb Z_{\neq 0}) \mid n_1n'_2=n'_1n_2}$ \\
$\Rightarrow \mathbb Q$ ist eine \"Aquivalenzrelation auf $\mathbb Z \times \mathbb Z_{\neq 0}$ \\
$\newline$

setze $\mathbb Q := {[\frac{n}{n'}] \mid (n,n') \in \mathbb Z \times \mathbb Z_{\neq 0}}$ Menge der
rationalen Zahlen \\
beachte: unendlich viele Symbole $\frac{n}{n'}$ f\"ur gleiche Zahl $[\frac{n}{n'}]$ \\
wir schreiben sp\"ater $\frac{n}{n'}$ f\"ur die Zahl $[\frac{n}{n'}]$ \\
$\newline$

offenbar gilt die K\"urzungsregel: $[\frac{n}{n'}]=[\frac{kn}{kn'}] \quad \forall k \in 
\mathbb Z_{\neq 0}$ \\
$\newline$

\textbf{Rechenoperationen auf $\mathbb Q$:} \\
\begin{compactitem}
	\item Addition: $[\frac{m}{m'}]+[\frac{n}{n'}] := [\frac{mn'+m'n}{m'n'}]$
	\item Multiplikation: $[\frac{m}{m'}] \cdot [\frac{n}{n'}] := [\frac{mn}{m'n'}]$
\end{compactitem}
$\newline$

\begin{framed}
	\textbf{Satz:} Mit Addition und Multiplikation ist $\mathbb Q$ ein K\"orper mit
	\begin{compactitem}
		\item neutralen Elementen: $0=[\frac{0_{\mathbb Z}}{1_{\mathbb Z}}]=
		[\frac{0_{\mathbb Z}}{n_{\mathbb Z}}], 1:=[\frac{1_{\mathbb Z}}{1_{\mathbb Z}}]=[\frac n n]
		\neq 0$
		\item inversen Elementen: $-[\frac{n}{n'}]=[\frac{-n}{n}], [\frac{n}{n'}]^{-1}=[\frac{n'}{n}]$
	\end{compactitem}
\end{framed}
$\newline$

Ordnung auf $\mathbb Q:$ f\"ur $[\frac{n}{n'}] \in \mathbb Q$ kann man stets $n'>0$ annehmen \\
Realtion: $R:=\{([\frac{m}{m'}],[\frac{n}{n'}]) \in \mathbb Q \times \mathbb Q \mid mn' \le m'n, 
m',n' > 0\}$ gibt Ordnung $\le$ \\
$\newline$

\begin{framed}
	\textbf{Satz:} $\mathbb Q$ ist ein angeordneter K\"orper (d.h. $\le$ ist eine Totalordnung und
	vertr\"aglich mit Addition und Multiplikation)
\end{framed}
$\newline$

Notation: schreibe vereinfacht nur noch $\frac{n}{n'}$ f\"ur die Zahl $[\frac{n}{n'}] \in \mathbb Q$
und verwende auch Symbole $p,q,...$ f\"ur Elemente aus $\mathbb Q$ \\
$\newline$

Gleichung $p \cdot x = q$ hat stets eindeutige L\"osung: $x=q \cdot p^{-1}$ ($p,q \in \mathbb Q, 
p \neq 0$) \\
$\newline$

\textbf{Frage:} $\mathbb N \subset \mathbb Z \to \mathbb Z \subset \mathbb Q$?
\textbf{Antwort:} Sei $\mathbb Z_{\mathbb Q} := {\frac n 1 \in \mathbb Q \mid n \mathbb Z}, I:
\mathbb Z \to \mathbb Z_{\mathbb Q}$ mit $I(n)=\frac n 1$ \\
$\Rightarrow I$ ist Isomorphismus bez\"uglich Addition, Multiplikation und Ordnung. \\
In diesem Sinn: $\mathbb N \subset \mathbb Z \subset \mathbb Q$ \\
$\newline$

\begin{framed}
	\textbf{Folgerung:} K\"orper $\mathbb Q$ ist archimedisch angeordnet, d.h. f\"ur alle $q \in 
	\mathbb Q \exists n \in \mathbb N: q<_{\mathbb Q} n$
\end{framed}

\textit{Beweis: \\
	Sei $q = [\frac{k}{k'}]$ mit $k'>0$ \\
	$n := 0$ falls $k<0 \Rightarrow q=[\frac{k}{k'}] < [\frac{0}{k'}]=0=n$ \\
	$n := k+1$ falls $k \ge 0 \Rightarrow q=[\frac{k}{k'}] < [\frac{k+1}{k'}]=n$}
$\newline$